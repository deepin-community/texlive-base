% \iffalse meta-comment
% 
% This is file `subcaption.dtx'.
% 
% Copyright (C) 2007-2023 Axel Sommerfeldt (axel.sommerfeldt@f-m.fm)
% 
% --------------------------------------------------------------------------
% 
% This work may be distributed and/or modified under the
% conditions of the LaTeX Project Public License, either version 1.3
% of this license or (at your option) any later version.
% The latest version of this license is in
%   http://www.latex-project.org/lppl.txt
% and version 1.3 or later is part of all distributions of LaTeX
% version 2003/12/01 or later.
% 
% This work has the LPPL maintenance status "maintained".
% 
% This Current Maintainer of this work is Axel Sommerfeldt.
% 
% This work consists of the files
%   caption.ins, caption.dtx, caption-light.dtx, caption2.dtx, caption3.dtx,
%   caption-ams-smf.dtx, caption-beamer.dtx, caption-elsarticle.dtx,
%   caption-koma.dtx, caption-memoir.dtx, caption-ntg.dtx,
%   caption-thesis.dtx, bicaption.dtx, ltcaption.dtx, subcaption.dtx,
% the derived files
%   caption.sty, caption-light.sty, caption2.sty, caption3.sty,
%   caption-ams-smf.sto, caption-beamer.sto, caption-elsarticle.sto,
%   caption-koma.sto, caption-memoir.sto, caption-ntg.sto,
%   caption-thesis.sto, bicaption.sty, ltcaption.sty, subcaption.sty.
% 
% \fi
%
% \CheckSum{511}
%
% \iffalse
%<*driver>
\NeedsTeXFormat{LaTeX2e}[1994/12/01]
\ProvidesFile{subcaption.drv}[2023/08/13 v1.6 Adds a sub-caption feature to the caption package]
%\errorcontextlines=3
%
\documentclass{captiondoc}
\hypersetup{pdfkeywords={LaTeX, package, subcaption}}
%
\usepackage{caption}[2022/01/07]    % needs v3.6 or newer to typeset this document
\usepackage{subcaption}[2022/01/07] % needs v1.5 or newer to typeset this document
\DeclareCaptionSubType*[arabic]{table}
\captionsetup[subtable]{labelformat=simple,labelsep=colon}
%
\usepackage{array,diagbox,graphicx,overpic,pict2e,stackengine,tikz}
%
\begin{document}
  \DocInput{subcaption.dtx}
\end{document}
%</driver>
% \fi
%
% \def\thispackage{the \pkg{subcaption} package}
% \def\Thispackage{The \pkg{subcaption} package}
%
% \GetFileInfo{subcaption.drv}
% \let\docdate\filedate
% \let\docversion\fileversion
% \GetFileInfo{subcaption.sty}
%
% \title{\texorpdfstring{\Thispackage\thanks{%^^A
%          This package has version number \docversion.}}%^^A
%        {The subcaption package}}
% \author{Axel Sommerfeldt\\
%         \url{https://gitlab.com/axelsommerfeldt/caption}}
% \date{\docdate}
% \maketitle
% 
% \begin{abstract}
% This package offers an user interface to typeset sub-captions.
% \end{abstract}
%
% \begin{background}
% At the end of each section, text marked with the mountain symbol
% will contain background knowledge on how the particular command or
% environment is actually implemented.
% If you just want to use this package as it is, you don't have to read or
% understand them.
% \end{background}
%
% \begin{background*}
% Since version $3.1$ the \pkg{caption} package offers a low-level interface to typeset sub-captions:
% |\Declare|\-|Caption|\-|Sub|\-|Type| defines the required counters and internal commands,
% |\set|\-|caption|\-|sub|\-|type| switches to the sub-caption mode, and
% |\caption@sub|\-|type|\-|hook| could be extended to apply own code when a switch to the sub-caption mode is in progress.
%
% This package demonstrates its usage by offering a high-level user interface additionally.
% \end{background*}
% 
% \clearpage
% \setcounter{tocdepth}{2}
% \tableofcontents
%
% \iffalse
% --------------------------------------------------------------------------- %
% \fi
%
% \clearpage
% \section{Loading the package}
%
% Load this package using
% \begin{quote}
%   |\usepackage|\oarg{options}|{subcaption}|\quad.
% \end{quote}
% The options for \thispackage\ are the same ones as for the \pkg{caption}
% package, but specify settings which are used for sub-captions
% \emph{additionally}.
% In fact
% \begin{quote}
%   |\usepackage|\oarg{options}|{subcaption}|
% \end{quote}
% is identical to
% \begin{quote}
%   |\usepackage{subcaption}|\\
%   |\subcaptionsetup|\marg{options}\quad.
% \end{quote}
%
% \bigskip
%
% \pagebreak[3]
% The default settings for sub-captions are:
% \begin{quote}
%   |margin=0pt,font+=smaller,labelformat=parens,labelsep=space,|\\
%   |skip=6pt,list=false,hypcap=false|~\footnote{%^^A
%     This means that sub-captions are not listed in the List of Figures
%     or Tables by default, but you can enable it by
%     specifying the option \texttt{list=true}.}
% \end{quote}
%
% \iffalse
% --------------------------------------------------------------------------- %
% \fi
%
% \PageBreak
% \section{Setting options}
% \label{sec:options}
%
% \DescribeMacro\subcaptionsetup
% \NEWfeature*{v1.6}
% The |\subcaptionsetup| command sets options specifically for sub-captions.
%
% \begin{quote}
%   |\subcaptionsetup|\marg{options}
% \end{quote}
% sets options valid for all sub-captions.
%
% \pagebreak[3]
% An example:
% \begin{quote}
%   |\subcaptionsetup{font=it}|
% \end{quote}
% sets the font to ``italic'' for all sub-captions.
%
% \medskip
%
% \begin{quote}
%   |\subcaptionsetup|\oarg{environment}\marg{options}
% \end{quote}
% is an alternative syntax to
% \begin{quote}
%   |\caption|\-|setup[sub|\meta{environment}|]|\marg{options}\quad.
% \end{quote}
%
% \pagebreak[3]
% An example:
% \begin{quote}
%   |\subcaptionsetup[table]{labelformat=simple}|
% \end{quote}
% is identical to
% \begin{quote}
%   |\captionsetup[subtable]{labelformat=simple}|
% \end{quote}
% and sets the labelformat to ``simple'' for all sub-captions inside tables.
%
% \bigskip
%
% Options specified with |\usepackage[|\ldots|]{sub|\-|caption}| and
% |\sub|\-|caption|\-|setup{|\ldots|}| will override the ones specified by
% |\caption|\-|setup{|\ldots|}| and |\caption|\-|setup[fig|\-|ure]{|\ldots|}|,
% but are again overwritten by |\subcaption|\-|setup[figure]{|\ldots|}|
% (same for `table'). So finally we have the following order how
% settings for sub-captions are applied:
% \begin{enumerate}
% \item Global settings
%  {\small(|\usepackage[|\ldots|]{caption}| and |\captionsetup{|\ldots|}|)}
% \item Environmental settings
%  {\small(|\captionsetup[figure|\emph{ -or- }|table]{|\ldots|}|)}
% \item Local settings
%  {\small(|\captionsetup{|\ldots|}| inside |figure| or |table| environment)}
% \item Default `sub' settings
%  {\small(|margin=0pt,font+=smaller,|\ldots, see above)}
% \item Custom `sub' settings
%  {\small(|\usepackage[|\ldots|]{subcaption}| and |\subcaptionsetup{|\ldots|}|)}
% \item Environmental `sub' settings
%  {\small(|\subcaptionsetup[figure|\emph{ -or- }|table]{|\ldots|}|)}
% \item Local `sub' settings
%  {\small(|\captionsetup{|\ldots|}| inside |sub|\-|figure| or |sub|\-|table|)}
% \end{enumerate}
%
% \pagebreak[3]
% An example:
% \begin{quote}
%   |\usepackage[labelsep=quad,indention=10pt]{caption}|\\
%   |\usepackage[labelfont=bf,list=true]{subcaption}|\\
%   |\captionsetup[table]{textfont=it,position=top}|\\
%   |\subcaptionsetup[table]{textfont=sf}|
% \end{quote}
% causes sub-captions inside |table| environments to be typeset with
% the settings
% \begin{quote}
%   |indention=10pt,position=top,margin=0pt,font=small,|\\
%   |labelformat=parens,labelsep=space,skip=6pt,hypcap=false,|\\
%   |labelfont=bf,list=true,textfont=sf|\quad.
% \end{quote}
%
% \iffalse
% --------------------------------------------------------------------------- %
% \fi
%
% \PageBreak
% \section{The \cs{subcaptionbox} command}
%
% \DescribeMacro\subcaptionbox
% The |\sub|\-|caption|\-|box| command typesets given content and caption.
% It automatically aligns the sub-figures or sub-tables by their very first caption line.
%
% \pagebreak[3]
% Its syntax is:
% \begingroup
% \leftmargini=12pt
% \begin{quote}
%   |\subcaptionbox|\oarg{list entry}\marg{heading}\oarg{width}\oarg{inner-pos}\marg{contents}\\
%   |\subcaptionbox*|\marg{heading}\oarg{width}\oarg{inner-pos}\marg{contents}
% \end{quote}
% \endgroup
% \begin{small}
% The arguments \meta{list entry} \& \meta{heading} will be used for typesetting the |\caption|.
%
% \meta{width} is the width of the resulting |\par|\-|box|; the default value is the width of the contents.
%
% \meta{inner-pos} specifies how the contents will be justified inside the resulting |\parbox|;
% it can be either `c' (centered, using |\centering|), `l' (left-justified, using |\ragged|\-|right|),
% `r' (right-justified, using |\ragged|\-|left|), or `s' (for no special justification).
% The default is `c'.
% (But you can use any justification defined with
% |\Declare|\-|Caption|\-|Jus|\-|ti|\-|fi|\-|ca|\-|tion| as well,
% e.g.: `|center|\-|last|')
% \end{small}
%
% \bigskip
%
% When using |\sub|\-|caption|\-|box|, the baseline of the resulting box will be placed
% right between contents and heading. So usually you don't have to care about the
% vertical alignment of the sub-figures for yourself.
% Also the hyperlink anchor is placed properly with respect to the |hyp|\-|cap=| setting.
%
% \pagebreak[3]
% An example:
% \begin{quote}
%   |\begin{figure}|\\
%   |  \centering|\\
%   |  \subcaptionbox{A cat\label{cat}}|\\
%   |    {\includegraphics{cat}}|\\
%   |  \subcaptionbox{An elephant\label{elephant}}|\\
%   |    {\includegraphics{elephant}}|\\
%   |  \caption{Two animals}\label{animals}|\\
%   |\end{figure}|
% \end{quote}
% \noindent\begin{minipage}{\linewidth}
%   \captionsetup{type=figure}
%   \centering
%   \subcaptionbox{A cat\label{cat}}
%     {\includegraphics[width=30pt]{cat}}
%   \subcaptionbox{An elephant\label{elephant}}
%     {\includegraphics[width=140pt]{elephant}}
%   \caption[Two animals]{Two animals~\footnotemark}\label{animals}
% \end{minipage}
% \footnotetext{The pictures were taken with permission from the
%   \LaTeX\ Companion\cite{TLC2} examples.}
%
% \bigskip
%
% As you see the result is not satisfying;
% the caption below the cat looks ugly because of the small width of the
% graphic. This can be solved by using the optional arguments of
% |\sub|\-|caption|\-|box|, increasing the width of the resulting box:
% \begin{quote}
%   |  |\ldots\\
%   |  \subcaptionbox{A cat\label{cat}}|\\
%   |    |\textcolor{blue}{\csoarg{2.5cm}}|{\includegraphics{cat}}|\\
%   |  |\ldots
% \end{quote}
% \noindent\begin{minipage}{\linewidth}
%   \captionsetup{type=figure}
%   \centering
%   \subcaptionbox{A cat\label{cat2}}
%     [2.5cm]{\includegraphics[width=30pt]{cat}}
%   \subcaptionbox{An elephant\label{elephant2}}
%     {\includegraphics[width=140pt]{elephant}}
%   \caption{Two animals}\label{animals2}
% \end{minipage}
%
% \bigskip
%
% Furthermore the main caption, which is centered with respect to the
% |\text|\-|width|, looks mis-aligned with respect to the sub-captions.
% This can (again) be solved by using the optional arguments of
% |\sub|\-|caption|\-|box|, giving both boxes the same width, for example:
% \begin{quote}
%   |  |\ldots\\
%   |  \subcaptionbox{A cat\label{cat}}|\\
%   |    |\textcolor{blue}{\csoarg{.4\cs{textwidth}}}|{\includegraphics{cat}}%|\\
%   |  \subcaptionbox{An elephant\label{elephant}}|\\
%   |    |\textcolor{blue}{\csoarg{.4\cs{textwidth}}}|{\includegraphics{elephant}}|\\
%   |  |\ldots
% \end{quote}
% \noindent\begin{minipage}{\linewidth}
%   \captionsetup{type=figure}
%   \centering
%   \subcaptionbox{A cat\label{cat3}}
%     [.4\textwidth]{\includegraphics[width=30pt]{cat}}%^^A
%   \subcaptionbox{An elephant\label{elephant3}}
%     [.4\textwidth]{\includegraphics[width=140pt]{elephant}}
%   \caption{Two animals}\label{animals3}
% \end{minipage}
%
% \bigskip
%
% \begin{background}
% The |\sub|\-|caption|\-|box| is a |\par|\-|box| with
% |\set|\-|caption|\-|sub|\-|type| as first contents line.
% \end{background}
%
% \subsection{Comparison with \cs{captionbox}}
%
% Both, |\caption|\-|box| (offered by the \pkg{caption} package) and |\sub|\-|caption|\-|box|,
% put its contents and caption into a |\par|\-|box| of either natural or given width and share the
% same (mandatory and optional) arguments, but while |\caption|\-|box| uses a regular caption,
% |\sub|\-|caption|\-|box| uses a sub-caption instead, like ``(a)'' or ``2.1''.
%
% \pagebreak[3]
% So for example the last example would look like this when using |\caption|\-|box|
% instead of |\sub|\-|caption|\-|box|:
% \begin{quote}
%   |\begin{figure}|\\
%   |  \centering|\\
%   |  \captionbox{A cat\label{cat}}|\\
%   |    |\csoarg{.4\cs{textwidth}}|{\includegraphics{cat}}%|\\
%   |  \captionbox{An elephant\label{elephant}}|\\
%   |    |\csoarg{.4\cs{textwidth}}|{\includegraphics{elephant}}|\\
%   |\end{figure}|
% \end{quote}
% \noindent\begin{minipage}{\linewidth}
%   \captionsetup{type=figure}
%   \centering
%   \captionbox{A cat\label{cat3.1}}
%     [.4\textwidth]{\includegraphics[width=30pt]{cat}}%^^A
%   \captionbox{An elephant\label{elephant3.1}}
%     [.4\textwidth]{\includegraphics[width=140pt]{elephant}}
% \end{minipage}
%
% \iffalse
% --------------------------------------------------------------------------- %
% \fi
%
% \PageBreak
% \section{The \env{subcaptionblock} environment}
%
% \DescribeEnv{subcaptionblock}
% \NEWfeature*{v1.5}
% The |sub|\-|caption|\-|block| environment makes a box with given width.
% Inside this box the regular caption commands (like |\caption|, |\phantom|\-|caption|, \ldots) could be used to typeset sub-captions.
%
% \pagebreak[3]
% |sub|\-|caption|\-|block| has the same (optional \& mandatory) arguments as the |mini|\-|page| environment:
% \begingroup
% \leftmargini=12pt
% \begin{quote}
%   |\begin{subcaptionblock}|\oarg{outer-pos}\oarg{height}\oarg{inner-pos}\marg{width}\\
%   |  |\ldots\\
%   |\end{subcaptionblock}|
% \end{quote}
% \endgroup
%
% The default value for \meta{outer-pos} is either `b' (if no main caption was typeset so far) or `t' (if a main caption was already typeset),
% and the default value for \meta{inner-pos} is `s'.
%
% \NEWfeature{v1.2}
% Beside the \meta{outer-pos} values of `c', `t', and `b', \thispackage\ also offers the
% values `T' and `B' additionally which align the contents at the
% very top or bottom. (In contrast `t' and `b' align the contents at the
% top or bottom \textit{baseline}.)
%
% \bigskip
%
% \pagebreak[3]
% The same example as \autoref{animals3}, but this time using the |sub|\-|caption|\-|block| environment
% instead of |\sub|\-|caption|\-|box|:
% \begin{quote}
%   |\begin{figure}|\\
%   |  \centering|\\
%   |  |\textcolor{blue}{\cs{begin}\csmarg{subcaptionblock}\csmarg{.4\cs{textwidth}}}\\
%   |    \centering|\\
%   |    \includegraphics{cat}|\\
%   |    |\textcolor{blue}{\cs{caption}}|{A cat}\label{cat}|\\
%   |  |\textcolor{blue}{\cs{end}\csmarg{subcaptionblock}}|%|\\
%   |  |\textcolor{blue}{\cs{begin}\csmarg{subcaptionblock}\csmarg{.4\cs{textwidth}}}\\
%   |    \centering|\\
%   |    \includegraphics{elephant}|\\
%   |    |\textcolor{blue}{\cs{caption}}|{An elephant}\label{elephant}|\\
%   |  |\textcolor{blue}{\cs{end}\csmarg{subcaptionblock}}|%|\\
%   |  \caption{Two animals}\label{animals}|\\
%   |\end{figure}|
% \end{quote}
% \noindent\begin{minipage}{\linewidth}
%   \captionsetup{type=figure}
%   \centering
%   \begin{subcaptionblock}{.4\textwidth}
%     \centering
%     \includegraphics[width=30pt]{cat}
%     \caption{A cat}\label{cat4}
%   \end{subcaptionblock}%
%   \begin{subcaptionblock}{.4\textwidth}
%     \centering
%     \includegraphics[width=140pt]{elephant}
%     \caption{An elephant}\label{elephant4}
%   \end{subcaptionblock}
%   \caption{Two animals}\label{animals4}
% \end{minipage}
%
% \medskip
%
% \pagebreak[3]
% Some additional notes:
% \begin{itemize}
% \item
% You can override the setttings for a specific subcaption
% with a |\caption|\-|setup| inside the |sub|\-|caption|\-|block|, e.g.:
% \begin{quote}
%   |\begin{subcaptionblock}{.4\textwidth}|\\
%   |  \centering|\\
%   |  \includegraphics{owl}|\\
%   |  |\textcolor{blue}{\cs{captionsetup}\csmarg{skip=3pt}}\\
%   |  \caption{An owl}\label{owl}|\\
%   |\end{subcaptionblock}|
% \end{quote}
% \item
% Just like |figure| or |table|, a |sub|\-|caption|\-|block| could have multiple captions, e.g.:
% \begin{quote}
%   |\begin{subcaptionblock}{.4\textwidth}|\\
%   |  \centering|\\
%   |  \includegraphics{cat}|\\
%   |  \caption{A cat}\label{cat}|\\
%   |  \medskip|\\
%   |  \includegraphics{elephant}|\\
%   |  \caption{An elephant}\label{elephant}|\\
%   |\end{subcaptionblock}|
% \end{quote}
% \item
% Hyperlinks targeted to this sub-figure will jump to the beginning
% of the |sub|\-|caption|\-|block|, and not to the |\caption| inside it
% (if |hypcap=true| is set for sub-captions).
% \See{\longref{hypcap}}
% \end{itemize}
%
% \DescribeEnv{subfigure}
% \DescribeEnv{subtable}
% The |sub|\-|caption|\-|block| environment is also offered as |sub|\-|figure| or |sub|\-|table|.
% (And prior version $1.5$ of \thispackage\ it was only available as |sub|\-|figure| or |sub|\-|table|.)
%
% There is no difference in them except the environment name should match the current floating environment,
% i.e. inside a |figure| a |sub|\-|figure| should be used, and inside a |table| a |sub|\-|table| should be used.
% Using the wrong sub-environment will cause a warning since \version{1.5} of the \pkg{subcaption} package.
%
% So if in doubt, or when writing own \LaTeX\ commands which should work in every floating environment,
% using |sub|\-|caption|\-|block| is the correct choice.
%
% (Note that the default value for \meta{outer-pos} has changed from `c' to either `b' or `t' in version $1.5$ of \thispackage.)
%
% \smallskip
%
% \begin{background}
% The |sub|\-|caption|\-|block|, |sub|\-|figure|, and |sub|\-|table| environments are |mini|\-|page|
% environments with |\set|\-|caption|\-|sub|\-|type| as first contents line.
%
% |sub|\-|figure| and |sub|\-|table| are defined with the help of
% |\For|\-|Each|\-|Caption|\-|Sub|\-|Type| offered by the \pkg{caption} package,
% which executes code for every sub-type declared with |\Declare|\-|Caption|\-|Sub|\-|Type|.
% \end{background}
%
% \iffalse
% --------------------------------------------------------------------------- %
% \fi
%
% \PageBreak
% \section{The \env{subcaptiongroup} environment}
%
% \DescribeEnv{subcaptiongroup}
% \NEWfeature*{v1.5}
% The |sub|\-|caption|\-|group| environment is only switching to the sub-caption mode inside an own \TeX\ group.
% Inside this environment the regular caption commands (like |\caption|, |\phantom|\-|caption|, \ldots) could be used to typeset sub-captions.
%
% \pagebreak[3]
% Its syntax is:
% \begingroup
% \leftmargini=12pt
% \begin{quote}
%   |\begin{subcaptiongroup}|\\
%   |  |\ldots\\
%   |\end{subcaptiongroup}|
% \end{quote}
% \endgroup
%
% \pagebreak[3]
% There is a starred variant of this environment as well which uses |\set|\-|caption|\-|sub|\-|type*|
% instead of |\set|\-|caption|\-|sub|\-|type| internally:
% \begingroup
% \leftmargini=12pt
% \begin{quote}
%   |\begin{subcaptiongroup*}|\\
%   |  |\ldots\\
%   |\end{subcaptiongroup*}|
% \end{quote}
% \endgroup
%
% While this gives you great flexibility, it also offers you no help formatting its contents.
%
% \pagebreak[3]
% The same example as \autoref{animals4}, but this time using the |sub|\-|caption|\-|group| environment
% instead of |\sub|\-|caption|\-|block|:
% \begin{quote}
%   |\begin{figure}|\\
%   |  \centering|\\
%   |  |\textcolor{blue}{\cs{begin}\csmarg{subcaptiongroup}}\\
%   |    \centering|\\
%   |    \parbox[b]{.4\textwidth}{%|\\
%   |      \centering|\\
%   |      \includegraphics{cat}|\\
%   |      \caption{A cat}\label{cat}}%|\\
%   |    \parbox[b]{.4\textwidth}{%|\\
%   |      \centering|\\
%   |      \includegraphics{elephant}|\\
%   |      \caption{An elephant}\label{elephant}}%|\\
%   |  |\textcolor{blue}{\cs{end}\csmarg{subcaptiongroup}}\\
%   |  \caption{Two animals}\label{animals}|\\
%   |\end{figure}|
% \end{quote}
% --or--
% \begin{quote}
%   |\begin{figure}|\\
%   |  \centering|\\
%   |  |\textcolor{blue}{\cs{begin}\csmarg{subcaptiongroup}}\\
%   |    \centering|\\
%   |    \begin{minipage}[b]{.4\textwidth}|\\
%   |      \centering|\\
%   |      \includegraphics{cat}|\\
%   |      \caption{A cat}\label{cat}|\\
%   |    \end{minipage}%|\\
%   |    \begin{minipage}[b]{.4\textwidth}|\\
%   |      \centering|\\
%   |      \includegraphics{elephant}|\\
%   |      \caption{An elephant}\label{elephant}|\\
%   |    \end{minipage}|\\
%   |  |\textcolor{blue}{\cs{end}\csmarg{subcaptiongroup}}\\
%   |  \caption{Two animals}\label{animals}|\\
%   |\end{figure}|
% \end{quote}
% \noindent\begin{minipage}{\linewidth}
%   \captionsetup{type=figure}
%   \centering
%   \begin{subcaptiongroup}
%     \centering
%     \begin{minipage}[b]{.4\textwidth}
%       \centering
%       \includegraphics[width=30pt]{cat}
%       \caption{A cat}\label{cat5}
%     \end{minipage}%
%     \begin{minipage}[b]{.4\textwidth}
%       \centering
%       \includegraphics[width=140pt]{elephant}
%       \caption{An elephant}\label{elephant5}
%     \end{minipage}
%   \end{subcaptiongroup}
%   \caption{Two animals}\label{animals5}
% \end{minipage}
%
% \bigskip
%
% \begin{background}
% The |sub|\-|caption|\-|group| environment is a \LaTeX\ environment
% with |\set|\-|caption|\-|sub|\-|type| as first contents line.
% \end{background}
%
% \iffalse
% --------------------------------------------------------------------------- %
% \fi
%
% \PageBreak
% \section{The \cs{DeclareCaptionSubType} command}
%
% \DescribeMacro\DeclareCaptionSubType
% For using the sub-caption feature of the \pkg{caption} package some
% commands and counters must be prepared. This is done with
% \begin{quote}
%  |\DeclareCaptionSubType|\oarg{numbering scheme}\marg{type}\\
%  |\DeclareCaptionSubType*|\oarg{numbering scheme}\marg{type}
% \end{quote}
% For the environments |figure| \& |table|, and all the ones
% defined with |\Declare|\-|Floating|\-|Environment| offered by the
% \pkg{newfloat} package, this will be done automatically,
% but for other environments (e.g.~the ones defined with |\newfloat| offered by the
% \pkg{float} package or |\Declare|\-|New|\-|Float|\-|Type| offered by the
% \pkg{floatrow} package) this has to be done manually.
%
% \medskip
%
% The starred variant provides the sub-caption numbering format
% \meta{type}|.|\meta{subtype} (for example `|1.2|') while the non-starred variant
% simply uses \meta{subtype} (for example `|a|').
%
% \pagebreak[3]
% Own numbering formats can be created by redefining |\thesub|\meta{type}, e.g.:
% \begingroup
% \leftmargini=12pt
% \begin{quote}
%   |\DeclareCaptionSubType*{figure}|\\
%   |\renewcommand\thesubfigure{\thefigure\alph{subfigure}}|
% \end{quote}
% \endgroup
% would give you sub-caption numbers like `|1b|'.
%
% The default numbering scheme is |alph|, but you can use any \LaTeX\ (or self-defined)
% command name here which converts a counter to a text value, e.g.: |arabic|, |roman|,
% |Roman|, |alph|, |Alph|, |fnsymbol|, \ldots
%
% \pagebreak[3]
% But |\DeclareCaptionSubType| is not only for defining new sub-caption types,
% you can use this command for re-definitions as well, e.g.:
% \begingroup
% \leftmargini=12pt
% \begin{quote}
%   |\DeclareCaptionSubType*[arabic]{table}|\\
%   |\subcaptionsetup[table]{labelformat=simple,labelsep=colon}|
% \end{quote}
% \endgroup
% \pagebreak[2]
% will give you sub-captions in |table|s like these ones:
% \par\bigskip
% \noindent\begin{minipage}{\linewidth}
%   \captionsetup{type=table}
%   \centering
%   \caption{Two tables}
%   \subcaptionbox{Table one}[3cm][c]{\begin{tabular}{cc}A & B\\ C & D\\ \end{tabular}}
%   \subcaptionbox{Table two}[3cm][c]{\begin{tabular}{cc}E & F\\ G & H\\ \end{tabular}}
% \end{minipage}
%
% \begin{background}
% |\Declare|\-|Caption|\-|Sub|\-|Type| and |\For|\-|Each|\-|Caption|\-|Sub|\-|Type|
% are integral parts of the \pkg{caption} package kernel.
% \end{background}
%
% \iffalse
% --------------------------------------------------------------------------- %
% \fi
%
% \PageBreak
% \section{Cross Referencing}
% \label{crossreferencing}
%
% The macro |\the|\-\meta{counter} is not only responsible for the look of the \meta{counter},
% but for the look of the references typeset with |\ref|, too. References will be prefixed by
% \LaTeX{} with the internal macro |\p@|\-\meta{counter}.
%
% |\Declare|\-|Caption|\-|Sub|\-|Type| will define both of them for sub-captions
% (e.g. |sub|\-|figure| and |sub|\-|table|), and as you have seen in the last section
% |\Declare|\-|Caption|\-|Sub|\-|Type| will give you some options to control the
% internal (re-)definition of |\the|\-\meta{counter} and |\p@|\-\meta{counter}.
%
% \pagebreak[3]
% \DescribeMacro\thesubfigure
% \DescribeMacro\p@subfigure
% For example |\thesubfigure| and |\p@subfigure| are (as default) internally defined as
% \begin{quote}
%   |\newcommand\thesubfigure{\alph{subfigure}}|\\
%   |\newcommand\p@subfigure{\thefigure}|
% \end{quote}
% so the label of sub-captions will look like `|a|' (decorated by the selected label format),
% while references will look like `|1a|' since they are prefixed by |\p@sub|\-|figure| $=$
% |\the|\-|figure|.
%
% After |\Declare|\-|Caption|\-|Sub|\-|Type*[arabic]{figure}|, |\the|\-|sub|\-|figure| and
% |\p@sub|\-|figure| will look like
% \begin{quote}
%   |\renewcommand\thesubfigure{\thefigure.\arabic{subfigure}}|\\
%   |\renewcommand\p@subfigure{}|
% \end{quote}
%
% But if you want detailed control on how the references will look like,
% the options of |\Declare|\-|Caption|\-|Sub|\-|Type| are potentially not sufficient.
% In this case one need to redefine these two macros on his/her own.
% Some examples:
%
% If you want parentheses around the sub-figure part of the reference,
% so they will look like `|1(a)|', you may get them this way:
% \begin{quote}
%   |\usepackage[labelformat=simple]{subcaption}|\\
%   |\renewcommand\thesubfigure{(\alph{subfigure})}|
% \end{quote}
% {\small (\emph{Note:} Since |parens| is the default label format you will get double
% parentheses in sub-captions when not specifiying a different label format, e.g. |simple|.)}
%
% But if you want only a closing parenthesis, so references will look like `|1a)|',
% but the sub-captions itself should still look like `|(a)|',
% this would be a possible solution:
% \begin{quote}
%   |\usepackage{subcaption}|\\
%   |\renewcommand\thesubfigure{\alph{subfigure})}|\\
%   |\DeclareCaptionLabelFormat{opening}{(#2}|\\
%   |\subcaptionsetup[figure]{labelformat=opening}|
% \end{quote}
%
% \bigskip
%
% \pagebreak[3]
% \INFO
% {\small Please note that you need to surround redefinitions of |\p@|\-\meta{counter}
% with |\make|\-|at|\-|letter| and |\make|\-|at|\-|other|. See
% \url{http://tex.stackexchange.com/questions/8351/}
% for details.
%
% Since October 2019 |\p@|\-\meta{counter} could be redefined using |\label|\-|format| instead,
% dropping the need to use |\make|\-|at|\-|letter| and |\make|\-|at|\-|other|.
% Furthermore not only prefixes could be specified this way,
% instead the counter representation could be used at any position in the text.
% So for example
% \begin{quote}
%   |\makeatletter|\\
%   |\renewcommand\p@subfigure{\thefigure}|\\
%   |\makeatother|\\
% \end{quote}
% (which prefixes the subfigure counter representation with the figure counter representation)
% could be replaced by
% \begin{quote}
%   |\labelformat{subfigure}{\thefigure #1}|
% \end{quote}}
%
% \pagebreak[3]
% \subsection{The \cs{subref} command}
% \label{subref}
%
% While |\ref|\marg{key} (and |\ref*|\marg{key}, if the \pkg{hyperref}
% package is used) usually gives a combined result representing the main
% caption counter and the sub-caption one, it is sometimes useful to have
% a reference to the sub-caption only. For this purpose you can use
% \begin{quote}
%   |\subref|\marg{key}\\
%   |\subref*|\marg{key}~\footnote{%^^A
%     Like \cs{ref*}, \cs{subref*} is only available if the \pkg{hyperref}
%     package\cite{hyperref} is used.}%^^A
%     \qquad.
% \end{quote}
% So for example |\ref{cat}| gives the result `\ref{cat}' but |\subref{cat}|
% gives `\subref{cat}'.
%
% \begin{small}
% \emph{Note:} If the sub-caption was (re-)defined with the starred variant
% |\Declare|\-|Caption|\-|Sub|\-|Type*|, both |\ref| and |\sub|\-|ref| usually gives
% the same result.
% \end{small}
%
% \begin{background}
% The |\sub|\-|ref| command demonstrates the usage of |\caption@sub|\-|type|\-|hook|
% which will be called during |\caption|\-|setup{sub|\-|type}|.
% \end{background}
%
% \pagebreak[3]
% \subsection{The \opt{subrefformat=} option}
% \label{subrefformat}
%
% \DescribeMacro{subrefformat=}
% \NEWfeature*{v1.1}
% By applying |\Declare|\-|Caption|\-|Sub|\-|Type|, or by redefining |\the|\-\meta{counter}
% and |\p@|\-\meta{counter}, you will change the look of references typeset with |\ref|
% \emph{and} |\sub|\-|ref|.
%
% But maybe you only want to change the output of |\sub|\-|ref| without
% affecting the references typeset with |\ref|?
%
% \pagebreak[3]
% This is possible, too, by using the option \opt{subrefformat}:
% \begin{quote}
%   |\captionsetup{subrefformat=|\meta{label format}|}|
% \end{quote}
% This one will choose a label format (either a pre-defined one, or a one defined with
% |\Declare|\-|Caption|\-|Label|\-|Format|) as decorative element to sub-references.
% The default one is |simple| which has no decorative elements but simply typeset
% the reference as it is.
%
% \pagebreak[3]
% For example
% \begin{quote}
%   |\captionsetup{subrefformat=parens}|
% \end{quote}
% will result in references (typeset with |\ref|) like `|1a|' but sub-references
% (typeset with |\subref|) like `|(a)|'.
%
% \subsection{Referencing sub-figures without sub-captions}
% \label{without}
%
% \DescribeMacro\phantomcaption
% If you don't want to give a sub-figure a caption (yet), because the picture itself
% already contains the caption, or for some other reason, you could use the command
% \begin{quote}
%   |\phantomcaption|
% \end{quote}
% instead of |\caption|.
%
% |\phantom|\-|caption| is offered by the \pkg{caption} package since version $3.2$ and
% does not generate any output but increases the sub-figure or sub-table counter
% and gives you an anchor for a |\label| command which can be placed after it.
%
% \pagebreak[3]
% An example:
% \begin{quote}
%   |\begin{figure}|\\
%   |  \centering|\\
%   |  \begin{subcaptiongroup}|\\
%   |    \includegraphics{cat_with_a}|\\
%   |    |\textcolor{blue}{\cs{phantomcaption}}|\label{cat}|\\
%   |    \includegraphics{elephant_with_b}|\\
%   |    |\textcolor{blue}{\cs{phantomcaption}}|\label{elephant}|\\
%   |  \end{subcaptiongroup}|\\
%   |  \captionsetup{subrefformat=parens}|\\
%   |  \caption{Two animals: \subref{cat} a cat,|\\
%   |           and \subref{elephant} an elephant}|\\
%   |  \label{animals}|\\
%   |\end{figure}|
% \end{quote}
% \noindent\begin{minipage}{\linewidth}
%   \captionsetup{type=figure}
%   \centering
%   \begin{subcaptiongroup}
%     \phantomcaption\label{cat6.3}
%     \begin{overpic}[width=30pt]{cat}
%       \put(40,34){\captiontext*{}}
%     \end{overpic}
%     \phantomcaption\label{elephant6.3}
%     \begin{overpic}[width=140pt]{elephant}
%       \put(58,40){\captiontext*{}}
%     \end{overpic}
%   \end{subcaptiongroup}
%   \captionsetup{subrefformat=parens}
%   \caption{Two animals: \subref{cat6.3} a cat, and \subref{elephant6.3} an elephant}
% \end{minipage}
%
% \bigskip
%
% \pagebreak[3]
% \DescribeMacro\captionlistentry
% If you don't want to give a sub-figure a caption (yet), because the picture itself
% already contains the caption, or for some other reason, you could also use the command
% \begin{quote}
%   |\captionlistentry|\marg{list entry}
% \end{quote}
% instead of |\caption|.
%
% |\caption|\-|list|\-|entry| is offered by the \pkg{caption} package since version $3.3$ and
% (just like |\phantom|\-|caption|) does not generate any output but increases
% the sub-figure or sub-table counter
% and gives you an anchor for a |\label| command which can be placed after it.
% Additionally to |\phantom|\-|caption| this command puts an entry into the list
% of figures or tables.
%
% \pagebreak[3]
% An example:
% \begin{quote}
%   |\begin{figure}|\\
%   |  \centering|\\
%   |  \begin{subcaptiongroup}|\\
%   |    \includegraphics{cat_with_a}|\\
%   |    |\textcolor{blue}{\cs{captionlistentry}\csmarg{A cat}}\\
%   |    \label{cat}|\\
%   |    \includegraphics{elephant_with_b}|\\
%   |    |\textcolor{blue}{\cs{captionlistentry}\csmarg{An elephant}}\\
%   |    \label{elephant}|\\
%   |  \end{subcaptiongroup}|\\
%   |  \captionsetup{subrefformat=parens}|\\
%   |  \caption{Two animals: \subref{cat} a cat,|\\
%   |           and \subref{elephant} an elephant}|\\
%   |  \label{animals}|\\
%   |\end{figure}|
% \end{quote}
%
% \pagebreak[3]
% \subsection{Typesetting sub-captions without generating a (new) reference}
% \label{without2}
%
% The |\caption| command is a multi-purpose command:
% \begin{enumerate}
% \item It increments the sub-figure or sub-table counter and generates an internal reference which could be used with |\label|
% \item It puts an entry into the list of figures or tables
% \item It finally typesets a caption
% \end{enumerate}
%
% When put into a command or into an environment which either evaluates its content more than once or
% does not like one of the first two actions (for whatever reason), the result could be either an error message or an incorrect
% result, for example a sub-figure or sub-table counter which was incremented more than once.
%
% \DescribeMacro\captiontext
% In these cases the |\caption| command could be split into
% |\phantom|\-|caption| which performs step one only (or |\caption|\-|list|\-|entry| which performs steps one and two),
% and |\caption|\-|text| which performs step three only.
% This way critical steps could be out-sourced from the target command or environment,
% for example by prepending |\phantom|\-|caption| and using |\caption|\-|text| inside.
%
% \pagebreak[3]
% The syntax of |caption|\-|text| is
% \begin{quote}
%   |\captiontext|\oarg{number}\marg{text of sub-caption}\\
%   |\captiontext*|\oarg{number}\marg{text of sub-caption}
% \end{quote}
% |\caption|\-|text| is offered by the \pkg{caption} package since version $3.6$ and
% as opposite to |\caption| it does not increase the sub-figure or sub-table counter
% and does not give you an anchor for a |\label| command.
% It typesets the caption only, using existing counter values unless a \meta{number} is given explicitly.
%
% For example code please take a look at \longref{inside}.
%
% \pagebreak[3]
% \subsection{Where to place the \cs{label} command?}
% \label{label}
%
% When |\caption|
% inside a |caption|\-|sub|\-|block|, |sub|\-|figure|, |sub|\-|table|, or |caption|\-|sub|\-|group| environment,
% the |\label| can be either placed inside the caption text or right after the command, e.g.:
% \begin{quote}
%   |\caption{Some text here\label{text}}|\\
%    \ldots\\
%   |\caption{Some other text}\label{othertext}|\\
%    \ldots\\
%   |\caption{Something completely different}|\\
%   |\label{differenttext}|
% \end{quote}
%
% When using |\phantom|\-|caption| or |\caption|\-|list|\-|entry|
% inside a |caption|\-|sub|\-|block|, |sub|\-|figure|, |sub|\-|table|, or |caption|\-|sub|\-|group| environment,
% the |\label| should be placed right after the command, e.g.:
% \begin{quote}
%   |\phantomcaption\label{this}|\\
%    \ldots\\
%   |\phantomcaption|\\
%   |\label{that}|
% \end{quote}
%
% But when using the |\sub|\-|caption|\-|box| command, the |\label| should be placed inside
% the caption text, e.g.:
% \begin{quote}
%   |\subcaptionbox{A description here\label{todo1}}|\\
%   |              {Some content here}|\\
%    \ldots\\
%   |\subcaptionbox[List-of-Figures entry]|\\
%   |              {A description here\label{todo2}}|\\
%   |              {Some content here}|
% \end{quote}
% Placing |\label| outside the |\sub|\-|caption|\-|box| would produce an incorrect reference.
%
% \pagebreak[3]
% \subsection{Where do hyperlinks jump?}
% \label{hypcap}
%
% For the |caption|\-|sub|\-|block|, |sub|\-|figure|, |sub|\-|table|, and |caption|\-|sub|\-|group| environments,
% and for the |\sub|\-|caption|\-|box| command (and all other constructs which use |\set|\-|caption|\-|sub|\-|type|)
% the hyperlink anchors will be placed in respect to the |hypcap=| setting.
% While usage of this option is straight-forward for ordinary captions,
% the usage for sub-captions depends on the setting regarding the main captions.
%
% This table gives you an overview where the hyperlinks will jump:
%
% \bigskip
%
% \begin{small}
% \centering
% \DeleteShortVerb{\|}
% \renewcommand\arraystretch{1.5}
% \begin{tabular}{|l|p{3cm}p{3cm}|}
% \hline
% \backslashbox{subcaption}{caption} &
%   \multicolumn{1}{c}{\texttt{hypcap=false}} &
%   \multicolumn{1}{c|}{\texttt{hypcap=true}} \\
% \hline
% \raisebox{-1.5ex}[1.5ex]{\texttt{hypcap=false}} &
%   sub-caption & figure or table\par\hfill\textit{(default setting)}\\
% \raisebox{-1.5ex}[1.5ex]{\texttt{hypcap=true}} &
%   sub-figure or\par sub-table & sub-figure or\par sub-table \\
% \hline
% \end{tabular}\par
% \MakeShortVerb{\|}
% \end{small}
% \bigskip
%
% But if the |caption|\-|sub|\-|group*| environment is used
% (or a different construct which uses |\set|\-|caption|\-|sub|\-|type*|)
% and |hypcap=true| is set for sub-captions,
% \thispackage\ does not know where the sub-figure or sub-table actually
% begins, so it will jump to the sub-caption instead.
%
% \medskip
% \emph{Remember:} If you use the \pkg{hypcap} package\cite{hypcap},
% it controls the placement of the hyperlink anchors, making the rules
% above invalid.
%
% \bigskip
% {\small(See also the documentation of the \pkg{caption} package,
% sections about \pkg{hyperref} \& \pkg{hypcap}.)}
%
% \iffalse
% --------------------------------------------------------------------------- %
% \fi
%
% \PageBreak
% \section{Captions inside sub-figures}
% \label{inside}
%
% Nearly all code examples so far have placed the caption either above or below the sub-figure.
% But it is possible to put the caption inside the sub-figure, too:
% \begin{itemize}
% \item The caption could already be part of the image. (This case is already handled in \longref{without}.)
% \item \LaTeX\ packages like \pkg{overpic}, \pkg{stackengine}, or \pkg{tikz} could be used.
% Note that the commands or environments offered by these packages usually evaluate their content more than once,
% resulting in either errors or wrong reference counters.
% For this reason |\caption| should not be used here, instead
% |\phantomcaption| or |\captionlistentry| should be used outside the command or environment and
% |\captiontext| should be used inside it.
% See also \longref{without2}
% \item The options |skip=| and |margin=| could be used to place the caption onto the image.
% \end{itemize}
%
% \pagebreak[3]
% \subsection{Using the overpic package}
% \label{overpic}
%
% The same example as in section \longref{without}, but using the \env{overpic} environment
% offered by the \pkg{overpic} package~\cite{overpic}
% to place the captions inside the pictures:
% \begin{quote}
%   |\usepackage{overpic}|\\
%   \ldots\\
%   |\begin{figure}|\\
%   |  \centering|\\
%   |  \begin{subcaptiongroup}|\\
%   |    |\textcolor{blue}{\cs{subcaptionlistentry}\csmarg{A cat}}\\
%   |    \label{cat}|\\
%   |    \begin{overpic}[width=60pt]{cat}|\\
%   |      \put(40,34){|\textcolor{blue}{\cs{captiontext*}\csmarg{}}|}|\\
%   |    \end{overpic}|\\
%   |    |\textcolor{blue}{\cs{subcaptionlistentry}\csmarg{An elephant}}\\
%   |    \label{elephant}|\\
%   |    \begin{overpic}[width=140pt]{elephant}|\\
%   |      \put(58,40){|\textcolor{blue}{\cs{captiontext*}\csmarg{}}|}|\\
%   |    \end{overpic}|\\
%   |  \end{subcaptiongroup}|\\
%   |  \captionsetup{subrefformat=parens}|\\
%   |  \caption{Two animals: \subref{cat} a huge cat,|\\
%   |           and \subref{elephant} an elephant}|\\
%   |\end{figure}|
% \end{quote}
% \noindent\begin{minipage}{\linewidth}
%   \captionsetup{type=figure}
%   \centering
%   \begin{subcaptiongroup}
%     \subcaptionlistentry{A cat}\label{cat7.1}
%     \begin{overpic}[width=60pt]{cat}
%       \put(40,34){\captiontext*{}}
%     \end{overpic}
%     \subcaptionlistentry{An elephant}\label{elephant7.1}
%     \begin{overpic}[width=140pt]{elephant}
%       \put(58,40){\captiontext*{}}
%     \end{overpic}
%   \end{subcaptiongroup}
%   \captionsetup{subrefformat=parens}
%   \caption{Two animals: \subref{cat7.1} a huge cat, and \subref{elephant7.1} an elephant}
% \end{minipage}
%
% \bigskip
%
% If neither a reference to the sub-figures nor an entry in the List of Figures is needed,
% the usage of |\phantom|\-|caption| or |\caption|\-|list|\-|entry| could be dropped.
% Since this leaves |\caption|\-|text| without a valid sub-figure number value it must be
% given to it explicitly as optional argument.
%
% Furthermore the usage of the |sub|\-|caption|\-|group| environment could be dropped here,
% instead it would be sufficient to replace |\caption|\-|text| with |\sub|\-|caption|\-|text|.
% \See{\longref{subcaption}}
%
% \begin{quote}
%   |\usepackage{overpic}|\\
%   \ldots\\
%   |\begin{figure}|\\
%   |  \centering|\\
%   |  \begin{overpic}[width=60pt]{cat}|\\
%   |    \put(40,34){|\textcolor{blue}{\cs{subcaptiontext*}\csoarg{1}\csmarg{}}|}|\\
%   |  \end{overpic}|\\
%   |  \begin{overpic}[width=140pt]{elephant}|\\
%   |    \put(58,40){|\textcolor{blue}{\cs{subcaptiontext*}\csoarg{2}\csmarg{}}|}|\\
%   |  \end{overpic}|\\
%   |  \caption{Two animals: A huge cat and an elephant}|\\
%   |\end{figure}|
% \end{quote}
%
% \pagebreak[3]
% \subsection{Using the stackengine package}
% \label{stackengine}
%
% The same example as in section \longref{without}, but using the |\stack|\-|inset| command
% offered by the \pkg{stackengine} package~\cite{stackengine}
% to place the captions inside the pictures:
% \begin{quote}
%   |\usepackage{stackengine}|\\
%   \ldots\\
%   |\begin{figure}|\\
%   |  \centering|\\
%   |  \begin{subcaptiongroup}|\\
%   |    |\textcolor{blue}{\cs{subcaptionlistentry}\csmarg{A cat}}\\
%   |    \label{cat}|\\
%   |    \stackinset{l}{25pt}{b}{20pt}{|\textcolor{blue}{\cs{captiontext*}\csmarg{}}|}|\\
%   |      {\includegraphics[width=60pt]{cat}}|\\
%   |    |\textcolor{blue}{\cs{subcaptionlistentry}\csmarg{An elephant}}\\
%   |    \label{elephant}|\\
%   |    \stackinset{l}{80pt}{b}{60pt}{|\textcolor{blue}{\cs{captiontext*}\csmarg{}}|}|\\
%   |      {\includegraphics[width=140pt]{elephant}}|\\
%   |  \end{subcaptiongroup}|\\
%   |  \captionsetup{subrefformat=parens}|\\
%   |  \caption{Two animals: \subref{cat} a huge cat,|\\
%   |           and \subref{elephant} an elephant}|\\
%   |\end{figure}|
% \end{quote}
% \noindent\begin{minipage}{\linewidth}
%   \captionsetup{type=figure}
%   \centering
%   \begin{subcaptiongroup}
%     \subcaptionlistentry{A cat}\label{cat7.2}
%     \stackinset{l}{25pt}{b}{20pt}{\captiontext*{}}{\includegraphics[width=60pt]{cat}}
%     \subcaptionlistentry{An elephant}\label{elephant7.2}
%     \stackinset{l}{80pt}{b}{60pt}{\captiontext*{}}{\includegraphics[width=140pt]{elephant}}
%   \end{subcaptiongroup}
%   \captionsetup{subrefformat=parens}
%   \caption{Two animals: \subref{cat7.2} a huge cat, and \subref{elephant7.2} an elephant}
% \end{minipage}
%
% \bigskip
%
% \pagebreak[3]
% If neither a reference to the sub-figures nor an entry in the List of Figures is needed,
% this code could be simpified to:
% \begin{quote}
%   |\usepackage{stackengine}|\\
%   \ldots\\
%   |\begin{figure}|\\
%   |  \centering|\\
%   |  \stackinset{l}{25pt}{b}{20pt}{|\textcolor{blue}{\cs{subcaptiontext*}\csoarg{1}\csmarg{}}|}|\\
%   |    {\includegraphics[width=60pt]{cat}}|\\
%   |  \stackinset{l}{80pt}{b}{60pt}{|\textcolor{blue}{\cs{subcaptiontext*}\csoarg{2}\csmarg{}}|}|\\
%   |    {\includegraphics[width=140pt]{elephant}}|\\
%   |  \caption{Two animals: A huge cat and an elephant}|\\
%   |\end{figure}|
% \end{quote}
%
% \pagebreak[3]
% \subsection{Using the tikz package}
% \label{tikz}
%
% The same example as in section \longref{without}, but using the \env{tikzpicture} environment
% offered by the \pkg{tikz} package~\cite{tikz}
% to place the captions inside the pictures:
% \begin{quote}
%   |\usepackage{tikz}|\\
%   \ldots\\
%   |\begin{figure}|\\
%   |  \centering|\\
%   |  \begin{subcaptiongroup}|\\
%   |    |\textcolor{blue}{\cs{subcaptionlistentry}\csmarg{A cat}}\\
%   |    \label{cat}|\\
%   |    \begin{tikzpicture}|\\
%   |      \node (cat) at (0,0)|\\
%   |        {\includegraphics[width=60pt]{cat}};|\\
%   |      \node at (0.1,-0.1) {|\textcolor{blue}{\cs{captiontext*}\csmarg{}}|};|\\
%   |    \end{tikzpicture}|\\
%   |    |\textcolor{blue}{\cs{subcaptionlistentry}\csmarg{An elephant}}\\
%   |    \label{elephant}|\\
%   |    \begin{tikzpicture}|\\
%   |      \node (elephant) at (0,0)|\\
%   |        {\includegraphics[width=140pt]{elephant}};|\\
%   |      \node at (0.5,-0.1) {|\textcolor{blue}{\cs{captiontext*}\csmarg{}}|};|\\
%   |    \end{tikzpicture}|\\
%   |  \end{subcaptiongroup}|\\
%   |  \captionsetup{subrefformat=parens}|\\
%   |  \caption{Two animals: \subref{cat} a huge cat,|\\
%   |           and \subref{elephant} an elephant}|\\
%   |\end{figure}|
% \end{quote}
%
% \noindent\begin{minipage}{\linewidth}
%   \captionsetup{type=figure}
%   \centering
%   \begin{subcaptiongroup}
%     \subcaptionlistentry{A cat}\label{cat7.3}
%     \begin{tikzpicture}
%       \node (cat) at (0,0) {\includegraphics[width=60pt]{cat}};
%       \node at (0.1,-0.1) {\captiontext*{}};
%     \end{tikzpicture}
%     \subcaptionlistentry{An elephant}\label{elephant7.3}
%     \begin{tikzpicture}
%       \node (cat) at (0,0) {\includegraphics[width=140pt]{elephant}};
%       \node at (0.5,-0.1) {\captiontext*{}};
%     \end{tikzpicture}
%   \end{subcaptiongroup}
%   \captionsetup{subrefformat=parens}
%   \caption{Two animals: \subref{cat7.3} a huge cat, and \subref{elephant7.3} an elephant}
% \end{minipage}
%
% \bigskip
%
% \pagebreak[3]
% If neither a reference to the sub-figures nor an entry in the List of Figures is needed,
% this code could be simpified to:
% \begin{quote}
%   |\usepackage{tikz}|\\
%   \ldots\\
%   |\begin{figure}|\\
%   |  \centering|\\
%   |  \begin{tikzpicture}|\\
%   |    \node (cat) at (0,0)|\\
%   |      {\includegraphics[width=60pt]{cat}};|\\
%   |    \node at (0.1,-0.1) {|\textcolor{blue}{\cs{subcaptiontext*}\csoarg{1}\csmarg{}}|};|\\
%   |  \end{tikzpicture}|\\
%   |  \begin{tikzpicture}|\\
%   |    \node (elephant) at (0,0)|\\
%   |      {\includegraphics[width=140pt]{elephant}};|\\
%   |    \node at (0.5,-0.1) {|\textcolor{blue}{\cs{subcaptiontext*}\csoarg{2}\csmarg{}}|};|\\
%   |  \end{tikzpicture}|\\
%   |  \caption{Two animals: A huge cat and an elephant}|\\
%   |\end{figure}|
% \end{quote}
%
% \pagebreak[3]
% \subsection{Using the skip and margin options}
% For a particular sub-caption the |skip=|\ldots could be set to a negative value so it will overlap with the image.
% Combined with |singlelinecheck=off| (to switch off the centering of short captions) and |margin=|\ldots it could be placed
% at a specific horizontal position within the image, too.
%
% \pagebreak[3]
% The same example as in section \longref{without}, but using the |skip=|\ldots and |margin=|\ldots options:
% \begin{quote}
%   |\begin{figure}|\\
%   |  \centering|\\
%   |  \subcaptionsetup[figure]|\\
%   |    {skip=-28pt,slc=off,margin={25pt,0pt}}|\\
%   |  \subcaptionbox{\label{cat}}|\\
%   |    {\includegraphics[width=60pt]{cat}}|\\
%   |  \subcaptionsetup[figure]|\\
%   |    {skip=-60pt,slc=off,margin={80pt,0pt}}|\\
%   |  \subcaptionbox{\label{elephant}}|\\
%   |    {\includegraphics[width=140pt]{elephant}}|\\
%   |  \captionsetup{subrefformat=parens}|\\
%   |  \caption{Two animals: \subref{cat} a huge cat,|\\
%   |           and \subref{elephant} an elephant}|\\
%   |\end{figure}|
% \end{quote}
%
% \noindent\begin{minipage}{\linewidth}
%   \captionsetup{type=figure}
%   \centering
%   \captionsetup[subfigure]{skip=-28pt,slc=off,margin={25pt,0pt}}
%   \subcaptionbox{\label{cat7.4}}{\includegraphics[width=60pt]{cat}}
%   \captionsetup[subfigure]{skip=-60pt,slc=off,margin={80pt,0pt}}
%   \subcaptionbox{\label{elephant7.4}}{\includegraphics[width=140pt]{elephant}}
%   \captionsetup{subrefformat=parens}
%   \caption{Two animals: \subref{cat7.4} a huge cat, and \subref{elephant7.4} an elephant}
% \end{minipage}
%
% \iffalse
% --------------------------------------------------------------------------- %
% \fi
%
% \PageBreak
% \section{Numbering}
%
% \subsection{Pittfall \#1: Using multiple main captions}
%
% When multiple main captions are used within a figure or table, and sub-captions
% are used as well, how does the \pkg{subcaption} package know which
% sub-captions belong to which caption, i.e. what is the main counter value for
% the sub-captions?
%
% \pagebreak[3]
% Let's illustrate this problem with an example document:
% \begin{quote}
%   |\documentclass{article}|\\
%   |\usepackage{graphicx,subcaption}|\\
%   |\begin{document}|\\
%   |\begin{figure}|\\
%   |  \centering|\\
%   |  \includegraphics[width=4cm]{example-image-c}|\\
%   |  \caption{Caption no. 1}|\\
%   |  \bigskip|\\
%   |  \subcaptionbox|\\
%   |    {\label{fig:2a}}|\\
%   |    {\includegraphics[width=2cm]{example-image-a}}|\\
%   |  \subcaptionbox|\\
%   |    {\label{fig:2b}}|\\
%   |    {\includegraphics[width=2cm]{example-image-b}}|\\
%   |  \caption{Caption no. 2}|\\
%   |\end{figure}|\\
%   |Look at sub-figures \ref{fig:2a} and \ref{fig:2b}.|\\
%   |\end{document}|
% \end{quote}
%
% It's obvious that the sub-captions belong to 2nd caption, and therefore
% |\ref{fig:2a}| will become ``2a'', isn't it?
% But since the \pkg{subcaption} package is only involved when using
% \LaTeX\ commands either defined or patched by the \pkg{caption} or
% \pkg{subcaption} package, this is what the \pkg{caption} package
% is aware of:
% \begin{quote}
%   |\usepackage{subcaption}|\\
%   |\begin{document}|\\
%   |\begin{figure}|\\
%   |  \caption{Caption no. 1}|\\
%   |  \subcaptionbox|\\
%   |    {\label{fig:2a}}|\\
%   |    {|\meta{unknown content}|}|\\
%   |  \subcaptionbox|\\
%   |    {\label{fig:2b}}|\\
%   |    {|\meta{unknown content}|}|\\
%   |  \caption{Caption no. 2}|\\
%   |\end{figure}|\\
%   |\end{document}|
% \end{quote}
%
% So from \pkg{caption}s point of view it's not easy to decide if the
% sub-captions belong to the 1st or 2nd main caption since they are placed
% between them. (Note: The \pkg{subcaption} package is only offering
% an user interface to the sub-caption feature of the \pkg{caption} package,
% and therefore this decision is the responsibility of the \pkg{caption} package.)
%
% But how does the \pkg{caption} package makes a decision?
% If in doubt, it rather clings to the past than to the future, i.e.~in this
% case it decides that the sub-captions belong to the 1st caption and therefore
% the result of |\ref{fig:2a}| is not ``2a'' but ``1a''.
% If the \pkg{caption} package is unsure about its decision (like in this
% case), a warning will be issued:
% \begingroup
% \leftmargini=12pt
% \begin{quote}
%   |Package caption Warning: Ambiguous sub-caption(s),|\\
%   |                         use \nextfloat on input line 15.|\\
%   |See the caption package documentation for explanation.|
% \end{quote}
% \endgroup
%
% \pagebreak[3]
% How to fix it? Use |\next|\-|float| to tell the \pkg{caption} package
% where the 2nd figure within the |figure| environment starts:
% \begin{quote}
%   |\documentclass{article}|\\
%   |\usepackage{graphicx,subcaption}|\\
%   |\begin{document}|\\
%   |\begin{figure}|\\
%   |  \centering|\\
%   |  \includegraphics[width=4cm]{example-image-c}|\\
%   |  \caption{Caption no. 1}|\\
%   |  \bigskip|\\
%   ||\textcolor{blue}{\cs{nextfloat}}\\
%   |  \subcaptionbox|\\
%   |    {\label{fig:2a}}|\\
%   |    {\includegraphics[width=2cm]{example-image-a}}|\\
%   |  \subcaptionbox|\\
%   |    {\label{fig:2b}}|\\
%   |    {\includegraphics[width=2cm]{example-image-b}}|\\
%   |  \caption{Caption no. 2}|\\
%   |\end{figure}|\\
%   |Look at sub-figures \ref{fig:2a} and \ref{fig:2b}.|\\
%   |\end{document}|
% \end{quote}
%
% \INFO
% Note that the |\next|\-|float| command was introduced in \pkg{caption} package
% \version{3.6}.
% Previous versions of the \pkg{caption} package have bound the decision to the
% |position=| setting of the floating environment instead, i.e.~sub-captions belonged to
% the caption above if |position=top| was set and they belonged to the caption below
% if |position=below| was set.
% While this would be beneficial in this case, there were several cases were it was not
% and especially it was not always comprehensible to the user why his references got an
% incorrect numbering.
%
% Furthermore the new decision algorithm always succeeds if there is only one caption
% within the figure or table (which is the case most of the time) while the old one did not.
%
% \pagebreak[3]
% If you still prefer the old decision algorithm (for example because you want to process
% an already existing document), you need to specify the \pkg{caption} package version
% explicitly, for example:
% \begin{quote}
%   |\documentclass{article}|\\
%   |\usepackage{caption}[=v3.5]|\\
%   |\usepackage{graphicx,subcaption}|\\
%   \ldots
% \end{quote}
%
% See also: \longref{sec:caption}
%
% \subsection{Pittfall \#2: Expecting \cs{caption} to increment the counter}
%
% \pagebreak[3]
% Usually |\caption| increments the figure or table counter and therefore
% it is usally safe to assume that the counter was not incremented yet in code
% used before |\caption|:
% \begin{quote}
%   |\documentclass{article}|\\
%   |\begin{document}|\\
%   |\begin{figure}|\\
%   |  Figure counter before caption: \thefigure|\\
%   |  \caption{Some text}|\\
%   |  Figure counter after caption: \thefigure|\\
%   |\end{figure}|\\
%   |\end{document}|
% \end{quote}
%
% Unsurprisingly the results are 0 and 1 for the counter values.
%
% \pagebreak[3]
% But this happens if we use the \pkg{subcaption} package:
% \begin{quote}
%   |\documentclass{article}|\\
%   |\usepackage{subcaption}|\\
%   |\begin{document}|\\
%   |\begin{figure}|\\
%   |  \subcaptionbox{}{some content}|\\
%   |  \subcaptionbox{}{some content}|\\
%   |  Figure counter before caption: \thefigure|\\
%   |  \caption{Some text}|\\
%   |  Figure counter after caption: \thefigure|\\
%   |\end{figure}|\\
%   |\end{document}|
% \end{quote}
%
% Here the results are 1 and 1 for the counter values. Why?
%
% Since the \pkg{caption} package assumes that there will be a |\caption|
% following |\sub|\-|caption|\-|box| it decides that the sub-captions belong
% to the upcoming main caption and therefore share the same main counter value
% which needs to be incremented before its use.
% Therefore the first |\sub|\-|caption|\-|box| in the figure increments the
% figure counter while the second one and especially the |\caption| does not.
%
% Usually this is no problem at all but keep this in mind if you are doing
% tricky stuff with the figure or table counter within figures or tables.
%
% \iffalse
% --------------------------------------------------------------------------- %
% \fi
%
% \PageBreak
% \section{Abbreviatory commands}
% \label{subcaption}
%
% \DescribeMacro\subcaption
% \DescribeMacro\phantomsubcaption
% \DescribeMacro\subcaptionlistentry
% \DescribeMacro\subcaptiontext
% \hskip1pt %^^A work-around for bug in \DescribeMacro to prevent red colored margin note
% As we have seen in sections {\longref{overpic}, \longref{stackengine}, and \longref{tikz}}
% it's sometimes inconvenient to use the commands and environments
% described so far.
%
% \pagebreak[3]
% For example it would be inconvenient to use \env{captiongroup} just for
% a single |\caption|\-|text| in this particular case:
% \begin{quote}
%   \ldots\\
%   |\begin{overpic}[width=60pt]{cat}|\\
%   |  \put(40,34)|\\
%   |    {\begin{captiongroup*}|\\
%   |       \captiontext*[1]{}|\\
%   |     \end{captiongroup*}}|\\
%   |\end{overpic}|\\
%   \ldots
% \end{quote}
%
% \pagebreak[3]
% Since |\caption|\-|text| is already used in an extra environment (|overpic|)
% we don't need an extra |caption|\-|group*| here,
% instead using |\set|\-|caption|\-|sub|\-|type*| (which switches into the
% sub-caption mode without making an \pkg{hyperref} anchor) would be sufficient:
% \begin{quote}
%   \ldots\\
%   |\begin{overpic}[width=60pt]{cat}|\\
%   |  \put(40,34)|\\
%   |    {\setcaptionsubtype*|\\
%   |     \captiontext*[1]{}}|\\
%   |\end{overpic}|\\
%   \ldots
% \end{quote}
%
% \pagebreak[3]
% But this is still inconvenient when used many times, and therefore \thispackage\ defines
% several extra commands which are prefixed with |\set|\-|caption|\-|sub|\-|type*|:
%
% \medskip
%
% \begin{minipage}{\textwidth}
% \centering
% \begin{tabular}{l|l|l}
%   Regular command     & using \cs{setcaptionsubtype*} & available since \\
%   \hline
%   |\caption|          & |\subcaption|                 & \version{1.0} \\
%   |\phantomcaption|   & |\phantomsubcaption|          & \version{1.1} \\
%   |\captionlistentry| & |\subcaptionlistentry|        & \version{1.5} \\
%   |\captiontext|      & |\subcaptiontext|             & \version{1.5} \\
% \end{tabular}
% \end{minipage}
%
% \bigskip
%
% \pagebreak[3]
% This way our code snipped above could be simplified to:
% \begin{quote}
%   \ldots\\
%   |\begin{overpic}[width=60pt]{cat}|\\
%   |  \put(40,34){\subcaptiontext*[1]{}}|\\
%   |\end{overpic}|\\
%   \ldots
% \end{quote}
%
% \INFO
% Since |\set|\-|caption|\-|sub|\-|type| should only be used within an
% extra group or environment, the same applies to these commands as well.
% (In this case |\sub|\-|caption|\-|text| is encapsulated by the
% |overpic| environment, so we are ok here.
%
% \begin{background}
% The |\sub|\-|caption| command is just a simple combination of
% |\set|\-|caption|\-|sub|\-|type*| and |\caption|. Same for all other
% abbreviatory commands here.
% \end{background}
%
% \iffalse
% --------------------------------------------------------------------------- %
% \fi
%
% \PageBreak
% \section{The \cs{subfloat} command}
% \label{subfloat}
%
% \DescribeMacro\subfloat
% \NEWfeature*{v1.3}
% To allow a smoother transition from the \pkg{subfig} package~\cite{subfig}
% (which is unmaintained for over 16 years)
% this package also offers |\sub|\-|float| with the same syntax:
% \begin{quote}
%   |\subfloat|\oarg{list\_entry}\oarg{sub-caption}\marg{body}
% \end{quote}
%
% \iffalse
% --------------------------------------------------------------------------- %
% \fi
%
% \section{Required packages}
% \label{sec:caption}
%
% \NEWfeature{v1.4}
% Starting with version $1.4$ the \pkg{subcaption} package requires at least version $3.1$
% of the \pkg{caption} package and loads it automatically.
% (Older versions of the \pkg{subcaption} package have required exactly the version of the
% \pkg{caption} package which was released with it.)
%
% \pagebreak[3]
% If you need to use a specific version of the \pkg{caption} package you need to load it
% \emph{before} the \pkg{subcaption} package, e.g.:
% \begin{quote}
%   |\usepackage[| \ldots |]{caption}[=v3.5]|\\
%   |\usepackage[| \ldots |]{subcaption}|
% \end{quote}
%
% \pagebreak[3]
% Note that there are limitations if an older version of the \pkg{caption} package
% is used:
% \begin{itemize}
%   \item The |\phantomsubcaption| command needs at least \pkg{caption} \version{3.2}.
%   \item The |\subcaptionlistentry| command needs at least \pkg{caption} \version{3.3}.
%   \item The |\subfloat| emulation needs at least \pkg{caption} \version{3.4}.
%   \item The |\subcaptiontext| command needs at least \pkg{caption} \version{3.6}.
%   \item The |\subcaptionsetup| command needs at least \pkg{caption} \version{3.6l}.
% \end{itemize}
%
% \iffalse
% --------------------------------------------------------------------------- %
% \fi
%
% \section{Other packages with similar offerings}
% \label{sec:otherpackages}
%
% The \pkg{subcaption} package is not the only one defining sub-figure counters and offering commands or environments to typeset sub-figures.
% The most established are:
% \begin{itemize}
% \item The \cls{memoir} document class
% \item The \pkg{subfigure} package and its successor, the \pkg{subfig} package
% \end{itemize}
%
% \NEWfeature{v1.6}
% Prior subcaption version $1.6$ an error message was issued if the `subfigure' or `subtable' counter was already defined by any of them.
% Since subcaption version $1.6$ the `subfigure' and `subtable' counters will only be defined by the \pkg{subcaption} package if they are not defined yet,
% but the usage of any of the commands or environments described in this document will result in an error message ``The counter sub\ldots was defined by \ldots''
% if the corresponding counter was not defined by the \pkg{subcaption} package.
% Equally the commands or environments offered by the above document classes or packages should not be used with counters defined by the \pkg{subcaption} package,
% this will result in either \LaTeX\ errors or unwanted behaviour.
%
% So the rule of thumb is:
% If the counter `subfigure' or `subtable' is defined by document class or package \textit{xyz},
% then the corresponding commands and environments offered by the document class or package \textit{xyz} should be used.
%
% \pagebreak[3]
% An example document:
% \begingroup\small\leftmargini=12pt\relax
% \begin{quote}
%   |\documentclass{memoir}|\\
%   |\newsubfloat{table}      % defines the `subtable' counter|\\
%   |\usepackage{subcaption}  % defines the `subfigure' counter|\\
%   |\begin{document}|\\
%   |\begin{figure}|\\
%   |\subcaptionbox{...}{...} % <- This is fine|\\
%   |\caption{...}|\\
%   |\end{figure}|\\
%   |\begin{table}|\\
%   |\caption{...}|\\
%   |\subcaptionbox{...}{...} % <- This results in an error,|\\
%   |\end{table}                   \subtop should be used instead|\\
%   |\end{document}|\\
% \end{quote}
% \endgroup
%
% \iffalse
% --------------------------------------------------------------------------- %
% \fi
%
% \PageBreak
% \section{Beyond this package}
% \label{keyfloat}\label{floatrow}
%
% For a more advanced usage of the sub-caption feature of the
% \pkg{caption} package, please take a look at the excellent
% \pkg{keyfloat} package\cite{keyfloat} which provides the environments
% |key|\-|sub|\-|figs|, |key|\-|sub|\-|tabs|, and |key|\-|sub|\-|floats| for
% typesetting sub-figures and sub-tables.
%
% Furthermore the \pkg{floatrow} package\cite{floatrow} provides the
% |sub|\-|float|\-|row| environment for typesetting sub-figures.
%
% \iffalse
% --------------------------------------------------------------------------- %
% \fi
%
% \PageBreak
% \section{Thanks}
%
% I would like to thank
% Stephen Dalton
% who helped to make this package a better one.
%
% \iffalse
% --------------------------------------------------------------------------- %
% \fi
%
% \StopEventually{%^^A
% \begin{thebibliography}{9}
%
%   \bibitem{TLC2}
%   Frank Mittelbach and Michel Goossens:\\
%   \newblock {\em The {\LaTeX} Companion (2nd.~Ed.)},\\
%   \newblock Addison-Wesley, 2004.
%
%   \bibitem{caption}
%   Axel Sommerfeldt:\\
%   \href{http://www.ctan.org/pkg/caption}%
%        {\emph{Customizing captions of floating environments}},\\
%   2022/01/07
%
%   \bibitem{floatrow}
%   Olga Lapko:\\
%   \href{http://www.ctan.org/pkg/floatrow}%
%        {\emph{The floatrow package documentation}},\\
%   2007/12/24
%
%   \bibitem{hyperref}
%   Sebastian Rahtz \& Heiko Oberdiek:\\
%   \href{http://www.ctan.org/pkg/hyperref}%
%        {\emph{Hypertext marks in \LaTeX}},\\
%   November 12, 2007
%
%   \bibitem{hypcap}
%   Heiko Oberdiek:\\
%   \href{http://www.ctan.org/pkg/hypcap}%
%        {\emph{The hypcap package -- Adjusting anchors of captions}},\\
%   2007/04/09
%
%   \bibitem{keyfloat}
%   Brian Duun:\\
%   \href{http://www.ctan.org/pkg/keyfloat}%
%        {\emph{The \LaTeX\ keyfloat Package}},\\
%   2019/09/23
%
%   \bibitem{overpic}
%   Rolf Niepraschk:\\
%   \href{http://www.ctan.org/pkg/overpic}%
%        {\emph{The overpic package}},\\
%   2020/02/22
%
%   \bibitem{stackengine}
%   Steven B. Segletes:\\
%   \href{http://www.ctan.org/pkg/stackengine}%
%        {\emph{The stackengine Package}},\\
%   July 22, 2021
%
%   \bibitem{tikz}
%   Till Tantau:\\
%   \href{http://www.ctan.org/pkg/tikz}%
%        {\emph{The TikZ and PGF Packages}},\\
%   May 15, 2021
%
%   \bibitem{subfig}
%   Steven D. Cochran:\\
%   \href{http://www.ctan.org/pkg/subfig}%
%        {\emph{The subfig package}},\\
%   2005/07/05
%
% \end{thebibliography}
% }
%
% \iffalse
% --------------------------------------------------------------------------- %
% \fi
%
% \DoNotIndex{\\,\_,\ ,\@@par}
% \DoNotIndex{\@bsphack}
% \DoNotIndex{\@car,\@cdr,\@classoptionslist,\@cons,\@currext,\@currname}
% \DoNotIndex{\@ehc,\@ehd,\@empty,\@esphack,\@expandtwoargs}
% \DoNotIndex{\@for,\@firstofone,\@firstoftwo}
% \DoNotIndex{\@gobble,\@gobblefour,\@gobbletwo,\@hangfrom}
% \DoNotIndex{\if@minipage,\@ifnextchar,\@ifpackagelater,\@ifpackageloaded}
% \DoNotIndex{\@ifstar,\@ifundefined,\@latex@error,\@namedef,\@nameuse}
% \DoNotIndex{\@onlypreamble,\@parboxrestore,\@plus,\@ptionlist}
% \DoNotIndex{\@removeelement,\@restorepar,\@secondoftwo,\@setpar}
% \DoNotIndex{\@tempa,\@tempboxa,\@tempdima,\@tempdimb,\@tempdimc,\@tempb,\@tempc}
% \DoNotIndex{\@testopt}
% \DoNotIndex{\@undefined,\@unprocessedoptions,\@unusedoptionlist}
% \DoNotIndex{\p@,\z@}
% \DoNotIndex{\active,\addtocounter,\addtolength,\advance,\aftergroup}
% \DoNotIndex{\baselineskip,\begin,\begingroup,\bfseries,\box}
% \DoNotIndex{\catcode,\centering,\changes,\csname,\def,\divide,\do,\downarrow}
% \DoNotIndex{\edef,\else,\empty,\end,\endcsname,\endgraf,\endgroup,\expandafter}
% \DoNotIndex{\fi,\footnotesize,\global}
% \DoNotIndex{\hangindent,\hbox,\hfil,\hsize,\hskip,\hspace,\hss}
% \DoNotIndex{\ifcase,\ifdim,\ifnum,\ifodd,\ifvoid,\ifvmode}
% \DoNotIndex{\ifx,\ignorespaces,\itshape}
% \DoNotIndex{\kernel@ifnextchar}
% \DoNotIndex{\Large,\large,\leavevmode,\leftmargini,\leftskip,\let,\linewidth}
% \DoNotIndex{\llap,\long,\m@ne,\margin,\mdseries,\message}
% \DoNotIndex{\newcommand,\newdimen,\newlength,\newline,\newif,\newsavebox}
% \DoNotIndex{\next,\nobreak,\nobreakspace,\noexpand,\noindent,\numberline}
% \DoNotIndex{\normalcolor,\normalfont,\normalsize,\or,\par,\parbox,\parfillskip}
% \DoNotIndex{\parindent,\parskip,\prevdepth,\protect,\protected@edef,\protected@write}
% \DoNotIndex{\providecommand,\quad}
% \DoNotIndex{\raggedleft,\raggedright,\relax,\renewcommand,\RequirePackage}
% \DoNotIndex{\rightskip,\rmfamily}
% \DoNotIndex{\sbox,\scriptsize,\scshape,\setbox,\setlength,\sffamily,\slshape}
% \DoNotIndex{\small,\string,\space,\strut}
% \DoNotIndex{\textheight,\the,\toks@,\typeout,\ttfamily}
% \DoNotIndex{\unvbox,\uparrow,\upshape,\usebox,\usepackage}
% \DoNotIndex{\value,\vbox,\vsize,\vskip,\wd,\width,\z@skip}
% \DoNotIndex{\AtBeginDocument,\AtEndOfPackage,\CurrentOption,\DeclareOption}
% \DoNotIndex{\ExecuteOptions,\GenericWarning,\IfFileExists,\InputIfFileExists}
% \DoNotIndex{\NeedsTeXFormat,\MessageBreak}
% \DoNotIndex{\PackageError,\PackageInfo,\PackageWarning,\PackageWarningNoLine}
% \DoNotIndex{\PassOptionsToPackage,\ProcessOptions,\ProvidesPackage}
%
% \iffalse
% --------------------------------------------------------------------------- %
% \fi
%
% \setlength{\parskip}{0pt plus 1pt}
% \changes{v0.1}{2007/09/01}{First demo}
% \changes{v0.2}{2007/11/11}{\cs{subcaptionbox} added}
% \changes{v0.3}{2007/12/06}{Adapted to \pkg{caption} package \version{3.1f}}
% \changes{v1.0}{2008/03/16}{\cs{subfloat} added}
% \changes{v1.0}{2010/10/27}{An error message will be issued when the subfigure or subfig package is loaded}
% \changes{v1.0}{2011/01/22}{Undocumented command \cs{subfloat} removed}
%
% \newcommand*\Note[2][Note]{\par{\small\emph{#1:} #2}}
%
% \iffalse
% --------------------------------------------------------------------------- %
% \fi
%
% \clearpage
% \section{The implementation}
%
% \iffalse
%<*package>
% \fi
%
% \subsection{Identification}
%
% We need at least \LaTeX2e\ version 1994/12/01.
%    \begin{macrocode}
\NeedsTeXFormat{LaTeX2e}[1994/12/01]
%    \end{macrocode}
%
% Bypass the release declarations in case the \LaTeX\ kernel doesn’t know how to deal with them
% (as suggested by \url{https://www.latex-project.org/publications/2018-FMi-TUB-tb122mitt-version-rollback.pdf}).
%    \begin{macrocode}
\providecommand\DeclareRelease[3]{}
\providecommand\DeclareCurrentRelease[2]{}
%    \end{macrocode}
%
% Declare all supported releases.
%    \begin{macrocode}
\DeclareCurrentRelease{v1}{2007/12/06}
%    \end{macrocode}
%
% Identify the current version of the package.
%    \begin{macrocode}
\ProvidesPackage{subcaption}[2023/07/28 v1.6b Sub-captions (AR)]
%    \end{macrocode}
%
% Since we base on the \pkg{caption} package we load it here.
%    \begin{macrocode}
\RequirePackage{caption}[2010/01/09] % we need at least v3.1m
%    \end{macrocode}
%
% \subsection{Initial code}
%
% \begin{macro}{\subcaption@Info}
% \changes{v1.6}{2022/04/07}{This macro added}
%   |\subcaption@Info|\marg{message}\\
%   issues an info message (with code line indication).
%    \begin{macrocode}
\newcommand*\subcaption@Info{%
  \PackageInfo{subcaption}}
%    \end{macrocode}
% \end{macro}
%
% \begin{macro}{\subcaption@Warning}
% \changes{v1.5}{2022/01/06}{This macro added}
%   |\subcaption@Warning|\marg{message}\\
%   issues an warning message (with code line indication).
%    \begin{macrocode}
\newcommand*\subcaption@Warning{%
  \PackageWarning{subcaption}}
%    \end{macrocode}
% \end{macro}
%
% \begin{macro}{\subcaption@Error}
% \changes{v1.3h}{2020/09/28}{This macro added}
%   |\subcaption@Error|\marg{message}\\
%   issues an error message (with code line indication).
%    \begin{macrocode}
\newcommand*\subcaption@Error[1]{%
  \PackageError{subcaption}{#1}{\caption@@eh{subcaption}}}
%    \end{macrocode}
%    \begin{macrocode}
\providecommand*\caption@@eh[1]{%
  If you do not understand this error, please take a closer look\MessageBreak
  at the documentation of the `#1' package, especially the\MessageBreak
  section about errors.\MessageBreak\@ehc}
%    \end{macrocode}
% \end{macro}
%
% \begin{macro}{\subcaption@OutsideFloat}
% \changes{v1.5}{2022/01/07}{This macro definition added}
% |\subcaption@OutsideFloat}| issues a ``\meta{command} outside float'' error.
%    \begin{macrocode}
\newcommand*\subcaption@OutsideFloat[1]{%
  \subcaption@Error{\string#1 outside float}}
%    \end{macrocode}
% \end{macro}
%
% \subsection{Main code}
%
% \begin{macro}{\setcaptionsubtype}
% \changes{v1.4}{2020/12/24}{This macro definition added}
% \changes{v1.5}{2022/01/07}{Check added if used inside a floating environment}
% |\setcaptionsubtype| is available since \pkg{caption} package \version{3.2},
% so we need to define it first if only an older version is loaded (in fallback mode).
%    \begin{macrocode}
\providecommand\setcaptionsubtype{%
  \caption@iftype
    {\@ifstar{\captionsetup{subtype*}}{\captionsetup{subtype}}}%
    {\subcaption@OutsideFloat\setcaptionsubtype}}
%    \end{macrocode}
% \end{macro}
%
% \subsubsection{The \env{subcaptiongroup} environment}
%
% \begin{environment}{subcaptiongroup}
% \changes{v1.5}{2022/01/07}{This environment definition added}
% \env{subcaptiongroup} is the simplest of all environments or commands this package has to offer.
% It's simply an environment containing |\set|\-|caption|\-|sub|\-|type|, that's all.
% A starred variant of this environment is available, too, which uses |\set|\-|caption|\-|sub|\-|type*|.
%    \begin{macrocode}
\newenvironment{subcaptiongroup}
  {\caption@iftype
     {\setcaptionsubtype\relax}%
     {\subcaption@OutsideFloat{subcaptiongroup}}}
  {}
%    \end{macrocode}
%    \begin{macrocode}
\newenvironment{subcaptiongroup*}
  {\caption@iftype
     {\setcaptionsubtype*}%
     {\subcaption@OutsideFloat{subcaptiongroup*}}}
  {}
%    \end{macrocode}
% \end{environment}
%
% \subsubsection{The \env{subcaptionblock} environment (and aliases)}
%
% \begin{macro}{\subcaption@minipage}
% \changes{v1.2}{2016/02/21}{Optional argument values `B' and `T' added}
% \changes{v1.3d}{2020/01/22}{Definition and usage of \cs{@subfloatboxreset} added}
% \changes{v1.5}{2022/01/07}{This macro definition out-sourced from \cs{subcaption@newminipage}}
% \changes{v1.5}{2022/01/07}{Usage of \cs{@ifnextchar} replaced by \cs{@testopt}}
% \changes{v1.5}{2022/01/07}{Default value of \meta{outer-pos} changed from `c' to `b'}
% \changes{v1.5a}{2022/04/16}{Default value of \meta{outer-pos} changed from `b' to either `b' or `t'}
% This is a \env{minipage} with |\set|\-|caption|\-|sub|\-|type| as first contents line.
%    \begin{macrocode}
\newcommand*\subcaption@minipage{%
  \kernel@ifnextchar[%]
    \subcaption@iminipage
    {\caption@ifcaption{\subcaption@iminipage[t]}{\subcaption@iminipage[b]}}}
%    \end{macrocode}
% We pass all other optional arguments using the generic helper macro
% |\caption@with|\-|opt|\-|args| offered by the \pkg{caption} kernel).
%    \begin{macrocode}
\def\subcaption@iminipage[#1]{%
% \caption@withoptargs{\subcaption@iiminipage{#1}}}  % would need at least caption3 v1.5
  \def\subcaption@tempa{\subcaption@iiminipage{#1}}%
  \caption@withoptargs\subcaption@tempa}
%    \end{macrocode}
% `B' and `T' will add a |\vspace{0pt}|, all other values (and additional
% optional arguments) will be passed unseen to the |minipage| environment.
%    \begin{macrocode}
\newcommand*\subcaption@iiminipage[3]{%
  \let\subcaption@endminipage@hook\@empty
  \if#1B%
    \minipage[b]#2{#3}%
    \def\subcaption@endminipage@hook{\vspace{0pt}}%
  \else\if#1T%
    \minipage[t]#2{#3}%
    \vspace{0pt}%
  \else
    \minipage[#1]#2{#3}%
  \fi\fi
  \@subfloatboxreset
  \setcaptionsubtype\relax}
%    \end{macrocode}
%    \begin{macrocode}
\newcommand*\subcaption@endminipage{%
  \subcaption@endminipage@hook
  \endminipage}
%    \end{macrocode}
%    \begin{macrocode}
\providecommand*\@subfloatboxreset{}
%    \end{macrocode}
%    \begin{macrocode}
\providecommand*\caption@ifcaption{\caption@ifflag2} % caption >= v3.3 < v3.6
\providecommand*\caption@ifflag[1]{\@secondoftwo}    % caption < v3.3
%    \end{macrocode}
% \end{macro}
%
% \begin{environment}{subcaptionblock}
% \changes{v1.5}{2022/01/07}{This environment definition added}
% \env{subcaptionblock} is a \env{minipage} with |\set|\-|caption|\-|sub|\-|type| as first contents line.
%    \begin{macrocode}
\newenvironment{subcaptionblock}{\subcaption@minipage}{\subcaption@endminipage}
%    \end{macrocode}
% \end{environment}
%
% \begin{macro}{\subcaption@newminipage}
% \changes{v1.4d}{2022/01/04}{Test of environment name added, should match `sub\cs{@captype}'}
% \changes{v1.5}{2022/01/06}{Test of environment name revised, results in a warning now instead of an error}
% \changes{v1.5}{2022/01/07}{This command renamed from \cs{subcaption@newenvironment} to \cs{subcaption@newminipage} and revised}
% |\subcaption@newminipage|\marg{name of new environment}\\
% defines a new environment containing |\sub|\-|caption@mini|\-|page| and |\sub|\-|caption@end|\-|mini|\-|page|.
% Furthermore a test will be included which checks if the environment name matches |sub\@captype|.
%    \begin{macrocode}
\newcommand*\subcaption@newminipage[1]{%
  \newenvironment{#1}{\subcaption@minipage@{#1}}{\subcaption@endminipage}}
%    \end{macrocode}
%    \begin{macrocode}
\newcommand*\subcaption@minipage@[1]{%
  \caption@iftype
    {\edef\caption@tempa{#1}%
     \edef\caption@tempb{sub\@captype}%
     \ifx\caption@tempa\caption@tempb \else
       \subcaption@Warning{%
         `\caption@tempa' is treated as `\caption@tempb'\MessageBreak}%
     \fi}%
    {\subcaption@OutsideFloat{#1}}%
  \subcaption@minipage}
%    \end{macrocode}
% \end{macro}
%
% \begin{environment}{subfigure}
% \begin{environment}{subtable}
% \changes{v1.3e}{2020/07/29}{Uses \cs{ForEachCaptionSubType} (instead of \cs{caption@For}) to define the environments now}
% The sub-environments will be defined using the helper macro |\For|\-|Each|\-|Caption|\-|Sub|\-|Type|
% offered by the \pkg{caption} kernel v1.13 (2020/07/29), so for every caption sub-type
% declared with |\Declare|\-|Caption|\-|Sub|\-|Type| a corresponding `sub' environment will be defined automatically.
% (If the \pkg{caption} kernel v1.13 is not available we use \cs{caption@For} as fallback.)
%    \begin{macrocode}
\@ifundefined{ForEachCaptionSubType}  % caption3 v1.13
  {\caption@For{subtypelist}{\subcaption@newminipage{sub#1}}}
  {\ForEachCaptionSubType{\subcaption@newminipage{#1}}}
%    \end{macrocode}
% \end{environment}
% \end{environment}
%
% \subsubsection{The \cs{subcaptionbox} command}
%
% \begin{macro}{\subcaptionbox}
% \changes{v1.0}{2008/05/06}{Adapted to the \opt{rule} option of the \pkg{caption} package}
% \changes{v1.0}{2008/08/31}{Definition and usage of \cs{subcaption@hrule} added}
% \changes{v1.0}{2010/12/17}{Uses \cs{caption@box} now}
% \changes{v1.1}{2011/08/16}{Adapted to actual version of \cs{caption@box}}
% \changes{v1.1}{2012/04/09}{Adapted to actual version of \cs{caption@ibox}}
% \changes{v1.4}{2020/12/24}{Fallback definition of \cs{caption@ibox} added}
% A |\parbox| with contents and sub-caption, separated by an invisible |\hrule|.\par
% The code of this macro was moved to the \pkg{caption} package \version{3.2}
% (so it could offer it as |\caption|\-|box| as well), so since then it's
% sufficient to simply use |\caption@ibox| here.\par
% But if only \pkg{caption} package \version{3.1} is available (since the user
% decided to use this particular fallback version), we still must define the code
% on our own.
%    \begin{macrocode}
\@ifundefined{caption@ibox}{%
%    \end{macrocode}
%    \begin{macrocode}
  \newcommand*\subcaptionbox{%  caption v3.1
    \def\subcaption@tempa{\caption@ibox\setcaptionsubtype\relax}%
    \caption@withoptargs\subcaption@tempa}
%    \end{macrocode}
%    \begin{macrocode}
  \newcommand\caption@ibox[3]{%
    \kernel@ifnextchar[%]
      {\caption@iibox{#1}{#2}{#3}}%
      {\caption@iibox@{#1}{#2}{#3}}}
%    \end{macrocode}
%    \begin{macrocode}
  \long\def\caption@iibox#1#2#3[#4]{%
    \@testopt{\caption@iiibox{#1}{#2}{#3}[{#4}]}\captionbox@innerpos@default}
%    \end{macrocode}
%    \begin{macrocode}
  \long\def\caption@iibox@#1#2#3#4{%
    \setbox\@tempboxa\hbox{#4}%
    \caption@iiibox{#1}{#2}{#3}%
      [\wd\@tempboxa]%
      [\captionbox@innerpos@default]%
      {\unhbox\@tempboxa}}
%    \end{macrocode}
%    \begin{macrocode}
  \long\def\caption@iiibox#1{%
    \caption@iiiibox{#1}\vbox\vtop}
%    \end{macrocode}
%    \begin{macrocode}
  \long\def\caption@iiiibox#1#2#3#4#5[#6][#7]#8{%
    \@ifundefined{caption@hj@#7}%
      {\subcaption@Error{Undefined justification `#7'}\@gobble}%
      {\@firstofone}%
    {\begingroup
     #1*% set \caption@position so \caption@iftop expands correctly
     \caption@iftop{%
       \endgroup
       \parbox[t]{#6}{%
         #1\relax
         \caption@setposition t%
         #2{\caption#4{#5}}%
         \captionbox@hrule
         \csname caption@hj@#7\endcsname
         #8}%
     }{%
       \endgroup
       \parbox[b]{#6}{%
         #1\relax
         \caption@setposition b%
         \csname caption@hj@#7\endcsname
         #8%
         \captionbox@hrule
         #3{\caption#4{#5}}}%
     }}}
%    \end{macrocode}
%    \begin{macrocode}
  \newcommand*\captionbox@innerpos@default{c}
  \newcommand*\captionbox@hrule{\hrule\@height\z@\relax}
%    \end{macrocode}
%    \begin{macrocode}
  \providecommand*\caption@hj@c{\centering}
  \providecommand*\caption@hj@l{\raggedright}
  \providecommand*\caption@hj@r{\raggedleft}
  \providecommand*\caption@hj@s{}
%    \end{macrocode}
%    \begin{macrocode}
}{\@ifundefined{caption@iiibox}{%
%    \end{macrocode}
%    \begin{macrocode}
  \newcommand*\subcaptionbox{%  caption v3.2
    \def\captionbox@type{subtype}%
    \let\captionbox@settype\setcaptionsubtype
    \caption@withoptargs\caption@box}
%    \end{macrocode}
%    \begin{macrocode}
}{%
%    \end{macrocode}
%    \begin{macrocode}
  \newcommand*\subcaptionbox{%  caption >= v3.3
    \caption@withoptargs{\caption@ibox\setcaptionsubtype}}
%    \end{macrocode}
%    \begin{macrocode}
}}
%    \end{macrocode}
% \end{macro}
%
% \subsubsection{The \cs{subref} command}
%
% We redefine |\label| to |\subcaption@label| at |\setcaptionsubtype|.
%    \begin{macrocode}
\g@addto@macro\caption@subtypehook{%
  \ifx\label\subcaption@label \else
    \let\subcaption@ORI@label\label
    \let\label\subcaption@label
  \fi}
%    \end{macrocode}
%
% \begin{macro}{\subcaption@label}
% \changes{v1.1}{2011/09/12}{Adapted to the \pkg{showkeys} package}
% \changes{v1.1}{2011/09/12}{Uses \cs{caption@withoptargs} now}
% \changes{v1.1}{2016/02/20}{Unwanted space removed}
% \changes{v1.4b}{2021/01/08}{Local re-definitions of \cs{@bsphack} and \cs{@esphack} added}
% \changes{v1.4c}{2021/04/10}{Adaption to the \pkg{showkeys} package out-sourced to \cs{subcaption@prepare@label}}
% When a label will be placed for a sub-caption, we automatically place
% a second one for |\sub|\-|ref|, too. This second label will contain
% the sub-type counter only.
%    \begin{macrocode}
\newcommand*\subcaption@label{%
  \caption@withoptargs\subcaption@@label}
%    \end{macrocode}
%    \begin{macrocode}
\newcommand*\subcaption@@label[2]{%
  \@bsphack\begingroup
    \let\@bsphack\relax
    \let\@esphack\relax
%    \end{macrocode}
% Label |\@currentlabel| by expanding the original |\label| code.
%    \begin{macrocode}
    \subcaption@ORI@label#1{#2}%
%    \end{macrocode}
% Set |\@currentlabel| to the `sub' counter value and expand the original |\label| code again.
% (But this time without optional arguments.)
%    \begin{macrocode}
    \subcaption@prepare@label
    \protected@edef\@currentlabel{\csname thesub\@captype\endcsname}%
    \subcaption@ORI@label{sub@#2}%
  \endgroup\@esphack}
%    \end{macrocode}
% \end{macro}
%
% \begin{macro}{\subcaption@prepare@label}
% \changes{v1.4c}{2021/04/10}{This macro added}
% \changes{v1.4c}{2021/04/10}{Adapted to the \pkg{showlabels} package}
%    \begin{macrocode}
\newcommand*\subcaption@prepare@label{%
%    \end{macrocode}
% Adaption to the \pkg{showkeys} package: Hide the `sub' label from it.
%    \begin{macrocode}
  \let\SK@\@gobbletwo
%    \end{macrocode}
% Adaption to the \pkg{showlabels} package: Hide the `sub' label from it
% by expanding to the original definition saved to |\SL@orig#1|
% (with |#1| $=$ |\SL@orig|\-|label|, \ldots).
%    \begin{macrocode}
  \def\SL@showlabels##1{\@nameuse{SL@orig##1}}}
%    \end{macrocode}
% \end{macro}
%
% \begin{macro}{\subref}
% \changes{v1.1}{2011/08/14}{Caption option \opt{subrefformat=} added}
% \changes{v1.1}{2011/08/18}{Uses \cs{caption@setoptions*} now}
% \changes{v1.1}{2012/01/12}{Usage of \cs{caption@setoptions*} replaced by \cs{caption@setoptions}}
% \changes{v1.1}{2012/04/09}{Revised}
% \changes{v1.1}{2018/12/26}{Bugfix: Missing curly braces added}
% This one calls |\ref| with the second label. (see |\subcaption@label|)
%    \begin{macrocode}
\DeclareRobustCommand*\subref{%
  \@ifstar
    {\caption@withoptargs\subcaption@ref*}%
    {\caption@withoptargs\@subref}}
\newcommand*\@subref[2]{%
  \@ifundefined{hyperref}%
    {\subcaption@ref{#1}{#2}}%
    {\hyperref[{#2}]{\subcaption@ref{*#1}{#2}}}}
%    \end{macrocode}
%    \begin{macrocode}
\newcommand*\subcaption@ref[2]{%
  \begingroup
    \caption@setoptions{sub}%
    \subcaption@reffmt\p@subref{\ref#1{sub@#2}}%
  \endgroup}
%    \end{macrocode}
%    \begin{macrocode}
\newcommand*\p@subref{}
%    \end{macrocode}
% \end{macro}
%
%    \begin{macrocode}
\DeclareCaptionOption{subrefformat}{\subcaption@setrefformat{#1}}
%    \end{macrocode}
%
% \begin{macro}{\subcaption@setrefformat}
% \changes{v1.3g}{2020/07/27}{Adapted to \pkg{caption3} v2.0}
% \changes{v1.3h}{2020/09/28}{Usage of \cs{caption@Error} replaced by \cs{subcaption@Error}}
% \changes{v1.3j}{2020/10/07}{Definition of the label formats \texttt{subsimple} and \texttt{subparens} added}
% \changes{v1.4}{2020/12/24}{Adapted to older versions of the caption package (fallback)}
% \changes{v1.6b}{2023/07/28}{Adapted to the \pkg{subfig} package}
%  |\subcaption@setrefformat|\marg{name}\par
%  Selecting a subref format simply means saving the code (in |\subcaption@reffmt|).
%    \begin{macrocode}
\newcommand*\subcaption@setrefformat[1]{%
  \@ifundefined{caption@labelformat@#1}%  caption3 v2.x
    {\@ifundefined{caption@lfmt@#1}%      caption3 v1.x
       {\@ifundefined{caption@subreffmt@#1}{\subcaption@Error{Undefined label format `#1'}}{}}%
       {\expandafter\let\expandafter\subcaption@reffmt\csname caption@lfmt@#1\endcsname}}%
    {\expandafter\let\expandafter\subcaption@reffmt\csname caption@labelformat@#1\endcsname}%
  \@ifundefined{caption@subreffmt@#1}{}{\caption@setsubrefformat{#1}}} % subfig v1.3
%    \end{macrocode}
%    \begin{macrocode}
\subcaption@setrefformat{simple}
%    \end{macrocode}
%  To offer a smooth transition from the \pkg{subfig} to the \pkg{subcaption} package
%  we offer the options |sub|\-|ref|\-|format=sub|\-|simple| and |sub|\-|ref|\-|format=sub|\-|parens|, too.
%    \begin{macrocode}
\DeclareCaptionLabelFormat{subsimple}{#2}
\DeclareCaptionLabelFormat{subparens}{(#2)}
%    \end{macrocode}
% \end{macro}
%
% \subsection{Execution of options}
%
% \changes{v1.1}{2018/05/01}{Option \texttt{font+=small} changed to \texttt{font+=smaller}}        \changenote{released 2018/05/02}
% \changes{v1.2}{2018/05/13}{Option \texttt{font+=smaller} changed to \texttt{size=smaller}}       \changenote{merged from former SVN/trunk 2019/03/16, released 2019/08/18}
% \changes{v1.3e}{2020/07/27}{Option \texttt{size=smaller} changed back to \texttt{font+=smaller}} \changenote{merged from feature/documentclass 2020/07/29, released 2020/08/30}
% We use |\caption@Execute|\-|Options| and |\caption@Process|\-|Options| here to add
% the options to the `|sub|' option list instead of executing them immediately.
%    \begin{macrocode}
\let\caption@setkeys@ORI\caption@setkeys
\@ifundefined{caption@SetupOptions}  % caption3 v1.3
  {\renewcommand\caption@setkeys[2]{\captionsetup[sub]{#2}}}
  {\caption@SetupOptions{subcaption}{\captionsetup[sub]{#2}}}
%    \end{macrocode}
%    \begin{macrocode}
\@ifundefined{caption@smaller}       % caption3 v1.7-169
  {\caption@ExecuteOptions{subcaption}{%
     font+=small,labelformat=parens,labelsep=space,skip=6pt,list=0,hypcap=0}}
  {\caption@ExecuteOptions{subcaption}{%
     font+=smaller,labelformat=parens,labelsep=space,skip=6pt,list=0,hypcap=0}}
%    \end{macrocode}
%    \begin{macrocode}
\caption@ProcessOptions*{subcaption}
%    \end{macrocode}
%    \begin{macrocode}
\let\caption@setkeys\caption@setkeys@ORI
\let\caption@setkeys@ORI\@undefined
%    \end{macrocode}
%
% \begin{macro}{\subcaption@DeclareType}
% \changes{v1.6}{2022/04/07}{This macro added}
%   |\subcaption@DeclareType|\marg{type}\\
%   defines the sub-counter via |\Declare|\-|Caption|\-|Sub|\-|Type| offered by the \pkg{caption} package.
%    \begin{macrocode}
\newcommand*\subcaption@DeclareType[1]{%
  \@ifundefined{c@sub#1}%
    {\DeclareCaptionSubType{#1}}%
    {\caption@subtypesource\caption@subtype@source{sub#1}%
     \subcaption@Info{The counter `sub#1' was already defined by\MessageBreak\caption@subtype@source}}}
%    \end{macrocode}
%    \begin{macrocode}
\providecommand*\caption@subtypesource[2]{\def#1{a different package}} % caption3 v2.4
%    \end{macrocode}
% \end{macro}
%
% We call |\Declare|\-|Caption|\-|Sub|\-|Type| for |figure| and |table|,
% and each floating environment declared with |\Declare|\-|Floating|\-|Environment| here.
%    \begin{macrocode}
\@ifundefined{ForEachCaptionType}      % caption3 v1.13
  {\@ifundefined{caption@ForEachType}  % caption3 v1.4a
     {\@ifundefined{c@figure}{}{\subcaption@DeclareType{figure}}%
      \@ifundefined{c@table}{}{\subcaption@DeclareType{table}}%
      \caption@For{typelist}{\subcaption@DeclareType{#1}}}
     {\caption@ForEachType{\subcaption@DeclareType{#1}}}}
  {\ForEachCaptionType{\subcaption@DeclareType{#1}}}
%    \end{macrocode}
%
% \subsection{Bonus material}
%
% \subsubsection{The \cs{subcaption}, \cs{phantomsubcaption}, \cs{subcaptionlistentry}, and \cs{subcaptiontext} commands}
%
% \begin{macro}{\subcaption@newabbreviation}
% \changes{v1.5}{2022/01/07}{This macro added}
% |\subcaption@newabbreviation|\marg{new command}\marg{existing command}\marg{extra code in case of error}\\
% defines a new command as abbreviation of |\set|\-|caption|\-|sub|\-|type*| plus \meta{command}.
% (With a prefacing |\set|\-|caption|\-|sub|\-|type|, \meta{new command} is reduced to \meta{command}.
% Unfortunately we have to do this on our own since using |\set|\-|caption|\-|sub|\-|type| multiple times
% is not suppressed until \pkg{caption} package \version{3.6}.)
% \Note{Since \cs{setcaptionsubtype} is used, the new command is designed to be used inside an own group!}
%    \begin{macrocode}
\newcommand*\subcaption@newabbreviation[3]{%
%    \end{macrocode}
%    \begin{macrocode}
  \newcommand*{#1}{%
    \caption@iftype
      {\setcaptionsubtype*#2}%
      {\subcaption@OutsideFloat#1#3}}%
%    \end{macrocode}
%    \begin{macrocode}
  \g@addto@macro\caption@subtypehook{\let#1#2}}% needed for caption < 3.6
%    \end{macrocode}
%    \begin{macrocode}
\@onlypreamble\subcaption@newabbreviation
%    \end{macrocode}
% \end{macro}
%
% \begin{macro}{\caption@gobble}
% \changes{v1.4}{2020/12/24}{This macro added}
%   |\caption@gobble*|\oarg{arg}\oarg{\ldots}\marg{arg}\\
%   is similar to |\@gobble| but gobbles a star and optional arguments as well.
%    \begin{macrocode}
\@ifundefined{caption@gobble}{%
  \DeclareRobustCommand*\caption@gobble{%  caption3 < v1.4
    \caption@withoptargs\@gobbletwo}%
}{}
%    \end{macrocode}
% \end{macro}
%
% \begin{macro}{\subcaption}
% \changes{v1.1}{2012/04/06}{Overwrites the definition defined by the \pkg{memoir} document class}
% |\subcaption| is an abbreviation of |\set|\-|caption|\-|sub|\-|type*| plus |\caption|.
%    \begin{macrocode}
\@ifclassloaded{memoir}{\let\subcaption\undefined}{}
\subcaption@newabbreviation\subcaption\caption\caption@gobble
%    \end{macrocode}
% \end{macro}
%
% \begin{macro}{\phantomsubcaption}
% \changes{v1.1}{2011/08/17}{This macro added}
% \changes{v1.3h}{2020/09/28}{Usage of \cs{caption@Error} replaced by \cs{subcaption@Error}}
% |\phantom|\-|sub|\-|caption| is an abbreviation of |\set|\-|caption|\-|sub|\-|type*| plus |\phantom|\-|caption|.
% \Note{This commands needs at least \pkg{caption} package \version{3.2}.}
%    \begin{macrocode}
\subcaption@newabbreviation\phantomsubcaption\phantomcaption\relax
%    \end{macrocode}
% \end{macro}
%
% \begin{macro}{\subcaptionlistentry}
% \changes{v1.5}{2022/01/05}{This macro added}
% |\sub|\-|caption|\-|list|\-|entry| is an abbreviation of |\set|\-|caption|\-|sub|\-|type*| plus |\caption|\-|list|\-|entry|.
% \Note{This commands needs at least \pkg{caption} package \version{3.3}.}
%    \begin{macrocode}
\subcaption@newabbreviation\subcaptionlistentry\captionlistentry\caption@gobble
%    \end{macrocode}
% \end{macro}
%
% \begin{macro}{\subcaptiontext}
% \changes{v1.5}{2022/01/05}{This macro added}
% |\sub|\-|caption|\-|text| is an abbreviation of |\set|\-|caption|\-|sub|\-|type*| plus |\caption|\-|text|.
% \Note{This commands needs at least \pkg{caption} package \version{3.6}.}
%    \begin{macrocode}
\subcaption@newabbreviation\subcaptiontext\captiontext\caption@gobble
%    \end{macrocode}
% \end{macro}
%
% \begin{macro}{\caption@subfloatrow@hook}
% \changes{v1.6}{2022/04/17}{Support of this hook added}
% Inside the |sub|\-|float|\-|row| environment offered by the \pkg{floatrow} package
% the usage of |\caption| etc.~needs to be mapped to |\sub|\-|caption| etc.
%    \begin{macrocode}
\providecommand*\caption@subfloatrow@hook{} % caption < v3.7
%    \end{macrocode}
%    \begin{macrocode}
\g@addto@macro\caption@subfloatrow@hook{%
  \let\caption\subcaption
  \let\phantomcaption\phantomsubcaption
  \let\captionlistentry\subcaptionlistentry
  \let\captiontext\subcaptiontext}
%    \end{macrocode}
% \end{macro}
%
% \subsubsection{The \cs{subfloat} command}
%
% \begin{macro}{\subfloat}
% \changes{v0.4}{2008/03/01}{This macro added}
% \changes{v1.0c}{2011/01/22}{This macro removed since it's too incompatible with the one from \pkg{subfig}}
% \changes{v1.3}{2019/08/31}{This macro re-added and revised}
% \changes{v1.3a}{2019/09/01}{\cs{ignorespaces} copied from \cs{sf@@@subfloat}}
% \changes{v1.3b}{2020/01/03}{Previous definition of \cs{subfloat} will be overwritten}
% \changes{v1.3f}{2020/07/29}{Bugfix: Usage of \cs{caption@hj@default} replaced by \cs{captionbox@innerpos@default}}
% \changes{v1.3i}{2020/09/28}{This macro will be defined with \cs{providecommand} now to preserve an original definition}
% \changes{v1.4}{2020/12/24}{Usage of \cs{caption@getlabel} replaced by \cs{subcaption@getlabel}}
% \changes{v1.4a}{2020/12/26}{Clearance of \cs{caption@thelabel} replaced by \cs{subcaption@clrlabel}}
% \changes{v1.5}{2022/01/07}{Usage of \cs{@ifnextchar} replaced by \cs{kernel@ifnextchar}}
% \changes{v1.5}{2022/01/07}{Overwrites the definition done by the \cls{memoir} document class}
% \changes{v1.6}{2022/04/16}{Emulates the definition done by the \cls{memoir} document class}
%  |\subfloat|\oarg{list\_entry}\oarg{sub-caption}\marg{body}\par
%  If \meta{sub-caption} is given, we map this to |\sub|\-|caption|\-|box|
%  but transfer the |\label| from \meta{body} to \meta{sub-caption}.
%  If not, we do the same as |\sub|\-|caption|\-|box| does, but use |\phantom|\-|caption| instead of |\caption|.
%  In both cases we do a |\ignore|\-|spaces| at the end since the original implementation of |\sub|\-|float| does this, too.
%    \begin{macrocode}
\caption@AtBeginDocument{%
  \begingroup
    \newenvironment{subcaption@memoir@subfloat}{}{}%
    \ifx\subfloat\subcaption@memoir@subfloat
      \endgroup
      % "subfloat" = empty environment, defined by memoir
      \renewcommand*\subfloat{%
        \def\subcaption@currenvir{subfloat}%
        \ifx\@currenvir\subcaption@currenvir
          % emulate (empty) subfloat environment
        \else
          \expandafter\subcaption@subfloat
        \fi}%
    \else
      \endgroup
      \providecommand*\subfloat{\subcaption@subfloat}%
    \fi}
%    \end{macrocode}
%    \begin{macrocode}
\newcommand*\subcaption@subfloat{%
  \kernel@ifnextchar[%]
    \subcaption@@subfloat
    \subcaption@subfloat@}
%    \end{macrocode}
%    \begin{macrocode}
\long\def\subcaption@@subfloat[#1]{%
  \kernel@ifnextchar[%]
    {\subcaption@@@subfloat{#1}}%
    {\subcaption@@subfloat@\subcaptionbox{#1}}}
%    \end{macrocode}
%    \begin{macrocode}
\long\def\subcaption@@@subfloat#1[#2]{%
  \subcaption@@subfloat@{\subcaptionbox[{#1}]}{#2}}
%    \end{macrocode}
%    \begin{macrocode}
\long\def\subcaption@@subfloat@#1#2#3{%
  \subcaption@getlabel{#3}%
  #1{#2\caption@thelabel}{\let\label\caption@gobble#3}%
  \subcaption@clrlabel
  \ignorespaces}
%    \end{macrocode}
%    \begin{macrocode}
\def\subcaption@subfloat@#1{%
  \setbox\@tempboxa\hbox{#1}%
  \caption@iiiibox
    \setcaptionsubtype
    {\phantomcaption\@gobble}{\phantomcaption\@gobble}% no box with \caption
    {}% no optional arguments for \caption
    {}% no sub-caption
    [\wd\@tempboxa][\captionbox@innerpos@default]%
    {\unhbox\@tempboxa}%
  \ignorespaces}
%    \end{macrocode}
% \end{macro}
%
% \begin{macro}{\subcaption@getlabel}
% \changes{v1.4}{2020/12/24}{This macro added}
%   |\subcaption@getlabel|\marg{text}\\
%   gets the label command out of the given caption text and stores it to |\caption@the|\-|label|.
%   It uses |\caption@get|\-|label| for this purpose which interface unfortunately has changed over time.
%   (Changing the interface was a bad idea in the first place but now it is as it is).
%    \begin{macrocode}
\@ifundefined{caption@getlabel}{%
%    \end{macrocode}
%    \begin{macrocode}
  \newcommand\subcaption@getlabel[1]{%  caption3 < v1.7
    \subcaption@Error{\noexpand\subfloat needs at least caption v3.4}%
    \let\caption@thelabel\relax}
%    \end{macrocode}
%    \begin{macrocode}
}{\@ifundefined{caption@@@@getlabel}{%
%    \end{macrocode}
%    \begin{macrocode}
  \newcommand\subcaption@getlabel[1]{%  caption3 >= v1.7
    \caption@getlabel#1\label{}\@nil}
%    \end{macrocode}
%    \begin{macrocode}
}{%
%    \end{macrocode}
%    \begin{macrocode}
  \newcommand*\subcaption@getlabel{%    caption3 >= v2.0
    \caption@getlabel}
%    \end{macrocode}
%    \begin{macrocode}
}}
%    \end{macrocode}
% \end{macro}
%
% \begin{macro}{\subcaption@clrlabel}
% \changes{v1.4a}{2020/12/26}{This macro added}
%   |\subcaption@clrlabel|\\
%   resets |\caption@the|\-|label| to |\relax|.
%    \begin{macrocode}
\@ifundefined{caption@clrlabel}{%
%    \end{macrocode}
%    \begin{macrocode}
  \newcommand*\subcaption@clrlabel{\let\caption@thelabel\relax} %  caption3 < v2.3
%    \end{macrocode}
%    \begin{macrocode}
}{%
%    \end{macrocode}
%    \begin{macrocode}
  \newcommand*\subcaption@clrlabel{\caption@clrlabel}           %  caption3 >= v2.3
%    \end{macrocode}
%    \begin{macrocode}
}
%    \end{macrocode}
% \end{macro}
%
% \iffalse
%</package>
% \fi
%
% \iffalse
% --------------------------------------------------------------------------- %
% \fi
%
% \Finale
%
\endinput
