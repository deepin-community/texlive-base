%%^^A%%  fontspec-code-msg.dtx -- part of FONTSPEC <latex3.github.io/fontspec>
%
% \section{Error/warning/info messages}
%
% \iffalse
%    \begin{macrocode}
%<*fontspec>
%    \end{macrocode}
% \fi
%
% Shorthands for messages:
%    \begin{macrocode}
\cs_new:Npn \@@_error:n     { \msg_error:nn     {fontspec} }
\cs_new:Npn \@@_error:nn    { \msg_error:nnn    {fontspec} }
\cs_new:Npn \@@_error:nx    { \msg_error:nnx    {fontspec} }
\cs_new:Npn \@@_error:nxx   { \msg_error:nnxx   {fontspec} }
\cs_new:Npn \@@_warning:n   { \msg_warning:nn   {fontspec} }
\cs_new:Npn \@@_warning:nx  { \msg_warning:nnx  {fontspec} }
\cs_new:Npn \@@_warning:nxx { \msg_warning:nnxx {fontspec} }
\cs_new:Npn \@@_info:n      { \msg_info:nn      {fontspec} }
\cs_new:Npn \@@_info:nx     { \msg_info:nnx     {fontspec} }
\cs_new:Npn \@@_info:nxx    { \msg_info:nnxx    {fontspec} }
\cs_new:Npn \@@_trace:n     { \msg_trace:nn     {fontspec} }
%    \end{macrocode}
%
% Allow messages to be written with spaces acting as normal:
%    \begin{macrocode}
\cs_generate_variant:Nn \msg_new:nnn  {nnx}
\cs_generate_variant:Nn \msg_new:nnnn {nnxx}
\cs_new:Nn \@@_msg_new:nn
  { \msg_new:nnx {fontspec} {#1} { ^^J \tl_trim_spaces:n {#2} } }
\cs_new:Nn \@@_msg_new:nnn
  { \msg_new:nnxx {fontspec} {#1} { ^^J \tl_trim_spaces:n {#2} } { \tl_trim_spaces:n {#3} } }
\char_set_catcode_space:n {32}
%    \end{macrocode}
%
% \subsection{Errors}
%
%    \begin{macrocode}
\@@_msg_new:nn {only-inside-encdef}
 {
  \exp_not:N #1 can only be used in the second argument
  to \string\DeclareUnicodeEncoding.
 }
\@@_msg_new:nn {no-size-info}
 {
  Size information must be supplied.\\
  For example, SizeFeatures={Size={8-12},...}.
 }
\@@_msg_new:nnn {font-not-found}
 {
  The font "#1" cannot be found; this may be but usually is not
  a fontspec bug. Either there is a typo in the font name/file,
  the font is not installed (correctly), or there is a bug
  in the underlying font loading engine (XeTeX/luaotfload).
 }
 {
  A font might not be found for many reasons.\\
  Check the spelling, where the font is installed etc. etc.\\\\
  When in doubt, ask someone for help!
 }
\@@_msg_new:nnn {rename-feature-not-exist}
 {
  The feature #1 doesn't appear to be defined.
 }
 {
  It looks like you're trying to rename a feature that doesn't exist.
 }
\@@_msg_new:nn {no-glyph}
 {
  '#1' does not contain glyph #2.
 }
\@@_msg_new:nnn {euler-too-late}
 {
  The euler package must be loaded BEFORE fontspec.
 }
 {
  fontspec only overwrites euler's attempt to
  define the maths text fonts if fontspec is
  loaded after euler. Type <return> to proceed
  with incorrect \string\mathit, \string\mathbf, etc.
 }
\@@_msg_new:nnn {no-xcolor}
 {
  Cannot load named colours without the xcolor package.
 }
 {
  Sorry, I can't do anything to help. Instead of loading
  the color package, use xcolor instead.
 }
\@@_msg_new:nnn {unknown-color-model}
 {
  Error loading colour `#1'; unknown colour model.
 }
 {
  Sorry, I can't do anything to help. Please report this error
  to my developer with a minimal example that causes the problem.
 }
\@@_msg_new:nnn {not-in-addfontfeatures}
 {
  The "#1" font feature cannot be used in \string\addfontfeatures.
 }
 {
  This is due to how TeX loads fonts; such settings
  are global so adding them mid-document within a group causes
  confusion. You'll need to define multiple font families to achieve
  what you want.
 }
%    \end{macrocode}
%
% \subsection{Warnings}
%
%    \begin{macrocode}
\@@_msg_new:nn {tu-clash}
 {
  I have found the tuenc.def encoding definition file but the TU encoding is not
  defined by the LaTeX2e kernel; attempting to correct but you really should update
  to the latest version of LaTeX2e.
 }
\@@_msg_new:nn {tu-missing}
 {
  The TU encoding seems to be missing; please update to the latest version of LaTeX2e.
 }
\@@_msg_new:nn {addfontfeatures-ignored}
 {
  \string\addfontfeature (s) ignored \msg_line_context:;
  it cannot be used with a font that wasn't selected by a fontspec command.\\
  \\
  The current font is "\use:c{font@name}".\\
  \int_compare:nTF { \clist_count:n {#1} = 1 }
    { The requested feature is "#1". }
    { The requested features are "#1". }
 }
\@@_msg_new:nn {feature-option-overwrite}
 {
  Option '#2' of font feature '#1' overwritten.
 }
\@@_msg_new:nn {ot-tag-too-long}
 {
  OpenType tag '#1' is too long; script, language, and feature tags must be four characters or fewer.
 }
\@@_msg_new:nn {aat-feature-not-exist}
 {
  '\l_keys_key_tl=\l_keys_value_tl' feature not supported
  for AAT font '\l_fontspec_fontname_tl'.
 }
\@@_msg_new:nn {aat-feature-not-exist-in-font}
 {
  AAT feature '\l_keys_key_tl=\l_keys_value_tl' (#1) not available
  in font '\l_fontspec_fontname_tl'.
 }
\@@_msg_new:nn {no-opticals}
 {
  '#1' doesn't appear to have an Optical Size axis.
 }
\@@_msg_new:nn {script-not-exist}
 {
  Script '#2' not explicitly supported within font '#1'.
  Check the typeset output, and if it is okay then ignore this warning.
  Otherwise a different font should be chosen.
 }
\@@_msg_new:nn {language-not-exist}
 {
  Language '#1' not explicitly supported
  within font '\l_fontspec_fontname_tl'
  with script '\l_@@_script_name_tl'.
  Check the typeset output, and if it is okay then ignore this warning.
  Otherwise a different font should be chosen.
 }
\@@_msg_new:nn {only-xetex-feature}
 {
  Ignored XeTeX-only feature: '#1'.
 }
\@@_msg_new:nn {only-luatex-feature}
 {
  Ignored LuaTeX-only feature: '#1'.
 }
\@@_msg_new:nn {unknown-renderer}
 {
  Renderer '#1' unknown. Assuming Harfbuzz with 'shaper=#1'.
  Please raise a fontspec issue to add this shaper to the interface.
 }
\@@_msg_new:nn {no-mapping}
 {
  Input mapping not supported in LuaTeX.
 }
\@@_msg_new:nn {no-mapping-ligtex}
 {
  Input mapping not supported in LuaTeX.\\
  Use "Ligatures=TeX" instead of "Mapping=tex-text".
 }
%    \end{macrocode}
% message for package options must be loaded earlier
%    \begin{macrocode}
%</fontspec>
%<*options>
\msg_new:nnn {fontspec} {cm-default-obsolete}
 {
  The~"cm-default"~package~option~is~obsolete.
 }
\msg_new:nnn {fontspec} {enc-obsolete}
 {
  The~"#1"~package~option~is~obsolete.~TU~is~the~default~encoding.
 }
%</options>
%<*fontspec>
\@@_msg_new:nn {font-index-needs-ttc}
 {
  The "FontIndex" feature is only supported by TTC (TrueType Collection) fonts.\\
  Feature ignored.
 }
\@@_msg_new:nn {feat-cannot-remove}
 {
  The "#1" feature cannot be deactivated. Request ignored.
 }
%    \end{macrocode}
%
% \subsection{Info messages}
%
%    \begin{macrocode}
\@@_msg_new:nn {defining-font}
 {
  Font family '\g_@@_nfss_family_tl' created for font '#2'
  with options [\l_@@_all_features_clist].\\
  \\
  This font family consists of the following NFSS series/shapes:\\
  \g_@@_defined_shapes_tl
 }
\@@_msg_new:nn {no-font-shape}
 {
  Could not resolve font "#1" (it probably doesn't exist).
 }
\@@_msg_new:nn {set-scale}
 {
  \l_fontspec_fontname_tl\space scale = \l_@@_scale_tl.
 }
\@@_msg_new:nn {setup-math}
 {
  Adjusting the maths setup (use [no-math] to avoid this).
 }
\@@_msg_new:nn {opa-twice}
 {
  Opacity set twice, in both Colour and Opacity.\\
  Using specification "Opacity=#1".
 }
\@@_msg_new:nn {opa-twice-col}
 {
  Opacity set twice, in both Opacity and Colour.\\
  Using an opacity specification in hex of "#1/FF".
 }
\@@_msg_new:nn {bad-colour}
 {
  Bad colour declaration "#1".
  Colour must be one of:\\
  * a named xcolor colour\\
  * a six-digit hex colour RRGGBB\\
  * an eight-digit hex colour RRGGBBTT with opacity
 }
%    \end{macrocode}
%
% Reset `space' behaviour:
%    \begin{macrocode}
\char_set_catcode_ignore:n {32}
%    \end{macrocode}
%
% \iffalse
%    \begin{macrocode}
%</fontspec>
%    \end{macrocode}
% \fi


\endinput

% /©
% ------------------------------------------------
% The FONTSPEC package  <latex3.github.io/fontspec>
% ------------------------------------------------
% Copyright  2022-2024  The LaTeX project,  LPPL "maintainer"
% Copyright  2004-2022  Will Robertson
% Copyright  2009-2015  Khaled Hosny
% Copyright  2013       Philipp Gesang
% Copyright  2013-2016  Joseph Wright
% ------------------------------------------------
% This package is free software and may be redistributed and/or modified under
% the conditions of the LaTeX Project Public License, version 1.3c or higher
% (your choice): <http://www.latex-project.org/lppl/>.
% ------------------------------------------------
% ©/
