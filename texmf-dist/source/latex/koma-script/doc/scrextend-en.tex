% ======================================================================
% scrextend-en.tex
% Copyright (c) Markus Kohm, 2002-2022
%
% This file is part of the LaTeX2e KOMA-Script bundle.
%
% This work may be distributed and/or modified under the conditions of
% the LaTeX Project Public License, version 1.3c of the license.
% The latest version of this license is in
%   http://www.latex-project.org/lppl.txt
% and version 1.3c or later is part of all distributions of LaTeX 
% version 2005/12/01 or later and of this work.
%
% This work has the LPPL maintenance status "author-maintained".
%
% The Current Maintainer and author of this work is Markus Kohm.
%
% This work consists of all files listed in MANIFEST.md.
% ======================================================================
%
% Package scrextend for Document Writers
% Maintained by Markus Kohm
%
% ======================================================================

\KOMAProvidesFile{scrextend-en.tex}
                 [$Date: 2023-09-18 09:02:03 +0200 (Mo, 18. Sep 2023) $
                  KOMA-Script package scrextend]
\translator{Markus Kohm\and Karl Hagen}

\chapter[{\KOMAScript{} Features for Other Classes with \Package{scrextend}}]
  {Using Basic Features of the \KOMAScript{} Classes in Other Classes with the
    \Package{scrextend} Package}
\labelbase{scrextend}
\BeginIndexGroup%
\BeginIndex{Package}{scrextend}%

There are some features that are common to all \KOMAScript{} classes. This
applies not only to the \Class{scrbook}, \Class{scrreprt}, and
\Class{scrartcl} classes, which are intended to replace the standard classes
\Class{book}, \Class{report}, and \Class{article}, but also to a large extend
the \KOMAScript{} class \Class{scrlttr2}, the successor to \Class{scrlettr},
which is intended for letters. These basic features, which can be found in the
classes mentioned above, are also provided by package \Package{scrextend} from
\KOMAScript{} version~3.00 onward. This\textnote{Attention!} package should
not be used with \KOMAScript{} classes. It is intended for use with other
classes only. If you attempt to load the package with a \KOMAScript{} class,
\Package{scrextend} will detect this and reject loading it with a warning
message.

The fact that \hyperref[cha:scrlttr2]{\Package{scrletter}}%
\IndexPackage{scrletter} can be used not only with \KOMAScript{} classes but
also with the standard classes is partly due to \Package{scrextend}. If
\hyperref[cha:scrlttr2]{\Package{scrletter}} detects that it is not being used
with a \KOMAScript{} class, it automatically loads \Package{scrextend}. Doing
so makes all required \KOMAScript{} classes available.

Of course, there is no guarantee that \Package{scrextend} will work with all
classes. The package has been designed primarily to extend the standard
classes and those derived from them. In any case, before you use
\Package{scrextend}, you should first make sure that the class you are using
does not already provide the feature you need.

In addition to the features described in this chapter, there are a few more
that are primarily intended for authors of classes and packages. These can be
found in \autoref{cha:scrbase}, starting on \autopageref{cha:scrbase}. The 
\hyperref[cha:scrbase]{\Package{scrbase}}%
\important{\hyperref[cha:scrbase]{\Package{scrbase}}}\IndexPackage{scrbase}
package documented there is used by all \KOMAScript{} classes and the
\Package{scrextend} package.

All \KOMAScript{} classes and the \Package{scrextend} package also load the
\hyperref[cha:scrlfile]{\Package{scrlfile}}%
\important{\hyperref[cha:scrlfile]{\Package{scrlfile}}}\IndexPackage{scrlfile}
package described in \autoref{cha:scrlfile} starting on
\autopageref{cha:scrlfile}. Therefore the features of this package are also
available when using \Package{scrextend}.

\iftrue % Umbruchkorrekturtext
In contrast, only the \KOMAScript{} classes \Class{scrbook}, \Class{scrreprt},
and \Class{scrartcl} load the \hyperref[cha:tocbasic]{\Package{tocbasic}}
package (see \autoref{cha:tocbasic} starting on \autopageref{cha:tocbasic}),
which is designed for authors of classes and packages. For this reason, the
features defined there are found only in those classes and not in
\Package{scrextend}. Of course you can still use
\hyperref[cha:tocbasic]{\Package{tocbasic}} together with
\Package{scrextend}.%
\fi

\LoadCommonFile{options}% \section{Early or late Selection of Options}

\LoadCommonFile{compatibility}% \section{Compatibility with Earlier Versions of \KOMAScript}


\section{Optional, Extended Features}
\seclabel{optionalFeatures}

The \Package{scrextend} package provides some optional, extended features.
These features are not available by default but can be activated. These
features are optional because, for example, they may conflict with features of
the standard classes of other commonly used packages.

\begin{Declaration}
  \OptionVName{extendedfeature}{feature}
\end{Declaration}
With this option, you can activate an extended \PName{feature} of
\Package{scrextend}. This option is available only while loading
\Package{scrextend}. You must therefore specify this option in the optional
argument of \DescRef{\LabelBase.cmd.usepackage}\PParameter{scrextend}. %
\iffree{%
  An overview of all available features is shown in
  \autoref{tab:scrextend.optionalFeatures}.

  \begin{table}
    \caption[{Available extended features of
      \Package{scrextend}}]{Overview of the optional extended
      features of \Package{scrextend}}
    \label{tab:scrextend.optionalFeatures}
    \begin{desctabular}
      \entry{\PName{title}}{%
        title pages have the additional features of the \KOMAScript{} classes;
        this applies not only to the commands for the title page but also to
        the \DescRef{\LabelBase.option.titlepage} option (see
        \autoref{sec:scrextend.titlepage}, from
        \autopageref{sec:scrextend.titlepage})%
      }%
    \end{desctabular}
  \end{table}
}{%
  \par%
  Currently the only available extended \PName{feature} is 
  \PValue{title}\IndexOption[indexmain]{extendedfeature~=\textKValue{title}}%
    \important{\OptionValue{extendedfeature}{title}}.
  This \PName{feature} gives title pages the features of the \KOMAScript{}
  classes, as described in \autoref{sec:scrextend.titlepage} starting on
  \autopageref{sec:scrextend.titlepage}.%
}%
%
\EndIndexGroup


\LoadCommonFile{draftmode}% \section{Draft Mode}

\LoadCommonFile{fontsize}%

\LoadCommonFile{textmarkup}% \section{Text Markup}

\LoadCommonFile{titles}% \section{Document Title Pages}

\LoadCommonFile{oddorevenpage}% \section{Detection of Odd and Even Pages}

\section{Choosing a Predefined Page Style}
\seclabel{pagestyle}

One of the basic features of a document is the page
style\Index[indexmain]{page>style}. In \LaTeX{}, the page style primarily
determines the content of headers and footers. The \Package{scrextend} package
does not define any page styles itself. Instead it uses the page styles of the
\LaTeX{} kernel.


\begin{Declaration}
  \Macro{titlepagestyle}
\end{Declaration}%
\Index{title>page style}%
On some pages \DescRef{maincls.cmd.thispagestyle}\IndexCmd{thispagestyle}
automatically selects a different page style. Currently, \Package{scrextend}
only does this for title pages, and only if
\OptionValueRef{\LabelBase}{extendedfeature}{title} has been used (see
\autoref{sec:scrextend.optionalFeatures},
\DescPageRef{scrextend.option.extendedfeature}). In this case the page style
specified in \DescRef{maincls.cmd.thispagestyle} will be used. The default for
\DescRef{maincls.cmd.thispagestyle} is
\PageStyle{plain}\IndexPagestyle{plain}. This page style is defined by the
\LaTeX{} kernel, so it should always be available.%
\EndIndexGroup

\LoadCommonFile{interleafpage}% \section{Interleaf Pages}

\LoadCommonFile{footnotes}% \section{Footnotes}

\LoadCommonFile{dictum}% \section{Dicta}

\LoadCommonFile{lists}% \section{Lists}

\LoadCommonFile{marginpar}% \section{Margin Notes}
%
\EndIndexGroup

\endinput

%%% Local Variables: 
%%% mode: latex
%%% TeX-master: "scrguide-en.tex"
%%% coding: utf-8
%%% ispell-local-dictionary: "en_GB"
%%% eval: (flyspell-mode 1)
%%% End: 
