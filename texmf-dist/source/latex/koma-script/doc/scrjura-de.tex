% ======================================================================
% scrjura-de.tex
% Copyright (c) Markus Kohm, 2011-2024
%
% This file is part of the LaTeX2e KOMA-Script bundle.
%
% This work may be distributed and/or modified under the conditions of
% the LaTeX Project Public License, version 1.3c of the license.
% The latest version of this license is in
%   http://www.latex-project.org/lppl.txt
% and version 1.3c or later is part of all distributions of LaTeX 
% version 2005/12/01 or later and of this work.
%
% This work has the LPPL maintenance status "author-maintained".
%
% The Current Maintainer and author of this work is Markus Kohm.
%
% This work consists of all files listed in MANIFEST.md.
% ======================================================================
%
% Chapter about scrjura of the KOMA-Script guide
% Maintained by Markus Kohm
%
% ======================================================================

\KOMAProvidesFile{scrjura-de.tex}%
                 [$Date: 2024-10-15 10:33:25 +0200 (Di, 15. Okt 2024) $
                  KOMA-Script guide (chapter: scrjura)]

\chapter{Unterstützung für die Anwaltspraxis durch \Package{scrjura}}
\labelbase{scrjura}
\BeginIndexGroup
\BeginIndex{Package}{scrjura}
\BeginIndex{Package}{contract}

Bis einschließlich Version~3.41 enthielt \KOMAScript{} offiziell das Paket
\Package{scrjura} zur Unterstützung anwaltlicher Dokumente. Dabei ging es vor
allem um Satzungen, Gesetze, Kommentare dazu oder im weitesten Sinn um
Verträge alle Art. Im Zuge der Restrukturierung von \KOMAScript{} wurde das
Paket ausgelagert. Da der Vertrag das zentrale Element des Pakets ist, bekam
es dabei den neuen Namen \Package{contract}. Unter diesem Namen ist es nicht
nur auf CTAN zu finden (siehe \cite{package:contract}). Es wurde auch bereits
in gängige \TeX-Distributionen aufgenommen und kann daher über deren
Paketmanager installiert werden.

Aus Kompatibilitätsgründen wird es für eine begrenzte Zeit in \KOMAScript{}
weiterhin ein Paket \Package{scrjura} geben. Dabei handelt es sich aber
lediglich um eine andere Verpackung des neuen Pakets \Package{contract}, bei
dem ein Teil der Inkompatibilitäten zwischen dem neuen Paket und dem früheren
\Package{scrjura} bereinigt wurden. Damit sollte es in der Regel möglich sein,
bisherige Dokumente auf Basis von \Package{scrjura} weiterhin zu
verarbeiten. Für neue Dokumente wird unbedingt empfohlen, stattdessen zu
\Package{contract} zu wechseln. Bei der Überarbeitung alter Dokumente ist eine
Umstellung ebenfalls zu empfehlen. Für die dabei zu berücksichtigenden
Änderungen sind die Angaben in der Anleitung zum Paket \Package{contract}
maßgeblich.

\begin{Declaration}
  \Macro{Clause}\Parameter{Optionen}%
  \Macro{SubClause}\Parameter{Optionen}
\end{Declaration}
Der wichtigste Unterschied bei Verwendung von \Package{scrjura} gegenüber
\Package{contract} ist, dass die \PName{Optionen} zu den beiden Anweisungen
\Macro{Clause} und \Macro{SubClause} innerhalb einer
\Environment{contract}-Umgebung bei \Package{contract} ein optionales Argument
darstellen, also in eckigen Klammern anzugeben sind. Beim Paket
\Package{scrjura} waren die \PName{Optionen} hingegen immer ein erforderliches
Argument also in geschweiften Klammern anzugeben. Das ist mit
\Package{scrjura} auch weiterhin der Fall.%
\EndIndexGroup
%
\EndIndexGroup

%%% Local Variables: 
%%% mode: latex
%%% TeX-master: "scrguide-de.tex"
%%% coding: utf-8
%%% ispell-local-dictionary: "de_DE"
%%% eval: (flyspell-mode 1)
%%% End: 

% LocalWords:  anwaltlicher Paketmanager Kompatibilitätspaket
% LocalWords:  Originaldefinition
