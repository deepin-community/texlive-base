% \iffalse meta-comment
%
% Copyright (C) 1993-2024
% The LaTeX Project and any individual authors listed elsewhere
% in this file.
%
% This file is part of the LaTeX base system.
% -------------------------------------------
%
% It may be distributed and/or modified under the
% conditions of the LaTeX Project Public License, either version 1.3c
% of this license or (at your option) any later version.
% The latest version of this license is in
%    https://www.latex-project.org/lppl.txt
% and version 1.3c or later is part of all distributions of LaTeX
% version 2008 or later.
%
% This file has the LPPL maintenance status "maintained".
%
% The list of all files belonging to the LaTeX base distribution is
% given in the file `manifest.txt'. See also `legal.txt' for additional
% information.
%
% The list of derived (unpacked) files belonging to the distribution
% and covered by LPPL is defined by the unpacking scripts (with
% extension .ins) which are part of the distribution.
%
% \fi
% \iffalse
%%
%% File `ifthen.dtx'.
%% Copyright (C) 1991 by Leslie Lamport
%% Copyright (C) 1994-2001 LaTeX project, David Carlisle
%%                       all rights reserved.
%%
%
%<*dtx>
          \ProvidesFile{ifthen.dtx}
%</dtx>
%<package>\NeedsTeXFormat{LaTeX2e}[1994/12/01]
%<package>\ProvidesPackage{ifthen}
%<driver>\ProvidesFile{ifthen.drv}
% \fi
%         \ProvidesFile{ifthen.dtx}
          [2022/04/13 v1.1d Standard LaTeX ifthen package (DPC)]
%
% \iffalse
%<*driver>
\documentclass{ltxdoc}
\begin{document}
\DocInput{ifthen.dtx}
\end{document}
%</driver>
% \fi
%
% \GetFileInfo{ifthen.dtx}
%
% \title{The \textsf{ifthen} package\thanks{This file
%         has version number \fileversion, last
%         revised \filedate.}}
% \author{David Carlisle}
% \date{\filedate}
% \MaintainedByLaTeXTeam{latex}
%  \maketitle
%
%
%
% \begin{abstract}
% This file implements an |\ifthenelse| command for \LaTeXe.
% The algorithm used is compatible with that used in the \LaTeX~2.09
% |ifthen| style option. It has been recoded, making the resulting
% definitions somewhat more compact and efficient.
% \end{abstract}
%
% \changes{v1.0a}{1993/10/15}{New implementation}
% \changes{v1.0b}{1993/12/17}{Upgrade to LaTeX2e}
% \changes{v1.0e}{1994/02/11}{Improve documentation}
% \changes{v1.0f}{1994/03/02}{Remove need for dtx file}
% \changes{v1.0g}{1994/03/14}{Modify for the new ltxdoc.cls}
% \changes{v1.0k}{1995/04/25}{Fix `driver' docstrip guards.}
% \changes{v1.0n}{1997/11/03}{Documentation fix.}
% \changes{v1.0o}{1998/08/17}{Documentation fix.}
% \changes{v1.1d}{2022/04/13}{Make reference command locally expandable.}
%
% \section{Introduction}
%
% \DescribeMacro{\ifthenelse}
% |\ifthenelse{|^^A
%       \meta{test}|}{|\meta{then clause}|}{|\meta{else clause}|}|
%
% Evaluates \meta{test} as a boolean function, and then executes
% either \meta{then clause} or \meta{else clause}.
%
% \meta{test} is a boolean expression using the infix connectives,
% |\and|, |\or|, the unary |\not| and parentheses |\( \)|.
%
% As an alternative notation |\AND|, |\OR| and |\NOT| can be
%    used. This is safer since it can't be misinterpreted
%    when appearing inside a \TeX-conditional in which |\or| has a
%    different meaning.
%
% The atomic  propositions are:\\
% \meta{number} |<| \meta{number} \\
% \meta{number} |=| \meta{number} \\
% \meta{number} |>| \meta{number} \\
% |\isodd{| \meta{number} |}|\\
% |\isundefined{| \meta{command name} |}|\\
% |\equal{|\meta{string}|}{|\meta{string}|}|\\
% |\lengthtest{|\meta{dimen}|<|\meta{dimen}|}|\\
% |\lengthtest{|\meta{dimen}|=|\meta{dimen}|}|\\
% |\lengthtest{|\meta{dimen}|>|\meta{dimen}|}|\\
% |\boolean{|\meta{name}|}|
%
% The \meta{string}s tested by |\equal| may be any sequence of commands
% that expand to a list of tokens. If these expansions are equal, then
% the proposition is true.
%
% |\isodd| is true if the \meta{number} is odd, and false otherwise
% (even if the argument is not a number).
%
% |\isundefined{\cmd}| is true if |\cmd| is not defined.
%
% |\boolean{xyz}| returns the truth value contained in the primitive
% \TeX\ |\if|, |\ifxyz|. This is usually used with boolean flags
% created with |\newboolean| and  |\provideboolean| described below.
% It can also be used with the names of |\newif| created tokens, and
% primitive \TeX\ |\if| constructs, for example |\boolean{true}|
% (|\iftrue|),  |\boolean{mmode}| (|\ifmmode|) etc.
%
%
%  The commands:\\
%  |\newboolean|\marg{name}\DescribeMacro{\newboolean}\ and\
%  |\provideboolean|\marg{name}\DescribeMacro{\provideboolean}\
%  are provided so the user can easily create  new boolean flags.
%  As for |\newcommand|, |\newboolean| generates an error if the
%  command name is not new. |\provideboolean| silently does nothing in
%  that case.
%
%  The boolean flags may be set with:\\
%  |\setboolean|\marg{name}\marg{value}\DescribeMacro{\setboolean}\\
% \meta{value} may be either |true| or |false| (any CaSe).
%
% Note that there is no precedence between |\and| and |\or|.
% The proposition is evaluated in a left right manner. |\not| only
% applies to the immediately following proposition. (This is consistent
% with Lamport's |ifthen.sty|.) In this style, though the test is
% `lazily' evaluated, so for instance if the first proposition in an
% |\or| is true, the second one is skipped. (On the second pass---the
% first pass in an |\edef| expands clauses in all propositions.)
%
% Apart from the addition of the extra atomic propositions |\isodd|,
% |\boolean|, |\lengthtest| and |\isundefined|,
% the only known incompatibility is that
% in this package the expression|\not\not|\meta{P} is equivalent to
% \meta{P}.
% However in the original style it was equivalent to |\not|\meta{P}.
% This is intentional (bug fix:-).
%
% \DescribeMacro{\whiledo}
% The command |\whiledo| is also defined (copied directly from
% the \LaTeX2.09 definition).
%
% |\whiledo{|\meta{test}|}{|\meta{while clause}|}|
%
% With \meta{test} as above, repeatedly executes \meta{while clause}
% while the test remains true.
%
% \MaybeStop{}
%
% \section{The Implementation}
%
%    \begin{macrocode}
%<*package>
%    \end{macrocode}
%
% \begin{macro}{\TE@throw}
% In order to support the syntax of |ifthen.sty|, which allows access
% to the primitive \TeX\ syntax for a numeric test, rather than a |{}|
% delimited argument form, it is most convenient to work `within' an
% |\ifnum|. |\ift@throw| `throws' you out of the current |\ifnum| so
% that you can (eg) start an |\ifdim| for the length tests.
%    \begin{macrocode}
\def\TE@throw{\@ne=\@ne\noexpand\fi}
%    \end{macrocode}
% \end{macro}
%
% \begin{macro}{\boolean}
% A non-standard extension to |ifthen|, supporting boolean was
% previously available, this is a simpler implementation.
%    \begin{macrocode}
\def\boolean#1#2{%
  \TE@throw\expandafter\noexpand\csname if#1\endcsname#2}
%    \end{macrocode}
% \end{macro}
%
% \begin{macro}{\TE@length}
% Testing lengths. |#1| is the test. The extra argument gobbles spaces.
%    \begin{macrocode}
\def\TE@length#1#2{\TE@throw\noexpand\ifdim#1#2}
%    \end{macrocode}
% \end{macro}
%
% \begin{macro}{\TE@odd}
% \begin{macro}{\TE@@odd}
% Testing odd/even. This is true if |#1| is an odd number, and false
% otherwise (even if |#1| is not a number at all).
%
% It is hard to make this completely reliable. Here I have erred on the
% side of safety. This should not generate a \TeX\ error if given any
% robust commands as its argument. However it returns true on any
% argument that \emph{starts} with an odd number |11xx| which is bad,
% and it can not deal with \TeX's count registers, although \LaTeX\
% counters work (via |\value|).
% \changes{v1.0b}{1993/12/17}{Improve \cs{isodd}.}
% \changes{v1.0c}{1994/01/20}{Improve \cs{isodd} again.}
%    \begin{macrocode}
\def\TE@odd#1#2{%
  \TE@throw\noexpand\TE@@odd#1\noexpand\@nil\noexpand\ifodd\count@#2}
%    \end{macrocode}
%
% |\TE@@odd| is not expanded on the first pass.
%    \begin{macrocode}
\def\TE@@odd#1#2\@nil{%
  \@defaultunits
  \count@\if-#1-0\else0\expandafter#1\fi#2\relax\@nnil}
%    \end{macrocode}
% \end{macro}
% \end{macro}
%
%  \begin{macro}{\TE@repl}
% |\TE@repl| replaces the single token |#1| by |#2|. (Not within |{}|
% groups.) It is used to replace |\or| by |\TE@or| without the need to
% redefine |\or|. Earlier versions just |\let\or\TE@or| but this has a
% bad effect on the expansion of commands which use the primitive
% |\or| internally, eg |\alph|, and so caused surprising results if
% these commands were used inside |\equal|.
% \changes{v1.0h}{1994/05/14}{macro added}
%    \begin{macrocode}
\def\TE@repl#1#2{%
  \long\def\@tempc##1#1##2{%
    \def\@tempa{##2}\def\@tempb{\@tempc}%
    \ifx\@tempa\@tempb
     \toks@\expandafter{\the\toks@##1}%
     \expandafter\@gobble
    \else
      \toks@\expandafter{\the\toks@##1#2}%
    \expandafter\@tempc
    \fi
    ##2}%
  \expandafter\toks@\expandafter{\expandafter}%
  \expandafter\@tempc\the\toks@#1\@tempc}
%    \end{macrocode}
%  \end{macro}
%
%
% \begin{macro}{\ifthenelse}
% The remaining macros in this file are derived from the ones in
% |ifthen.sty| but recoded and simplified. The main simplification is
% that the original style (and the |\boolean| extensions) expressed
% logical values always in terms of |\ifnum|. As |\fi| is `untyped' this
% is not necessary, so for example the length tests can return values
% via |\ifdim|, the trailing |\fi| will not complain, even though it was
% `expecting' an |\ifnum|. Also the system of passing information via
% macros expanding to |T| or |F| has been completely replaced by a
% simpler system using |\iftrue|, which furthermore allows lazy
% evaluation on the second pass.
% With a LaTeX 2022/06/01 we have to ensure that |\pageref| is expandable.
%    \begin{macrocode}
\@ifl@t@r\fmtversion{2022/06/01}
  {\def\TE@ref@exp{\let\pageref\@kernel@pageref@exp
                  \let\ref\@kernel@ref@exp}}
  {\def\TE@ref@exp{\def\@setref##1##2##3{%
         \ifx##1\relax\z@\else\expandafter##2##1\fi}}}
%    \end{macrocode}
%    \begin{macrocode}
\long\def\ifthenelse#1{%
%    \end{macrocode}
% \changes{v1.0h}{1994/05/14}{Use \cs{TE@repl}}
%    \begin{macrocode}
  \toks@{#1}%
  \TE@repl\or\TE@or
  \TE@repl\and\TE@and
  \TE@repl\not\TE@neg
%    \end{macrocode}
%
% \changes{v1.1c}{2001/05/25}{Added \cs{AND}, \cs{OR} and \cs{NOT} as
%    boolean operators as alternative to the lower case versions
%    (pr/3297)}
%    Support alternate names for the boolean operators (strictly
%    speaking only |\OR| would be necessary).
%    \begin{macrocode}
  \TE@repl\OR\TE@or
  \TE@repl\AND\TE@and
  \TE@repl\NOT\TE@neg
%    \end{macrocode}
% The original |ifthen.sty| processed everything inside a box
% assignment, to catch any extra spaces before they appeared in the
% output. Instead I have added extra arguments to the commands so they
% each remove any following space.
%
% Set up the user level names |\not| etc.
% \changes{v1.0c}{1994/01/20}{Modify \cs{protect} and \cs{value}}
% \changes{v1.0j}{1994/11/15}{Modify \cs{protect} add \cs{@setref}}
% \changes{v1.0l}{1996/03/22}
%      {Use \cs{begingroup} not \cs{bgroup} for latex/2105}
% \changes{v1.1a}{1999/01/07}{\cs{isundefined} added for /2824}
% \changes{v1.1d}{2022/04/13}{changed method to make \cs{pageref} and \cs{ref} locally expandable.}
%    \begin{macrocode}
    \begingroup
        \let\protect\@unexpandable@protect
        \TE@ref@exp
        \def\value##1{\the\csname c@##1\endcsname}%
        \let\equal\TE@equal \let\(\TE@lparen \let\)\TE@rparen
        \let\isodd\TE@odd \let\lengthtest\TE@length
        \let\isundefined\TE@undef
%    \end{macrocode}
% For the first pass, in a group, make various tokens non-expandable.
%
% It is unfortunate that in order to remain compatible with |ifthen|
% syntax, it is necessary to have a two pass system. The first pass
% inside an |\edef| `exposes' the |\if|\ldots\ |\fi| tokens, so the
% correct clauses may be skipped on the second pass. This means that the
% whole |\ifthenelse| command does not work by expansion, and so
% possibly has only limited usefulness for macro code writers.
% The main problem with the |ifthen:| syntax is that (unique for \LaTeX)
% it does not uses a brace delimited argument form, and exposes the
% primitive \TeX\ syntax for \meta{number}. Pretty much the only way of
% parsing |1 > 2 \or 2 < 1| is to actually evaluate the primitive
% |\ifnum|s. A syntax such as:\\%
% |\or{\numtest{1<2}}{\lengthtest{1pt<1in}}|\\
%  could easily be evaluated in a one pass way, operating directly via
%  expansion, and leaving no extra tokens in the token stream.
%
% Still, on with the code\ldots\ make |\@tempa| and |\@tempb| tokens
% non-expandable on the first pass.
% \changes{v1.0l}{1996/03/22}
%      {Use \cs{begingroup} not \{ for latex/2105}
%    \begin{macrocode}
        \begingroup
          \let\@tempa\relax\let\@tempb\relax
          \xdef\@gtempa{\expandafter\TE@eval\the\toks@\TE@endeval}%
        \endgroup
%    \end{macrocode}
% Now outside the group, execute |\@gtempa| which causes all the
% |\if|s etc., to be evaluated, the final truth value is contained in
% the |\newif| token |\ifTE@val|. Finally this is tested and either the
% first or second following argument is chosen accordingly.
% \changes{v1.0d}{1994/01/24}
%     {Use \cs{@firstoftwo} not \cs{@leftmark}.}
%    \begin{macrocode}
        \@gtempa
        \expandafter\endgroup\ifTE@val
          \expandafter\@firstoftwo
        \else
          \expandafter\@secondoftwo
        \fi}
%    \end{macrocode}
% \end{macro}
%
% \begin{macro}{\TE@eval}
% Initialise a term. (Expanded on the first pass).
%    \begin{macrocode}
\def\TE@eval{\noexpand\TE@negatefalse\noexpand\iftrue\noexpand\ifnum}
%    \end{macrocode}
% \end{macro}
%
% \begin{macro}{\ifTE@val}
% \begin{macro}{\ifTE@negate}
% Two |\newif|s the first holds the current truth value of the
% expression. The second is a temporary flag which is true if we need to
% negate the current proposition.
%    \begin{macrocode}
\newif\ifTE@val
\newif\ifTE@negate
%    \end{macrocode}
% \end{macro}
% \end{macro}
%
% \begin{macro}{\TE@endeval}
% Finalize a term. (Expanded on the first pass).
%    \begin{macrocode}
\def\TE@endeval{\relax
      \noexpand\TE@setvaltrue\noexpand
    \else
      \noexpand\TE@setvalfalse\noexpand
    \fi
    \noexpand\TE@negatefalse\noexpand
  \fi}
%    \end{macrocode}
% \end{macro}
%
% \begin{macro}{\TE@setvaltrue}
% \begin{macro}{\TE@setvalfalse}
% Set the |\ifTE@val| to true or false depending on the value of the
% current proposition, and the negate flag. (Not expanded on the first
% pass.)
%    \begin{macrocode}
\def\TE@setvaltrue{%
  \ifTE@negate\TE@valfalse\else\TE@valtrue\fi}
\def\TE@setvalfalse{\let\ifTE@val\ifTE@negate}
%    \end{macrocode}
% \end{macro}
% \end{macro}
%
% \begin{macro}{\TE@or}
% The internal version of |\or|. Ends the current term.
% If true skip the remaining terms.
%    \begin{macrocode}
\def\TE@or{\TE@endeval\noexpand\ifTE@val\noexpand\else\noexpand\ifnum}
%    \end{macrocode}
% \end{macro}
%
% \begin{macro}{\TE@and}
% The internal version of |\and|. If false skip the remaining terms.
%    \begin{macrocode}
\def\TE@and{\TE@endeval\noexpand\ifTE@val\noexpand\ifnum}
%    \end{macrocode}
% \end{macro}
%
% \begin{macro}{\TE@neg}
% \begin{macro}{\TE@negswitch}
% |\not|. Throw the current context, set a negate flag, then restart
% the |\ifnum|. |\TE@negswitch| is not expanded on the first pass.
%    \begin{macrocode}
\def\TE@neg{\TE@throw\noexpand\TE@negswitch\noexpand\ifnum}
\def\TE@negswitch{\ifTE@negate\TE@negatefalse\else\TE@negatetrue\fi}
%    \end{macrocode}
% \end{macro}
% \end{macro}
%
% \begin{macro}{\TE@lparen}
% |\(|. Throw the current context, then restart a term inside a group.
%    \begin{macrocode}
\def\TE@lparen#1{\TE@throw\begingroup\TE@eval#1}
%    \end{macrocode}
% \end{macro}
%
% \begin{macro}{\TE@rparen}
% |\)| end the current term, and the local group started by |\(|, but
% pass on the boolean value in |\if\@val T|. The |\noexpand| stops the
% |\expandafter| from expanding on the first pass.
%    \begin{macrocode}
\def\TE@rparen#1{%
  \TE@endeval
  \noexpand\expandafter\endgroup\noexpand\ifTE@val#1}
%    \end{macrocode}
% \end{macro}
%
% \begin{macro}{\TE@equal}
% |\equal| greatly simplified from the original. |\def| may be used
% rather than |\edef| as the whole thing is expanded anyway in the
% first pass. The boolean can be directly encoded with the |\ifx|,
% there is no need to start an equivalent |\ifnum|.
% \changes{v1.0h}{1994/05/14}{make long}
%    \begin{macrocode}
\long\def\TE@equal#1#2#3{\TE@throw
      \def\@tempa{#1}\def\@tempb{#2}%
      \noexpand\ifx\@tempa\@tempb#3}
%    \end{macrocode}
% \end{macro}
%
% \begin{macro}{\setboolean}
% |\setboolean| takes |true| or |false|, as |#2|, and sets |#1|
% accordingly.
% \changes{v1.0i}{1994/05/27}{New style error commands}
%    \begin{macrocode}
\def\setboolean#1#2{%
  \lowercase{\def\@tempa{#2}}%
  \@ifundefined{@tempswa\@tempa}%
    {\PackageError{ifthen}%
       {You can only set a boolean to `true' or `false'}\@ehc}%
    {\@ifundefined{#1\@tempa}%
      {\PackageError{ifthen}{Boolean #1 undefined}\@ehc}%
      {\csname#1\@tempa\endcsname}}}
%    \end{macrocode}
% \end{macro}
%
% \begin{macro}{\newboolean}
% \changes{v1.0m}{1996/08/02}{Add \cs{@ifdefinable} test for latex/2220}
% Define a new `boolean'.
%    \begin{macrocode}
\def\newboolean#1{%
  \expandafter\@ifdefinable\csname if#1\endcsname{%
    \expandafter\newif\csname if#1\endcsname}}
%    \end{macrocode}
% \end{macro}
%
% \begin{macro}{\provideboolean}
% \changes{v1.0m}{1996/08/02}{Macro added for latex/2220}
% Define a new `boolean' if it is not already defined.
%    \begin{macrocode}
\def\provideboolean#1{%
  \@ifundefined{if#1}{%
    \expandafter\newif\csname if#1\endcsname}\relax}
%    \end{macrocode}
% \end{macro}
%
% \begin{macro}{\whiledo}
% |\whiledo| copied directly from the original.\\
% |\whiledo{|\meta{test}|}{|\meta{body}|}|\\
% repeatedly evaluates \meta{body} until \meta{test} is true.
%    \begin{macrocode}
\long\def\whiledo#1#2{%
  \ifthenelse{#1}%
    {\@whiledotrue
     \@whilesw\if@whiledo\fi
       {#2%
        \ifthenelse{#1}\@whiledotrue\@whiledofalse}}%
    {}%
  }
%    \end{macrocode}
% \end{macro}
%
% \begin{macro}{\TE@undef}
% \changes{v1.1a}{1999/01/07}{Macro added for /2824}
% test if csname is defined. |\ifx| test.
%    \begin{macrocode}
\def\TE@undef#1#2{%
  \TE@throw\noexpand\ifx\noexpand\@undefined\noexpand#1#2}
%    \end{macrocode}
% \end{macro}
%
% \begin{macro}{\if@whiledo}
% Internal switch for |\whiledo|.
%    \begin{macrocode}
\newif\if@whiledo
%    \end{macrocode}
% \end{macro}
%
%    \begin{macrocode}
%</package>
%    \end{macrocode}
%
% \Finale
\endinput
