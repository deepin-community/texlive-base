% \iffalse meta-comment
%
% File: luacolor.dtx
% Version: 2023-08-18 v1.18
% Info: Color support via LuaTeX's attributes
%
% Copyright (C)
%    2007, 2009-2011 Heiko Oberdiek
%    2016-2023 Oberdiek Package Support Group
%    https://github.com/ho-tex/luacolor/issues
%
% This work may be distributed and/or modified under the
% conditions of the LaTeX Project Public License, either
% version 1.3c of this license or (at your option) any later
% version. This version of this license is in
%    https://www.latex-project.org/lppl/lppl-1-3c.txt
% and the latest version of this license is in
%    https://www.latex-project.org/lppl.txt
% and version 1.3 or later is part of all distributions of
% LaTeX version 2005/12/01 or later.
%
% This work has the LPPL maintenance status "maintained".
%
% The Current Maintainers of this work are
% Heiko Oberdiek and the Oberdiek Package Support Group
% https://github.com/ho-tex/luacolor/issues
%
% This work consists of the main source file luacolor.dtx
% and the derived files
%    luacolor.sty, luacolor.pdf, luacolor.lua,
%
% Distribution:
%    CTAN:macros/latex/contrib/luacolor/luacolor.dtx
%    CTAN:macros/latex/contrib/luacolor/luacolor.pdf
%
% Unpacking:
%    (a) If luacolor.ins is present:
%           tex luacolor.ins
%    (b) Without luacolor.ins:
%           tex luacolor.dtx
%    (c) If you insist on using LaTeX
%           latex \let\install=y% \iffalse meta-comment
%
% File: luacolor.dtx
% Version: 2023-08-18 v1.18
% Info: Color support via LuaTeX's attributes
%
% Copyright (C)
%    2007, 2009-2011 Heiko Oberdiek
%    2016-2023 Oberdiek Package Support Group
%    https://github.com/ho-tex/luacolor/issues
%
% This work may be distributed and/or modified under the
% conditions of the LaTeX Project Public License, either
% version 1.3c of this license or (at your option) any later
% version. This version of this license is in
%    https://www.latex-project.org/lppl/lppl-1-3c.txt
% and the latest version of this license is in
%    https://www.latex-project.org/lppl.txt
% and version 1.3 or later is part of all distributions of
% LaTeX version 2005/12/01 or later.
%
% This work has the LPPL maintenance status "maintained".
%
% The Current Maintainers of this work are
% Heiko Oberdiek and the Oberdiek Package Support Group
% https://github.com/ho-tex/luacolor/issues
%
% This work consists of the main source file luacolor.dtx
% and the derived files
%    luacolor.sty, luacolor.pdf, luacolor.lua,
%
% Distribution:
%    CTAN:macros/latex/contrib/luacolor/luacolor.dtx
%    CTAN:macros/latex/contrib/luacolor/luacolor.pdf
%
% Unpacking:
%    (a) If luacolor.ins is present:
%           tex luacolor.ins
%    (b) Without luacolor.ins:
%           tex luacolor.dtx
%    (c) If you insist on using LaTeX
%           latex \let\install=y% \iffalse meta-comment
%
% File: luacolor.dtx
% Version: 2023-08-18 v1.18
% Info: Color support via LuaTeX's attributes
%
% Copyright (C)
%    2007, 2009-2011 Heiko Oberdiek
%    2016-2023 Oberdiek Package Support Group
%    https://github.com/ho-tex/luacolor/issues
%
% This work may be distributed and/or modified under the
% conditions of the LaTeX Project Public License, either
% version 1.3c of this license or (at your option) any later
% version. This version of this license is in
%    https://www.latex-project.org/lppl/lppl-1-3c.txt
% and the latest version of this license is in
%    https://www.latex-project.org/lppl.txt
% and version 1.3 or later is part of all distributions of
% LaTeX version 2005/12/01 or later.
%
% This work has the LPPL maintenance status "maintained".
%
% The Current Maintainers of this work are
% Heiko Oberdiek and the Oberdiek Package Support Group
% https://github.com/ho-tex/luacolor/issues
%
% This work consists of the main source file luacolor.dtx
% and the derived files
%    luacolor.sty, luacolor.pdf, luacolor.lua,
%
% Distribution:
%    CTAN:macros/latex/contrib/luacolor/luacolor.dtx
%    CTAN:macros/latex/contrib/luacolor/luacolor.pdf
%
% Unpacking:
%    (a) If luacolor.ins is present:
%           tex luacolor.ins
%    (b) Without luacolor.ins:
%           tex luacolor.dtx
%    (c) If you insist on using LaTeX
%           latex \let\install=y% \iffalse meta-comment
%
% File: luacolor.dtx
% Version: 2023-08-18 v1.18
% Info: Color support via LuaTeX's attributes
%
% Copyright (C)
%    2007, 2009-2011 Heiko Oberdiek
%    2016-2023 Oberdiek Package Support Group
%    https://github.com/ho-tex/luacolor/issues
%
% This work may be distributed and/or modified under the
% conditions of the LaTeX Project Public License, either
% version 1.3c of this license or (at your option) any later
% version. This version of this license is in
%    https://www.latex-project.org/lppl/lppl-1-3c.txt
% and the latest version of this license is in
%    https://www.latex-project.org/lppl.txt
% and version 1.3 or later is part of all distributions of
% LaTeX version 2005/12/01 or later.
%
% This work has the LPPL maintenance status "maintained".
%
% The Current Maintainers of this work are
% Heiko Oberdiek and the Oberdiek Package Support Group
% https://github.com/ho-tex/luacolor/issues
%
% This work consists of the main source file luacolor.dtx
% and the derived files
%    luacolor.sty, luacolor.pdf, luacolor.lua,
%
% Distribution:
%    CTAN:macros/latex/contrib/luacolor/luacolor.dtx
%    CTAN:macros/latex/contrib/luacolor/luacolor.pdf
%
% Unpacking:
%    (a) If luacolor.ins is present:
%           tex luacolor.ins
%    (b) Without luacolor.ins:
%           tex luacolor.dtx
%    (c) If you insist on using LaTeX
%           latex \let\install=y\input{luacolor.dtx}
%        (quote the arguments according to the demands of your shell)
%
% Documentation:
%    (a) If luacolor.drv is present:
%           latex luacolor.drv
%    (b) Without luacolor.drv:
%           latex luacolor.dtx; ...
%    The class ltxdoc loads the configuration file ltxdoc.cfg
%    if available. Here you can specify further options, e.g.
%    use A4 as paper format:
%       \PassOptionsToClass{a4paper}{article}
%
%    Programm calls to get the documentation (example):
%       pdflatex luacolor.dtx
%       makeindex -s gind.ist luacolor.idx
%       pdflatex luacolor.dtx
%       makeindex -s gind.ist luacolor.idx
%       pdflatex luacolor.dtx
%
% Installation:
%    TDS:tex/latex/luacolor/luacolor.sty
%    TDS:scripts/luacolor/luacolor.lua
%    TDS:doc/latex/luacolor/luacolor.pdf
%    TDS:source/latex/luacolor/luacolor.dtx
%
%<*ignore>
\begingroup
  \catcode123=1 %
  \catcode125=2 %
  \def\x{LaTeX2e}%
\expandafter\endgroup
\ifcase 0\ifx\install y1\fi\expandafter
         \ifx\csname processbatchFile\endcsname\relax\else1\fi
         \ifx\fmtname\x\else 1\fi\relax
\else\csname fi\endcsname
%</ignore>
%<*install>
\input docstrip.tex
\Msg{************************************************************************}
\Msg{* Installation}
\Msg{* Package: luacolor 2023-08-18 v1.18 Color support via LuaTeX's attributes (HO)}
\Msg{************************************************************************}

\keepsilent
\askforoverwritefalse

\let\MetaPrefix\relax
\preamble

This is a generated file.

Project: luacolor
Version: 2023-08-18 v1.18

Copyright (C)
   2007, 2009-2011 Heiko Oberdiek
   2016-2023 Oberdiek Package Support Group

This work may be distributed and/or modified under the
conditions of the LaTeX Project Public License, either
version 1.3c of this license or (at your option) any later
version. This version of this license is in
   https://www.latex-project.org/lppl/lppl-1-3c.txt
and the latest version of this license is in
   https://www.latex-project.org/lppl.txt
and version 1.3 or later is part of all distributions of
LaTeX version 2005/12/01 or later.

This work has the LPPL maintenance status "maintained".

The Current Maintainers of this work are
Heiko Oberdiek and the Oberdiek Package Support Group
https://github.com/ho-tex/luacolor/issues


This work consists of the main source file luacolor.dtx
and the derived files
   luacolor.sty, luacolor.pdf, luacolor.ins, luacolor.drv,
   luacolor.lua,

\endpreamble
\let\MetaPrefix\DoubleperCent

\generate{%
  \file{luacolor.ins}{\from{luacolor.dtx}{install}}%
  \file{luacolor.drv}{\from{luacolor.dtx}{driver}}%
  \usedir{tex/latex/luacolor}%
  \file{luacolor.sty}{\from{luacolor.dtx}{package}}%
}
\def\MetaPrefix{-- }
\def\defaultpostamble{%
  \MetaPrefix^^J%
  \MetaPrefix\space End of File `\outFileName'.%
}
\def\currentpostamble{\defaultpostamble}%
\generate{%
  \usedir{scripts/luacolor}%
%  \file{oberdiek.luacolor.lua}{\from{luacolor.dtx}{lua}}%
  \file{luacolor.lua}{\from{luacolor.dtx}{lua}}%
}

\catcode32=13\relax% active space
\let =\space%
\Msg{************************************************************************}
\Msg{*}
\Msg{* To finish the installation you have to move the following}
\Msg{* file into a directory searched by TeX:}
\Msg{*}
\Msg{*     luacolor.sty}
\Msg{*}
\Msg{* And install the following script file:}
\Msg{*}
\Msg{*     luacolor.lua,}
\Msg{*}
\Msg{* To produce the documentation run the file `luacolor.dtx'}
\Msg{* through LaTeX.}
\Msg{*}
\Msg{* Happy TeXing!}
\Msg{*}
\Msg{************************************************************************}

\endbatchfile
%</install>
%<*ignore>
\fi
%</ignore>
%<*driver>
\NeedsTeXFormat{LaTeX2e}
\ProvidesFile{luacolor.drv}%
  [2023-08-18 v1.18 Color support via LuaTeX's attributes (HO)]%
\documentclass{ltxdoc}
\usepackage{holtxdoc}[2011/11/22]
\usepackage{paralist}
\makeatletter
\g@addto@macro\MakePrivateLetters{%
  \@makeother\_%
}
\makeatother
\begin{document}
  \DocInput{luacolor.dtx}%
\end{document}
%</driver>
% \fi
%
%
%
% \GetFileInfo{luacolor.drv}
%
% \title{The \xpackage{luacolor} package}
% \date{2023-08-18 v1.18}
% \author{Heiko Oberdiek\thanks
% {Please report any issues at \url{https://github.com/ho-tex/luacolor/issues}}}
%
% \maketitle
%
% \begin{abstract}
% Package \xpackage{luacolor} implements color support based
% on \LuaTeX's node attributes.
% \end{abstract}
%
% \tableofcontents
%
% \section{Documentation}
%
% \subsection{Introduction}
%
% This package uses a \LuaTeX's attribute register to
% to annotate nodes with color information.
% If a color is set, then the attribute register
% is set to this color and all nodes created in its scope
% (current group) are annotated with this attribute.
% Now the color property behaves much the same way
% as the font property.
%
% \subsection{Usage}
%
% Package \xpackage{color} is loaded automatically by this
% package \xpackage{luacolor}. If you need a special driver
% option or you prefer package \xpackage{xcolor}, then load
% it before package \xpackage{luacolor}, for example:
% \begin{quote}
%   |\usepackage[dvipdfmx]{xcolor}|
% \end{quote}
% The package \xpackage{luacolor} is loaded without options:
% \begin{quote}
%   |\usepackage{luacolor}|
% \end{quote}
% It is able to detect PDF mode and DVI drivers are
% differentiated by its color specials. Therefore the
% package do need driver options.
%
% Then it redefines the color setting commands to set
% attributes instead of whatsits for color.
%
% At last the attribute annotations of the nodes in the
% output box must be analyzed to insert the necessary color whatsits.
% (With older \LuaTeX\ that lack the appropriate callback function 
% the package \xpackage{atbegshi} is used to get control
% before a box is shipped out.)
%
% \begin{declcs}{luacolorProcessBox} \M{box}
% \end{declcs}
% Macro \cs{luacolorProcessBox} processes the box \meta{box}
% in the previously described manner. It is automatically
% called for pages, but not for XForm objects.
% Before passing a box to \cs{pdfxform}, call \cs{luacolorProcessBox}
% first.
%
% \subsection{Limitations}
%
% \begin{description}
% \item[Ligatures with different colored components:]
%   Package \xpackage{luacolor} sees the ligature after
%   the paragraph building and page breaking, when a
%   page is to be shipped out. Therefore it cannot
%   break ligatures, because the components might occupy
%   different space. Therefore it is the responsibility
%   of the ligature forming process to deal with different
%   colored glyphs that form a ligature. The user can avoid
%   the problem entirely by explicitly breaking the ligature
%   at the places where the color changes.
%  \item \dots
% \end{description}
%
% \StopEventually{
% }
%
% \section{Implementation}
%
%    \begin{macrocode}
%<*package>
%    \end{macrocode}
%
% \subsection{Catcodes and identification}
%
%    \begin{macrocode}
\begingroup\catcode61\catcode48\catcode32=10\relax%
  \catcode13=5 % ^^M
  \endlinechar=13 %
  \catcode123=1 % {
  \catcode125=2 % }
  \catcode64=11 % @
  \def\x{\endgroup
    \expandafter\edef\csname LuaCol@AtEnd\endcsname{%
      \endlinechar=\the\endlinechar\relax
      \catcode13=\the\catcode13\relax
      \catcode32=\the\catcode32\relax
      \catcode35=\the\catcode35\relax
      \catcode61=\the\catcode61\relax
      \catcode64=\the\catcode64\relax
      \catcode123=\the\catcode123\relax
      \catcode125=\the\catcode125\relax
    }%
  }%
\x\catcode61\catcode48\catcode32=10\relax%
\catcode13=5 % ^^M
\endlinechar=13 %
\catcode35=6 % #
\catcode64=11 % @
\catcode123=1 % {
\catcode125=2 % }
\def\TMP@EnsureCode#1#2{%
  \edef\LuaCol@AtEnd{%
    \LuaCol@AtEnd
    \catcode#1=\the\catcode#1\relax
  }%
  \catcode#1=#2\relax
}
\TMP@EnsureCode{34}{12}% "
\TMP@EnsureCode{39}{12}% '
\TMP@EnsureCode{40}{12}% (
\TMP@EnsureCode{41}{12}% )
\TMP@EnsureCode{42}{12}% *
\TMP@EnsureCode{43}{12}% +
\TMP@EnsureCode{44}{12}% ,
\TMP@EnsureCode{45}{12}% -
\TMP@EnsureCode{46}{12}% .
\TMP@EnsureCode{47}{12}% /
\TMP@EnsureCode{58}{12}% :
\TMP@EnsureCode{60}{12}% <
\TMP@EnsureCode{62}{12}% >
\TMP@EnsureCode{91}{12}% [
\TMP@EnsureCode{93}{12}% ]
\TMP@EnsureCode{95}{12}% _ (other!)
\TMP@EnsureCode{96}{12}% `
\edef\LuaCol@AtEnd{\LuaCol@AtEnd\noexpand\endinput}
%    \end{macrocode}
%
%    Package identification.
%    \begin{macrocode}
\NeedsTeXFormat{LaTeX2e}
\ProvidesPackage{luacolor}%
  [2023-08-18 v1.18 Color support via LuaTeX's attributes (HO)]
%    \end{macrocode}
%
% \subsection{Check for \LuaTeX}
%
%    Without \LuaTeX\ there is no point in using this package.
%    \begin{macrocode}
\RequirePackage{color}
%    \end{macrocode}
%
%    \begin{macrocode}
\ifx\directlua\@undefined
  \PackageError{luacolor}{%
    This package may only be run using LuaTeX%
  }\@ehc
  \expandafter\LuaCol@AtEnd
\fi%
%    \end{macrocode}
%
% \subsection{Check for disabled colors}
%
%    \begin{macrocode}
\ifcolors@
\else
  \PackageWarningNoLine{luacolor}{%
    Colors are disabled by option `monochrome'%
  }%
  \def\set@color{}%
  \def\reset@color{}%
  \def\set@page@color{}%
  \def\define@color#1#2{}%
  \expandafter\LuaCol@AtEnd
\fi%
%    \end{macrocode}
%
% \subsection{Load module and check version}
%
%    \begin{macrocode}
\directlua{%
  require("luacolor")%
}
%    \end{macrocode}
%    \begin{macrocode}
\begingroup
  \edef\x{\directlua{tex.write("2023-08-18 v1.18")}}%
  \edef\y{%
    \directlua{%
      if oberdiek.luacolor.getversion then %
        oberdiek.luacolor.getversion()%
      end%
    }%
  }%
  \ifx\x\y
  \else
    \PackageError{luacolor}{%
      Wrong version of lua module.\MessageBreak
      Package version: \x\MessageBreak
      Lua module: \y
    }\@ehc
  \fi
\endgroup
%    \end{macrocode}
%
% \subsection{Find driver}
%
%    \begin{macrocode}
\ifnum\outputmode=\@ne
\else
  \begingroup
    \def\current@color{}%
    \def\reset@color{}%
    \setbox\z@=\hbox{%
      \begingroup
        \set@color
      \endgroup
    }%
    \edef\reserved@a{%
      \directlua{%
        oberdiek.luacolor.dvidetect()%
      }%
    }%
    \ifx\reserved@a\@empty
      \PackageError{luacolor}{%
        DVI driver detection failed because of\MessageBreak
        unrecognized color \string\special
      }\@ehc
      \endgroup
      \expandafter\expandafter\expandafter\LuaCol@AtEnd
    \else
      \PackageInfo{luacolor}{%
        Type of color \string\special: \reserved@a
      \@gobble}%
    \fi%
  \endgroup
\fi
%    \end{macrocode}
%
% \subsection{Attribute setting}
%
%    \begin{macro}{\LuaCol@Attribute}
%    \begin{macrocode}
\newattribute\LuaCol@Attribute
\let\LuaCol@setattribute\setattribute
\directlua{%
  oberdiek.luacolor.setattribute(\number\allocationnumber)%
}
%    \end{macrocode}
%    \end{macro}
%
%    \begin{macro}{\set@color}
% change 2023-08-18: added \cs{reset@color} so that \cs{mathcolor} can gobble it,
% issue 7.   
%    \begin{macrocode}
\protected\def\set@color{%
  \LuaCol@setattribute\LuaCol@Attribute{%
    \directlua{%
      oberdiek.luacolor.get("\luaescapestring{\current@color}")%
    }%
  }%
 \aftergroup\reset@color 
}
%    \end{macrocode}
%    \end{macro}
%    \begin{macro}{\reset@color}
%    \begin{macrocode}
\def\reset@color{}
%    \end{macrocode}
%    \end{macro}
%
% \subsection{Whatsit insertion}
%
%    \begin{macro}{\luacolorProcessBox}
%    \begin{macrocode}
\def\luacolorProcessBox#1{%
  \directlua{%
    oberdiek.luacolor.process(\number#1)%
  }%
}
%    \end{macrocode}
%    \end{macro}
%
%    \begin{macrocode}
\directlua{%
  if luatexbase.callbacktypes.pre_shipout_filter then
    token.get_next()
  end
}\@secondoftwo\@gobble{
  \RequirePackage{atbegshi}[2011/01/30]
  \AtBeginShipout{%
    \luacolorProcessBox\AtBeginShipoutBox
  }
}
%    \end{macrocode}
%
%    Set default color.
%    \begin{macrocode}
\set@color
%    \end{macrocode}
%
% \subsection{\cs{pdfxform}/\cs{saveboxresource} support}
%
%    \begin{macrocode}
\ifnum\outputmode=\@ne
    \let\LuaCol@org@pdfxform\saveboxresource
%    \end{macrocode}
% First we need some helpers to allow expandable code to parse keyword style arguments:
%    \begin{macrocode}
    \def\LuaCol@iii@i@ii#1#2#3{#3{#1}{#2}}
    \def\LuaCol@ii@i#1#2{{#2#1}}
    \def\LuaCol@if@keyword#1#2#3{%
      \expanded{\unexpanded{\LuaCol@iii@i@ii{#2}{#3}}\expandafter}%
      \directlua{%
        token.put_next(token.create(token.scan_keyword(token.scan_string())
        and '@firstoftwo'
        or '@secondoftwo'))
      }{#1}%
    }
%    \end{macrocode}
% The following macro scans a integer and expands to a token equivalent to a chardef
% whose value corresponds to the scanned integer. This allows the integer to be passed
% around as a undelimited argument.
%    \begin{macrocode}
    \def\LuaCol@scan@number{%
      \directlua{
        token.put_next(token.new(token.scan_int(), token.command_id'char_given'))
      }%
    }
%    \end{macrocode}
% \TeX\ primitives like \cs{saveboxresource} read braced arguments in a special way.
% Especially they expand everything until they find a left brace. To simulate this, we
% use Lua to expand everything else:
%    \begin{macrocode}
    \def\LuaCol@scan@tobrace{%
      \directlua{
        local relax, space = token.command_id'relax', token.command_id'spacer'
        local t
        repeat
          t = token.scan_token()
        until not (t.command == relax or t.command == space)
        token.put_next(t)
      }%
    }
    \def\LuaCol@scan@boxresource@i#1#2{%
      \LuaCol@if@keyword{attr}{%
        \expanded{\unexpanded{\LuaCol@scan@boxresource@iI{#1#2attr}}%
          \expandafter\expandafter\expandafter}%
        \LuaCol@scan@tobrace
      }{%
        \LuaCol@scan@boxresource@ii{#1#2}%
      }%
    }
    \def\LuaCol@scan@boxresource@iI#1#2{\LuaCol@scan@boxresource@ii{#1{#2}}}
    \def\LuaCol@scan@boxresource@ii#1{%
      \LuaCol@if@keyword{resources}{%
        \expanded{\unexpanded{\LuaCol@scan@boxresource@iiI{#1resources}}%
          \expandafter\expandafter\expandafter}%
        \LuaCol@scan@tobrace
      }{%
        \LuaCol@scan@boxresource@iii{#1}%
      }%
    }
    \def\LuaCol@scan@boxresource@iiI#1#2{\LuaCol@scan@boxresource@iii{#1{#2}}}
    \def\LuaCol@scan@boxresource@iii#1{%
      \LuaCol@if@keyword{margin}{%
        \expanded{\unexpanded{\LuaCol@scan@boxresource@iv{#1margin }}%
          \expandafter\expandafter\expandafter}%
        \LuaCol@scan@number
      }{%
        \LuaCol@scan@boxresource@iv{#1}{}%
      }%
    }
    \def\LuaCol@scan@boxresource@iv#1#2{%
      \expanded{\unexpanded{\LuaCol@scan@boxresource@v{#1#2}}%
        \expandafter\expandafter\expandafter}%
      \LuaCol@scan@number
    }
    \def\LuaCol@scan@boxresource@v#1#2{%
      \luacolorProcessBox{#2}%
      \LuaCol@org@pdfxform#1#2%
    }

%    \end{macrocode}
% This could be written in Lua, but at least upto Lua\TeX~1.11, feeding back too many tokens
% from Lua to \TeX\ triggers a segmentation fault.
% This is written in Lua so the integer setting is expandable and does not interfere with
% a preceding \verb|\immediate|.
%    \begin{macrocode}
    \protected\def\saveboxresource{%
      \LuaCol@if@keyword{type}{%
        \expandafter
        \expanded{\unexpanded{\LuaCol@scan@boxresource@i{type }}%
          \expandafter\expandafter\expandafter}%
        \LuaCol@scan@number
      }{%
        \LuaCol@scan@boxresource@i{}{}%
      }%
    }
%    \end{macrocode}
%    Legacy alias.
%    \begin{macrocode}
    \let\pdfxform\saveboxresource
\fi
%    \end{macrocode}
%
%    \begin{macrocode}
\LuaCol@AtEnd%
%</package>
%    \end{macrocode}
%
% \subsection{Lua module}
%
%    \begin{macrocode}
%<*lua>
%    \end{macrocode}
%    Box zero contains a \cs{hbox} with the color \cs{special}.
%    That is analyzed to get the prefix for the color setting
%    \cs{special}.
%    \begin{macrocode}
oberdiek = oberdiek or {}
local luacolor = oberdiek.luacolor or {}
oberdiek.luacolor = luacolor
%    \end{macrocode}
%    \begin{macro}{getversion()}
%    \begin{macrocode}
function luacolor.getversion()
  tex.write("2023-08-18 v1.18")
end
%    \end{macrocode}
%    \end{macro}
%
% \subsubsection{Driver detection}
%
%    \begin{macrocode}
local ifpdf = tonumber(tex.outputmode or tex.pdfoutput) > 0
local prefix
local prefixes = {
  dvips   = "color ",
  dvipdfm = "pdf:sc ",
  truetex = "textcolor:",
  pctexps = "ps::",
}
local patterns = {
  ["^color "]            = "dvips",
  ["^pdf: *begincolor "] = "dvipdfm",
  ["^pdf: *bcolor "]     = "dvipdfm",
  ["^pdf: *bc "]         = "dvipdfm",
  ["^pdf: *setcolor "]   = "dvipdfm",
  ["^pdf: *scolor "]     = "dvipdfm",
  ["^pdf: *sc "]         = "dvipdfm",
  ["^textcolor:"]        = "truetex",
  ["^ps::"]              = "pctexps",
}
%    \end{macrocode}
%    \begin{macro}{info()}
%    \begin{macrocode}
local function info(msg, term)
  local target = "log"
  if term then
    target = "term and log"
  end
  texio.write_nl(target, "Package luacolor info: " .. msg .. ".")
  texio.write_nl(target, "")
end
%    \end{macrocode}
%    \end{macro}
%    \begin{macro}{dvidetect()}
%    \begin{macrocode}
function luacolor.dvidetect()
  local v = tex.box[0]
  assert(v.id == node.id("hlist"))
  for v in node.traverse_id(node.id("whatsit"), v.head) do
    if v and v.subtype == node.subtype("special") then
      local data = v.data
      for pattern, driver in pairs(patterns) do
        if string.find(data, pattern) then
          prefix = prefixes[driver]
          tex.write(driver)
          return
        end
      end
      info("\\special{" .. data .. "}", true)
      return
    end
  end
  info("Missing \\special", true)
end
%    \end{macrocode}
%    \end{macro}
%
% \subsubsection{Color strings}
%
%    \begin{macrocode}
local map = {
  n = 0,
}
%    \end{macrocode}
%    \begin{macro}{get()}
%    \begin{macrocode}
function luacolor.get(color)
  tex.write("" .. luacolor.getvalue(color))
end
%    \end{macrocode}
%    \end{macro}
%    \begin{macro}{getvalue()}
%    \begin{macrocode}
function luacolor.getvalue(color)
  local n = map[color]
  if not n then
    n = map.n + 1
    map.n = n
    map[n] = color
    map[color] = n
  end
  return n
end
%    \end{macrocode}
%    \end{macro}
%
% \subsubsection{Attribute register}
%
%    \begin{macro}{setattribute()}
%    \begin{macrocode}
local attribute
function luacolor.setattribute(attr)
  attribute = attr
end
%    \end{macrocode}
%    \end{macro}
%
%    \begin{macro}{getattribute()}
%    \begin{macrocode}
function luacolor.getattribute()
  return attribute
end
%    \end{macrocode}
%    \end{macro}
%
% \subsubsection{Whatsit insertion}
%
%    \begin{macrocode}
local LIST = 1
local LIST_LEADERS = 2
local LIST_DISC = 3
local COLOR = 4
local NOCOLOR = 5
local RULE = node.id("rule")
local node_types = {
  [node.id("hlist")] = LIST,
  [node.id("vlist")] = LIST,
  [node.id("rule")]  = COLOR,
  [node.id("glyph")] = COLOR,
  [node.id("disc")]  = LIST_DISC,
  [node.id("whatsit")] = {
    [node.subtype("pdf_colorstack")] =
      function(n)
        return n.stack == 0 and NOCOLOR or nil
      end,
    [node.subtype("special")] = COLOR,
    [node.subtype("pdf_literal")] = COLOR,
    [node.subtype("pdf_save")] = COLOR,
    [node.subtype("pdf_restore")] = COLOR, -- probably not needed
-- TODO (DPC)    [node.subtype("pdf_refximage")] = COLOR,
  },
  [node.id("glue")] =
    function(n)
      if n.subtype >= 100 then -- leaders
        if n.leader.id == RULE then
          return COLOR
        else
          return LIST_LEADERS
        end
      end
    end,
}
%    \end{macrocode}
%    \begin{macro}{get_type()}
%    \begin{macrocode}
local function get_type(n)
  local ret = node_types[n.id]
  if type(ret) == 'table' then
    ret = ret[n.subtype]
  end
  if type(ret) == 'function' then
    ret = ret(n)
  end
  return ret
end
%    \end{macrocode}
%    \end{macro}
%    \begin{macrocode}
local mode = 2 -- luatex.pdfliteral.direct
local WHATSIT = node.id("whatsit")
local SPECIAL = node.subtype("special")
local PDFLITERAL = node.subtype("pdf_literal")
local DRY_FALSE = false
local DRY_TRUE = true
%    \end{macrocode}
%    \begin{macro}{traverse()}
%    \begin{macrocode}
local function traverse(list, color, dry)
  if not list then
    return color
  end
  local head
  if get_type(list) == LIST then
    head = list.head
  elseif get_type(list) == LIST_DISC then
    head = list.replace
  else
    texio.write_nl("!!! Error: Wrong list type: " .. node.type(list.id))
    return color
  end
%<debug>texio.write_nl("traverse: " .. node.type(list.id))
  for n in node.traverse(head) do
%<debug>texio.write_nl("  node: " .. node.type(n.id))
    local t = get_type(n)
%<debug>texio.write_nl("TYPE "..tostring(t).. " "..tostring(node.type(node.getid(n))).." ".. tostring(node.getsubtype(n)))
    if t == LIST or t == LIST_DISC then
      color = traverse(n, color, dry)
    elseif t == LIST_LEADERS then
      local color_after = traverse(n.leader, color, DRY_TRUE)
      if color == color_after then
        traverse(n.leader, color, DRY_FALSE or dry)
      else
        traverse(n.leader, '', DRY_FALSE or dry)
%    \end{macrocode}
% The color status is unknown here, because the leader box
% will or will not be set.
%    \begin{macrocode}
        color = ''
      end
    elseif t == COLOR then
      local v = node.has_attribute(n, attribute)
      if v then
        local newColor = map[v]
        if newColor ~= color then
          color = newColor
          if dry == DRY_FALSE then
            local newNode
            if ifpdf then
              newNode = node.new(WHATSIT, PDFLITERAL)
              newNode.mode = mode
              newNode.data = color
            else
              newNode = node.new(WHATSIT, SPECIAL)
              newNode.data = prefix .. color
            end
            head = node.insert_before(head, n, newNode)
          end
        end
      end
    elseif t == NOCOLOR then
      color = ''
    end
  end
  if get_type(list) == LIST then
    list.head = head
  else
    list.replace = head
  end
  return color
end
%    \end{macrocode}
%    \end{macro}
%
%    \begin{macro}{process()}
%    \begin{macrocode}
function luacolor.process(box)
  local color = ""
  local list = tex.getbox(box)
  traverse(list, color, DRY_FALSE)
end

if luatexbase.callbacktypes.pre_shipout_filter then
  luatexbase.add_to_callback("pre_shipout_filter", function(list)
    traverse(list, "", DRY_FALSE)
    return true
  end, "luacolor.process")
end
%    \end{macrocode}
%    \end{macro}
% For recent versions of luaotfload, we can register a callback to
% control how coloring glyph is handled for the color feature.
%    \begin{macrocode}
if luaotfload.set_colorhandler then
  local set_attribute = node.direct.set_attribute
  luaotfload.set_colorhandler(function(head, n, color)
    set_attribute(n, attribute, luacolor.getvalue(color))
    return head, n
  end)
end
%    \end{macrocode}
%
%    \begin{macrocode}
%</lua>
%    \end{macrocode}
% \section{Installation}
%
% \subsection{Download}
%
% \paragraph{Package.} This package is available on
% CTAN\footnote{\CTANpkg{luacolor}}:
% \begin{description}
% \item[\CTAN{macros/latex/contrib/luacolor/luacolor.dtx}] The source file.
% \item[\CTAN{macros/latex/contrib/luacolor/luacolor.pdf}] Documentation.
% \end{description}
%
%
% \paragraph{Bundle.} All the packages of the bundle `luacolor'
% are also available in a TDS compliant ZIP archive. There
% the packages are already unpacked and the documentation files
% are generated. The files and directories obey the TDS standard.
% \begin{description}
% \item[\CTANinstall{install/macros/latex/contrib/luacolor.tds.zip}]
% \end{description}
% \emph{TDS} refers to the standard ``A Directory Structure
% for \TeX\ Files'' (\CTANpkg{tds}). Directories
% with \xfile{texmf} in their name are usually organized this way.
%
% \subsection{Bundle installation}
%
% \paragraph{Unpacking.} Unpack the \xfile{luacolor.tds.zip} in the
% TDS tree (also known as \xfile{texmf} tree) of your choice.
% Example (linux):
% \begin{quote}
%   |unzip luacolor.tds.zip -d ~/texmf|
% \end{quote}
%
% \paragraph{Script installation.}
% Check the directory \xfile{TDS:scripts/luacolor/} for
% scripts that need further installation steps.
%
% \subsection{Package installation}
%
% \paragraph{Unpacking.} The \xfile{.dtx} file is a self-extracting
% \docstrip\ archive. The files are extracted by running the
% \xfile{.dtx} through \plainTeX:
% \begin{quote}
%   \verb|tex luacolor.dtx|
% \end{quote}
%
% \paragraph{TDS.} Now the different files must be moved into
% the different directories in your installation TDS tree
% (also known as \xfile{texmf} tree):
% \begin{quote}
% \def\t{^^A
% \begin{tabular}{@{}>{\ttfamily}l@{ $\rightarrow$ }>{\ttfamily}l@{}}
%   luacolor.sty & tex/latex/luacolor/luacolor.sty\\
%   luacolor.lua & scripts/luacolor/luacolor.lua\\
%   luacolor.pdf & doc/latex/luacolor/luacolor.pdf\\
%   luacolor.dtx & source/latex/luacolor/luacolor.dtx\\
% \end{tabular}^^A
% }^^A
% \sbox0{\t}^^A
% \ifdim\wd0>\linewidth
%   \begingroup
%     \advance\linewidth by\leftmargin
%     \advance\linewidth by\rightmargin
%   \edef\x{\endgroup
%     \def\noexpand\lw{\the\linewidth}^^A
%   }\x
%   \def\lwbox{^^A
%     \leavevmode
%     \hbox to \linewidth{^^A
%       \kern-\leftmargin\relax
%       \hss
%       \usebox0
%       \hss
%       \kern-\rightmargin\relax
%     }^^A
%   }^^A
%   \ifdim\wd0>\lw
%     \sbox0{\small\t}^^A
%     \ifdim\wd0>\linewidth
%       \ifdim\wd0>\lw
%         \sbox0{\footnotesize\t}^^A
%         \ifdim\wd0>\linewidth
%           \ifdim\wd0>\lw
%             \sbox0{\scriptsize\t}^^A
%             \ifdim\wd0>\linewidth
%               \ifdim\wd0>\lw
%                 \sbox0{\tiny\t}^^A
%                 \ifdim\wd0>\linewidth
%                   \lwbox
%                 \else
%                   \usebox0
%                 \fi
%               \else
%                 \lwbox
%               \fi
%             \else
%               \usebox0
%             \fi
%           \else
%             \lwbox
%           \fi
%         \else
%           \usebox0
%         \fi
%       \else
%         \lwbox
%       \fi
%     \else
%       \usebox0
%     \fi
%   \else
%     \lwbox
%   \fi
% \else
%   \usebox0
% \fi
% \end{quote}
% If you have a \xfile{docstrip.cfg} that configures and enables \docstrip's
% TDS installing feature, then some files can already be in the right
% place, see the documentation of \docstrip.
%
% \subsection{Refresh file name databases}
%
% If your \TeX~distribution
% (\TeX\,Live, \mikTeX, \dots) relies on file name databases, you must refresh
% these. For example, \TeX\,Live\ users run \verb|texhash| or
% \verb|mktexlsr|.
%
% \subsection{Some details for the interested}
%
% \paragraph{Unpacking with \LaTeX.}
% The \xfile{.dtx} chooses its action depending on the format:
% \begin{description}
% \item[\plainTeX:] Run \docstrip\ and extract the files.
% \item[\LaTeX:] Generate the documentation.
% \end{description}
% If you insist on using \LaTeX\ for \docstrip\ (really,
% \docstrip\ does not need \LaTeX), then inform the autodetect routine
% about your intention:
% \begin{quote}
%   \verb|latex \let\install=y\input{luacolor.dtx}|
% \end{quote}
% Do not forget to quote the argument according to the demands
% of your shell.
%
% \paragraph{Generating the documentation.}
% You can use both the \xfile{.dtx} or the \xfile{.drv} to generate
% the documentation. The process can be configured by the
% configuration file \xfile{ltxdoc.cfg}. For instance, put this
% line into this file, if you want to have A4 as paper format:
% \begin{quote}
%   \verb|\PassOptionsToClass{a4paper}{article}|
% \end{quote}
% An example follows how to generate the
% documentation with pdf\LaTeX:
% \begin{quote}
%\begin{verbatim}
%pdflatex luacolor.dtx
%makeindex -s gind.ist luacolor.idx
%pdflatex luacolor.dtx
%makeindex -s gind.ist luacolor.idx
%pdflatex luacolor.dtx
%\end{verbatim}
% \end{quote}
%
% \begin{History}
%   \begin{Version}{2007/12/12 v1.0}
%   \item
%     First public version.
%   \end{Version}
%   \begin{Version}{2009/04/10 v1.1}
%   \item
%     Fixes for changed syntax of \cs{directlua} in \LuaTeX\ 0.36.
%   \end{Version}
%   \begin{Version}{2010/03/09 v1.2}
%   \item
%     Adaptation for package \xpackage{luatex} 2010/03/09 v0.4.
%   \end{Version}
%   \begin{Version}{2010/12/13 v1.3}
%   \item
%     Support for \cs{pdfxform} added.
%   \item
%     Loaded package \xpackage{luatexbase-attr} recognized.
%   \item
%     Update for \hologo{LuaTeX}: `list' fields renamed to `head' in v0.65.0.
%   \end{Version}
%   \begin{Version}{2011/03/29 v1.4}
%   \item
%     Avoid whatsit insertion if option \xoption{monochrome} is used
%     (thanks Manuel P\'egouri\'e-Gonnard).
%   \end{Version}
%   \begin{Version}{2011/04/22 v1.5}
%   \item
%     Bug fix by Manuel P\'egouri\'e-Gonnard: A typo prevented the
%     detection of whatsits and applying color changes for
%     \cs{pdfliteral} and \cs{special} nodes that might contain
%     typesetting material.
%   \item
%     Bug fix by Manuel P\'egouri\'e-Gonnard: Now colors are also
%     applied to leader boxes.
%   \item
%     Unnecessary color settings are removed for leaders boxes,
%     if after the leader box the color has not changed.
%     The costs are a little runtime, leader boxes are processed twice.
%   \item
%     Additional whatsits that are colored: \texttt{pdf\_refximage}.
%   \item
%     Workaround for bug with \texttt{node.insert\_before}
%     removed for the version after \hologo{LuaTeX} 0.65, because
%     bug was fixed in 0.27. (Thanks Manuel P\'egouri\'e-Gonnard.)
%   \end{Version}
%   \begin{Version}{2011/04/23 v1.6}
%   \item
%     Bug fix for nested leader boxes.
%   \item
%     Bug fix for leader boxes that change color, but
%     are not set because of missing place.
%   \item
%     Version check for Lua module added.
%   \end{Version}
%   \begin{Version}{2011/10/22 v1.7}
%   \item
%     Lua functions \texttt{getattribute} and \texttt{getvalue}
%     added to tell other external Lua functions
%     the attribute register number for coloring.
%   \end{Version}
%   \begin{Version}{2011/11/01 v1.8}
%   \item
%     Use of \texttt{node.subtype} instead of magic numbers.
%   \end{Version}
%   \begin{Version}{2016/05/13 v1.9}
%   \item
%     More use of \texttt{node.subtype} instead of magic numbers.
%   \item luatex 85 updates
%   \end{Version}
%   \begin{Version}{2016/05/16 v1.10}
%   \item
%     Documentation updates.
%   \end{Version}
%   \begin{Version}{2018/11/22 v1.11}
%   \item
%     handle issue 43.
%   \item
%     removed pre-0.65 stuff
%   \end{Version}
%   \begin{Version}{2019/07/25 v1.12}
%   \item
%     removed uses of module function, see PR70
%   \end{Version}
%   \begin{Version}{2019/11/29 v1.13}
%   \item
%     Documentation updates.
%   \item  Use \xpackage{iftex} directly.
%   \end{Version}
%   \begin{Version}{2020-02-22 v1.14}
%   \item Drop use of \textsf{iftex} \textsf{ltxcmds} and \textsf{infwarerr}.
%   \item Assume \textsf{ltluatex} preloaded into format (true since 2015).
%   \item Patch \verb|\saveboxresource| rather than \verb|\pdfxform| (keep old name as alias).
%   \item Grab the number via Lua so that a \verb|\immediate| prefix still works with
%         \verb|\saveboxresource|\slash \verb|\pdfxform|.
%   \item Added handler for the color feature of luaotfload      
%   \end{Version}
%   \begin{Version}{2020-02-24 v1.15}
%   \item Grab all possible arguments for \verb|\saveboxresource|\slash \verb|\pdfxform|
%   \end{Version}
%   \begin{Version}{2020-04-04 v1.16}
%   \item Reset color after \verb|pdf_colorstack| whatsits.
%   \end{Version}
%   \begin{Version}{2021-02-17 v1.17}
%   \item Use \LaTeXe's new \verb|pre_shipout_filter| callback if it's available to
%         allow coloring background and foreground layer material
%   \end{Version}
%   \begin{Version}{2023-08-18 v1.18}
%   \item added \verb|\reset@color| to \verb|\set@color| for \verb|\mathcolor|, issue 7.
%   \end{Version}
% \end{History}
%
% \PrintIndex
%
% \Finale
\endinput

%        (quote the arguments according to the demands of your shell)
%
% Documentation:
%    (a) If luacolor.drv is present:
%           latex luacolor.drv
%    (b) Without luacolor.drv:
%           latex luacolor.dtx; ...
%    The class ltxdoc loads the configuration file ltxdoc.cfg
%    if available. Here you can specify further options, e.g.
%    use A4 as paper format:
%       \PassOptionsToClass{a4paper}{article}
%
%    Programm calls to get the documentation (example):
%       pdflatex luacolor.dtx
%       makeindex -s gind.ist luacolor.idx
%       pdflatex luacolor.dtx
%       makeindex -s gind.ist luacolor.idx
%       pdflatex luacolor.dtx
%
% Installation:
%    TDS:tex/latex/luacolor/luacolor.sty
%    TDS:scripts/luacolor/luacolor.lua
%    TDS:doc/latex/luacolor/luacolor.pdf
%    TDS:source/latex/luacolor/luacolor.dtx
%
%<*ignore>
\begingroup
  \catcode123=1 %
  \catcode125=2 %
  \def\x{LaTeX2e}%
\expandafter\endgroup
\ifcase 0\ifx\install y1\fi\expandafter
         \ifx\csname processbatchFile\endcsname\relax\else1\fi
         \ifx\fmtname\x\else 1\fi\relax
\else\csname fi\endcsname
%</ignore>
%<*install>
\input docstrip.tex
\Msg{************************************************************************}
\Msg{* Installation}
\Msg{* Package: luacolor 2023-08-18 v1.18 Color support via LuaTeX's attributes (HO)}
\Msg{************************************************************************}

\keepsilent
\askforoverwritefalse

\let\MetaPrefix\relax
\preamble

This is a generated file.

Project: luacolor
Version: 2023-08-18 v1.18

Copyright (C)
   2007, 2009-2011 Heiko Oberdiek
   2016-2023 Oberdiek Package Support Group

This work may be distributed and/or modified under the
conditions of the LaTeX Project Public License, either
version 1.3c of this license or (at your option) any later
version. This version of this license is in
   https://www.latex-project.org/lppl/lppl-1-3c.txt
and the latest version of this license is in
   https://www.latex-project.org/lppl.txt
and version 1.3 or later is part of all distributions of
LaTeX version 2005/12/01 or later.

This work has the LPPL maintenance status "maintained".

The Current Maintainers of this work are
Heiko Oberdiek and the Oberdiek Package Support Group
https://github.com/ho-tex/luacolor/issues


This work consists of the main source file luacolor.dtx
and the derived files
   luacolor.sty, luacolor.pdf, luacolor.ins, luacolor.drv,
   luacolor.lua,

\endpreamble
\let\MetaPrefix\DoubleperCent

\generate{%
  \file{luacolor.ins}{\from{luacolor.dtx}{install}}%
  \file{luacolor.drv}{\from{luacolor.dtx}{driver}}%
  \usedir{tex/latex/luacolor}%
  \file{luacolor.sty}{\from{luacolor.dtx}{package}}%
}
\def\MetaPrefix{-- }
\def\defaultpostamble{%
  \MetaPrefix^^J%
  \MetaPrefix\space End of File `\outFileName'.%
}
\def\currentpostamble{\defaultpostamble}%
\generate{%
  \usedir{scripts/luacolor}%
%  \file{oberdiek.luacolor.lua}{\from{luacolor.dtx}{lua}}%
  \file{luacolor.lua}{\from{luacolor.dtx}{lua}}%
}

\catcode32=13\relax% active space
\let =\space%
\Msg{************************************************************************}
\Msg{*}
\Msg{* To finish the installation you have to move the following}
\Msg{* file into a directory searched by TeX:}
\Msg{*}
\Msg{*     luacolor.sty}
\Msg{*}
\Msg{* And install the following script file:}
\Msg{*}
\Msg{*     luacolor.lua,}
\Msg{*}
\Msg{* To produce the documentation run the file `luacolor.dtx'}
\Msg{* through LaTeX.}
\Msg{*}
\Msg{* Happy TeXing!}
\Msg{*}
\Msg{************************************************************************}

\endbatchfile
%</install>
%<*ignore>
\fi
%</ignore>
%<*driver>
\NeedsTeXFormat{LaTeX2e}
\ProvidesFile{luacolor.drv}%
  [2023-08-18 v1.18 Color support via LuaTeX's attributes (HO)]%
\documentclass{ltxdoc}
\usepackage{holtxdoc}[2011/11/22]
\usepackage{paralist}
\makeatletter
\g@addto@macro\MakePrivateLetters{%
  \@makeother\_%
}
\makeatother
\begin{document}
  \DocInput{luacolor.dtx}%
\end{document}
%</driver>
% \fi
%
%
%
% \GetFileInfo{luacolor.drv}
%
% \title{The \xpackage{luacolor} package}
% \date{2023-08-18 v1.18}
% \author{Heiko Oberdiek\thanks
% {Please report any issues at \url{https://github.com/ho-tex/luacolor/issues}}}
%
% \maketitle
%
% \begin{abstract}
% Package \xpackage{luacolor} implements color support based
% on \LuaTeX's node attributes.
% \end{abstract}
%
% \tableofcontents
%
% \section{Documentation}
%
% \subsection{Introduction}
%
% This package uses a \LuaTeX's attribute register to
% to annotate nodes with color information.
% If a color is set, then the attribute register
% is set to this color and all nodes created in its scope
% (current group) are annotated with this attribute.
% Now the color property behaves much the same way
% as the font property.
%
% \subsection{Usage}
%
% Package \xpackage{color} is loaded automatically by this
% package \xpackage{luacolor}. If you need a special driver
% option or you prefer package \xpackage{xcolor}, then load
% it before package \xpackage{luacolor}, for example:
% \begin{quote}
%   |\usepackage[dvipdfmx]{xcolor}|
% \end{quote}
% The package \xpackage{luacolor} is loaded without options:
% \begin{quote}
%   |\usepackage{luacolor}|
% \end{quote}
% It is able to detect PDF mode and DVI drivers are
% differentiated by its color specials. Therefore the
% package do need driver options.
%
% Then it redefines the color setting commands to set
% attributes instead of whatsits for color.
%
% At last the attribute annotations of the nodes in the
% output box must be analyzed to insert the necessary color whatsits.
% (With older \LuaTeX\ that lack the appropriate callback function 
% the package \xpackage{atbegshi} is used to get control
% before a box is shipped out.)
%
% \begin{declcs}{luacolorProcessBox} \M{box}
% \end{declcs}
% Macro \cs{luacolorProcessBox} processes the box \meta{box}
% in the previously described manner. It is automatically
% called for pages, but not for XForm objects.
% Before passing a box to \cs{pdfxform}, call \cs{luacolorProcessBox}
% first.
%
% \subsection{Limitations}
%
% \begin{description}
% \item[Ligatures with different colored components:]
%   Package \xpackage{luacolor} sees the ligature after
%   the paragraph building and page breaking, when a
%   page is to be shipped out. Therefore it cannot
%   break ligatures, because the components might occupy
%   different space. Therefore it is the responsibility
%   of the ligature forming process to deal with different
%   colored glyphs that form a ligature. The user can avoid
%   the problem entirely by explicitly breaking the ligature
%   at the places where the color changes.
%  \item \dots
% \end{description}
%
% \StopEventually{
% }
%
% \section{Implementation}
%
%    \begin{macrocode}
%<*package>
%    \end{macrocode}
%
% \subsection{Catcodes and identification}
%
%    \begin{macrocode}
\begingroup\catcode61\catcode48\catcode32=10\relax%
  \catcode13=5 % ^^M
  \endlinechar=13 %
  \catcode123=1 % {
  \catcode125=2 % }
  \catcode64=11 % @
  \def\x{\endgroup
    \expandafter\edef\csname LuaCol@AtEnd\endcsname{%
      \endlinechar=\the\endlinechar\relax
      \catcode13=\the\catcode13\relax
      \catcode32=\the\catcode32\relax
      \catcode35=\the\catcode35\relax
      \catcode61=\the\catcode61\relax
      \catcode64=\the\catcode64\relax
      \catcode123=\the\catcode123\relax
      \catcode125=\the\catcode125\relax
    }%
  }%
\x\catcode61\catcode48\catcode32=10\relax%
\catcode13=5 % ^^M
\endlinechar=13 %
\catcode35=6 % #
\catcode64=11 % @
\catcode123=1 % {
\catcode125=2 % }
\def\TMP@EnsureCode#1#2{%
  \edef\LuaCol@AtEnd{%
    \LuaCol@AtEnd
    \catcode#1=\the\catcode#1\relax
  }%
  \catcode#1=#2\relax
}
\TMP@EnsureCode{34}{12}% "
\TMP@EnsureCode{39}{12}% '
\TMP@EnsureCode{40}{12}% (
\TMP@EnsureCode{41}{12}% )
\TMP@EnsureCode{42}{12}% *
\TMP@EnsureCode{43}{12}% +
\TMP@EnsureCode{44}{12}% ,
\TMP@EnsureCode{45}{12}% -
\TMP@EnsureCode{46}{12}% .
\TMP@EnsureCode{47}{12}% /
\TMP@EnsureCode{58}{12}% :
\TMP@EnsureCode{60}{12}% <
\TMP@EnsureCode{62}{12}% >
\TMP@EnsureCode{91}{12}% [
\TMP@EnsureCode{93}{12}% ]
\TMP@EnsureCode{95}{12}% _ (other!)
\TMP@EnsureCode{96}{12}% `
\edef\LuaCol@AtEnd{\LuaCol@AtEnd\noexpand\endinput}
%    \end{macrocode}
%
%    Package identification.
%    \begin{macrocode}
\NeedsTeXFormat{LaTeX2e}
\ProvidesPackage{luacolor}%
  [2023-08-18 v1.18 Color support via LuaTeX's attributes (HO)]
%    \end{macrocode}
%
% \subsection{Check for \LuaTeX}
%
%    Without \LuaTeX\ there is no point in using this package.
%    \begin{macrocode}
\RequirePackage{color}
%    \end{macrocode}
%
%    \begin{macrocode}
\ifx\directlua\@undefined
  \PackageError{luacolor}{%
    This package may only be run using LuaTeX%
  }\@ehc
  \expandafter\LuaCol@AtEnd
\fi%
%    \end{macrocode}
%
% \subsection{Check for disabled colors}
%
%    \begin{macrocode}
\ifcolors@
\else
  \PackageWarningNoLine{luacolor}{%
    Colors are disabled by option `monochrome'%
  }%
  \def\set@color{}%
  \def\reset@color{}%
  \def\set@page@color{}%
  \def\define@color#1#2{}%
  \expandafter\LuaCol@AtEnd
\fi%
%    \end{macrocode}
%
% \subsection{Load module and check version}
%
%    \begin{macrocode}
\directlua{%
  require("luacolor")%
}
%    \end{macrocode}
%    \begin{macrocode}
\begingroup
  \edef\x{\directlua{tex.write("2023-08-18 v1.18")}}%
  \edef\y{%
    \directlua{%
      if oberdiek.luacolor.getversion then %
        oberdiek.luacolor.getversion()%
      end%
    }%
  }%
  \ifx\x\y
  \else
    \PackageError{luacolor}{%
      Wrong version of lua module.\MessageBreak
      Package version: \x\MessageBreak
      Lua module: \y
    }\@ehc
  \fi
\endgroup
%    \end{macrocode}
%
% \subsection{Find driver}
%
%    \begin{macrocode}
\ifnum\outputmode=\@ne
\else
  \begingroup
    \def\current@color{}%
    \def\reset@color{}%
    \setbox\z@=\hbox{%
      \begingroup
        \set@color
      \endgroup
    }%
    \edef\reserved@a{%
      \directlua{%
        oberdiek.luacolor.dvidetect()%
      }%
    }%
    \ifx\reserved@a\@empty
      \PackageError{luacolor}{%
        DVI driver detection failed because of\MessageBreak
        unrecognized color \string\special
      }\@ehc
      \endgroup
      \expandafter\expandafter\expandafter\LuaCol@AtEnd
    \else
      \PackageInfo{luacolor}{%
        Type of color \string\special: \reserved@a
      \@gobble}%
    \fi%
  \endgroup
\fi
%    \end{macrocode}
%
% \subsection{Attribute setting}
%
%    \begin{macro}{\LuaCol@Attribute}
%    \begin{macrocode}
\newattribute\LuaCol@Attribute
\let\LuaCol@setattribute\setattribute
\directlua{%
  oberdiek.luacolor.setattribute(\number\allocationnumber)%
}
%    \end{macrocode}
%    \end{macro}
%
%    \begin{macro}{\set@color}
% change 2023-08-18: added \cs{reset@color} so that \cs{mathcolor} can gobble it,
% issue 7.   
%    \begin{macrocode}
\protected\def\set@color{%
  \LuaCol@setattribute\LuaCol@Attribute{%
    \directlua{%
      oberdiek.luacolor.get("\luaescapestring{\current@color}")%
    }%
  }%
 \aftergroup\reset@color 
}
%    \end{macrocode}
%    \end{macro}
%    \begin{macro}{\reset@color}
%    \begin{macrocode}
\def\reset@color{}
%    \end{macrocode}
%    \end{macro}
%
% \subsection{Whatsit insertion}
%
%    \begin{macro}{\luacolorProcessBox}
%    \begin{macrocode}
\def\luacolorProcessBox#1{%
  \directlua{%
    oberdiek.luacolor.process(\number#1)%
  }%
}
%    \end{macrocode}
%    \end{macro}
%
%    \begin{macrocode}
\directlua{%
  if luatexbase.callbacktypes.pre_shipout_filter then
    token.get_next()
  end
}\@secondoftwo\@gobble{
  \RequirePackage{atbegshi}[2011/01/30]
  \AtBeginShipout{%
    \luacolorProcessBox\AtBeginShipoutBox
  }
}
%    \end{macrocode}
%
%    Set default color.
%    \begin{macrocode}
\set@color
%    \end{macrocode}
%
% \subsection{\cs{pdfxform}/\cs{saveboxresource} support}
%
%    \begin{macrocode}
\ifnum\outputmode=\@ne
    \let\LuaCol@org@pdfxform\saveboxresource
%    \end{macrocode}
% First we need some helpers to allow expandable code to parse keyword style arguments:
%    \begin{macrocode}
    \def\LuaCol@iii@i@ii#1#2#3{#3{#1}{#2}}
    \def\LuaCol@ii@i#1#2{{#2#1}}
    \def\LuaCol@if@keyword#1#2#3{%
      \expanded{\unexpanded{\LuaCol@iii@i@ii{#2}{#3}}\expandafter}%
      \directlua{%
        token.put_next(token.create(token.scan_keyword(token.scan_string())
        and '@firstoftwo'
        or '@secondoftwo'))
      }{#1}%
    }
%    \end{macrocode}
% The following macro scans a integer and expands to a token equivalent to a chardef
% whose value corresponds to the scanned integer. This allows the integer to be passed
% around as a undelimited argument.
%    \begin{macrocode}
    \def\LuaCol@scan@number{%
      \directlua{
        token.put_next(token.new(token.scan_int(), token.command_id'char_given'))
      }%
    }
%    \end{macrocode}
% \TeX\ primitives like \cs{saveboxresource} read braced arguments in a special way.
% Especially they expand everything until they find a left brace. To simulate this, we
% use Lua to expand everything else:
%    \begin{macrocode}
    \def\LuaCol@scan@tobrace{%
      \directlua{
        local relax, space = token.command_id'relax', token.command_id'spacer'
        local t
        repeat
          t = token.scan_token()
        until not (t.command == relax or t.command == space)
        token.put_next(t)
      }%
    }
    \def\LuaCol@scan@boxresource@i#1#2{%
      \LuaCol@if@keyword{attr}{%
        \expanded{\unexpanded{\LuaCol@scan@boxresource@iI{#1#2attr}}%
          \expandafter\expandafter\expandafter}%
        \LuaCol@scan@tobrace
      }{%
        \LuaCol@scan@boxresource@ii{#1#2}%
      }%
    }
    \def\LuaCol@scan@boxresource@iI#1#2{\LuaCol@scan@boxresource@ii{#1{#2}}}
    \def\LuaCol@scan@boxresource@ii#1{%
      \LuaCol@if@keyword{resources}{%
        \expanded{\unexpanded{\LuaCol@scan@boxresource@iiI{#1resources}}%
          \expandafter\expandafter\expandafter}%
        \LuaCol@scan@tobrace
      }{%
        \LuaCol@scan@boxresource@iii{#1}%
      }%
    }
    \def\LuaCol@scan@boxresource@iiI#1#2{\LuaCol@scan@boxresource@iii{#1{#2}}}
    \def\LuaCol@scan@boxresource@iii#1{%
      \LuaCol@if@keyword{margin}{%
        \expanded{\unexpanded{\LuaCol@scan@boxresource@iv{#1margin }}%
          \expandafter\expandafter\expandafter}%
        \LuaCol@scan@number
      }{%
        \LuaCol@scan@boxresource@iv{#1}{}%
      }%
    }
    \def\LuaCol@scan@boxresource@iv#1#2{%
      \expanded{\unexpanded{\LuaCol@scan@boxresource@v{#1#2}}%
        \expandafter\expandafter\expandafter}%
      \LuaCol@scan@number
    }
    \def\LuaCol@scan@boxresource@v#1#2{%
      \luacolorProcessBox{#2}%
      \LuaCol@org@pdfxform#1#2%
    }

%    \end{macrocode}
% This could be written in Lua, but at least upto Lua\TeX~1.11, feeding back too many tokens
% from Lua to \TeX\ triggers a segmentation fault.
% This is written in Lua so the integer setting is expandable and does not interfere with
% a preceding \verb|\immediate|.
%    \begin{macrocode}
    \protected\def\saveboxresource{%
      \LuaCol@if@keyword{type}{%
        \expandafter
        \expanded{\unexpanded{\LuaCol@scan@boxresource@i{type }}%
          \expandafter\expandafter\expandafter}%
        \LuaCol@scan@number
      }{%
        \LuaCol@scan@boxresource@i{}{}%
      }%
    }
%    \end{macrocode}
%    Legacy alias.
%    \begin{macrocode}
    \let\pdfxform\saveboxresource
\fi
%    \end{macrocode}
%
%    \begin{macrocode}
\LuaCol@AtEnd%
%</package>
%    \end{macrocode}
%
% \subsection{Lua module}
%
%    \begin{macrocode}
%<*lua>
%    \end{macrocode}
%    Box zero contains a \cs{hbox} with the color \cs{special}.
%    That is analyzed to get the prefix for the color setting
%    \cs{special}.
%    \begin{macrocode}
oberdiek = oberdiek or {}
local luacolor = oberdiek.luacolor or {}
oberdiek.luacolor = luacolor
%    \end{macrocode}
%    \begin{macro}{getversion()}
%    \begin{macrocode}
function luacolor.getversion()
  tex.write("2023-08-18 v1.18")
end
%    \end{macrocode}
%    \end{macro}
%
% \subsubsection{Driver detection}
%
%    \begin{macrocode}
local ifpdf = tonumber(tex.outputmode or tex.pdfoutput) > 0
local prefix
local prefixes = {
  dvips   = "color ",
  dvipdfm = "pdf:sc ",
  truetex = "textcolor:",
  pctexps = "ps::",
}
local patterns = {
  ["^color "]            = "dvips",
  ["^pdf: *begincolor "] = "dvipdfm",
  ["^pdf: *bcolor "]     = "dvipdfm",
  ["^pdf: *bc "]         = "dvipdfm",
  ["^pdf: *setcolor "]   = "dvipdfm",
  ["^pdf: *scolor "]     = "dvipdfm",
  ["^pdf: *sc "]         = "dvipdfm",
  ["^textcolor:"]        = "truetex",
  ["^ps::"]              = "pctexps",
}
%    \end{macrocode}
%    \begin{macro}{info()}
%    \begin{macrocode}
local function info(msg, term)
  local target = "log"
  if term then
    target = "term and log"
  end
  texio.write_nl(target, "Package luacolor info: " .. msg .. ".")
  texio.write_nl(target, "")
end
%    \end{macrocode}
%    \end{macro}
%    \begin{macro}{dvidetect()}
%    \begin{macrocode}
function luacolor.dvidetect()
  local v = tex.box[0]
  assert(v.id == node.id("hlist"))
  for v in node.traverse_id(node.id("whatsit"), v.head) do
    if v and v.subtype == node.subtype("special") then
      local data = v.data
      for pattern, driver in pairs(patterns) do
        if string.find(data, pattern) then
          prefix = prefixes[driver]
          tex.write(driver)
          return
        end
      end
      info("\\special{" .. data .. "}", true)
      return
    end
  end
  info("Missing \\special", true)
end
%    \end{macrocode}
%    \end{macro}
%
% \subsubsection{Color strings}
%
%    \begin{macrocode}
local map = {
  n = 0,
}
%    \end{macrocode}
%    \begin{macro}{get()}
%    \begin{macrocode}
function luacolor.get(color)
  tex.write("" .. luacolor.getvalue(color))
end
%    \end{macrocode}
%    \end{macro}
%    \begin{macro}{getvalue()}
%    \begin{macrocode}
function luacolor.getvalue(color)
  local n = map[color]
  if not n then
    n = map.n + 1
    map.n = n
    map[n] = color
    map[color] = n
  end
  return n
end
%    \end{macrocode}
%    \end{macro}
%
% \subsubsection{Attribute register}
%
%    \begin{macro}{setattribute()}
%    \begin{macrocode}
local attribute
function luacolor.setattribute(attr)
  attribute = attr
end
%    \end{macrocode}
%    \end{macro}
%
%    \begin{macro}{getattribute()}
%    \begin{macrocode}
function luacolor.getattribute()
  return attribute
end
%    \end{macrocode}
%    \end{macro}
%
% \subsubsection{Whatsit insertion}
%
%    \begin{macrocode}
local LIST = 1
local LIST_LEADERS = 2
local LIST_DISC = 3
local COLOR = 4
local NOCOLOR = 5
local RULE = node.id("rule")
local node_types = {
  [node.id("hlist")] = LIST,
  [node.id("vlist")] = LIST,
  [node.id("rule")]  = COLOR,
  [node.id("glyph")] = COLOR,
  [node.id("disc")]  = LIST_DISC,
  [node.id("whatsit")] = {
    [node.subtype("pdf_colorstack")] =
      function(n)
        return n.stack == 0 and NOCOLOR or nil
      end,
    [node.subtype("special")] = COLOR,
    [node.subtype("pdf_literal")] = COLOR,
    [node.subtype("pdf_save")] = COLOR,
    [node.subtype("pdf_restore")] = COLOR, -- probably not needed
-- TODO (DPC)    [node.subtype("pdf_refximage")] = COLOR,
  },
  [node.id("glue")] =
    function(n)
      if n.subtype >= 100 then -- leaders
        if n.leader.id == RULE then
          return COLOR
        else
          return LIST_LEADERS
        end
      end
    end,
}
%    \end{macrocode}
%    \begin{macro}{get_type()}
%    \begin{macrocode}
local function get_type(n)
  local ret = node_types[n.id]
  if type(ret) == 'table' then
    ret = ret[n.subtype]
  end
  if type(ret) == 'function' then
    ret = ret(n)
  end
  return ret
end
%    \end{macrocode}
%    \end{macro}
%    \begin{macrocode}
local mode = 2 -- luatex.pdfliteral.direct
local WHATSIT = node.id("whatsit")
local SPECIAL = node.subtype("special")
local PDFLITERAL = node.subtype("pdf_literal")
local DRY_FALSE = false
local DRY_TRUE = true
%    \end{macrocode}
%    \begin{macro}{traverse()}
%    \begin{macrocode}
local function traverse(list, color, dry)
  if not list then
    return color
  end
  local head
  if get_type(list) == LIST then
    head = list.head
  elseif get_type(list) == LIST_DISC then
    head = list.replace
  else
    texio.write_nl("!!! Error: Wrong list type: " .. node.type(list.id))
    return color
  end
%<debug>texio.write_nl("traverse: " .. node.type(list.id))
  for n in node.traverse(head) do
%<debug>texio.write_nl("  node: " .. node.type(n.id))
    local t = get_type(n)
%<debug>texio.write_nl("TYPE "..tostring(t).. " "..tostring(node.type(node.getid(n))).." ".. tostring(node.getsubtype(n)))
    if t == LIST or t == LIST_DISC then
      color = traverse(n, color, dry)
    elseif t == LIST_LEADERS then
      local color_after = traverse(n.leader, color, DRY_TRUE)
      if color == color_after then
        traverse(n.leader, color, DRY_FALSE or dry)
      else
        traverse(n.leader, '', DRY_FALSE or dry)
%    \end{macrocode}
% The color status is unknown here, because the leader box
% will or will not be set.
%    \begin{macrocode}
        color = ''
      end
    elseif t == COLOR then
      local v = node.has_attribute(n, attribute)
      if v then
        local newColor = map[v]
        if newColor ~= color then
          color = newColor
          if dry == DRY_FALSE then
            local newNode
            if ifpdf then
              newNode = node.new(WHATSIT, PDFLITERAL)
              newNode.mode = mode
              newNode.data = color
            else
              newNode = node.new(WHATSIT, SPECIAL)
              newNode.data = prefix .. color
            end
            head = node.insert_before(head, n, newNode)
          end
        end
      end
    elseif t == NOCOLOR then
      color = ''
    end
  end
  if get_type(list) == LIST then
    list.head = head
  else
    list.replace = head
  end
  return color
end
%    \end{macrocode}
%    \end{macro}
%
%    \begin{macro}{process()}
%    \begin{macrocode}
function luacolor.process(box)
  local color = ""
  local list = tex.getbox(box)
  traverse(list, color, DRY_FALSE)
end

if luatexbase.callbacktypes.pre_shipout_filter then
  luatexbase.add_to_callback("pre_shipout_filter", function(list)
    traverse(list, "", DRY_FALSE)
    return true
  end, "luacolor.process")
end
%    \end{macrocode}
%    \end{macro}
% For recent versions of luaotfload, we can register a callback to
% control how coloring glyph is handled for the color feature.
%    \begin{macrocode}
if luaotfload.set_colorhandler then
  local set_attribute = node.direct.set_attribute
  luaotfload.set_colorhandler(function(head, n, color)
    set_attribute(n, attribute, luacolor.getvalue(color))
    return head, n
  end)
end
%    \end{macrocode}
%
%    \begin{macrocode}
%</lua>
%    \end{macrocode}
% \section{Installation}
%
% \subsection{Download}
%
% \paragraph{Package.} This package is available on
% CTAN\footnote{\CTANpkg{luacolor}}:
% \begin{description}
% \item[\CTAN{macros/latex/contrib/luacolor/luacolor.dtx}] The source file.
% \item[\CTAN{macros/latex/contrib/luacolor/luacolor.pdf}] Documentation.
% \end{description}
%
%
% \paragraph{Bundle.} All the packages of the bundle `luacolor'
% are also available in a TDS compliant ZIP archive. There
% the packages are already unpacked and the documentation files
% are generated. The files and directories obey the TDS standard.
% \begin{description}
% \item[\CTANinstall{install/macros/latex/contrib/luacolor.tds.zip}]
% \end{description}
% \emph{TDS} refers to the standard ``A Directory Structure
% for \TeX\ Files'' (\CTANpkg{tds}). Directories
% with \xfile{texmf} in their name are usually organized this way.
%
% \subsection{Bundle installation}
%
% \paragraph{Unpacking.} Unpack the \xfile{luacolor.tds.zip} in the
% TDS tree (also known as \xfile{texmf} tree) of your choice.
% Example (linux):
% \begin{quote}
%   |unzip luacolor.tds.zip -d ~/texmf|
% \end{quote}
%
% \paragraph{Script installation.}
% Check the directory \xfile{TDS:scripts/luacolor/} for
% scripts that need further installation steps.
%
% \subsection{Package installation}
%
% \paragraph{Unpacking.} The \xfile{.dtx} file is a self-extracting
% \docstrip\ archive. The files are extracted by running the
% \xfile{.dtx} through \plainTeX:
% \begin{quote}
%   \verb|tex luacolor.dtx|
% \end{quote}
%
% \paragraph{TDS.} Now the different files must be moved into
% the different directories in your installation TDS tree
% (also known as \xfile{texmf} tree):
% \begin{quote}
% \def\t{^^A
% \begin{tabular}{@{}>{\ttfamily}l@{ $\rightarrow$ }>{\ttfamily}l@{}}
%   luacolor.sty & tex/latex/luacolor/luacolor.sty\\
%   luacolor.lua & scripts/luacolor/luacolor.lua\\
%   luacolor.pdf & doc/latex/luacolor/luacolor.pdf\\
%   luacolor.dtx & source/latex/luacolor/luacolor.dtx\\
% \end{tabular}^^A
% }^^A
% \sbox0{\t}^^A
% \ifdim\wd0>\linewidth
%   \begingroup
%     \advance\linewidth by\leftmargin
%     \advance\linewidth by\rightmargin
%   \edef\x{\endgroup
%     \def\noexpand\lw{\the\linewidth}^^A
%   }\x
%   \def\lwbox{^^A
%     \leavevmode
%     \hbox to \linewidth{^^A
%       \kern-\leftmargin\relax
%       \hss
%       \usebox0
%       \hss
%       \kern-\rightmargin\relax
%     }^^A
%   }^^A
%   \ifdim\wd0>\lw
%     \sbox0{\small\t}^^A
%     \ifdim\wd0>\linewidth
%       \ifdim\wd0>\lw
%         \sbox0{\footnotesize\t}^^A
%         \ifdim\wd0>\linewidth
%           \ifdim\wd0>\lw
%             \sbox0{\scriptsize\t}^^A
%             \ifdim\wd0>\linewidth
%               \ifdim\wd0>\lw
%                 \sbox0{\tiny\t}^^A
%                 \ifdim\wd0>\linewidth
%                   \lwbox
%                 \else
%                   \usebox0
%                 \fi
%               \else
%                 \lwbox
%               \fi
%             \else
%               \usebox0
%             \fi
%           \else
%             \lwbox
%           \fi
%         \else
%           \usebox0
%         \fi
%       \else
%         \lwbox
%       \fi
%     \else
%       \usebox0
%     \fi
%   \else
%     \lwbox
%   \fi
% \else
%   \usebox0
% \fi
% \end{quote}
% If you have a \xfile{docstrip.cfg} that configures and enables \docstrip's
% TDS installing feature, then some files can already be in the right
% place, see the documentation of \docstrip.
%
% \subsection{Refresh file name databases}
%
% If your \TeX~distribution
% (\TeX\,Live, \mikTeX, \dots) relies on file name databases, you must refresh
% these. For example, \TeX\,Live\ users run \verb|texhash| or
% \verb|mktexlsr|.
%
% \subsection{Some details for the interested}
%
% \paragraph{Unpacking with \LaTeX.}
% The \xfile{.dtx} chooses its action depending on the format:
% \begin{description}
% \item[\plainTeX:] Run \docstrip\ and extract the files.
% \item[\LaTeX:] Generate the documentation.
% \end{description}
% If you insist on using \LaTeX\ for \docstrip\ (really,
% \docstrip\ does not need \LaTeX), then inform the autodetect routine
% about your intention:
% \begin{quote}
%   \verb|latex \let\install=y% \iffalse meta-comment
%
% File: luacolor.dtx
% Version: 2023-08-18 v1.18
% Info: Color support via LuaTeX's attributes
%
% Copyright (C)
%    2007, 2009-2011 Heiko Oberdiek
%    2016-2023 Oberdiek Package Support Group
%    https://github.com/ho-tex/luacolor/issues
%
% This work may be distributed and/or modified under the
% conditions of the LaTeX Project Public License, either
% version 1.3c of this license or (at your option) any later
% version. This version of this license is in
%    https://www.latex-project.org/lppl/lppl-1-3c.txt
% and the latest version of this license is in
%    https://www.latex-project.org/lppl.txt
% and version 1.3 or later is part of all distributions of
% LaTeX version 2005/12/01 or later.
%
% This work has the LPPL maintenance status "maintained".
%
% The Current Maintainers of this work are
% Heiko Oberdiek and the Oberdiek Package Support Group
% https://github.com/ho-tex/luacolor/issues
%
% This work consists of the main source file luacolor.dtx
% and the derived files
%    luacolor.sty, luacolor.pdf, luacolor.lua,
%
% Distribution:
%    CTAN:macros/latex/contrib/luacolor/luacolor.dtx
%    CTAN:macros/latex/contrib/luacolor/luacolor.pdf
%
% Unpacking:
%    (a) If luacolor.ins is present:
%           tex luacolor.ins
%    (b) Without luacolor.ins:
%           tex luacolor.dtx
%    (c) If you insist on using LaTeX
%           latex \let\install=y\input{luacolor.dtx}
%        (quote the arguments according to the demands of your shell)
%
% Documentation:
%    (a) If luacolor.drv is present:
%           latex luacolor.drv
%    (b) Without luacolor.drv:
%           latex luacolor.dtx; ...
%    The class ltxdoc loads the configuration file ltxdoc.cfg
%    if available. Here you can specify further options, e.g.
%    use A4 as paper format:
%       \PassOptionsToClass{a4paper}{article}
%
%    Programm calls to get the documentation (example):
%       pdflatex luacolor.dtx
%       makeindex -s gind.ist luacolor.idx
%       pdflatex luacolor.dtx
%       makeindex -s gind.ist luacolor.idx
%       pdflatex luacolor.dtx
%
% Installation:
%    TDS:tex/latex/luacolor/luacolor.sty
%    TDS:scripts/luacolor/luacolor.lua
%    TDS:doc/latex/luacolor/luacolor.pdf
%    TDS:source/latex/luacolor/luacolor.dtx
%
%<*ignore>
\begingroup
  \catcode123=1 %
  \catcode125=2 %
  \def\x{LaTeX2e}%
\expandafter\endgroup
\ifcase 0\ifx\install y1\fi\expandafter
         \ifx\csname processbatchFile\endcsname\relax\else1\fi
         \ifx\fmtname\x\else 1\fi\relax
\else\csname fi\endcsname
%</ignore>
%<*install>
\input docstrip.tex
\Msg{************************************************************************}
\Msg{* Installation}
\Msg{* Package: luacolor 2023-08-18 v1.18 Color support via LuaTeX's attributes (HO)}
\Msg{************************************************************************}

\keepsilent
\askforoverwritefalse

\let\MetaPrefix\relax
\preamble

This is a generated file.

Project: luacolor
Version: 2023-08-18 v1.18

Copyright (C)
   2007, 2009-2011 Heiko Oberdiek
   2016-2023 Oberdiek Package Support Group

This work may be distributed and/or modified under the
conditions of the LaTeX Project Public License, either
version 1.3c of this license or (at your option) any later
version. This version of this license is in
   https://www.latex-project.org/lppl/lppl-1-3c.txt
and the latest version of this license is in
   https://www.latex-project.org/lppl.txt
and version 1.3 or later is part of all distributions of
LaTeX version 2005/12/01 or later.

This work has the LPPL maintenance status "maintained".

The Current Maintainers of this work are
Heiko Oberdiek and the Oberdiek Package Support Group
https://github.com/ho-tex/luacolor/issues


This work consists of the main source file luacolor.dtx
and the derived files
   luacolor.sty, luacolor.pdf, luacolor.ins, luacolor.drv,
   luacolor.lua,

\endpreamble
\let\MetaPrefix\DoubleperCent

\generate{%
  \file{luacolor.ins}{\from{luacolor.dtx}{install}}%
  \file{luacolor.drv}{\from{luacolor.dtx}{driver}}%
  \usedir{tex/latex/luacolor}%
  \file{luacolor.sty}{\from{luacolor.dtx}{package}}%
}
\def\MetaPrefix{-- }
\def\defaultpostamble{%
  \MetaPrefix^^J%
  \MetaPrefix\space End of File `\outFileName'.%
}
\def\currentpostamble{\defaultpostamble}%
\generate{%
  \usedir{scripts/luacolor}%
%  \file{oberdiek.luacolor.lua}{\from{luacolor.dtx}{lua}}%
  \file{luacolor.lua}{\from{luacolor.dtx}{lua}}%
}

\catcode32=13\relax% active space
\let =\space%
\Msg{************************************************************************}
\Msg{*}
\Msg{* To finish the installation you have to move the following}
\Msg{* file into a directory searched by TeX:}
\Msg{*}
\Msg{*     luacolor.sty}
\Msg{*}
\Msg{* And install the following script file:}
\Msg{*}
\Msg{*     luacolor.lua,}
\Msg{*}
\Msg{* To produce the documentation run the file `luacolor.dtx'}
\Msg{* through LaTeX.}
\Msg{*}
\Msg{* Happy TeXing!}
\Msg{*}
\Msg{************************************************************************}

\endbatchfile
%</install>
%<*ignore>
\fi
%</ignore>
%<*driver>
\NeedsTeXFormat{LaTeX2e}
\ProvidesFile{luacolor.drv}%
  [2023-08-18 v1.18 Color support via LuaTeX's attributes (HO)]%
\documentclass{ltxdoc}
\usepackage{holtxdoc}[2011/11/22]
\usepackage{paralist}
\makeatletter
\g@addto@macro\MakePrivateLetters{%
  \@makeother\_%
}
\makeatother
\begin{document}
  \DocInput{luacolor.dtx}%
\end{document}
%</driver>
% \fi
%
%
%
% \GetFileInfo{luacolor.drv}
%
% \title{The \xpackage{luacolor} package}
% \date{2023-08-18 v1.18}
% \author{Heiko Oberdiek\thanks
% {Please report any issues at \url{https://github.com/ho-tex/luacolor/issues}}}
%
% \maketitle
%
% \begin{abstract}
% Package \xpackage{luacolor} implements color support based
% on \LuaTeX's node attributes.
% \end{abstract}
%
% \tableofcontents
%
% \section{Documentation}
%
% \subsection{Introduction}
%
% This package uses a \LuaTeX's attribute register to
% to annotate nodes with color information.
% If a color is set, then the attribute register
% is set to this color and all nodes created in its scope
% (current group) are annotated with this attribute.
% Now the color property behaves much the same way
% as the font property.
%
% \subsection{Usage}
%
% Package \xpackage{color} is loaded automatically by this
% package \xpackage{luacolor}. If you need a special driver
% option or you prefer package \xpackage{xcolor}, then load
% it before package \xpackage{luacolor}, for example:
% \begin{quote}
%   |\usepackage[dvipdfmx]{xcolor}|
% \end{quote}
% The package \xpackage{luacolor} is loaded without options:
% \begin{quote}
%   |\usepackage{luacolor}|
% \end{quote}
% It is able to detect PDF mode and DVI drivers are
% differentiated by its color specials. Therefore the
% package do need driver options.
%
% Then it redefines the color setting commands to set
% attributes instead of whatsits for color.
%
% At last the attribute annotations of the nodes in the
% output box must be analyzed to insert the necessary color whatsits.
% (With older \LuaTeX\ that lack the appropriate callback function 
% the package \xpackage{atbegshi} is used to get control
% before a box is shipped out.)
%
% \begin{declcs}{luacolorProcessBox} \M{box}
% \end{declcs}
% Macro \cs{luacolorProcessBox} processes the box \meta{box}
% in the previously described manner. It is automatically
% called for pages, but not for XForm objects.
% Before passing a box to \cs{pdfxform}, call \cs{luacolorProcessBox}
% first.
%
% \subsection{Limitations}
%
% \begin{description}
% \item[Ligatures with different colored components:]
%   Package \xpackage{luacolor} sees the ligature after
%   the paragraph building and page breaking, when a
%   page is to be shipped out. Therefore it cannot
%   break ligatures, because the components might occupy
%   different space. Therefore it is the responsibility
%   of the ligature forming process to deal with different
%   colored glyphs that form a ligature. The user can avoid
%   the problem entirely by explicitly breaking the ligature
%   at the places where the color changes.
%  \item \dots
% \end{description}
%
% \StopEventually{
% }
%
% \section{Implementation}
%
%    \begin{macrocode}
%<*package>
%    \end{macrocode}
%
% \subsection{Catcodes and identification}
%
%    \begin{macrocode}
\begingroup\catcode61\catcode48\catcode32=10\relax%
  \catcode13=5 % ^^M
  \endlinechar=13 %
  \catcode123=1 % {
  \catcode125=2 % }
  \catcode64=11 % @
  \def\x{\endgroup
    \expandafter\edef\csname LuaCol@AtEnd\endcsname{%
      \endlinechar=\the\endlinechar\relax
      \catcode13=\the\catcode13\relax
      \catcode32=\the\catcode32\relax
      \catcode35=\the\catcode35\relax
      \catcode61=\the\catcode61\relax
      \catcode64=\the\catcode64\relax
      \catcode123=\the\catcode123\relax
      \catcode125=\the\catcode125\relax
    }%
  }%
\x\catcode61\catcode48\catcode32=10\relax%
\catcode13=5 % ^^M
\endlinechar=13 %
\catcode35=6 % #
\catcode64=11 % @
\catcode123=1 % {
\catcode125=2 % }
\def\TMP@EnsureCode#1#2{%
  \edef\LuaCol@AtEnd{%
    \LuaCol@AtEnd
    \catcode#1=\the\catcode#1\relax
  }%
  \catcode#1=#2\relax
}
\TMP@EnsureCode{34}{12}% "
\TMP@EnsureCode{39}{12}% '
\TMP@EnsureCode{40}{12}% (
\TMP@EnsureCode{41}{12}% )
\TMP@EnsureCode{42}{12}% *
\TMP@EnsureCode{43}{12}% +
\TMP@EnsureCode{44}{12}% ,
\TMP@EnsureCode{45}{12}% -
\TMP@EnsureCode{46}{12}% .
\TMP@EnsureCode{47}{12}% /
\TMP@EnsureCode{58}{12}% :
\TMP@EnsureCode{60}{12}% <
\TMP@EnsureCode{62}{12}% >
\TMP@EnsureCode{91}{12}% [
\TMP@EnsureCode{93}{12}% ]
\TMP@EnsureCode{95}{12}% _ (other!)
\TMP@EnsureCode{96}{12}% `
\edef\LuaCol@AtEnd{\LuaCol@AtEnd\noexpand\endinput}
%    \end{macrocode}
%
%    Package identification.
%    \begin{macrocode}
\NeedsTeXFormat{LaTeX2e}
\ProvidesPackage{luacolor}%
  [2023-08-18 v1.18 Color support via LuaTeX's attributes (HO)]
%    \end{macrocode}
%
% \subsection{Check for \LuaTeX}
%
%    Without \LuaTeX\ there is no point in using this package.
%    \begin{macrocode}
\RequirePackage{color}
%    \end{macrocode}
%
%    \begin{macrocode}
\ifx\directlua\@undefined
  \PackageError{luacolor}{%
    This package may only be run using LuaTeX%
  }\@ehc
  \expandafter\LuaCol@AtEnd
\fi%
%    \end{macrocode}
%
% \subsection{Check for disabled colors}
%
%    \begin{macrocode}
\ifcolors@
\else
  \PackageWarningNoLine{luacolor}{%
    Colors are disabled by option `monochrome'%
  }%
  \def\set@color{}%
  \def\reset@color{}%
  \def\set@page@color{}%
  \def\define@color#1#2{}%
  \expandafter\LuaCol@AtEnd
\fi%
%    \end{macrocode}
%
% \subsection{Load module and check version}
%
%    \begin{macrocode}
\directlua{%
  require("luacolor")%
}
%    \end{macrocode}
%    \begin{macrocode}
\begingroup
  \edef\x{\directlua{tex.write("2023-08-18 v1.18")}}%
  \edef\y{%
    \directlua{%
      if oberdiek.luacolor.getversion then %
        oberdiek.luacolor.getversion()%
      end%
    }%
  }%
  \ifx\x\y
  \else
    \PackageError{luacolor}{%
      Wrong version of lua module.\MessageBreak
      Package version: \x\MessageBreak
      Lua module: \y
    }\@ehc
  \fi
\endgroup
%    \end{macrocode}
%
% \subsection{Find driver}
%
%    \begin{macrocode}
\ifnum\outputmode=\@ne
\else
  \begingroup
    \def\current@color{}%
    \def\reset@color{}%
    \setbox\z@=\hbox{%
      \begingroup
        \set@color
      \endgroup
    }%
    \edef\reserved@a{%
      \directlua{%
        oberdiek.luacolor.dvidetect()%
      }%
    }%
    \ifx\reserved@a\@empty
      \PackageError{luacolor}{%
        DVI driver detection failed because of\MessageBreak
        unrecognized color \string\special
      }\@ehc
      \endgroup
      \expandafter\expandafter\expandafter\LuaCol@AtEnd
    \else
      \PackageInfo{luacolor}{%
        Type of color \string\special: \reserved@a
      \@gobble}%
    \fi%
  \endgroup
\fi
%    \end{macrocode}
%
% \subsection{Attribute setting}
%
%    \begin{macro}{\LuaCol@Attribute}
%    \begin{macrocode}
\newattribute\LuaCol@Attribute
\let\LuaCol@setattribute\setattribute
\directlua{%
  oberdiek.luacolor.setattribute(\number\allocationnumber)%
}
%    \end{macrocode}
%    \end{macro}
%
%    \begin{macro}{\set@color}
% change 2023-08-18: added \cs{reset@color} so that \cs{mathcolor} can gobble it,
% issue 7.   
%    \begin{macrocode}
\protected\def\set@color{%
  \LuaCol@setattribute\LuaCol@Attribute{%
    \directlua{%
      oberdiek.luacolor.get("\luaescapestring{\current@color}")%
    }%
  }%
 \aftergroup\reset@color 
}
%    \end{macrocode}
%    \end{macro}
%    \begin{macro}{\reset@color}
%    \begin{macrocode}
\def\reset@color{}
%    \end{macrocode}
%    \end{macro}
%
% \subsection{Whatsit insertion}
%
%    \begin{macro}{\luacolorProcessBox}
%    \begin{macrocode}
\def\luacolorProcessBox#1{%
  \directlua{%
    oberdiek.luacolor.process(\number#1)%
  }%
}
%    \end{macrocode}
%    \end{macro}
%
%    \begin{macrocode}
\directlua{%
  if luatexbase.callbacktypes.pre_shipout_filter then
    token.get_next()
  end
}\@secondoftwo\@gobble{
  \RequirePackage{atbegshi}[2011/01/30]
  \AtBeginShipout{%
    \luacolorProcessBox\AtBeginShipoutBox
  }
}
%    \end{macrocode}
%
%    Set default color.
%    \begin{macrocode}
\set@color
%    \end{macrocode}
%
% \subsection{\cs{pdfxform}/\cs{saveboxresource} support}
%
%    \begin{macrocode}
\ifnum\outputmode=\@ne
    \let\LuaCol@org@pdfxform\saveboxresource
%    \end{macrocode}
% First we need some helpers to allow expandable code to parse keyword style arguments:
%    \begin{macrocode}
    \def\LuaCol@iii@i@ii#1#2#3{#3{#1}{#2}}
    \def\LuaCol@ii@i#1#2{{#2#1}}
    \def\LuaCol@if@keyword#1#2#3{%
      \expanded{\unexpanded{\LuaCol@iii@i@ii{#2}{#3}}\expandafter}%
      \directlua{%
        token.put_next(token.create(token.scan_keyword(token.scan_string())
        and '@firstoftwo'
        or '@secondoftwo'))
      }{#1}%
    }
%    \end{macrocode}
% The following macro scans a integer and expands to a token equivalent to a chardef
% whose value corresponds to the scanned integer. This allows the integer to be passed
% around as a undelimited argument.
%    \begin{macrocode}
    \def\LuaCol@scan@number{%
      \directlua{
        token.put_next(token.new(token.scan_int(), token.command_id'char_given'))
      }%
    }
%    \end{macrocode}
% \TeX\ primitives like \cs{saveboxresource} read braced arguments in a special way.
% Especially they expand everything until they find a left brace. To simulate this, we
% use Lua to expand everything else:
%    \begin{macrocode}
    \def\LuaCol@scan@tobrace{%
      \directlua{
        local relax, space = token.command_id'relax', token.command_id'spacer'
        local t
        repeat
          t = token.scan_token()
        until not (t.command == relax or t.command == space)
        token.put_next(t)
      }%
    }
    \def\LuaCol@scan@boxresource@i#1#2{%
      \LuaCol@if@keyword{attr}{%
        \expanded{\unexpanded{\LuaCol@scan@boxresource@iI{#1#2attr}}%
          \expandafter\expandafter\expandafter}%
        \LuaCol@scan@tobrace
      }{%
        \LuaCol@scan@boxresource@ii{#1#2}%
      }%
    }
    \def\LuaCol@scan@boxresource@iI#1#2{\LuaCol@scan@boxresource@ii{#1{#2}}}
    \def\LuaCol@scan@boxresource@ii#1{%
      \LuaCol@if@keyword{resources}{%
        \expanded{\unexpanded{\LuaCol@scan@boxresource@iiI{#1resources}}%
          \expandafter\expandafter\expandafter}%
        \LuaCol@scan@tobrace
      }{%
        \LuaCol@scan@boxresource@iii{#1}%
      }%
    }
    \def\LuaCol@scan@boxresource@iiI#1#2{\LuaCol@scan@boxresource@iii{#1{#2}}}
    \def\LuaCol@scan@boxresource@iii#1{%
      \LuaCol@if@keyword{margin}{%
        \expanded{\unexpanded{\LuaCol@scan@boxresource@iv{#1margin }}%
          \expandafter\expandafter\expandafter}%
        \LuaCol@scan@number
      }{%
        \LuaCol@scan@boxresource@iv{#1}{}%
      }%
    }
    \def\LuaCol@scan@boxresource@iv#1#2{%
      \expanded{\unexpanded{\LuaCol@scan@boxresource@v{#1#2}}%
        \expandafter\expandafter\expandafter}%
      \LuaCol@scan@number
    }
    \def\LuaCol@scan@boxresource@v#1#2{%
      \luacolorProcessBox{#2}%
      \LuaCol@org@pdfxform#1#2%
    }

%    \end{macrocode}
% This could be written in Lua, but at least upto Lua\TeX~1.11, feeding back too many tokens
% from Lua to \TeX\ triggers a segmentation fault.
% This is written in Lua so the integer setting is expandable and does not interfere with
% a preceding \verb|\immediate|.
%    \begin{macrocode}
    \protected\def\saveboxresource{%
      \LuaCol@if@keyword{type}{%
        \expandafter
        \expanded{\unexpanded{\LuaCol@scan@boxresource@i{type }}%
          \expandafter\expandafter\expandafter}%
        \LuaCol@scan@number
      }{%
        \LuaCol@scan@boxresource@i{}{}%
      }%
    }
%    \end{macrocode}
%    Legacy alias.
%    \begin{macrocode}
    \let\pdfxform\saveboxresource
\fi
%    \end{macrocode}
%
%    \begin{macrocode}
\LuaCol@AtEnd%
%</package>
%    \end{macrocode}
%
% \subsection{Lua module}
%
%    \begin{macrocode}
%<*lua>
%    \end{macrocode}
%    Box zero contains a \cs{hbox} with the color \cs{special}.
%    That is analyzed to get the prefix for the color setting
%    \cs{special}.
%    \begin{macrocode}
oberdiek = oberdiek or {}
local luacolor = oberdiek.luacolor or {}
oberdiek.luacolor = luacolor
%    \end{macrocode}
%    \begin{macro}{getversion()}
%    \begin{macrocode}
function luacolor.getversion()
  tex.write("2023-08-18 v1.18")
end
%    \end{macrocode}
%    \end{macro}
%
% \subsubsection{Driver detection}
%
%    \begin{macrocode}
local ifpdf = tonumber(tex.outputmode or tex.pdfoutput) > 0
local prefix
local prefixes = {
  dvips   = "color ",
  dvipdfm = "pdf:sc ",
  truetex = "textcolor:",
  pctexps = "ps::",
}
local patterns = {
  ["^color "]            = "dvips",
  ["^pdf: *begincolor "] = "dvipdfm",
  ["^pdf: *bcolor "]     = "dvipdfm",
  ["^pdf: *bc "]         = "dvipdfm",
  ["^pdf: *setcolor "]   = "dvipdfm",
  ["^pdf: *scolor "]     = "dvipdfm",
  ["^pdf: *sc "]         = "dvipdfm",
  ["^textcolor:"]        = "truetex",
  ["^ps::"]              = "pctexps",
}
%    \end{macrocode}
%    \begin{macro}{info()}
%    \begin{macrocode}
local function info(msg, term)
  local target = "log"
  if term then
    target = "term and log"
  end
  texio.write_nl(target, "Package luacolor info: " .. msg .. ".")
  texio.write_nl(target, "")
end
%    \end{macrocode}
%    \end{macro}
%    \begin{macro}{dvidetect()}
%    \begin{macrocode}
function luacolor.dvidetect()
  local v = tex.box[0]
  assert(v.id == node.id("hlist"))
  for v in node.traverse_id(node.id("whatsit"), v.head) do
    if v and v.subtype == node.subtype("special") then
      local data = v.data
      for pattern, driver in pairs(patterns) do
        if string.find(data, pattern) then
          prefix = prefixes[driver]
          tex.write(driver)
          return
        end
      end
      info("\\special{" .. data .. "}", true)
      return
    end
  end
  info("Missing \\special", true)
end
%    \end{macrocode}
%    \end{macro}
%
% \subsubsection{Color strings}
%
%    \begin{macrocode}
local map = {
  n = 0,
}
%    \end{macrocode}
%    \begin{macro}{get()}
%    \begin{macrocode}
function luacolor.get(color)
  tex.write("" .. luacolor.getvalue(color))
end
%    \end{macrocode}
%    \end{macro}
%    \begin{macro}{getvalue()}
%    \begin{macrocode}
function luacolor.getvalue(color)
  local n = map[color]
  if not n then
    n = map.n + 1
    map.n = n
    map[n] = color
    map[color] = n
  end
  return n
end
%    \end{macrocode}
%    \end{macro}
%
% \subsubsection{Attribute register}
%
%    \begin{macro}{setattribute()}
%    \begin{macrocode}
local attribute
function luacolor.setattribute(attr)
  attribute = attr
end
%    \end{macrocode}
%    \end{macro}
%
%    \begin{macro}{getattribute()}
%    \begin{macrocode}
function luacolor.getattribute()
  return attribute
end
%    \end{macrocode}
%    \end{macro}
%
% \subsubsection{Whatsit insertion}
%
%    \begin{macrocode}
local LIST = 1
local LIST_LEADERS = 2
local LIST_DISC = 3
local COLOR = 4
local NOCOLOR = 5
local RULE = node.id("rule")
local node_types = {
  [node.id("hlist")] = LIST,
  [node.id("vlist")] = LIST,
  [node.id("rule")]  = COLOR,
  [node.id("glyph")] = COLOR,
  [node.id("disc")]  = LIST_DISC,
  [node.id("whatsit")] = {
    [node.subtype("pdf_colorstack")] =
      function(n)
        return n.stack == 0 and NOCOLOR or nil
      end,
    [node.subtype("special")] = COLOR,
    [node.subtype("pdf_literal")] = COLOR,
    [node.subtype("pdf_save")] = COLOR,
    [node.subtype("pdf_restore")] = COLOR, -- probably not needed
-- TODO (DPC)    [node.subtype("pdf_refximage")] = COLOR,
  },
  [node.id("glue")] =
    function(n)
      if n.subtype >= 100 then -- leaders
        if n.leader.id == RULE then
          return COLOR
        else
          return LIST_LEADERS
        end
      end
    end,
}
%    \end{macrocode}
%    \begin{macro}{get_type()}
%    \begin{macrocode}
local function get_type(n)
  local ret = node_types[n.id]
  if type(ret) == 'table' then
    ret = ret[n.subtype]
  end
  if type(ret) == 'function' then
    ret = ret(n)
  end
  return ret
end
%    \end{macrocode}
%    \end{macro}
%    \begin{macrocode}
local mode = 2 -- luatex.pdfliteral.direct
local WHATSIT = node.id("whatsit")
local SPECIAL = node.subtype("special")
local PDFLITERAL = node.subtype("pdf_literal")
local DRY_FALSE = false
local DRY_TRUE = true
%    \end{macrocode}
%    \begin{macro}{traverse()}
%    \begin{macrocode}
local function traverse(list, color, dry)
  if not list then
    return color
  end
  local head
  if get_type(list) == LIST then
    head = list.head
  elseif get_type(list) == LIST_DISC then
    head = list.replace
  else
    texio.write_nl("!!! Error: Wrong list type: " .. node.type(list.id))
    return color
  end
%<debug>texio.write_nl("traverse: " .. node.type(list.id))
  for n in node.traverse(head) do
%<debug>texio.write_nl("  node: " .. node.type(n.id))
    local t = get_type(n)
%<debug>texio.write_nl("TYPE "..tostring(t).. " "..tostring(node.type(node.getid(n))).." ".. tostring(node.getsubtype(n)))
    if t == LIST or t == LIST_DISC then
      color = traverse(n, color, dry)
    elseif t == LIST_LEADERS then
      local color_after = traverse(n.leader, color, DRY_TRUE)
      if color == color_after then
        traverse(n.leader, color, DRY_FALSE or dry)
      else
        traverse(n.leader, '', DRY_FALSE or dry)
%    \end{macrocode}
% The color status is unknown here, because the leader box
% will or will not be set.
%    \begin{macrocode}
        color = ''
      end
    elseif t == COLOR then
      local v = node.has_attribute(n, attribute)
      if v then
        local newColor = map[v]
        if newColor ~= color then
          color = newColor
          if dry == DRY_FALSE then
            local newNode
            if ifpdf then
              newNode = node.new(WHATSIT, PDFLITERAL)
              newNode.mode = mode
              newNode.data = color
            else
              newNode = node.new(WHATSIT, SPECIAL)
              newNode.data = prefix .. color
            end
            head = node.insert_before(head, n, newNode)
          end
        end
      end
    elseif t == NOCOLOR then
      color = ''
    end
  end
  if get_type(list) == LIST then
    list.head = head
  else
    list.replace = head
  end
  return color
end
%    \end{macrocode}
%    \end{macro}
%
%    \begin{macro}{process()}
%    \begin{macrocode}
function luacolor.process(box)
  local color = ""
  local list = tex.getbox(box)
  traverse(list, color, DRY_FALSE)
end

if luatexbase.callbacktypes.pre_shipout_filter then
  luatexbase.add_to_callback("pre_shipout_filter", function(list)
    traverse(list, "", DRY_FALSE)
    return true
  end, "luacolor.process")
end
%    \end{macrocode}
%    \end{macro}
% For recent versions of luaotfload, we can register a callback to
% control how coloring glyph is handled for the color feature.
%    \begin{macrocode}
if luaotfload.set_colorhandler then
  local set_attribute = node.direct.set_attribute
  luaotfload.set_colorhandler(function(head, n, color)
    set_attribute(n, attribute, luacolor.getvalue(color))
    return head, n
  end)
end
%    \end{macrocode}
%
%    \begin{macrocode}
%</lua>
%    \end{macrocode}
% \section{Installation}
%
% \subsection{Download}
%
% \paragraph{Package.} This package is available on
% CTAN\footnote{\CTANpkg{luacolor}}:
% \begin{description}
% \item[\CTAN{macros/latex/contrib/luacolor/luacolor.dtx}] The source file.
% \item[\CTAN{macros/latex/contrib/luacolor/luacolor.pdf}] Documentation.
% \end{description}
%
%
% \paragraph{Bundle.} All the packages of the bundle `luacolor'
% are also available in a TDS compliant ZIP archive. There
% the packages are already unpacked and the documentation files
% are generated. The files and directories obey the TDS standard.
% \begin{description}
% \item[\CTANinstall{install/macros/latex/contrib/luacolor.tds.zip}]
% \end{description}
% \emph{TDS} refers to the standard ``A Directory Structure
% for \TeX\ Files'' (\CTANpkg{tds}). Directories
% with \xfile{texmf} in their name are usually organized this way.
%
% \subsection{Bundle installation}
%
% \paragraph{Unpacking.} Unpack the \xfile{luacolor.tds.zip} in the
% TDS tree (also known as \xfile{texmf} tree) of your choice.
% Example (linux):
% \begin{quote}
%   |unzip luacolor.tds.zip -d ~/texmf|
% \end{quote}
%
% \paragraph{Script installation.}
% Check the directory \xfile{TDS:scripts/luacolor/} for
% scripts that need further installation steps.
%
% \subsection{Package installation}
%
% \paragraph{Unpacking.} The \xfile{.dtx} file is a self-extracting
% \docstrip\ archive. The files are extracted by running the
% \xfile{.dtx} through \plainTeX:
% \begin{quote}
%   \verb|tex luacolor.dtx|
% \end{quote}
%
% \paragraph{TDS.} Now the different files must be moved into
% the different directories in your installation TDS tree
% (also known as \xfile{texmf} tree):
% \begin{quote}
% \def\t{^^A
% \begin{tabular}{@{}>{\ttfamily}l@{ $\rightarrow$ }>{\ttfamily}l@{}}
%   luacolor.sty & tex/latex/luacolor/luacolor.sty\\
%   luacolor.lua & scripts/luacolor/luacolor.lua\\
%   luacolor.pdf & doc/latex/luacolor/luacolor.pdf\\
%   luacolor.dtx & source/latex/luacolor/luacolor.dtx\\
% \end{tabular}^^A
% }^^A
% \sbox0{\t}^^A
% \ifdim\wd0>\linewidth
%   \begingroup
%     \advance\linewidth by\leftmargin
%     \advance\linewidth by\rightmargin
%   \edef\x{\endgroup
%     \def\noexpand\lw{\the\linewidth}^^A
%   }\x
%   \def\lwbox{^^A
%     \leavevmode
%     \hbox to \linewidth{^^A
%       \kern-\leftmargin\relax
%       \hss
%       \usebox0
%       \hss
%       \kern-\rightmargin\relax
%     }^^A
%   }^^A
%   \ifdim\wd0>\lw
%     \sbox0{\small\t}^^A
%     \ifdim\wd0>\linewidth
%       \ifdim\wd0>\lw
%         \sbox0{\footnotesize\t}^^A
%         \ifdim\wd0>\linewidth
%           \ifdim\wd0>\lw
%             \sbox0{\scriptsize\t}^^A
%             \ifdim\wd0>\linewidth
%               \ifdim\wd0>\lw
%                 \sbox0{\tiny\t}^^A
%                 \ifdim\wd0>\linewidth
%                   \lwbox
%                 \else
%                   \usebox0
%                 \fi
%               \else
%                 \lwbox
%               \fi
%             \else
%               \usebox0
%             \fi
%           \else
%             \lwbox
%           \fi
%         \else
%           \usebox0
%         \fi
%       \else
%         \lwbox
%       \fi
%     \else
%       \usebox0
%     \fi
%   \else
%     \lwbox
%   \fi
% \else
%   \usebox0
% \fi
% \end{quote}
% If you have a \xfile{docstrip.cfg} that configures and enables \docstrip's
% TDS installing feature, then some files can already be in the right
% place, see the documentation of \docstrip.
%
% \subsection{Refresh file name databases}
%
% If your \TeX~distribution
% (\TeX\,Live, \mikTeX, \dots) relies on file name databases, you must refresh
% these. For example, \TeX\,Live\ users run \verb|texhash| or
% \verb|mktexlsr|.
%
% \subsection{Some details for the interested}
%
% \paragraph{Unpacking with \LaTeX.}
% The \xfile{.dtx} chooses its action depending on the format:
% \begin{description}
% \item[\plainTeX:] Run \docstrip\ and extract the files.
% \item[\LaTeX:] Generate the documentation.
% \end{description}
% If you insist on using \LaTeX\ for \docstrip\ (really,
% \docstrip\ does not need \LaTeX), then inform the autodetect routine
% about your intention:
% \begin{quote}
%   \verb|latex \let\install=y\input{luacolor.dtx}|
% \end{quote}
% Do not forget to quote the argument according to the demands
% of your shell.
%
% \paragraph{Generating the documentation.}
% You can use both the \xfile{.dtx} or the \xfile{.drv} to generate
% the documentation. The process can be configured by the
% configuration file \xfile{ltxdoc.cfg}. For instance, put this
% line into this file, if you want to have A4 as paper format:
% \begin{quote}
%   \verb|\PassOptionsToClass{a4paper}{article}|
% \end{quote}
% An example follows how to generate the
% documentation with pdf\LaTeX:
% \begin{quote}
%\begin{verbatim}
%pdflatex luacolor.dtx
%makeindex -s gind.ist luacolor.idx
%pdflatex luacolor.dtx
%makeindex -s gind.ist luacolor.idx
%pdflatex luacolor.dtx
%\end{verbatim}
% \end{quote}
%
% \begin{History}
%   \begin{Version}{2007/12/12 v1.0}
%   \item
%     First public version.
%   \end{Version}
%   \begin{Version}{2009/04/10 v1.1}
%   \item
%     Fixes for changed syntax of \cs{directlua} in \LuaTeX\ 0.36.
%   \end{Version}
%   \begin{Version}{2010/03/09 v1.2}
%   \item
%     Adaptation for package \xpackage{luatex} 2010/03/09 v0.4.
%   \end{Version}
%   \begin{Version}{2010/12/13 v1.3}
%   \item
%     Support for \cs{pdfxform} added.
%   \item
%     Loaded package \xpackage{luatexbase-attr} recognized.
%   \item
%     Update for \hologo{LuaTeX}: `list' fields renamed to `head' in v0.65.0.
%   \end{Version}
%   \begin{Version}{2011/03/29 v1.4}
%   \item
%     Avoid whatsit insertion if option \xoption{monochrome} is used
%     (thanks Manuel P\'egouri\'e-Gonnard).
%   \end{Version}
%   \begin{Version}{2011/04/22 v1.5}
%   \item
%     Bug fix by Manuel P\'egouri\'e-Gonnard: A typo prevented the
%     detection of whatsits and applying color changes for
%     \cs{pdfliteral} and \cs{special} nodes that might contain
%     typesetting material.
%   \item
%     Bug fix by Manuel P\'egouri\'e-Gonnard: Now colors are also
%     applied to leader boxes.
%   \item
%     Unnecessary color settings are removed for leaders boxes,
%     if after the leader box the color has not changed.
%     The costs are a little runtime, leader boxes are processed twice.
%   \item
%     Additional whatsits that are colored: \texttt{pdf\_refximage}.
%   \item
%     Workaround for bug with \texttt{node.insert\_before}
%     removed for the version after \hologo{LuaTeX} 0.65, because
%     bug was fixed in 0.27. (Thanks Manuel P\'egouri\'e-Gonnard.)
%   \end{Version}
%   \begin{Version}{2011/04/23 v1.6}
%   \item
%     Bug fix for nested leader boxes.
%   \item
%     Bug fix for leader boxes that change color, but
%     are not set because of missing place.
%   \item
%     Version check for Lua module added.
%   \end{Version}
%   \begin{Version}{2011/10/22 v1.7}
%   \item
%     Lua functions \texttt{getattribute} and \texttt{getvalue}
%     added to tell other external Lua functions
%     the attribute register number for coloring.
%   \end{Version}
%   \begin{Version}{2011/11/01 v1.8}
%   \item
%     Use of \texttt{node.subtype} instead of magic numbers.
%   \end{Version}
%   \begin{Version}{2016/05/13 v1.9}
%   \item
%     More use of \texttt{node.subtype} instead of magic numbers.
%   \item luatex 85 updates
%   \end{Version}
%   \begin{Version}{2016/05/16 v1.10}
%   \item
%     Documentation updates.
%   \end{Version}
%   \begin{Version}{2018/11/22 v1.11}
%   \item
%     handle issue 43.
%   \item
%     removed pre-0.65 stuff
%   \end{Version}
%   \begin{Version}{2019/07/25 v1.12}
%   \item
%     removed uses of module function, see PR70
%   \end{Version}
%   \begin{Version}{2019/11/29 v1.13}
%   \item
%     Documentation updates.
%   \item  Use \xpackage{iftex} directly.
%   \end{Version}
%   \begin{Version}{2020-02-22 v1.14}
%   \item Drop use of \textsf{iftex} \textsf{ltxcmds} and \textsf{infwarerr}.
%   \item Assume \textsf{ltluatex} preloaded into format (true since 2015).
%   \item Patch \verb|\saveboxresource| rather than \verb|\pdfxform| (keep old name as alias).
%   \item Grab the number via Lua so that a \verb|\immediate| prefix still works with
%         \verb|\saveboxresource|\slash \verb|\pdfxform|.
%   \item Added handler for the color feature of luaotfload      
%   \end{Version}
%   \begin{Version}{2020-02-24 v1.15}
%   \item Grab all possible arguments for \verb|\saveboxresource|\slash \verb|\pdfxform|
%   \end{Version}
%   \begin{Version}{2020-04-04 v1.16}
%   \item Reset color after \verb|pdf_colorstack| whatsits.
%   \end{Version}
%   \begin{Version}{2021-02-17 v1.17}
%   \item Use \LaTeXe's new \verb|pre_shipout_filter| callback if it's available to
%         allow coloring background and foreground layer material
%   \end{Version}
%   \begin{Version}{2023-08-18 v1.18}
%   \item added \verb|\reset@color| to \verb|\set@color| for \verb|\mathcolor|, issue 7.
%   \end{Version}
% \end{History}
%
% \PrintIndex
%
% \Finale
\endinput
|
% \end{quote}
% Do not forget to quote the argument according to the demands
% of your shell.
%
% \paragraph{Generating the documentation.}
% You can use both the \xfile{.dtx} or the \xfile{.drv} to generate
% the documentation. The process can be configured by the
% configuration file \xfile{ltxdoc.cfg}. For instance, put this
% line into this file, if you want to have A4 as paper format:
% \begin{quote}
%   \verb|\PassOptionsToClass{a4paper}{article}|
% \end{quote}
% An example follows how to generate the
% documentation with pdf\LaTeX:
% \begin{quote}
%\begin{verbatim}
%pdflatex luacolor.dtx
%makeindex -s gind.ist luacolor.idx
%pdflatex luacolor.dtx
%makeindex -s gind.ist luacolor.idx
%pdflatex luacolor.dtx
%\end{verbatim}
% \end{quote}
%
% \begin{History}
%   \begin{Version}{2007/12/12 v1.0}
%   \item
%     First public version.
%   \end{Version}
%   \begin{Version}{2009/04/10 v1.1}
%   \item
%     Fixes for changed syntax of \cs{directlua} in \LuaTeX\ 0.36.
%   \end{Version}
%   \begin{Version}{2010/03/09 v1.2}
%   \item
%     Adaptation for package \xpackage{luatex} 2010/03/09 v0.4.
%   \end{Version}
%   \begin{Version}{2010/12/13 v1.3}
%   \item
%     Support for \cs{pdfxform} added.
%   \item
%     Loaded package \xpackage{luatexbase-attr} recognized.
%   \item
%     Update for \hologo{LuaTeX}: `list' fields renamed to `head' in v0.65.0.
%   \end{Version}
%   \begin{Version}{2011/03/29 v1.4}
%   \item
%     Avoid whatsit insertion if option \xoption{monochrome} is used
%     (thanks Manuel P\'egouri\'e-Gonnard).
%   \end{Version}
%   \begin{Version}{2011/04/22 v1.5}
%   \item
%     Bug fix by Manuel P\'egouri\'e-Gonnard: A typo prevented the
%     detection of whatsits and applying color changes for
%     \cs{pdfliteral} and \cs{special} nodes that might contain
%     typesetting material.
%   \item
%     Bug fix by Manuel P\'egouri\'e-Gonnard: Now colors are also
%     applied to leader boxes.
%   \item
%     Unnecessary color settings are removed for leaders boxes,
%     if after the leader box the color has not changed.
%     The costs are a little runtime, leader boxes are processed twice.
%   \item
%     Additional whatsits that are colored: \texttt{pdf\_refximage}.
%   \item
%     Workaround for bug with \texttt{node.insert\_before}
%     removed for the version after \hologo{LuaTeX} 0.65, because
%     bug was fixed in 0.27. (Thanks Manuel P\'egouri\'e-Gonnard.)
%   \end{Version}
%   \begin{Version}{2011/04/23 v1.6}
%   \item
%     Bug fix for nested leader boxes.
%   \item
%     Bug fix for leader boxes that change color, but
%     are not set because of missing place.
%   \item
%     Version check for Lua module added.
%   \end{Version}
%   \begin{Version}{2011/10/22 v1.7}
%   \item
%     Lua functions \texttt{getattribute} and \texttt{getvalue}
%     added to tell other external Lua functions
%     the attribute register number for coloring.
%   \end{Version}
%   \begin{Version}{2011/11/01 v1.8}
%   \item
%     Use of \texttt{node.subtype} instead of magic numbers.
%   \end{Version}
%   \begin{Version}{2016/05/13 v1.9}
%   \item
%     More use of \texttt{node.subtype} instead of magic numbers.
%   \item luatex 85 updates
%   \end{Version}
%   \begin{Version}{2016/05/16 v1.10}
%   \item
%     Documentation updates.
%   \end{Version}
%   \begin{Version}{2018/11/22 v1.11}
%   \item
%     handle issue 43.
%   \item
%     removed pre-0.65 stuff
%   \end{Version}
%   \begin{Version}{2019/07/25 v1.12}
%   \item
%     removed uses of module function, see PR70
%   \end{Version}
%   \begin{Version}{2019/11/29 v1.13}
%   \item
%     Documentation updates.
%   \item  Use \xpackage{iftex} directly.
%   \end{Version}
%   \begin{Version}{2020-02-22 v1.14}
%   \item Drop use of \textsf{iftex} \textsf{ltxcmds} and \textsf{infwarerr}.
%   \item Assume \textsf{ltluatex} preloaded into format (true since 2015).
%   \item Patch \verb|\saveboxresource| rather than \verb|\pdfxform| (keep old name as alias).
%   \item Grab the number via Lua so that a \verb|\immediate| prefix still works with
%         \verb|\saveboxresource|\slash \verb|\pdfxform|.
%   \item Added handler for the color feature of luaotfload      
%   \end{Version}
%   \begin{Version}{2020-02-24 v1.15}
%   \item Grab all possible arguments for \verb|\saveboxresource|\slash \verb|\pdfxform|
%   \end{Version}
%   \begin{Version}{2020-04-04 v1.16}
%   \item Reset color after \verb|pdf_colorstack| whatsits.
%   \end{Version}
%   \begin{Version}{2021-02-17 v1.17}
%   \item Use \LaTeXe's new \verb|pre_shipout_filter| callback if it's available to
%         allow coloring background and foreground layer material
%   \end{Version}
%   \begin{Version}{2023-08-18 v1.18}
%   \item added \verb|\reset@color| to \verb|\set@color| for \verb|\mathcolor|, issue 7.
%   \end{Version}
% \end{History}
%
% \PrintIndex
%
% \Finale
\endinput

%        (quote the arguments according to the demands of your shell)
%
% Documentation:
%    (a) If luacolor.drv is present:
%           latex luacolor.drv
%    (b) Without luacolor.drv:
%           latex luacolor.dtx; ...
%    The class ltxdoc loads the configuration file ltxdoc.cfg
%    if available. Here you can specify further options, e.g.
%    use A4 as paper format:
%       \PassOptionsToClass{a4paper}{article}
%
%    Programm calls to get the documentation (example):
%       pdflatex luacolor.dtx
%       makeindex -s gind.ist luacolor.idx
%       pdflatex luacolor.dtx
%       makeindex -s gind.ist luacolor.idx
%       pdflatex luacolor.dtx
%
% Installation:
%    TDS:tex/latex/luacolor/luacolor.sty
%    TDS:scripts/luacolor/luacolor.lua
%    TDS:doc/latex/luacolor/luacolor.pdf
%    TDS:source/latex/luacolor/luacolor.dtx
%
%<*ignore>
\begingroup
  \catcode123=1 %
  \catcode125=2 %
  \def\x{LaTeX2e}%
\expandafter\endgroup
\ifcase 0\ifx\install y1\fi\expandafter
         \ifx\csname processbatchFile\endcsname\relax\else1\fi
         \ifx\fmtname\x\else 1\fi\relax
\else\csname fi\endcsname
%</ignore>
%<*install>
\input docstrip.tex
\Msg{************************************************************************}
\Msg{* Installation}
\Msg{* Package: luacolor 2023-08-18 v1.18 Color support via LuaTeX's attributes (HO)}
\Msg{************************************************************************}

\keepsilent
\askforoverwritefalse

\let\MetaPrefix\relax
\preamble

This is a generated file.

Project: luacolor
Version: 2023-08-18 v1.18

Copyright (C)
   2007, 2009-2011 Heiko Oberdiek
   2016-2023 Oberdiek Package Support Group

This work may be distributed and/or modified under the
conditions of the LaTeX Project Public License, either
version 1.3c of this license or (at your option) any later
version. This version of this license is in
   https://www.latex-project.org/lppl/lppl-1-3c.txt
and the latest version of this license is in
   https://www.latex-project.org/lppl.txt
and version 1.3 or later is part of all distributions of
LaTeX version 2005/12/01 or later.

This work has the LPPL maintenance status "maintained".

The Current Maintainers of this work are
Heiko Oberdiek and the Oberdiek Package Support Group
https://github.com/ho-tex/luacolor/issues


This work consists of the main source file luacolor.dtx
and the derived files
   luacolor.sty, luacolor.pdf, luacolor.ins, luacolor.drv,
   luacolor.lua,

\endpreamble
\let\MetaPrefix\DoubleperCent

\generate{%
  \file{luacolor.ins}{\from{luacolor.dtx}{install}}%
  \file{luacolor.drv}{\from{luacolor.dtx}{driver}}%
  \usedir{tex/latex/luacolor}%
  \file{luacolor.sty}{\from{luacolor.dtx}{package}}%
}
\def\MetaPrefix{-- }
\def\defaultpostamble{%
  \MetaPrefix^^J%
  \MetaPrefix\space End of File `\outFileName'.%
}
\def\currentpostamble{\defaultpostamble}%
\generate{%
  \usedir{scripts/luacolor}%
%  \file{oberdiek.luacolor.lua}{\from{luacolor.dtx}{lua}}%
  \file{luacolor.lua}{\from{luacolor.dtx}{lua}}%
}

\catcode32=13\relax% active space
\let =\space%
\Msg{************************************************************************}
\Msg{*}
\Msg{* To finish the installation you have to move the following}
\Msg{* file into a directory searched by TeX:}
\Msg{*}
\Msg{*     luacolor.sty}
\Msg{*}
\Msg{* And install the following script file:}
\Msg{*}
\Msg{*     luacolor.lua,}
\Msg{*}
\Msg{* To produce the documentation run the file `luacolor.dtx'}
\Msg{* through LaTeX.}
\Msg{*}
\Msg{* Happy TeXing!}
\Msg{*}
\Msg{************************************************************************}

\endbatchfile
%</install>
%<*ignore>
\fi
%</ignore>
%<*driver>
\NeedsTeXFormat{LaTeX2e}
\ProvidesFile{luacolor.drv}%
  [2023-08-18 v1.18 Color support via LuaTeX's attributes (HO)]%
\documentclass{ltxdoc}
\usepackage{holtxdoc}[2011/11/22]
\usepackage{paralist}
\makeatletter
\g@addto@macro\MakePrivateLetters{%
  \@makeother\_%
}
\makeatother
\begin{document}
  \DocInput{luacolor.dtx}%
\end{document}
%</driver>
% \fi
%
%
%
% \GetFileInfo{luacolor.drv}
%
% \title{The \xpackage{luacolor} package}
% \date{2023-08-18 v1.18}
% \author{Heiko Oberdiek\thanks
% {Please report any issues at \url{https://github.com/ho-tex/luacolor/issues}}}
%
% \maketitle
%
% \begin{abstract}
% Package \xpackage{luacolor} implements color support based
% on \LuaTeX's node attributes.
% \end{abstract}
%
% \tableofcontents
%
% \section{Documentation}
%
% \subsection{Introduction}
%
% This package uses a \LuaTeX's attribute register to
% to annotate nodes with color information.
% If a color is set, then the attribute register
% is set to this color and all nodes created in its scope
% (current group) are annotated with this attribute.
% Now the color property behaves much the same way
% as the font property.
%
% \subsection{Usage}
%
% Package \xpackage{color} is loaded automatically by this
% package \xpackage{luacolor}. If you need a special driver
% option or you prefer package \xpackage{xcolor}, then load
% it before package \xpackage{luacolor}, for example:
% \begin{quote}
%   |\usepackage[dvipdfmx]{xcolor}|
% \end{quote}
% The package \xpackage{luacolor} is loaded without options:
% \begin{quote}
%   |\usepackage{luacolor}|
% \end{quote}
% It is able to detect PDF mode and DVI drivers are
% differentiated by its color specials. Therefore the
% package do need driver options.
%
% Then it redefines the color setting commands to set
% attributes instead of whatsits for color.
%
% At last the attribute annotations of the nodes in the
% output box must be analyzed to insert the necessary color whatsits.
% (With older \LuaTeX\ that lack the appropriate callback function 
% the package \xpackage{atbegshi} is used to get control
% before a box is shipped out.)
%
% \begin{declcs}{luacolorProcessBox} \M{box}
% \end{declcs}
% Macro \cs{luacolorProcessBox} processes the box \meta{box}
% in the previously described manner. It is automatically
% called for pages, but not for XForm objects.
% Before passing a box to \cs{pdfxform}, call \cs{luacolorProcessBox}
% first.
%
% \subsection{Limitations}
%
% \begin{description}
% \item[Ligatures with different colored components:]
%   Package \xpackage{luacolor} sees the ligature after
%   the paragraph building and page breaking, when a
%   page is to be shipped out. Therefore it cannot
%   break ligatures, because the components might occupy
%   different space. Therefore it is the responsibility
%   of the ligature forming process to deal with different
%   colored glyphs that form a ligature. The user can avoid
%   the problem entirely by explicitly breaking the ligature
%   at the places where the color changes.
%  \item \dots
% \end{description}
%
% \StopEventually{
% }
%
% \section{Implementation}
%
%    \begin{macrocode}
%<*package>
%    \end{macrocode}
%
% \subsection{Catcodes and identification}
%
%    \begin{macrocode}
\begingroup\catcode61\catcode48\catcode32=10\relax%
  \catcode13=5 % ^^M
  \endlinechar=13 %
  \catcode123=1 % {
  \catcode125=2 % }
  \catcode64=11 % @
  \def\x{\endgroup
    \expandafter\edef\csname LuaCol@AtEnd\endcsname{%
      \endlinechar=\the\endlinechar\relax
      \catcode13=\the\catcode13\relax
      \catcode32=\the\catcode32\relax
      \catcode35=\the\catcode35\relax
      \catcode61=\the\catcode61\relax
      \catcode64=\the\catcode64\relax
      \catcode123=\the\catcode123\relax
      \catcode125=\the\catcode125\relax
    }%
  }%
\x\catcode61\catcode48\catcode32=10\relax%
\catcode13=5 % ^^M
\endlinechar=13 %
\catcode35=6 % #
\catcode64=11 % @
\catcode123=1 % {
\catcode125=2 % }
\def\TMP@EnsureCode#1#2{%
  \edef\LuaCol@AtEnd{%
    \LuaCol@AtEnd
    \catcode#1=\the\catcode#1\relax
  }%
  \catcode#1=#2\relax
}
\TMP@EnsureCode{34}{12}% "
\TMP@EnsureCode{39}{12}% '
\TMP@EnsureCode{40}{12}% (
\TMP@EnsureCode{41}{12}% )
\TMP@EnsureCode{42}{12}% *
\TMP@EnsureCode{43}{12}% +
\TMP@EnsureCode{44}{12}% ,
\TMP@EnsureCode{45}{12}% -
\TMP@EnsureCode{46}{12}% .
\TMP@EnsureCode{47}{12}% /
\TMP@EnsureCode{58}{12}% :
\TMP@EnsureCode{60}{12}% <
\TMP@EnsureCode{62}{12}% >
\TMP@EnsureCode{91}{12}% [
\TMP@EnsureCode{93}{12}% ]
\TMP@EnsureCode{95}{12}% _ (other!)
\TMP@EnsureCode{96}{12}% `
\edef\LuaCol@AtEnd{\LuaCol@AtEnd\noexpand\endinput}
%    \end{macrocode}
%
%    Package identification.
%    \begin{macrocode}
\NeedsTeXFormat{LaTeX2e}
\ProvidesPackage{luacolor}%
  [2023-08-18 v1.18 Color support via LuaTeX's attributes (HO)]
%    \end{macrocode}
%
% \subsection{Check for \LuaTeX}
%
%    Without \LuaTeX\ there is no point in using this package.
%    \begin{macrocode}
\RequirePackage{color}
%    \end{macrocode}
%
%    \begin{macrocode}
\ifx\directlua\@undefined
  \PackageError{luacolor}{%
    This package may only be run using LuaTeX%
  }\@ehc
  \expandafter\LuaCol@AtEnd
\fi%
%    \end{macrocode}
%
% \subsection{Check for disabled colors}
%
%    \begin{macrocode}
\ifcolors@
\else
  \PackageWarningNoLine{luacolor}{%
    Colors are disabled by option `monochrome'%
  }%
  \def\set@color{}%
  \def\reset@color{}%
  \def\set@page@color{}%
  \def\define@color#1#2{}%
  \expandafter\LuaCol@AtEnd
\fi%
%    \end{macrocode}
%
% \subsection{Load module and check version}
%
%    \begin{macrocode}
\directlua{%
  require("luacolor")%
}
%    \end{macrocode}
%    \begin{macrocode}
\begingroup
  \edef\x{\directlua{tex.write("2023-08-18 v1.18")}}%
  \edef\y{%
    \directlua{%
      if oberdiek.luacolor.getversion then %
        oberdiek.luacolor.getversion()%
      end%
    }%
  }%
  \ifx\x\y
  \else
    \PackageError{luacolor}{%
      Wrong version of lua module.\MessageBreak
      Package version: \x\MessageBreak
      Lua module: \y
    }\@ehc
  \fi
\endgroup
%    \end{macrocode}
%
% \subsection{Find driver}
%
%    \begin{macrocode}
\ifnum\outputmode=\@ne
\else
  \begingroup
    \def\current@color{}%
    \def\reset@color{}%
    \setbox\z@=\hbox{%
      \begingroup
        \set@color
      \endgroup
    }%
    \edef\reserved@a{%
      \directlua{%
        oberdiek.luacolor.dvidetect()%
      }%
    }%
    \ifx\reserved@a\@empty
      \PackageError{luacolor}{%
        DVI driver detection failed because of\MessageBreak
        unrecognized color \string\special
      }\@ehc
      \endgroup
      \expandafter\expandafter\expandafter\LuaCol@AtEnd
    \else
      \PackageInfo{luacolor}{%
        Type of color \string\special: \reserved@a
      \@gobble}%
    \fi%
  \endgroup
\fi
%    \end{macrocode}
%
% \subsection{Attribute setting}
%
%    \begin{macro}{\LuaCol@Attribute}
%    \begin{macrocode}
\newattribute\LuaCol@Attribute
\let\LuaCol@setattribute\setattribute
\directlua{%
  oberdiek.luacolor.setattribute(\number\allocationnumber)%
}
%    \end{macrocode}
%    \end{macro}
%
%    \begin{macro}{\set@color}
% change 2023-08-18: added \cs{reset@color} so that \cs{mathcolor} can gobble it,
% issue 7.   
%    \begin{macrocode}
\protected\def\set@color{%
  \LuaCol@setattribute\LuaCol@Attribute{%
    \directlua{%
      oberdiek.luacolor.get("\luaescapestring{\current@color}")%
    }%
  }%
 \aftergroup\reset@color 
}
%    \end{macrocode}
%    \end{macro}
%    \begin{macro}{\reset@color}
%    \begin{macrocode}
\def\reset@color{}
%    \end{macrocode}
%    \end{macro}
%
% \subsection{Whatsit insertion}
%
%    \begin{macro}{\luacolorProcessBox}
%    \begin{macrocode}
\def\luacolorProcessBox#1{%
  \directlua{%
    oberdiek.luacolor.process(\number#1)%
  }%
}
%    \end{macrocode}
%    \end{macro}
%
%    \begin{macrocode}
\directlua{%
  if luatexbase.callbacktypes.pre_shipout_filter then
    token.get_next()
  end
}\@secondoftwo\@gobble{
  \RequirePackage{atbegshi}[2011/01/30]
  \AtBeginShipout{%
    \luacolorProcessBox\AtBeginShipoutBox
  }
}
%    \end{macrocode}
%
%    Set default color.
%    \begin{macrocode}
\set@color
%    \end{macrocode}
%
% \subsection{\cs{pdfxform}/\cs{saveboxresource} support}
%
%    \begin{macrocode}
\ifnum\outputmode=\@ne
    \let\LuaCol@org@pdfxform\saveboxresource
%    \end{macrocode}
% First we need some helpers to allow expandable code to parse keyword style arguments:
%    \begin{macrocode}
    \def\LuaCol@iii@i@ii#1#2#3{#3{#1}{#2}}
    \def\LuaCol@ii@i#1#2{{#2#1}}
    \def\LuaCol@if@keyword#1#2#3{%
      \expanded{\unexpanded{\LuaCol@iii@i@ii{#2}{#3}}\expandafter}%
      \directlua{%
        token.put_next(token.create(token.scan_keyword(token.scan_string())
        and '@firstoftwo'
        or '@secondoftwo'))
      }{#1}%
    }
%    \end{macrocode}
% The following macro scans a integer and expands to a token equivalent to a chardef
% whose value corresponds to the scanned integer. This allows the integer to be passed
% around as a undelimited argument.
%    \begin{macrocode}
    \def\LuaCol@scan@number{%
      \directlua{
        token.put_next(token.new(token.scan_int(), token.command_id'char_given'))
      }%
    }
%    \end{macrocode}
% \TeX\ primitives like \cs{saveboxresource} read braced arguments in a special way.
% Especially they expand everything until they find a left brace. To simulate this, we
% use Lua to expand everything else:
%    \begin{macrocode}
    \def\LuaCol@scan@tobrace{%
      \directlua{
        local relax, space = token.command_id'relax', token.command_id'spacer'
        local t
        repeat
          t = token.scan_token()
        until not (t.command == relax or t.command == space)
        token.put_next(t)
      }%
    }
    \def\LuaCol@scan@boxresource@i#1#2{%
      \LuaCol@if@keyword{attr}{%
        \expanded{\unexpanded{\LuaCol@scan@boxresource@iI{#1#2attr}}%
          \expandafter\expandafter\expandafter}%
        \LuaCol@scan@tobrace
      }{%
        \LuaCol@scan@boxresource@ii{#1#2}%
      }%
    }
    \def\LuaCol@scan@boxresource@iI#1#2{\LuaCol@scan@boxresource@ii{#1{#2}}}
    \def\LuaCol@scan@boxresource@ii#1{%
      \LuaCol@if@keyword{resources}{%
        \expanded{\unexpanded{\LuaCol@scan@boxresource@iiI{#1resources}}%
          \expandafter\expandafter\expandafter}%
        \LuaCol@scan@tobrace
      }{%
        \LuaCol@scan@boxresource@iii{#1}%
      }%
    }
    \def\LuaCol@scan@boxresource@iiI#1#2{\LuaCol@scan@boxresource@iii{#1{#2}}}
    \def\LuaCol@scan@boxresource@iii#1{%
      \LuaCol@if@keyword{margin}{%
        \expanded{\unexpanded{\LuaCol@scan@boxresource@iv{#1margin }}%
          \expandafter\expandafter\expandafter}%
        \LuaCol@scan@number
      }{%
        \LuaCol@scan@boxresource@iv{#1}{}%
      }%
    }
    \def\LuaCol@scan@boxresource@iv#1#2{%
      \expanded{\unexpanded{\LuaCol@scan@boxresource@v{#1#2}}%
        \expandafter\expandafter\expandafter}%
      \LuaCol@scan@number
    }
    \def\LuaCol@scan@boxresource@v#1#2{%
      \luacolorProcessBox{#2}%
      \LuaCol@org@pdfxform#1#2%
    }

%    \end{macrocode}
% This could be written in Lua, but at least upto Lua\TeX~1.11, feeding back too many tokens
% from Lua to \TeX\ triggers a segmentation fault.
% This is written in Lua so the integer setting is expandable and does not interfere with
% a preceding \verb|\immediate|.
%    \begin{macrocode}
    \protected\def\saveboxresource{%
      \LuaCol@if@keyword{type}{%
        \expandafter
        \expanded{\unexpanded{\LuaCol@scan@boxresource@i{type }}%
          \expandafter\expandafter\expandafter}%
        \LuaCol@scan@number
      }{%
        \LuaCol@scan@boxresource@i{}{}%
      }%
    }
%    \end{macrocode}
%    Legacy alias.
%    \begin{macrocode}
    \let\pdfxform\saveboxresource
\fi
%    \end{macrocode}
%
%    \begin{macrocode}
\LuaCol@AtEnd%
%</package>
%    \end{macrocode}
%
% \subsection{Lua module}
%
%    \begin{macrocode}
%<*lua>
%    \end{macrocode}
%    Box zero contains a \cs{hbox} with the color \cs{special}.
%    That is analyzed to get the prefix for the color setting
%    \cs{special}.
%    \begin{macrocode}
oberdiek = oberdiek or {}
local luacolor = oberdiek.luacolor or {}
oberdiek.luacolor = luacolor
%    \end{macrocode}
%    \begin{macro}{getversion()}
%    \begin{macrocode}
function luacolor.getversion()
  tex.write("2023-08-18 v1.18")
end
%    \end{macrocode}
%    \end{macro}
%
% \subsubsection{Driver detection}
%
%    \begin{macrocode}
local ifpdf = tonumber(tex.outputmode or tex.pdfoutput) > 0
local prefix
local prefixes = {
  dvips   = "color ",
  dvipdfm = "pdf:sc ",
  truetex = "textcolor:",
  pctexps = "ps::",
}
local patterns = {
  ["^color "]            = "dvips",
  ["^pdf: *begincolor "] = "dvipdfm",
  ["^pdf: *bcolor "]     = "dvipdfm",
  ["^pdf: *bc "]         = "dvipdfm",
  ["^pdf: *setcolor "]   = "dvipdfm",
  ["^pdf: *scolor "]     = "dvipdfm",
  ["^pdf: *sc "]         = "dvipdfm",
  ["^textcolor:"]        = "truetex",
  ["^ps::"]              = "pctexps",
}
%    \end{macrocode}
%    \begin{macro}{info()}
%    \begin{macrocode}
local function info(msg, term)
  local target = "log"
  if term then
    target = "term and log"
  end
  texio.write_nl(target, "Package luacolor info: " .. msg .. ".")
  texio.write_nl(target, "")
end
%    \end{macrocode}
%    \end{macro}
%    \begin{macro}{dvidetect()}
%    \begin{macrocode}
function luacolor.dvidetect()
  local v = tex.box[0]
  assert(v.id == node.id("hlist"))
  for v in node.traverse_id(node.id("whatsit"), v.head) do
    if v and v.subtype == node.subtype("special") then
      local data = v.data
      for pattern, driver in pairs(patterns) do
        if string.find(data, pattern) then
          prefix = prefixes[driver]
          tex.write(driver)
          return
        end
      end
      info("\\special{" .. data .. "}", true)
      return
    end
  end
  info("Missing \\special", true)
end
%    \end{macrocode}
%    \end{macro}
%
% \subsubsection{Color strings}
%
%    \begin{macrocode}
local map = {
  n = 0,
}
%    \end{macrocode}
%    \begin{macro}{get()}
%    \begin{macrocode}
function luacolor.get(color)
  tex.write("" .. luacolor.getvalue(color))
end
%    \end{macrocode}
%    \end{macro}
%    \begin{macro}{getvalue()}
%    \begin{macrocode}
function luacolor.getvalue(color)
  local n = map[color]
  if not n then
    n = map.n + 1
    map.n = n
    map[n] = color
    map[color] = n
  end
  return n
end
%    \end{macrocode}
%    \end{macro}
%
% \subsubsection{Attribute register}
%
%    \begin{macro}{setattribute()}
%    \begin{macrocode}
local attribute
function luacolor.setattribute(attr)
  attribute = attr
end
%    \end{macrocode}
%    \end{macro}
%
%    \begin{macro}{getattribute()}
%    \begin{macrocode}
function luacolor.getattribute()
  return attribute
end
%    \end{macrocode}
%    \end{macro}
%
% \subsubsection{Whatsit insertion}
%
%    \begin{macrocode}
local LIST = 1
local LIST_LEADERS = 2
local LIST_DISC = 3
local COLOR = 4
local NOCOLOR = 5
local RULE = node.id("rule")
local node_types = {
  [node.id("hlist")] = LIST,
  [node.id("vlist")] = LIST,
  [node.id("rule")]  = COLOR,
  [node.id("glyph")] = COLOR,
  [node.id("disc")]  = LIST_DISC,
  [node.id("whatsit")] = {
    [node.subtype("pdf_colorstack")] =
      function(n)
        return n.stack == 0 and NOCOLOR or nil
      end,
    [node.subtype("special")] = COLOR,
    [node.subtype("pdf_literal")] = COLOR,
    [node.subtype("pdf_save")] = COLOR,
    [node.subtype("pdf_restore")] = COLOR, -- probably not needed
-- TODO (DPC)    [node.subtype("pdf_refximage")] = COLOR,
  },
  [node.id("glue")] =
    function(n)
      if n.subtype >= 100 then -- leaders
        if n.leader.id == RULE then
          return COLOR
        else
          return LIST_LEADERS
        end
      end
    end,
}
%    \end{macrocode}
%    \begin{macro}{get_type()}
%    \begin{macrocode}
local function get_type(n)
  local ret = node_types[n.id]
  if type(ret) == 'table' then
    ret = ret[n.subtype]
  end
  if type(ret) == 'function' then
    ret = ret(n)
  end
  return ret
end
%    \end{macrocode}
%    \end{macro}
%    \begin{macrocode}
local mode = 2 -- luatex.pdfliteral.direct
local WHATSIT = node.id("whatsit")
local SPECIAL = node.subtype("special")
local PDFLITERAL = node.subtype("pdf_literal")
local DRY_FALSE = false
local DRY_TRUE = true
%    \end{macrocode}
%    \begin{macro}{traverse()}
%    \begin{macrocode}
local function traverse(list, color, dry)
  if not list then
    return color
  end
  local head
  if get_type(list) == LIST then
    head = list.head
  elseif get_type(list) == LIST_DISC then
    head = list.replace
  else
    texio.write_nl("!!! Error: Wrong list type: " .. node.type(list.id))
    return color
  end
%<debug>texio.write_nl("traverse: " .. node.type(list.id))
  for n in node.traverse(head) do
%<debug>texio.write_nl("  node: " .. node.type(n.id))
    local t = get_type(n)
%<debug>texio.write_nl("TYPE "..tostring(t).. " "..tostring(node.type(node.getid(n))).." ".. tostring(node.getsubtype(n)))
    if t == LIST or t == LIST_DISC then
      color = traverse(n, color, dry)
    elseif t == LIST_LEADERS then
      local color_after = traverse(n.leader, color, DRY_TRUE)
      if color == color_after then
        traverse(n.leader, color, DRY_FALSE or dry)
      else
        traverse(n.leader, '', DRY_FALSE or dry)
%    \end{macrocode}
% The color status is unknown here, because the leader box
% will or will not be set.
%    \begin{macrocode}
        color = ''
      end
    elseif t == COLOR then
      local v = node.has_attribute(n, attribute)
      if v then
        local newColor = map[v]
        if newColor ~= color then
          color = newColor
          if dry == DRY_FALSE then
            local newNode
            if ifpdf then
              newNode = node.new(WHATSIT, PDFLITERAL)
              newNode.mode = mode
              newNode.data = color
            else
              newNode = node.new(WHATSIT, SPECIAL)
              newNode.data = prefix .. color
            end
            head = node.insert_before(head, n, newNode)
          end
        end
      end
    elseif t == NOCOLOR then
      color = ''
    end
  end
  if get_type(list) == LIST then
    list.head = head
  else
    list.replace = head
  end
  return color
end
%    \end{macrocode}
%    \end{macro}
%
%    \begin{macro}{process()}
%    \begin{macrocode}
function luacolor.process(box)
  local color = ""
  local list = tex.getbox(box)
  traverse(list, color, DRY_FALSE)
end

if luatexbase.callbacktypes.pre_shipout_filter then
  luatexbase.add_to_callback("pre_shipout_filter", function(list)
    traverse(list, "", DRY_FALSE)
    return true
  end, "luacolor.process")
end
%    \end{macrocode}
%    \end{macro}
% For recent versions of luaotfload, we can register a callback to
% control how coloring glyph is handled for the color feature.
%    \begin{macrocode}
if luaotfload.set_colorhandler then
  local set_attribute = node.direct.set_attribute
  luaotfload.set_colorhandler(function(head, n, color)
    set_attribute(n, attribute, luacolor.getvalue(color))
    return head, n
  end)
end
%    \end{macrocode}
%
%    \begin{macrocode}
%</lua>
%    \end{macrocode}
% \section{Installation}
%
% \subsection{Download}
%
% \paragraph{Package.} This package is available on
% CTAN\footnote{\CTANpkg{luacolor}}:
% \begin{description}
% \item[\CTAN{macros/latex/contrib/luacolor/luacolor.dtx}] The source file.
% \item[\CTAN{macros/latex/contrib/luacolor/luacolor.pdf}] Documentation.
% \end{description}
%
%
% \paragraph{Bundle.} All the packages of the bundle `luacolor'
% are also available in a TDS compliant ZIP archive. There
% the packages are already unpacked and the documentation files
% are generated. The files and directories obey the TDS standard.
% \begin{description}
% \item[\CTANinstall{install/macros/latex/contrib/luacolor.tds.zip}]
% \end{description}
% \emph{TDS} refers to the standard ``A Directory Structure
% for \TeX\ Files'' (\CTANpkg{tds}). Directories
% with \xfile{texmf} in their name are usually organized this way.
%
% \subsection{Bundle installation}
%
% \paragraph{Unpacking.} Unpack the \xfile{luacolor.tds.zip} in the
% TDS tree (also known as \xfile{texmf} tree) of your choice.
% Example (linux):
% \begin{quote}
%   |unzip luacolor.tds.zip -d ~/texmf|
% \end{quote}
%
% \paragraph{Script installation.}
% Check the directory \xfile{TDS:scripts/luacolor/} for
% scripts that need further installation steps.
%
% \subsection{Package installation}
%
% \paragraph{Unpacking.} The \xfile{.dtx} file is a self-extracting
% \docstrip\ archive. The files are extracted by running the
% \xfile{.dtx} through \plainTeX:
% \begin{quote}
%   \verb|tex luacolor.dtx|
% \end{quote}
%
% \paragraph{TDS.} Now the different files must be moved into
% the different directories in your installation TDS tree
% (also known as \xfile{texmf} tree):
% \begin{quote}
% \def\t{^^A
% \begin{tabular}{@{}>{\ttfamily}l@{ $\rightarrow$ }>{\ttfamily}l@{}}
%   luacolor.sty & tex/latex/luacolor/luacolor.sty\\
%   luacolor.lua & scripts/luacolor/luacolor.lua\\
%   luacolor.pdf & doc/latex/luacolor/luacolor.pdf\\
%   luacolor.dtx & source/latex/luacolor/luacolor.dtx\\
% \end{tabular}^^A
% }^^A
% \sbox0{\t}^^A
% \ifdim\wd0>\linewidth
%   \begingroup
%     \advance\linewidth by\leftmargin
%     \advance\linewidth by\rightmargin
%   \edef\x{\endgroup
%     \def\noexpand\lw{\the\linewidth}^^A
%   }\x
%   \def\lwbox{^^A
%     \leavevmode
%     \hbox to \linewidth{^^A
%       \kern-\leftmargin\relax
%       \hss
%       \usebox0
%       \hss
%       \kern-\rightmargin\relax
%     }^^A
%   }^^A
%   \ifdim\wd0>\lw
%     \sbox0{\small\t}^^A
%     \ifdim\wd0>\linewidth
%       \ifdim\wd0>\lw
%         \sbox0{\footnotesize\t}^^A
%         \ifdim\wd0>\linewidth
%           \ifdim\wd0>\lw
%             \sbox0{\scriptsize\t}^^A
%             \ifdim\wd0>\linewidth
%               \ifdim\wd0>\lw
%                 \sbox0{\tiny\t}^^A
%                 \ifdim\wd0>\linewidth
%                   \lwbox
%                 \else
%                   \usebox0
%                 \fi
%               \else
%                 \lwbox
%               \fi
%             \else
%               \usebox0
%             \fi
%           \else
%             \lwbox
%           \fi
%         \else
%           \usebox0
%         \fi
%       \else
%         \lwbox
%       \fi
%     \else
%       \usebox0
%     \fi
%   \else
%     \lwbox
%   \fi
% \else
%   \usebox0
% \fi
% \end{quote}
% If you have a \xfile{docstrip.cfg} that configures and enables \docstrip's
% TDS installing feature, then some files can already be in the right
% place, see the documentation of \docstrip.
%
% \subsection{Refresh file name databases}
%
% If your \TeX~distribution
% (\TeX\,Live, \mikTeX, \dots) relies on file name databases, you must refresh
% these. For example, \TeX\,Live\ users run \verb|texhash| or
% \verb|mktexlsr|.
%
% \subsection{Some details for the interested}
%
% \paragraph{Unpacking with \LaTeX.}
% The \xfile{.dtx} chooses its action depending on the format:
% \begin{description}
% \item[\plainTeX:] Run \docstrip\ and extract the files.
% \item[\LaTeX:] Generate the documentation.
% \end{description}
% If you insist on using \LaTeX\ for \docstrip\ (really,
% \docstrip\ does not need \LaTeX), then inform the autodetect routine
% about your intention:
% \begin{quote}
%   \verb|latex \let\install=y% \iffalse meta-comment
%
% File: luacolor.dtx
% Version: 2023-08-18 v1.18
% Info: Color support via LuaTeX's attributes
%
% Copyright (C)
%    2007, 2009-2011 Heiko Oberdiek
%    2016-2023 Oberdiek Package Support Group
%    https://github.com/ho-tex/luacolor/issues
%
% This work may be distributed and/or modified under the
% conditions of the LaTeX Project Public License, either
% version 1.3c of this license or (at your option) any later
% version. This version of this license is in
%    https://www.latex-project.org/lppl/lppl-1-3c.txt
% and the latest version of this license is in
%    https://www.latex-project.org/lppl.txt
% and version 1.3 or later is part of all distributions of
% LaTeX version 2005/12/01 or later.
%
% This work has the LPPL maintenance status "maintained".
%
% The Current Maintainers of this work are
% Heiko Oberdiek and the Oberdiek Package Support Group
% https://github.com/ho-tex/luacolor/issues
%
% This work consists of the main source file luacolor.dtx
% and the derived files
%    luacolor.sty, luacolor.pdf, luacolor.lua,
%
% Distribution:
%    CTAN:macros/latex/contrib/luacolor/luacolor.dtx
%    CTAN:macros/latex/contrib/luacolor/luacolor.pdf
%
% Unpacking:
%    (a) If luacolor.ins is present:
%           tex luacolor.ins
%    (b) Without luacolor.ins:
%           tex luacolor.dtx
%    (c) If you insist on using LaTeX
%           latex \let\install=y% \iffalse meta-comment
%
% File: luacolor.dtx
% Version: 2023-08-18 v1.18
% Info: Color support via LuaTeX's attributes
%
% Copyright (C)
%    2007, 2009-2011 Heiko Oberdiek
%    2016-2023 Oberdiek Package Support Group
%    https://github.com/ho-tex/luacolor/issues
%
% This work may be distributed and/or modified under the
% conditions of the LaTeX Project Public License, either
% version 1.3c of this license or (at your option) any later
% version. This version of this license is in
%    https://www.latex-project.org/lppl/lppl-1-3c.txt
% and the latest version of this license is in
%    https://www.latex-project.org/lppl.txt
% and version 1.3 or later is part of all distributions of
% LaTeX version 2005/12/01 or later.
%
% This work has the LPPL maintenance status "maintained".
%
% The Current Maintainers of this work are
% Heiko Oberdiek and the Oberdiek Package Support Group
% https://github.com/ho-tex/luacolor/issues
%
% This work consists of the main source file luacolor.dtx
% and the derived files
%    luacolor.sty, luacolor.pdf, luacolor.lua,
%
% Distribution:
%    CTAN:macros/latex/contrib/luacolor/luacolor.dtx
%    CTAN:macros/latex/contrib/luacolor/luacolor.pdf
%
% Unpacking:
%    (a) If luacolor.ins is present:
%           tex luacolor.ins
%    (b) Without luacolor.ins:
%           tex luacolor.dtx
%    (c) If you insist on using LaTeX
%           latex \let\install=y\input{luacolor.dtx}
%        (quote the arguments according to the demands of your shell)
%
% Documentation:
%    (a) If luacolor.drv is present:
%           latex luacolor.drv
%    (b) Without luacolor.drv:
%           latex luacolor.dtx; ...
%    The class ltxdoc loads the configuration file ltxdoc.cfg
%    if available. Here you can specify further options, e.g.
%    use A4 as paper format:
%       \PassOptionsToClass{a4paper}{article}
%
%    Programm calls to get the documentation (example):
%       pdflatex luacolor.dtx
%       makeindex -s gind.ist luacolor.idx
%       pdflatex luacolor.dtx
%       makeindex -s gind.ist luacolor.idx
%       pdflatex luacolor.dtx
%
% Installation:
%    TDS:tex/latex/luacolor/luacolor.sty
%    TDS:scripts/luacolor/luacolor.lua
%    TDS:doc/latex/luacolor/luacolor.pdf
%    TDS:source/latex/luacolor/luacolor.dtx
%
%<*ignore>
\begingroup
  \catcode123=1 %
  \catcode125=2 %
  \def\x{LaTeX2e}%
\expandafter\endgroup
\ifcase 0\ifx\install y1\fi\expandafter
         \ifx\csname processbatchFile\endcsname\relax\else1\fi
         \ifx\fmtname\x\else 1\fi\relax
\else\csname fi\endcsname
%</ignore>
%<*install>
\input docstrip.tex
\Msg{************************************************************************}
\Msg{* Installation}
\Msg{* Package: luacolor 2023-08-18 v1.18 Color support via LuaTeX's attributes (HO)}
\Msg{************************************************************************}

\keepsilent
\askforoverwritefalse

\let\MetaPrefix\relax
\preamble

This is a generated file.

Project: luacolor
Version: 2023-08-18 v1.18

Copyright (C)
   2007, 2009-2011 Heiko Oberdiek
   2016-2023 Oberdiek Package Support Group

This work may be distributed and/or modified under the
conditions of the LaTeX Project Public License, either
version 1.3c of this license or (at your option) any later
version. This version of this license is in
   https://www.latex-project.org/lppl/lppl-1-3c.txt
and the latest version of this license is in
   https://www.latex-project.org/lppl.txt
and version 1.3 or later is part of all distributions of
LaTeX version 2005/12/01 or later.

This work has the LPPL maintenance status "maintained".

The Current Maintainers of this work are
Heiko Oberdiek and the Oberdiek Package Support Group
https://github.com/ho-tex/luacolor/issues


This work consists of the main source file luacolor.dtx
and the derived files
   luacolor.sty, luacolor.pdf, luacolor.ins, luacolor.drv,
   luacolor.lua,

\endpreamble
\let\MetaPrefix\DoubleperCent

\generate{%
  \file{luacolor.ins}{\from{luacolor.dtx}{install}}%
  \file{luacolor.drv}{\from{luacolor.dtx}{driver}}%
  \usedir{tex/latex/luacolor}%
  \file{luacolor.sty}{\from{luacolor.dtx}{package}}%
}
\def\MetaPrefix{-- }
\def\defaultpostamble{%
  \MetaPrefix^^J%
  \MetaPrefix\space End of File `\outFileName'.%
}
\def\currentpostamble{\defaultpostamble}%
\generate{%
  \usedir{scripts/luacolor}%
%  \file{oberdiek.luacolor.lua}{\from{luacolor.dtx}{lua}}%
  \file{luacolor.lua}{\from{luacolor.dtx}{lua}}%
}

\catcode32=13\relax% active space
\let =\space%
\Msg{************************************************************************}
\Msg{*}
\Msg{* To finish the installation you have to move the following}
\Msg{* file into a directory searched by TeX:}
\Msg{*}
\Msg{*     luacolor.sty}
\Msg{*}
\Msg{* And install the following script file:}
\Msg{*}
\Msg{*     luacolor.lua,}
\Msg{*}
\Msg{* To produce the documentation run the file `luacolor.dtx'}
\Msg{* through LaTeX.}
\Msg{*}
\Msg{* Happy TeXing!}
\Msg{*}
\Msg{************************************************************************}

\endbatchfile
%</install>
%<*ignore>
\fi
%</ignore>
%<*driver>
\NeedsTeXFormat{LaTeX2e}
\ProvidesFile{luacolor.drv}%
  [2023-08-18 v1.18 Color support via LuaTeX's attributes (HO)]%
\documentclass{ltxdoc}
\usepackage{holtxdoc}[2011/11/22]
\usepackage{paralist}
\makeatletter
\g@addto@macro\MakePrivateLetters{%
  \@makeother\_%
}
\makeatother
\begin{document}
  \DocInput{luacolor.dtx}%
\end{document}
%</driver>
% \fi
%
%
%
% \GetFileInfo{luacolor.drv}
%
% \title{The \xpackage{luacolor} package}
% \date{2023-08-18 v1.18}
% \author{Heiko Oberdiek\thanks
% {Please report any issues at \url{https://github.com/ho-tex/luacolor/issues}}}
%
% \maketitle
%
% \begin{abstract}
% Package \xpackage{luacolor} implements color support based
% on \LuaTeX's node attributes.
% \end{abstract}
%
% \tableofcontents
%
% \section{Documentation}
%
% \subsection{Introduction}
%
% This package uses a \LuaTeX's attribute register to
% to annotate nodes with color information.
% If a color is set, then the attribute register
% is set to this color and all nodes created in its scope
% (current group) are annotated with this attribute.
% Now the color property behaves much the same way
% as the font property.
%
% \subsection{Usage}
%
% Package \xpackage{color} is loaded automatically by this
% package \xpackage{luacolor}. If you need a special driver
% option or you prefer package \xpackage{xcolor}, then load
% it before package \xpackage{luacolor}, for example:
% \begin{quote}
%   |\usepackage[dvipdfmx]{xcolor}|
% \end{quote}
% The package \xpackage{luacolor} is loaded without options:
% \begin{quote}
%   |\usepackage{luacolor}|
% \end{quote}
% It is able to detect PDF mode and DVI drivers are
% differentiated by its color specials. Therefore the
% package do need driver options.
%
% Then it redefines the color setting commands to set
% attributes instead of whatsits for color.
%
% At last the attribute annotations of the nodes in the
% output box must be analyzed to insert the necessary color whatsits.
% (With older \LuaTeX\ that lack the appropriate callback function 
% the package \xpackage{atbegshi} is used to get control
% before a box is shipped out.)
%
% \begin{declcs}{luacolorProcessBox} \M{box}
% \end{declcs}
% Macro \cs{luacolorProcessBox} processes the box \meta{box}
% in the previously described manner. It is automatically
% called for pages, but not for XForm objects.
% Before passing a box to \cs{pdfxform}, call \cs{luacolorProcessBox}
% first.
%
% \subsection{Limitations}
%
% \begin{description}
% \item[Ligatures with different colored components:]
%   Package \xpackage{luacolor} sees the ligature after
%   the paragraph building and page breaking, when a
%   page is to be shipped out. Therefore it cannot
%   break ligatures, because the components might occupy
%   different space. Therefore it is the responsibility
%   of the ligature forming process to deal with different
%   colored glyphs that form a ligature. The user can avoid
%   the problem entirely by explicitly breaking the ligature
%   at the places where the color changes.
%  \item \dots
% \end{description}
%
% \StopEventually{
% }
%
% \section{Implementation}
%
%    \begin{macrocode}
%<*package>
%    \end{macrocode}
%
% \subsection{Catcodes and identification}
%
%    \begin{macrocode}
\begingroup\catcode61\catcode48\catcode32=10\relax%
  \catcode13=5 % ^^M
  \endlinechar=13 %
  \catcode123=1 % {
  \catcode125=2 % }
  \catcode64=11 % @
  \def\x{\endgroup
    \expandafter\edef\csname LuaCol@AtEnd\endcsname{%
      \endlinechar=\the\endlinechar\relax
      \catcode13=\the\catcode13\relax
      \catcode32=\the\catcode32\relax
      \catcode35=\the\catcode35\relax
      \catcode61=\the\catcode61\relax
      \catcode64=\the\catcode64\relax
      \catcode123=\the\catcode123\relax
      \catcode125=\the\catcode125\relax
    }%
  }%
\x\catcode61\catcode48\catcode32=10\relax%
\catcode13=5 % ^^M
\endlinechar=13 %
\catcode35=6 % #
\catcode64=11 % @
\catcode123=1 % {
\catcode125=2 % }
\def\TMP@EnsureCode#1#2{%
  \edef\LuaCol@AtEnd{%
    \LuaCol@AtEnd
    \catcode#1=\the\catcode#1\relax
  }%
  \catcode#1=#2\relax
}
\TMP@EnsureCode{34}{12}% "
\TMP@EnsureCode{39}{12}% '
\TMP@EnsureCode{40}{12}% (
\TMP@EnsureCode{41}{12}% )
\TMP@EnsureCode{42}{12}% *
\TMP@EnsureCode{43}{12}% +
\TMP@EnsureCode{44}{12}% ,
\TMP@EnsureCode{45}{12}% -
\TMP@EnsureCode{46}{12}% .
\TMP@EnsureCode{47}{12}% /
\TMP@EnsureCode{58}{12}% :
\TMP@EnsureCode{60}{12}% <
\TMP@EnsureCode{62}{12}% >
\TMP@EnsureCode{91}{12}% [
\TMP@EnsureCode{93}{12}% ]
\TMP@EnsureCode{95}{12}% _ (other!)
\TMP@EnsureCode{96}{12}% `
\edef\LuaCol@AtEnd{\LuaCol@AtEnd\noexpand\endinput}
%    \end{macrocode}
%
%    Package identification.
%    \begin{macrocode}
\NeedsTeXFormat{LaTeX2e}
\ProvidesPackage{luacolor}%
  [2023-08-18 v1.18 Color support via LuaTeX's attributes (HO)]
%    \end{macrocode}
%
% \subsection{Check for \LuaTeX}
%
%    Without \LuaTeX\ there is no point in using this package.
%    \begin{macrocode}
\RequirePackage{color}
%    \end{macrocode}
%
%    \begin{macrocode}
\ifx\directlua\@undefined
  \PackageError{luacolor}{%
    This package may only be run using LuaTeX%
  }\@ehc
  \expandafter\LuaCol@AtEnd
\fi%
%    \end{macrocode}
%
% \subsection{Check for disabled colors}
%
%    \begin{macrocode}
\ifcolors@
\else
  \PackageWarningNoLine{luacolor}{%
    Colors are disabled by option `monochrome'%
  }%
  \def\set@color{}%
  \def\reset@color{}%
  \def\set@page@color{}%
  \def\define@color#1#2{}%
  \expandafter\LuaCol@AtEnd
\fi%
%    \end{macrocode}
%
% \subsection{Load module and check version}
%
%    \begin{macrocode}
\directlua{%
  require("luacolor")%
}
%    \end{macrocode}
%    \begin{macrocode}
\begingroup
  \edef\x{\directlua{tex.write("2023-08-18 v1.18")}}%
  \edef\y{%
    \directlua{%
      if oberdiek.luacolor.getversion then %
        oberdiek.luacolor.getversion()%
      end%
    }%
  }%
  \ifx\x\y
  \else
    \PackageError{luacolor}{%
      Wrong version of lua module.\MessageBreak
      Package version: \x\MessageBreak
      Lua module: \y
    }\@ehc
  \fi
\endgroup
%    \end{macrocode}
%
% \subsection{Find driver}
%
%    \begin{macrocode}
\ifnum\outputmode=\@ne
\else
  \begingroup
    \def\current@color{}%
    \def\reset@color{}%
    \setbox\z@=\hbox{%
      \begingroup
        \set@color
      \endgroup
    }%
    \edef\reserved@a{%
      \directlua{%
        oberdiek.luacolor.dvidetect()%
      }%
    }%
    \ifx\reserved@a\@empty
      \PackageError{luacolor}{%
        DVI driver detection failed because of\MessageBreak
        unrecognized color \string\special
      }\@ehc
      \endgroup
      \expandafter\expandafter\expandafter\LuaCol@AtEnd
    \else
      \PackageInfo{luacolor}{%
        Type of color \string\special: \reserved@a
      \@gobble}%
    \fi%
  \endgroup
\fi
%    \end{macrocode}
%
% \subsection{Attribute setting}
%
%    \begin{macro}{\LuaCol@Attribute}
%    \begin{macrocode}
\newattribute\LuaCol@Attribute
\let\LuaCol@setattribute\setattribute
\directlua{%
  oberdiek.luacolor.setattribute(\number\allocationnumber)%
}
%    \end{macrocode}
%    \end{macro}
%
%    \begin{macro}{\set@color}
% change 2023-08-18: added \cs{reset@color} so that \cs{mathcolor} can gobble it,
% issue 7.   
%    \begin{macrocode}
\protected\def\set@color{%
  \LuaCol@setattribute\LuaCol@Attribute{%
    \directlua{%
      oberdiek.luacolor.get("\luaescapestring{\current@color}")%
    }%
  }%
 \aftergroup\reset@color 
}
%    \end{macrocode}
%    \end{macro}
%    \begin{macro}{\reset@color}
%    \begin{macrocode}
\def\reset@color{}
%    \end{macrocode}
%    \end{macro}
%
% \subsection{Whatsit insertion}
%
%    \begin{macro}{\luacolorProcessBox}
%    \begin{macrocode}
\def\luacolorProcessBox#1{%
  \directlua{%
    oberdiek.luacolor.process(\number#1)%
  }%
}
%    \end{macrocode}
%    \end{macro}
%
%    \begin{macrocode}
\directlua{%
  if luatexbase.callbacktypes.pre_shipout_filter then
    token.get_next()
  end
}\@secondoftwo\@gobble{
  \RequirePackage{atbegshi}[2011/01/30]
  \AtBeginShipout{%
    \luacolorProcessBox\AtBeginShipoutBox
  }
}
%    \end{macrocode}
%
%    Set default color.
%    \begin{macrocode}
\set@color
%    \end{macrocode}
%
% \subsection{\cs{pdfxform}/\cs{saveboxresource} support}
%
%    \begin{macrocode}
\ifnum\outputmode=\@ne
    \let\LuaCol@org@pdfxform\saveboxresource
%    \end{macrocode}
% First we need some helpers to allow expandable code to parse keyword style arguments:
%    \begin{macrocode}
    \def\LuaCol@iii@i@ii#1#2#3{#3{#1}{#2}}
    \def\LuaCol@ii@i#1#2{{#2#1}}
    \def\LuaCol@if@keyword#1#2#3{%
      \expanded{\unexpanded{\LuaCol@iii@i@ii{#2}{#3}}\expandafter}%
      \directlua{%
        token.put_next(token.create(token.scan_keyword(token.scan_string())
        and '@firstoftwo'
        or '@secondoftwo'))
      }{#1}%
    }
%    \end{macrocode}
% The following macro scans a integer and expands to a token equivalent to a chardef
% whose value corresponds to the scanned integer. This allows the integer to be passed
% around as a undelimited argument.
%    \begin{macrocode}
    \def\LuaCol@scan@number{%
      \directlua{
        token.put_next(token.new(token.scan_int(), token.command_id'char_given'))
      }%
    }
%    \end{macrocode}
% \TeX\ primitives like \cs{saveboxresource} read braced arguments in a special way.
% Especially they expand everything until they find a left brace. To simulate this, we
% use Lua to expand everything else:
%    \begin{macrocode}
    \def\LuaCol@scan@tobrace{%
      \directlua{
        local relax, space = token.command_id'relax', token.command_id'spacer'
        local t
        repeat
          t = token.scan_token()
        until not (t.command == relax or t.command == space)
        token.put_next(t)
      }%
    }
    \def\LuaCol@scan@boxresource@i#1#2{%
      \LuaCol@if@keyword{attr}{%
        \expanded{\unexpanded{\LuaCol@scan@boxresource@iI{#1#2attr}}%
          \expandafter\expandafter\expandafter}%
        \LuaCol@scan@tobrace
      }{%
        \LuaCol@scan@boxresource@ii{#1#2}%
      }%
    }
    \def\LuaCol@scan@boxresource@iI#1#2{\LuaCol@scan@boxresource@ii{#1{#2}}}
    \def\LuaCol@scan@boxresource@ii#1{%
      \LuaCol@if@keyword{resources}{%
        \expanded{\unexpanded{\LuaCol@scan@boxresource@iiI{#1resources}}%
          \expandafter\expandafter\expandafter}%
        \LuaCol@scan@tobrace
      }{%
        \LuaCol@scan@boxresource@iii{#1}%
      }%
    }
    \def\LuaCol@scan@boxresource@iiI#1#2{\LuaCol@scan@boxresource@iii{#1{#2}}}
    \def\LuaCol@scan@boxresource@iii#1{%
      \LuaCol@if@keyword{margin}{%
        \expanded{\unexpanded{\LuaCol@scan@boxresource@iv{#1margin }}%
          \expandafter\expandafter\expandafter}%
        \LuaCol@scan@number
      }{%
        \LuaCol@scan@boxresource@iv{#1}{}%
      }%
    }
    \def\LuaCol@scan@boxresource@iv#1#2{%
      \expanded{\unexpanded{\LuaCol@scan@boxresource@v{#1#2}}%
        \expandafter\expandafter\expandafter}%
      \LuaCol@scan@number
    }
    \def\LuaCol@scan@boxresource@v#1#2{%
      \luacolorProcessBox{#2}%
      \LuaCol@org@pdfxform#1#2%
    }

%    \end{macrocode}
% This could be written in Lua, but at least upto Lua\TeX~1.11, feeding back too many tokens
% from Lua to \TeX\ triggers a segmentation fault.
% This is written in Lua so the integer setting is expandable and does not interfere with
% a preceding \verb|\immediate|.
%    \begin{macrocode}
    \protected\def\saveboxresource{%
      \LuaCol@if@keyword{type}{%
        \expandafter
        \expanded{\unexpanded{\LuaCol@scan@boxresource@i{type }}%
          \expandafter\expandafter\expandafter}%
        \LuaCol@scan@number
      }{%
        \LuaCol@scan@boxresource@i{}{}%
      }%
    }
%    \end{macrocode}
%    Legacy alias.
%    \begin{macrocode}
    \let\pdfxform\saveboxresource
\fi
%    \end{macrocode}
%
%    \begin{macrocode}
\LuaCol@AtEnd%
%</package>
%    \end{macrocode}
%
% \subsection{Lua module}
%
%    \begin{macrocode}
%<*lua>
%    \end{macrocode}
%    Box zero contains a \cs{hbox} with the color \cs{special}.
%    That is analyzed to get the prefix for the color setting
%    \cs{special}.
%    \begin{macrocode}
oberdiek = oberdiek or {}
local luacolor = oberdiek.luacolor or {}
oberdiek.luacolor = luacolor
%    \end{macrocode}
%    \begin{macro}{getversion()}
%    \begin{macrocode}
function luacolor.getversion()
  tex.write("2023-08-18 v1.18")
end
%    \end{macrocode}
%    \end{macro}
%
% \subsubsection{Driver detection}
%
%    \begin{macrocode}
local ifpdf = tonumber(tex.outputmode or tex.pdfoutput) > 0
local prefix
local prefixes = {
  dvips   = "color ",
  dvipdfm = "pdf:sc ",
  truetex = "textcolor:",
  pctexps = "ps::",
}
local patterns = {
  ["^color "]            = "dvips",
  ["^pdf: *begincolor "] = "dvipdfm",
  ["^pdf: *bcolor "]     = "dvipdfm",
  ["^pdf: *bc "]         = "dvipdfm",
  ["^pdf: *setcolor "]   = "dvipdfm",
  ["^pdf: *scolor "]     = "dvipdfm",
  ["^pdf: *sc "]         = "dvipdfm",
  ["^textcolor:"]        = "truetex",
  ["^ps::"]              = "pctexps",
}
%    \end{macrocode}
%    \begin{macro}{info()}
%    \begin{macrocode}
local function info(msg, term)
  local target = "log"
  if term then
    target = "term and log"
  end
  texio.write_nl(target, "Package luacolor info: " .. msg .. ".")
  texio.write_nl(target, "")
end
%    \end{macrocode}
%    \end{macro}
%    \begin{macro}{dvidetect()}
%    \begin{macrocode}
function luacolor.dvidetect()
  local v = tex.box[0]
  assert(v.id == node.id("hlist"))
  for v in node.traverse_id(node.id("whatsit"), v.head) do
    if v and v.subtype == node.subtype("special") then
      local data = v.data
      for pattern, driver in pairs(patterns) do
        if string.find(data, pattern) then
          prefix = prefixes[driver]
          tex.write(driver)
          return
        end
      end
      info("\\special{" .. data .. "}", true)
      return
    end
  end
  info("Missing \\special", true)
end
%    \end{macrocode}
%    \end{macro}
%
% \subsubsection{Color strings}
%
%    \begin{macrocode}
local map = {
  n = 0,
}
%    \end{macrocode}
%    \begin{macro}{get()}
%    \begin{macrocode}
function luacolor.get(color)
  tex.write("" .. luacolor.getvalue(color))
end
%    \end{macrocode}
%    \end{macro}
%    \begin{macro}{getvalue()}
%    \begin{macrocode}
function luacolor.getvalue(color)
  local n = map[color]
  if not n then
    n = map.n + 1
    map.n = n
    map[n] = color
    map[color] = n
  end
  return n
end
%    \end{macrocode}
%    \end{macro}
%
% \subsubsection{Attribute register}
%
%    \begin{macro}{setattribute()}
%    \begin{macrocode}
local attribute
function luacolor.setattribute(attr)
  attribute = attr
end
%    \end{macrocode}
%    \end{macro}
%
%    \begin{macro}{getattribute()}
%    \begin{macrocode}
function luacolor.getattribute()
  return attribute
end
%    \end{macrocode}
%    \end{macro}
%
% \subsubsection{Whatsit insertion}
%
%    \begin{macrocode}
local LIST = 1
local LIST_LEADERS = 2
local LIST_DISC = 3
local COLOR = 4
local NOCOLOR = 5
local RULE = node.id("rule")
local node_types = {
  [node.id("hlist")] = LIST,
  [node.id("vlist")] = LIST,
  [node.id("rule")]  = COLOR,
  [node.id("glyph")] = COLOR,
  [node.id("disc")]  = LIST_DISC,
  [node.id("whatsit")] = {
    [node.subtype("pdf_colorstack")] =
      function(n)
        return n.stack == 0 and NOCOLOR or nil
      end,
    [node.subtype("special")] = COLOR,
    [node.subtype("pdf_literal")] = COLOR,
    [node.subtype("pdf_save")] = COLOR,
    [node.subtype("pdf_restore")] = COLOR, -- probably not needed
-- TODO (DPC)    [node.subtype("pdf_refximage")] = COLOR,
  },
  [node.id("glue")] =
    function(n)
      if n.subtype >= 100 then -- leaders
        if n.leader.id == RULE then
          return COLOR
        else
          return LIST_LEADERS
        end
      end
    end,
}
%    \end{macrocode}
%    \begin{macro}{get_type()}
%    \begin{macrocode}
local function get_type(n)
  local ret = node_types[n.id]
  if type(ret) == 'table' then
    ret = ret[n.subtype]
  end
  if type(ret) == 'function' then
    ret = ret(n)
  end
  return ret
end
%    \end{macrocode}
%    \end{macro}
%    \begin{macrocode}
local mode = 2 -- luatex.pdfliteral.direct
local WHATSIT = node.id("whatsit")
local SPECIAL = node.subtype("special")
local PDFLITERAL = node.subtype("pdf_literal")
local DRY_FALSE = false
local DRY_TRUE = true
%    \end{macrocode}
%    \begin{macro}{traverse()}
%    \begin{macrocode}
local function traverse(list, color, dry)
  if not list then
    return color
  end
  local head
  if get_type(list) == LIST then
    head = list.head
  elseif get_type(list) == LIST_DISC then
    head = list.replace
  else
    texio.write_nl("!!! Error: Wrong list type: " .. node.type(list.id))
    return color
  end
%<debug>texio.write_nl("traverse: " .. node.type(list.id))
  for n in node.traverse(head) do
%<debug>texio.write_nl("  node: " .. node.type(n.id))
    local t = get_type(n)
%<debug>texio.write_nl("TYPE "..tostring(t).. " "..tostring(node.type(node.getid(n))).." ".. tostring(node.getsubtype(n)))
    if t == LIST or t == LIST_DISC then
      color = traverse(n, color, dry)
    elseif t == LIST_LEADERS then
      local color_after = traverse(n.leader, color, DRY_TRUE)
      if color == color_after then
        traverse(n.leader, color, DRY_FALSE or dry)
      else
        traverse(n.leader, '', DRY_FALSE or dry)
%    \end{macrocode}
% The color status is unknown here, because the leader box
% will or will not be set.
%    \begin{macrocode}
        color = ''
      end
    elseif t == COLOR then
      local v = node.has_attribute(n, attribute)
      if v then
        local newColor = map[v]
        if newColor ~= color then
          color = newColor
          if dry == DRY_FALSE then
            local newNode
            if ifpdf then
              newNode = node.new(WHATSIT, PDFLITERAL)
              newNode.mode = mode
              newNode.data = color
            else
              newNode = node.new(WHATSIT, SPECIAL)
              newNode.data = prefix .. color
            end
            head = node.insert_before(head, n, newNode)
          end
        end
      end
    elseif t == NOCOLOR then
      color = ''
    end
  end
  if get_type(list) == LIST then
    list.head = head
  else
    list.replace = head
  end
  return color
end
%    \end{macrocode}
%    \end{macro}
%
%    \begin{macro}{process()}
%    \begin{macrocode}
function luacolor.process(box)
  local color = ""
  local list = tex.getbox(box)
  traverse(list, color, DRY_FALSE)
end

if luatexbase.callbacktypes.pre_shipout_filter then
  luatexbase.add_to_callback("pre_shipout_filter", function(list)
    traverse(list, "", DRY_FALSE)
    return true
  end, "luacolor.process")
end
%    \end{macrocode}
%    \end{macro}
% For recent versions of luaotfload, we can register a callback to
% control how coloring glyph is handled for the color feature.
%    \begin{macrocode}
if luaotfload.set_colorhandler then
  local set_attribute = node.direct.set_attribute
  luaotfload.set_colorhandler(function(head, n, color)
    set_attribute(n, attribute, luacolor.getvalue(color))
    return head, n
  end)
end
%    \end{macrocode}
%
%    \begin{macrocode}
%</lua>
%    \end{macrocode}
% \section{Installation}
%
% \subsection{Download}
%
% \paragraph{Package.} This package is available on
% CTAN\footnote{\CTANpkg{luacolor}}:
% \begin{description}
% \item[\CTAN{macros/latex/contrib/luacolor/luacolor.dtx}] The source file.
% \item[\CTAN{macros/latex/contrib/luacolor/luacolor.pdf}] Documentation.
% \end{description}
%
%
% \paragraph{Bundle.} All the packages of the bundle `luacolor'
% are also available in a TDS compliant ZIP archive. There
% the packages are already unpacked and the documentation files
% are generated. The files and directories obey the TDS standard.
% \begin{description}
% \item[\CTANinstall{install/macros/latex/contrib/luacolor.tds.zip}]
% \end{description}
% \emph{TDS} refers to the standard ``A Directory Structure
% for \TeX\ Files'' (\CTANpkg{tds}). Directories
% with \xfile{texmf} in their name are usually organized this way.
%
% \subsection{Bundle installation}
%
% \paragraph{Unpacking.} Unpack the \xfile{luacolor.tds.zip} in the
% TDS tree (also known as \xfile{texmf} tree) of your choice.
% Example (linux):
% \begin{quote}
%   |unzip luacolor.tds.zip -d ~/texmf|
% \end{quote}
%
% \paragraph{Script installation.}
% Check the directory \xfile{TDS:scripts/luacolor/} for
% scripts that need further installation steps.
%
% \subsection{Package installation}
%
% \paragraph{Unpacking.} The \xfile{.dtx} file is a self-extracting
% \docstrip\ archive. The files are extracted by running the
% \xfile{.dtx} through \plainTeX:
% \begin{quote}
%   \verb|tex luacolor.dtx|
% \end{quote}
%
% \paragraph{TDS.} Now the different files must be moved into
% the different directories in your installation TDS tree
% (also known as \xfile{texmf} tree):
% \begin{quote}
% \def\t{^^A
% \begin{tabular}{@{}>{\ttfamily}l@{ $\rightarrow$ }>{\ttfamily}l@{}}
%   luacolor.sty & tex/latex/luacolor/luacolor.sty\\
%   luacolor.lua & scripts/luacolor/luacolor.lua\\
%   luacolor.pdf & doc/latex/luacolor/luacolor.pdf\\
%   luacolor.dtx & source/latex/luacolor/luacolor.dtx\\
% \end{tabular}^^A
% }^^A
% \sbox0{\t}^^A
% \ifdim\wd0>\linewidth
%   \begingroup
%     \advance\linewidth by\leftmargin
%     \advance\linewidth by\rightmargin
%   \edef\x{\endgroup
%     \def\noexpand\lw{\the\linewidth}^^A
%   }\x
%   \def\lwbox{^^A
%     \leavevmode
%     \hbox to \linewidth{^^A
%       \kern-\leftmargin\relax
%       \hss
%       \usebox0
%       \hss
%       \kern-\rightmargin\relax
%     }^^A
%   }^^A
%   \ifdim\wd0>\lw
%     \sbox0{\small\t}^^A
%     \ifdim\wd0>\linewidth
%       \ifdim\wd0>\lw
%         \sbox0{\footnotesize\t}^^A
%         \ifdim\wd0>\linewidth
%           \ifdim\wd0>\lw
%             \sbox0{\scriptsize\t}^^A
%             \ifdim\wd0>\linewidth
%               \ifdim\wd0>\lw
%                 \sbox0{\tiny\t}^^A
%                 \ifdim\wd0>\linewidth
%                   \lwbox
%                 \else
%                   \usebox0
%                 \fi
%               \else
%                 \lwbox
%               \fi
%             \else
%               \usebox0
%             \fi
%           \else
%             \lwbox
%           \fi
%         \else
%           \usebox0
%         \fi
%       \else
%         \lwbox
%       \fi
%     \else
%       \usebox0
%     \fi
%   \else
%     \lwbox
%   \fi
% \else
%   \usebox0
% \fi
% \end{quote}
% If you have a \xfile{docstrip.cfg} that configures and enables \docstrip's
% TDS installing feature, then some files can already be in the right
% place, see the documentation of \docstrip.
%
% \subsection{Refresh file name databases}
%
% If your \TeX~distribution
% (\TeX\,Live, \mikTeX, \dots) relies on file name databases, you must refresh
% these. For example, \TeX\,Live\ users run \verb|texhash| or
% \verb|mktexlsr|.
%
% \subsection{Some details for the interested}
%
% \paragraph{Unpacking with \LaTeX.}
% The \xfile{.dtx} chooses its action depending on the format:
% \begin{description}
% \item[\plainTeX:] Run \docstrip\ and extract the files.
% \item[\LaTeX:] Generate the documentation.
% \end{description}
% If you insist on using \LaTeX\ for \docstrip\ (really,
% \docstrip\ does not need \LaTeX), then inform the autodetect routine
% about your intention:
% \begin{quote}
%   \verb|latex \let\install=y\input{luacolor.dtx}|
% \end{quote}
% Do not forget to quote the argument according to the demands
% of your shell.
%
% \paragraph{Generating the documentation.}
% You can use both the \xfile{.dtx} or the \xfile{.drv} to generate
% the documentation. The process can be configured by the
% configuration file \xfile{ltxdoc.cfg}. For instance, put this
% line into this file, if you want to have A4 as paper format:
% \begin{quote}
%   \verb|\PassOptionsToClass{a4paper}{article}|
% \end{quote}
% An example follows how to generate the
% documentation with pdf\LaTeX:
% \begin{quote}
%\begin{verbatim}
%pdflatex luacolor.dtx
%makeindex -s gind.ist luacolor.idx
%pdflatex luacolor.dtx
%makeindex -s gind.ist luacolor.idx
%pdflatex luacolor.dtx
%\end{verbatim}
% \end{quote}
%
% \begin{History}
%   \begin{Version}{2007/12/12 v1.0}
%   \item
%     First public version.
%   \end{Version}
%   \begin{Version}{2009/04/10 v1.1}
%   \item
%     Fixes for changed syntax of \cs{directlua} in \LuaTeX\ 0.36.
%   \end{Version}
%   \begin{Version}{2010/03/09 v1.2}
%   \item
%     Adaptation for package \xpackage{luatex} 2010/03/09 v0.4.
%   \end{Version}
%   \begin{Version}{2010/12/13 v1.3}
%   \item
%     Support for \cs{pdfxform} added.
%   \item
%     Loaded package \xpackage{luatexbase-attr} recognized.
%   \item
%     Update for \hologo{LuaTeX}: `list' fields renamed to `head' in v0.65.0.
%   \end{Version}
%   \begin{Version}{2011/03/29 v1.4}
%   \item
%     Avoid whatsit insertion if option \xoption{monochrome} is used
%     (thanks Manuel P\'egouri\'e-Gonnard).
%   \end{Version}
%   \begin{Version}{2011/04/22 v1.5}
%   \item
%     Bug fix by Manuel P\'egouri\'e-Gonnard: A typo prevented the
%     detection of whatsits and applying color changes for
%     \cs{pdfliteral} and \cs{special} nodes that might contain
%     typesetting material.
%   \item
%     Bug fix by Manuel P\'egouri\'e-Gonnard: Now colors are also
%     applied to leader boxes.
%   \item
%     Unnecessary color settings are removed for leaders boxes,
%     if after the leader box the color has not changed.
%     The costs are a little runtime, leader boxes are processed twice.
%   \item
%     Additional whatsits that are colored: \texttt{pdf\_refximage}.
%   \item
%     Workaround for bug with \texttt{node.insert\_before}
%     removed for the version after \hologo{LuaTeX} 0.65, because
%     bug was fixed in 0.27. (Thanks Manuel P\'egouri\'e-Gonnard.)
%   \end{Version}
%   \begin{Version}{2011/04/23 v1.6}
%   \item
%     Bug fix for nested leader boxes.
%   \item
%     Bug fix for leader boxes that change color, but
%     are not set because of missing place.
%   \item
%     Version check for Lua module added.
%   \end{Version}
%   \begin{Version}{2011/10/22 v1.7}
%   \item
%     Lua functions \texttt{getattribute} and \texttt{getvalue}
%     added to tell other external Lua functions
%     the attribute register number for coloring.
%   \end{Version}
%   \begin{Version}{2011/11/01 v1.8}
%   \item
%     Use of \texttt{node.subtype} instead of magic numbers.
%   \end{Version}
%   \begin{Version}{2016/05/13 v1.9}
%   \item
%     More use of \texttt{node.subtype} instead of magic numbers.
%   \item luatex 85 updates
%   \end{Version}
%   \begin{Version}{2016/05/16 v1.10}
%   \item
%     Documentation updates.
%   \end{Version}
%   \begin{Version}{2018/11/22 v1.11}
%   \item
%     handle issue 43.
%   \item
%     removed pre-0.65 stuff
%   \end{Version}
%   \begin{Version}{2019/07/25 v1.12}
%   \item
%     removed uses of module function, see PR70
%   \end{Version}
%   \begin{Version}{2019/11/29 v1.13}
%   \item
%     Documentation updates.
%   \item  Use \xpackage{iftex} directly.
%   \end{Version}
%   \begin{Version}{2020-02-22 v1.14}
%   \item Drop use of \textsf{iftex} \textsf{ltxcmds} and \textsf{infwarerr}.
%   \item Assume \textsf{ltluatex} preloaded into format (true since 2015).
%   \item Patch \verb|\saveboxresource| rather than \verb|\pdfxform| (keep old name as alias).
%   \item Grab the number via Lua so that a \verb|\immediate| prefix still works with
%         \verb|\saveboxresource|\slash \verb|\pdfxform|.
%   \item Added handler for the color feature of luaotfload      
%   \end{Version}
%   \begin{Version}{2020-02-24 v1.15}
%   \item Grab all possible arguments for \verb|\saveboxresource|\slash \verb|\pdfxform|
%   \end{Version}
%   \begin{Version}{2020-04-04 v1.16}
%   \item Reset color after \verb|pdf_colorstack| whatsits.
%   \end{Version}
%   \begin{Version}{2021-02-17 v1.17}
%   \item Use \LaTeXe's new \verb|pre_shipout_filter| callback if it's available to
%         allow coloring background and foreground layer material
%   \end{Version}
%   \begin{Version}{2023-08-18 v1.18}
%   \item added \verb|\reset@color| to \verb|\set@color| for \verb|\mathcolor|, issue 7.
%   \end{Version}
% \end{History}
%
% \PrintIndex
%
% \Finale
\endinput

%        (quote the arguments according to the demands of your shell)
%
% Documentation:
%    (a) If luacolor.drv is present:
%           latex luacolor.drv
%    (b) Without luacolor.drv:
%           latex luacolor.dtx; ...
%    The class ltxdoc loads the configuration file ltxdoc.cfg
%    if available. Here you can specify further options, e.g.
%    use A4 as paper format:
%       \PassOptionsToClass{a4paper}{article}
%
%    Programm calls to get the documentation (example):
%       pdflatex luacolor.dtx
%       makeindex -s gind.ist luacolor.idx
%       pdflatex luacolor.dtx
%       makeindex -s gind.ist luacolor.idx
%       pdflatex luacolor.dtx
%
% Installation:
%    TDS:tex/latex/luacolor/luacolor.sty
%    TDS:scripts/luacolor/luacolor.lua
%    TDS:doc/latex/luacolor/luacolor.pdf
%    TDS:source/latex/luacolor/luacolor.dtx
%
%<*ignore>
\begingroup
  \catcode123=1 %
  \catcode125=2 %
  \def\x{LaTeX2e}%
\expandafter\endgroup
\ifcase 0\ifx\install y1\fi\expandafter
         \ifx\csname processbatchFile\endcsname\relax\else1\fi
         \ifx\fmtname\x\else 1\fi\relax
\else\csname fi\endcsname
%</ignore>
%<*install>
\input docstrip.tex
\Msg{************************************************************************}
\Msg{* Installation}
\Msg{* Package: luacolor 2023-08-18 v1.18 Color support via LuaTeX's attributes (HO)}
\Msg{************************************************************************}

\keepsilent
\askforoverwritefalse

\let\MetaPrefix\relax
\preamble

This is a generated file.

Project: luacolor
Version: 2023-08-18 v1.18

Copyright (C)
   2007, 2009-2011 Heiko Oberdiek
   2016-2023 Oberdiek Package Support Group

This work may be distributed and/or modified under the
conditions of the LaTeX Project Public License, either
version 1.3c of this license or (at your option) any later
version. This version of this license is in
   https://www.latex-project.org/lppl/lppl-1-3c.txt
and the latest version of this license is in
   https://www.latex-project.org/lppl.txt
and version 1.3 or later is part of all distributions of
LaTeX version 2005/12/01 or later.

This work has the LPPL maintenance status "maintained".

The Current Maintainers of this work are
Heiko Oberdiek and the Oberdiek Package Support Group
https://github.com/ho-tex/luacolor/issues


This work consists of the main source file luacolor.dtx
and the derived files
   luacolor.sty, luacolor.pdf, luacolor.ins, luacolor.drv,
   luacolor.lua,

\endpreamble
\let\MetaPrefix\DoubleperCent

\generate{%
  \file{luacolor.ins}{\from{luacolor.dtx}{install}}%
  \file{luacolor.drv}{\from{luacolor.dtx}{driver}}%
  \usedir{tex/latex/luacolor}%
  \file{luacolor.sty}{\from{luacolor.dtx}{package}}%
}
\def\MetaPrefix{-- }
\def\defaultpostamble{%
  \MetaPrefix^^J%
  \MetaPrefix\space End of File `\outFileName'.%
}
\def\currentpostamble{\defaultpostamble}%
\generate{%
  \usedir{scripts/luacolor}%
%  \file{oberdiek.luacolor.lua}{\from{luacolor.dtx}{lua}}%
  \file{luacolor.lua}{\from{luacolor.dtx}{lua}}%
}

\catcode32=13\relax% active space
\let =\space%
\Msg{************************************************************************}
\Msg{*}
\Msg{* To finish the installation you have to move the following}
\Msg{* file into a directory searched by TeX:}
\Msg{*}
\Msg{*     luacolor.sty}
\Msg{*}
\Msg{* And install the following script file:}
\Msg{*}
\Msg{*     luacolor.lua,}
\Msg{*}
\Msg{* To produce the documentation run the file `luacolor.dtx'}
\Msg{* through LaTeX.}
\Msg{*}
\Msg{* Happy TeXing!}
\Msg{*}
\Msg{************************************************************************}

\endbatchfile
%</install>
%<*ignore>
\fi
%</ignore>
%<*driver>
\NeedsTeXFormat{LaTeX2e}
\ProvidesFile{luacolor.drv}%
  [2023-08-18 v1.18 Color support via LuaTeX's attributes (HO)]%
\documentclass{ltxdoc}
\usepackage{holtxdoc}[2011/11/22]
\usepackage{paralist}
\makeatletter
\g@addto@macro\MakePrivateLetters{%
  \@makeother\_%
}
\makeatother
\begin{document}
  \DocInput{luacolor.dtx}%
\end{document}
%</driver>
% \fi
%
%
%
% \GetFileInfo{luacolor.drv}
%
% \title{The \xpackage{luacolor} package}
% \date{2023-08-18 v1.18}
% \author{Heiko Oberdiek\thanks
% {Please report any issues at \url{https://github.com/ho-tex/luacolor/issues}}}
%
% \maketitle
%
% \begin{abstract}
% Package \xpackage{luacolor} implements color support based
% on \LuaTeX's node attributes.
% \end{abstract}
%
% \tableofcontents
%
% \section{Documentation}
%
% \subsection{Introduction}
%
% This package uses a \LuaTeX's attribute register to
% to annotate nodes with color information.
% If a color is set, then the attribute register
% is set to this color and all nodes created in its scope
% (current group) are annotated with this attribute.
% Now the color property behaves much the same way
% as the font property.
%
% \subsection{Usage}
%
% Package \xpackage{color} is loaded automatically by this
% package \xpackage{luacolor}. If you need a special driver
% option or you prefer package \xpackage{xcolor}, then load
% it before package \xpackage{luacolor}, for example:
% \begin{quote}
%   |\usepackage[dvipdfmx]{xcolor}|
% \end{quote}
% The package \xpackage{luacolor} is loaded without options:
% \begin{quote}
%   |\usepackage{luacolor}|
% \end{quote}
% It is able to detect PDF mode and DVI drivers are
% differentiated by its color specials. Therefore the
% package do need driver options.
%
% Then it redefines the color setting commands to set
% attributes instead of whatsits for color.
%
% At last the attribute annotations of the nodes in the
% output box must be analyzed to insert the necessary color whatsits.
% (With older \LuaTeX\ that lack the appropriate callback function 
% the package \xpackage{atbegshi} is used to get control
% before a box is shipped out.)
%
% \begin{declcs}{luacolorProcessBox} \M{box}
% \end{declcs}
% Macro \cs{luacolorProcessBox} processes the box \meta{box}
% in the previously described manner. It is automatically
% called for pages, but not for XForm objects.
% Before passing a box to \cs{pdfxform}, call \cs{luacolorProcessBox}
% first.
%
% \subsection{Limitations}
%
% \begin{description}
% \item[Ligatures with different colored components:]
%   Package \xpackage{luacolor} sees the ligature after
%   the paragraph building and page breaking, when a
%   page is to be shipped out. Therefore it cannot
%   break ligatures, because the components might occupy
%   different space. Therefore it is the responsibility
%   of the ligature forming process to deal with different
%   colored glyphs that form a ligature. The user can avoid
%   the problem entirely by explicitly breaking the ligature
%   at the places where the color changes.
%  \item \dots
% \end{description}
%
% \StopEventually{
% }
%
% \section{Implementation}
%
%    \begin{macrocode}
%<*package>
%    \end{macrocode}
%
% \subsection{Catcodes and identification}
%
%    \begin{macrocode}
\begingroup\catcode61\catcode48\catcode32=10\relax%
  \catcode13=5 % ^^M
  \endlinechar=13 %
  \catcode123=1 % {
  \catcode125=2 % }
  \catcode64=11 % @
  \def\x{\endgroup
    \expandafter\edef\csname LuaCol@AtEnd\endcsname{%
      \endlinechar=\the\endlinechar\relax
      \catcode13=\the\catcode13\relax
      \catcode32=\the\catcode32\relax
      \catcode35=\the\catcode35\relax
      \catcode61=\the\catcode61\relax
      \catcode64=\the\catcode64\relax
      \catcode123=\the\catcode123\relax
      \catcode125=\the\catcode125\relax
    }%
  }%
\x\catcode61\catcode48\catcode32=10\relax%
\catcode13=5 % ^^M
\endlinechar=13 %
\catcode35=6 % #
\catcode64=11 % @
\catcode123=1 % {
\catcode125=2 % }
\def\TMP@EnsureCode#1#2{%
  \edef\LuaCol@AtEnd{%
    \LuaCol@AtEnd
    \catcode#1=\the\catcode#1\relax
  }%
  \catcode#1=#2\relax
}
\TMP@EnsureCode{34}{12}% "
\TMP@EnsureCode{39}{12}% '
\TMP@EnsureCode{40}{12}% (
\TMP@EnsureCode{41}{12}% )
\TMP@EnsureCode{42}{12}% *
\TMP@EnsureCode{43}{12}% +
\TMP@EnsureCode{44}{12}% ,
\TMP@EnsureCode{45}{12}% -
\TMP@EnsureCode{46}{12}% .
\TMP@EnsureCode{47}{12}% /
\TMP@EnsureCode{58}{12}% :
\TMP@EnsureCode{60}{12}% <
\TMP@EnsureCode{62}{12}% >
\TMP@EnsureCode{91}{12}% [
\TMP@EnsureCode{93}{12}% ]
\TMP@EnsureCode{95}{12}% _ (other!)
\TMP@EnsureCode{96}{12}% `
\edef\LuaCol@AtEnd{\LuaCol@AtEnd\noexpand\endinput}
%    \end{macrocode}
%
%    Package identification.
%    \begin{macrocode}
\NeedsTeXFormat{LaTeX2e}
\ProvidesPackage{luacolor}%
  [2023-08-18 v1.18 Color support via LuaTeX's attributes (HO)]
%    \end{macrocode}
%
% \subsection{Check for \LuaTeX}
%
%    Without \LuaTeX\ there is no point in using this package.
%    \begin{macrocode}
\RequirePackage{color}
%    \end{macrocode}
%
%    \begin{macrocode}
\ifx\directlua\@undefined
  \PackageError{luacolor}{%
    This package may only be run using LuaTeX%
  }\@ehc
  \expandafter\LuaCol@AtEnd
\fi%
%    \end{macrocode}
%
% \subsection{Check for disabled colors}
%
%    \begin{macrocode}
\ifcolors@
\else
  \PackageWarningNoLine{luacolor}{%
    Colors are disabled by option `monochrome'%
  }%
  \def\set@color{}%
  \def\reset@color{}%
  \def\set@page@color{}%
  \def\define@color#1#2{}%
  \expandafter\LuaCol@AtEnd
\fi%
%    \end{macrocode}
%
% \subsection{Load module and check version}
%
%    \begin{macrocode}
\directlua{%
  require("luacolor")%
}
%    \end{macrocode}
%    \begin{macrocode}
\begingroup
  \edef\x{\directlua{tex.write("2023-08-18 v1.18")}}%
  \edef\y{%
    \directlua{%
      if oberdiek.luacolor.getversion then %
        oberdiek.luacolor.getversion()%
      end%
    }%
  }%
  \ifx\x\y
  \else
    \PackageError{luacolor}{%
      Wrong version of lua module.\MessageBreak
      Package version: \x\MessageBreak
      Lua module: \y
    }\@ehc
  \fi
\endgroup
%    \end{macrocode}
%
% \subsection{Find driver}
%
%    \begin{macrocode}
\ifnum\outputmode=\@ne
\else
  \begingroup
    \def\current@color{}%
    \def\reset@color{}%
    \setbox\z@=\hbox{%
      \begingroup
        \set@color
      \endgroup
    }%
    \edef\reserved@a{%
      \directlua{%
        oberdiek.luacolor.dvidetect()%
      }%
    }%
    \ifx\reserved@a\@empty
      \PackageError{luacolor}{%
        DVI driver detection failed because of\MessageBreak
        unrecognized color \string\special
      }\@ehc
      \endgroup
      \expandafter\expandafter\expandafter\LuaCol@AtEnd
    \else
      \PackageInfo{luacolor}{%
        Type of color \string\special: \reserved@a
      \@gobble}%
    \fi%
  \endgroup
\fi
%    \end{macrocode}
%
% \subsection{Attribute setting}
%
%    \begin{macro}{\LuaCol@Attribute}
%    \begin{macrocode}
\newattribute\LuaCol@Attribute
\let\LuaCol@setattribute\setattribute
\directlua{%
  oberdiek.luacolor.setattribute(\number\allocationnumber)%
}
%    \end{macrocode}
%    \end{macro}
%
%    \begin{macro}{\set@color}
% change 2023-08-18: added \cs{reset@color} so that \cs{mathcolor} can gobble it,
% issue 7.   
%    \begin{macrocode}
\protected\def\set@color{%
  \LuaCol@setattribute\LuaCol@Attribute{%
    \directlua{%
      oberdiek.luacolor.get("\luaescapestring{\current@color}")%
    }%
  }%
 \aftergroup\reset@color 
}
%    \end{macrocode}
%    \end{macro}
%    \begin{macro}{\reset@color}
%    \begin{macrocode}
\def\reset@color{}
%    \end{macrocode}
%    \end{macro}
%
% \subsection{Whatsit insertion}
%
%    \begin{macro}{\luacolorProcessBox}
%    \begin{macrocode}
\def\luacolorProcessBox#1{%
  \directlua{%
    oberdiek.luacolor.process(\number#1)%
  }%
}
%    \end{macrocode}
%    \end{macro}
%
%    \begin{macrocode}
\directlua{%
  if luatexbase.callbacktypes.pre_shipout_filter then
    token.get_next()
  end
}\@secondoftwo\@gobble{
  \RequirePackage{atbegshi}[2011/01/30]
  \AtBeginShipout{%
    \luacolorProcessBox\AtBeginShipoutBox
  }
}
%    \end{macrocode}
%
%    Set default color.
%    \begin{macrocode}
\set@color
%    \end{macrocode}
%
% \subsection{\cs{pdfxform}/\cs{saveboxresource} support}
%
%    \begin{macrocode}
\ifnum\outputmode=\@ne
    \let\LuaCol@org@pdfxform\saveboxresource
%    \end{macrocode}
% First we need some helpers to allow expandable code to parse keyword style arguments:
%    \begin{macrocode}
    \def\LuaCol@iii@i@ii#1#2#3{#3{#1}{#2}}
    \def\LuaCol@ii@i#1#2{{#2#1}}
    \def\LuaCol@if@keyword#1#2#3{%
      \expanded{\unexpanded{\LuaCol@iii@i@ii{#2}{#3}}\expandafter}%
      \directlua{%
        token.put_next(token.create(token.scan_keyword(token.scan_string())
        and '@firstoftwo'
        or '@secondoftwo'))
      }{#1}%
    }
%    \end{macrocode}
% The following macro scans a integer and expands to a token equivalent to a chardef
% whose value corresponds to the scanned integer. This allows the integer to be passed
% around as a undelimited argument.
%    \begin{macrocode}
    \def\LuaCol@scan@number{%
      \directlua{
        token.put_next(token.new(token.scan_int(), token.command_id'char_given'))
      }%
    }
%    \end{macrocode}
% \TeX\ primitives like \cs{saveboxresource} read braced arguments in a special way.
% Especially they expand everything until they find a left brace. To simulate this, we
% use Lua to expand everything else:
%    \begin{macrocode}
    \def\LuaCol@scan@tobrace{%
      \directlua{
        local relax, space = token.command_id'relax', token.command_id'spacer'
        local t
        repeat
          t = token.scan_token()
        until not (t.command == relax or t.command == space)
        token.put_next(t)
      }%
    }
    \def\LuaCol@scan@boxresource@i#1#2{%
      \LuaCol@if@keyword{attr}{%
        \expanded{\unexpanded{\LuaCol@scan@boxresource@iI{#1#2attr}}%
          \expandafter\expandafter\expandafter}%
        \LuaCol@scan@tobrace
      }{%
        \LuaCol@scan@boxresource@ii{#1#2}%
      }%
    }
    \def\LuaCol@scan@boxresource@iI#1#2{\LuaCol@scan@boxresource@ii{#1{#2}}}
    \def\LuaCol@scan@boxresource@ii#1{%
      \LuaCol@if@keyword{resources}{%
        \expanded{\unexpanded{\LuaCol@scan@boxresource@iiI{#1resources}}%
          \expandafter\expandafter\expandafter}%
        \LuaCol@scan@tobrace
      }{%
        \LuaCol@scan@boxresource@iii{#1}%
      }%
    }
    \def\LuaCol@scan@boxresource@iiI#1#2{\LuaCol@scan@boxresource@iii{#1{#2}}}
    \def\LuaCol@scan@boxresource@iii#1{%
      \LuaCol@if@keyword{margin}{%
        \expanded{\unexpanded{\LuaCol@scan@boxresource@iv{#1margin }}%
          \expandafter\expandafter\expandafter}%
        \LuaCol@scan@number
      }{%
        \LuaCol@scan@boxresource@iv{#1}{}%
      }%
    }
    \def\LuaCol@scan@boxresource@iv#1#2{%
      \expanded{\unexpanded{\LuaCol@scan@boxresource@v{#1#2}}%
        \expandafter\expandafter\expandafter}%
      \LuaCol@scan@number
    }
    \def\LuaCol@scan@boxresource@v#1#2{%
      \luacolorProcessBox{#2}%
      \LuaCol@org@pdfxform#1#2%
    }

%    \end{macrocode}
% This could be written in Lua, but at least upto Lua\TeX~1.11, feeding back too many tokens
% from Lua to \TeX\ triggers a segmentation fault.
% This is written in Lua so the integer setting is expandable and does not interfere with
% a preceding \verb|\immediate|.
%    \begin{macrocode}
    \protected\def\saveboxresource{%
      \LuaCol@if@keyword{type}{%
        \expandafter
        \expanded{\unexpanded{\LuaCol@scan@boxresource@i{type }}%
          \expandafter\expandafter\expandafter}%
        \LuaCol@scan@number
      }{%
        \LuaCol@scan@boxresource@i{}{}%
      }%
    }
%    \end{macrocode}
%    Legacy alias.
%    \begin{macrocode}
    \let\pdfxform\saveboxresource
\fi
%    \end{macrocode}
%
%    \begin{macrocode}
\LuaCol@AtEnd%
%</package>
%    \end{macrocode}
%
% \subsection{Lua module}
%
%    \begin{macrocode}
%<*lua>
%    \end{macrocode}
%    Box zero contains a \cs{hbox} with the color \cs{special}.
%    That is analyzed to get the prefix for the color setting
%    \cs{special}.
%    \begin{macrocode}
oberdiek = oberdiek or {}
local luacolor = oberdiek.luacolor or {}
oberdiek.luacolor = luacolor
%    \end{macrocode}
%    \begin{macro}{getversion()}
%    \begin{macrocode}
function luacolor.getversion()
  tex.write("2023-08-18 v1.18")
end
%    \end{macrocode}
%    \end{macro}
%
% \subsubsection{Driver detection}
%
%    \begin{macrocode}
local ifpdf = tonumber(tex.outputmode or tex.pdfoutput) > 0
local prefix
local prefixes = {
  dvips   = "color ",
  dvipdfm = "pdf:sc ",
  truetex = "textcolor:",
  pctexps = "ps::",
}
local patterns = {
  ["^color "]            = "dvips",
  ["^pdf: *begincolor "] = "dvipdfm",
  ["^pdf: *bcolor "]     = "dvipdfm",
  ["^pdf: *bc "]         = "dvipdfm",
  ["^pdf: *setcolor "]   = "dvipdfm",
  ["^pdf: *scolor "]     = "dvipdfm",
  ["^pdf: *sc "]         = "dvipdfm",
  ["^textcolor:"]        = "truetex",
  ["^ps::"]              = "pctexps",
}
%    \end{macrocode}
%    \begin{macro}{info()}
%    \begin{macrocode}
local function info(msg, term)
  local target = "log"
  if term then
    target = "term and log"
  end
  texio.write_nl(target, "Package luacolor info: " .. msg .. ".")
  texio.write_nl(target, "")
end
%    \end{macrocode}
%    \end{macro}
%    \begin{macro}{dvidetect()}
%    \begin{macrocode}
function luacolor.dvidetect()
  local v = tex.box[0]
  assert(v.id == node.id("hlist"))
  for v in node.traverse_id(node.id("whatsit"), v.head) do
    if v and v.subtype == node.subtype("special") then
      local data = v.data
      for pattern, driver in pairs(patterns) do
        if string.find(data, pattern) then
          prefix = prefixes[driver]
          tex.write(driver)
          return
        end
      end
      info("\\special{" .. data .. "}", true)
      return
    end
  end
  info("Missing \\special", true)
end
%    \end{macrocode}
%    \end{macro}
%
% \subsubsection{Color strings}
%
%    \begin{macrocode}
local map = {
  n = 0,
}
%    \end{macrocode}
%    \begin{macro}{get()}
%    \begin{macrocode}
function luacolor.get(color)
  tex.write("" .. luacolor.getvalue(color))
end
%    \end{macrocode}
%    \end{macro}
%    \begin{macro}{getvalue()}
%    \begin{macrocode}
function luacolor.getvalue(color)
  local n = map[color]
  if not n then
    n = map.n + 1
    map.n = n
    map[n] = color
    map[color] = n
  end
  return n
end
%    \end{macrocode}
%    \end{macro}
%
% \subsubsection{Attribute register}
%
%    \begin{macro}{setattribute()}
%    \begin{macrocode}
local attribute
function luacolor.setattribute(attr)
  attribute = attr
end
%    \end{macrocode}
%    \end{macro}
%
%    \begin{macro}{getattribute()}
%    \begin{macrocode}
function luacolor.getattribute()
  return attribute
end
%    \end{macrocode}
%    \end{macro}
%
% \subsubsection{Whatsit insertion}
%
%    \begin{macrocode}
local LIST = 1
local LIST_LEADERS = 2
local LIST_DISC = 3
local COLOR = 4
local NOCOLOR = 5
local RULE = node.id("rule")
local node_types = {
  [node.id("hlist")] = LIST,
  [node.id("vlist")] = LIST,
  [node.id("rule")]  = COLOR,
  [node.id("glyph")] = COLOR,
  [node.id("disc")]  = LIST_DISC,
  [node.id("whatsit")] = {
    [node.subtype("pdf_colorstack")] =
      function(n)
        return n.stack == 0 and NOCOLOR or nil
      end,
    [node.subtype("special")] = COLOR,
    [node.subtype("pdf_literal")] = COLOR,
    [node.subtype("pdf_save")] = COLOR,
    [node.subtype("pdf_restore")] = COLOR, -- probably not needed
-- TODO (DPC)    [node.subtype("pdf_refximage")] = COLOR,
  },
  [node.id("glue")] =
    function(n)
      if n.subtype >= 100 then -- leaders
        if n.leader.id == RULE then
          return COLOR
        else
          return LIST_LEADERS
        end
      end
    end,
}
%    \end{macrocode}
%    \begin{macro}{get_type()}
%    \begin{macrocode}
local function get_type(n)
  local ret = node_types[n.id]
  if type(ret) == 'table' then
    ret = ret[n.subtype]
  end
  if type(ret) == 'function' then
    ret = ret(n)
  end
  return ret
end
%    \end{macrocode}
%    \end{macro}
%    \begin{macrocode}
local mode = 2 -- luatex.pdfliteral.direct
local WHATSIT = node.id("whatsit")
local SPECIAL = node.subtype("special")
local PDFLITERAL = node.subtype("pdf_literal")
local DRY_FALSE = false
local DRY_TRUE = true
%    \end{macrocode}
%    \begin{macro}{traverse()}
%    \begin{macrocode}
local function traverse(list, color, dry)
  if not list then
    return color
  end
  local head
  if get_type(list) == LIST then
    head = list.head
  elseif get_type(list) == LIST_DISC then
    head = list.replace
  else
    texio.write_nl("!!! Error: Wrong list type: " .. node.type(list.id))
    return color
  end
%<debug>texio.write_nl("traverse: " .. node.type(list.id))
  for n in node.traverse(head) do
%<debug>texio.write_nl("  node: " .. node.type(n.id))
    local t = get_type(n)
%<debug>texio.write_nl("TYPE "..tostring(t).. " "..tostring(node.type(node.getid(n))).." ".. tostring(node.getsubtype(n)))
    if t == LIST or t == LIST_DISC then
      color = traverse(n, color, dry)
    elseif t == LIST_LEADERS then
      local color_after = traverse(n.leader, color, DRY_TRUE)
      if color == color_after then
        traverse(n.leader, color, DRY_FALSE or dry)
      else
        traverse(n.leader, '', DRY_FALSE or dry)
%    \end{macrocode}
% The color status is unknown here, because the leader box
% will or will not be set.
%    \begin{macrocode}
        color = ''
      end
    elseif t == COLOR then
      local v = node.has_attribute(n, attribute)
      if v then
        local newColor = map[v]
        if newColor ~= color then
          color = newColor
          if dry == DRY_FALSE then
            local newNode
            if ifpdf then
              newNode = node.new(WHATSIT, PDFLITERAL)
              newNode.mode = mode
              newNode.data = color
            else
              newNode = node.new(WHATSIT, SPECIAL)
              newNode.data = prefix .. color
            end
            head = node.insert_before(head, n, newNode)
          end
        end
      end
    elseif t == NOCOLOR then
      color = ''
    end
  end
  if get_type(list) == LIST then
    list.head = head
  else
    list.replace = head
  end
  return color
end
%    \end{macrocode}
%    \end{macro}
%
%    \begin{macro}{process()}
%    \begin{macrocode}
function luacolor.process(box)
  local color = ""
  local list = tex.getbox(box)
  traverse(list, color, DRY_FALSE)
end

if luatexbase.callbacktypes.pre_shipout_filter then
  luatexbase.add_to_callback("pre_shipout_filter", function(list)
    traverse(list, "", DRY_FALSE)
    return true
  end, "luacolor.process")
end
%    \end{macrocode}
%    \end{macro}
% For recent versions of luaotfload, we can register a callback to
% control how coloring glyph is handled for the color feature.
%    \begin{macrocode}
if luaotfload.set_colorhandler then
  local set_attribute = node.direct.set_attribute
  luaotfload.set_colorhandler(function(head, n, color)
    set_attribute(n, attribute, luacolor.getvalue(color))
    return head, n
  end)
end
%    \end{macrocode}
%
%    \begin{macrocode}
%</lua>
%    \end{macrocode}
% \section{Installation}
%
% \subsection{Download}
%
% \paragraph{Package.} This package is available on
% CTAN\footnote{\CTANpkg{luacolor}}:
% \begin{description}
% \item[\CTAN{macros/latex/contrib/luacolor/luacolor.dtx}] The source file.
% \item[\CTAN{macros/latex/contrib/luacolor/luacolor.pdf}] Documentation.
% \end{description}
%
%
% \paragraph{Bundle.} All the packages of the bundle `luacolor'
% are also available in a TDS compliant ZIP archive. There
% the packages are already unpacked and the documentation files
% are generated. The files and directories obey the TDS standard.
% \begin{description}
% \item[\CTANinstall{install/macros/latex/contrib/luacolor.tds.zip}]
% \end{description}
% \emph{TDS} refers to the standard ``A Directory Structure
% for \TeX\ Files'' (\CTANpkg{tds}). Directories
% with \xfile{texmf} in their name are usually organized this way.
%
% \subsection{Bundle installation}
%
% \paragraph{Unpacking.} Unpack the \xfile{luacolor.tds.zip} in the
% TDS tree (also known as \xfile{texmf} tree) of your choice.
% Example (linux):
% \begin{quote}
%   |unzip luacolor.tds.zip -d ~/texmf|
% \end{quote}
%
% \paragraph{Script installation.}
% Check the directory \xfile{TDS:scripts/luacolor/} for
% scripts that need further installation steps.
%
% \subsection{Package installation}
%
% \paragraph{Unpacking.} The \xfile{.dtx} file is a self-extracting
% \docstrip\ archive. The files are extracted by running the
% \xfile{.dtx} through \plainTeX:
% \begin{quote}
%   \verb|tex luacolor.dtx|
% \end{quote}
%
% \paragraph{TDS.} Now the different files must be moved into
% the different directories in your installation TDS tree
% (also known as \xfile{texmf} tree):
% \begin{quote}
% \def\t{^^A
% \begin{tabular}{@{}>{\ttfamily}l@{ $\rightarrow$ }>{\ttfamily}l@{}}
%   luacolor.sty & tex/latex/luacolor/luacolor.sty\\
%   luacolor.lua & scripts/luacolor/luacolor.lua\\
%   luacolor.pdf & doc/latex/luacolor/luacolor.pdf\\
%   luacolor.dtx & source/latex/luacolor/luacolor.dtx\\
% \end{tabular}^^A
% }^^A
% \sbox0{\t}^^A
% \ifdim\wd0>\linewidth
%   \begingroup
%     \advance\linewidth by\leftmargin
%     \advance\linewidth by\rightmargin
%   \edef\x{\endgroup
%     \def\noexpand\lw{\the\linewidth}^^A
%   }\x
%   \def\lwbox{^^A
%     \leavevmode
%     \hbox to \linewidth{^^A
%       \kern-\leftmargin\relax
%       \hss
%       \usebox0
%       \hss
%       \kern-\rightmargin\relax
%     }^^A
%   }^^A
%   \ifdim\wd0>\lw
%     \sbox0{\small\t}^^A
%     \ifdim\wd0>\linewidth
%       \ifdim\wd0>\lw
%         \sbox0{\footnotesize\t}^^A
%         \ifdim\wd0>\linewidth
%           \ifdim\wd0>\lw
%             \sbox0{\scriptsize\t}^^A
%             \ifdim\wd0>\linewidth
%               \ifdim\wd0>\lw
%                 \sbox0{\tiny\t}^^A
%                 \ifdim\wd0>\linewidth
%                   \lwbox
%                 \else
%                   \usebox0
%                 \fi
%               \else
%                 \lwbox
%               \fi
%             \else
%               \usebox0
%             \fi
%           \else
%             \lwbox
%           \fi
%         \else
%           \usebox0
%         \fi
%       \else
%         \lwbox
%       \fi
%     \else
%       \usebox0
%     \fi
%   \else
%     \lwbox
%   \fi
% \else
%   \usebox0
% \fi
% \end{quote}
% If you have a \xfile{docstrip.cfg} that configures and enables \docstrip's
% TDS installing feature, then some files can already be in the right
% place, see the documentation of \docstrip.
%
% \subsection{Refresh file name databases}
%
% If your \TeX~distribution
% (\TeX\,Live, \mikTeX, \dots) relies on file name databases, you must refresh
% these. For example, \TeX\,Live\ users run \verb|texhash| or
% \verb|mktexlsr|.
%
% \subsection{Some details for the interested}
%
% \paragraph{Unpacking with \LaTeX.}
% The \xfile{.dtx} chooses its action depending on the format:
% \begin{description}
% \item[\plainTeX:] Run \docstrip\ and extract the files.
% \item[\LaTeX:] Generate the documentation.
% \end{description}
% If you insist on using \LaTeX\ for \docstrip\ (really,
% \docstrip\ does not need \LaTeX), then inform the autodetect routine
% about your intention:
% \begin{quote}
%   \verb|latex \let\install=y% \iffalse meta-comment
%
% File: luacolor.dtx
% Version: 2023-08-18 v1.18
% Info: Color support via LuaTeX's attributes
%
% Copyright (C)
%    2007, 2009-2011 Heiko Oberdiek
%    2016-2023 Oberdiek Package Support Group
%    https://github.com/ho-tex/luacolor/issues
%
% This work may be distributed and/or modified under the
% conditions of the LaTeX Project Public License, either
% version 1.3c of this license or (at your option) any later
% version. This version of this license is in
%    https://www.latex-project.org/lppl/lppl-1-3c.txt
% and the latest version of this license is in
%    https://www.latex-project.org/lppl.txt
% and version 1.3 or later is part of all distributions of
% LaTeX version 2005/12/01 or later.
%
% This work has the LPPL maintenance status "maintained".
%
% The Current Maintainers of this work are
% Heiko Oberdiek and the Oberdiek Package Support Group
% https://github.com/ho-tex/luacolor/issues
%
% This work consists of the main source file luacolor.dtx
% and the derived files
%    luacolor.sty, luacolor.pdf, luacolor.lua,
%
% Distribution:
%    CTAN:macros/latex/contrib/luacolor/luacolor.dtx
%    CTAN:macros/latex/contrib/luacolor/luacolor.pdf
%
% Unpacking:
%    (a) If luacolor.ins is present:
%           tex luacolor.ins
%    (b) Without luacolor.ins:
%           tex luacolor.dtx
%    (c) If you insist on using LaTeX
%           latex \let\install=y\input{luacolor.dtx}
%        (quote the arguments according to the demands of your shell)
%
% Documentation:
%    (a) If luacolor.drv is present:
%           latex luacolor.drv
%    (b) Without luacolor.drv:
%           latex luacolor.dtx; ...
%    The class ltxdoc loads the configuration file ltxdoc.cfg
%    if available. Here you can specify further options, e.g.
%    use A4 as paper format:
%       \PassOptionsToClass{a4paper}{article}
%
%    Programm calls to get the documentation (example):
%       pdflatex luacolor.dtx
%       makeindex -s gind.ist luacolor.idx
%       pdflatex luacolor.dtx
%       makeindex -s gind.ist luacolor.idx
%       pdflatex luacolor.dtx
%
% Installation:
%    TDS:tex/latex/luacolor/luacolor.sty
%    TDS:scripts/luacolor/luacolor.lua
%    TDS:doc/latex/luacolor/luacolor.pdf
%    TDS:source/latex/luacolor/luacolor.dtx
%
%<*ignore>
\begingroup
  \catcode123=1 %
  \catcode125=2 %
  \def\x{LaTeX2e}%
\expandafter\endgroup
\ifcase 0\ifx\install y1\fi\expandafter
         \ifx\csname processbatchFile\endcsname\relax\else1\fi
         \ifx\fmtname\x\else 1\fi\relax
\else\csname fi\endcsname
%</ignore>
%<*install>
\input docstrip.tex
\Msg{************************************************************************}
\Msg{* Installation}
\Msg{* Package: luacolor 2023-08-18 v1.18 Color support via LuaTeX's attributes (HO)}
\Msg{************************************************************************}

\keepsilent
\askforoverwritefalse

\let\MetaPrefix\relax
\preamble

This is a generated file.

Project: luacolor
Version: 2023-08-18 v1.18

Copyright (C)
   2007, 2009-2011 Heiko Oberdiek
   2016-2023 Oberdiek Package Support Group

This work may be distributed and/or modified under the
conditions of the LaTeX Project Public License, either
version 1.3c of this license or (at your option) any later
version. This version of this license is in
   https://www.latex-project.org/lppl/lppl-1-3c.txt
and the latest version of this license is in
   https://www.latex-project.org/lppl.txt
and version 1.3 or later is part of all distributions of
LaTeX version 2005/12/01 or later.

This work has the LPPL maintenance status "maintained".

The Current Maintainers of this work are
Heiko Oberdiek and the Oberdiek Package Support Group
https://github.com/ho-tex/luacolor/issues


This work consists of the main source file luacolor.dtx
and the derived files
   luacolor.sty, luacolor.pdf, luacolor.ins, luacolor.drv,
   luacolor.lua,

\endpreamble
\let\MetaPrefix\DoubleperCent

\generate{%
  \file{luacolor.ins}{\from{luacolor.dtx}{install}}%
  \file{luacolor.drv}{\from{luacolor.dtx}{driver}}%
  \usedir{tex/latex/luacolor}%
  \file{luacolor.sty}{\from{luacolor.dtx}{package}}%
}
\def\MetaPrefix{-- }
\def\defaultpostamble{%
  \MetaPrefix^^J%
  \MetaPrefix\space End of File `\outFileName'.%
}
\def\currentpostamble{\defaultpostamble}%
\generate{%
  \usedir{scripts/luacolor}%
%  \file{oberdiek.luacolor.lua}{\from{luacolor.dtx}{lua}}%
  \file{luacolor.lua}{\from{luacolor.dtx}{lua}}%
}

\catcode32=13\relax% active space
\let =\space%
\Msg{************************************************************************}
\Msg{*}
\Msg{* To finish the installation you have to move the following}
\Msg{* file into a directory searched by TeX:}
\Msg{*}
\Msg{*     luacolor.sty}
\Msg{*}
\Msg{* And install the following script file:}
\Msg{*}
\Msg{*     luacolor.lua,}
\Msg{*}
\Msg{* To produce the documentation run the file `luacolor.dtx'}
\Msg{* through LaTeX.}
\Msg{*}
\Msg{* Happy TeXing!}
\Msg{*}
\Msg{************************************************************************}

\endbatchfile
%</install>
%<*ignore>
\fi
%</ignore>
%<*driver>
\NeedsTeXFormat{LaTeX2e}
\ProvidesFile{luacolor.drv}%
  [2023-08-18 v1.18 Color support via LuaTeX's attributes (HO)]%
\documentclass{ltxdoc}
\usepackage{holtxdoc}[2011/11/22]
\usepackage{paralist}
\makeatletter
\g@addto@macro\MakePrivateLetters{%
  \@makeother\_%
}
\makeatother
\begin{document}
  \DocInput{luacolor.dtx}%
\end{document}
%</driver>
% \fi
%
%
%
% \GetFileInfo{luacolor.drv}
%
% \title{The \xpackage{luacolor} package}
% \date{2023-08-18 v1.18}
% \author{Heiko Oberdiek\thanks
% {Please report any issues at \url{https://github.com/ho-tex/luacolor/issues}}}
%
% \maketitle
%
% \begin{abstract}
% Package \xpackage{luacolor} implements color support based
% on \LuaTeX's node attributes.
% \end{abstract}
%
% \tableofcontents
%
% \section{Documentation}
%
% \subsection{Introduction}
%
% This package uses a \LuaTeX's attribute register to
% to annotate nodes with color information.
% If a color is set, then the attribute register
% is set to this color and all nodes created in its scope
% (current group) are annotated with this attribute.
% Now the color property behaves much the same way
% as the font property.
%
% \subsection{Usage}
%
% Package \xpackage{color} is loaded automatically by this
% package \xpackage{luacolor}. If you need a special driver
% option or you prefer package \xpackage{xcolor}, then load
% it before package \xpackage{luacolor}, for example:
% \begin{quote}
%   |\usepackage[dvipdfmx]{xcolor}|
% \end{quote}
% The package \xpackage{luacolor} is loaded without options:
% \begin{quote}
%   |\usepackage{luacolor}|
% \end{quote}
% It is able to detect PDF mode and DVI drivers are
% differentiated by its color specials. Therefore the
% package do need driver options.
%
% Then it redefines the color setting commands to set
% attributes instead of whatsits for color.
%
% At last the attribute annotations of the nodes in the
% output box must be analyzed to insert the necessary color whatsits.
% (With older \LuaTeX\ that lack the appropriate callback function 
% the package \xpackage{atbegshi} is used to get control
% before a box is shipped out.)
%
% \begin{declcs}{luacolorProcessBox} \M{box}
% \end{declcs}
% Macro \cs{luacolorProcessBox} processes the box \meta{box}
% in the previously described manner. It is automatically
% called for pages, but not for XForm objects.
% Before passing a box to \cs{pdfxform}, call \cs{luacolorProcessBox}
% first.
%
% \subsection{Limitations}
%
% \begin{description}
% \item[Ligatures with different colored components:]
%   Package \xpackage{luacolor} sees the ligature after
%   the paragraph building and page breaking, when a
%   page is to be shipped out. Therefore it cannot
%   break ligatures, because the components might occupy
%   different space. Therefore it is the responsibility
%   of the ligature forming process to deal with different
%   colored glyphs that form a ligature. The user can avoid
%   the problem entirely by explicitly breaking the ligature
%   at the places where the color changes.
%  \item \dots
% \end{description}
%
% \StopEventually{
% }
%
% \section{Implementation}
%
%    \begin{macrocode}
%<*package>
%    \end{macrocode}
%
% \subsection{Catcodes and identification}
%
%    \begin{macrocode}
\begingroup\catcode61\catcode48\catcode32=10\relax%
  \catcode13=5 % ^^M
  \endlinechar=13 %
  \catcode123=1 % {
  \catcode125=2 % }
  \catcode64=11 % @
  \def\x{\endgroup
    \expandafter\edef\csname LuaCol@AtEnd\endcsname{%
      \endlinechar=\the\endlinechar\relax
      \catcode13=\the\catcode13\relax
      \catcode32=\the\catcode32\relax
      \catcode35=\the\catcode35\relax
      \catcode61=\the\catcode61\relax
      \catcode64=\the\catcode64\relax
      \catcode123=\the\catcode123\relax
      \catcode125=\the\catcode125\relax
    }%
  }%
\x\catcode61\catcode48\catcode32=10\relax%
\catcode13=5 % ^^M
\endlinechar=13 %
\catcode35=6 % #
\catcode64=11 % @
\catcode123=1 % {
\catcode125=2 % }
\def\TMP@EnsureCode#1#2{%
  \edef\LuaCol@AtEnd{%
    \LuaCol@AtEnd
    \catcode#1=\the\catcode#1\relax
  }%
  \catcode#1=#2\relax
}
\TMP@EnsureCode{34}{12}% "
\TMP@EnsureCode{39}{12}% '
\TMP@EnsureCode{40}{12}% (
\TMP@EnsureCode{41}{12}% )
\TMP@EnsureCode{42}{12}% *
\TMP@EnsureCode{43}{12}% +
\TMP@EnsureCode{44}{12}% ,
\TMP@EnsureCode{45}{12}% -
\TMP@EnsureCode{46}{12}% .
\TMP@EnsureCode{47}{12}% /
\TMP@EnsureCode{58}{12}% :
\TMP@EnsureCode{60}{12}% <
\TMP@EnsureCode{62}{12}% >
\TMP@EnsureCode{91}{12}% [
\TMP@EnsureCode{93}{12}% ]
\TMP@EnsureCode{95}{12}% _ (other!)
\TMP@EnsureCode{96}{12}% `
\edef\LuaCol@AtEnd{\LuaCol@AtEnd\noexpand\endinput}
%    \end{macrocode}
%
%    Package identification.
%    \begin{macrocode}
\NeedsTeXFormat{LaTeX2e}
\ProvidesPackage{luacolor}%
  [2023-08-18 v1.18 Color support via LuaTeX's attributes (HO)]
%    \end{macrocode}
%
% \subsection{Check for \LuaTeX}
%
%    Without \LuaTeX\ there is no point in using this package.
%    \begin{macrocode}
\RequirePackage{color}
%    \end{macrocode}
%
%    \begin{macrocode}
\ifx\directlua\@undefined
  \PackageError{luacolor}{%
    This package may only be run using LuaTeX%
  }\@ehc
  \expandafter\LuaCol@AtEnd
\fi%
%    \end{macrocode}
%
% \subsection{Check for disabled colors}
%
%    \begin{macrocode}
\ifcolors@
\else
  \PackageWarningNoLine{luacolor}{%
    Colors are disabled by option `monochrome'%
  }%
  \def\set@color{}%
  \def\reset@color{}%
  \def\set@page@color{}%
  \def\define@color#1#2{}%
  \expandafter\LuaCol@AtEnd
\fi%
%    \end{macrocode}
%
% \subsection{Load module and check version}
%
%    \begin{macrocode}
\directlua{%
  require("luacolor")%
}
%    \end{macrocode}
%    \begin{macrocode}
\begingroup
  \edef\x{\directlua{tex.write("2023-08-18 v1.18")}}%
  \edef\y{%
    \directlua{%
      if oberdiek.luacolor.getversion then %
        oberdiek.luacolor.getversion()%
      end%
    }%
  }%
  \ifx\x\y
  \else
    \PackageError{luacolor}{%
      Wrong version of lua module.\MessageBreak
      Package version: \x\MessageBreak
      Lua module: \y
    }\@ehc
  \fi
\endgroup
%    \end{macrocode}
%
% \subsection{Find driver}
%
%    \begin{macrocode}
\ifnum\outputmode=\@ne
\else
  \begingroup
    \def\current@color{}%
    \def\reset@color{}%
    \setbox\z@=\hbox{%
      \begingroup
        \set@color
      \endgroup
    }%
    \edef\reserved@a{%
      \directlua{%
        oberdiek.luacolor.dvidetect()%
      }%
    }%
    \ifx\reserved@a\@empty
      \PackageError{luacolor}{%
        DVI driver detection failed because of\MessageBreak
        unrecognized color \string\special
      }\@ehc
      \endgroup
      \expandafter\expandafter\expandafter\LuaCol@AtEnd
    \else
      \PackageInfo{luacolor}{%
        Type of color \string\special: \reserved@a
      \@gobble}%
    \fi%
  \endgroup
\fi
%    \end{macrocode}
%
% \subsection{Attribute setting}
%
%    \begin{macro}{\LuaCol@Attribute}
%    \begin{macrocode}
\newattribute\LuaCol@Attribute
\let\LuaCol@setattribute\setattribute
\directlua{%
  oberdiek.luacolor.setattribute(\number\allocationnumber)%
}
%    \end{macrocode}
%    \end{macro}
%
%    \begin{macro}{\set@color}
% change 2023-08-18: added \cs{reset@color} so that \cs{mathcolor} can gobble it,
% issue 7.   
%    \begin{macrocode}
\protected\def\set@color{%
  \LuaCol@setattribute\LuaCol@Attribute{%
    \directlua{%
      oberdiek.luacolor.get("\luaescapestring{\current@color}")%
    }%
  }%
 \aftergroup\reset@color 
}
%    \end{macrocode}
%    \end{macro}
%    \begin{macro}{\reset@color}
%    \begin{macrocode}
\def\reset@color{}
%    \end{macrocode}
%    \end{macro}
%
% \subsection{Whatsit insertion}
%
%    \begin{macro}{\luacolorProcessBox}
%    \begin{macrocode}
\def\luacolorProcessBox#1{%
  \directlua{%
    oberdiek.luacolor.process(\number#1)%
  }%
}
%    \end{macrocode}
%    \end{macro}
%
%    \begin{macrocode}
\directlua{%
  if luatexbase.callbacktypes.pre_shipout_filter then
    token.get_next()
  end
}\@secondoftwo\@gobble{
  \RequirePackage{atbegshi}[2011/01/30]
  \AtBeginShipout{%
    \luacolorProcessBox\AtBeginShipoutBox
  }
}
%    \end{macrocode}
%
%    Set default color.
%    \begin{macrocode}
\set@color
%    \end{macrocode}
%
% \subsection{\cs{pdfxform}/\cs{saveboxresource} support}
%
%    \begin{macrocode}
\ifnum\outputmode=\@ne
    \let\LuaCol@org@pdfxform\saveboxresource
%    \end{macrocode}
% First we need some helpers to allow expandable code to parse keyword style arguments:
%    \begin{macrocode}
    \def\LuaCol@iii@i@ii#1#2#3{#3{#1}{#2}}
    \def\LuaCol@ii@i#1#2{{#2#1}}
    \def\LuaCol@if@keyword#1#2#3{%
      \expanded{\unexpanded{\LuaCol@iii@i@ii{#2}{#3}}\expandafter}%
      \directlua{%
        token.put_next(token.create(token.scan_keyword(token.scan_string())
        and '@firstoftwo'
        or '@secondoftwo'))
      }{#1}%
    }
%    \end{macrocode}
% The following macro scans a integer and expands to a token equivalent to a chardef
% whose value corresponds to the scanned integer. This allows the integer to be passed
% around as a undelimited argument.
%    \begin{macrocode}
    \def\LuaCol@scan@number{%
      \directlua{
        token.put_next(token.new(token.scan_int(), token.command_id'char_given'))
      }%
    }
%    \end{macrocode}
% \TeX\ primitives like \cs{saveboxresource} read braced arguments in a special way.
% Especially they expand everything until they find a left brace. To simulate this, we
% use Lua to expand everything else:
%    \begin{macrocode}
    \def\LuaCol@scan@tobrace{%
      \directlua{
        local relax, space = token.command_id'relax', token.command_id'spacer'
        local t
        repeat
          t = token.scan_token()
        until not (t.command == relax or t.command == space)
        token.put_next(t)
      }%
    }
    \def\LuaCol@scan@boxresource@i#1#2{%
      \LuaCol@if@keyword{attr}{%
        \expanded{\unexpanded{\LuaCol@scan@boxresource@iI{#1#2attr}}%
          \expandafter\expandafter\expandafter}%
        \LuaCol@scan@tobrace
      }{%
        \LuaCol@scan@boxresource@ii{#1#2}%
      }%
    }
    \def\LuaCol@scan@boxresource@iI#1#2{\LuaCol@scan@boxresource@ii{#1{#2}}}
    \def\LuaCol@scan@boxresource@ii#1{%
      \LuaCol@if@keyword{resources}{%
        \expanded{\unexpanded{\LuaCol@scan@boxresource@iiI{#1resources}}%
          \expandafter\expandafter\expandafter}%
        \LuaCol@scan@tobrace
      }{%
        \LuaCol@scan@boxresource@iii{#1}%
      }%
    }
    \def\LuaCol@scan@boxresource@iiI#1#2{\LuaCol@scan@boxresource@iii{#1{#2}}}
    \def\LuaCol@scan@boxresource@iii#1{%
      \LuaCol@if@keyword{margin}{%
        \expanded{\unexpanded{\LuaCol@scan@boxresource@iv{#1margin }}%
          \expandafter\expandafter\expandafter}%
        \LuaCol@scan@number
      }{%
        \LuaCol@scan@boxresource@iv{#1}{}%
      }%
    }
    \def\LuaCol@scan@boxresource@iv#1#2{%
      \expanded{\unexpanded{\LuaCol@scan@boxresource@v{#1#2}}%
        \expandafter\expandafter\expandafter}%
      \LuaCol@scan@number
    }
    \def\LuaCol@scan@boxresource@v#1#2{%
      \luacolorProcessBox{#2}%
      \LuaCol@org@pdfxform#1#2%
    }

%    \end{macrocode}
% This could be written in Lua, but at least upto Lua\TeX~1.11, feeding back too many tokens
% from Lua to \TeX\ triggers a segmentation fault.
% This is written in Lua so the integer setting is expandable and does not interfere with
% a preceding \verb|\immediate|.
%    \begin{macrocode}
    \protected\def\saveboxresource{%
      \LuaCol@if@keyword{type}{%
        \expandafter
        \expanded{\unexpanded{\LuaCol@scan@boxresource@i{type }}%
          \expandafter\expandafter\expandafter}%
        \LuaCol@scan@number
      }{%
        \LuaCol@scan@boxresource@i{}{}%
      }%
    }
%    \end{macrocode}
%    Legacy alias.
%    \begin{macrocode}
    \let\pdfxform\saveboxresource
\fi
%    \end{macrocode}
%
%    \begin{macrocode}
\LuaCol@AtEnd%
%</package>
%    \end{macrocode}
%
% \subsection{Lua module}
%
%    \begin{macrocode}
%<*lua>
%    \end{macrocode}
%    Box zero contains a \cs{hbox} with the color \cs{special}.
%    That is analyzed to get the prefix for the color setting
%    \cs{special}.
%    \begin{macrocode}
oberdiek = oberdiek or {}
local luacolor = oberdiek.luacolor or {}
oberdiek.luacolor = luacolor
%    \end{macrocode}
%    \begin{macro}{getversion()}
%    \begin{macrocode}
function luacolor.getversion()
  tex.write("2023-08-18 v1.18")
end
%    \end{macrocode}
%    \end{macro}
%
% \subsubsection{Driver detection}
%
%    \begin{macrocode}
local ifpdf = tonumber(tex.outputmode or tex.pdfoutput) > 0
local prefix
local prefixes = {
  dvips   = "color ",
  dvipdfm = "pdf:sc ",
  truetex = "textcolor:",
  pctexps = "ps::",
}
local patterns = {
  ["^color "]            = "dvips",
  ["^pdf: *begincolor "] = "dvipdfm",
  ["^pdf: *bcolor "]     = "dvipdfm",
  ["^pdf: *bc "]         = "dvipdfm",
  ["^pdf: *setcolor "]   = "dvipdfm",
  ["^pdf: *scolor "]     = "dvipdfm",
  ["^pdf: *sc "]         = "dvipdfm",
  ["^textcolor:"]        = "truetex",
  ["^ps::"]              = "pctexps",
}
%    \end{macrocode}
%    \begin{macro}{info()}
%    \begin{macrocode}
local function info(msg, term)
  local target = "log"
  if term then
    target = "term and log"
  end
  texio.write_nl(target, "Package luacolor info: " .. msg .. ".")
  texio.write_nl(target, "")
end
%    \end{macrocode}
%    \end{macro}
%    \begin{macro}{dvidetect()}
%    \begin{macrocode}
function luacolor.dvidetect()
  local v = tex.box[0]
  assert(v.id == node.id("hlist"))
  for v in node.traverse_id(node.id("whatsit"), v.head) do
    if v and v.subtype == node.subtype("special") then
      local data = v.data
      for pattern, driver in pairs(patterns) do
        if string.find(data, pattern) then
          prefix = prefixes[driver]
          tex.write(driver)
          return
        end
      end
      info("\\special{" .. data .. "}", true)
      return
    end
  end
  info("Missing \\special", true)
end
%    \end{macrocode}
%    \end{macro}
%
% \subsubsection{Color strings}
%
%    \begin{macrocode}
local map = {
  n = 0,
}
%    \end{macrocode}
%    \begin{macro}{get()}
%    \begin{macrocode}
function luacolor.get(color)
  tex.write("" .. luacolor.getvalue(color))
end
%    \end{macrocode}
%    \end{macro}
%    \begin{macro}{getvalue()}
%    \begin{macrocode}
function luacolor.getvalue(color)
  local n = map[color]
  if not n then
    n = map.n + 1
    map.n = n
    map[n] = color
    map[color] = n
  end
  return n
end
%    \end{macrocode}
%    \end{macro}
%
% \subsubsection{Attribute register}
%
%    \begin{macro}{setattribute()}
%    \begin{macrocode}
local attribute
function luacolor.setattribute(attr)
  attribute = attr
end
%    \end{macrocode}
%    \end{macro}
%
%    \begin{macro}{getattribute()}
%    \begin{macrocode}
function luacolor.getattribute()
  return attribute
end
%    \end{macrocode}
%    \end{macro}
%
% \subsubsection{Whatsit insertion}
%
%    \begin{macrocode}
local LIST = 1
local LIST_LEADERS = 2
local LIST_DISC = 3
local COLOR = 4
local NOCOLOR = 5
local RULE = node.id("rule")
local node_types = {
  [node.id("hlist")] = LIST,
  [node.id("vlist")] = LIST,
  [node.id("rule")]  = COLOR,
  [node.id("glyph")] = COLOR,
  [node.id("disc")]  = LIST_DISC,
  [node.id("whatsit")] = {
    [node.subtype("pdf_colorstack")] =
      function(n)
        return n.stack == 0 and NOCOLOR or nil
      end,
    [node.subtype("special")] = COLOR,
    [node.subtype("pdf_literal")] = COLOR,
    [node.subtype("pdf_save")] = COLOR,
    [node.subtype("pdf_restore")] = COLOR, -- probably not needed
-- TODO (DPC)    [node.subtype("pdf_refximage")] = COLOR,
  },
  [node.id("glue")] =
    function(n)
      if n.subtype >= 100 then -- leaders
        if n.leader.id == RULE then
          return COLOR
        else
          return LIST_LEADERS
        end
      end
    end,
}
%    \end{macrocode}
%    \begin{macro}{get_type()}
%    \begin{macrocode}
local function get_type(n)
  local ret = node_types[n.id]
  if type(ret) == 'table' then
    ret = ret[n.subtype]
  end
  if type(ret) == 'function' then
    ret = ret(n)
  end
  return ret
end
%    \end{macrocode}
%    \end{macro}
%    \begin{macrocode}
local mode = 2 -- luatex.pdfliteral.direct
local WHATSIT = node.id("whatsit")
local SPECIAL = node.subtype("special")
local PDFLITERAL = node.subtype("pdf_literal")
local DRY_FALSE = false
local DRY_TRUE = true
%    \end{macrocode}
%    \begin{macro}{traverse()}
%    \begin{macrocode}
local function traverse(list, color, dry)
  if not list then
    return color
  end
  local head
  if get_type(list) == LIST then
    head = list.head
  elseif get_type(list) == LIST_DISC then
    head = list.replace
  else
    texio.write_nl("!!! Error: Wrong list type: " .. node.type(list.id))
    return color
  end
%<debug>texio.write_nl("traverse: " .. node.type(list.id))
  for n in node.traverse(head) do
%<debug>texio.write_nl("  node: " .. node.type(n.id))
    local t = get_type(n)
%<debug>texio.write_nl("TYPE "..tostring(t).. " "..tostring(node.type(node.getid(n))).." ".. tostring(node.getsubtype(n)))
    if t == LIST or t == LIST_DISC then
      color = traverse(n, color, dry)
    elseif t == LIST_LEADERS then
      local color_after = traverse(n.leader, color, DRY_TRUE)
      if color == color_after then
        traverse(n.leader, color, DRY_FALSE or dry)
      else
        traverse(n.leader, '', DRY_FALSE or dry)
%    \end{macrocode}
% The color status is unknown here, because the leader box
% will or will not be set.
%    \begin{macrocode}
        color = ''
      end
    elseif t == COLOR then
      local v = node.has_attribute(n, attribute)
      if v then
        local newColor = map[v]
        if newColor ~= color then
          color = newColor
          if dry == DRY_FALSE then
            local newNode
            if ifpdf then
              newNode = node.new(WHATSIT, PDFLITERAL)
              newNode.mode = mode
              newNode.data = color
            else
              newNode = node.new(WHATSIT, SPECIAL)
              newNode.data = prefix .. color
            end
            head = node.insert_before(head, n, newNode)
          end
        end
      end
    elseif t == NOCOLOR then
      color = ''
    end
  end
  if get_type(list) == LIST then
    list.head = head
  else
    list.replace = head
  end
  return color
end
%    \end{macrocode}
%    \end{macro}
%
%    \begin{macro}{process()}
%    \begin{macrocode}
function luacolor.process(box)
  local color = ""
  local list = tex.getbox(box)
  traverse(list, color, DRY_FALSE)
end

if luatexbase.callbacktypes.pre_shipout_filter then
  luatexbase.add_to_callback("pre_shipout_filter", function(list)
    traverse(list, "", DRY_FALSE)
    return true
  end, "luacolor.process")
end
%    \end{macrocode}
%    \end{macro}
% For recent versions of luaotfload, we can register a callback to
% control how coloring glyph is handled for the color feature.
%    \begin{macrocode}
if luaotfload.set_colorhandler then
  local set_attribute = node.direct.set_attribute
  luaotfload.set_colorhandler(function(head, n, color)
    set_attribute(n, attribute, luacolor.getvalue(color))
    return head, n
  end)
end
%    \end{macrocode}
%
%    \begin{macrocode}
%</lua>
%    \end{macrocode}
% \section{Installation}
%
% \subsection{Download}
%
% \paragraph{Package.} This package is available on
% CTAN\footnote{\CTANpkg{luacolor}}:
% \begin{description}
% \item[\CTAN{macros/latex/contrib/luacolor/luacolor.dtx}] The source file.
% \item[\CTAN{macros/latex/contrib/luacolor/luacolor.pdf}] Documentation.
% \end{description}
%
%
% \paragraph{Bundle.} All the packages of the bundle `luacolor'
% are also available in a TDS compliant ZIP archive. There
% the packages are already unpacked and the documentation files
% are generated. The files and directories obey the TDS standard.
% \begin{description}
% \item[\CTANinstall{install/macros/latex/contrib/luacolor.tds.zip}]
% \end{description}
% \emph{TDS} refers to the standard ``A Directory Structure
% for \TeX\ Files'' (\CTANpkg{tds}). Directories
% with \xfile{texmf} in their name are usually organized this way.
%
% \subsection{Bundle installation}
%
% \paragraph{Unpacking.} Unpack the \xfile{luacolor.tds.zip} in the
% TDS tree (also known as \xfile{texmf} tree) of your choice.
% Example (linux):
% \begin{quote}
%   |unzip luacolor.tds.zip -d ~/texmf|
% \end{quote}
%
% \paragraph{Script installation.}
% Check the directory \xfile{TDS:scripts/luacolor/} for
% scripts that need further installation steps.
%
% \subsection{Package installation}
%
% \paragraph{Unpacking.} The \xfile{.dtx} file is a self-extracting
% \docstrip\ archive. The files are extracted by running the
% \xfile{.dtx} through \plainTeX:
% \begin{quote}
%   \verb|tex luacolor.dtx|
% \end{quote}
%
% \paragraph{TDS.} Now the different files must be moved into
% the different directories in your installation TDS tree
% (also known as \xfile{texmf} tree):
% \begin{quote}
% \def\t{^^A
% \begin{tabular}{@{}>{\ttfamily}l@{ $\rightarrow$ }>{\ttfamily}l@{}}
%   luacolor.sty & tex/latex/luacolor/luacolor.sty\\
%   luacolor.lua & scripts/luacolor/luacolor.lua\\
%   luacolor.pdf & doc/latex/luacolor/luacolor.pdf\\
%   luacolor.dtx & source/latex/luacolor/luacolor.dtx\\
% \end{tabular}^^A
% }^^A
% \sbox0{\t}^^A
% \ifdim\wd0>\linewidth
%   \begingroup
%     \advance\linewidth by\leftmargin
%     \advance\linewidth by\rightmargin
%   \edef\x{\endgroup
%     \def\noexpand\lw{\the\linewidth}^^A
%   }\x
%   \def\lwbox{^^A
%     \leavevmode
%     \hbox to \linewidth{^^A
%       \kern-\leftmargin\relax
%       \hss
%       \usebox0
%       \hss
%       \kern-\rightmargin\relax
%     }^^A
%   }^^A
%   \ifdim\wd0>\lw
%     \sbox0{\small\t}^^A
%     \ifdim\wd0>\linewidth
%       \ifdim\wd0>\lw
%         \sbox0{\footnotesize\t}^^A
%         \ifdim\wd0>\linewidth
%           \ifdim\wd0>\lw
%             \sbox0{\scriptsize\t}^^A
%             \ifdim\wd0>\linewidth
%               \ifdim\wd0>\lw
%                 \sbox0{\tiny\t}^^A
%                 \ifdim\wd0>\linewidth
%                   \lwbox
%                 \else
%                   \usebox0
%                 \fi
%               \else
%                 \lwbox
%               \fi
%             \else
%               \usebox0
%             \fi
%           \else
%             \lwbox
%           \fi
%         \else
%           \usebox0
%         \fi
%       \else
%         \lwbox
%       \fi
%     \else
%       \usebox0
%     \fi
%   \else
%     \lwbox
%   \fi
% \else
%   \usebox0
% \fi
% \end{quote}
% If you have a \xfile{docstrip.cfg} that configures and enables \docstrip's
% TDS installing feature, then some files can already be in the right
% place, see the documentation of \docstrip.
%
% \subsection{Refresh file name databases}
%
% If your \TeX~distribution
% (\TeX\,Live, \mikTeX, \dots) relies on file name databases, you must refresh
% these. For example, \TeX\,Live\ users run \verb|texhash| or
% \verb|mktexlsr|.
%
% \subsection{Some details for the interested}
%
% \paragraph{Unpacking with \LaTeX.}
% The \xfile{.dtx} chooses its action depending on the format:
% \begin{description}
% \item[\plainTeX:] Run \docstrip\ and extract the files.
% \item[\LaTeX:] Generate the documentation.
% \end{description}
% If you insist on using \LaTeX\ for \docstrip\ (really,
% \docstrip\ does not need \LaTeX), then inform the autodetect routine
% about your intention:
% \begin{quote}
%   \verb|latex \let\install=y\input{luacolor.dtx}|
% \end{quote}
% Do not forget to quote the argument according to the demands
% of your shell.
%
% \paragraph{Generating the documentation.}
% You can use both the \xfile{.dtx} or the \xfile{.drv} to generate
% the documentation. The process can be configured by the
% configuration file \xfile{ltxdoc.cfg}. For instance, put this
% line into this file, if you want to have A4 as paper format:
% \begin{quote}
%   \verb|\PassOptionsToClass{a4paper}{article}|
% \end{quote}
% An example follows how to generate the
% documentation with pdf\LaTeX:
% \begin{quote}
%\begin{verbatim}
%pdflatex luacolor.dtx
%makeindex -s gind.ist luacolor.idx
%pdflatex luacolor.dtx
%makeindex -s gind.ist luacolor.idx
%pdflatex luacolor.dtx
%\end{verbatim}
% \end{quote}
%
% \begin{History}
%   \begin{Version}{2007/12/12 v1.0}
%   \item
%     First public version.
%   \end{Version}
%   \begin{Version}{2009/04/10 v1.1}
%   \item
%     Fixes for changed syntax of \cs{directlua} in \LuaTeX\ 0.36.
%   \end{Version}
%   \begin{Version}{2010/03/09 v1.2}
%   \item
%     Adaptation for package \xpackage{luatex} 2010/03/09 v0.4.
%   \end{Version}
%   \begin{Version}{2010/12/13 v1.3}
%   \item
%     Support for \cs{pdfxform} added.
%   \item
%     Loaded package \xpackage{luatexbase-attr} recognized.
%   \item
%     Update for \hologo{LuaTeX}: `list' fields renamed to `head' in v0.65.0.
%   \end{Version}
%   \begin{Version}{2011/03/29 v1.4}
%   \item
%     Avoid whatsit insertion if option \xoption{monochrome} is used
%     (thanks Manuel P\'egouri\'e-Gonnard).
%   \end{Version}
%   \begin{Version}{2011/04/22 v1.5}
%   \item
%     Bug fix by Manuel P\'egouri\'e-Gonnard: A typo prevented the
%     detection of whatsits and applying color changes for
%     \cs{pdfliteral} and \cs{special} nodes that might contain
%     typesetting material.
%   \item
%     Bug fix by Manuel P\'egouri\'e-Gonnard: Now colors are also
%     applied to leader boxes.
%   \item
%     Unnecessary color settings are removed for leaders boxes,
%     if after the leader box the color has not changed.
%     The costs are a little runtime, leader boxes are processed twice.
%   \item
%     Additional whatsits that are colored: \texttt{pdf\_refximage}.
%   \item
%     Workaround for bug with \texttt{node.insert\_before}
%     removed for the version after \hologo{LuaTeX} 0.65, because
%     bug was fixed in 0.27. (Thanks Manuel P\'egouri\'e-Gonnard.)
%   \end{Version}
%   \begin{Version}{2011/04/23 v1.6}
%   \item
%     Bug fix for nested leader boxes.
%   \item
%     Bug fix for leader boxes that change color, but
%     are not set because of missing place.
%   \item
%     Version check for Lua module added.
%   \end{Version}
%   \begin{Version}{2011/10/22 v1.7}
%   \item
%     Lua functions \texttt{getattribute} and \texttt{getvalue}
%     added to tell other external Lua functions
%     the attribute register number for coloring.
%   \end{Version}
%   \begin{Version}{2011/11/01 v1.8}
%   \item
%     Use of \texttt{node.subtype} instead of magic numbers.
%   \end{Version}
%   \begin{Version}{2016/05/13 v1.9}
%   \item
%     More use of \texttt{node.subtype} instead of magic numbers.
%   \item luatex 85 updates
%   \end{Version}
%   \begin{Version}{2016/05/16 v1.10}
%   \item
%     Documentation updates.
%   \end{Version}
%   \begin{Version}{2018/11/22 v1.11}
%   \item
%     handle issue 43.
%   \item
%     removed pre-0.65 stuff
%   \end{Version}
%   \begin{Version}{2019/07/25 v1.12}
%   \item
%     removed uses of module function, see PR70
%   \end{Version}
%   \begin{Version}{2019/11/29 v1.13}
%   \item
%     Documentation updates.
%   \item  Use \xpackage{iftex} directly.
%   \end{Version}
%   \begin{Version}{2020-02-22 v1.14}
%   \item Drop use of \textsf{iftex} \textsf{ltxcmds} and \textsf{infwarerr}.
%   \item Assume \textsf{ltluatex} preloaded into format (true since 2015).
%   \item Patch \verb|\saveboxresource| rather than \verb|\pdfxform| (keep old name as alias).
%   \item Grab the number via Lua so that a \verb|\immediate| prefix still works with
%         \verb|\saveboxresource|\slash \verb|\pdfxform|.
%   \item Added handler for the color feature of luaotfload      
%   \end{Version}
%   \begin{Version}{2020-02-24 v1.15}
%   \item Grab all possible arguments for \verb|\saveboxresource|\slash \verb|\pdfxform|
%   \end{Version}
%   \begin{Version}{2020-04-04 v1.16}
%   \item Reset color after \verb|pdf_colorstack| whatsits.
%   \end{Version}
%   \begin{Version}{2021-02-17 v1.17}
%   \item Use \LaTeXe's new \verb|pre_shipout_filter| callback if it's available to
%         allow coloring background and foreground layer material
%   \end{Version}
%   \begin{Version}{2023-08-18 v1.18}
%   \item added \verb|\reset@color| to \verb|\set@color| for \verb|\mathcolor|, issue 7.
%   \end{Version}
% \end{History}
%
% \PrintIndex
%
% \Finale
\endinput
|
% \end{quote}
% Do not forget to quote the argument according to the demands
% of your shell.
%
% \paragraph{Generating the documentation.}
% You can use both the \xfile{.dtx} or the \xfile{.drv} to generate
% the documentation. The process can be configured by the
% configuration file \xfile{ltxdoc.cfg}. For instance, put this
% line into this file, if you want to have A4 as paper format:
% \begin{quote}
%   \verb|\PassOptionsToClass{a4paper}{article}|
% \end{quote}
% An example follows how to generate the
% documentation with pdf\LaTeX:
% \begin{quote}
%\begin{verbatim}
%pdflatex luacolor.dtx
%makeindex -s gind.ist luacolor.idx
%pdflatex luacolor.dtx
%makeindex -s gind.ist luacolor.idx
%pdflatex luacolor.dtx
%\end{verbatim}
% \end{quote}
%
% \begin{History}
%   \begin{Version}{2007/12/12 v1.0}
%   \item
%     First public version.
%   \end{Version}
%   \begin{Version}{2009/04/10 v1.1}
%   \item
%     Fixes for changed syntax of \cs{directlua} in \LuaTeX\ 0.36.
%   \end{Version}
%   \begin{Version}{2010/03/09 v1.2}
%   \item
%     Adaptation for package \xpackage{luatex} 2010/03/09 v0.4.
%   \end{Version}
%   \begin{Version}{2010/12/13 v1.3}
%   \item
%     Support for \cs{pdfxform} added.
%   \item
%     Loaded package \xpackage{luatexbase-attr} recognized.
%   \item
%     Update for \hologo{LuaTeX}: `list' fields renamed to `head' in v0.65.0.
%   \end{Version}
%   \begin{Version}{2011/03/29 v1.4}
%   \item
%     Avoid whatsit insertion if option \xoption{monochrome} is used
%     (thanks Manuel P\'egouri\'e-Gonnard).
%   \end{Version}
%   \begin{Version}{2011/04/22 v1.5}
%   \item
%     Bug fix by Manuel P\'egouri\'e-Gonnard: A typo prevented the
%     detection of whatsits and applying color changes for
%     \cs{pdfliteral} and \cs{special} nodes that might contain
%     typesetting material.
%   \item
%     Bug fix by Manuel P\'egouri\'e-Gonnard: Now colors are also
%     applied to leader boxes.
%   \item
%     Unnecessary color settings are removed for leaders boxes,
%     if after the leader box the color has not changed.
%     The costs are a little runtime, leader boxes are processed twice.
%   \item
%     Additional whatsits that are colored: \texttt{pdf\_refximage}.
%   \item
%     Workaround for bug with \texttt{node.insert\_before}
%     removed for the version after \hologo{LuaTeX} 0.65, because
%     bug was fixed in 0.27. (Thanks Manuel P\'egouri\'e-Gonnard.)
%   \end{Version}
%   \begin{Version}{2011/04/23 v1.6}
%   \item
%     Bug fix for nested leader boxes.
%   \item
%     Bug fix for leader boxes that change color, but
%     are not set because of missing place.
%   \item
%     Version check for Lua module added.
%   \end{Version}
%   \begin{Version}{2011/10/22 v1.7}
%   \item
%     Lua functions \texttt{getattribute} and \texttt{getvalue}
%     added to tell other external Lua functions
%     the attribute register number for coloring.
%   \end{Version}
%   \begin{Version}{2011/11/01 v1.8}
%   \item
%     Use of \texttt{node.subtype} instead of magic numbers.
%   \end{Version}
%   \begin{Version}{2016/05/13 v1.9}
%   \item
%     More use of \texttt{node.subtype} instead of magic numbers.
%   \item luatex 85 updates
%   \end{Version}
%   \begin{Version}{2016/05/16 v1.10}
%   \item
%     Documentation updates.
%   \end{Version}
%   \begin{Version}{2018/11/22 v1.11}
%   \item
%     handle issue 43.
%   \item
%     removed pre-0.65 stuff
%   \end{Version}
%   \begin{Version}{2019/07/25 v1.12}
%   \item
%     removed uses of module function, see PR70
%   \end{Version}
%   \begin{Version}{2019/11/29 v1.13}
%   \item
%     Documentation updates.
%   \item  Use \xpackage{iftex} directly.
%   \end{Version}
%   \begin{Version}{2020-02-22 v1.14}
%   \item Drop use of \textsf{iftex} \textsf{ltxcmds} and \textsf{infwarerr}.
%   \item Assume \textsf{ltluatex} preloaded into format (true since 2015).
%   \item Patch \verb|\saveboxresource| rather than \verb|\pdfxform| (keep old name as alias).
%   \item Grab the number via Lua so that a \verb|\immediate| prefix still works with
%         \verb|\saveboxresource|\slash \verb|\pdfxform|.
%   \item Added handler for the color feature of luaotfload      
%   \end{Version}
%   \begin{Version}{2020-02-24 v1.15}
%   \item Grab all possible arguments for \verb|\saveboxresource|\slash \verb|\pdfxform|
%   \end{Version}
%   \begin{Version}{2020-04-04 v1.16}
%   \item Reset color after \verb|pdf_colorstack| whatsits.
%   \end{Version}
%   \begin{Version}{2021-02-17 v1.17}
%   \item Use \LaTeXe's new \verb|pre_shipout_filter| callback if it's available to
%         allow coloring background and foreground layer material
%   \end{Version}
%   \begin{Version}{2023-08-18 v1.18}
%   \item added \verb|\reset@color| to \verb|\set@color| for \verb|\mathcolor|, issue 7.
%   \end{Version}
% \end{History}
%
% \PrintIndex
%
% \Finale
\endinput
|
% \end{quote}
% Do not forget to quote the argument according to the demands
% of your shell.
%
% \paragraph{Generating the documentation.}
% You can use both the \xfile{.dtx} or the \xfile{.drv} to generate
% the documentation. The process can be configured by the
% configuration file \xfile{ltxdoc.cfg}. For instance, put this
% line into this file, if you want to have A4 as paper format:
% \begin{quote}
%   \verb|\PassOptionsToClass{a4paper}{article}|
% \end{quote}
% An example follows how to generate the
% documentation with pdf\LaTeX:
% \begin{quote}
%\begin{verbatim}
%pdflatex luacolor.dtx
%makeindex -s gind.ist luacolor.idx
%pdflatex luacolor.dtx
%makeindex -s gind.ist luacolor.idx
%pdflatex luacolor.dtx
%\end{verbatim}
% \end{quote}
%
% \begin{History}
%   \begin{Version}{2007/12/12 v1.0}
%   \item
%     First public version.
%   \end{Version}
%   \begin{Version}{2009/04/10 v1.1}
%   \item
%     Fixes for changed syntax of \cs{directlua} in \LuaTeX\ 0.36.
%   \end{Version}
%   \begin{Version}{2010/03/09 v1.2}
%   \item
%     Adaptation for package \xpackage{luatex} 2010/03/09 v0.4.
%   \end{Version}
%   \begin{Version}{2010/12/13 v1.3}
%   \item
%     Support for \cs{pdfxform} added.
%   \item
%     Loaded package \xpackage{luatexbase-attr} recognized.
%   \item
%     Update for \hologo{LuaTeX}: `list' fields renamed to `head' in v0.65.0.
%   \end{Version}
%   \begin{Version}{2011/03/29 v1.4}
%   \item
%     Avoid whatsit insertion if option \xoption{monochrome} is used
%     (thanks Manuel P\'egouri\'e-Gonnard).
%   \end{Version}
%   \begin{Version}{2011/04/22 v1.5}
%   \item
%     Bug fix by Manuel P\'egouri\'e-Gonnard: A typo prevented the
%     detection of whatsits and applying color changes for
%     \cs{pdfliteral} and \cs{special} nodes that might contain
%     typesetting material.
%   \item
%     Bug fix by Manuel P\'egouri\'e-Gonnard: Now colors are also
%     applied to leader boxes.
%   \item
%     Unnecessary color settings are removed for leaders boxes,
%     if after the leader box the color has not changed.
%     The costs are a little runtime, leader boxes are processed twice.
%   \item
%     Additional whatsits that are colored: \texttt{pdf\_refximage}.
%   \item
%     Workaround for bug with \texttt{node.insert\_before}
%     removed for the version after \hologo{LuaTeX} 0.65, because
%     bug was fixed in 0.27. (Thanks Manuel P\'egouri\'e-Gonnard.)
%   \end{Version}
%   \begin{Version}{2011/04/23 v1.6}
%   \item
%     Bug fix for nested leader boxes.
%   \item
%     Bug fix for leader boxes that change color, but
%     are not set because of missing place.
%   \item
%     Version check for Lua module added.
%   \end{Version}
%   \begin{Version}{2011/10/22 v1.7}
%   \item
%     Lua functions \texttt{getattribute} and \texttt{getvalue}
%     added to tell other external Lua functions
%     the attribute register number for coloring.
%   \end{Version}
%   \begin{Version}{2011/11/01 v1.8}
%   \item
%     Use of \texttt{node.subtype} instead of magic numbers.
%   \end{Version}
%   \begin{Version}{2016/05/13 v1.9}
%   \item
%     More use of \texttt{node.subtype} instead of magic numbers.
%   \item luatex 85 updates
%   \end{Version}
%   \begin{Version}{2016/05/16 v1.10}
%   \item
%     Documentation updates.
%   \end{Version}
%   \begin{Version}{2018/11/22 v1.11}
%   \item
%     handle issue 43.
%   \item
%     removed pre-0.65 stuff
%   \end{Version}
%   \begin{Version}{2019/07/25 v1.12}
%   \item
%     removed uses of module function, see PR70
%   \end{Version}
%   \begin{Version}{2019/11/29 v1.13}
%   \item
%     Documentation updates.
%   \item  Use \xpackage{iftex} directly.
%   \end{Version}
%   \begin{Version}{2020-02-22 v1.14}
%   \item Drop use of \textsf{iftex} \textsf{ltxcmds} and \textsf{infwarerr}.
%   \item Assume \textsf{ltluatex} preloaded into format (true since 2015).
%   \item Patch \verb|\saveboxresource| rather than \verb|\pdfxform| (keep old name as alias).
%   \item Grab the number via Lua so that a \verb|\immediate| prefix still works with
%         \verb|\saveboxresource|\slash \verb|\pdfxform|.
%   \item Added handler for the color feature of luaotfload      
%   \end{Version}
%   \begin{Version}{2020-02-24 v1.15}
%   \item Grab all possible arguments for \verb|\saveboxresource|\slash \verb|\pdfxform|
%   \end{Version}
%   \begin{Version}{2020-04-04 v1.16}
%   \item Reset color after \verb|pdf_colorstack| whatsits.
%   \end{Version}
%   \begin{Version}{2021-02-17 v1.17}
%   \item Use \LaTeXe's new \verb|pre_shipout_filter| callback if it's available to
%         allow coloring background and foreground layer material
%   \end{Version}
%   \begin{Version}{2023-08-18 v1.18}
%   \item added \verb|\reset@color| to \verb|\set@color| for \verb|\mathcolor|, issue 7.
%   \end{Version}
% \end{History}
%
% \PrintIndex
%
% \Finale
\endinput

%        (quote the arguments according to the demands of your shell)
%
% Documentation:
%    (a) If luacolor.drv is present:
%           latex luacolor.drv
%    (b) Without luacolor.drv:
%           latex luacolor.dtx; ...
%    The class ltxdoc loads the configuration file ltxdoc.cfg
%    if available. Here you can specify further options, e.g.
%    use A4 as paper format:
%       \PassOptionsToClass{a4paper}{article}
%
%    Programm calls to get the documentation (example):
%       pdflatex luacolor.dtx
%       makeindex -s gind.ist luacolor.idx
%       pdflatex luacolor.dtx
%       makeindex -s gind.ist luacolor.idx
%       pdflatex luacolor.dtx
%
% Installation:
%    TDS:tex/latex/luacolor/luacolor.sty
%    TDS:scripts/luacolor/luacolor.lua
%    TDS:doc/latex/luacolor/luacolor.pdf
%    TDS:source/latex/luacolor/luacolor.dtx
%
%<*ignore>
\begingroup
  \catcode123=1 %
  \catcode125=2 %
  \def\x{LaTeX2e}%
\expandafter\endgroup
\ifcase 0\ifx\install y1\fi\expandafter
         \ifx\csname processbatchFile\endcsname\relax\else1\fi
         \ifx\fmtname\x\else 1\fi\relax
\else\csname fi\endcsname
%</ignore>
%<*install>
\input docstrip.tex
\Msg{************************************************************************}
\Msg{* Installation}
\Msg{* Package: luacolor 2023-08-18 v1.18 Color support via LuaTeX's attributes (HO)}
\Msg{************************************************************************}

\keepsilent
\askforoverwritefalse

\let\MetaPrefix\relax
\preamble

This is a generated file.

Project: luacolor
Version: 2023-08-18 v1.18

Copyright (C)
   2007, 2009-2011 Heiko Oberdiek
   2016-2023 Oberdiek Package Support Group

This work may be distributed and/or modified under the
conditions of the LaTeX Project Public License, either
version 1.3c of this license or (at your option) any later
version. This version of this license is in
   https://www.latex-project.org/lppl/lppl-1-3c.txt
and the latest version of this license is in
   https://www.latex-project.org/lppl.txt
and version 1.3 or later is part of all distributions of
LaTeX version 2005/12/01 or later.

This work has the LPPL maintenance status "maintained".

The Current Maintainers of this work are
Heiko Oberdiek and the Oberdiek Package Support Group
https://github.com/ho-tex/luacolor/issues


This work consists of the main source file luacolor.dtx
and the derived files
   luacolor.sty, luacolor.pdf, luacolor.ins, luacolor.drv,
   luacolor.lua,

\endpreamble
\let\MetaPrefix\DoubleperCent

\generate{%
  \file{luacolor.ins}{\from{luacolor.dtx}{install}}%
  \file{luacolor.drv}{\from{luacolor.dtx}{driver}}%
  \usedir{tex/latex/luacolor}%
  \file{luacolor.sty}{\from{luacolor.dtx}{package}}%
}
\def\MetaPrefix{-- }
\def\defaultpostamble{%
  \MetaPrefix^^J%
  \MetaPrefix\space End of File `\outFileName'.%
}
\def\currentpostamble{\defaultpostamble}%
\generate{%
  \usedir{scripts/luacolor}%
%  \file{oberdiek.luacolor.lua}{\from{luacolor.dtx}{lua}}%
  \file{luacolor.lua}{\from{luacolor.dtx}{lua}}%
}

\catcode32=13\relax% active space
\let =\space%
\Msg{************************************************************************}
\Msg{*}
\Msg{* To finish the installation you have to move the following}
\Msg{* file into a directory searched by TeX:}
\Msg{*}
\Msg{*     luacolor.sty}
\Msg{*}
\Msg{* And install the following script file:}
\Msg{*}
\Msg{*     luacolor.lua,}
\Msg{*}
\Msg{* To produce the documentation run the file `luacolor.dtx'}
\Msg{* through LaTeX.}
\Msg{*}
\Msg{* Happy TeXing!}
\Msg{*}
\Msg{************************************************************************}

\endbatchfile
%</install>
%<*ignore>
\fi
%</ignore>
%<*driver>
\NeedsTeXFormat{LaTeX2e}
\ProvidesFile{luacolor.drv}%
  [2023-08-18 v1.18 Color support via LuaTeX's attributes (HO)]%
\documentclass{ltxdoc}
\usepackage{holtxdoc}[2011/11/22]
\usepackage{paralist}
\makeatletter
\g@addto@macro\MakePrivateLetters{%
  \@makeother\_%
}
\makeatother
\begin{document}
  \DocInput{luacolor.dtx}%
\end{document}
%</driver>
% \fi
%
%
%
% \GetFileInfo{luacolor.drv}
%
% \title{The \xpackage{luacolor} package}
% \date{2023-08-18 v1.18}
% \author{Heiko Oberdiek\thanks
% {Please report any issues at \url{https://github.com/ho-tex/luacolor/issues}}}
%
% \maketitle
%
% \begin{abstract}
% Package \xpackage{luacolor} implements color support based
% on \LuaTeX's node attributes.
% \end{abstract}
%
% \tableofcontents
%
% \section{Documentation}
%
% \subsection{Introduction}
%
% This package uses a \LuaTeX's attribute register to
% to annotate nodes with color information.
% If a color is set, then the attribute register
% is set to this color and all nodes created in its scope
% (current group) are annotated with this attribute.
% Now the color property behaves much the same way
% as the font property.
%
% \subsection{Usage}
%
% Package \xpackage{color} is loaded automatically by this
% package \xpackage{luacolor}. If you need a special driver
% option or you prefer package \xpackage{xcolor}, then load
% it before package \xpackage{luacolor}, for example:
% \begin{quote}
%   |\usepackage[dvipdfmx]{xcolor}|
% \end{quote}
% The package \xpackage{luacolor} is loaded without options:
% \begin{quote}
%   |\usepackage{luacolor}|
% \end{quote}
% It is able to detect PDF mode and DVI drivers are
% differentiated by its color specials. Therefore the
% package do need driver options.
%
% Then it redefines the color setting commands to set
% attributes instead of whatsits for color.
%
% At last the attribute annotations of the nodes in the
% output box must be analyzed to insert the necessary color whatsits.
% (With older \LuaTeX\ that lack the appropriate callback function 
% the package \xpackage{atbegshi} is used to get control
% before a box is shipped out.)
%
% \begin{declcs}{luacolorProcessBox} \M{box}
% \end{declcs}
% Macro \cs{luacolorProcessBox} processes the box \meta{box}
% in the previously described manner. It is automatically
% called for pages, but not for XForm objects.
% Before passing a box to \cs{pdfxform}, call \cs{luacolorProcessBox}
% first.
%
% \subsection{Limitations}
%
% \begin{description}
% \item[Ligatures with different colored components:]
%   Package \xpackage{luacolor} sees the ligature after
%   the paragraph building and page breaking, when a
%   page is to be shipped out. Therefore it cannot
%   break ligatures, because the components might occupy
%   different space. Therefore it is the responsibility
%   of the ligature forming process to deal with different
%   colored glyphs that form a ligature. The user can avoid
%   the problem entirely by explicitly breaking the ligature
%   at the places where the color changes.
%  \item \dots
% \end{description}
%
% \StopEventually{
% }
%
% \section{Implementation}
%
%    \begin{macrocode}
%<*package>
%    \end{macrocode}
%
% \subsection{Catcodes and identification}
%
%    \begin{macrocode}
\begingroup\catcode61\catcode48\catcode32=10\relax%
  \catcode13=5 % ^^M
  \endlinechar=13 %
  \catcode123=1 % {
  \catcode125=2 % }
  \catcode64=11 % @
  \def\x{\endgroup
    \expandafter\edef\csname LuaCol@AtEnd\endcsname{%
      \endlinechar=\the\endlinechar\relax
      \catcode13=\the\catcode13\relax
      \catcode32=\the\catcode32\relax
      \catcode35=\the\catcode35\relax
      \catcode61=\the\catcode61\relax
      \catcode64=\the\catcode64\relax
      \catcode123=\the\catcode123\relax
      \catcode125=\the\catcode125\relax
    }%
  }%
\x\catcode61\catcode48\catcode32=10\relax%
\catcode13=5 % ^^M
\endlinechar=13 %
\catcode35=6 % #
\catcode64=11 % @
\catcode123=1 % {
\catcode125=2 % }
\def\TMP@EnsureCode#1#2{%
  \edef\LuaCol@AtEnd{%
    \LuaCol@AtEnd
    \catcode#1=\the\catcode#1\relax
  }%
  \catcode#1=#2\relax
}
\TMP@EnsureCode{34}{12}% "
\TMP@EnsureCode{39}{12}% '
\TMP@EnsureCode{40}{12}% (
\TMP@EnsureCode{41}{12}% )
\TMP@EnsureCode{42}{12}% *
\TMP@EnsureCode{43}{12}% +
\TMP@EnsureCode{44}{12}% ,
\TMP@EnsureCode{45}{12}% -
\TMP@EnsureCode{46}{12}% .
\TMP@EnsureCode{47}{12}% /
\TMP@EnsureCode{58}{12}% :
\TMP@EnsureCode{60}{12}% <
\TMP@EnsureCode{62}{12}% >
\TMP@EnsureCode{91}{12}% [
\TMP@EnsureCode{93}{12}% ]
\TMP@EnsureCode{95}{12}% _ (other!)
\TMP@EnsureCode{96}{12}% `
\edef\LuaCol@AtEnd{\LuaCol@AtEnd\noexpand\endinput}
%    \end{macrocode}
%
%    Package identification.
%    \begin{macrocode}
\NeedsTeXFormat{LaTeX2e}
\ProvidesPackage{luacolor}%
  [2023-08-18 v1.18 Color support via LuaTeX's attributes (HO)]
%    \end{macrocode}
%
% \subsection{Check for \LuaTeX}
%
%    Without \LuaTeX\ there is no point in using this package.
%    \begin{macrocode}
\RequirePackage{color}
%    \end{macrocode}
%
%    \begin{macrocode}
\ifx\directlua\@undefined
  \PackageError{luacolor}{%
    This package may only be run using LuaTeX%
  }\@ehc
  \expandafter\LuaCol@AtEnd
\fi%
%    \end{macrocode}
%
% \subsection{Check for disabled colors}
%
%    \begin{macrocode}
\ifcolors@
\else
  \PackageWarningNoLine{luacolor}{%
    Colors are disabled by option `monochrome'%
  }%
  \def\set@color{}%
  \def\reset@color{}%
  \def\set@page@color{}%
  \def\define@color#1#2{}%
  \expandafter\LuaCol@AtEnd
\fi%
%    \end{macrocode}
%
% \subsection{Load module and check version}
%
%    \begin{macrocode}
\directlua{%
  require("luacolor")%
}
%    \end{macrocode}
%    \begin{macrocode}
\begingroup
  \edef\x{\directlua{tex.write("2023-08-18 v1.18")}}%
  \edef\y{%
    \directlua{%
      if oberdiek.luacolor.getversion then %
        oberdiek.luacolor.getversion()%
      end%
    }%
  }%
  \ifx\x\y
  \else
    \PackageError{luacolor}{%
      Wrong version of lua module.\MessageBreak
      Package version: \x\MessageBreak
      Lua module: \y
    }\@ehc
  \fi
\endgroup
%    \end{macrocode}
%
% \subsection{Find driver}
%
%    \begin{macrocode}
\ifnum\outputmode=\@ne
\else
  \begingroup
    \def\current@color{}%
    \def\reset@color{}%
    \setbox\z@=\hbox{%
      \begingroup
        \set@color
      \endgroup
    }%
    \edef\reserved@a{%
      \directlua{%
        oberdiek.luacolor.dvidetect()%
      }%
    }%
    \ifx\reserved@a\@empty
      \PackageError{luacolor}{%
        DVI driver detection failed because of\MessageBreak
        unrecognized color \string\special
      }\@ehc
      \endgroup
      \expandafter\expandafter\expandafter\LuaCol@AtEnd
    \else
      \PackageInfo{luacolor}{%
        Type of color \string\special: \reserved@a
      \@gobble}%
    \fi%
  \endgroup
\fi
%    \end{macrocode}
%
% \subsection{Attribute setting}
%
%    \begin{macro}{\LuaCol@Attribute}
%    \begin{macrocode}
\newattribute\LuaCol@Attribute
\let\LuaCol@setattribute\setattribute
\directlua{%
  oberdiek.luacolor.setattribute(\number\allocationnumber)%
}
%    \end{macrocode}
%    \end{macro}
%
%    \begin{macro}{\set@color}
% change 2023-08-18: added \cs{reset@color} so that \cs{mathcolor} can gobble it,
% issue 7.   
%    \begin{macrocode}
\protected\def\set@color{%
  \LuaCol@setattribute\LuaCol@Attribute{%
    \directlua{%
      oberdiek.luacolor.get("\luaescapestring{\current@color}")%
    }%
  }%
 \aftergroup\reset@color 
}
%    \end{macrocode}
%    \end{macro}
%    \begin{macro}{\reset@color}
%    \begin{macrocode}
\def\reset@color{}
%    \end{macrocode}
%    \end{macro}
%
% \subsection{Whatsit insertion}
%
%    \begin{macro}{\luacolorProcessBox}
%    \begin{macrocode}
\def\luacolorProcessBox#1{%
  \directlua{%
    oberdiek.luacolor.process(\number#1)%
  }%
}
%    \end{macrocode}
%    \end{macro}
%
%    \begin{macrocode}
\directlua{%
  if luatexbase.callbacktypes.pre_shipout_filter then
    token.get_next()
  end
}\@secondoftwo\@gobble{
  \RequirePackage{atbegshi}[2011/01/30]
  \AtBeginShipout{%
    \luacolorProcessBox\AtBeginShipoutBox
  }
}
%    \end{macrocode}
%
%    Set default color.
%    \begin{macrocode}
\set@color
%    \end{macrocode}
%
% \subsection{\cs{pdfxform}/\cs{saveboxresource} support}
%
%    \begin{macrocode}
\ifnum\outputmode=\@ne
    \let\LuaCol@org@pdfxform\saveboxresource
%    \end{macrocode}
% First we need some helpers to allow expandable code to parse keyword style arguments:
%    \begin{macrocode}
    \def\LuaCol@iii@i@ii#1#2#3{#3{#1}{#2}}
    \def\LuaCol@ii@i#1#2{{#2#1}}
    \def\LuaCol@if@keyword#1#2#3{%
      \expanded{\unexpanded{\LuaCol@iii@i@ii{#2}{#3}}\expandafter}%
      \directlua{%
        token.put_next(token.create(token.scan_keyword(token.scan_string())
        and '@firstoftwo'
        or '@secondoftwo'))
      }{#1}%
    }
%    \end{macrocode}
% The following macro scans a integer and expands to a token equivalent to a chardef
% whose value corresponds to the scanned integer. This allows the integer to be passed
% around as a undelimited argument.
%    \begin{macrocode}
    \def\LuaCol@scan@number{%
      \directlua{
        token.put_next(token.new(token.scan_int(), token.command_id'char_given'))
      }%
    }
%    \end{macrocode}
% \TeX\ primitives like \cs{saveboxresource} read braced arguments in a special way.
% Especially they expand everything until they find a left brace. To simulate this, we
% use Lua to expand everything else:
%    \begin{macrocode}
    \def\LuaCol@scan@tobrace{%
      \directlua{
        local relax, space = token.command_id'relax', token.command_id'spacer'
        local t
        repeat
          t = token.scan_token()
        until not (t.command == relax or t.command == space)
        token.put_next(t)
      }%
    }
    \def\LuaCol@scan@boxresource@i#1#2{%
      \LuaCol@if@keyword{attr}{%
        \expanded{\unexpanded{\LuaCol@scan@boxresource@iI{#1#2attr}}%
          \expandafter\expandafter\expandafter}%
        \LuaCol@scan@tobrace
      }{%
        \LuaCol@scan@boxresource@ii{#1#2}%
      }%
    }
    \def\LuaCol@scan@boxresource@iI#1#2{\LuaCol@scan@boxresource@ii{#1{#2}}}
    \def\LuaCol@scan@boxresource@ii#1{%
      \LuaCol@if@keyword{resources}{%
        \expanded{\unexpanded{\LuaCol@scan@boxresource@iiI{#1resources}}%
          \expandafter\expandafter\expandafter}%
        \LuaCol@scan@tobrace
      }{%
        \LuaCol@scan@boxresource@iii{#1}%
      }%
    }
    \def\LuaCol@scan@boxresource@iiI#1#2{\LuaCol@scan@boxresource@iii{#1{#2}}}
    \def\LuaCol@scan@boxresource@iii#1{%
      \LuaCol@if@keyword{margin}{%
        \expanded{\unexpanded{\LuaCol@scan@boxresource@iv{#1margin }}%
          \expandafter\expandafter\expandafter}%
        \LuaCol@scan@number
      }{%
        \LuaCol@scan@boxresource@iv{#1}{}%
      }%
    }
    \def\LuaCol@scan@boxresource@iv#1#2{%
      \expanded{\unexpanded{\LuaCol@scan@boxresource@v{#1#2}}%
        \expandafter\expandafter\expandafter}%
      \LuaCol@scan@number
    }
    \def\LuaCol@scan@boxresource@v#1#2{%
      \luacolorProcessBox{#2}%
      \LuaCol@org@pdfxform#1#2%
    }

%    \end{macrocode}
% This could be written in Lua, but at least upto Lua\TeX~1.11, feeding back too many tokens
% from Lua to \TeX\ triggers a segmentation fault.
% This is written in Lua so the integer setting is expandable and does not interfere with
% a preceding \verb|\immediate|.
%    \begin{macrocode}
    \protected\def\saveboxresource{%
      \LuaCol@if@keyword{type}{%
        \expandafter
        \expanded{\unexpanded{\LuaCol@scan@boxresource@i{type }}%
          \expandafter\expandafter\expandafter}%
        \LuaCol@scan@number
      }{%
        \LuaCol@scan@boxresource@i{}{}%
      }%
    }
%    \end{macrocode}
%    Legacy alias.
%    \begin{macrocode}
    \let\pdfxform\saveboxresource
\fi
%    \end{macrocode}
%
%    \begin{macrocode}
\LuaCol@AtEnd%
%</package>
%    \end{macrocode}
%
% \subsection{Lua module}
%
%    \begin{macrocode}
%<*lua>
%    \end{macrocode}
%    Box zero contains a \cs{hbox} with the color \cs{special}.
%    That is analyzed to get the prefix for the color setting
%    \cs{special}.
%    \begin{macrocode}
oberdiek = oberdiek or {}
local luacolor = oberdiek.luacolor or {}
oberdiek.luacolor = luacolor
%    \end{macrocode}
%    \begin{macro}{getversion()}
%    \begin{macrocode}
function luacolor.getversion()
  tex.write("2023-08-18 v1.18")
end
%    \end{macrocode}
%    \end{macro}
%
% \subsubsection{Driver detection}
%
%    \begin{macrocode}
local ifpdf = tonumber(tex.outputmode or tex.pdfoutput) > 0
local prefix
local prefixes = {
  dvips   = "color ",
  dvipdfm = "pdf:sc ",
  truetex = "textcolor:",
  pctexps = "ps::",
}
local patterns = {
  ["^color "]            = "dvips",
  ["^pdf: *begincolor "] = "dvipdfm",
  ["^pdf: *bcolor "]     = "dvipdfm",
  ["^pdf: *bc "]         = "dvipdfm",
  ["^pdf: *setcolor "]   = "dvipdfm",
  ["^pdf: *scolor "]     = "dvipdfm",
  ["^pdf: *sc "]         = "dvipdfm",
  ["^textcolor:"]        = "truetex",
  ["^ps::"]              = "pctexps",
}
%    \end{macrocode}
%    \begin{macro}{info()}
%    \begin{macrocode}
local function info(msg, term)
  local target = "log"
  if term then
    target = "term and log"
  end
  texio.write_nl(target, "Package luacolor info: " .. msg .. ".")
  texio.write_nl(target, "")
end
%    \end{macrocode}
%    \end{macro}
%    \begin{macro}{dvidetect()}
%    \begin{macrocode}
function luacolor.dvidetect()
  local v = tex.box[0]
  assert(v.id == node.id("hlist"))
  for v in node.traverse_id(node.id("whatsit"), v.head) do
    if v and v.subtype == node.subtype("special") then
      local data = v.data
      for pattern, driver in pairs(patterns) do
        if string.find(data, pattern) then
          prefix = prefixes[driver]
          tex.write(driver)
          return
        end
      end
      info("\\special{" .. data .. "}", true)
      return
    end
  end
  info("Missing \\special", true)
end
%    \end{macrocode}
%    \end{macro}
%
% \subsubsection{Color strings}
%
%    \begin{macrocode}
local map = {
  n = 0,
}
%    \end{macrocode}
%    \begin{macro}{get()}
%    \begin{macrocode}
function luacolor.get(color)
  tex.write("" .. luacolor.getvalue(color))
end
%    \end{macrocode}
%    \end{macro}
%    \begin{macro}{getvalue()}
%    \begin{macrocode}
function luacolor.getvalue(color)
  local n = map[color]
  if not n then
    n = map.n + 1
    map.n = n
    map[n] = color
    map[color] = n
  end
  return n
end
%    \end{macrocode}
%    \end{macro}
%
% \subsubsection{Attribute register}
%
%    \begin{macro}{setattribute()}
%    \begin{macrocode}
local attribute
function luacolor.setattribute(attr)
  attribute = attr
end
%    \end{macrocode}
%    \end{macro}
%
%    \begin{macro}{getattribute()}
%    \begin{macrocode}
function luacolor.getattribute()
  return attribute
end
%    \end{macrocode}
%    \end{macro}
%
% \subsubsection{Whatsit insertion}
%
%    \begin{macrocode}
local LIST = 1
local LIST_LEADERS = 2
local LIST_DISC = 3
local COLOR = 4
local NOCOLOR = 5
local RULE = node.id("rule")
local node_types = {
  [node.id("hlist")] = LIST,
  [node.id("vlist")] = LIST,
  [node.id("rule")]  = COLOR,
  [node.id("glyph")] = COLOR,
  [node.id("disc")]  = LIST_DISC,
  [node.id("whatsit")] = {
    [node.subtype("pdf_colorstack")] =
      function(n)
        return n.stack == 0 and NOCOLOR or nil
      end,
    [node.subtype("special")] = COLOR,
    [node.subtype("pdf_literal")] = COLOR,
    [node.subtype("pdf_save")] = COLOR,
    [node.subtype("pdf_restore")] = COLOR, -- probably not needed
-- TODO (DPC)    [node.subtype("pdf_refximage")] = COLOR,
  },
  [node.id("glue")] =
    function(n)
      if n.subtype >= 100 then -- leaders
        if n.leader.id == RULE then
          return COLOR
        else
          return LIST_LEADERS
        end
      end
    end,
}
%    \end{macrocode}
%    \begin{macro}{get_type()}
%    \begin{macrocode}
local function get_type(n)
  local ret = node_types[n.id]
  if type(ret) == 'table' then
    ret = ret[n.subtype]
  end
  if type(ret) == 'function' then
    ret = ret(n)
  end
  return ret
end
%    \end{macrocode}
%    \end{macro}
%    \begin{macrocode}
local mode = 2 -- luatex.pdfliteral.direct
local WHATSIT = node.id("whatsit")
local SPECIAL = node.subtype("special")
local PDFLITERAL = node.subtype("pdf_literal")
local DRY_FALSE = false
local DRY_TRUE = true
%    \end{macrocode}
%    \begin{macro}{traverse()}
%    \begin{macrocode}
local function traverse(list, color, dry)
  if not list then
    return color
  end
  local head
  if get_type(list) == LIST then
    head = list.head
  elseif get_type(list) == LIST_DISC then
    head = list.replace
  else
    texio.write_nl("!!! Error: Wrong list type: " .. node.type(list.id))
    return color
  end
%<debug>texio.write_nl("traverse: " .. node.type(list.id))
  for n in node.traverse(head) do
%<debug>texio.write_nl("  node: " .. node.type(n.id))
    local t = get_type(n)
%<debug>texio.write_nl("TYPE "..tostring(t).. " "..tostring(node.type(node.getid(n))).." ".. tostring(node.getsubtype(n)))
    if t == LIST or t == LIST_DISC then
      color = traverse(n, color, dry)
    elseif t == LIST_LEADERS then
      local color_after = traverse(n.leader, color, DRY_TRUE)
      if color == color_after then
        traverse(n.leader, color, DRY_FALSE or dry)
      else
        traverse(n.leader, '', DRY_FALSE or dry)
%    \end{macrocode}
% The color status is unknown here, because the leader box
% will or will not be set.
%    \begin{macrocode}
        color = ''
      end
    elseif t == COLOR then
      local v = node.has_attribute(n, attribute)
      if v then
        local newColor = map[v]
        if newColor ~= color then
          color = newColor
          if dry == DRY_FALSE then
            local newNode
            if ifpdf then
              newNode = node.new(WHATSIT, PDFLITERAL)
              newNode.mode = mode
              newNode.data = color
            else
              newNode = node.new(WHATSIT, SPECIAL)
              newNode.data = prefix .. color
            end
            head = node.insert_before(head, n, newNode)
          end
        end
      end
    elseif t == NOCOLOR then
      color = ''
    end
  end
  if get_type(list) == LIST then
    list.head = head
  else
    list.replace = head
  end
  return color
end
%    \end{macrocode}
%    \end{macro}
%
%    \begin{macro}{process()}
%    \begin{macrocode}
function luacolor.process(box)
  local color = ""
  local list = tex.getbox(box)
  traverse(list, color, DRY_FALSE)
end

if luatexbase.callbacktypes.pre_shipout_filter then
  luatexbase.add_to_callback("pre_shipout_filter", function(list)
    traverse(list, "", DRY_FALSE)
    return true
  end, "luacolor.process")
end
%    \end{macrocode}
%    \end{macro}
% For recent versions of luaotfload, we can register a callback to
% control how coloring glyph is handled for the color feature.
%    \begin{macrocode}
if luaotfload.set_colorhandler then
  local set_attribute = node.direct.set_attribute
  luaotfload.set_colorhandler(function(head, n, color)
    set_attribute(n, attribute, luacolor.getvalue(color))
    return head, n
  end)
end
%    \end{macrocode}
%
%    \begin{macrocode}
%</lua>
%    \end{macrocode}
% \section{Installation}
%
% \subsection{Download}
%
% \paragraph{Package.} This package is available on
% CTAN\footnote{\CTANpkg{luacolor}}:
% \begin{description}
% \item[\CTAN{macros/latex/contrib/luacolor/luacolor.dtx}] The source file.
% \item[\CTAN{macros/latex/contrib/luacolor/luacolor.pdf}] Documentation.
% \end{description}
%
%
% \paragraph{Bundle.} All the packages of the bundle `luacolor'
% are also available in a TDS compliant ZIP archive. There
% the packages are already unpacked and the documentation files
% are generated. The files and directories obey the TDS standard.
% \begin{description}
% \item[\CTANinstall{install/macros/latex/contrib/luacolor.tds.zip}]
% \end{description}
% \emph{TDS} refers to the standard ``A Directory Structure
% for \TeX\ Files'' (\CTANpkg{tds}). Directories
% with \xfile{texmf} in their name are usually organized this way.
%
% \subsection{Bundle installation}
%
% \paragraph{Unpacking.} Unpack the \xfile{luacolor.tds.zip} in the
% TDS tree (also known as \xfile{texmf} tree) of your choice.
% Example (linux):
% \begin{quote}
%   |unzip luacolor.tds.zip -d ~/texmf|
% \end{quote}
%
% \paragraph{Script installation.}
% Check the directory \xfile{TDS:scripts/luacolor/} for
% scripts that need further installation steps.
%
% \subsection{Package installation}
%
% \paragraph{Unpacking.} The \xfile{.dtx} file is a self-extracting
% \docstrip\ archive. The files are extracted by running the
% \xfile{.dtx} through \plainTeX:
% \begin{quote}
%   \verb|tex luacolor.dtx|
% \end{quote}
%
% \paragraph{TDS.} Now the different files must be moved into
% the different directories in your installation TDS tree
% (also known as \xfile{texmf} tree):
% \begin{quote}
% \def\t{^^A
% \begin{tabular}{@{}>{\ttfamily}l@{ $\rightarrow$ }>{\ttfamily}l@{}}
%   luacolor.sty & tex/latex/luacolor/luacolor.sty\\
%   luacolor.lua & scripts/luacolor/luacolor.lua\\
%   luacolor.pdf & doc/latex/luacolor/luacolor.pdf\\
%   luacolor.dtx & source/latex/luacolor/luacolor.dtx\\
% \end{tabular}^^A
% }^^A
% \sbox0{\t}^^A
% \ifdim\wd0>\linewidth
%   \begingroup
%     \advance\linewidth by\leftmargin
%     \advance\linewidth by\rightmargin
%   \edef\x{\endgroup
%     \def\noexpand\lw{\the\linewidth}^^A
%   }\x
%   \def\lwbox{^^A
%     \leavevmode
%     \hbox to \linewidth{^^A
%       \kern-\leftmargin\relax
%       \hss
%       \usebox0
%       \hss
%       \kern-\rightmargin\relax
%     }^^A
%   }^^A
%   \ifdim\wd0>\lw
%     \sbox0{\small\t}^^A
%     \ifdim\wd0>\linewidth
%       \ifdim\wd0>\lw
%         \sbox0{\footnotesize\t}^^A
%         \ifdim\wd0>\linewidth
%           \ifdim\wd0>\lw
%             \sbox0{\scriptsize\t}^^A
%             \ifdim\wd0>\linewidth
%               \ifdim\wd0>\lw
%                 \sbox0{\tiny\t}^^A
%                 \ifdim\wd0>\linewidth
%                   \lwbox
%                 \else
%                   \usebox0
%                 \fi
%               \else
%                 \lwbox
%               \fi
%             \else
%               \usebox0
%             \fi
%           \else
%             \lwbox
%           \fi
%         \else
%           \usebox0
%         \fi
%       \else
%         \lwbox
%       \fi
%     \else
%       \usebox0
%     \fi
%   \else
%     \lwbox
%   \fi
% \else
%   \usebox0
% \fi
% \end{quote}
% If you have a \xfile{docstrip.cfg} that configures and enables \docstrip's
% TDS installing feature, then some files can already be in the right
% place, see the documentation of \docstrip.
%
% \subsection{Refresh file name databases}
%
% If your \TeX~distribution
% (\TeX\,Live, \mikTeX, \dots) relies on file name databases, you must refresh
% these. For example, \TeX\,Live\ users run \verb|texhash| or
% \verb|mktexlsr|.
%
% \subsection{Some details for the interested}
%
% \paragraph{Unpacking with \LaTeX.}
% The \xfile{.dtx} chooses its action depending on the format:
% \begin{description}
% \item[\plainTeX:] Run \docstrip\ and extract the files.
% \item[\LaTeX:] Generate the documentation.
% \end{description}
% If you insist on using \LaTeX\ for \docstrip\ (really,
% \docstrip\ does not need \LaTeX), then inform the autodetect routine
% about your intention:
% \begin{quote}
%   \verb|latex \let\install=y% \iffalse meta-comment
%
% File: luacolor.dtx
% Version: 2023-08-18 v1.18
% Info: Color support via LuaTeX's attributes
%
% Copyright (C)
%    2007, 2009-2011 Heiko Oberdiek
%    2016-2023 Oberdiek Package Support Group
%    https://github.com/ho-tex/luacolor/issues
%
% This work may be distributed and/or modified under the
% conditions of the LaTeX Project Public License, either
% version 1.3c of this license or (at your option) any later
% version. This version of this license is in
%    https://www.latex-project.org/lppl/lppl-1-3c.txt
% and the latest version of this license is in
%    https://www.latex-project.org/lppl.txt
% and version 1.3 or later is part of all distributions of
% LaTeX version 2005/12/01 or later.
%
% This work has the LPPL maintenance status "maintained".
%
% The Current Maintainers of this work are
% Heiko Oberdiek and the Oberdiek Package Support Group
% https://github.com/ho-tex/luacolor/issues
%
% This work consists of the main source file luacolor.dtx
% and the derived files
%    luacolor.sty, luacolor.pdf, luacolor.lua,
%
% Distribution:
%    CTAN:macros/latex/contrib/luacolor/luacolor.dtx
%    CTAN:macros/latex/contrib/luacolor/luacolor.pdf
%
% Unpacking:
%    (a) If luacolor.ins is present:
%           tex luacolor.ins
%    (b) Without luacolor.ins:
%           tex luacolor.dtx
%    (c) If you insist on using LaTeX
%           latex \let\install=y% \iffalse meta-comment
%
% File: luacolor.dtx
% Version: 2023-08-18 v1.18
% Info: Color support via LuaTeX's attributes
%
% Copyright (C)
%    2007, 2009-2011 Heiko Oberdiek
%    2016-2023 Oberdiek Package Support Group
%    https://github.com/ho-tex/luacolor/issues
%
% This work may be distributed and/or modified under the
% conditions of the LaTeX Project Public License, either
% version 1.3c of this license or (at your option) any later
% version. This version of this license is in
%    https://www.latex-project.org/lppl/lppl-1-3c.txt
% and the latest version of this license is in
%    https://www.latex-project.org/lppl.txt
% and version 1.3 or later is part of all distributions of
% LaTeX version 2005/12/01 or later.
%
% This work has the LPPL maintenance status "maintained".
%
% The Current Maintainers of this work are
% Heiko Oberdiek and the Oberdiek Package Support Group
% https://github.com/ho-tex/luacolor/issues
%
% This work consists of the main source file luacolor.dtx
% and the derived files
%    luacolor.sty, luacolor.pdf, luacolor.lua,
%
% Distribution:
%    CTAN:macros/latex/contrib/luacolor/luacolor.dtx
%    CTAN:macros/latex/contrib/luacolor/luacolor.pdf
%
% Unpacking:
%    (a) If luacolor.ins is present:
%           tex luacolor.ins
%    (b) Without luacolor.ins:
%           tex luacolor.dtx
%    (c) If you insist on using LaTeX
%           latex \let\install=y% \iffalse meta-comment
%
% File: luacolor.dtx
% Version: 2023-08-18 v1.18
% Info: Color support via LuaTeX's attributes
%
% Copyright (C)
%    2007, 2009-2011 Heiko Oberdiek
%    2016-2023 Oberdiek Package Support Group
%    https://github.com/ho-tex/luacolor/issues
%
% This work may be distributed and/or modified under the
% conditions of the LaTeX Project Public License, either
% version 1.3c of this license or (at your option) any later
% version. This version of this license is in
%    https://www.latex-project.org/lppl/lppl-1-3c.txt
% and the latest version of this license is in
%    https://www.latex-project.org/lppl.txt
% and version 1.3 or later is part of all distributions of
% LaTeX version 2005/12/01 or later.
%
% This work has the LPPL maintenance status "maintained".
%
% The Current Maintainers of this work are
% Heiko Oberdiek and the Oberdiek Package Support Group
% https://github.com/ho-tex/luacolor/issues
%
% This work consists of the main source file luacolor.dtx
% and the derived files
%    luacolor.sty, luacolor.pdf, luacolor.lua,
%
% Distribution:
%    CTAN:macros/latex/contrib/luacolor/luacolor.dtx
%    CTAN:macros/latex/contrib/luacolor/luacolor.pdf
%
% Unpacking:
%    (a) If luacolor.ins is present:
%           tex luacolor.ins
%    (b) Without luacolor.ins:
%           tex luacolor.dtx
%    (c) If you insist on using LaTeX
%           latex \let\install=y\input{luacolor.dtx}
%        (quote the arguments according to the demands of your shell)
%
% Documentation:
%    (a) If luacolor.drv is present:
%           latex luacolor.drv
%    (b) Without luacolor.drv:
%           latex luacolor.dtx; ...
%    The class ltxdoc loads the configuration file ltxdoc.cfg
%    if available. Here you can specify further options, e.g.
%    use A4 as paper format:
%       \PassOptionsToClass{a4paper}{article}
%
%    Programm calls to get the documentation (example):
%       pdflatex luacolor.dtx
%       makeindex -s gind.ist luacolor.idx
%       pdflatex luacolor.dtx
%       makeindex -s gind.ist luacolor.idx
%       pdflatex luacolor.dtx
%
% Installation:
%    TDS:tex/latex/luacolor/luacolor.sty
%    TDS:scripts/luacolor/luacolor.lua
%    TDS:doc/latex/luacolor/luacolor.pdf
%    TDS:source/latex/luacolor/luacolor.dtx
%
%<*ignore>
\begingroup
  \catcode123=1 %
  \catcode125=2 %
  \def\x{LaTeX2e}%
\expandafter\endgroup
\ifcase 0\ifx\install y1\fi\expandafter
         \ifx\csname processbatchFile\endcsname\relax\else1\fi
         \ifx\fmtname\x\else 1\fi\relax
\else\csname fi\endcsname
%</ignore>
%<*install>
\input docstrip.tex
\Msg{************************************************************************}
\Msg{* Installation}
\Msg{* Package: luacolor 2023-08-18 v1.18 Color support via LuaTeX's attributes (HO)}
\Msg{************************************************************************}

\keepsilent
\askforoverwritefalse

\let\MetaPrefix\relax
\preamble

This is a generated file.

Project: luacolor
Version: 2023-08-18 v1.18

Copyright (C)
   2007, 2009-2011 Heiko Oberdiek
   2016-2023 Oberdiek Package Support Group

This work may be distributed and/or modified under the
conditions of the LaTeX Project Public License, either
version 1.3c of this license or (at your option) any later
version. This version of this license is in
   https://www.latex-project.org/lppl/lppl-1-3c.txt
and the latest version of this license is in
   https://www.latex-project.org/lppl.txt
and version 1.3 or later is part of all distributions of
LaTeX version 2005/12/01 or later.

This work has the LPPL maintenance status "maintained".

The Current Maintainers of this work are
Heiko Oberdiek and the Oberdiek Package Support Group
https://github.com/ho-tex/luacolor/issues


This work consists of the main source file luacolor.dtx
and the derived files
   luacolor.sty, luacolor.pdf, luacolor.ins, luacolor.drv,
   luacolor.lua,

\endpreamble
\let\MetaPrefix\DoubleperCent

\generate{%
  \file{luacolor.ins}{\from{luacolor.dtx}{install}}%
  \file{luacolor.drv}{\from{luacolor.dtx}{driver}}%
  \usedir{tex/latex/luacolor}%
  \file{luacolor.sty}{\from{luacolor.dtx}{package}}%
}
\def\MetaPrefix{-- }
\def\defaultpostamble{%
  \MetaPrefix^^J%
  \MetaPrefix\space End of File `\outFileName'.%
}
\def\currentpostamble{\defaultpostamble}%
\generate{%
  \usedir{scripts/luacolor}%
%  \file{oberdiek.luacolor.lua}{\from{luacolor.dtx}{lua}}%
  \file{luacolor.lua}{\from{luacolor.dtx}{lua}}%
}

\catcode32=13\relax% active space
\let =\space%
\Msg{************************************************************************}
\Msg{*}
\Msg{* To finish the installation you have to move the following}
\Msg{* file into a directory searched by TeX:}
\Msg{*}
\Msg{*     luacolor.sty}
\Msg{*}
\Msg{* And install the following script file:}
\Msg{*}
\Msg{*     luacolor.lua,}
\Msg{*}
\Msg{* To produce the documentation run the file `luacolor.dtx'}
\Msg{* through LaTeX.}
\Msg{*}
\Msg{* Happy TeXing!}
\Msg{*}
\Msg{************************************************************************}

\endbatchfile
%</install>
%<*ignore>
\fi
%</ignore>
%<*driver>
\NeedsTeXFormat{LaTeX2e}
\ProvidesFile{luacolor.drv}%
  [2023-08-18 v1.18 Color support via LuaTeX's attributes (HO)]%
\documentclass{ltxdoc}
\usepackage{holtxdoc}[2011/11/22]
\usepackage{paralist}
\makeatletter
\g@addto@macro\MakePrivateLetters{%
  \@makeother\_%
}
\makeatother
\begin{document}
  \DocInput{luacolor.dtx}%
\end{document}
%</driver>
% \fi
%
%
%
% \GetFileInfo{luacolor.drv}
%
% \title{The \xpackage{luacolor} package}
% \date{2023-08-18 v1.18}
% \author{Heiko Oberdiek\thanks
% {Please report any issues at \url{https://github.com/ho-tex/luacolor/issues}}}
%
% \maketitle
%
% \begin{abstract}
% Package \xpackage{luacolor} implements color support based
% on \LuaTeX's node attributes.
% \end{abstract}
%
% \tableofcontents
%
% \section{Documentation}
%
% \subsection{Introduction}
%
% This package uses a \LuaTeX's attribute register to
% to annotate nodes with color information.
% If a color is set, then the attribute register
% is set to this color and all nodes created in its scope
% (current group) are annotated with this attribute.
% Now the color property behaves much the same way
% as the font property.
%
% \subsection{Usage}
%
% Package \xpackage{color} is loaded automatically by this
% package \xpackage{luacolor}. If you need a special driver
% option or you prefer package \xpackage{xcolor}, then load
% it before package \xpackage{luacolor}, for example:
% \begin{quote}
%   |\usepackage[dvipdfmx]{xcolor}|
% \end{quote}
% The package \xpackage{luacolor} is loaded without options:
% \begin{quote}
%   |\usepackage{luacolor}|
% \end{quote}
% It is able to detect PDF mode and DVI drivers are
% differentiated by its color specials. Therefore the
% package do need driver options.
%
% Then it redefines the color setting commands to set
% attributes instead of whatsits for color.
%
% At last the attribute annotations of the nodes in the
% output box must be analyzed to insert the necessary color whatsits.
% (With older \LuaTeX\ that lack the appropriate callback function 
% the package \xpackage{atbegshi} is used to get control
% before a box is shipped out.)
%
% \begin{declcs}{luacolorProcessBox} \M{box}
% \end{declcs}
% Macro \cs{luacolorProcessBox} processes the box \meta{box}
% in the previously described manner. It is automatically
% called for pages, but not for XForm objects.
% Before passing a box to \cs{pdfxform}, call \cs{luacolorProcessBox}
% first.
%
% \subsection{Limitations}
%
% \begin{description}
% \item[Ligatures with different colored components:]
%   Package \xpackage{luacolor} sees the ligature after
%   the paragraph building and page breaking, when a
%   page is to be shipped out. Therefore it cannot
%   break ligatures, because the components might occupy
%   different space. Therefore it is the responsibility
%   of the ligature forming process to deal with different
%   colored glyphs that form a ligature. The user can avoid
%   the problem entirely by explicitly breaking the ligature
%   at the places where the color changes.
%  \item \dots
% \end{description}
%
% \StopEventually{
% }
%
% \section{Implementation}
%
%    \begin{macrocode}
%<*package>
%    \end{macrocode}
%
% \subsection{Catcodes and identification}
%
%    \begin{macrocode}
\begingroup\catcode61\catcode48\catcode32=10\relax%
  \catcode13=5 % ^^M
  \endlinechar=13 %
  \catcode123=1 % {
  \catcode125=2 % }
  \catcode64=11 % @
  \def\x{\endgroup
    \expandafter\edef\csname LuaCol@AtEnd\endcsname{%
      \endlinechar=\the\endlinechar\relax
      \catcode13=\the\catcode13\relax
      \catcode32=\the\catcode32\relax
      \catcode35=\the\catcode35\relax
      \catcode61=\the\catcode61\relax
      \catcode64=\the\catcode64\relax
      \catcode123=\the\catcode123\relax
      \catcode125=\the\catcode125\relax
    }%
  }%
\x\catcode61\catcode48\catcode32=10\relax%
\catcode13=5 % ^^M
\endlinechar=13 %
\catcode35=6 % #
\catcode64=11 % @
\catcode123=1 % {
\catcode125=2 % }
\def\TMP@EnsureCode#1#2{%
  \edef\LuaCol@AtEnd{%
    \LuaCol@AtEnd
    \catcode#1=\the\catcode#1\relax
  }%
  \catcode#1=#2\relax
}
\TMP@EnsureCode{34}{12}% "
\TMP@EnsureCode{39}{12}% '
\TMP@EnsureCode{40}{12}% (
\TMP@EnsureCode{41}{12}% )
\TMP@EnsureCode{42}{12}% *
\TMP@EnsureCode{43}{12}% +
\TMP@EnsureCode{44}{12}% ,
\TMP@EnsureCode{45}{12}% -
\TMP@EnsureCode{46}{12}% .
\TMP@EnsureCode{47}{12}% /
\TMP@EnsureCode{58}{12}% :
\TMP@EnsureCode{60}{12}% <
\TMP@EnsureCode{62}{12}% >
\TMP@EnsureCode{91}{12}% [
\TMP@EnsureCode{93}{12}% ]
\TMP@EnsureCode{95}{12}% _ (other!)
\TMP@EnsureCode{96}{12}% `
\edef\LuaCol@AtEnd{\LuaCol@AtEnd\noexpand\endinput}
%    \end{macrocode}
%
%    Package identification.
%    \begin{macrocode}
\NeedsTeXFormat{LaTeX2e}
\ProvidesPackage{luacolor}%
  [2023-08-18 v1.18 Color support via LuaTeX's attributes (HO)]
%    \end{macrocode}
%
% \subsection{Check for \LuaTeX}
%
%    Without \LuaTeX\ there is no point in using this package.
%    \begin{macrocode}
\RequirePackage{color}
%    \end{macrocode}
%
%    \begin{macrocode}
\ifx\directlua\@undefined
  \PackageError{luacolor}{%
    This package may only be run using LuaTeX%
  }\@ehc
  \expandafter\LuaCol@AtEnd
\fi%
%    \end{macrocode}
%
% \subsection{Check for disabled colors}
%
%    \begin{macrocode}
\ifcolors@
\else
  \PackageWarningNoLine{luacolor}{%
    Colors are disabled by option `monochrome'%
  }%
  \def\set@color{}%
  \def\reset@color{}%
  \def\set@page@color{}%
  \def\define@color#1#2{}%
  \expandafter\LuaCol@AtEnd
\fi%
%    \end{macrocode}
%
% \subsection{Load module and check version}
%
%    \begin{macrocode}
\directlua{%
  require("luacolor")%
}
%    \end{macrocode}
%    \begin{macrocode}
\begingroup
  \edef\x{\directlua{tex.write("2023-08-18 v1.18")}}%
  \edef\y{%
    \directlua{%
      if oberdiek.luacolor.getversion then %
        oberdiek.luacolor.getversion()%
      end%
    }%
  }%
  \ifx\x\y
  \else
    \PackageError{luacolor}{%
      Wrong version of lua module.\MessageBreak
      Package version: \x\MessageBreak
      Lua module: \y
    }\@ehc
  \fi
\endgroup
%    \end{macrocode}
%
% \subsection{Find driver}
%
%    \begin{macrocode}
\ifnum\outputmode=\@ne
\else
  \begingroup
    \def\current@color{}%
    \def\reset@color{}%
    \setbox\z@=\hbox{%
      \begingroup
        \set@color
      \endgroup
    }%
    \edef\reserved@a{%
      \directlua{%
        oberdiek.luacolor.dvidetect()%
      }%
    }%
    \ifx\reserved@a\@empty
      \PackageError{luacolor}{%
        DVI driver detection failed because of\MessageBreak
        unrecognized color \string\special
      }\@ehc
      \endgroup
      \expandafter\expandafter\expandafter\LuaCol@AtEnd
    \else
      \PackageInfo{luacolor}{%
        Type of color \string\special: \reserved@a
      \@gobble}%
    \fi%
  \endgroup
\fi
%    \end{macrocode}
%
% \subsection{Attribute setting}
%
%    \begin{macro}{\LuaCol@Attribute}
%    \begin{macrocode}
\newattribute\LuaCol@Attribute
\let\LuaCol@setattribute\setattribute
\directlua{%
  oberdiek.luacolor.setattribute(\number\allocationnumber)%
}
%    \end{macrocode}
%    \end{macro}
%
%    \begin{macro}{\set@color}
% change 2023-08-18: added \cs{reset@color} so that \cs{mathcolor} can gobble it,
% issue 7.   
%    \begin{macrocode}
\protected\def\set@color{%
  \LuaCol@setattribute\LuaCol@Attribute{%
    \directlua{%
      oberdiek.luacolor.get("\luaescapestring{\current@color}")%
    }%
  }%
 \aftergroup\reset@color 
}
%    \end{macrocode}
%    \end{macro}
%    \begin{macro}{\reset@color}
%    \begin{macrocode}
\def\reset@color{}
%    \end{macrocode}
%    \end{macro}
%
% \subsection{Whatsit insertion}
%
%    \begin{macro}{\luacolorProcessBox}
%    \begin{macrocode}
\def\luacolorProcessBox#1{%
  \directlua{%
    oberdiek.luacolor.process(\number#1)%
  }%
}
%    \end{macrocode}
%    \end{macro}
%
%    \begin{macrocode}
\directlua{%
  if luatexbase.callbacktypes.pre_shipout_filter then
    token.get_next()
  end
}\@secondoftwo\@gobble{
  \RequirePackage{atbegshi}[2011/01/30]
  \AtBeginShipout{%
    \luacolorProcessBox\AtBeginShipoutBox
  }
}
%    \end{macrocode}
%
%    Set default color.
%    \begin{macrocode}
\set@color
%    \end{macrocode}
%
% \subsection{\cs{pdfxform}/\cs{saveboxresource} support}
%
%    \begin{macrocode}
\ifnum\outputmode=\@ne
    \let\LuaCol@org@pdfxform\saveboxresource
%    \end{macrocode}
% First we need some helpers to allow expandable code to parse keyword style arguments:
%    \begin{macrocode}
    \def\LuaCol@iii@i@ii#1#2#3{#3{#1}{#2}}
    \def\LuaCol@ii@i#1#2{{#2#1}}
    \def\LuaCol@if@keyword#1#2#3{%
      \expanded{\unexpanded{\LuaCol@iii@i@ii{#2}{#3}}\expandafter}%
      \directlua{%
        token.put_next(token.create(token.scan_keyword(token.scan_string())
        and '@firstoftwo'
        or '@secondoftwo'))
      }{#1}%
    }
%    \end{macrocode}
% The following macro scans a integer and expands to a token equivalent to a chardef
% whose value corresponds to the scanned integer. This allows the integer to be passed
% around as a undelimited argument.
%    \begin{macrocode}
    \def\LuaCol@scan@number{%
      \directlua{
        token.put_next(token.new(token.scan_int(), token.command_id'char_given'))
      }%
    }
%    \end{macrocode}
% \TeX\ primitives like \cs{saveboxresource} read braced arguments in a special way.
% Especially they expand everything until they find a left brace. To simulate this, we
% use Lua to expand everything else:
%    \begin{macrocode}
    \def\LuaCol@scan@tobrace{%
      \directlua{
        local relax, space = token.command_id'relax', token.command_id'spacer'
        local t
        repeat
          t = token.scan_token()
        until not (t.command == relax or t.command == space)
        token.put_next(t)
      }%
    }
    \def\LuaCol@scan@boxresource@i#1#2{%
      \LuaCol@if@keyword{attr}{%
        \expanded{\unexpanded{\LuaCol@scan@boxresource@iI{#1#2attr}}%
          \expandafter\expandafter\expandafter}%
        \LuaCol@scan@tobrace
      }{%
        \LuaCol@scan@boxresource@ii{#1#2}%
      }%
    }
    \def\LuaCol@scan@boxresource@iI#1#2{\LuaCol@scan@boxresource@ii{#1{#2}}}
    \def\LuaCol@scan@boxresource@ii#1{%
      \LuaCol@if@keyword{resources}{%
        \expanded{\unexpanded{\LuaCol@scan@boxresource@iiI{#1resources}}%
          \expandafter\expandafter\expandafter}%
        \LuaCol@scan@tobrace
      }{%
        \LuaCol@scan@boxresource@iii{#1}%
      }%
    }
    \def\LuaCol@scan@boxresource@iiI#1#2{\LuaCol@scan@boxresource@iii{#1{#2}}}
    \def\LuaCol@scan@boxresource@iii#1{%
      \LuaCol@if@keyword{margin}{%
        \expanded{\unexpanded{\LuaCol@scan@boxresource@iv{#1margin }}%
          \expandafter\expandafter\expandafter}%
        \LuaCol@scan@number
      }{%
        \LuaCol@scan@boxresource@iv{#1}{}%
      }%
    }
    \def\LuaCol@scan@boxresource@iv#1#2{%
      \expanded{\unexpanded{\LuaCol@scan@boxresource@v{#1#2}}%
        \expandafter\expandafter\expandafter}%
      \LuaCol@scan@number
    }
    \def\LuaCol@scan@boxresource@v#1#2{%
      \luacolorProcessBox{#2}%
      \LuaCol@org@pdfxform#1#2%
    }

%    \end{macrocode}
% This could be written in Lua, but at least upto Lua\TeX~1.11, feeding back too many tokens
% from Lua to \TeX\ triggers a segmentation fault.
% This is written in Lua so the integer setting is expandable and does not interfere with
% a preceding \verb|\immediate|.
%    \begin{macrocode}
    \protected\def\saveboxresource{%
      \LuaCol@if@keyword{type}{%
        \expandafter
        \expanded{\unexpanded{\LuaCol@scan@boxresource@i{type }}%
          \expandafter\expandafter\expandafter}%
        \LuaCol@scan@number
      }{%
        \LuaCol@scan@boxresource@i{}{}%
      }%
    }
%    \end{macrocode}
%    Legacy alias.
%    \begin{macrocode}
    \let\pdfxform\saveboxresource
\fi
%    \end{macrocode}
%
%    \begin{macrocode}
\LuaCol@AtEnd%
%</package>
%    \end{macrocode}
%
% \subsection{Lua module}
%
%    \begin{macrocode}
%<*lua>
%    \end{macrocode}
%    Box zero contains a \cs{hbox} with the color \cs{special}.
%    That is analyzed to get the prefix for the color setting
%    \cs{special}.
%    \begin{macrocode}
oberdiek = oberdiek or {}
local luacolor = oberdiek.luacolor or {}
oberdiek.luacolor = luacolor
%    \end{macrocode}
%    \begin{macro}{getversion()}
%    \begin{macrocode}
function luacolor.getversion()
  tex.write("2023-08-18 v1.18")
end
%    \end{macrocode}
%    \end{macro}
%
% \subsubsection{Driver detection}
%
%    \begin{macrocode}
local ifpdf = tonumber(tex.outputmode or tex.pdfoutput) > 0
local prefix
local prefixes = {
  dvips   = "color ",
  dvipdfm = "pdf:sc ",
  truetex = "textcolor:",
  pctexps = "ps::",
}
local patterns = {
  ["^color "]            = "dvips",
  ["^pdf: *begincolor "] = "dvipdfm",
  ["^pdf: *bcolor "]     = "dvipdfm",
  ["^pdf: *bc "]         = "dvipdfm",
  ["^pdf: *setcolor "]   = "dvipdfm",
  ["^pdf: *scolor "]     = "dvipdfm",
  ["^pdf: *sc "]         = "dvipdfm",
  ["^textcolor:"]        = "truetex",
  ["^ps::"]              = "pctexps",
}
%    \end{macrocode}
%    \begin{macro}{info()}
%    \begin{macrocode}
local function info(msg, term)
  local target = "log"
  if term then
    target = "term and log"
  end
  texio.write_nl(target, "Package luacolor info: " .. msg .. ".")
  texio.write_nl(target, "")
end
%    \end{macrocode}
%    \end{macro}
%    \begin{macro}{dvidetect()}
%    \begin{macrocode}
function luacolor.dvidetect()
  local v = tex.box[0]
  assert(v.id == node.id("hlist"))
  for v in node.traverse_id(node.id("whatsit"), v.head) do
    if v and v.subtype == node.subtype("special") then
      local data = v.data
      for pattern, driver in pairs(patterns) do
        if string.find(data, pattern) then
          prefix = prefixes[driver]
          tex.write(driver)
          return
        end
      end
      info("\\special{" .. data .. "}", true)
      return
    end
  end
  info("Missing \\special", true)
end
%    \end{macrocode}
%    \end{macro}
%
% \subsubsection{Color strings}
%
%    \begin{macrocode}
local map = {
  n = 0,
}
%    \end{macrocode}
%    \begin{macro}{get()}
%    \begin{macrocode}
function luacolor.get(color)
  tex.write("" .. luacolor.getvalue(color))
end
%    \end{macrocode}
%    \end{macro}
%    \begin{macro}{getvalue()}
%    \begin{macrocode}
function luacolor.getvalue(color)
  local n = map[color]
  if not n then
    n = map.n + 1
    map.n = n
    map[n] = color
    map[color] = n
  end
  return n
end
%    \end{macrocode}
%    \end{macro}
%
% \subsubsection{Attribute register}
%
%    \begin{macro}{setattribute()}
%    \begin{macrocode}
local attribute
function luacolor.setattribute(attr)
  attribute = attr
end
%    \end{macrocode}
%    \end{macro}
%
%    \begin{macro}{getattribute()}
%    \begin{macrocode}
function luacolor.getattribute()
  return attribute
end
%    \end{macrocode}
%    \end{macro}
%
% \subsubsection{Whatsit insertion}
%
%    \begin{macrocode}
local LIST = 1
local LIST_LEADERS = 2
local LIST_DISC = 3
local COLOR = 4
local NOCOLOR = 5
local RULE = node.id("rule")
local node_types = {
  [node.id("hlist")] = LIST,
  [node.id("vlist")] = LIST,
  [node.id("rule")]  = COLOR,
  [node.id("glyph")] = COLOR,
  [node.id("disc")]  = LIST_DISC,
  [node.id("whatsit")] = {
    [node.subtype("pdf_colorstack")] =
      function(n)
        return n.stack == 0 and NOCOLOR or nil
      end,
    [node.subtype("special")] = COLOR,
    [node.subtype("pdf_literal")] = COLOR,
    [node.subtype("pdf_save")] = COLOR,
    [node.subtype("pdf_restore")] = COLOR, -- probably not needed
-- TODO (DPC)    [node.subtype("pdf_refximage")] = COLOR,
  },
  [node.id("glue")] =
    function(n)
      if n.subtype >= 100 then -- leaders
        if n.leader.id == RULE then
          return COLOR
        else
          return LIST_LEADERS
        end
      end
    end,
}
%    \end{macrocode}
%    \begin{macro}{get_type()}
%    \begin{macrocode}
local function get_type(n)
  local ret = node_types[n.id]
  if type(ret) == 'table' then
    ret = ret[n.subtype]
  end
  if type(ret) == 'function' then
    ret = ret(n)
  end
  return ret
end
%    \end{macrocode}
%    \end{macro}
%    \begin{macrocode}
local mode = 2 -- luatex.pdfliteral.direct
local WHATSIT = node.id("whatsit")
local SPECIAL = node.subtype("special")
local PDFLITERAL = node.subtype("pdf_literal")
local DRY_FALSE = false
local DRY_TRUE = true
%    \end{macrocode}
%    \begin{macro}{traverse()}
%    \begin{macrocode}
local function traverse(list, color, dry)
  if not list then
    return color
  end
  local head
  if get_type(list) == LIST then
    head = list.head
  elseif get_type(list) == LIST_DISC then
    head = list.replace
  else
    texio.write_nl("!!! Error: Wrong list type: " .. node.type(list.id))
    return color
  end
%<debug>texio.write_nl("traverse: " .. node.type(list.id))
  for n in node.traverse(head) do
%<debug>texio.write_nl("  node: " .. node.type(n.id))
    local t = get_type(n)
%<debug>texio.write_nl("TYPE "..tostring(t).. " "..tostring(node.type(node.getid(n))).." ".. tostring(node.getsubtype(n)))
    if t == LIST or t == LIST_DISC then
      color = traverse(n, color, dry)
    elseif t == LIST_LEADERS then
      local color_after = traverse(n.leader, color, DRY_TRUE)
      if color == color_after then
        traverse(n.leader, color, DRY_FALSE or dry)
      else
        traverse(n.leader, '', DRY_FALSE or dry)
%    \end{macrocode}
% The color status is unknown here, because the leader box
% will or will not be set.
%    \begin{macrocode}
        color = ''
      end
    elseif t == COLOR then
      local v = node.has_attribute(n, attribute)
      if v then
        local newColor = map[v]
        if newColor ~= color then
          color = newColor
          if dry == DRY_FALSE then
            local newNode
            if ifpdf then
              newNode = node.new(WHATSIT, PDFLITERAL)
              newNode.mode = mode
              newNode.data = color
            else
              newNode = node.new(WHATSIT, SPECIAL)
              newNode.data = prefix .. color
            end
            head = node.insert_before(head, n, newNode)
          end
        end
      end
    elseif t == NOCOLOR then
      color = ''
    end
  end
  if get_type(list) == LIST then
    list.head = head
  else
    list.replace = head
  end
  return color
end
%    \end{macrocode}
%    \end{macro}
%
%    \begin{macro}{process()}
%    \begin{macrocode}
function luacolor.process(box)
  local color = ""
  local list = tex.getbox(box)
  traverse(list, color, DRY_FALSE)
end

if luatexbase.callbacktypes.pre_shipout_filter then
  luatexbase.add_to_callback("pre_shipout_filter", function(list)
    traverse(list, "", DRY_FALSE)
    return true
  end, "luacolor.process")
end
%    \end{macrocode}
%    \end{macro}
% For recent versions of luaotfload, we can register a callback to
% control how coloring glyph is handled for the color feature.
%    \begin{macrocode}
if luaotfload.set_colorhandler then
  local set_attribute = node.direct.set_attribute
  luaotfload.set_colorhandler(function(head, n, color)
    set_attribute(n, attribute, luacolor.getvalue(color))
    return head, n
  end)
end
%    \end{macrocode}
%
%    \begin{macrocode}
%</lua>
%    \end{macrocode}
% \section{Installation}
%
% \subsection{Download}
%
% \paragraph{Package.} This package is available on
% CTAN\footnote{\CTANpkg{luacolor}}:
% \begin{description}
% \item[\CTAN{macros/latex/contrib/luacolor/luacolor.dtx}] The source file.
% \item[\CTAN{macros/latex/contrib/luacolor/luacolor.pdf}] Documentation.
% \end{description}
%
%
% \paragraph{Bundle.} All the packages of the bundle `luacolor'
% are also available in a TDS compliant ZIP archive. There
% the packages are already unpacked and the documentation files
% are generated. The files and directories obey the TDS standard.
% \begin{description}
% \item[\CTANinstall{install/macros/latex/contrib/luacolor.tds.zip}]
% \end{description}
% \emph{TDS} refers to the standard ``A Directory Structure
% for \TeX\ Files'' (\CTANpkg{tds}). Directories
% with \xfile{texmf} in their name are usually organized this way.
%
% \subsection{Bundle installation}
%
% \paragraph{Unpacking.} Unpack the \xfile{luacolor.tds.zip} in the
% TDS tree (also known as \xfile{texmf} tree) of your choice.
% Example (linux):
% \begin{quote}
%   |unzip luacolor.tds.zip -d ~/texmf|
% \end{quote}
%
% \paragraph{Script installation.}
% Check the directory \xfile{TDS:scripts/luacolor/} for
% scripts that need further installation steps.
%
% \subsection{Package installation}
%
% \paragraph{Unpacking.} The \xfile{.dtx} file is a self-extracting
% \docstrip\ archive. The files are extracted by running the
% \xfile{.dtx} through \plainTeX:
% \begin{quote}
%   \verb|tex luacolor.dtx|
% \end{quote}
%
% \paragraph{TDS.} Now the different files must be moved into
% the different directories in your installation TDS tree
% (also known as \xfile{texmf} tree):
% \begin{quote}
% \def\t{^^A
% \begin{tabular}{@{}>{\ttfamily}l@{ $\rightarrow$ }>{\ttfamily}l@{}}
%   luacolor.sty & tex/latex/luacolor/luacolor.sty\\
%   luacolor.lua & scripts/luacolor/luacolor.lua\\
%   luacolor.pdf & doc/latex/luacolor/luacolor.pdf\\
%   luacolor.dtx & source/latex/luacolor/luacolor.dtx\\
% \end{tabular}^^A
% }^^A
% \sbox0{\t}^^A
% \ifdim\wd0>\linewidth
%   \begingroup
%     \advance\linewidth by\leftmargin
%     \advance\linewidth by\rightmargin
%   \edef\x{\endgroup
%     \def\noexpand\lw{\the\linewidth}^^A
%   }\x
%   \def\lwbox{^^A
%     \leavevmode
%     \hbox to \linewidth{^^A
%       \kern-\leftmargin\relax
%       \hss
%       \usebox0
%       \hss
%       \kern-\rightmargin\relax
%     }^^A
%   }^^A
%   \ifdim\wd0>\lw
%     \sbox0{\small\t}^^A
%     \ifdim\wd0>\linewidth
%       \ifdim\wd0>\lw
%         \sbox0{\footnotesize\t}^^A
%         \ifdim\wd0>\linewidth
%           \ifdim\wd0>\lw
%             \sbox0{\scriptsize\t}^^A
%             \ifdim\wd0>\linewidth
%               \ifdim\wd0>\lw
%                 \sbox0{\tiny\t}^^A
%                 \ifdim\wd0>\linewidth
%                   \lwbox
%                 \else
%                   \usebox0
%                 \fi
%               \else
%                 \lwbox
%               \fi
%             \else
%               \usebox0
%             \fi
%           \else
%             \lwbox
%           \fi
%         \else
%           \usebox0
%         \fi
%       \else
%         \lwbox
%       \fi
%     \else
%       \usebox0
%     \fi
%   \else
%     \lwbox
%   \fi
% \else
%   \usebox0
% \fi
% \end{quote}
% If you have a \xfile{docstrip.cfg} that configures and enables \docstrip's
% TDS installing feature, then some files can already be in the right
% place, see the documentation of \docstrip.
%
% \subsection{Refresh file name databases}
%
% If your \TeX~distribution
% (\TeX\,Live, \mikTeX, \dots) relies on file name databases, you must refresh
% these. For example, \TeX\,Live\ users run \verb|texhash| or
% \verb|mktexlsr|.
%
% \subsection{Some details for the interested}
%
% \paragraph{Unpacking with \LaTeX.}
% The \xfile{.dtx} chooses its action depending on the format:
% \begin{description}
% \item[\plainTeX:] Run \docstrip\ and extract the files.
% \item[\LaTeX:] Generate the documentation.
% \end{description}
% If you insist on using \LaTeX\ for \docstrip\ (really,
% \docstrip\ does not need \LaTeX), then inform the autodetect routine
% about your intention:
% \begin{quote}
%   \verb|latex \let\install=y\input{luacolor.dtx}|
% \end{quote}
% Do not forget to quote the argument according to the demands
% of your shell.
%
% \paragraph{Generating the documentation.}
% You can use both the \xfile{.dtx} or the \xfile{.drv} to generate
% the documentation. The process can be configured by the
% configuration file \xfile{ltxdoc.cfg}. For instance, put this
% line into this file, if you want to have A4 as paper format:
% \begin{quote}
%   \verb|\PassOptionsToClass{a4paper}{article}|
% \end{quote}
% An example follows how to generate the
% documentation with pdf\LaTeX:
% \begin{quote}
%\begin{verbatim}
%pdflatex luacolor.dtx
%makeindex -s gind.ist luacolor.idx
%pdflatex luacolor.dtx
%makeindex -s gind.ist luacolor.idx
%pdflatex luacolor.dtx
%\end{verbatim}
% \end{quote}
%
% \begin{History}
%   \begin{Version}{2007/12/12 v1.0}
%   \item
%     First public version.
%   \end{Version}
%   \begin{Version}{2009/04/10 v1.1}
%   \item
%     Fixes for changed syntax of \cs{directlua} in \LuaTeX\ 0.36.
%   \end{Version}
%   \begin{Version}{2010/03/09 v1.2}
%   \item
%     Adaptation for package \xpackage{luatex} 2010/03/09 v0.4.
%   \end{Version}
%   \begin{Version}{2010/12/13 v1.3}
%   \item
%     Support for \cs{pdfxform} added.
%   \item
%     Loaded package \xpackage{luatexbase-attr} recognized.
%   \item
%     Update for \hologo{LuaTeX}: `list' fields renamed to `head' in v0.65.0.
%   \end{Version}
%   \begin{Version}{2011/03/29 v1.4}
%   \item
%     Avoid whatsit insertion if option \xoption{monochrome} is used
%     (thanks Manuel P\'egouri\'e-Gonnard).
%   \end{Version}
%   \begin{Version}{2011/04/22 v1.5}
%   \item
%     Bug fix by Manuel P\'egouri\'e-Gonnard: A typo prevented the
%     detection of whatsits and applying color changes for
%     \cs{pdfliteral} and \cs{special} nodes that might contain
%     typesetting material.
%   \item
%     Bug fix by Manuel P\'egouri\'e-Gonnard: Now colors are also
%     applied to leader boxes.
%   \item
%     Unnecessary color settings are removed for leaders boxes,
%     if after the leader box the color has not changed.
%     The costs are a little runtime, leader boxes are processed twice.
%   \item
%     Additional whatsits that are colored: \texttt{pdf\_refximage}.
%   \item
%     Workaround for bug with \texttt{node.insert\_before}
%     removed for the version after \hologo{LuaTeX} 0.65, because
%     bug was fixed in 0.27. (Thanks Manuel P\'egouri\'e-Gonnard.)
%   \end{Version}
%   \begin{Version}{2011/04/23 v1.6}
%   \item
%     Bug fix for nested leader boxes.
%   \item
%     Bug fix for leader boxes that change color, but
%     are not set because of missing place.
%   \item
%     Version check for Lua module added.
%   \end{Version}
%   \begin{Version}{2011/10/22 v1.7}
%   \item
%     Lua functions \texttt{getattribute} and \texttt{getvalue}
%     added to tell other external Lua functions
%     the attribute register number for coloring.
%   \end{Version}
%   \begin{Version}{2011/11/01 v1.8}
%   \item
%     Use of \texttt{node.subtype} instead of magic numbers.
%   \end{Version}
%   \begin{Version}{2016/05/13 v1.9}
%   \item
%     More use of \texttt{node.subtype} instead of magic numbers.
%   \item luatex 85 updates
%   \end{Version}
%   \begin{Version}{2016/05/16 v1.10}
%   \item
%     Documentation updates.
%   \end{Version}
%   \begin{Version}{2018/11/22 v1.11}
%   \item
%     handle issue 43.
%   \item
%     removed pre-0.65 stuff
%   \end{Version}
%   \begin{Version}{2019/07/25 v1.12}
%   \item
%     removed uses of module function, see PR70
%   \end{Version}
%   \begin{Version}{2019/11/29 v1.13}
%   \item
%     Documentation updates.
%   \item  Use \xpackage{iftex} directly.
%   \end{Version}
%   \begin{Version}{2020-02-22 v1.14}
%   \item Drop use of \textsf{iftex} \textsf{ltxcmds} and \textsf{infwarerr}.
%   \item Assume \textsf{ltluatex} preloaded into format (true since 2015).
%   \item Patch \verb|\saveboxresource| rather than \verb|\pdfxform| (keep old name as alias).
%   \item Grab the number via Lua so that a \verb|\immediate| prefix still works with
%         \verb|\saveboxresource|\slash \verb|\pdfxform|.
%   \item Added handler for the color feature of luaotfload      
%   \end{Version}
%   \begin{Version}{2020-02-24 v1.15}
%   \item Grab all possible arguments for \verb|\saveboxresource|\slash \verb|\pdfxform|
%   \end{Version}
%   \begin{Version}{2020-04-04 v1.16}
%   \item Reset color after \verb|pdf_colorstack| whatsits.
%   \end{Version}
%   \begin{Version}{2021-02-17 v1.17}
%   \item Use \LaTeXe's new \verb|pre_shipout_filter| callback if it's available to
%         allow coloring background and foreground layer material
%   \end{Version}
%   \begin{Version}{2023-08-18 v1.18}
%   \item added \verb|\reset@color| to \verb|\set@color| for \verb|\mathcolor|, issue 7.
%   \end{Version}
% \end{History}
%
% \PrintIndex
%
% \Finale
\endinput

%        (quote the arguments according to the demands of your shell)
%
% Documentation:
%    (a) If luacolor.drv is present:
%           latex luacolor.drv
%    (b) Without luacolor.drv:
%           latex luacolor.dtx; ...
%    The class ltxdoc loads the configuration file ltxdoc.cfg
%    if available. Here you can specify further options, e.g.
%    use A4 as paper format:
%       \PassOptionsToClass{a4paper}{article}
%
%    Programm calls to get the documentation (example):
%       pdflatex luacolor.dtx
%       makeindex -s gind.ist luacolor.idx
%       pdflatex luacolor.dtx
%       makeindex -s gind.ist luacolor.idx
%       pdflatex luacolor.dtx
%
% Installation:
%    TDS:tex/latex/luacolor/luacolor.sty
%    TDS:scripts/luacolor/luacolor.lua
%    TDS:doc/latex/luacolor/luacolor.pdf
%    TDS:source/latex/luacolor/luacolor.dtx
%
%<*ignore>
\begingroup
  \catcode123=1 %
  \catcode125=2 %
  \def\x{LaTeX2e}%
\expandafter\endgroup
\ifcase 0\ifx\install y1\fi\expandafter
         \ifx\csname processbatchFile\endcsname\relax\else1\fi
         \ifx\fmtname\x\else 1\fi\relax
\else\csname fi\endcsname
%</ignore>
%<*install>
\input docstrip.tex
\Msg{************************************************************************}
\Msg{* Installation}
\Msg{* Package: luacolor 2023-08-18 v1.18 Color support via LuaTeX's attributes (HO)}
\Msg{************************************************************************}

\keepsilent
\askforoverwritefalse

\let\MetaPrefix\relax
\preamble

This is a generated file.

Project: luacolor
Version: 2023-08-18 v1.18

Copyright (C)
   2007, 2009-2011 Heiko Oberdiek
   2016-2023 Oberdiek Package Support Group

This work may be distributed and/or modified under the
conditions of the LaTeX Project Public License, either
version 1.3c of this license or (at your option) any later
version. This version of this license is in
   https://www.latex-project.org/lppl/lppl-1-3c.txt
and the latest version of this license is in
   https://www.latex-project.org/lppl.txt
and version 1.3 or later is part of all distributions of
LaTeX version 2005/12/01 or later.

This work has the LPPL maintenance status "maintained".

The Current Maintainers of this work are
Heiko Oberdiek and the Oberdiek Package Support Group
https://github.com/ho-tex/luacolor/issues


This work consists of the main source file luacolor.dtx
and the derived files
   luacolor.sty, luacolor.pdf, luacolor.ins, luacolor.drv,
   luacolor.lua,

\endpreamble
\let\MetaPrefix\DoubleperCent

\generate{%
  \file{luacolor.ins}{\from{luacolor.dtx}{install}}%
  \file{luacolor.drv}{\from{luacolor.dtx}{driver}}%
  \usedir{tex/latex/luacolor}%
  \file{luacolor.sty}{\from{luacolor.dtx}{package}}%
}
\def\MetaPrefix{-- }
\def\defaultpostamble{%
  \MetaPrefix^^J%
  \MetaPrefix\space End of File `\outFileName'.%
}
\def\currentpostamble{\defaultpostamble}%
\generate{%
  \usedir{scripts/luacolor}%
%  \file{oberdiek.luacolor.lua}{\from{luacolor.dtx}{lua}}%
  \file{luacolor.lua}{\from{luacolor.dtx}{lua}}%
}

\catcode32=13\relax% active space
\let =\space%
\Msg{************************************************************************}
\Msg{*}
\Msg{* To finish the installation you have to move the following}
\Msg{* file into a directory searched by TeX:}
\Msg{*}
\Msg{*     luacolor.sty}
\Msg{*}
\Msg{* And install the following script file:}
\Msg{*}
\Msg{*     luacolor.lua,}
\Msg{*}
\Msg{* To produce the documentation run the file `luacolor.dtx'}
\Msg{* through LaTeX.}
\Msg{*}
\Msg{* Happy TeXing!}
\Msg{*}
\Msg{************************************************************************}

\endbatchfile
%</install>
%<*ignore>
\fi
%</ignore>
%<*driver>
\NeedsTeXFormat{LaTeX2e}
\ProvidesFile{luacolor.drv}%
  [2023-08-18 v1.18 Color support via LuaTeX's attributes (HO)]%
\documentclass{ltxdoc}
\usepackage{holtxdoc}[2011/11/22]
\usepackage{paralist}
\makeatletter
\g@addto@macro\MakePrivateLetters{%
  \@makeother\_%
}
\makeatother
\begin{document}
  \DocInput{luacolor.dtx}%
\end{document}
%</driver>
% \fi
%
%
%
% \GetFileInfo{luacolor.drv}
%
% \title{The \xpackage{luacolor} package}
% \date{2023-08-18 v1.18}
% \author{Heiko Oberdiek\thanks
% {Please report any issues at \url{https://github.com/ho-tex/luacolor/issues}}}
%
% \maketitle
%
% \begin{abstract}
% Package \xpackage{luacolor} implements color support based
% on \LuaTeX's node attributes.
% \end{abstract}
%
% \tableofcontents
%
% \section{Documentation}
%
% \subsection{Introduction}
%
% This package uses a \LuaTeX's attribute register to
% to annotate nodes with color information.
% If a color is set, then the attribute register
% is set to this color and all nodes created in its scope
% (current group) are annotated with this attribute.
% Now the color property behaves much the same way
% as the font property.
%
% \subsection{Usage}
%
% Package \xpackage{color} is loaded automatically by this
% package \xpackage{luacolor}. If you need a special driver
% option or you prefer package \xpackage{xcolor}, then load
% it before package \xpackage{luacolor}, for example:
% \begin{quote}
%   |\usepackage[dvipdfmx]{xcolor}|
% \end{quote}
% The package \xpackage{luacolor} is loaded without options:
% \begin{quote}
%   |\usepackage{luacolor}|
% \end{quote}
% It is able to detect PDF mode and DVI drivers are
% differentiated by its color specials. Therefore the
% package do need driver options.
%
% Then it redefines the color setting commands to set
% attributes instead of whatsits for color.
%
% At last the attribute annotations of the nodes in the
% output box must be analyzed to insert the necessary color whatsits.
% (With older \LuaTeX\ that lack the appropriate callback function 
% the package \xpackage{atbegshi} is used to get control
% before a box is shipped out.)
%
% \begin{declcs}{luacolorProcessBox} \M{box}
% \end{declcs}
% Macro \cs{luacolorProcessBox} processes the box \meta{box}
% in the previously described manner. It is automatically
% called for pages, but not for XForm objects.
% Before passing a box to \cs{pdfxform}, call \cs{luacolorProcessBox}
% first.
%
% \subsection{Limitations}
%
% \begin{description}
% \item[Ligatures with different colored components:]
%   Package \xpackage{luacolor} sees the ligature after
%   the paragraph building and page breaking, when a
%   page is to be shipped out. Therefore it cannot
%   break ligatures, because the components might occupy
%   different space. Therefore it is the responsibility
%   of the ligature forming process to deal with different
%   colored glyphs that form a ligature. The user can avoid
%   the problem entirely by explicitly breaking the ligature
%   at the places where the color changes.
%  \item \dots
% \end{description}
%
% \StopEventually{
% }
%
% \section{Implementation}
%
%    \begin{macrocode}
%<*package>
%    \end{macrocode}
%
% \subsection{Catcodes and identification}
%
%    \begin{macrocode}
\begingroup\catcode61\catcode48\catcode32=10\relax%
  \catcode13=5 % ^^M
  \endlinechar=13 %
  \catcode123=1 % {
  \catcode125=2 % }
  \catcode64=11 % @
  \def\x{\endgroup
    \expandafter\edef\csname LuaCol@AtEnd\endcsname{%
      \endlinechar=\the\endlinechar\relax
      \catcode13=\the\catcode13\relax
      \catcode32=\the\catcode32\relax
      \catcode35=\the\catcode35\relax
      \catcode61=\the\catcode61\relax
      \catcode64=\the\catcode64\relax
      \catcode123=\the\catcode123\relax
      \catcode125=\the\catcode125\relax
    }%
  }%
\x\catcode61\catcode48\catcode32=10\relax%
\catcode13=5 % ^^M
\endlinechar=13 %
\catcode35=6 % #
\catcode64=11 % @
\catcode123=1 % {
\catcode125=2 % }
\def\TMP@EnsureCode#1#2{%
  \edef\LuaCol@AtEnd{%
    \LuaCol@AtEnd
    \catcode#1=\the\catcode#1\relax
  }%
  \catcode#1=#2\relax
}
\TMP@EnsureCode{34}{12}% "
\TMP@EnsureCode{39}{12}% '
\TMP@EnsureCode{40}{12}% (
\TMP@EnsureCode{41}{12}% )
\TMP@EnsureCode{42}{12}% *
\TMP@EnsureCode{43}{12}% +
\TMP@EnsureCode{44}{12}% ,
\TMP@EnsureCode{45}{12}% -
\TMP@EnsureCode{46}{12}% .
\TMP@EnsureCode{47}{12}% /
\TMP@EnsureCode{58}{12}% :
\TMP@EnsureCode{60}{12}% <
\TMP@EnsureCode{62}{12}% >
\TMP@EnsureCode{91}{12}% [
\TMP@EnsureCode{93}{12}% ]
\TMP@EnsureCode{95}{12}% _ (other!)
\TMP@EnsureCode{96}{12}% `
\edef\LuaCol@AtEnd{\LuaCol@AtEnd\noexpand\endinput}
%    \end{macrocode}
%
%    Package identification.
%    \begin{macrocode}
\NeedsTeXFormat{LaTeX2e}
\ProvidesPackage{luacolor}%
  [2023-08-18 v1.18 Color support via LuaTeX's attributes (HO)]
%    \end{macrocode}
%
% \subsection{Check for \LuaTeX}
%
%    Without \LuaTeX\ there is no point in using this package.
%    \begin{macrocode}
\RequirePackage{color}
%    \end{macrocode}
%
%    \begin{macrocode}
\ifx\directlua\@undefined
  \PackageError{luacolor}{%
    This package may only be run using LuaTeX%
  }\@ehc
  \expandafter\LuaCol@AtEnd
\fi%
%    \end{macrocode}
%
% \subsection{Check for disabled colors}
%
%    \begin{macrocode}
\ifcolors@
\else
  \PackageWarningNoLine{luacolor}{%
    Colors are disabled by option `monochrome'%
  }%
  \def\set@color{}%
  \def\reset@color{}%
  \def\set@page@color{}%
  \def\define@color#1#2{}%
  \expandafter\LuaCol@AtEnd
\fi%
%    \end{macrocode}
%
% \subsection{Load module and check version}
%
%    \begin{macrocode}
\directlua{%
  require("luacolor")%
}
%    \end{macrocode}
%    \begin{macrocode}
\begingroup
  \edef\x{\directlua{tex.write("2023-08-18 v1.18")}}%
  \edef\y{%
    \directlua{%
      if oberdiek.luacolor.getversion then %
        oberdiek.luacolor.getversion()%
      end%
    }%
  }%
  \ifx\x\y
  \else
    \PackageError{luacolor}{%
      Wrong version of lua module.\MessageBreak
      Package version: \x\MessageBreak
      Lua module: \y
    }\@ehc
  \fi
\endgroup
%    \end{macrocode}
%
% \subsection{Find driver}
%
%    \begin{macrocode}
\ifnum\outputmode=\@ne
\else
  \begingroup
    \def\current@color{}%
    \def\reset@color{}%
    \setbox\z@=\hbox{%
      \begingroup
        \set@color
      \endgroup
    }%
    \edef\reserved@a{%
      \directlua{%
        oberdiek.luacolor.dvidetect()%
      }%
    }%
    \ifx\reserved@a\@empty
      \PackageError{luacolor}{%
        DVI driver detection failed because of\MessageBreak
        unrecognized color \string\special
      }\@ehc
      \endgroup
      \expandafter\expandafter\expandafter\LuaCol@AtEnd
    \else
      \PackageInfo{luacolor}{%
        Type of color \string\special: \reserved@a
      \@gobble}%
    \fi%
  \endgroup
\fi
%    \end{macrocode}
%
% \subsection{Attribute setting}
%
%    \begin{macro}{\LuaCol@Attribute}
%    \begin{macrocode}
\newattribute\LuaCol@Attribute
\let\LuaCol@setattribute\setattribute
\directlua{%
  oberdiek.luacolor.setattribute(\number\allocationnumber)%
}
%    \end{macrocode}
%    \end{macro}
%
%    \begin{macro}{\set@color}
% change 2023-08-18: added \cs{reset@color} so that \cs{mathcolor} can gobble it,
% issue 7.   
%    \begin{macrocode}
\protected\def\set@color{%
  \LuaCol@setattribute\LuaCol@Attribute{%
    \directlua{%
      oberdiek.luacolor.get("\luaescapestring{\current@color}")%
    }%
  }%
 \aftergroup\reset@color 
}
%    \end{macrocode}
%    \end{macro}
%    \begin{macro}{\reset@color}
%    \begin{macrocode}
\def\reset@color{}
%    \end{macrocode}
%    \end{macro}
%
% \subsection{Whatsit insertion}
%
%    \begin{macro}{\luacolorProcessBox}
%    \begin{macrocode}
\def\luacolorProcessBox#1{%
  \directlua{%
    oberdiek.luacolor.process(\number#1)%
  }%
}
%    \end{macrocode}
%    \end{macro}
%
%    \begin{macrocode}
\directlua{%
  if luatexbase.callbacktypes.pre_shipout_filter then
    token.get_next()
  end
}\@secondoftwo\@gobble{
  \RequirePackage{atbegshi}[2011/01/30]
  \AtBeginShipout{%
    \luacolorProcessBox\AtBeginShipoutBox
  }
}
%    \end{macrocode}
%
%    Set default color.
%    \begin{macrocode}
\set@color
%    \end{macrocode}
%
% \subsection{\cs{pdfxform}/\cs{saveboxresource} support}
%
%    \begin{macrocode}
\ifnum\outputmode=\@ne
    \let\LuaCol@org@pdfxform\saveboxresource
%    \end{macrocode}
% First we need some helpers to allow expandable code to parse keyword style arguments:
%    \begin{macrocode}
    \def\LuaCol@iii@i@ii#1#2#3{#3{#1}{#2}}
    \def\LuaCol@ii@i#1#2{{#2#1}}
    \def\LuaCol@if@keyword#1#2#3{%
      \expanded{\unexpanded{\LuaCol@iii@i@ii{#2}{#3}}\expandafter}%
      \directlua{%
        token.put_next(token.create(token.scan_keyword(token.scan_string())
        and '@firstoftwo'
        or '@secondoftwo'))
      }{#1}%
    }
%    \end{macrocode}
% The following macro scans a integer and expands to a token equivalent to a chardef
% whose value corresponds to the scanned integer. This allows the integer to be passed
% around as a undelimited argument.
%    \begin{macrocode}
    \def\LuaCol@scan@number{%
      \directlua{
        token.put_next(token.new(token.scan_int(), token.command_id'char_given'))
      }%
    }
%    \end{macrocode}
% \TeX\ primitives like \cs{saveboxresource} read braced arguments in a special way.
% Especially they expand everything until they find a left brace. To simulate this, we
% use Lua to expand everything else:
%    \begin{macrocode}
    \def\LuaCol@scan@tobrace{%
      \directlua{
        local relax, space = token.command_id'relax', token.command_id'spacer'
        local t
        repeat
          t = token.scan_token()
        until not (t.command == relax or t.command == space)
        token.put_next(t)
      }%
    }
    \def\LuaCol@scan@boxresource@i#1#2{%
      \LuaCol@if@keyword{attr}{%
        \expanded{\unexpanded{\LuaCol@scan@boxresource@iI{#1#2attr}}%
          \expandafter\expandafter\expandafter}%
        \LuaCol@scan@tobrace
      }{%
        \LuaCol@scan@boxresource@ii{#1#2}%
      }%
    }
    \def\LuaCol@scan@boxresource@iI#1#2{\LuaCol@scan@boxresource@ii{#1{#2}}}
    \def\LuaCol@scan@boxresource@ii#1{%
      \LuaCol@if@keyword{resources}{%
        \expanded{\unexpanded{\LuaCol@scan@boxresource@iiI{#1resources}}%
          \expandafter\expandafter\expandafter}%
        \LuaCol@scan@tobrace
      }{%
        \LuaCol@scan@boxresource@iii{#1}%
      }%
    }
    \def\LuaCol@scan@boxresource@iiI#1#2{\LuaCol@scan@boxresource@iii{#1{#2}}}
    \def\LuaCol@scan@boxresource@iii#1{%
      \LuaCol@if@keyword{margin}{%
        \expanded{\unexpanded{\LuaCol@scan@boxresource@iv{#1margin }}%
          \expandafter\expandafter\expandafter}%
        \LuaCol@scan@number
      }{%
        \LuaCol@scan@boxresource@iv{#1}{}%
      }%
    }
    \def\LuaCol@scan@boxresource@iv#1#2{%
      \expanded{\unexpanded{\LuaCol@scan@boxresource@v{#1#2}}%
        \expandafter\expandafter\expandafter}%
      \LuaCol@scan@number
    }
    \def\LuaCol@scan@boxresource@v#1#2{%
      \luacolorProcessBox{#2}%
      \LuaCol@org@pdfxform#1#2%
    }

%    \end{macrocode}
% This could be written in Lua, but at least upto Lua\TeX~1.11, feeding back too many tokens
% from Lua to \TeX\ triggers a segmentation fault.
% This is written in Lua so the integer setting is expandable and does not interfere with
% a preceding \verb|\immediate|.
%    \begin{macrocode}
    \protected\def\saveboxresource{%
      \LuaCol@if@keyword{type}{%
        \expandafter
        \expanded{\unexpanded{\LuaCol@scan@boxresource@i{type }}%
          \expandafter\expandafter\expandafter}%
        \LuaCol@scan@number
      }{%
        \LuaCol@scan@boxresource@i{}{}%
      }%
    }
%    \end{macrocode}
%    Legacy alias.
%    \begin{macrocode}
    \let\pdfxform\saveboxresource
\fi
%    \end{macrocode}
%
%    \begin{macrocode}
\LuaCol@AtEnd%
%</package>
%    \end{macrocode}
%
% \subsection{Lua module}
%
%    \begin{macrocode}
%<*lua>
%    \end{macrocode}
%    Box zero contains a \cs{hbox} with the color \cs{special}.
%    That is analyzed to get the prefix for the color setting
%    \cs{special}.
%    \begin{macrocode}
oberdiek = oberdiek or {}
local luacolor = oberdiek.luacolor or {}
oberdiek.luacolor = luacolor
%    \end{macrocode}
%    \begin{macro}{getversion()}
%    \begin{macrocode}
function luacolor.getversion()
  tex.write("2023-08-18 v1.18")
end
%    \end{macrocode}
%    \end{macro}
%
% \subsubsection{Driver detection}
%
%    \begin{macrocode}
local ifpdf = tonumber(tex.outputmode or tex.pdfoutput) > 0
local prefix
local prefixes = {
  dvips   = "color ",
  dvipdfm = "pdf:sc ",
  truetex = "textcolor:",
  pctexps = "ps::",
}
local patterns = {
  ["^color "]            = "dvips",
  ["^pdf: *begincolor "] = "dvipdfm",
  ["^pdf: *bcolor "]     = "dvipdfm",
  ["^pdf: *bc "]         = "dvipdfm",
  ["^pdf: *setcolor "]   = "dvipdfm",
  ["^pdf: *scolor "]     = "dvipdfm",
  ["^pdf: *sc "]         = "dvipdfm",
  ["^textcolor:"]        = "truetex",
  ["^ps::"]              = "pctexps",
}
%    \end{macrocode}
%    \begin{macro}{info()}
%    \begin{macrocode}
local function info(msg, term)
  local target = "log"
  if term then
    target = "term and log"
  end
  texio.write_nl(target, "Package luacolor info: " .. msg .. ".")
  texio.write_nl(target, "")
end
%    \end{macrocode}
%    \end{macro}
%    \begin{macro}{dvidetect()}
%    \begin{macrocode}
function luacolor.dvidetect()
  local v = tex.box[0]
  assert(v.id == node.id("hlist"))
  for v in node.traverse_id(node.id("whatsit"), v.head) do
    if v and v.subtype == node.subtype("special") then
      local data = v.data
      for pattern, driver in pairs(patterns) do
        if string.find(data, pattern) then
          prefix = prefixes[driver]
          tex.write(driver)
          return
        end
      end
      info("\\special{" .. data .. "}", true)
      return
    end
  end
  info("Missing \\special", true)
end
%    \end{macrocode}
%    \end{macro}
%
% \subsubsection{Color strings}
%
%    \begin{macrocode}
local map = {
  n = 0,
}
%    \end{macrocode}
%    \begin{macro}{get()}
%    \begin{macrocode}
function luacolor.get(color)
  tex.write("" .. luacolor.getvalue(color))
end
%    \end{macrocode}
%    \end{macro}
%    \begin{macro}{getvalue()}
%    \begin{macrocode}
function luacolor.getvalue(color)
  local n = map[color]
  if not n then
    n = map.n + 1
    map.n = n
    map[n] = color
    map[color] = n
  end
  return n
end
%    \end{macrocode}
%    \end{macro}
%
% \subsubsection{Attribute register}
%
%    \begin{macro}{setattribute()}
%    \begin{macrocode}
local attribute
function luacolor.setattribute(attr)
  attribute = attr
end
%    \end{macrocode}
%    \end{macro}
%
%    \begin{macro}{getattribute()}
%    \begin{macrocode}
function luacolor.getattribute()
  return attribute
end
%    \end{macrocode}
%    \end{macro}
%
% \subsubsection{Whatsit insertion}
%
%    \begin{macrocode}
local LIST = 1
local LIST_LEADERS = 2
local LIST_DISC = 3
local COLOR = 4
local NOCOLOR = 5
local RULE = node.id("rule")
local node_types = {
  [node.id("hlist")] = LIST,
  [node.id("vlist")] = LIST,
  [node.id("rule")]  = COLOR,
  [node.id("glyph")] = COLOR,
  [node.id("disc")]  = LIST_DISC,
  [node.id("whatsit")] = {
    [node.subtype("pdf_colorstack")] =
      function(n)
        return n.stack == 0 and NOCOLOR or nil
      end,
    [node.subtype("special")] = COLOR,
    [node.subtype("pdf_literal")] = COLOR,
    [node.subtype("pdf_save")] = COLOR,
    [node.subtype("pdf_restore")] = COLOR, -- probably not needed
-- TODO (DPC)    [node.subtype("pdf_refximage")] = COLOR,
  },
  [node.id("glue")] =
    function(n)
      if n.subtype >= 100 then -- leaders
        if n.leader.id == RULE then
          return COLOR
        else
          return LIST_LEADERS
        end
      end
    end,
}
%    \end{macrocode}
%    \begin{macro}{get_type()}
%    \begin{macrocode}
local function get_type(n)
  local ret = node_types[n.id]
  if type(ret) == 'table' then
    ret = ret[n.subtype]
  end
  if type(ret) == 'function' then
    ret = ret(n)
  end
  return ret
end
%    \end{macrocode}
%    \end{macro}
%    \begin{macrocode}
local mode = 2 -- luatex.pdfliteral.direct
local WHATSIT = node.id("whatsit")
local SPECIAL = node.subtype("special")
local PDFLITERAL = node.subtype("pdf_literal")
local DRY_FALSE = false
local DRY_TRUE = true
%    \end{macrocode}
%    \begin{macro}{traverse()}
%    \begin{macrocode}
local function traverse(list, color, dry)
  if not list then
    return color
  end
  local head
  if get_type(list) == LIST then
    head = list.head
  elseif get_type(list) == LIST_DISC then
    head = list.replace
  else
    texio.write_nl("!!! Error: Wrong list type: " .. node.type(list.id))
    return color
  end
%<debug>texio.write_nl("traverse: " .. node.type(list.id))
  for n in node.traverse(head) do
%<debug>texio.write_nl("  node: " .. node.type(n.id))
    local t = get_type(n)
%<debug>texio.write_nl("TYPE "..tostring(t).. " "..tostring(node.type(node.getid(n))).." ".. tostring(node.getsubtype(n)))
    if t == LIST or t == LIST_DISC then
      color = traverse(n, color, dry)
    elseif t == LIST_LEADERS then
      local color_after = traverse(n.leader, color, DRY_TRUE)
      if color == color_after then
        traverse(n.leader, color, DRY_FALSE or dry)
      else
        traverse(n.leader, '', DRY_FALSE or dry)
%    \end{macrocode}
% The color status is unknown here, because the leader box
% will or will not be set.
%    \begin{macrocode}
        color = ''
      end
    elseif t == COLOR then
      local v = node.has_attribute(n, attribute)
      if v then
        local newColor = map[v]
        if newColor ~= color then
          color = newColor
          if dry == DRY_FALSE then
            local newNode
            if ifpdf then
              newNode = node.new(WHATSIT, PDFLITERAL)
              newNode.mode = mode
              newNode.data = color
            else
              newNode = node.new(WHATSIT, SPECIAL)
              newNode.data = prefix .. color
            end
            head = node.insert_before(head, n, newNode)
          end
        end
      end
    elseif t == NOCOLOR then
      color = ''
    end
  end
  if get_type(list) == LIST then
    list.head = head
  else
    list.replace = head
  end
  return color
end
%    \end{macrocode}
%    \end{macro}
%
%    \begin{macro}{process()}
%    \begin{macrocode}
function luacolor.process(box)
  local color = ""
  local list = tex.getbox(box)
  traverse(list, color, DRY_FALSE)
end

if luatexbase.callbacktypes.pre_shipout_filter then
  luatexbase.add_to_callback("pre_shipout_filter", function(list)
    traverse(list, "", DRY_FALSE)
    return true
  end, "luacolor.process")
end
%    \end{macrocode}
%    \end{macro}
% For recent versions of luaotfload, we can register a callback to
% control how coloring glyph is handled for the color feature.
%    \begin{macrocode}
if luaotfload.set_colorhandler then
  local set_attribute = node.direct.set_attribute
  luaotfload.set_colorhandler(function(head, n, color)
    set_attribute(n, attribute, luacolor.getvalue(color))
    return head, n
  end)
end
%    \end{macrocode}
%
%    \begin{macrocode}
%</lua>
%    \end{macrocode}
% \section{Installation}
%
% \subsection{Download}
%
% \paragraph{Package.} This package is available on
% CTAN\footnote{\CTANpkg{luacolor}}:
% \begin{description}
% \item[\CTAN{macros/latex/contrib/luacolor/luacolor.dtx}] The source file.
% \item[\CTAN{macros/latex/contrib/luacolor/luacolor.pdf}] Documentation.
% \end{description}
%
%
% \paragraph{Bundle.} All the packages of the bundle `luacolor'
% are also available in a TDS compliant ZIP archive. There
% the packages are already unpacked and the documentation files
% are generated. The files and directories obey the TDS standard.
% \begin{description}
% \item[\CTANinstall{install/macros/latex/contrib/luacolor.tds.zip}]
% \end{description}
% \emph{TDS} refers to the standard ``A Directory Structure
% for \TeX\ Files'' (\CTANpkg{tds}). Directories
% with \xfile{texmf} in their name are usually organized this way.
%
% \subsection{Bundle installation}
%
% \paragraph{Unpacking.} Unpack the \xfile{luacolor.tds.zip} in the
% TDS tree (also known as \xfile{texmf} tree) of your choice.
% Example (linux):
% \begin{quote}
%   |unzip luacolor.tds.zip -d ~/texmf|
% \end{quote}
%
% \paragraph{Script installation.}
% Check the directory \xfile{TDS:scripts/luacolor/} for
% scripts that need further installation steps.
%
% \subsection{Package installation}
%
% \paragraph{Unpacking.} The \xfile{.dtx} file is a self-extracting
% \docstrip\ archive. The files are extracted by running the
% \xfile{.dtx} through \plainTeX:
% \begin{quote}
%   \verb|tex luacolor.dtx|
% \end{quote}
%
% \paragraph{TDS.} Now the different files must be moved into
% the different directories in your installation TDS tree
% (also known as \xfile{texmf} tree):
% \begin{quote}
% \def\t{^^A
% \begin{tabular}{@{}>{\ttfamily}l@{ $\rightarrow$ }>{\ttfamily}l@{}}
%   luacolor.sty & tex/latex/luacolor/luacolor.sty\\
%   luacolor.lua & scripts/luacolor/luacolor.lua\\
%   luacolor.pdf & doc/latex/luacolor/luacolor.pdf\\
%   luacolor.dtx & source/latex/luacolor/luacolor.dtx\\
% \end{tabular}^^A
% }^^A
% \sbox0{\t}^^A
% \ifdim\wd0>\linewidth
%   \begingroup
%     \advance\linewidth by\leftmargin
%     \advance\linewidth by\rightmargin
%   \edef\x{\endgroup
%     \def\noexpand\lw{\the\linewidth}^^A
%   }\x
%   \def\lwbox{^^A
%     \leavevmode
%     \hbox to \linewidth{^^A
%       \kern-\leftmargin\relax
%       \hss
%       \usebox0
%       \hss
%       \kern-\rightmargin\relax
%     }^^A
%   }^^A
%   \ifdim\wd0>\lw
%     \sbox0{\small\t}^^A
%     \ifdim\wd0>\linewidth
%       \ifdim\wd0>\lw
%         \sbox0{\footnotesize\t}^^A
%         \ifdim\wd0>\linewidth
%           \ifdim\wd0>\lw
%             \sbox0{\scriptsize\t}^^A
%             \ifdim\wd0>\linewidth
%               \ifdim\wd0>\lw
%                 \sbox0{\tiny\t}^^A
%                 \ifdim\wd0>\linewidth
%                   \lwbox
%                 \else
%                   \usebox0
%                 \fi
%               \else
%                 \lwbox
%               \fi
%             \else
%               \usebox0
%             \fi
%           \else
%             \lwbox
%           \fi
%         \else
%           \usebox0
%         \fi
%       \else
%         \lwbox
%       \fi
%     \else
%       \usebox0
%     \fi
%   \else
%     \lwbox
%   \fi
% \else
%   \usebox0
% \fi
% \end{quote}
% If you have a \xfile{docstrip.cfg} that configures and enables \docstrip's
% TDS installing feature, then some files can already be in the right
% place, see the documentation of \docstrip.
%
% \subsection{Refresh file name databases}
%
% If your \TeX~distribution
% (\TeX\,Live, \mikTeX, \dots) relies on file name databases, you must refresh
% these. For example, \TeX\,Live\ users run \verb|texhash| or
% \verb|mktexlsr|.
%
% \subsection{Some details for the interested}
%
% \paragraph{Unpacking with \LaTeX.}
% The \xfile{.dtx} chooses its action depending on the format:
% \begin{description}
% \item[\plainTeX:] Run \docstrip\ and extract the files.
% \item[\LaTeX:] Generate the documentation.
% \end{description}
% If you insist on using \LaTeX\ for \docstrip\ (really,
% \docstrip\ does not need \LaTeX), then inform the autodetect routine
% about your intention:
% \begin{quote}
%   \verb|latex \let\install=y% \iffalse meta-comment
%
% File: luacolor.dtx
% Version: 2023-08-18 v1.18
% Info: Color support via LuaTeX's attributes
%
% Copyright (C)
%    2007, 2009-2011 Heiko Oberdiek
%    2016-2023 Oberdiek Package Support Group
%    https://github.com/ho-tex/luacolor/issues
%
% This work may be distributed and/or modified under the
% conditions of the LaTeX Project Public License, either
% version 1.3c of this license or (at your option) any later
% version. This version of this license is in
%    https://www.latex-project.org/lppl/lppl-1-3c.txt
% and the latest version of this license is in
%    https://www.latex-project.org/lppl.txt
% and version 1.3 or later is part of all distributions of
% LaTeX version 2005/12/01 or later.
%
% This work has the LPPL maintenance status "maintained".
%
% The Current Maintainers of this work are
% Heiko Oberdiek and the Oberdiek Package Support Group
% https://github.com/ho-tex/luacolor/issues
%
% This work consists of the main source file luacolor.dtx
% and the derived files
%    luacolor.sty, luacolor.pdf, luacolor.lua,
%
% Distribution:
%    CTAN:macros/latex/contrib/luacolor/luacolor.dtx
%    CTAN:macros/latex/contrib/luacolor/luacolor.pdf
%
% Unpacking:
%    (a) If luacolor.ins is present:
%           tex luacolor.ins
%    (b) Without luacolor.ins:
%           tex luacolor.dtx
%    (c) If you insist on using LaTeX
%           latex \let\install=y\input{luacolor.dtx}
%        (quote the arguments according to the demands of your shell)
%
% Documentation:
%    (a) If luacolor.drv is present:
%           latex luacolor.drv
%    (b) Without luacolor.drv:
%           latex luacolor.dtx; ...
%    The class ltxdoc loads the configuration file ltxdoc.cfg
%    if available. Here you can specify further options, e.g.
%    use A4 as paper format:
%       \PassOptionsToClass{a4paper}{article}
%
%    Programm calls to get the documentation (example):
%       pdflatex luacolor.dtx
%       makeindex -s gind.ist luacolor.idx
%       pdflatex luacolor.dtx
%       makeindex -s gind.ist luacolor.idx
%       pdflatex luacolor.dtx
%
% Installation:
%    TDS:tex/latex/luacolor/luacolor.sty
%    TDS:scripts/luacolor/luacolor.lua
%    TDS:doc/latex/luacolor/luacolor.pdf
%    TDS:source/latex/luacolor/luacolor.dtx
%
%<*ignore>
\begingroup
  \catcode123=1 %
  \catcode125=2 %
  \def\x{LaTeX2e}%
\expandafter\endgroup
\ifcase 0\ifx\install y1\fi\expandafter
         \ifx\csname processbatchFile\endcsname\relax\else1\fi
         \ifx\fmtname\x\else 1\fi\relax
\else\csname fi\endcsname
%</ignore>
%<*install>
\input docstrip.tex
\Msg{************************************************************************}
\Msg{* Installation}
\Msg{* Package: luacolor 2023-08-18 v1.18 Color support via LuaTeX's attributes (HO)}
\Msg{************************************************************************}

\keepsilent
\askforoverwritefalse

\let\MetaPrefix\relax
\preamble

This is a generated file.

Project: luacolor
Version: 2023-08-18 v1.18

Copyright (C)
   2007, 2009-2011 Heiko Oberdiek
   2016-2023 Oberdiek Package Support Group

This work may be distributed and/or modified under the
conditions of the LaTeX Project Public License, either
version 1.3c of this license or (at your option) any later
version. This version of this license is in
   https://www.latex-project.org/lppl/lppl-1-3c.txt
and the latest version of this license is in
   https://www.latex-project.org/lppl.txt
and version 1.3 or later is part of all distributions of
LaTeX version 2005/12/01 or later.

This work has the LPPL maintenance status "maintained".

The Current Maintainers of this work are
Heiko Oberdiek and the Oberdiek Package Support Group
https://github.com/ho-tex/luacolor/issues


This work consists of the main source file luacolor.dtx
and the derived files
   luacolor.sty, luacolor.pdf, luacolor.ins, luacolor.drv,
   luacolor.lua,

\endpreamble
\let\MetaPrefix\DoubleperCent

\generate{%
  \file{luacolor.ins}{\from{luacolor.dtx}{install}}%
  \file{luacolor.drv}{\from{luacolor.dtx}{driver}}%
  \usedir{tex/latex/luacolor}%
  \file{luacolor.sty}{\from{luacolor.dtx}{package}}%
}
\def\MetaPrefix{-- }
\def\defaultpostamble{%
  \MetaPrefix^^J%
  \MetaPrefix\space End of File `\outFileName'.%
}
\def\currentpostamble{\defaultpostamble}%
\generate{%
  \usedir{scripts/luacolor}%
%  \file{oberdiek.luacolor.lua}{\from{luacolor.dtx}{lua}}%
  \file{luacolor.lua}{\from{luacolor.dtx}{lua}}%
}

\catcode32=13\relax% active space
\let =\space%
\Msg{************************************************************************}
\Msg{*}
\Msg{* To finish the installation you have to move the following}
\Msg{* file into a directory searched by TeX:}
\Msg{*}
\Msg{*     luacolor.sty}
\Msg{*}
\Msg{* And install the following script file:}
\Msg{*}
\Msg{*     luacolor.lua,}
\Msg{*}
\Msg{* To produce the documentation run the file `luacolor.dtx'}
\Msg{* through LaTeX.}
\Msg{*}
\Msg{* Happy TeXing!}
\Msg{*}
\Msg{************************************************************************}

\endbatchfile
%</install>
%<*ignore>
\fi
%</ignore>
%<*driver>
\NeedsTeXFormat{LaTeX2e}
\ProvidesFile{luacolor.drv}%
  [2023-08-18 v1.18 Color support via LuaTeX's attributes (HO)]%
\documentclass{ltxdoc}
\usepackage{holtxdoc}[2011/11/22]
\usepackage{paralist}
\makeatletter
\g@addto@macro\MakePrivateLetters{%
  \@makeother\_%
}
\makeatother
\begin{document}
  \DocInput{luacolor.dtx}%
\end{document}
%</driver>
% \fi
%
%
%
% \GetFileInfo{luacolor.drv}
%
% \title{The \xpackage{luacolor} package}
% \date{2023-08-18 v1.18}
% \author{Heiko Oberdiek\thanks
% {Please report any issues at \url{https://github.com/ho-tex/luacolor/issues}}}
%
% \maketitle
%
% \begin{abstract}
% Package \xpackage{luacolor} implements color support based
% on \LuaTeX's node attributes.
% \end{abstract}
%
% \tableofcontents
%
% \section{Documentation}
%
% \subsection{Introduction}
%
% This package uses a \LuaTeX's attribute register to
% to annotate nodes with color information.
% If a color is set, then the attribute register
% is set to this color and all nodes created in its scope
% (current group) are annotated with this attribute.
% Now the color property behaves much the same way
% as the font property.
%
% \subsection{Usage}
%
% Package \xpackage{color} is loaded automatically by this
% package \xpackage{luacolor}. If you need a special driver
% option or you prefer package \xpackage{xcolor}, then load
% it before package \xpackage{luacolor}, for example:
% \begin{quote}
%   |\usepackage[dvipdfmx]{xcolor}|
% \end{quote}
% The package \xpackage{luacolor} is loaded without options:
% \begin{quote}
%   |\usepackage{luacolor}|
% \end{quote}
% It is able to detect PDF mode and DVI drivers are
% differentiated by its color specials. Therefore the
% package do need driver options.
%
% Then it redefines the color setting commands to set
% attributes instead of whatsits for color.
%
% At last the attribute annotations of the nodes in the
% output box must be analyzed to insert the necessary color whatsits.
% (With older \LuaTeX\ that lack the appropriate callback function 
% the package \xpackage{atbegshi} is used to get control
% before a box is shipped out.)
%
% \begin{declcs}{luacolorProcessBox} \M{box}
% \end{declcs}
% Macro \cs{luacolorProcessBox} processes the box \meta{box}
% in the previously described manner. It is automatically
% called for pages, but not for XForm objects.
% Before passing a box to \cs{pdfxform}, call \cs{luacolorProcessBox}
% first.
%
% \subsection{Limitations}
%
% \begin{description}
% \item[Ligatures with different colored components:]
%   Package \xpackage{luacolor} sees the ligature after
%   the paragraph building and page breaking, when a
%   page is to be shipped out. Therefore it cannot
%   break ligatures, because the components might occupy
%   different space. Therefore it is the responsibility
%   of the ligature forming process to deal with different
%   colored glyphs that form a ligature. The user can avoid
%   the problem entirely by explicitly breaking the ligature
%   at the places where the color changes.
%  \item \dots
% \end{description}
%
% \StopEventually{
% }
%
% \section{Implementation}
%
%    \begin{macrocode}
%<*package>
%    \end{macrocode}
%
% \subsection{Catcodes and identification}
%
%    \begin{macrocode}
\begingroup\catcode61\catcode48\catcode32=10\relax%
  \catcode13=5 % ^^M
  \endlinechar=13 %
  \catcode123=1 % {
  \catcode125=2 % }
  \catcode64=11 % @
  \def\x{\endgroup
    \expandafter\edef\csname LuaCol@AtEnd\endcsname{%
      \endlinechar=\the\endlinechar\relax
      \catcode13=\the\catcode13\relax
      \catcode32=\the\catcode32\relax
      \catcode35=\the\catcode35\relax
      \catcode61=\the\catcode61\relax
      \catcode64=\the\catcode64\relax
      \catcode123=\the\catcode123\relax
      \catcode125=\the\catcode125\relax
    }%
  }%
\x\catcode61\catcode48\catcode32=10\relax%
\catcode13=5 % ^^M
\endlinechar=13 %
\catcode35=6 % #
\catcode64=11 % @
\catcode123=1 % {
\catcode125=2 % }
\def\TMP@EnsureCode#1#2{%
  \edef\LuaCol@AtEnd{%
    \LuaCol@AtEnd
    \catcode#1=\the\catcode#1\relax
  }%
  \catcode#1=#2\relax
}
\TMP@EnsureCode{34}{12}% "
\TMP@EnsureCode{39}{12}% '
\TMP@EnsureCode{40}{12}% (
\TMP@EnsureCode{41}{12}% )
\TMP@EnsureCode{42}{12}% *
\TMP@EnsureCode{43}{12}% +
\TMP@EnsureCode{44}{12}% ,
\TMP@EnsureCode{45}{12}% -
\TMP@EnsureCode{46}{12}% .
\TMP@EnsureCode{47}{12}% /
\TMP@EnsureCode{58}{12}% :
\TMP@EnsureCode{60}{12}% <
\TMP@EnsureCode{62}{12}% >
\TMP@EnsureCode{91}{12}% [
\TMP@EnsureCode{93}{12}% ]
\TMP@EnsureCode{95}{12}% _ (other!)
\TMP@EnsureCode{96}{12}% `
\edef\LuaCol@AtEnd{\LuaCol@AtEnd\noexpand\endinput}
%    \end{macrocode}
%
%    Package identification.
%    \begin{macrocode}
\NeedsTeXFormat{LaTeX2e}
\ProvidesPackage{luacolor}%
  [2023-08-18 v1.18 Color support via LuaTeX's attributes (HO)]
%    \end{macrocode}
%
% \subsection{Check for \LuaTeX}
%
%    Without \LuaTeX\ there is no point in using this package.
%    \begin{macrocode}
\RequirePackage{color}
%    \end{macrocode}
%
%    \begin{macrocode}
\ifx\directlua\@undefined
  \PackageError{luacolor}{%
    This package may only be run using LuaTeX%
  }\@ehc
  \expandafter\LuaCol@AtEnd
\fi%
%    \end{macrocode}
%
% \subsection{Check for disabled colors}
%
%    \begin{macrocode}
\ifcolors@
\else
  \PackageWarningNoLine{luacolor}{%
    Colors are disabled by option `monochrome'%
  }%
  \def\set@color{}%
  \def\reset@color{}%
  \def\set@page@color{}%
  \def\define@color#1#2{}%
  \expandafter\LuaCol@AtEnd
\fi%
%    \end{macrocode}
%
% \subsection{Load module and check version}
%
%    \begin{macrocode}
\directlua{%
  require("luacolor")%
}
%    \end{macrocode}
%    \begin{macrocode}
\begingroup
  \edef\x{\directlua{tex.write("2023-08-18 v1.18")}}%
  \edef\y{%
    \directlua{%
      if oberdiek.luacolor.getversion then %
        oberdiek.luacolor.getversion()%
      end%
    }%
  }%
  \ifx\x\y
  \else
    \PackageError{luacolor}{%
      Wrong version of lua module.\MessageBreak
      Package version: \x\MessageBreak
      Lua module: \y
    }\@ehc
  \fi
\endgroup
%    \end{macrocode}
%
% \subsection{Find driver}
%
%    \begin{macrocode}
\ifnum\outputmode=\@ne
\else
  \begingroup
    \def\current@color{}%
    \def\reset@color{}%
    \setbox\z@=\hbox{%
      \begingroup
        \set@color
      \endgroup
    }%
    \edef\reserved@a{%
      \directlua{%
        oberdiek.luacolor.dvidetect()%
      }%
    }%
    \ifx\reserved@a\@empty
      \PackageError{luacolor}{%
        DVI driver detection failed because of\MessageBreak
        unrecognized color \string\special
      }\@ehc
      \endgroup
      \expandafter\expandafter\expandafter\LuaCol@AtEnd
    \else
      \PackageInfo{luacolor}{%
        Type of color \string\special: \reserved@a
      \@gobble}%
    \fi%
  \endgroup
\fi
%    \end{macrocode}
%
% \subsection{Attribute setting}
%
%    \begin{macro}{\LuaCol@Attribute}
%    \begin{macrocode}
\newattribute\LuaCol@Attribute
\let\LuaCol@setattribute\setattribute
\directlua{%
  oberdiek.luacolor.setattribute(\number\allocationnumber)%
}
%    \end{macrocode}
%    \end{macro}
%
%    \begin{macro}{\set@color}
% change 2023-08-18: added \cs{reset@color} so that \cs{mathcolor} can gobble it,
% issue 7.   
%    \begin{macrocode}
\protected\def\set@color{%
  \LuaCol@setattribute\LuaCol@Attribute{%
    \directlua{%
      oberdiek.luacolor.get("\luaescapestring{\current@color}")%
    }%
  }%
 \aftergroup\reset@color 
}
%    \end{macrocode}
%    \end{macro}
%    \begin{macro}{\reset@color}
%    \begin{macrocode}
\def\reset@color{}
%    \end{macrocode}
%    \end{macro}
%
% \subsection{Whatsit insertion}
%
%    \begin{macro}{\luacolorProcessBox}
%    \begin{macrocode}
\def\luacolorProcessBox#1{%
  \directlua{%
    oberdiek.luacolor.process(\number#1)%
  }%
}
%    \end{macrocode}
%    \end{macro}
%
%    \begin{macrocode}
\directlua{%
  if luatexbase.callbacktypes.pre_shipout_filter then
    token.get_next()
  end
}\@secondoftwo\@gobble{
  \RequirePackage{atbegshi}[2011/01/30]
  \AtBeginShipout{%
    \luacolorProcessBox\AtBeginShipoutBox
  }
}
%    \end{macrocode}
%
%    Set default color.
%    \begin{macrocode}
\set@color
%    \end{macrocode}
%
% \subsection{\cs{pdfxform}/\cs{saveboxresource} support}
%
%    \begin{macrocode}
\ifnum\outputmode=\@ne
    \let\LuaCol@org@pdfxform\saveboxresource
%    \end{macrocode}
% First we need some helpers to allow expandable code to parse keyword style arguments:
%    \begin{macrocode}
    \def\LuaCol@iii@i@ii#1#2#3{#3{#1}{#2}}
    \def\LuaCol@ii@i#1#2{{#2#1}}
    \def\LuaCol@if@keyword#1#2#3{%
      \expanded{\unexpanded{\LuaCol@iii@i@ii{#2}{#3}}\expandafter}%
      \directlua{%
        token.put_next(token.create(token.scan_keyword(token.scan_string())
        and '@firstoftwo'
        or '@secondoftwo'))
      }{#1}%
    }
%    \end{macrocode}
% The following macro scans a integer and expands to a token equivalent to a chardef
% whose value corresponds to the scanned integer. This allows the integer to be passed
% around as a undelimited argument.
%    \begin{macrocode}
    \def\LuaCol@scan@number{%
      \directlua{
        token.put_next(token.new(token.scan_int(), token.command_id'char_given'))
      }%
    }
%    \end{macrocode}
% \TeX\ primitives like \cs{saveboxresource} read braced arguments in a special way.
% Especially they expand everything until they find a left brace. To simulate this, we
% use Lua to expand everything else:
%    \begin{macrocode}
    \def\LuaCol@scan@tobrace{%
      \directlua{
        local relax, space = token.command_id'relax', token.command_id'spacer'
        local t
        repeat
          t = token.scan_token()
        until not (t.command == relax or t.command == space)
        token.put_next(t)
      }%
    }
    \def\LuaCol@scan@boxresource@i#1#2{%
      \LuaCol@if@keyword{attr}{%
        \expanded{\unexpanded{\LuaCol@scan@boxresource@iI{#1#2attr}}%
          \expandafter\expandafter\expandafter}%
        \LuaCol@scan@tobrace
      }{%
        \LuaCol@scan@boxresource@ii{#1#2}%
      }%
    }
    \def\LuaCol@scan@boxresource@iI#1#2{\LuaCol@scan@boxresource@ii{#1{#2}}}
    \def\LuaCol@scan@boxresource@ii#1{%
      \LuaCol@if@keyword{resources}{%
        \expanded{\unexpanded{\LuaCol@scan@boxresource@iiI{#1resources}}%
          \expandafter\expandafter\expandafter}%
        \LuaCol@scan@tobrace
      }{%
        \LuaCol@scan@boxresource@iii{#1}%
      }%
    }
    \def\LuaCol@scan@boxresource@iiI#1#2{\LuaCol@scan@boxresource@iii{#1{#2}}}
    \def\LuaCol@scan@boxresource@iii#1{%
      \LuaCol@if@keyword{margin}{%
        \expanded{\unexpanded{\LuaCol@scan@boxresource@iv{#1margin }}%
          \expandafter\expandafter\expandafter}%
        \LuaCol@scan@number
      }{%
        \LuaCol@scan@boxresource@iv{#1}{}%
      }%
    }
    \def\LuaCol@scan@boxresource@iv#1#2{%
      \expanded{\unexpanded{\LuaCol@scan@boxresource@v{#1#2}}%
        \expandafter\expandafter\expandafter}%
      \LuaCol@scan@number
    }
    \def\LuaCol@scan@boxresource@v#1#2{%
      \luacolorProcessBox{#2}%
      \LuaCol@org@pdfxform#1#2%
    }

%    \end{macrocode}
% This could be written in Lua, but at least upto Lua\TeX~1.11, feeding back too many tokens
% from Lua to \TeX\ triggers a segmentation fault.
% This is written in Lua so the integer setting is expandable and does not interfere with
% a preceding \verb|\immediate|.
%    \begin{macrocode}
    \protected\def\saveboxresource{%
      \LuaCol@if@keyword{type}{%
        \expandafter
        \expanded{\unexpanded{\LuaCol@scan@boxresource@i{type }}%
          \expandafter\expandafter\expandafter}%
        \LuaCol@scan@number
      }{%
        \LuaCol@scan@boxresource@i{}{}%
      }%
    }
%    \end{macrocode}
%    Legacy alias.
%    \begin{macrocode}
    \let\pdfxform\saveboxresource
\fi
%    \end{macrocode}
%
%    \begin{macrocode}
\LuaCol@AtEnd%
%</package>
%    \end{macrocode}
%
% \subsection{Lua module}
%
%    \begin{macrocode}
%<*lua>
%    \end{macrocode}
%    Box zero contains a \cs{hbox} with the color \cs{special}.
%    That is analyzed to get the prefix for the color setting
%    \cs{special}.
%    \begin{macrocode}
oberdiek = oberdiek or {}
local luacolor = oberdiek.luacolor or {}
oberdiek.luacolor = luacolor
%    \end{macrocode}
%    \begin{macro}{getversion()}
%    \begin{macrocode}
function luacolor.getversion()
  tex.write("2023-08-18 v1.18")
end
%    \end{macrocode}
%    \end{macro}
%
% \subsubsection{Driver detection}
%
%    \begin{macrocode}
local ifpdf = tonumber(tex.outputmode or tex.pdfoutput) > 0
local prefix
local prefixes = {
  dvips   = "color ",
  dvipdfm = "pdf:sc ",
  truetex = "textcolor:",
  pctexps = "ps::",
}
local patterns = {
  ["^color "]            = "dvips",
  ["^pdf: *begincolor "] = "dvipdfm",
  ["^pdf: *bcolor "]     = "dvipdfm",
  ["^pdf: *bc "]         = "dvipdfm",
  ["^pdf: *setcolor "]   = "dvipdfm",
  ["^pdf: *scolor "]     = "dvipdfm",
  ["^pdf: *sc "]         = "dvipdfm",
  ["^textcolor:"]        = "truetex",
  ["^ps::"]              = "pctexps",
}
%    \end{macrocode}
%    \begin{macro}{info()}
%    \begin{macrocode}
local function info(msg, term)
  local target = "log"
  if term then
    target = "term and log"
  end
  texio.write_nl(target, "Package luacolor info: " .. msg .. ".")
  texio.write_nl(target, "")
end
%    \end{macrocode}
%    \end{macro}
%    \begin{macro}{dvidetect()}
%    \begin{macrocode}
function luacolor.dvidetect()
  local v = tex.box[0]
  assert(v.id == node.id("hlist"))
  for v in node.traverse_id(node.id("whatsit"), v.head) do
    if v and v.subtype == node.subtype("special") then
      local data = v.data
      for pattern, driver in pairs(patterns) do
        if string.find(data, pattern) then
          prefix = prefixes[driver]
          tex.write(driver)
          return
        end
      end
      info("\\special{" .. data .. "}", true)
      return
    end
  end
  info("Missing \\special", true)
end
%    \end{macrocode}
%    \end{macro}
%
% \subsubsection{Color strings}
%
%    \begin{macrocode}
local map = {
  n = 0,
}
%    \end{macrocode}
%    \begin{macro}{get()}
%    \begin{macrocode}
function luacolor.get(color)
  tex.write("" .. luacolor.getvalue(color))
end
%    \end{macrocode}
%    \end{macro}
%    \begin{macro}{getvalue()}
%    \begin{macrocode}
function luacolor.getvalue(color)
  local n = map[color]
  if not n then
    n = map.n + 1
    map.n = n
    map[n] = color
    map[color] = n
  end
  return n
end
%    \end{macrocode}
%    \end{macro}
%
% \subsubsection{Attribute register}
%
%    \begin{macro}{setattribute()}
%    \begin{macrocode}
local attribute
function luacolor.setattribute(attr)
  attribute = attr
end
%    \end{macrocode}
%    \end{macro}
%
%    \begin{macro}{getattribute()}
%    \begin{macrocode}
function luacolor.getattribute()
  return attribute
end
%    \end{macrocode}
%    \end{macro}
%
% \subsubsection{Whatsit insertion}
%
%    \begin{macrocode}
local LIST = 1
local LIST_LEADERS = 2
local LIST_DISC = 3
local COLOR = 4
local NOCOLOR = 5
local RULE = node.id("rule")
local node_types = {
  [node.id("hlist")] = LIST,
  [node.id("vlist")] = LIST,
  [node.id("rule")]  = COLOR,
  [node.id("glyph")] = COLOR,
  [node.id("disc")]  = LIST_DISC,
  [node.id("whatsit")] = {
    [node.subtype("pdf_colorstack")] =
      function(n)
        return n.stack == 0 and NOCOLOR or nil
      end,
    [node.subtype("special")] = COLOR,
    [node.subtype("pdf_literal")] = COLOR,
    [node.subtype("pdf_save")] = COLOR,
    [node.subtype("pdf_restore")] = COLOR, -- probably not needed
-- TODO (DPC)    [node.subtype("pdf_refximage")] = COLOR,
  },
  [node.id("glue")] =
    function(n)
      if n.subtype >= 100 then -- leaders
        if n.leader.id == RULE then
          return COLOR
        else
          return LIST_LEADERS
        end
      end
    end,
}
%    \end{macrocode}
%    \begin{macro}{get_type()}
%    \begin{macrocode}
local function get_type(n)
  local ret = node_types[n.id]
  if type(ret) == 'table' then
    ret = ret[n.subtype]
  end
  if type(ret) == 'function' then
    ret = ret(n)
  end
  return ret
end
%    \end{macrocode}
%    \end{macro}
%    \begin{macrocode}
local mode = 2 -- luatex.pdfliteral.direct
local WHATSIT = node.id("whatsit")
local SPECIAL = node.subtype("special")
local PDFLITERAL = node.subtype("pdf_literal")
local DRY_FALSE = false
local DRY_TRUE = true
%    \end{macrocode}
%    \begin{macro}{traverse()}
%    \begin{macrocode}
local function traverse(list, color, dry)
  if not list then
    return color
  end
  local head
  if get_type(list) == LIST then
    head = list.head
  elseif get_type(list) == LIST_DISC then
    head = list.replace
  else
    texio.write_nl("!!! Error: Wrong list type: " .. node.type(list.id))
    return color
  end
%<debug>texio.write_nl("traverse: " .. node.type(list.id))
  for n in node.traverse(head) do
%<debug>texio.write_nl("  node: " .. node.type(n.id))
    local t = get_type(n)
%<debug>texio.write_nl("TYPE "..tostring(t).. " "..tostring(node.type(node.getid(n))).." ".. tostring(node.getsubtype(n)))
    if t == LIST or t == LIST_DISC then
      color = traverse(n, color, dry)
    elseif t == LIST_LEADERS then
      local color_after = traverse(n.leader, color, DRY_TRUE)
      if color == color_after then
        traverse(n.leader, color, DRY_FALSE or dry)
      else
        traverse(n.leader, '', DRY_FALSE or dry)
%    \end{macrocode}
% The color status is unknown here, because the leader box
% will or will not be set.
%    \begin{macrocode}
        color = ''
      end
    elseif t == COLOR then
      local v = node.has_attribute(n, attribute)
      if v then
        local newColor = map[v]
        if newColor ~= color then
          color = newColor
          if dry == DRY_FALSE then
            local newNode
            if ifpdf then
              newNode = node.new(WHATSIT, PDFLITERAL)
              newNode.mode = mode
              newNode.data = color
            else
              newNode = node.new(WHATSIT, SPECIAL)
              newNode.data = prefix .. color
            end
            head = node.insert_before(head, n, newNode)
          end
        end
      end
    elseif t == NOCOLOR then
      color = ''
    end
  end
  if get_type(list) == LIST then
    list.head = head
  else
    list.replace = head
  end
  return color
end
%    \end{macrocode}
%    \end{macro}
%
%    \begin{macro}{process()}
%    \begin{macrocode}
function luacolor.process(box)
  local color = ""
  local list = tex.getbox(box)
  traverse(list, color, DRY_FALSE)
end

if luatexbase.callbacktypes.pre_shipout_filter then
  luatexbase.add_to_callback("pre_shipout_filter", function(list)
    traverse(list, "", DRY_FALSE)
    return true
  end, "luacolor.process")
end
%    \end{macrocode}
%    \end{macro}
% For recent versions of luaotfload, we can register a callback to
% control how coloring glyph is handled for the color feature.
%    \begin{macrocode}
if luaotfload.set_colorhandler then
  local set_attribute = node.direct.set_attribute
  luaotfload.set_colorhandler(function(head, n, color)
    set_attribute(n, attribute, luacolor.getvalue(color))
    return head, n
  end)
end
%    \end{macrocode}
%
%    \begin{macrocode}
%</lua>
%    \end{macrocode}
% \section{Installation}
%
% \subsection{Download}
%
% \paragraph{Package.} This package is available on
% CTAN\footnote{\CTANpkg{luacolor}}:
% \begin{description}
% \item[\CTAN{macros/latex/contrib/luacolor/luacolor.dtx}] The source file.
% \item[\CTAN{macros/latex/contrib/luacolor/luacolor.pdf}] Documentation.
% \end{description}
%
%
% \paragraph{Bundle.} All the packages of the bundle `luacolor'
% are also available in a TDS compliant ZIP archive. There
% the packages are already unpacked and the documentation files
% are generated. The files and directories obey the TDS standard.
% \begin{description}
% \item[\CTANinstall{install/macros/latex/contrib/luacolor.tds.zip}]
% \end{description}
% \emph{TDS} refers to the standard ``A Directory Structure
% for \TeX\ Files'' (\CTANpkg{tds}). Directories
% with \xfile{texmf} in their name are usually organized this way.
%
% \subsection{Bundle installation}
%
% \paragraph{Unpacking.} Unpack the \xfile{luacolor.tds.zip} in the
% TDS tree (also known as \xfile{texmf} tree) of your choice.
% Example (linux):
% \begin{quote}
%   |unzip luacolor.tds.zip -d ~/texmf|
% \end{quote}
%
% \paragraph{Script installation.}
% Check the directory \xfile{TDS:scripts/luacolor/} for
% scripts that need further installation steps.
%
% \subsection{Package installation}
%
% \paragraph{Unpacking.} The \xfile{.dtx} file is a self-extracting
% \docstrip\ archive. The files are extracted by running the
% \xfile{.dtx} through \plainTeX:
% \begin{quote}
%   \verb|tex luacolor.dtx|
% \end{quote}
%
% \paragraph{TDS.} Now the different files must be moved into
% the different directories in your installation TDS tree
% (also known as \xfile{texmf} tree):
% \begin{quote}
% \def\t{^^A
% \begin{tabular}{@{}>{\ttfamily}l@{ $\rightarrow$ }>{\ttfamily}l@{}}
%   luacolor.sty & tex/latex/luacolor/luacolor.sty\\
%   luacolor.lua & scripts/luacolor/luacolor.lua\\
%   luacolor.pdf & doc/latex/luacolor/luacolor.pdf\\
%   luacolor.dtx & source/latex/luacolor/luacolor.dtx\\
% \end{tabular}^^A
% }^^A
% \sbox0{\t}^^A
% \ifdim\wd0>\linewidth
%   \begingroup
%     \advance\linewidth by\leftmargin
%     \advance\linewidth by\rightmargin
%   \edef\x{\endgroup
%     \def\noexpand\lw{\the\linewidth}^^A
%   }\x
%   \def\lwbox{^^A
%     \leavevmode
%     \hbox to \linewidth{^^A
%       \kern-\leftmargin\relax
%       \hss
%       \usebox0
%       \hss
%       \kern-\rightmargin\relax
%     }^^A
%   }^^A
%   \ifdim\wd0>\lw
%     \sbox0{\small\t}^^A
%     \ifdim\wd0>\linewidth
%       \ifdim\wd0>\lw
%         \sbox0{\footnotesize\t}^^A
%         \ifdim\wd0>\linewidth
%           \ifdim\wd0>\lw
%             \sbox0{\scriptsize\t}^^A
%             \ifdim\wd0>\linewidth
%               \ifdim\wd0>\lw
%                 \sbox0{\tiny\t}^^A
%                 \ifdim\wd0>\linewidth
%                   \lwbox
%                 \else
%                   \usebox0
%                 \fi
%               \else
%                 \lwbox
%               \fi
%             \else
%               \usebox0
%             \fi
%           \else
%             \lwbox
%           \fi
%         \else
%           \usebox0
%         \fi
%       \else
%         \lwbox
%       \fi
%     \else
%       \usebox0
%     \fi
%   \else
%     \lwbox
%   \fi
% \else
%   \usebox0
% \fi
% \end{quote}
% If you have a \xfile{docstrip.cfg} that configures and enables \docstrip's
% TDS installing feature, then some files can already be in the right
% place, see the documentation of \docstrip.
%
% \subsection{Refresh file name databases}
%
% If your \TeX~distribution
% (\TeX\,Live, \mikTeX, \dots) relies on file name databases, you must refresh
% these. For example, \TeX\,Live\ users run \verb|texhash| or
% \verb|mktexlsr|.
%
% \subsection{Some details for the interested}
%
% \paragraph{Unpacking with \LaTeX.}
% The \xfile{.dtx} chooses its action depending on the format:
% \begin{description}
% \item[\plainTeX:] Run \docstrip\ and extract the files.
% \item[\LaTeX:] Generate the documentation.
% \end{description}
% If you insist on using \LaTeX\ for \docstrip\ (really,
% \docstrip\ does not need \LaTeX), then inform the autodetect routine
% about your intention:
% \begin{quote}
%   \verb|latex \let\install=y\input{luacolor.dtx}|
% \end{quote}
% Do not forget to quote the argument according to the demands
% of your shell.
%
% \paragraph{Generating the documentation.}
% You can use both the \xfile{.dtx} or the \xfile{.drv} to generate
% the documentation. The process can be configured by the
% configuration file \xfile{ltxdoc.cfg}. For instance, put this
% line into this file, if you want to have A4 as paper format:
% \begin{quote}
%   \verb|\PassOptionsToClass{a4paper}{article}|
% \end{quote}
% An example follows how to generate the
% documentation with pdf\LaTeX:
% \begin{quote}
%\begin{verbatim}
%pdflatex luacolor.dtx
%makeindex -s gind.ist luacolor.idx
%pdflatex luacolor.dtx
%makeindex -s gind.ist luacolor.idx
%pdflatex luacolor.dtx
%\end{verbatim}
% \end{quote}
%
% \begin{History}
%   \begin{Version}{2007/12/12 v1.0}
%   \item
%     First public version.
%   \end{Version}
%   \begin{Version}{2009/04/10 v1.1}
%   \item
%     Fixes for changed syntax of \cs{directlua} in \LuaTeX\ 0.36.
%   \end{Version}
%   \begin{Version}{2010/03/09 v1.2}
%   \item
%     Adaptation for package \xpackage{luatex} 2010/03/09 v0.4.
%   \end{Version}
%   \begin{Version}{2010/12/13 v1.3}
%   \item
%     Support for \cs{pdfxform} added.
%   \item
%     Loaded package \xpackage{luatexbase-attr} recognized.
%   \item
%     Update for \hologo{LuaTeX}: `list' fields renamed to `head' in v0.65.0.
%   \end{Version}
%   \begin{Version}{2011/03/29 v1.4}
%   \item
%     Avoid whatsit insertion if option \xoption{monochrome} is used
%     (thanks Manuel P\'egouri\'e-Gonnard).
%   \end{Version}
%   \begin{Version}{2011/04/22 v1.5}
%   \item
%     Bug fix by Manuel P\'egouri\'e-Gonnard: A typo prevented the
%     detection of whatsits and applying color changes for
%     \cs{pdfliteral} and \cs{special} nodes that might contain
%     typesetting material.
%   \item
%     Bug fix by Manuel P\'egouri\'e-Gonnard: Now colors are also
%     applied to leader boxes.
%   \item
%     Unnecessary color settings are removed for leaders boxes,
%     if after the leader box the color has not changed.
%     The costs are a little runtime, leader boxes are processed twice.
%   \item
%     Additional whatsits that are colored: \texttt{pdf\_refximage}.
%   \item
%     Workaround for bug with \texttt{node.insert\_before}
%     removed for the version after \hologo{LuaTeX} 0.65, because
%     bug was fixed in 0.27. (Thanks Manuel P\'egouri\'e-Gonnard.)
%   \end{Version}
%   \begin{Version}{2011/04/23 v1.6}
%   \item
%     Bug fix for nested leader boxes.
%   \item
%     Bug fix for leader boxes that change color, but
%     are not set because of missing place.
%   \item
%     Version check for Lua module added.
%   \end{Version}
%   \begin{Version}{2011/10/22 v1.7}
%   \item
%     Lua functions \texttt{getattribute} and \texttt{getvalue}
%     added to tell other external Lua functions
%     the attribute register number for coloring.
%   \end{Version}
%   \begin{Version}{2011/11/01 v1.8}
%   \item
%     Use of \texttt{node.subtype} instead of magic numbers.
%   \end{Version}
%   \begin{Version}{2016/05/13 v1.9}
%   \item
%     More use of \texttt{node.subtype} instead of magic numbers.
%   \item luatex 85 updates
%   \end{Version}
%   \begin{Version}{2016/05/16 v1.10}
%   \item
%     Documentation updates.
%   \end{Version}
%   \begin{Version}{2018/11/22 v1.11}
%   \item
%     handle issue 43.
%   \item
%     removed pre-0.65 stuff
%   \end{Version}
%   \begin{Version}{2019/07/25 v1.12}
%   \item
%     removed uses of module function, see PR70
%   \end{Version}
%   \begin{Version}{2019/11/29 v1.13}
%   \item
%     Documentation updates.
%   \item  Use \xpackage{iftex} directly.
%   \end{Version}
%   \begin{Version}{2020-02-22 v1.14}
%   \item Drop use of \textsf{iftex} \textsf{ltxcmds} and \textsf{infwarerr}.
%   \item Assume \textsf{ltluatex} preloaded into format (true since 2015).
%   \item Patch \verb|\saveboxresource| rather than \verb|\pdfxform| (keep old name as alias).
%   \item Grab the number via Lua so that a \verb|\immediate| prefix still works with
%         \verb|\saveboxresource|\slash \verb|\pdfxform|.
%   \item Added handler for the color feature of luaotfload      
%   \end{Version}
%   \begin{Version}{2020-02-24 v1.15}
%   \item Grab all possible arguments for \verb|\saveboxresource|\slash \verb|\pdfxform|
%   \end{Version}
%   \begin{Version}{2020-04-04 v1.16}
%   \item Reset color after \verb|pdf_colorstack| whatsits.
%   \end{Version}
%   \begin{Version}{2021-02-17 v1.17}
%   \item Use \LaTeXe's new \verb|pre_shipout_filter| callback if it's available to
%         allow coloring background and foreground layer material
%   \end{Version}
%   \begin{Version}{2023-08-18 v1.18}
%   \item added \verb|\reset@color| to \verb|\set@color| for \verb|\mathcolor|, issue 7.
%   \end{Version}
% \end{History}
%
% \PrintIndex
%
% \Finale
\endinput
|
% \end{quote}
% Do not forget to quote the argument according to the demands
% of your shell.
%
% \paragraph{Generating the documentation.}
% You can use both the \xfile{.dtx} or the \xfile{.drv} to generate
% the documentation. The process can be configured by the
% configuration file \xfile{ltxdoc.cfg}. For instance, put this
% line into this file, if you want to have A4 as paper format:
% \begin{quote}
%   \verb|\PassOptionsToClass{a4paper}{article}|
% \end{quote}
% An example follows how to generate the
% documentation with pdf\LaTeX:
% \begin{quote}
%\begin{verbatim}
%pdflatex luacolor.dtx
%makeindex -s gind.ist luacolor.idx
%pdflatex luacolor.dtx
%makeindex -s gind.ist luacolor.idx
%pdflatex luacolor.dtx
%\end{verbatim}
% \end{quote}
%
% \begin{History}
%   \begin{Version}{2007/12/12 v1.0}
%   \item
%     First public version.
%   \end{Version}
%   \begin{Version}{2009/04/10 v1.1}
%   \item
%     Fixes for changed syntax of \cs{directlua} in \LuaTeX\ 0.36.
%   \end{Version}
%   \begin{Version}{2010/03/09 v1.2}
%   \item
%     Adaptation for package \xpackage{luatex} 2010/03/09 v0.4.
%   \end{Version}
%   \begin{Version}{2010/12/13 v1.3}
%   \item
%     Support for \cs{pdfxform} added.
%   \item
%     Loaded package \xpackage{luatexbase-attr} recognized.
%   \item
%     Update for \hologo{LuaTeX}: `list' fields renamed to `head' in v0.65.0.
%   \end{Version}
%   \begin{Version}{2011/03/29 v1.4}
%   \item
%     Avoid whatsit insertion if option \xoption{monochrome} is used
%     (thanks Manuel P\'egouri\'e-Gonnard).
%   \end{Version}
%   \begin{Version}{2011/04/22 v1.5}
%   \item
%     Bug fix by Manuel P\'egouri\'e-Gonnard: A typo prevented the
%     detection of whatsits and applying color changes for
%     \cs{pdfliteral} and \cs{special} nodes that might contain
%     typesetting material.
%   \item
%     Bug fix by Manuel P\'egouri\'e-Gonnard: Now colors are also
%     applied to leader boxes.
%   \item
%     Unnecessary color settings are removed for leaders boxes,
%     if after the leader box the color has not changed.
%     The costs are a little runtime, leader boxes are processed twice.
%   \item
%     Additional whatsits that are colored: \texttt{pdf\_refximage}.
%   \item
%     Workaround for bug with \texttt{node.insert\_before}
%     removed for the version after \hologo{LuaTeX} 0.65, because
%     bug was fixed in 0.27. (Thanks Manuel P\'egouri\'e-Gonnard.)
%   \end{Version}
%   \begin{Version}{2011/04/23 v1.6}
%   \item
%     Bug fix for nested leader boxes.
%   \item
%     Bug fix for leader boxes that change color, but
%     are not set because of missing place.
%   \item
%     Version check for Lua module added.
%   \end{Version}
%   \begin{Version}{2011/10/22 v1.7}
%   \item
%     Lua functions \texttt{getattribute} and \texttt{getvalue}
%     added to tell other external Lua functions
%     the attribute register number for coloring.
%   \end{Version}
%   \begin{Version}{2011/11/01 v1.8}
%   \item
%     Use of \texttt{node.subtype} instead of magic numbers.
%   \end{Version}
%   \begin{Version}{2016/05/13 v1.9}
%   \item
%     More use of \texttt{node.subtype} instead of magic numbers.
%   \item luatex 85 updates
%   \end{Version}
%   \begin{Version}{2016/05/16 v1.10}
%   \item
%     Documentation updates.
%   \end{Version}
%   \begin{Version}{2018/11/22 v1.11}
%   \item
%     handle issue 43.
%   \item
%     removed pre-0.65 stuff
%   \end{Version}
%   \begin{Version}{2019/07/25 v1.12}
%   \item
%     removed uses of module function, see PR70
%   \end{Version}
%   \begin{Version}{2019/11/29 v1.13}
%   \item
%     Documentation updates.
%   \item  Use \xpackage{iftex} directly.
%   \end{Version}
%   \begin{Version}{2020-02-22 v1.14}
%   \item Drop use of \textsf{iftex} \textsf{ltxcmds} and \textsf{infwarerr}.
%   \item Assume \textsf{ltluatex} preloaded into format (true since 2015).
%   \item Patch \verb|\saveboxresource| rather than \verb|\pdfxform| (keep old name as alias).
%   \item Grab the number via Lua so that a \verb|\immediate| prefix still works with
%         \verb|\saveboxresource|\slash \verb|\pdfxform|.
%   \item Added handler for the color feature of luaotfload      
%   \end{Version}
%   \begin{Version}{2020-02-24 v1.15}
%   \item Grab all possible arguments for \verb|\saveboxresource|\slash \verb|\pdfxform|
%   \end{Version}
%   \begin{Version}{2020-04-04 v1.16}
%   \item Reset color after \verb|pdf_colorstack| whatsits.
%   \end{Version}
%   \begin{Version}{2021-02-17 v1.17}
%   \item Use \LaTeXe's new \verb|pre_shipout_filter| callback if it's available to
%         allow coloring background and foreground layer material
%   \end{Version}
%   \begin{Version}{2023-08-18 v1.18}
%   \item added \verb|\reset@color| to \verb|\set@color| for \verb|\mathcolor|, issue 7.
%   \end{Version}
% \end{History}
%
% \PrintIndex
%
% \Finale
\endinput

%        (quote the arguments according to the demands of your shell)
%
% Documentation:
%    (a) If luacolor.drv is present:
%           latex luacolor.drv
%    (b) Without luacolor.drv:
%           latex luacolor.dtx; ...
%    The class ltxdoc loads the configuration file ltxdoc.cfg
%    if available. Here you can specify further options, e.g.
%    use A4 as paper format:
%       \PassOptionsToClass{a4paper}{article}
%
%    Programm calls to get the documentation (example):
%       pdflatex luacolor.dtx
%       makeindex -s gind.ist luacolor.idx
%       pdflatex luacolor.dtx
%       makeindex -s gind.ist luacolor.idx
%       pdflatex luacolor.dtx
%
% Installation:
%    TDS:tex/latex/luacolor/luacolor.sty
%    TDS:scripts/luacolor/luacolor.lua
%    TDS:doc/latex/luacolor/luacolor.pdf
%    TDS:source/latex/luacolor/luacolor.dtx
%
%<*ignore>
\begingroup
  \catcode123=1 %
  \catcode125=2 %
  \def\x{LaTeX2e}%
\expandafter\endgroup
\ifcase 0\ifx\install y1\fi\expandafter
         \ifx\csname processbatchFile\endcsname\relax\else1\fi
         \ifx\fmtname\x\else 1\fi\relax
\else\csname fi\endcsname
%</ignore>
%<*install>
\input docstrip.tex
\Msg{************************************************************************}
\Msg{* Installation}
\Msg{* Package: luacolor 2023-08-18 v1.18 Color support via LuaTeX's attributes (HO)}
\Msg{************************************************************************}

\keepsilent
\askforoverwritefalse

\let\MetaPrefix\relax
\preamble

This is a generated file.

Project: luacolor
Version: 2023-08-18 v1.18

Copyright (C)
   2007, 2009-2011 Heiko Oberdiek
   2016-2023 Oberdiek Package Support Group

This work may be distributed and/or modified under the
conditions of the LaTeX Project Public License, either
version 1.3c of this license or (at your option) any later
version. This version of this license is in
   https://www.latex-project.org/lppl/lppl-1-3c.txt
and the latest version of this license is in
   https://www.latex-project.org/lppl.txt
and version 1.3 or later is part of all distributions of
LaTeX version 2005/12/01 or later.

This work has the LPPL maintenance status "maintained".

The Current Maintainers of this work are
Heiko Oberdiek and the Oberdiek Package Support Group
https://github.com/ho-tex/luacolor/issues


This work consists of the main source file luacolor.dtx
and the derived files
   luacolor.sty, luacolor.pdf, luacolor.ins, luacolor.drv,
   luacolor.lua,

\endpreamble
\let\MetaPrefix\DoubleperCent

\generate{%
  \file{luacolor.ins}{\from{luacolor.dtx}{install}}%
  \file{luacolor.drv}{\from{luacolor.dtx}{driver}}%
  \usedir{tex/latex/luacolor}%
  \file{luacolor.sty}{\from{luacolor.dtx}{package}}%
}
\def\MetaPrefix{-- }
\def\defaultpostamble{%
  \MetaPrefix^^J%
  \MetaPrefix\space End of File `\outFileName'.%
}
\def\currentpostamble{\defaultpostamble}%
\generate{%
  \usedir{scripts/luacolor}%
%  \file{oberdiek.luacolor.lua}{\from{luacolor.dtx}{lua}}%
  \file{luacolor.lua}{\from{luacolor.dtx}{lua}}%
}

\catcode32=13\relax% active space
\let =\space%
\Msg{************************************************************************}
\Msg{*}
\Msg{* To finish the installation you have to move the following}
\Msg{* file into a directory searched by TeX:}
\Msg{*}
\Msg{*     luacolor.sty}
\Msg{*}
\Msg{* And install the following script file:}
\Msg{*}
\Msg{*     luacolor.lua,}
\Msg{*}
\Msg{* To produce the documentation run the file `luacolor.dtx'}
\Msg{* through LaTeX.}
\Msg{*}
\Msg{* Happy TeXing!}
\Msg{*}
\Msg{************************************************************************}

\endbatchfile
%</install>
%<*ignore>
\fi
%</ignore>
%<*driver>
\NeedsTeXFormat{LaTeX2e}
\ProvidesFile{luacolor.drv}%
  [2023-08-18 v1.18 Color support via LuaTeX's attributes (HO)]%
\documentclass{ltxdoc}
\usepackage{holtxdoc}[2011/11/22]
\usepackage{paralist}
\makeatletter
\g@addto@macro\MakePrivateLetters{%
  \@makeother\_%
}
\makeatother
\begin{document}
  \DocInput{luacolor.dtx}%
\end{document}
%</driver>
% \fi
%
%
%
% \GetFileInfo{luacolor.drv}
%
% \title{The \xpackage{luacolor} package}
% \date{2023-08-18 v1.18}
% \author{Heiko Oberdiek\thanks
% {Please report any issues at \url{https://github.com/ho-tex/luacolor/issues}}}
%
% \maketitle
%
% \begin{abstract}
% Package \xpackage{luacolor} implements color support based
% on \LuaTeX's node attributes.
% \end{abstract}
%
% \tableofcontents
%
% \section{Documentation}
%
% \subsection{Introduction}
%
% This package uses a \LuaTeX's attribute register to
% to annotate nodes with color information.
% If a color is set, then the attribute register
% is set to this color and all nodes created in its scope
% (current group) are annotated with this attribute.
% Now the color property behaves much the same way
% as the font property.
%
% \subsection{Usage}
%
% Package \xpackage{color} is loaded automatically by this
% package \xpackage{luacolor}. If you need a special driver
% option or you prefer package \xpackage{xcolor}, then load
% it before package \xpackage{luacolor}, for example:
% \begin{quote}
%   |\usepackage[dvipdfmx]{xcolor}|
% \end{quote}
% The package \xpackage{luacolor} is loaded without options:
% \begin{quote}
%   |\usepackage{luacolor}|
% \end{quote}
% It is able to detect PDF mode and DVI drivers are
% differentiated by its color specials. Therefore the
% package do need driver options.
%
% Then it redefines the color setting commands to set
% attributes instead of whatsits for color.
%
% At last the attribute annotations of the nodes in the
% output box must be analyzed to insert the necessary color whatsits.
% (With older \LuaTeX\ that lack the appropriate callback function 
% the package \xpackage{atbegshi} is used to get control
% before a box is shipped out.)
%
% \begin{declcs}{luacolorProcessBox} \M{box}
% \end{declcs}
% Macro \cs{luacolorProcessBox} processes the box \meta{box}
% in the previously described manner. It is automatically
% called for pages, but not for XForm objects.
% Before passing a box to \cs{pdfxform}, call \cs{luacolorProcessBox}
% first.
%
% \subsection{Limitations}
%
% \begin{description}
% \item[Ligatures with different colored components:]
%   Package \xpackage{luacolor} sees the ligature after
%   the paragraph building and page breaking, when a
%   page is to be shipped out. Therefore it cannot
%   break ligatures, because the components might occupy
%   different space. Therefore it is the responsibility
%   of the ligature forming process to deal with different
%   colored glyphs that form a ligature. The user can avoid
%   the problem entirely by explicitly breaking the ligature
%   at the places where the color changes.
%  \item \dots
% \end{description}
%
% \StopEventually{
% }
%
% \section{Implementation}
%
%    \begin{macrocode}
%<*package>
%    \end{macrocode}
%
% \subsection{Catcodes and identification}
%
%    \begin{macrocode}
\begingroup\catcode61\catcode48\catcode32=10\relax%
  \catcode13=5 % ^^M
  \endlinechar=13 %
  \catcode123=1 % {
  \catcode125=2 % }
  \catcode64=11 % @
  \def\x{\endgroup
    \expandafter\edef\csname LuaCol@AtEnd\endcsname{%
      \endlinechar=\the\endlinechar\relax
      \catcode13=\the\catcode13\relax
      \catcode32=\the\catcode32\relax
      \catcode35=\the\catcode35\relax
      \catcode61=\the\catcode61\relax
      \catcode64=\the\catcode64\relax
      \catcode123=\the\catcode123\relax
      \catcode125=\the\catcode125\relax
    }%
  }%
\x\catcode61\catcode48\catcode32=10\relax%
\catcode13=5 % ^^M
\endlinechar=13 %
\catcode35=6 % #
\catcode64=11 % @
\catcode123=1 % {
\catcode125=2 % }
\def\TMP@EnsureCode#1#2{%
  \edef\LuaCol@AtEnd{%
    \LuaCol@AtEnd
    \catcode#1=\the\catcode#1\relax
  }%
  \catcode#1=#2\relax
}
\TMP@EnsureCode{34}{12}% "
\TMP@EnsureCode{39}{12}% '
\TMP@EnsureCode{40}{12}% (
\TMP@EnsureCode{41}{12}% )
\TMP@EnsureCode{42}{12}% *
\TMP@EnsureCode{43}{12}% +
\TMP@EnsureCode{44}{12}% ,
\TMP@EnsureCode{45}{12}% -
\TMP@EnsureCode{46}{12}% .
\TMP@EnsureCode{47}{12}% /
\TMP@EnsureCode{58}{12}% :
\TMP@EnsureCode{60}{12}% <
\TMP@EnsureCode{62}{12}% >
\TMP@EnsureCode{91}{12}% [
\TMP@EnsureCode{93}{12}% ]
\TMP@EnsureCode{95}{12}% _ (other!)
\TMP@EnsureCode{96}{12}% `
\edef\LuaCol@AtEnd{\LuaCol@AtEnd\noexpand\endinput}
%    \end{macrocode}
%
%    Package identification.
%    \begin{macrocode}
\NeedsTeXFormat{LaTeX2e}
\ProvidesPackage{luacolor}%
  [2023-08-18 v1.18 Color support via LuaTeX's attributes (HO)]
%    \end{macrocode}
%
% \subsection{Check for \LuaTeX}
%
%    Without \LuaTeX\ there is no point in using this package.
%    \begin{macrocode}
\RequirePackage{color}
%    \end{macrocode}
%
%    \begin{macrocode}
\ifx\directlua\@undefined
  \PackageError{luacolor}{%
    This package may only be run using LuaTeX%
  }\@ehc
  \expandafter\LuaCol@AtEnd
\fi%
%    \end{macrocode}
%
% \subsection{Check for disabled colors}
%
%    \begin{macrocode}
\ifcolors@
\else
  \PackageWarningNoLine{luacolor}{%
    Colors are disabled by option `monochrome'%
  }%
  \def\set@color{}%
  \def\reset@color{}%
  \def\set@page@color{}%
  \def\define@color#1#2{}%
  \expandafter\LuaCol@AtEnd
\fi%
%    \end{macrocode}
%
% \subsection{Load module and check version}
%
%    \begin{macrocode}
\directlua{%
  require("luacolor")%
}
%    \end{macrocode}
%    \begin{macrocode}
\begingroup
  \edef\x{\directlua{tex.write("2023-08-18 v1.18")}}%
  \edef\y{%
    \directlua{%
      if oberdiek.luacolor.getversion then %
        oberdiek.luacolor.getversion()%
      end%
    }%
  }%
  \ifx\x\y
  \else
    \PackageError{luacolor}{%
      Wrong version of lua module.\MessageBreak
      Package version: \x\MessageBreak
      Lua module: \y
    }\@ehc
  \fi
\endgroup
%    \end{macrocode}
%
% \subsection{Find driver}
%
%    \begin{macrocode}
\ifnum\outputmode=\@ne
\else
  \begingroup
    \def\current@color{}%
    \def\reset@color{}%
    \setbox\z@=\hbox{%
      \begingroup
        \set@color
      \endgroup
    }%
    \edef\reserved@a{%
      \directlua{%
        oberdiek.luacolor.dvidetect()%
      }%
    }%
    \ifx\reserved@a\@empty
      \PackageError{luacolor}{%
        DVI driver detection failed because of\MessageBreak
        unrecognized color \string\special
      }\@ehc
      \endgroup
      \expandafter\expandafter\expandafter\LuaCol@AtEnd
    \else
      \PackageInfo{luacolor}{%
        Type of color \string\special: \reserved@a
      \@gobble}%
    \fi%
  \endgroup
\fi
%    \end{macrocode}
%
% \subsection{Attribute setting}
%
%    \begin{macro}{\LuaCol@Attribute}
%    \begin{macrocode}
\newattribute\LuaCol@Attribute
\let\LuaCol@setattribute\setattribute
\directlua{%
  oberdiek.luacolor.setattribute(\number\allocationnumber)%
}
%    \end{macrocode}
%    \end{macro}
%
%    \begin{macro}{\set@color}
% change 2023-08-18: added \cs{reset@color} so that \cs{mathcolor} can gobble it,
% issue 7.   
%    \begin{macrocode}
\protected\def\set@color{%
  \LuaCol@setattribute\LuaCol@Attribute{%
    \directlua{%
      oberdiek.luacolor.get("\luaescapestring{\current@color}")%
    }%
  }%
 \aftergroup\reset@color 
}
%    \end{macrocode}
%    \end{macro}
%    \begin{macro}{\reset@color}
%    \begin{macrocode}
\def\reset@color{}
%    \end{macrocode}
%    \end{macro}
%
% \subsection{Whatsit insertion}
%
%    \begin{macro}{\luacolorProcessBox}
%    \begin{macrocode}
\def\luacolorProcessBox#1{%
  \directlua{%
    oberdiek.luacolor.process(\number#1)%
  }%
}
%    \end{macrocode}
%    \end{macro}
%
%    \begin{macrocode}
\directlua{%
  if luatexbase.callbacktypes.pre_shipout_filter then
    token.get_next()
  end
}\@secondoftwo\@gobble{
  \RequirePackage{atbegshi}[2011/01/30]
  \AtBeginShipout{%
    \luacolorProcessBox\AtBeginShipoutBox
  }
}
%    \end{macrocode}
%
%    Set default color.
%    \begin{macrocode}
\set@color
%    \end{macrocode}
%
% \subsection{\cs{pdfxform}/\cs{saveboxresource} support}
%
%    \begin{macrocode}
\ifnum\outputmode=\@ne
    \let\LuaCol@org@pdfxform\saveboxresource
%    \end{macrocode}
% First we need some helpers to allow expandable code to parse keyword style arguments:
%    \begin{macrocode}
    \def\LuaCol@iii@i@ii#1#2#3{#3{#1}{#2}}
    \def\LuaCol@ii@i#1#2{{#2#1}}
    \def\LuaCol@if@keyword#1#2#3{%
      \expanded{\unexpanded{\LuaCol@iii@i@ii{#2}{#3}}\expandafter}%
      \directlua{%
        token.put_next(token.create(token.scan_keyword(token.scan_string())
        and '@firstoftwo'
        or '@secondoftwo'))
      }{#1}%
    }
%    \end{macrocode}
% The following macro scans a integer and expands to a token equivalent to a chardef
% whose value corresponds to the scanned integer. This allows the integer to be passed
% around as a undelimited argument.
%    \begin{macrocode}
    \def\LuaCol@scan@number{%
      \directlua{
        token.put_next(token.new(token.scan_int(), token.command_id'char_given'))
      }%
    }
%    \end{macrocode}
% \TeX\ primitives like \cs{saveboxresource} read braced arguments in a special way.
% Especially they expand everything until they find a left brace. To simulate this, we
% use Lua to expand everything else:
%    \begin{macrocode}
    \def\LuaCol@scan@tobrace{%
      \directlua{
        local relax, space = token.command_id'relax', token.command_id'spacer'
        local t
        repeat
          t = token.scan_token()
        until not (t.command == relax or t.command == space)
        token.put_next(t)
      }%
    }
    \def\LuaCol@scan@boxresource@i#1#2{%
      \LuaCol@if@keyword{attr}{%
        \expanded{\unexpanded{\LuaCol@scan@boxresource@iI{#1#2attr}}%
          \expandafter\expandafter\expandafter}%
        \LuaCol@scan@tobrace
      }{%
        \LuaCol@scan@boxresource@ii{#1#2}%
      }%
    }
    \def\LuaCol@scan@boxresource@iI#1#2{\LuaCol@scan@boxresource@ii{#1{#2}}}
    \def\LuaCol@scan@boxresource@ii#1{%
      \LuaCol@if@keyword{resources}{%
        \expanded{\unexpanded{\LuaCol@scan@boxresource@iiI{#1resources}}%
          \expandafter\expandafter\expandafter}%
        \LuaCol@scan@tobrace
      }{%
        \LuaCol@scan@boxresource@iii{#1}%
      }%
    }
    \def\LuaCol@scan@boxresource@iiI#1#2{\LuaCol@scan@boxresource@iii{#1{#2}}}
    \def\LuaCol@scan@boxresource@iii#1{%
      \LuaCol@if@keyword{margin}{%
        \expanded{\unexpanded{\LuaCol@scan@boxresource@iv{#1margin }}%
          \expandafter\expandafter\expandafter}%
        \LuaCol@scan@number
      }{%
        \LuaCol@scan@boxresource@iv{#1}{}%
      }%
    }
    \def\LuaCol@scan@boxresource@iv#1#2{%
      \expanded{\unexpanded{\LuaCol@scan@boxresource@v{#1#2}}%
        \expandafter\expandafter\expandafter}%
      \LuaCol@scan@number
    }
    \def\LuaCol@scan@boxresource@v#1#2{%
      \luacolorProcessBox{#2}%
      \LuaCol@org@pdfxform#1#2%
    }

%    \end{macrocode}
% This could be written in Lua, but at least upto Lua\TeX~1.11, feeding back too many tokens
% from Lua to \TeX\ triggers a segmentation fault.
% This is written in Lua so the integer setting is expandable and does not interfere with
% a preceding \verb|\immediate|.
%    \begin{macrocode}
    \protected\def\saveboxresource{%
      \LuaCol@if@keyword{type}{%
        \expandafter
        \expanded{\unexpanded{\LuaCol@scan@boxresource@i{type }}%
          \expandafter\expandafter\expandafter}%
        \LuaCol@scan@number
      }{%
        \LuaCol@scan@boxresource@i{}{}%
      }%
    }
%    \end{macrocode}
%    Legacy alias.
%    \begin{macrocode}
    \let\pdfxform\saveboxresource
\fi
%    \end{macrocode}
%
%    \begin{macrocode}
\LuaCol@AtEnd%
%</package>
%    \end{macrocode}
%
% \subsection{Lua module}
%
%    \begin{macrocode}
%<*lua>
%    \end{macrocode}
%    Box zero contains a \cs{hbox} with the color \cs{special}.
%    That is analyzed to get the prefix for the color setting
%    \cs{special}.
%    \begin{macrocode}
oberdiek = oberdiek or {}
local luacolor = oberdiek.luacolor or {}
oberdiek.luacolor = luacolor
%    \end{macrocode}
%    \begin{macro}{getversion()}
%    \begin{macrocode}
function luacolor.getversion()
  tex.write("2023-08-18 v1.18")
end
%    \end{macrocode}
%    \end{macro}
%
% \subsubsection{Driver detection}
%
%    \begin{macrocode}
local ifpdf = tonumber(tex.outputmode or tex.pdfoutput) > 0
local prefix
local prefixes = {
  dvips   = "color ",
  dvipdfm = "pdf:sc ",
  truetex = "textcolor:",
  pctexps = "ps::",
}
local patterns = {
  ["^color "]            = "dvips",
  ["^pdf: *begincolor "] = "dvipdfm",
  ["^pdf: *bcolor "]     = "dvipdfm",
  ["^pdf: *bc "]         = "dvipdfm",
  ["^pdf: *setcolor "]   = "dvipdfm",
  ["^pdf: *scolor "]     = "dvipdfm",
  ["^pdf: *sc "]         = "dvipdfm",
  ["^textcolor:"]        = "truetex",
  ["^ps::"]              = "pctexps",
}
%    \end{macrocode}
%    \begin{macro}{info()}
%    \begin{macrocode}
local function info(msg, term)
  local target = "log"
  if term then
    target = "term and log"
  end
  texio.write_nl(target, "Package luacolor info: " .. msg .. ".")
  texio.write_nl(target, "")
end
%    \end{macrocode}
%    \end{macro}
%    \begin{macro}{dvidetect()}
%    \begin{macrocode}
function luacolor.dvidetect()
  local v = tex.box[0]
  assert(v.id == node.id("hlist"))
  for v in node.traverse_id(node.id("whatsit"), v.head) do
    if v and v.subtype == node.subtype("special") then
      local data = v.data
      for pattern, driver in pairs(patterns) do
        if string.find(data, pattern) then
          prefix = prefixes[driver]
          tex.write(driver)
          return
        end
      end
      info("\\special{" .. data .. "}", true)
      return
    end
  end
  info("Missing \\special", true)
end
%    \end{macrocode}
%    \end{macro}
%
% \subsubsection{Color strings}
%
%    \begin{macrocode}
local map = {
  n = 0,
}
%    \end{macrocode}
%    \begin{macro}{get()}
%    \begin{macrocode}
function luacolor.get(color)
  tex.write("" .. luacolor.getvalue(color))
end
%    \end{macrocode}
%    \end{macro}
%    \begin{macro}{getvalue()}
%    \begin{macrocode}
function luacolor.getvalue(color)
  local n = map[color]
  if not n then
    n = map.n + 1
    map.n = n
    map[n] = color
    map[color] = n
  end
  return n
end
%    \end{macrocode}
%    \end{macro}
%
% \subsubsection{Attribute register}
%
%    \begin{macro}{setattribute()}
%    \begin{macrocode}
local attribute
function luacolor.setattribute(attr)
  attribute = attr
end
%    \end{macrocode}
%    \end{macro}
%
%    \begin{macro}{getattribute()}
%    \begin{macrocode}
function luacolor.getattribute()
  return attribute
end
%    \end{macrocode}
%    \end{macro}
%
% \subsubsection{Whatsit insertion}
%
%    \begin{macrocode}
local LIST = 1
local LIST_LEADERS = 2
local LIST_DISC = 3
local COLOR = 4
local NOCOLOR = 5
local RULE = node.id("rule")
local node_types = {
  [node.id("hlist")] = LIST,
  [node.id("vlist")] = LIST,
  [node.id("rule")]  = COLOR,
  [node.id("glyph")] = COLOR,
  [node.id("disc")]  = LIST_DISC,
  [node.id("whatsit")] = {
    [node.subtype("pdf_colorstack")] =
      function(n)
        return n.stack == 0 and NOCOLOR or nil
      end,
    [node.subtype("special")] = COLOR,
    [node.subtype("pdf_literal")] = COLOR,
    [node.subtype("pdf_save")] = COLOR,
    [node.subtype("pdf_restore")] = COLOR, -- probably not needed
-- TODO (DPC)    [node.subtype("pdf_refximage")] = COLOR,
  },
  [node.id("glue")] =
    function(n)
      if n.subtype >= 100 then -- leaders
        if n.leader.id == RULE then
          return COLOR
        else
          return LIST_LEADERS
        end
      end
    end,
}
%    \end{macrocode}
%    \begin{macro}{get_type()}
%    \begin{macrocode}
local function get_type(n)
  local ret = node_types[n.id]
  if type(ret) == 'table' then
    ret = ret[n.subtype]
  end
  if type(ret) == 'function' then
    ret = ret(n)
  end
  return ret
end
%    \end{macrocode}
%    \end{macro}
%    \begin{macrocode}
local mode = 2 -- luatex.pdfliteral.direct
local WHATSIT = node.id("whatsit")
local SPECIAL = node.subtype("special")
local PDFLITERAL = node.subtype("pdf_literal")
local DRY_FALSE = false
local DRY_TRUE = true
%    \end{macrocode}
%    \begin{macro}{traverse()}
%    \begin{macrocode}
local function traverse(list, color, dry)
  if not list then
    return color
  end
  local head
  if get_type(list) == LIST then
    head = list.head
  elseif get_type(list) == LIST_DISC then
    head = list.replace
  else
    texio.write_nl("!!! Error: Wrong list type: " .. node.type(list.id))
    return color
  end
%<debug>texio.write_nl("traverse: " .. node.type(list.id))
  for n in node.traverse(head) do
%<debug>texio.write_nl("  node: " .. node.type(n.id))
    local t = get_type(n)
%<debug>texio.write_nl("TYPE "..tostring(t).. " "..tostring(node.type(node.getid(n))).." ".. tostring(node.getsubtype(n)))
    if t == LIST or t == LIST_DISC then
      color = traverse(n, color, dry)
    elseif t == LIST_LEADERS then
      local color_after = traverse(n.leader, color, DRY_TRUE)
      if color == color_after then
        traverse(n.leader, color, DRY_FALSE or dry)
      else
        traverse(n.leader, '', DRY_FALSE or dry)
%    \end{macrocode}
% The color status is unknown here, because the leader box
% will or will not be set.
%    \begin{macrocode}
        color = ''
      end
    elseif t == COLOR then
      local v = node.has_attribute(n, attribute)
      if v then
        local newColor = map[v]
        if newColor ~= color then
          color = newColor
          if dry == DRY_FALSE then
            local newNode
            if ifpdf then
              newNode = node.new(WHATSIT, PDFLITERAL)
              newNode.mode = mode
              newNode.data = color
            else
              newNode = node.new(WHATSIT, SPECIAL)
              newNode.data = prefix .. color
            end
            head = node.insert_before(head, n, newNode)
          end
        end
      end
    elseif t == NOCOLOR then
      color = ''
    end
  end
  if get_type(list) == LIST then
    list.head = head
  else
    list.replace = head
  end
  return color
end
%    \end{macrocode}
%    \end{macro}
%
%    \begin{macro}{process()}
%    \begin{macrocode}
function luacolor.process(box)
  local color = ""
  local list = tex.getbox(box)
  traverse(list, color, DRY_FALSE)
end

if luatexbase.callbacktypes.pre_shipout_filter then
  luatexbase.add_to_callback("pre_shipout_filter", function(list)
    traverse(list, "", DRY_FALSE)
    return true
  end, "luacolor.process")
end
%    \end{macrocode}
%    \end{macro}
% For recent versions of luaotfload, we can register a callback to
% control how coloring glyph is handled for the color feature.
%    \begin{macrocode}
if luaotfload.set_colorhandler then
  local set_attribute = node.direct.set_attribute
  luaotfload.set_colorhandler(function(head, n, color)
    set_attribute(n, attribute, luacolor.getvalue(color))
    return head, n
  end)
end
%    \end{macrocode}
%
%    \begin{macrocode}
%</lua>
%    \end{macrocode}
% \section{Installation}
%
% \subsection{Download}
%
% \paragraph{Package.} This package is available on
% CTAN\footnote{\CTANpkg{luacolor}}:
% \begin{description}
% \item[\CTAN{macros/latex/contrib/luacolor/luacolor.dtx}] The source file.
% \item[\CTAN{macros/latex/contrib/luacolor/luacolor.pdf}] Documentation.
% \end{description}
%
%
% \paragraph{Bundle.} All the packages of the bundle `luacolor'
% are also available in a TDS compliant ZIP archive. There
% the packages are already unpacked and the documentation files
% are generated. The files and directories obey the TDS standard.
% \begin{description}
% \item[\CTANinstall{install/macros/latex/contrib/luacolor.tds.zip}]
% \end{description}
% \emph{TDS} refers to the standard ``A Directory Structure
% for \TeX\ Files'' (\CTANpkg{tds}). Directories
% with \xfile{texmf} in their name are usually organized this way.
%
% \subsection{Bundle installation}
%
% \paragraph{Unpacking.} Unpack the \xfile{luacolor.tds.zip} in the
% TDS tree (also known as \xfile{texmf} tree) of your choice.
% Example (linux):
% \begin{quote}
%   |unzip luacolor.tds.zip -d ~/texmf|
% \end{quote}
%
% \paragraph{Script installation.}
% Check the directory \xfile{TDS:scripts/luacolor/} for
% scripts that need further installation steps.
%
% \subsection{Package installation}
%
% \paragraph{Unpacking.} The \xfile{.dtx} file is a self-extracting
% \docstrip\ archive. The files are extracted by running the
% \xfile{.dtx} through \plainTeX:
% \begin{quote}
%   \verb|tex luacolor.dtx|
% \end{quote}
%
% \paragraph{TDS.} Now the different files must be moved into
% the different directories in your installation TDS tree
% (also known as \xfile{texmf} tree):
% \begin{quote}
% \def\t{^^A
% \begin{tabular}{@{}>{\ttfamily}l@{ $\rightarrow$ }>{\ttfamily}l@{}}
%   luacolor.sty & tex/latex/luacolor/luacolor.sty\\
%   luacolor.lua & scripts/luacolor/luacolor.lua\\
%   luacolor.pdf & doc/latex/luacolor/luacolor.pdf\\
%   luacolor.dtx & source/latex/luacolor/luacolor.dtx\\
% \end{tabular}^^A
% }^^A
% \sbox0{\t}^^A
% \ifdim\wd0>\linewidth
%   \begingroup
%     \advance\linewidth by\leftmargin
%     \advance\linewidth by\rightmargin
%   \edef\x{\endgroup
%     \def\noexpand\lw{\the\linewidth}^^A
%   }\x
%   \def\lwbox{^^A
%     \leavevmode
%     \hbox to \linewidth{^^A
%       \kern-\leftmargin\relax
%       \hss
%       \usebox0
%       \hss
%       \kern-\rightmargin\relax
%     }^^A
%   }^^A
%   \ifdim\wd0>\lw
%     \sbox0{\small\t}^^A
%     \ifdim\wd0>\linewidth
%       \ifdim\wd0>\lw
%         \sbox0{\footnotesize\t}^^A
%         \ifdim\wd0>\linewidth
%           \ifdim\wd0>\lw
%             \sbox0{\scriptsize\t}^^A
%             \ifdim\wd0>\linewidth
%               \ifdim\wd0>\lw
%                 \sbox0{\tiny\t}^^A
%                 \ifdim\wd0>\linewidth
%                   \lwbox
%                 \else
%                   \usebox0
%                 \fi
%               \else
%                 \lwbox
%               \fi
%             \else
%               \usebox0
%             \fi
%           \else
%             \lwbox
%           \fi
%         \else
%           \usebox0
%         \fi
%       \else
%         \lwbox
%       \fi
%     \else
%       \usebox0
%     \fi
%   \else
%     \lwbox
%   \fi
% \else
%   \usebox0
% \fi
% \end{quote}
% If you have a \xfile{docstrip.cfg} that configures and enables \docstrip's
% TDS installing feature, then some files can already be in the right
% place, see the documentation of \docstrip.
%
% \subsection{Refresh file name databases}
%
% If your \TeX~distribution
% (\TeX\,Live, \mikTeX, \dots) relies on file name databases, you must refresh
% these. For example, \TeX\,Live\ users run \verb|texhash| or
% \verb|mktexlsr|.
%
% \subsection{Some details for the interested}
%
% \paragraph{Unpacking with \LaTeX.}
% The \xfile{.dtx} chooses its action depending on the format:
% \begin{description}
% \item[\plainTeX:] Run \docstrip\ and extract the files.
% \item[\LaTeX:] Generate the documentation.
% \end{description}
% If you insist on using \LaTeX\ for \docstrip\ (really,
% \docstrip\ does not need \LaTeX), then inform the autodetect routine
% about your intention:
% \begin{quote}
%   \verb|latex \let\install=y% \iffalse meta-comment
%
% File: luacolor.dtx
% Version: 2023-08-18 v1.18
% Info: Color support via LuaTeX's attributes
%
% Copyright (C)
%    2007, 2009-2011 Heiko Oberdiek
%    2016-2023 Oberdiek Package Support Group
%    https://github.com/ho-tex/luacolor/issues
%
% This work may be distributed and/or modified under the
% conditions of the LaTeX Project Public License, either
% version 1.3c of this license or (at your option) any later
% version. This version of this license is in
%    https://www.latex-project.org/lppl/lppl-1-3c.txt
% and the latest version of this license is in
%    https://www.latex-project.org/lppl.txt
% and version 1.3 or later is part of all distributions of
% LaTeX version 2005/12/01 or later.
%
% This work has the LPPL maintenance status "maintained".
%
% The Current Maintainers of this work are
% Heiko Oberdiek and the Oberdiek Package Support Group
% https://github.com/ho-tex/luacolor/issues
%
% This work consists of the main source file luacolor.dtx
% and the derived files
%    luacolor.sty, luacolor.pdf, luacolor.lua,
%
% Distribution:
%    CTAN:macros/latex/contrib/luacolor/luacolor.dtx
%    CTAN:macros/latex/contrib/luacolor/luacolor.pdf
%
% Unpacking:
%    (a) If luacolor.ins is present:
%           tex luacolor.ins
%    (b) Without luacolor.ins:
%           tex luacolor.dtx
%    (c) If you insist on using LaTeX
%           latex \let\install=y% \iffalse meta-comment
%
% File: luacolor.dtx
% Version: 2023-08-18 v1.18
% Info: Color support via LuaTeX's attributes
%
% Copyright (C)
%    2007, 2009-2011 Heiko Oberdiek
%    2016-2023 Oberdiek Package Support Group
%    https://github.com/ho-tex/luacolor/issues
%
% This work may be distributed and/or modified under the
% conditions of the LaTeX Project Public License, either
% version 1.3c of this license or (at your option) any later
% version. This version of this license is in
%    https://www.latex-project.org/lppl/lppl-1-3c.txt
% and the latest version of this license is in
%    https://www.latex-project.org/lppl.txt
% and version 1.3 or later is part of all distributions of
% LaTeX version 2005/12/01 or later.
%
% This work has the LPPL maintenance status "maintained".
%
% The Current Maintainers of this work are
% Heiko Oberdiek and the Oberdiek Package Support Group
% https://github.com/ho-tex/luacolor/issues
%
% This work consists of the main source file luacolor.dtx
% and the derived files
%    luacolor.sty, luacolor.pdf, luacolor.lua,
%
% Distribution:
%    CTAN:macros/latex/contrib/luacolor/luacolor.dtx
%    CTAN:macros/latex/contrib/luacolor/luacolor.pdf
%
% Unpacking:
%    (a) If luacolor.ins is present:
%           tex luacolor.ins
%    (b) Without luacolor.ins:
%           tex luacolor.dtx
%    (c) If you insist on using LaTeX
%           latex \let\install=y\input{luacolor.dtx}
%        (quote the arguments according to the demands of your shell)
%
% Documentation:
%    (a) If luacolor.drv is present:
%           latex luacolor.drv
%    (b) Without luacolor.drv:
%           latex luacolor.dtx; ...
%    The class ltxdoc loads the configuration file ltxdoc.cfg
%    if available. Here you can specify further options, e.g.
%    use A4 as paper format:
%       \PassOptionsToClass{a4paper}{article}
%
%    Programm calls to get the documentation (example):
%       pdflatex luacolor.dtx
%       makeindex -s gind.ist luacolor.idx
%       pdflatex luacolor.dtx
%       makeindex -s gind.ist luacolor.idx
%       pdflatex luacolor.dtx
%
% Installation:
%    TDS:tex/latex/luacolor/luacolor.sty
%    TDS:scripts/luacolor/luacolor.lua
%    TDS:doc/latex/luacolor/luacolor.pdf
%    TDS:source/latex/luacolor/luacolor.dtx
%
%<*ignore>
\begingroup
  \catcode123=1 %
  \catcode125=2 %
  \def\x{LaTeX2e}%
\expandafter\endgroup
\ifcase 0\ifx\install y1\fi\expandafter
         \ifx\csname processbatchFile\endcsname\relax\else1\fi
         \ifx\fmtname\x\else 1\fi\relax
\else\csname fi\endcsname
%</ignore>
%<*install>
\input docstrip.tex
\Msg{************************************************************************}
\Msg{* Installation}
\Msg{* Package: luacolor 2023-08-18 v1.18 Color support via LuaTeX's attributes (HO)}
\Msg{************************************************************************}

\keepsilent
\askforoverwritefalse

\let\MetaPrefix\relax
\preamble

This is a generated file.

Project: luacolor
Version: 2023-08-18 v1.18

Copyright (C)
   2007, 2009-2011 Heiko Oberdiek
   2016-2023 Oberdiek Package Support Group

This work may be distributed and/or modified under the
conditions of the LaTeX Project Public License, either
version 1.3c of this license or (at your option) any later
version. This version of this license is in
   https://www.latex-project.org/lppl/lppl-1-3c.txt
and the latest version of this license is in
   https://www.latex-project.org/lppl.txt
and version 1.3 or later is part of all distributions of
LaTeX version 2005/12/01 or later.

This work has the LPPL maintenance status "maintained".

The Current Maintainers of this work are
Heiko Oberdiek and the Oberdiek Package Support Group
https://github.com/ho-tex/luacolor/issues


This work consists of the main source file luacolor.dtx
and the derived files
   luacolor.sty, luacolor.pdf, luacolor.ins, luacolor.drv,
   luacolor.lua,

\endpreamble
\let\MetaPrefix\DoubleperCent

\generate{%
  \file{luacolor.ins}{\from{luacolor.dtx}{install}}%
  \file{luacolor.drv}{\from{luacolor.dtx}{driver}}%
  \usedir{tex/latex/luacolor}%
  \file{luacolor.sty}{\from{luacolor.dtx}{package}}%
}
\def\MetaPrefix{-- }
\def\defaultpostamble{%
  \MetaPrefix^^J%
  \MetaPrefix\space End of File `\outFileName'.%
}
\def\currentpostamble{\defaultpostamble}%
\generate{%
  \usedir{scripts/luacolor}%
%  \file{oberdiek.luacolor.lua}{\from{luacolor.dtx}{lua}}%
  \file{luacolor.lua}{\from{luacolor.dtx}{lua}}%
}

\catcode32=13\relax% active space
\let =\space%
\Msg{************************************************************************}
\Msg{*}
\Msg{* To finish the installation you have to move the following}
\Msg{* file into a directory searched by TeX:}
\Msg{*}
\Msg{*     luacolor.sty}
\Msg{*}
\Msg{* And install the following script file:}
\Msg{*}
\Msg{*     luacolor.lua,}
\Msg{*}
\Msg{* To produce the documentation run the file `luacolor.dtx'}
\Msg{* through LaTeX.}
\Msg{*}
\Msg{* Happy TeXing!}
\Msg{*}
\Msg{************************************************************************}

\endbatchfile
%</install>
%<*ignore>
\fi
%</ignore>
%<*driver>
\NeedsTeXFormat{LaTeX2e}
\ProvidesFile{luacolor.drv}%
  [2023-08-18 v1.18 Color support via LuaTeX's attributes (HO)]%
\documentclass{ltxdoc}
\usepackage{holtxdoc}[2011/11/22]
\usepackage{paralist}
\makeatletter
\g@addto@macro\MakePrivateLetters{%
  \@makeother\_%
}
\makeatother
\begin{document}
  \DocInput{luacolor.dtx}%
\end{document}
%</driver>
% \fi
%
%
%
% \GetFileInfo{luacolor.drv}
%
% \title{The \xpackage{luacolor} package}
% \date{2023-08-18 v1.18}
% \author{Heiko Oberdiek\thanks
% {Please report any issues at \url{https://github.com/ho-tex/luacolor/issues}}}
%
% \maketitle
%
% \begin{abstract}
% Package \xpackage{luacolor} implements color support based
% on \LuaTeX's node attributes.
% \end{abstract}
%
% \tableofcontents
%
% \section{Documentation}
%
% \subsection{Introduction}
%
% This package uses a \LuaTeX's attribute register to
% to annotate nodes with color information.
% If a color is set, then the attribute register
% is set to this color and all nodes created in its scope
% (current group) are annotated with this attribute.
% Now the color property behaves much the same way
% as the font property.
%
% \subsection{Usage}
%
% Package \xpackage{color} is loaded automatically by this
% package \xpackage{luacolor}. If you need a special driver
% option or you prefer package \xpackage{xcolor}, then load
% it before package \xpackage{luacolor}, for example:
% \begin{quote}
%   |\usepackage[dvipdfmx]{xcolor}|
% \end{quote}
% The package \xpackage{luacolor} is loaded without options:
% \begin{quote}
%   |\usepackage{luacolor}|
% \end{quote}
% It is able to detect PDF mode and DVI drivers are
% differentiated by its color specials. Therefore the
% package do need driver options.
%
% Then it redefines the color setting commands to set
% attributes instead of whatsits for color.
%
% At last the attribute annotations of the nodes in the
% output box must be analyzed to insert the necessary color whatsits.
% (With older \LuaTeX\ that lack the appropriate callback function 
% the package \xpackage{atbegshi} is used to get control
% before a box is shipped out.)
%
% \begin{declcs}{luacolorProcessBox} \M{box}
% \end{declcs}
% Macro \cs{luacolorProcessBox} processes the box \meta{box}
% in the previously described manner. It is automatically
% called for pages, but not for XForm objects.
% Before passing a box to \cs{pdfxform}, call \cs{luacolorProcessBox}
% first.
%
% \subsection{Limitations}
%
% \begin{description}
% \item[Ligatures with different colored components:]
%   Package \xpackage{luacolor} sees the ligature after
%   the paragraph building and page breaking, when a
%   page is to be shipped out. Therefore it cannot
%   break ligatures, because the components might occupy
%   different space. Therefore it is the responsibility
%   of the ligature forming process to deal with different
%   colored glyphs that form a ligature. The user can avoid
%   the problem entirely by explicitly breaking the ligature
%   at the places where the color changes.
%  \item \dots
% \end{description}
%
% \StopEventually{
% }
%
% \section{Implementation}
%
%    \begin{macrocode}
%<*package>
%    \end{macrocode}
%
% \subsection{Catcodes and identification}
%
%    \begin{macrocode}
\begingroup\catcode61\catcode48\catcode32=10\relax%
  \catcode13=5 % ^^M
  \endlinechar=13 %
  \catcode123=1 % {
  \catcode125=2 % }
  \catcode64=11 % @
  \def\x{\endgroup
    \expandafter\edef\csname LuaCol@AtEnd\endcsname{%
      \endlinechar=\the\endlinechar\relax
      \catcode13=\the\catcode13\relax
      \catcode32=\the\catcode32\relax
      \catcode35=\the\catcode35\relax
      \catcode61=\the\catcode61\relax
      \catcode64=\the\catcode64\relax
      \catcode123=\the\catcode123\relax
      \catcode125=\the\catcode125\relax
    }%
  }%
\x\catcode61\catcode48\catcode32=10\relax%
\catcode13=5 % ^^M
\endlinechar=13 %
\catcode35=6 % #
\catcode64=11 % @
\catcode123=1 % {
\catcode125=2 % }
\def\TMP@EnsureCode#1#2{%
  \edef\LuaCol@AtEnd{%
    \LuaCol@AtEnd
    \catcode#1=\the\catcode#1\relax
  }%
  \catcode#1=#2\relax
}
\TMP@EnsureCode{34}{12}% "
\TMP@EnsureCode{39}{12}% '
\TMP@EnsureCode{40}{12}% (
\TMP@EnsureCode{41}{12}% )
\TMP@EnsureCode{42}{12}% *
\TMP@EnsureCode{43}{12}% +
\TMP@EnsureCode{44}{12}% ,
\TMP@EnsureCode{45}{12}% -
\TMP@EnsureCode{46}{12}% .
\TMP@EnsureCode{47}{12}% /
\TMP@EnsureCode{58}{12}% :
\TMP@EnsureCode{60}{12}% <
\TMP@EnsureCode{62}{12}% >
\TMP@EnsureCode{91}{12}% [
\TMP@EnsureCode{93}{12}% ]
\TMP@EnsureCode{95}{12}% _ (other!)
\TMP@EnsureCode{96}{12}% `
\edef\LuaCol@AtEnd{\LuaCol@AtEnd\noexpand\endinput}
%    \end{macrocode}
%
%    Package identification.
%    \begin{macrocode}
\NeedsTeXFormat{LaTeX2e}
\ProvidesPackage{luacolor}%
  [2023-08-18 v1.18 Color support via LuaTeX's attributes (HO)]
%    \end{macrocode}
%
% \subsection{Check for \LuaTeX}
%
%    Without \LuaTeX\ there is no point in using this package.
%    \begin{macrocode}
\RequirePackage{color}
%    \end{macrocode}
%
%    \begin{macrocode}
\ifx\directlua\@undefined
  \PackageError{luacolor}{%
    This package may only be run using LuaTeX%
  }\@ehc
  \expandafter\LuaCol@AtEnd
\fi%
%    \end{macrocode}
%
% \subsection{Check for disabled colors}
%
%    \begin{macrocode}
\ifcolors@
\else
  \PackageWarningNoLine{luacolor}{%
    Colors are disabled by option `monochrome'%
  }%
  \def\set@color{}%
  \def\reset@color{}%
  \def\set@page@color{}%
  \def\define@color#1#2{}%
  \expandafter\LuaCol@AtEnd
\fi%
%    \end{macrocode}
%
% \subsection{Load module and check version}
%
%    \begin{macrocode}
\directlua{%
  require("luacolor")%
}
%    \end{macrocode}
%    \begin{macrocode}
\begingroup
  \edef\x{\directlua{tex.write("2023-08-18 v1.18")}}%
  \edef\y{%
    \directlua{%
      if oberdiek.luacolor.getversion then %
        oberdiek.luacolor.getversion()%
      end%
    }%
  }%
  \ifx\x\y
  \else
    \PackageError{luacolor}{%
      Wrong version of lua module.\MessageBreak
      Package version: \x\MessageBreak
      Lua module: \y
    }\@ehc
  \fi
\endgroup
%    \end{macrocode}
%
% \subsection{Find driver}
%
%    \begin{macrocode}
\ifnum\outputmode=\@ne
\else
  \begingroup
    \def\current@color{}%
    \def\reset@color{}%
    \setbox\z@=\hbox{%
      \begingroup
        \set@color
      \endgroup
    }%
    \edef\reserved@a{%
      \directlua{%
        oberdiek.luacolor.dvidetect()%
      }%
    }%
    \ifx\reserved@a\@empty
      \PackageError{luacolor}{%
        DVI driver detection failed because of\MessageBreak
        unrecognized color \string\special
      }\@ehc
      \endgroup
      \expandafter\expandafter\expandafter\LuaCol@AtEnd
    \else
      \PackageInfo{luacolor}{%
        Type of color \string\special: \reserved@a
      \@gobble}%
    \fi%
  \endgroup
\fi
%    \end{macrocode}
%
% \subsection{Attribute setting}
%
%    \begin{macro}{\LuaCol@Attribute}
%    \begin{macrocode}
\newattribute\LuaCol@Attribute
\let\LuaCol@setattribute\setattribute
\directlua{%
  oberdiek.luacolor.setattribute(\number\allocationnumber)%
}
%    \end{macrocode}
%    \end{macro}
%
%    \begin{macro}{\set@color}
% change 2023-08-18: added \cs{reset@color} so that \cs{mathcolor} can gobble it,
% issue 7.   
%    \begin{macrocode}
\protected\def\set@color{%
  \LuaCol@setattribute\LuaCol@Attribute{%
    \directlua{%
      oberdiek.luacolor.get("\luaescapestring{\current@color}")%
    }%
  }%
 \aftergroup\reset@color 
}
%    \end{macrocode}
%    \end{macro}
%    \begin{macro}{\reset@color}
%    \begin{macrocode}
\def\reset@color{}
%    \end{macrocode}
%    \end{macro}
%
% \subsection{Whatsit insertion}
%
%    \begin{macro}{\luacolorProcessBox}
%    \begin{macrocode}
\def\luacolorProcessBox#1{%
  \directlua{%
    oberdiek.luacolor.process(\number#1)%
  }%
}
%    \end{macrocode}
%    \end{macro}
%
%    \begin{macrocode}
\directlua{%
  if luatexbase.callbacktypes.pre_shipout_filter then
    token.get_next()
  end
}\@secondoftwo\@gobble{
  \RequirePackage{atbegshi}[2011/01/30]
  \AtBeginShipout{%
    \luacolorProcessBox\AtBeginShipoutBox
  }
}
%    \end{macrocode}
%
%    Set default color.
%    \begin{macrocode}
\set@color
%    \end{macrocode}
%
% \subsection{\cs{pdfxform}/\cs{saveboxresource} support}
%
%    \begin{macrocode}
\ifnum\outputmode=\@ne
    \let\LuaCol@org@pdfxform\saveboxresource
%    \end{macrocode}
% First we need some helpers to allow expandable code to parse keyword style arguments:
%    \begin{macrocode}
    \def\LuaCol@iii@i@ii#1#2#3{#3{#1}{#2}}
    \def\LuaCol@ii@i#1#2{{#2#1}}
    \def\LuaCol@if@keyword#1#2#3{%
      \expanded{\unexpanded{\LuaCol@iii@i@ii{#2}{#3}}\expandafter}%
      \directlua{%
        token.put_next(token.create(token.scan_keyword(token.scan_string())
        and '@firstoftwo'
        or '@secondoftwo'))
      }{#1}%
    }
%    \end{macrocode}
% The following macro scans a integer and expands to a token equivalent to a chardef
% whose value corresponds to the scanned integer. This allows the integer to be passed
% around as a undelimited argument.
%    \begin{macrocode}
    \def\LuaCol@scan@number{%
      \directlua{
        token.put_next(token.new(token.scan_int(), token.command_id'char_given'))
      }%
    }
%    \end{macrocode}
% \TeX\ primitives like \cs{saveboxresource} read braced arguments in a special way.
% Especially they expand everything until they find a left brace. To simulate this, we
% use Lua to expand everything else:
%    \begin{macrocode}
    \def\LuaCol@scan@tobrace{%
      \directlua{
        local relax, space = token.command_id'relax', token.command_id'spacer'
        local t
        repeat
          t = token.scan_token()
        until not (t.command == relax or t.command == space)
        token.put_next(t)
      }%
    }
    \def\LuaCol@scan@boxresource@i#1#2{%
      \LuaCol@if@keyword{attr}{%
        \expanded{\unexpanded{\LuaCol@scan@boxresource@iI{#1#2attr}}%
          \expandafter\expandafter\expandafter}%
        \LuaCol@scan@tobrace
      }{%
        \LuaCol@scan@boxresource@ii{#1#2}%
      }%
    }
    \def\LuaCol@scan@boxresource@iI#1#2{\LuaCol@scan@boxresource@ii{#1{#2}}}
    \def\LuaCol@scan@boxresource@ii#1{%
      \LuaCol@if@keyword{resources}{%
        \expanded{\unexpanded{\LuaCol@scan@boxresource@iiI{#1resources}}%
          \expandafter\expandafter\expandafter}%
        \LuaCol@scan@tobrace
      }{%
        \LuaCol@scan@boxresource@iii{#1}%
      }%
    }
    \def\LuaCol@scan@boxresource@iiI#1#2{\LuaCol@scan@boxresource@iii{#1{#2}}}
    \def\LuaCol@scan@boxresource@iii#1{%
      \LuaCol@if@keyword{margin}{%
        \expanded{\unexpanded{\LuaCol@scan@boxresource@iv{#1margin }}%
          \expandafter\expandafter\expandafter}%
        \LuaCol@scan@number
      }{%
        \LuaCol@scan@boxresource@iv{#1}{}%
      }%
    }
    \def\LuaCol@scan@boxresource@iv#1#2{%
      \expanded{\unexpanded{\LuaCol@scan@boxresource@v{#1#2}}%
        \expandafter\expandafter\expandafter}%
      \LuaCol@scan@number
    }
    \def\LuaCol@scan@boxresource@v#1#2{%
      \luacolorProcessBox{#2}%
      \LuaCol@org@pdfxform#1#2%
    }

%    \end{macrocode}
% This could be written in Lua, but at least upto Lua\TeX~1.11, feeding back too many tokens
% from Lua to \TeX\ triggers a segmentation fault.
% This is written in Lua so the integer setting is expandable and does not interfere with
% a preceding \verb|\immediate|.
%    \begin{macrocode}
    \protected\def\saveboxresource{%
      \LuaCol@if@keyword{type}{%
        \expandafter
        \expanded{\unexpanded{\LuaCol@scan@boxresource@i{type }}%
          \expandafter\expandafter\expandafter}%
        \LuaCol@scan@number
      }{%
        \LuaCol@scan@boxresource@i{}{}%
      }%
    }
%    \end{macrocode}
%    Legacy alias.
%    \begin{macrocode}
    \let\pdfxform\saveboxresource
\fi
%    \end{macrocode}
%
%    \begin{macrocode}
\LuaCol@AtEnd%
%</package>
%    \end{macrocode}
%
% \subsection{Lua module}
%
%    \begin{macrocode}
%<*lua>
%    \end{macrocode}
%    Box zero contains a \cs{hbox} with the color \cs{special}.
%    That is analyzed to get the prefix for the color setting
%    \cs{special}.
%    \begin{macrocode}
oberdiek = oberdiek or {}
local luacolor = oberdiek.luacolor or {}
oberdiek.luacolor = luacolor
%    \end{macrocode}
%    \begin{macro}{getversion()}
%    \begin{macrocode}
function luacolor.getversion()
  tex.write("2023-08-18 v1.18")
end
%    \end{macrocode}
%    \end{macro}
%
% \subsubsection{Driver detection}
%
%    \begin{macrocode}
local ifpdf = tonumber(tex.outputmode or tex.pdfoutput) > 0
local prefix
local prefixes = {
  dvips   = "color ",
  dvipdfm = "pdf:sc ",
  truetex = "textcolor:",
  pctexps = "ps::",
}
local patterns = {
  ["^color "]            = "dvips",
  ["^pdf: *begincolor "] = "dvipdfm",
  ["^pdf: *bcolor "]     = "dvipdfm",
  ["^pdf: *bc "]         = "dvipdfm",
  ["^pdf: *setcolor "]   = "dvipdfm",
  ["^pdf: *scolor "]     = "dvipdfm",
  ["^pdf: *sc "]         = "dvipdfm",
  ["^textcolor:"]        = "truetex",
  ["^ps::"]              = "pctexps",
}
%    \end{macrocode}
%    \begin{macro}{info()}
%    \begin{macrocode}
local function info(msg, term)
  local target = "log"
  if term then
    target = "term and log"
  end
  texio.write_nl(target, "Package luacolor info: " .. msg .. ".")
  texio.write_nl(target, "")
end
%    \end{macrocode}
%    \end{macro}
%    \begin{macro}{dvidetect()}
%    \begin{macrocode}
function luacolor.dvidetect()
  local v = tex.box[0]
  assert(v.id == node.id("hlist"))
  for v in node.traverse_id(node.id("whatsit"), v.head) do
    if v and v.subtype == node.subtype("special") then
      local data = v.data
      for pattern, driver in pairs(patterns) do
        if string.find(data, pattern) then
          prefix = prefixes[driver]
          tex.write(driver)
          return
        end
      end
      info("\\special{" .. data .. "}", true)
      return
    end
  end
  info("Missing \\special", true)
end
%    \end{macrocode}
%    \end{macro}
%
% \subsubsection{Color strings}
%
%    \begin{macrocode}
local map = {
  n = 0,
}
%    \end{macrocode}
%    \begin{macro}{get()}
%    \begin{macrocode}
function luacolor.get(color)
  tex.write("" .. luacolor.getvalue(color))
end
%    \end{macrocode}
%    \end{macro}
%    \begin{macro}{getvalue()}
%    \begin{macrocode}
function luacolor.getvalue(color)
  local n = map[color]
  if not n then
    n = map.n + 1
    map.n = n
    map[n] = color
    map[color] = n
  end
  return n
end
%    \end{macrocode}
%    \end{macro}
%
% \subsubsection{Attribute register}
%
%    \begin{macro}{setattribute()}
%    \begin{macrocode}
local attribute
function luacolor.setattribute(attr)
  attribute = attr
end
%    \end{macrocode}
%    \end{macro}
%
%    \begin{macro}{getattribute()}
%    \begin{macrocode}
function luacolor.getattribute()
  return attribute
end
%    \end{macrocode}
%    \end{macro}
%
% \subsubsection{Whatsit insertion}
%
%    \begin{macrocode}
local LIST = 1
local LIST_LEADERS = 2
local LIST_DISC = 3
local COLOR = 4
local NOCOLOR = 5
local RULE = node.id("rule")
local node_types = {
  [node.id("hlist")] = LIST,
  [node.id("vlist")] = LIST,
  [node.id("rule")]  = COLOR,
  [node.id("glyph")] = COLOR,
  [node.id("disc")]  = LIST_DISC,
  [node.id("whatsit")] = {
    [node.subtype("pdf_colorstack")] =
      function(n)
        return n.stack == 0 and NOCOLOR or nil
      end,
    [node.subtype("special")] = COLOR,
    [node.subtype("pdf_literal")] = COLOR,
    [node.subtype("pdf_save")] = COLOR,
    [node.subtype("pdf_restore")] = COLOR, -- probably not needed
-- TODO (DPC)    [node.subtype("pdf_refximage")] = COLOR,
  },
  [node.id("glue")] =
    function(n)
      if n.subtype >= 100 then -- leaders
        if n.leader.id == RULE then
          return COLOR
        else
          return LIST_LEADERS
        end
      end
    end,
}
%    \end{macrocode}
%    \begin{macro}{get_type()}
%    \begin{macrocode}
local function get_type(n)
  local ret = node_types[n.id]
  if type(ret) == 'table' then
    ret = ret[n.subtype]
  end
  if type(ret) == 'function' then
    ret = ret(n)
  end
  return ret
end
%    \end{macrocode}
%    \end{macro}
%    \begin{macrocode}
local mode = 2 -- luatex.pdfliteral.direct
local WHATSIT = node.id("whatsit")
local SPECIAL = node.subtype("special")
local PDFLITERAL = node.subtype("pdf_literal")
local DRY_FALSE = false
local DRY_TRUE = true
%    \end{macrocode}
%    \begin{macro}{traverse()}
%    \begin{macrocode}
local function traverse(list, color, dry)
  if not list then
    return color
  end
  local head
  if get_type(list) == LIST then
    head = list.head
  elseif get_type(list) == LIST_DISC then
    head = list.replace
  else
    texio.write_nl("!!! Error: Wrong list type: " .. node.type(list.id))
    return color
  end
%<debug>texio.write_nl("traverse: " .. node.type(list.id))
  for n in node.traverse(head) do
%<debug>texio.write_nl("  node: " .. node.type(n.id))
    local t = get_type(n)
%<debug>texio.write_nl("TYPE "..tostring(t).. " "..tostring(node.type(node.getid(n))).." ".. tostring(node.getsubtype(n)))
    if t == LIST or t == LIST_DISC then
      color = traverse(n, color, dry)
    elseif t == LIST_LEADERS then
      local color_after = traverse(n.leader, color, DRY_TRUE)
      if color == color_after then
        traverse(n.leader, color, DRY_FALSE or dry)
      else
        traverse(n.leader, '', DRY_FALSE or dry)
%    \end{macrocode}
% The color status is unknown here, because the leader box
% will or will not be set.
%    \begin{macrocode}
        color = ''
      end
    elseif t == COLOR then
      local v = node.has_attribute(n, attribute)
      if v then
        local newColor = map[v]
        if newColor ~= color then
          color = newColor
          if dry == DRY_FALSE then
            local newNode
            if ifpdf then
              newNode = node.new(WHATSIT, PDFLITERAL)
              newNode.mode = mode
              newNode.data = color
            else
              newNode = node.new(WHATSIT, SPECIAL)
              newNode.data = prefix .. color
            end
            head = node.insert_before(head, n, newNode)
          end
        end
      end
    elseif t == NOCOLOR then
      color = ''
    end
  end
  if get_type(list) == LIST then
    list.head = head
  else
    list.replace = head
  end
  return color
end
%    \end{macrocode}
%    \end{macro}
%
%    \begin{macro}{process()}
%    \begin{macrocode}
function luacolor.process(box)
  local color = ""
  local list = tex.getbox(box)
  traverse(list, color, DRY_FALSE)
end

if luatexbase.callbacktypes.pre_shipout_filter then
  luatexbase.add_to_callback("pre_shipout_filter", function(list)
    traverse(list, "", DRY_FALSE)
    return true
  end, "luacolor.process")
end
%    \end{macrocode}
%    \end{macro}
% For recent versions of luaotfload, we can register a callback to
% control how coloring glyph is handled for the color feature.
%    \begin{macrocode}
if luaotfload.set_colorhandler then
  local set_attribute = node.direct.set_attribute
  luaotfload.set_colorhandler(function(head, n, color)
    set_attribute(n, attribute, luacolor.getvalue(color))
    return head, n
  end)
end
%    \end{macrocode}
%
%    \begin{macrocode}
%</lua>
%    \end{macrocode}
% \section{Installation}
%
% \subsection{Download}
%
% \paragraph{Package.} This package is available on
% CTAN\footnote{\CTANpkg{luacolor}}:
% \begin{description}
% \item[\CTAN{macros/latex/contrib/luacolor/luacolor.dtx}] The source file.
% \item[\CTAN{macros/latex/contrib/luacolor/luacolor.pdf}] Documentation.
% \end{description}
%
%
% \paragraph{Bundle.} All the packages of the bundle `luacolor'
% are also available in a TDS compliant ZIP archive. There
% the packages are already unpacked and the documentation files
% are generated. The files and directories obey the TDS standard.
% \begin{description}
% \item[\CTANinstall{install/macros/latex/contrib/luacolor.tds.zip}]
% \end{description}
% \emph{TDS} refers to the standard ``A Directory Structure
% for \TeX\ Files'' (\CTANpkg{tds}). Directories
% with \xfile{texmf} in their name are usually organized this way.
%
% \subsection{Bundle installation}
%
% \paragraph{Unpacking.} Unpack the \xfile{luacolor.tds.zip} in the
% TDS tree (also known as \xfile{texmf} tree) of your choice.
% Example (linux):
% \begin{quote}
%   |unzip luacolor.tds.zip -d ~/texmf|
% \end{quote}
%
% \paragraph{Script installation.}
% Check the directory \xfile{TDS:scripts/luacolor/} for
% scripts that need further installation steps.
%
% \subsection{Package installation}
%
% \paragraph{Unpacking.} The \xfile{.dtx} file is a self-extracting
% \docstrip\ archive. The files are extracted by running the
% \xfile{.dtx} through \plainTeX:
% \begin{quote}
%   \verb|tex luacolor.dtx|
% \end{quote}
%
% \paragraph{TDS.} Now the different files must be moved into
% the different directories in your installation TDS tree
% (also known as \xfile{texmf} tree):
% \begin{quote}
% \def\t{^^A
% \begin{tabular}{@{}>{\ttfamily}l@{ $\rightarrow$ }>{\ttfamily}l@{}}
%   luacolor.sty & tex/latex/luacolor/luacolor.sty\\
%   luacolor.lua & scripts/luacolor/luacolor.lua\\
%   luacolor.pdf & doc/latex/luacolor/luacolor.pdf\\
%   luacolor.dtx & source/latex/luacolor/luacolor.dtx\\
% \end{tabular}^^A
% }^^A
% \sbox0{\t}^^A
% \ifdim\wd0>\linewidth
%   \begingroup
%     \advance\linewidth by\leftmargin
%     \advance\linewidth by\rightmargin
%   \edef\x{\endgroup
%     \def\noexpand\lw{\the\linewidth}^^A
%   }\x
%   \def\lwbox{^^A
%     \leavevmode
%     \hbox to \linewidth{^^A
%       \kern-\leftmargin\relax
%       \hss
%       \usebox0
%       \hss
%       \kern-\rightmargin\relax
%     }^^A
%   }^^A
%   \ifdim\wd0>\lw
%     \sbox0{\small\t}^^A
%     \ifdim\wd0>\linewidth
%       \ifdim\wd0>\lw
%         \sbox0{\footnotesize\t}^^A
%         \ifdim\wd0>\linewidth
%           \ifdim\wd0>\lw
%             \sbox0{\scriptsize\t}^^A
%             \ifdim\wd0>\linewidth
%               \ifdim\wd0>\lw
%                 \sbox0{\tiny\t}^^A
%                 \ifdim\wd0>\linewidth
%                   \lwbox
%                 \else
%                   \usebox0
%                 \fi
%               \else
%                 \lwbox
%               \fi
%             \else
%               \usebox0
%             \fi
%           \else
%             \lwbox
%           \fi
%         \else
%           \usebox0
%         \fi
%       \else
%         \lwbox
%       \fi
%     \else
%       \usebox0
%     \fi
%   \else
%     \lwbox
%   \fi
% \else
%   \usebox0
% \fi
% \end{quote}
% If you have a \xfile{docstrip.cfg} that configures and enables \docstrip's
% TDS installing feature, then some files can already be in the right
% place, see the documentation of \docstrip.
%
% \subsection{Refresh file name databases}
%
% If your \TeX~distribution
% (\TeX\,Live, \mikTeX, \dots) relies on file name databases, you must refresh
% these. For example, \TeX\,Live\ users run \verb|texhash| or
% \verb|mktexlsr|.
%
% \subsection{Some details for the interested}
%
% \paragraph{Unpacking with \LaTeX.}
% The \xfile{.dtx} chooses its action depending on the format:
% \begin{description}
% \item[\plainTeX:] Run \docstrip\ and extract the files.
% \item[\LaTeX:] Generate the documentation.
% \end{description}
% If you insist on using \LaTeX\ for \docstrip\ (really,
% \docstrip\ does not need \LaTeX), then inform the autodetect routine
% about your intention:
% \begin{quote}
%   \verb|latex \let\install=y\input{luacolor.dtx}|
% \end{quote}
% Do not forget to quote the argument according to the demands
% of your shell.
%
% \paragraph{Generating the documentation.}
% You can use both the \xfile{.dtx} or the \xfile{.drv} to generate
% the documentation. The process can be configured by the
% configuration file \xfile{ltxdoc.cfg}. For instance, put this
% line into this file, if you want to have A4 as paper format:
% \begin{quote}
%   \verb|\PassOptionsToClass{a4paper}{article}|
% \end{quote}
% An example follows how to generate the
% documentation with pdf\LaTeX:
% \begin{quote}
%\begin{verbatim}
%pdflatex luacolor.dtx
%makeindex -s gind.ist luacolor.idx
%pdflatex luacolor.dtx
%makeindex -s gind.ist luacolor.idx
%pdflatex luacolor.dtx
%\end{verbatim}
% \end{quote}
%
% \begin{History}
%   \begin{Version}{2007/12/12 v1.0}
%   \item
%     First public version.
%   \end{Version}
%   \begin{Version}{2009/04/10 v1.1}
%   \item
%     Fixes for changed syntax of \cs{directlua} in \LuaTeX\ 0.36.
%   \end{Version}
%   \begin{Version}{2010/03/09 v1.2}
%   \item
%     Adaptation for package \xpackage{luatex} 2010/03/09 v0.4.
%   \end{Version}
%   \begin{Version}{2010/12/13 v1.3}
%   \item
%     Support for \cs{pdfxform} added.
%   \item
%     Loaded package \xpackage{luatexbase-attr} recognized.
%   \item
%     Update for \hologo{LuaTeX}: `list' fields renamed to `head' in v0.65.0.
%   \end{Version}
%   \begin{Version}{2011/03/29 v1.4}
%   \item
%     Avoid whatsit insertion if option \xoption{monochrome} is used
%     (thanks Manuel P\'egouri\'e-Gonnard).
%   \end{Version}
%   \begin{Version}{2011/04/22 v1.5}
%   \item
%     Bug fix by Manuel P\'egouri\'e-Gonnard: A typo prevented the
%     detection of whatsits and applying color changes for
%     \cs{pdfliteral} and \cs{special} nodes that might contain
%     typesetting material.
%   \item
%     Bug fix by Manuel P\'egouri\'e-Gonnard: Now colors are also
%     applied to leader boxes.
%   \item
%     Unnecessary color settings are removed for leaders boxes,
%     if after the leader box the color has not changed.
%     The costs are a little runtime, leader boxes are processed twice.
%   \item
%     Additional whatsits that are colored: \texttt{pdf\_refximage}.
%   \item
%     Workaround for bug with \texttt{node.insert\_before}
%     removed for the version after \hologo{LuaTeX} 0.65, because
%     bug was fixed in 0.27. (Thanks Manuel P\'egouri\'e-Gonnard.)
%   \end{Version}
%   \begin{Version}{2011/04/23 v1.6}
%   \item
%     Bug fix for nested leader boxes.
%   \item
%     Bug fix for leader boxes that change color, but
%     are not set because of missing place.
%   \item
%     Version check for Lua module added.
%   \end{Version}
%   \begin{Version}{2011/10/22 v1.7}
%   \item
%     Lua functions \texttt{getattribute} and \texttt{getvalue}
%     added to tell other external Lua functions
%     the attribute register number for coloring.
%   \end{Version}
%   \begin{Version}{2011/11/01 v1.8}
%   \item
%     Use of \texttt{node.subtype} instead of magic numbers.
%   \end{Version}
%   \begin{Version}{2016/05/13 v1.9}
%   \item
%     More use of \texttt{node.subtype} instead of magic numbers.
%   \item luatex 85 updates
%   \end{Version}
%   \begin{Version}{2016/05/16 v1.10}
%   \item
%     Documentation updates.
%   \end{Version}
%   \begin{Version}{2018/11/22 v1.11}
%   \item
%     handle issue 43.
%   \item
%     removed pre-0.65 stuff
%   \end{Version}
%   \begin{Version}{2019/07/25 v1.12}
%   \item
%     removed uses of module function, see PR70
%   \end{Version}
%   \begin{Version}{2019/11/29 v1.13}
%   \item
%     Documentation updates.
%   \item  Use \xpackage{iftex} directly.
%   \end{Version}
%   \begin{Version}{2020-02-22 v1.14}
%   \item Drop use of \textsf{iftex} \textsf{ltxcmds} and \textsf{infwarerr}.
%   \item Assume \textsf{ltluatex} preloaded into format (true since 2015).
%   \item Patch \verb|\saveboxresource| rather than \verb|\pdfxform| (keep old name as alias).
%   \item Grab the number via Lua so that a \verb|\immediate| prefix still works with
%         \verb|\saveboxresource|\slash \verb|\pdfxform|.
%   \item Added handler for the color feature of luaotfload      
%   \end{Version}
%   \begin{Version}{2020-02-24 v1.15}
%   \item Grab all possible arguments for \verb|\saveboxresource|\slash \verb|\pdfxform|
%   \end{Version}
%   \begin{Version}{2020-04-04 v1.16}
%   \item Reset color after \verb|pdf_colorstack| whatsits.
%   \end{Version}
%   \begin{Version}{2021-02-17 v1.17}
%   \item Use \LaTeXe's new \verb|pre_shipout_filter| callback if it's available to
%         allow coloring background and foreground layer material
%   \end{Version}
%   \begin{Version}{2023-08-18 v1.18}
%   \item added \verb|\reset@color| to \verb|\set@color| for \verb|\mathcolor|, issue 7.
%   \end{Version}
% \end{History}
%
% \PrintIndex
%
% \Finale
\endinput

%        (quote the arguments according to the demands of your shell)
%
% Documentation:
%    (a) If luacolor.drv is present:
%           latex luacolor.drv
%    (b) Without luacolor.drv:
%           latex luacolor.dtx; ...
%    The class ltxdoc loads the configuration file ltxdoc.cfg
%    if available. Here you can specify further options, e.g.
%    use A4 as paper format:
%       \PassOptionsToClass{a4paper}{article}
%
%    Programm calls to get the documentation (example):
%       pdflatex luacolor.dtx
%       makeindex -s gind.ist luacolor.idx
%       pdflatex luacolor.dtx
%       makeindex -s gind.ist luacolor.idx
%       pdflatex luacolor.dtx
%
% Installation:
%    TDS:tex/latex/luacolor/luacolor.sty
%    TDS:scripts/luacolor/luacolor.lua
%    TDS:doc/latex/luacolor/luacolor.pdf
%    TDS:source/latex/luacolor/luacolor.dtx
%
%<*ignore>
\begingroup
  \catcode123=1 %
  \catcode125=2 %
  \def\x{LaTeX2e}%
\expandafter\endgroup
\ifcase 0\ifx\install y1\fi\expandafter
         \ifx\csname processbatchFile\endcsname\relax\else1\fi
         \ifx\fmtname\x\else 1\fi\relax
\else\csname fi\endcsname
%</ignore>
%<*install>
\input docstrip.tex
\Msg{************************************************************************}
\Msg{* Installation}
\Msg{* Package: luacolor 2023-08-18 v1.18 Color support via LuaTeX's attributes (HO)}
\Msg{************************************************************************}

\keepsilent
\askforoverwritefalse

\let\MetaPrefix\relax
\preamble

This is a generated file.

Project: luacolor
Version: 2023-08-18 v1.18

Copyright (C)
   2007, 2009-2011 Heiko Oberdiek
   2016-2023 Oberdiek Package Support Group

This work may be distributed and/or modified under the
conditions of the LaTeX Project Public License, either
version 1.3c of this license or (at your option) any later
version. This version of this license is in
   https://www.latex-project.org/lppl/lppl-1-3c.txt
and the latest version of this license is in
   https://www.latex-project.org/lppl.txt
and version 1.3 or later is part of all distributions of
LaTeX version 2005/12/01 or later.

This work has the LPPL maintenance status "maintained".

The Current Maintainers of this work are
Heiko Oberdiek and the Oberdiek Package Support Group
https://github.com/ho-tex/luacolor/issues


This work consists of the main source file luacolor.dtx
and the derived files
   luacolor.sty, luacolor.pdf, luacolor.ins, luacolor.drv,
   luacolor.lua,

\endpreamble
\let\MetaPrefix\DoubleperCent

\generate{%
  \file{luacolor.ins}{\from{luacolor.dtx}{install}}%
  \file{luacolor.drv}{\from{luacolor.dtx}{driver}}%
  \usedir{tex/latex/luacolor}%
  \file{luacolor.sty}{\from{luacolor.dtx}{package}}%
}
\def\MetaPrefix{-- }
\def\defaultpostamble{%
  \MetaPrefix^^J%
  \MetaPrefix\space End of File `\outFileName'.%
}
\def\currentpostamble{\defaultpostamble}%
\generate{%
  \usedir{scripts/luacolor}%
%  \file{oberdiek.luacolor.lua}{\from{luacolor.dtx}{lua}}%
  \file{luacolor.lua}{\from{luacolor.dtx}{lua}}%
}

\catcode32=13\relax% active space
\let =\space%
\Msg{************************************************************************}
\Msg{*}
\Msg{* To finish the installation you have to move the following}
\Msg{* file into a directory searched by TeX:}
\Msg{*}
\Msg{*     luacolor.sty}
\Msg{*}
\Msg{* And install the following script file:}
\Msg{*}
\Msg{*     luacolor.lua,}
\Msg{*}
\Msg{* To produce the documentation run the file `luacolor.dtx'}
\Msg{* through LaTeX.}
\Msg{*}
\Msg{* Happy TeXing!}
\Msg{*}
\Msg{************************************************************************}

\endbatchfile
%</install>
%<*ignore>
\fi
%</ignore>
%<*driver>
\NeedsTeXFormat{LaTeX2e}
\ProvidesFile{luacolor.drv}%
  [2023-08-18 v1.18 Color support via LuaTeX's attributes (HO)]%
\documentclass{ltxdoc}
\usepackage{holtxdoc}[2011/11/22]
\usepackage{paralist}
\makeatletter
\g@addto@macro\MakePrivateLetters{%
  \@makeother\_%
}
\makeatother
\begin{document}
  \DocInput{luacolor.dtx}%
\end{document}
%</driver>
% \fi
%
%
%
% \GetFileInfo{luacolor.drv}
%
% \title{The \xpackage{luacolor} package}
% \date{2023-08-18 v1.18}
% \author{Heiko Oberdiek\thanks
% {Please report any issues at \url{https://github.com/ho-tex/luacolor/issues}}}
%
% \maketitle
%
% \begin{abstract}
% Package \xpackage{luacolor} implements color support based
% on \LuaTeX's node attributes.
% \end{abstract}
%
% \tableofcontents
%
% \section{Documentation}
%
% \subsection{Introduction}
%
% This package uses a \LuaTeX's attribute register to
% to annotate nodes with color information.
% If a color is set, then the attribute register
% is set to this color and all nodes created in its scope
% (current group) are annotated with this attribute.
% Now the color property behaves much the same way
% as the font property.
%
% \subsection{Usage}
%
% Package \xpackage{color} is loaded automatically by this
% package \xpackage{luacolor}. If you need a special driver
% option or you prefer package \xpackage{xcolor}, then load
% it before package \xpackage{luacolor}, for example:
% \begin{quote}
%   |\usepackage[dvipdfmx]{xcolor}|
% \end{quote}
% The package \xpackage{luacolor} is loaded without options:
% \begin{quote}
%   |\usepackage{luacolor}|
% \end{quote}
% It is able to detect PDF mode and DVI drivers are
% differentiated by its color specials. Therefore the
% package do need driver options.
%
% Then it redefines the color setting commands to set
% attributes instead of whatsits for color.
%
% At last the attribute annotations of the nodes in the
% output box must be analyzed to insert the necessary color whatsits.
% (With older \LuaTeX\ that lack the appropriate callback function 
% the package \xpackage{atbegshi} is used to get control
% before a box is shipped out.)
%
% \begin{declcs}{luacolorProcessBox} \M{box}
% \end{declcs}
% Macro \cs{luacolorProcessBox} processes the box \meta{box}
% in the previously described manner. It is automatically
% called for pages, but not for XForm objects.
% Before passing a box to \cs{pdfxform}, call \cs{luacolorProcessBox}
% first.
%
% \subsection{Limitations}
%
% \begin{description}
% \item[Ligatures with different colored components:]
%   Package \xpackage{luacolor} sees the ligature after
%   the paragraph building and page breaking, when a
%   page is to be shipped out. Therefore it cannot
%   break ligatures, because the components might occupy
%   different space. Therefore it is the responsibility
%   of the ligature forming process to deal with different
%   colored glyphs that form a ligature. The user can avoid
%   the problem entirely by explicitly breaking the ligature
%   at the places where the color changes.
%  \item \dots
% \end{description}
%
% \StopEventually{
% }
%
% \section{Implementation}
%
%    \begin{macrocode}
%<*package>
%    \end{macrocode}
%
% \subsection{Catcodes and identification}
%
%    \begin{macrocode}
\begingroup\catcode61\catcode48\catcode32=10\relax%
  \catcode13=5 % ^^M
  \endlinechar=13 %
  \catcode123=1 % {
  \catcode125=2 % }
  \catcode64=11 % @
  \def\x{\endgroup
    \expandafter\edef\csname LuaCol@AtEnd\endcsname{%
      \endlinechar=\the\endlinechar\relax
      \catcode13=\the\catcode13\relax
      \catcode32=\the\catcode32\relax
      \catcode35=\the\catcode35\relax
      \catcode61=\the\catcode61\relax
      \catcode64=\the\catcode64\relax
      \catcode123=\the\catcode123\relax
      \catcode125=\the\catcode125\relax
    }%
  }%
\x\catcode61\catcode48\catcode32=10\relax%
\catcode13=5 % ^^M
\endlinechar=13 %
\catcode35=6 % #
\catcode64=11 % @
\catcode123=1 % {
\catcode125=2 % }
\def\TMP@EnsureCode#1#2{%
  \edef\LuaCol@AtEnd{%
    \LuaCol@AtEnd
    \catcode#1=\the\catcode#1\relax
  }%
  \catcode#1=#2\relax
}
\TMP@EnsureCode{34}{12}% "
\TMP@EnsureCode{39}{12}% '
\TMP@EnsureCode{40}{12}% (
\TMP@EnsureCode{41}{12}% )
\TMP@EnsureCode{42}{12}% *
\TMP@EnsureCode{43}{12}% +
\TMP@EnsureCode{44}{12}% ,
\TMP@EnsureCode{45}{12}% -
\TMP@EnsureCode{46}{12}% .
\TMP@EnsureCode{47}{12}% /
\TMP@EnsureCode{58}{12}% :
\TMP@EnsureCode{60}{12}% <
\TMP@EnsureCode{62}{12}% >
\TMP@EnsureCode{91}{12}% [
\TMP@EnsureCode{93}{12}% ]
\TMP@EnsureCode{95}{12}% _ (other!)
\TMP@EnsureCode{96}{12}% `
\edef\LuaCol@AtEnd{\LuaCol@AtEnd\noexpand\endinput}
%    \end{macrocode}
%
%    Package identification.
%    \begin{macrocode}
\NeedsTeXFormat{LaTeX2e}
\ProvidesPackage{luacolor}%
  [2023-08-18 v1.18 Color support via LuaTeX's attributes (HO)]
%    \end{macrocode}
%
% \subsection{Check for \LuaTeX}
%
%    Without \LuaTeX\ there is no point in using this package.
%    \begin{macrocode}
\RequirePackage{color}
%    \end{macrocode}
%
%    \begin{macrocode}
\ifx\directlua\@undefined
  \PackageError{luacolor}{%
    This package may only be run using LuaTeX%
  }\@ehc
  \expandafter\LuaCol@AtEnd
\fi%
%    \end{macrocode}
%
% \subsection{Check for disabled colors}
%
%    \begin{macrocode}
\ifcolors@
\else
  \PackageWarningNoLine{luacolor}{%
    Colors are disabled by option `monochrome'%
  }%
  \def\set@color{}%
  \def\reset@color{}%
  \def\set@page@color{}%
  \def\define@color#1#2{}%
  \expandafter\LuaCol@AtEnd
\fi%
%    \end{macrocode}
%
% \subsection{Load module and check version}
%
%    \begin{macrocode}
\directlua{%
  require("luacolor")%
}
%    \end{macrocode}
%    \begin{macrocode}
\begingroup
  \edef\x{\directlua{tex.write("2023-08-18 v1.18")}}%
  \edef\y{%
    \directlua{%
      if oberdiek.luacolor.getversion then %
        oberdiek.luacolor.getversion()%
      end%
    }%
  }%
  \ifx\x\y
  \else
    \PackageError{luacolor}{%
      Wrong version of lua module.\MessageBreak
      Package version: \x\MessageBreak
      Lua module: \y
    }\@ehc
  \fi
\endgroup
%    \end{macrocode}
%
% \subsection{Find driver}
%
%    \begin{macrocode}
\ifnum\outputmode=\@ne
\else
  \begingroup
    \def\current@color{}%
    \def\reset@color{}%
    \setbox\z@=\hbox{%
      \begingroup
        \set@color
      \endgroup
    }%
    \edef\reserved@a{%
      \directlua{%
        oberdiek.luacolor.dvidetect()%
      }%
    }%
    \ifx\reserved@a\@empty
      \PackageError{luacolor}{%
        DVI driver detection failed because of\MessageBreak
        unrecognized color \string\special
      }\@ehc
      \endgroup
      \expandafter\expandafter\expandafter\LuaCol@AtEnd
    \else
      \PackageInfo{luacolor}{%
        Type of color \string\special: \reserved@a
      \@gobble}%
    \fi%
  \endgroup
\fi
%    \end{macrocode}
%
% \subsection{Attribute setting}
%
%    \begin{macro}{\LuaCol@Attribute}
%    \begin{macrocode}
\newattribute\LuaCol@Attribute
\let\LuaCol@setattribute\setattribute
\directlua{%
  oberdiek.luacolor.setattribute(\number\allocationnumber)%
}
%    \end{macrocode}
%    \end{macro}
%
%    \begin{macro}{\set@color}
% change 2023-08-18: added \cs{reset@color} so that \cs{mathcolor} can gobble it,
% issue 7.   
%    \begin{macrocode}
\protected\def\set@color{%
  \LuaCol@setattribute\LuaCol@Attribute{%
    \directlua{%
      oberdiek.luacolor.get("\luaescapestring{\current@color}")%
    }%
  }%
 \aftergroup\reset@color 
}
%    \end{macrocode}
%    \end{macro}
%    \begin{macro}{\reset@color}
%    \begin{macrocode}
\def\reset@color{}
%    \end{macrocode}
%    \end{macro}
%
% \subsection{Whatsit insertion}
%
%    \begin{macro}{\luacolorProcessBox}
%    \begin{macrocode}
\def\luacolorProcessBox#1{%
  \directlua{%
    oberdiek.luacolor.process(\number#1)%
  }%
}
%    \end{macrocode}
%    \end{macro}
%
%    \begin{macrocode}
\directlua{%
  if luatexbase.callbacktypes.pre_shipout_filter then
    token.get_next()
  end
}\@secondoftwo\@gobble{
  \RequirePackage{atbegshi}[2011/01/30]
  \AtBeginShipout{%
    \luacolorProcessBox\AtBeginShipoutBox
  }
}
%    \end{macrocode}
%
%    Set default color.
%    \begin{macrocode}
\set@color
%    \end{macrocode}
%
% \subsection{\cs{pdfxform}/\cs{saveboxresource} support}
%
%    \begin{macrocode}
\ifnum\outputmode=\@ne
    \let\LuaCol@org@pdfxform\saveboxresource
%    \end{macrocode}
% First we need some helpers to allow expandable code to parse keyword style arguments:
%    \begin{macrocode}
    \def\LuaCol@iii@i@ii#1#2#3{#3{#1}{#2}}
    \def\LuaCol@ii@i#1#2{{#2#1}}
    \def\LuaCol@if@keyword#1#2#3{%
      \expanded{\unexpanded{\LuaCol@iii@i@ii{#2}{#3}}\expandafter}%
      \directlua{%
        token.put_next(token.create(token.scan_keyword(token.scan_string())
        and '@firstoftwo'
        or '@secondoftwo'))
      }{#1}%
    }
%    \end{macrocode}
% The following macro scans a integer and expands to a token equivalent to a chardef
% whose value corresponds to the scanned integer. This allows the integer to be passed
% around as a undelimited argument.
%    \begin{macrocode}
    \def\LuaCol@scan@number{%
      \directlua{
        token.put_next(token.new(token.scan_int(), token.command_id'char_given'))
      }%
    }
%    \end{macrocode}
% \TeX\ primitives like \cs{saveboxresource} read braced arguments in a special way.
% Especially they expand everything until they find a left brace. To simulate this, we
% use Lua to expand everything else:
%    \begin{macrocode}
    \def\LuaCol@scan@tobrace{%
      \directlua{
        local relax, space = token.command_id'relax', token.command_id'spacer'
        local t
        repeat
          t = token.scan_token()
        until not (t.command == relax or t.command == space)
        token.put_next(t)
      }%
    }
    \def\LuaCol@scan@boxresource@i#1#2{%
      \LuaCol@if@keyword{attr}{%
        \expanded{\unexpanded{\LuaCol@scan@boxresource@iI{#1#2attr}}%
          \expandafter\expandafter\expandafter}%
        \LuaCol@scan@tobrace
      }{%
        \LuaCol@scan@boxresource@ii{#1#2}%
      }%
    }
    \def\LuaCol@scan@boxresource@iI#1#2{\LuaCol@scan@boxresource@ii{#1{#2}}}
    \def\LuaCol@scan@boxresource@ii#1{%
      \LuaCol@if@keyword{resources}{%
        \expanded{\unexpanded{\LuaCol@scan@boxresource@iiI{#1resources}}%
          \expandafter\expandafter\expandafter}%
        \LuaCol@scan@tobrace
      }{%
        \LuaCol@scan@boxresource@iii{#1}%
      }%
    }
    \def\LuaCol@scan@boxresource@iiI#1#2{\LuaCol@scan@boxresource@iii{#1{#2}}}
    \def\LuaCol@scan@boxresource@iii#1{%
      \LuaCol@if@keyword{margin}{%
        \expanded{\unexpanded{\LuaCol@scan@boxresource@iv{#1margin }}%
          \expandafter\expandafter\expandafter}%
        \LuaCol@scan@number
      }{%
        \LuaCol@scan@boxresource@iv{#1}{}%
      }%
    }
    \def\LuaCol@scan@boxresource@iv#1#2{%
      \expanded{\unexpanded{\LuaCol@scan@boxresource@v{#1#2}}%
        \expandafter\expandafter\expandafter}%
      \LuaCol@scan@number
    }
    \def\LuaCol@scan@boxresource@v#1#2{%
      \luacolorProcessBox{#2}%
      \LuaCol@org@pdfxform#1#2%
    }

%    \end{macrocode}
% This could be written in Lua, but at least upto Lua\TeX~1.11, feeding back too many tokens
% from Lua to \TeX\ triggers a segmentation fault.
% This is written in Lua so the integer setting is expandable and does not interfere with
% a preceding \verb|\immediate|.
%    \begin{macrocode}
    \protected\def\saveboxresource{%
      \LuaCol@if@keyword{type}{%
        \expandafter
        \expanded{\unexpanded{\LuaCol@scan@boxresource@i{type }}%
          \expandafter\expandafter\expandafter}%
        \LuaCol@scan@number
      }{%
        \LuaCol@scan@boxresource@i{}{}%
      }%
    }
%    \end{macrocode}
%    Legacy alias.
%    \begin{macrocode}
    \let\pdfxform\saveboxresource
\fi
%    \end{macrocode}
%
%    \begin{macrocode}
\LuaCol@AtEnd%
%</package>
%    \end{macrocode}
%
% \subsection{Lua module}
%
%    \begin{macrocode}
%<*lua>
%    \end{macrocode}
%    Box zero contains a \cs{hbox} with the color \cs{special}.
%    That is analyzed to get the prefix for the color setting
%    \cs{special}.
%    \begin{macrocode}
oberdiek = oberdiek or {}
local luacolor = oberdiek.luacolor or {}
oberdiek.luacolor = luacolor
%    \end{macrocode}
%    \begin{macro}{getversion()}
%    \begin{macrocode}
function luacolor.getversion()
  tex.write("2023-08-18 v1.18")
end
%    \end{macrocode}
%    \end{macro}
%
% \subsubsection{Driver detection}
%
%    \begin{macrocode}
local ifpdf = tonumber(tex.outputmode or tex.pdfoutput) > 0
local prefix
local prefixes = {
  dvips   = "color ",
  dvipdfm = "pdf:sc ",
  truetex = "textcolor:",
  pctexps = "ps::",
}
local patterns = {
  ["^color "]            = "dvips",
  ["^pdf: *begincolor "] = "dvipdfm",
  ["^pdf: *bcolor "]     = "dvipdfm",
  ["^pdf: *bc "]         = "dvipdfm",
  ["^pdf: *setcolor "]   = "dvipdfm",
  ["^pdf: *scolor "]     = "dvipdfm",
  ["^pdf: *sc "]         = "dvipdfm",
  ["^textcolor:"]        = "truetex",
  ["^ps::"]              = "pctexps",
}
%    \end{macrocode}
%    \begin{macro}{info()}
%    \begin{macrocode}
local function info(msg, term)
  local target = "log"
  if term then
    target = "term and log"
  end
  texio.write_nl(target, "Package luacolor info: " .. msg .. ".")
  texio.write_nl(target, "")
end
%    \end{macrocode}
%    \end{macro}
%    \begin{macro}{dvidetect()}
%    \begin{macrocode}
function luacolor.dvidetect()
  local v = tex.box[0]
  assert(v.id == node.id("hlist"))
  for v in node.traverse_id(node.id("whatsit"), v.head) do
    if v and v.subtype == node.subtype("special") then
      local data = v.data
      for pattern, driver in pairs(patterns) do
        if string.find(data, pattern) then
          prefix = prefixes[driver]
          tex.write(driver)
          return
        end
      end
      info("\\special{" .. data .. "}", true)
      return
    end
  end
  info("Missing \\special", true)
end
%    \end{macrocode}
%    \end{macro}
%
% \subsubsection{Color strings}
%
%    \begin{macrocode}
local map = {
  n = 0,
}
%    \end{macrocode}
%    \begin{macro}{get()}
%    \begin{macrocode}
function luacolor.get(color)
  tex.write("" .. luacolor.getvalue(color))
end
%    \end{macrocode}
%    \end{macro}
%    \begin{macro}{getvalue()}
%    \begin{macrocode}
function luacolor.getvalue(color)
  local n = map[color]
  if not n then
    n = map.n + 1
    map.n = n
    map[n] = color
    map[color] = n
  end
  return n
end
%    \end{macrocode}
%    \end{macro}
%
% \subsubsection{Attribute register}
%
%    \begin{macro}{setattribute()}
%    \begin{macrocode}
local attribute
function luacolor.setattribute(attr)
  attribute = attr
end
%    \end{macrocode}
%    \end{macro}
%
%    \begin{macro}{getattribute()}
%    \begin{macrocode}
function luacolor.getattribute()
  return attribute
end
%    \end{macrocode}
%    \end{macro}
%
% \subsubsection{Whatsit insertion}
%
%    \begin{macrocode}
local LIST = 1
local LIST_LEADERS = 2
local LIST_DISC = 3
local COLOR = 4
local NOCOLOR = 5
local RULE = node.id("rule")
local node_types = {
  [node.id("hlist")] = LIST,
  [node.id("vlist")] = LIST,
  [node.id("rule")]  = COLOR,
  [node.id("glyph")] = COLOR,
  [node.id("disc")]  = LIST_DISC,
  [node.id("whatsit")] = {
    [node.subtype("pdf_colorstack")] =
      function(n)
        return n.stack == 0 and NOCOLOR or nil
      end,
    [node.subtype("special")] = COLOR,
    [node.subtype("pdf_literal")] = COLOR,
    [node.subtype("pdf_save")] = COLOR,
    [node.subtype("pdf_restore")] = COLOR, -- probably not needed
-- TODO (DPC)    [node.subtype("pdf_refximage")] = COLOR,
  },
  [node.id("glue")] =
    function(n)
      if n.subtype >= 100 then -- leaders
        if n.leader.id == RULE then
          return COLOR
        else
          return LIST_LEADERS
        end
      end
    end,
}
%    \end{macrocode}
%    \begin{macro}{get_type()}
%    \begin{macrocode}
local function get_type(n)
  local ret = node_types[n.id]
  if type(ret) == 'table' then
    ret = ret[n.subtype]
  end
  if type(ret) == 'function' then
    ret = ret(n)
  end
  return ret
end
%    \end{macrocode}
%    \end{macro}
%    \begin{macrocode}
local mode = 2 -- luatex.pdfliteral.direct
local WHATSIT = node.id("whatsit")
local SPECIAL = node.subtype("special")
local PDFLITERAL = node.subtype("pdf_literal")
local DRY_FALSE = false
local DRY_TRUE = true
%    \end{macrocode}
%    \begin{macro}{traverse()}
%    \begin{macrocode}
local function traverse(list, color, dry)
  if not list then
    return color
  end
  local head
  if get_type(list) == LIST then
    head = list.head
  elseif get_type(list) == LIST_DISC then
    head = list.replace
  else
    texio.write_nl("!!! Error: Wrong list type: " .. node.type(list.id))
    return color
  end
%<debug>texio.write_nl("traverse: " .. node.type(list.id))
  for n in node.traverse(head) do
%<debug>texio.write_nl("  node: " .. node.type(n.id))
    local t = get_type(n)
%<debug>texio.write_nl("TYPE "..tostring(t).. " "..tostring(node.type(node.getid(n))).." ".. tostring(node.getsubtype(n)))
    if t == LIST or t == LIST_DISC then
      color = traverse(n, color, dry)
    elseif t == LIST_LEADERS then
      local color_after = traverse(n.leader, color, DRY_TRUE)
      if color == color_after then
        traverse(n.leader, color, DRY_FALSE or dry)
      else
        traverse(n.leader, '', DRY_FALSE or dry)
%    \end{macrocode}
% The color status is unknown here, because the leader box
% will or will not be set.
%    \begin{macrocode}
        color = ''
      end
    elseif t == COLOR then
      local v = node.has_attribute(n, attribute)
      if v then
        local newColor = map[v]
        if newColor ~= color then
          color = newColor
          if dry == DRY_FALSE then
            local newNode
            if ifpdf then
              newNode = node.new(WHATSIT, PDFLITERAL)
              newNode.mode = mode
              newNode.data = color
            else
              newNode = node.new(WHATSIT, SPECIAL)
              newNode.data = prefix .. color
            end
            head = node.insert_before(head, n, newNode)
          end
        end
      end
    elseif t == NOCOLOR then
      color = ''
    end
  end
  if get_type(list) == LIST then
    list.head = head
  else
    list.replace = head
  end
  return color
end
%    \end{macrocode}
%    \end{macro}
%
%    \begin{macro}{process()}
%    \begin{macrocode}
function luacolor.process(box)
  local color = ""
  local list = tex.getbox(box)
  traverse(list, color, DRY_FALSE)
end

if luatexbase.callbacktypes.pre_shipout_filter then
  luatexbase.add_to_callback("pre_shipout_filter", function(list)
    traverse(list, "", DRY_FALSE)
    return true
  end, "luacolor.process")
end
%    \end{macrocode}
%    \end{macro}
% For recent versions of luaotfload, we can register a callback to
% control how coloring glyph is handled for the color feature.
%    \begin{macrocode}
if luaotfload.set_colorhandler then
  local set_attribute = node.direct.set_attribute
  luaotfload.set_colorhandler(function(head, n, color)
    set_attribute(n, attribute, luacolor.getvalue(color))
    return head, n
  end)
end
%    \end{macrocode}
%
%    \begin{macrocode}
%</lua>
%    \end{macrocode}
% \section{Installation}
%
% \subsection{Download}
%
% \paragraph{Package.} This package is available on
% CTAN\footnote{\CTANpkg{luacolor}}:
% \begin{description}
% \item[\CTAN{macros/latex/contrib/luacolor/luacolor.dtx}] The source file.
% \item[\CTAN{macros/latex/contrib/luacolor/luacolor.pdf}] Documentation.
% \end{description}
%
%
% \paragraph{Bundle.} All the packages of the bundle `luacolor'
% are also available in a TDS compliant ZIP archive. There
% the packages are already unpacked and the documentation files
% are generated. The files and directories obey the TDS standard.
% \begin{description}
% \item[\CTANinstall{install/macros/latex/contrib/luacolor.tds.zip}]
% \end{description}
% \emph{TDS} refers to the standard ``A Directory Structure
% for \TeX\ Files'' (\CTANpkg{tds}). Directories
% with \xfile{texmf} in their name are usually organized this way.
%
% \subsection{Bundle installation}
%
% \paragraph{Unpacking.} Unpack the \xfile{luacolor.tds.zip} in the
% TDS tree (also known as \xfile{texmf} tree) of your choice.
% Example (linux):
% \begin{quote}
%   |unzip luacolor.tds.zip -d ~/texmf|
% \end{quote}
%
% \paragraph{Script installation.}
% Check the directory \xfile{TDS:scripts/luacolor/} for
% scripts that need further installation steps.
%
% \subsection{Package installation}
%
% \paragraph{Unpacking.} The \xfile{.dtx} file is a self-extracting
% \docstrip\ archive. The files are extracted by running the
% \xfile{.dtx} through \plainTeX:
% \begin{quote}
%   \verb|tex luacolor.dtx|
% \end{quote}
%
% \paragraph{TDS.} Now the different files must be moved into
% the different directories in your installation TDS tree
% (also known as \xfile{texmf} tree):
% \begin{quote}
% \def\t{^^A
% \begin{tabular}{@{}>{\ttfamily}l@{ $\rightarrow$ }>{\ttfamily}l@{}}
%   luacolor.sty & tex/latex/luacolor/luacolor.sty\\
%   luacolor.lua & scripts/luacolor/luacolor.lua\\
%   luacolor.pdf & doc/latex/luacolor/luacolor.pdf\\
%   luacolor.dtx & source/latex/luacolor/luacolor.dtx\\
% \end{tabular}^^A
% }^^A
% \sbox0{\t}^^A
% \ifdim\wd0>\linewidth
%   \begingroup
%     \advance\linewidth by\leftmargin
%     \advance\linewidth by\rightmargin
%   \edef\x{\endgroup
%     \def\noexpand\lw{\the\linewidth}^^A
%   }\x
%   \def\lwbox{^^A
%     \leavevmode
%     \hbox to \linewidth{^^A
%       \kern-\leftmargin\relax
%       \hss
%       \usebox0
%       \hss
%       \kern-\rightmargin\relax
%     }^^A
%   }^^A
%   \ifdim\wd0>\lw
%     \sbox0{\small\t}^^A
%     \ifdim\wd0>\linewidth
%       \ifdim\wd0>\lw
%         \sbox0{\footnotesize\t}^^A
%         \ifdim\wd0>\linewidth
%           \ifdim\wd0>\lw
%             \sbox0{\scriptsize\t}^^A
%             \ifdim\wd0>\linewidth
%               \ifdim\wd0>\lw
%                 \sbox0{\tiny\t}^^A
%                 \ifdim\wd0>\linewidth
%                   \lwbox
%                 \else
%                   \usebox0
%                 \fi
%               \else
%                 \lwbox
%               \fi
%             \else
%               \usebox0
%             \fi
%           \else
%             \lwbox
%           \fi
%         \else
%           \usebox0
%         \fi
%       \else
%         \lwbox
%       \fi
%     \else
%       \usebox0
%     \fi
%   \else
%     \lwbox
%   \fi
% \else
%   \usebox0
% \fi
% \end{quote}
% If you have a \xfile{docstrip.cfg} that configures and enables \docstrip's
% TDS installing feature, then some files can already be in the right
% place, see the documentation of \docstrip.
%
% \subsection{Refresh file name databases}
%
% If your \TeX~distribution
% (\TeX\,Live, \mikTeX, \dots) relies on file name databases, you must refresh
% these. For example, \TeX\,Live\ users run \verb|texhash| or
% \verb|mktexlsr|.
%
% \subsection{Some details for the interested}
%
% \paragraph{Unpacking with \LaTeX.}
% The \xfile{.dtx} chooses its action depending on the format:
% \begin{description}
% \item[\plainTeX:] Run \docstrip\ and extract the files.
% \item[\LaTeX:] Generate the documentation.
% \end{description}
% If you insist on using \LaTeX\ for \docstrip\ (really,
% \docstrip\ does not need \LaTeX), then inform the autodetect routine
% about your intention:
% \begin{quote}
%   \verb|latex \let\install=y% \iffalse meta-comment
%
% File: luacolor.dtx
% Version: 2023-08-18 v1.18
% Info: Color support via LuaTeX's attributes
%
% Copyright (C)
%    2007, 2009-2011 Heiko Oberdiek
%    2016-2023 Oberdiek Package Support Group
%    https://github.com/ho-tex/luacolor/issues
%
% This work may be distributed and/or modified under the
% conditions of the LaTeX Project Public License, either
% version 1.3c of this license or (at your option) any later
% version. This version of this license is in
%    https://www.latex-project.org/lppl/lppl-1-3c.txt
% and the latest version of this license is in
%    https://www.latex-project.org/lppl.txt
% and version 1.3 or later is part of all distributions of
% LaTeX version 2005/12/01 or later.
%
% This work has the LPPL maintenance status "maintained".
%
% The Current Maintainers of this work are
% Heiko Oberdiek and the Oberdiek Package Support Group
% https://github.com/ho-tex/luacolor/issues
%
% This work consists of the main source file luacolor.dtx
% and the derived files
%    luacolor.sty, luacolor.pdf, luacolor.lua,
%
% Distribution:
%    CTAN:macros/latex/contrib/luacolor/luacolor.dtx
%    CTAN:macros/latex/contrib/luacolor/luacolor.pdf
%
% Unpacking:
%    (a) If luacolor.ins is present:
%           tex luacolor.ins
%    (b) Without luacolor.ins:
%           tex luacolor.dtx
%    (c) If you insist on using LaTeX
%           latex \let\install=y\input{luacolor.dtx}
%        (quote the arguments according to the demands of your shell)
%
% Documentation:
%    (a) If luacolor.drv is present:
%           latex luacolor.drv
%    (b) Without luacolor.drv:
%           latex luacolor.dtx; ...
%    The class ltxdoc loads the configuration file ltxdoc.cfg
%    if available. Here you can specify further options, e.g.
%    use A4 as paper format:
%       \PassOptionsToClass{a4paper}{article}
%
%    Programm calls to get the documentation (example):
%       pdflatex luacolor.dtx
%       makeindex -s gind.ist luacolor.idx
%       pdflatex luacolor.dtx
%       makeindex -s gind.ist luacolor.idx
%       pdflatex luacolor.dtx
%
% Installation:
%    TDS:tex/latex/luacolor/luacolor.sty
%    TDS:scripts/luacolor/luacolor.lua
%    TDS:doc/latex/luacolor/luacolor.pdf
%    TDS:source/latex/luacolor/luacolor.dtx
%
%<*ignore>
\begingroup
  \catcode123=1 %
  \catcode125=2 %
  \def\x{LaTeX2e}%
\expandafter\endgroup
\ifcase 0\ifx\install y1\fi\expandafter
         \ifx\csname processbatchFile\endcsname\relax\else1\fi
         \ifx\fmtname\x\else 1\fi\relax
\else\csname fi\endcsname
%</ignore>
%<*install>
\input docstrip.tex
\Msg{************************************************************************}
\Msg{* Installation}
\Msg{* Package: luacolor 2023-08-18 v1.18 Color support via LuaTeX's attributes (HO)}
\Msg{************************************************************************}

\keepsilent
\askforoverwritefalse

\let\MetaPrefix\relax
\preamble

This is a generated file.

Project: luacolor
Version: 2023-08-18 v1.18

Copyright (C)
   2007, 2009-2011 Heiko Oberdiek
   2016-2023 Oberdiek Package Support Group

This work may be distributed and/or modified under the
conditions of the LaTeX Project Public License, either
version 1.3c of this license or (at your option) any later
version. This version of this license is in
   https://www.latex-project.org/lppl/lppl-1-3c.txt
and the latest version of this license is in
   https://www.latex-project.org/lppl.txt
and version 1.3 or later is part of all distributions of
LaTeX version 2005/12/01 or later.

This work has the LPPL maintenance status "maintained".

The Current Maintainers of this work are
Heiko Oberdiek and the Oberdiek Package Support Group
https://github.com/ho-tex/luacolor/issues


This work consists of the main source file luacolor.dtx
and the derived files
   luacolor.sty, luacolor.pdf, luacolor.ins, luacolor.drv,
   luacolor.lua,

\endpreamble
\let\MetaPrefix\DoubleperCent

\generate{%
  \file{luacolor.ins}{\from{luacolor.dtx}{install}}%
  \file{luacolor.drv}{\from{luacolor.dtx}{driver}}%
  \usedir{tex/latex/luacolor}%
  \file{luacolor.sty}{\from{luacolor.dtx}{package}}%
}
\def\MetaPrefix{-- }
\def\defaultpostamble{%
  \MetaPrefix^^J%
  \MetaPrefix\space End of File `\outFileName'.%
}
\def\currentpostamble{\defaultpostamble}%
\generate{%
  \usedir{scripts/luacolor}%
%  \file{oberdiek.luacolor.lua}{\from{luacolor.dtx}{lua}}%
  \file{luacolor.lua}{\from{luacolor.dtx}{lua}}%
}

\catcode32=13\relax% active space
\let =\space%
\Msg{************************************************************************}
\Msg{*}
\Msg{* To finish the installation you have to move the following}
\Msg{* file into a directory searched by TeX:}
\Msg{*}
\Msg{*     luacolor.sty}
\Msg{*}
\Msg{* And install the following script file:}
\Msg{*}
\Msg{*     luacolor.lua,}
\Msg{*}
\Msg{* To produce the documentation run the file `luacolor.dtx'}
\Msg{* through LaTeX.}
\Msg{*}
\Msg{* Happy TeXing!}
\Msg{*}
\Msg{************************************************************************}

\endbatchfile
%</install>
%<*ignore>
\fi
%</ignore>
%<*driver>
\NeedsTeXFormat{LaTeX2e}
\ProvidesFile{luacolor.drv}%
  [2023-08-18 v1.18 Color support via LuaTeX's attributes (HO)]%
\documentclass{ltxdoc}
\usepackage{holtxdoc}[2011/11/22]
\usepackage{paralist}
\makeatletter
\g@addto@macro\MakePrivateLetters{%
  \@makeother\_%
}
\makeatother
\begin{document}
  \DocInput{luacolor.dtx}%
\end{document}
%</driver>
% \fi
%
%
%
% \GetFileInfo{luacolor.drv}
%
% \title{The \xpackage{luacolor} package}
% \date{2023-08-18 v1.18}
% \author{Heiko Oberdiek\thanks
% {Please report any issues at \url{https://github.com/ho-tex/luacolor/issues}}}
%
% \maketitle
%
% \begin{abstract}
% Package \xpackage{luacolor} implements color support based
% on \LuaTeX's node attributes.
% \end{abstract}
%
% \tableofcontents
%
% \section{Documentation}
%
% \subsection{Introduction}
%
% This package uses a \LuaTeX's attribute register to
% to annotate nodes with color information.
% If a color is set, then the attribute register
% is set to this color and all nodes created in its scope
% (current group) are annotated with this attribute.
% Now the color property behaves much the same way
% as the font property.
%
% \subsection{Usage}
%
% Package \xpackage{color} is loaded automatically by this
% package \xpackage{luacolor}. If you need a special driver
% option or you prefer package \xpackage{xcolor}, then load
% it before package \xpackage{luacolor}, for example:
% \begin{quote}
%   |\usepackage[dvipdfmx]{xcolor}|
% \end{quote}
% The package \xpackage{luacolor} is loaded without options:
% \begin{quote}
%   |\usepackage{luacolor}|
% \end{quote}
% It is able to detect PDF mode and DVI drivers are
% differentiated by its color specials. Therefore the
% package do need driver options.
%
% Then it redefines the color setting commands to set
% attributes instead of whatsits for color.
%
% At last the attribute annotations of the nodes in the
% output box must be analyzed to insert the necessary color whatsits.
% (With older \LuaTeX\ that lack the appropriate callback function 
% the package \xpackage{atbegshi} is used to get control
% before a box is shipped out.)
%
% \begin{declcs}{luacolorProcessBox} \M{box}
% \end{declcs}
% Macro \cs{luacolorProcessBox} processes the box \meta{box}
% in the previously described manner. It is automatically
% called for pages, but not for XForm objects.
% Before passing a box to \cs{pdfxform}, call \cs{luacolorProcessBox}
% first.
%
% \subsection{Limitations}
%
% \begin{description}
% \item[Ligatures with different colored components:]
%   Package \xpackage{luacolor} sees the ligature after
%   the paragraph building and page breaking, when a
%   page is to be shipped out. Therefore it cannot
%   break ligatures, because the components might occupy
%   different space. Therefore it is the responsibility
%   of the ligature forming process to deal with different
%   colored glyphs that form a ligature. The user can avoid
%   the problem entirely by explicitly breaking the ligature
%   at the places where the color changes.
%  \item \dots
% \end{description}
%
% \StopEventually{
% }
%
% \section{Implementation}
%
%    \begin{macrocode}
%<*package>
%    \end{macrocode}
%
% \subsection{Catcodes and identification}
%
%    \begin{macrocode}
\begingroup\catcode61\catcode48\catcode32=10\relax%
  \catcode13=5 % ^^M
  \endlinechar=13 %
  \catcode123=1 % {
  \catcode125=2 % }
  \catcode64=11 % @
  \def\x{\endgroup
    \expandafter\edef\csname LuaCol@AtEnd\endcsname{%
      \endlinechar=\the\endlinechar\relax
      \catcode13=\the\catcode13\relax
      \catcode32=\the\catcode32\relax
      \catcode35=\the\catcode35\relax
      \catcode61=\the\catcode61\relax
      \catcode64=\the\catcode64\relax
      \catcode123=\the\catcode123\relax
      \catcode125=\the\catcode125\relax
    }%
  }%
\x\catcode61\catcode48\catcode32=10\relax%
\catcode13=5 % ^^M
\endlinechar=13 %
\catcode35=6 % #
\catcode64=11 % @
\catcode123=1 % {
\catcode125=2 % }
\def\TMP@EnsureCode#1#2{%
  \edef\LuaCol@AtEnd{%
    \LuaCol@AtEnd
    \catcode#1=\the\catcode#1\relax
  }%
  \catcode#1=#2\relax
}
\TMP@EnsureCode{34}{12}% "
\TMP@EnsureCode{39}{12}% '
\TMP@EnsureCode{40}{12}% (
\TMP@EnsureCode{41}{12}% )
\TMP@EnsureCode{42}{12}% *
\TMP@EnsureCode{43}{12}% +
\TMP@EnsureCode{44}{12}% ,
\TMP@EnsureCode{45}{12}% -
\TMP@EnsureCode{46}{12}% .
\TMP@EnsureCode{47}{12}% /
\TMP@EnsureCode{58}{12}% :
\TMP@EnsureCode{60}{12}% <
\TMP@EnsureCode{62}{12}% >
\TMP@EnsureCode{91}{12}% [
\TMP@EnsureCode{93}{12}% ]
\TMP@EnsureCode{95}{12}% _ (other!)
\TMP@EnsureCode{96}{12}% `
\edef\LuaCol@AtEnd{\LuaCol@AtEnd\noexpand\endinput}
%    \end{macrocode}
%
%    Package identification.
%    \begin{macrocode}
\NeedsTeXFormat{LaTeX2e}
\ProvidesPackage{luacolor}%
  [2023-08-18 v1.18 Color support via LuaTeX's attributes (HO)]
%    \end{macrocode}
%
% \subsection{Check for \LuaTeX}
%
%    Without \LuaTeX\ there is no point in using this package.
%    \begin{macrocode}
\RequirePackage{color}
%    \end{macrocode}
%
%    \begin{macrocode}
\ifx\directlua\@undefined
  \PackageError{luacolor}{%
    This package may only be run using LuaTeX%
  }\@ehc
  \expandafter\LuaCol@AtEnd
\fi%
%    \end{macrocode}
%
% \subsection{Check for disabled colors}
%
%    \begin{macrocode}
\ifcolors@
\else
  \PackageWarningNoLine{luacolor}{%
    Colors are disabled by option `monochrome'%
  }%
  \def\set@color{}%
  \def\reset@color{}%
  \def\set@page@color{}%
  \def\define@color#1#2{}%
  \expandafter\LuaCol@AtEnd
\fi%
%    \end{macrocode}
%
% \subsection{Load module and check version}
%
%    \begin{macrocode}
\directlua{%
  require("luacolor")%
}
%    \end{macrocode}
%    \begin{macrocode}
\begingroup
  \edef\x{\directlua{tex.write("2023-08-18 v1.18")}}%
  \edef\y{%
    \directlua{%
      if oberdiek.luacolor.getversion then %
        oberdiek.luacolor.getversion()%
      end%
    }%
  }%
  \ifx\x\y
  \else
    \PackageError{luacolor}{%
      Wrong version of lua module.\MessageBreak
      Package version: \x\MessageBreak
      Lua module: \y
    }\@ehc
  \fi
\endgroup
%    \end{macrocode}
%
% \subsection{Find driver}
%
%    \begin{macrocode}
\ifnum\outputmode=\@ne
\else
  \begingroup
    \def\current@color{}%
    \def\reset@color{}%
    \setbox\z@=\hbox{%
      \begingroup
        \set@color
      \endgroup
    }%
    \edef\reserved@a{%
      \directlua{%
        oberdiek.luacolor.dvidetect()%
      }%
    }%
    \ifx\reserved@a\@empty
      \PackageError{luacolor}{%
        DVI driver detection failed because of\MessageBreak
        unrecognized color \string\special
      }\@ehc
      \endgroup
      \expandafter\expandafter\expandafter\LuaCol@AtEnd
    \else
      \PackageInfo{luacolor}{%
        Type of color \string\special: \reserved@a
      \@gobble}%
    \fi%
  \endgroup
\fi
%    \end{macrocode}
%
% \subsection{Attribute setting}
%
%    \begin{macro}{\LuaCol@Attribute}
%    \begin{macrocode}
\newattribute\LuaCol@Attribute
\let\LuaCol@setattribute\setattribute
\directlua{%
  oberdiek.luacolor.setattribute(\number\allocationnumber)%
}
%    \end{macrocode}
%    \end{macro}
%
%    \begin{macro}{\set@color}
% change 2023-08-18: added \cs{reset@color} so that \cs{mathcolor} can gobble it,
% issue 7.   
%    \begin{macrocode}
\protected\def\set@color{%
  \LuaCol@setattribute\LuaCol@Attribute{%
    \directlua{%
      oberdiek.luacolor.get("\luaescapestring{\current@color}")%
    }%
  }%
 \aftergroup\reset@color 
}
%    \end{macrocode}
%    \end{macro}
%    \begin{macro}{\reset@color}
%    \begin{macrocode}
\def\reset@color{}
%    \end{macrocode}
%    \end{macro}
%
% \subsection{Whatsit insertion}
%
%    \begin{macro}{\luacolorProcessBox}
%    \begin{macrocode}
\def\luacolorProcessBox#1{%
  \directlua{%
    oberdiek.luacolor.process(\number#1)%
  }%
}
%    \end{macrocode}
%    \end{macro}
%
%    \begin{macrocode}
\directlua{%
  if luatexbase.callbacktypes.pre_shipout_filter then
    token.get_next()
  end
}\@secondoftwo\@gobble{
  \RequirePackage{atbegshi}[2011/01/30]
  \AtBeginShipout{%
    \luacolorProcessBox\AtBeginShipoutBox
  }
}
%    \end{macrocode}
%
%    Set default color.
%    \begin{macrocode}
\set@color
%    \end{macrocode}
%
% \subsection{\cs{pdfxform}/\cs{saveboxresource} support}
%
%    \begin{macrocode}
\ifnum\outputmode=\@ne
    \let\LuaCol@org@pdfxform\saveboxresource
%    \end{macrocode}
% First we need some helpers to allow expandable code to parse keyword style arguments:
%    \begin{macrocode}
    \def\LuaCol@iii@i@ii#1#2#3{#3{#1}{#2}}
    \def\LuaCol@ii@i#1#2{{#2#1}}
    \def\LuaCol@if@keyword#1#2#3{%
      \expanded{\unexpanded{\LuaCol@iii@i@ii{#2}{#3}}\expandafter}%
      \directlua{%
        token.put_next(token.create(token.scan_keyword(token.scan_string())
        and '@firstoftwo'
        or '@secondoftwo'))
      }{#1}%
    }
%    \end{macrocode}
% The following macro scans a integer and expands to a token equivalent to a chardef
% whose value corresponds to the scanned integer. This allows the integer to be passed
% around as a undelimited argument.
%    \begin{macrocode}
    \def\LuaCol@scan@number{%
      \directlua{
        token.put_next(token.new(token.scan_int(), token.command_id'char_given'))
      }%
    }
%    \end{macrocode}
% \TeX\ primitives like \cs{saveboxresource} read braced arguments in a special way.
% Especially they expand everything until they find a left brace. To simulate this, we
% use Lua to expand everything else:
%    \begin{macrocode}
    \def\LuaCol@scan@tobrace{%
      \directlua{
        local relax, space = token.command_id'relax', token.command_id'spacer'
        local t
        repeat
          t = token.scan_token()
        until not (t.command == relax or t.command == space)
        token.put_next(t)
      }%
    }
    \def\LuaCol@scan@boxresource@i#1#2{%
      \LuaCol@if@keyword{attr}{%
        \expanded{\unexpanded{\LuaCol@scan@boxresource@iI{#1#2attr}}%
          \expandafter\expandafter\expandafter}%
        \LuaCol@scan@tobrace
      }{%
        \LuaCol@scan@boxresource@ii{#1#2}%
      }%
    }
    \def\LuaCol@scan@boxresource@iI#1#2{\LuaCol@scan@boxresource@ii{#1{#2}}}
    \def\LuaCol@scan@boxresource@ii#1{%
      \LuaCol@if@keyword{resources}{%
        \expanded{\unexpanded{\LuaCol@scan@boxresource@iiI{#1resources}}%
          \expandafter\expandafter\expandafter}%
        \LuaCol@scan@tobrace
      }{%
        \LuaCol@scan@boxresource@iii{#1}%
      }%
    }
    \def\LuaCol@scan@boxresource@iiI#1#2{\LuaCol@scan@boxresource@iii{#1{#2}}}
    \def\LuaCol@scan@boxresource@iii#1{%
      \LuaCol@if@keyword{margin}{%
        \expanded{\unexpanded{\LuaCol@scan@boxresource@iv{#1margin }}%
          \expandafter\expandafter\expandafter}%
        \LuaCol@scan@number
      }{%
        \LuaCol@scan@boxresource@iv{#1}{}%
      }%
    }
    \def\LuaCol@scan@boxresource@iv#1#2{%
      \expanded{\unexpanded{\LuaCol@scan@boxresource@v{#1#2}}%
        \expandafter\expandafter\expandafter}%
      \LuaCol@scan@number
    }
    \def\LuaCol@scan@boxresource@v#1#2{%
      \luacolorProcessBox{#2}%
      \LuaCol@org@pdfxform#1#2%
    }

%    \end{macrocode}
% This could be written in Lua, but at least upto Lua\TeX~1.11, feeding back too many tokens
% from Lua to \TeX\ triggers a segmentation fault.
% This is written in Lua so the integer setting is expandable and does not interfere with
% a preceding \verb|\immediate|.
%    \begin{macrocode}
    \protected\def\saveboxresource{%
      \LuaCol@if@keyword{type}{%
        \expandafter
        \expanded{\unexpanded{\LuaCol@scan@boxresource@i{type }}%
          \expandafter\expandafter\expandafter}%
        \LuaCol@scan@number
      }{%
        \LuaCol@scan@boxresource@i{}{}%
      }%
    }
%    \end{macrocode}
%    Legacy alias.
%    \begin{macrocode}
    \let\pdfxform\saveboxresource
\fi
%    \end{macrocode}
%
%    \begin{macrocode}
\LuaCol@AtEnd%
%</package>
%    \end{macrocode}
%
% \subsection{Lua module}
%
%    \begin{macrocode}
%<*lua>
%    \end{macrocode}
%    Box zero contains a \cs{hbox} with the color \cs{special}.
%    That is analyzed to get the prefix for the color setting
%    \cs{special}.
%    \begin{macrocode}
oberdiek = oberdiek or {}
local luacolor = oberdiek.luacolor or {}
oberdiek.luacolor = luacolor
%    \end{macrocode}
%    \begin{macro}{getversion()}
%    \begin{macrocode}
function luacolor.getversion()
  tex.write("2023-08-18 v1.18")
end
%    \end{macrocode}
%    \end{macro}
%
% \subsubsection{Driver detection}
%
%    \begin{macrocode}
local ifpdf = tonumber(tex.outputmode or tex.pdfoutput) > 0
local prefix
local prefixes = {
  dvips   = "color ",
  dvipdfm = "pdf:sc ",
  truetex = "textcolor:",
  pctexps = "ps::",
}
local patterns = {
  ["^color "]            = "dvips",
  ["^pdf: *begincolor "] = "dvipdfm",
  ["^pdf: *bcolor "]     = "dvipdfm",
  ["^pdf: *bc "]         = "dvipdfm",
  ["^pdf: *setcolor "]   = "dvipdfm",
  ["^pdf: *scolor "]     = "dvipdfm",
  ["^pdf: *sc "]         = "dvipdfm",
  ["^textcolor:"]        = "truetex",
  ["^ps::"]              = "pctexps",
}
%    \end{macrocode}
%    \begin{macro}{info()}
%    \begin{macrocode}
local function info(msg, term)
  local target = "log"
  if term then
    target = "term and log"
  end
  texio.write_nl(target, "Package luacolor info: " .. msg .. ".")
  texio.write_nl(target, "")
end
%    \end{macrocode}
%    \end{macro}
%    \begin{macro}{dvidetect()}
%    \begin{macrocode}
function luacolor.dvidetect()
  local v = tex.box[0]
  assert(v.id == node.id("hlist"))
  for v in node.traverse_id(node.id("whatsit"), v.head) do
    if v and v.subtype == node.subtype("special") then
      local data = v.data
      for pattern, driver in pairs(patterns) do
        if string.find(data, pattern) then
          prefix = prefixes[driver]
          tex.write(driver)
          return
        end
      end
      info("\\special{" .. data .. "}", true)
      return
    end
  end
  info("Missing \\special", true)
end
%    \end{macrocode}
%    \end{macro}
%
% \subsubsection{Color strings}
%
%    \begin{macrocode}
local map = {
  n = 0,
}
%    \end{macrocode}
%    \begin{macro}{get()}
%    \begin{macrocode}
function luacolor.get(color)
  tex.write("" .. luacolor.getvalue(color))
end
%    \end{macrocode}
%    \end{macro}
%    \begin{macro}{getvalue()}
%    \begin{macrocode}
function luacolor.getvalue(color)
  local n = map[color]
  if not n then
    n = map.n + 1
    map.n = n
    map[n] = color
    map[color] = n
  end
  return n
end
%    \end{macrocode}
%    \end{macro}
%
% \subsubsection{Attribute register}
%
%    \begin{macro}{setattribute()}
%    \begin{macrocode}
local attribute
function luacolor.setattribute(attr)
  attribute = attr
end
%    \end{macrocode}
%    \end{macro}
%
%    \begin{macro}{getattribute()}
%    \begin{macrocode}
function luacolor.getattribute()
  return attribute
end
%    \end{macrocode}
%    \end{macro}
%
% \subsubsection{Whatsit insertion}
%
%    \begin{macrocode}
local LIST = 1
local LIST_LEADERS = 2
local LIST_DISC = 3
local COLOR = 4
local NOCOLOR = 5
local RULE = node.id("rule")
local node_types = {
  [node.id("hlist")] = LIST,
  [node.id("vlist")] = LIST,
  [node.id("rule")]  = COLOR,
  [node.id("glyph")] = COLOR,
  [node.id("disc")]  = LIST_DISC,
  [node.id("whatsit")] = {
    [node.subtype("pdf_colorstack")] =
      function(n)
        return n.stack == 0 and NOCOLOR or nil
      end,
    [node.subtype("special")] = COLOR,
    [node.subtype("pdf_literal")] = COLOR,
    [node.subtype("pdf_save")] = COLOR,
    [node.subtype("pdf_restore")] = COLOR, -- probably not needed
-- TODO (DPC)    [node.subtype("pdf_refximage")] = COLOR,
  },
  [node.id("glue")] =
    function(n)
      if n.subtype >= 100 then -- leaders
        if n.leader.id == RULE then
          return COLOR
        else
          return LIST_LEADERS
        end
      end
    end,
}
%    \end{macrocode}
%    \begin{macro}{get_type()}
%    \begin{macrocode}
local function get_type(n)
  local ret = node_types[n.id]
  if type(ret) == 'table' then
    ret = ret[n.subtype]
  end
  if type(ret) == 'function' then
    ret = ret(n)
  end
  return ret
end
%    \end{macrocode}
%    \end{macro}
%    \begin{macrocode}
local mode = 2 -- luatex.pdfliteral.direct
local WHATSIT = node.id("whatsit")
local SPECIAL = node.subtype("special")
local PDFLITERAL = node.subtype("pdf_literal")
local DRY_FALSE = false
local DRY_TRUE = true
%    \end{macrocode}
%    \begin{macro}{traverse()}
%    \begin{macrocode}
local function traverse(list, color, dry)
  if not list then
    return color
  end
  local head
  if get_type(list) == LIST then
    head = list.head
  elseif get_type(list) == LIST_DISC then
    head = list.replace
  else
    texio.write_nl("!!! Error: Wrong list type: " .. node.type(list.id))
    return color
  end
%<debug>texio.write_nl("traverse: " .. node.type(list.id))
  for n in node.traverse(head) do
%<debug>texio.write_nl("  node: " .. node.type(n.id))
    local t = get_type(n)
%<debug>texio.write_nl("TYPE "..tostring(t).. " "..tostring(node.type(node.getid(n))).." ".. tostring(node.getsubtype(n)))
    if t == LIST or t == LIST_DISC then
      color = traverse(n, color, dry)
    elseif t == LIST_LEADERS then
      local color_after = traverse(n.leader, color, DRY_TRUE)
      if color == color_after then
        traverse(n.leader, color, DRY_FALSE or dry)
      else
        traverse(n.leader, '', DRY_FALSE or dry)
%    \end{macrocode}
% The color status is unknown here, because the leader box
% will or will not be set.
%    \begin{macrocode}
        color = ''
      end
    elseif t == COLOR then
      local v = node.has_attribute(n, attribute)
      if v then
        local newColor = map[v]
        if newColor ~= color then
          color = newColor
          if dry == DRY_FALSE then
            local newNode
            if ifpdf then
              newNode = node.new(WHATSIT, PDFLITERAL)
              newNode.mode = mode
              newNode.data = color
            else
              newNode = node.new(WHATSIT, SPECIAL)
              newNode.data = prefix .. color
            end
            head = node.insert_before(head, n, newNode)
          end
        end
      end
    elseif t == NOCOLOR then
      color = ''
    end
  end
  if get_type(list) == LIST then
    list.head = head
  else
    list.replace = head
  end
  return color
end
%    \end{macrocode}
%    \end{macro}
%
%    \begin{macro}{process()}
%    \begin{macrocode}
function luacolor.process(box)
  local color = ""
  local list = tex.getbox(box)
  traverse(list, color, DRY_FALSE)
end

if luatexbase.callbacktypes.pre_shipout_filter then
  luatexbase.add_to_callback("pre_shipout_filter", function(list)
    traverse(list, "", DRY_FALSE)
    return true
  end, "luacolor.process")
end
%    \end{macrocode}
%    \end{macro}
% For recent versions of luaotfload, we can register a callback to
% control how coloring glyph is handled for the color feature.
%    \begin{macrocode}
if luaotfload.set_colorhandler then
  local set_attribute = node.direct.set_attribute
  luaotfload.set_colorhandler(function(head, n, color)
    set_attribute(n, attribute, luacolor.getvalue(color))
    return head, n
  end)
end
%    \end{macrocode}
%
%    \begin{macrocode}
%</lua>
%    \end{macrocode}
% \section{Installation}
%
% \subsection{Download}
%
% \paragraph{Package.} This package is available on
% CTAN\footnote{\CTANpkg{luacolor}}:
% \begin{description}
% \item[\CTAN{macros/latex/contrib/luacolor/luacolor.dtx}] The source file.
% \item[\CTAN{macros/latex/contrib/luacolor/luacolor.pdf}] Documentation.
% \end{description}
%
%
% \paragraph{Bundle.} All the packages of the bundle `luacolor'
% are also available in a TDS compliant ZIP archive. There
% the packages are already unpacked and the documentation files
% are generated. The files and directories obey the TDS standard.
% \begin{description}
% \item[\CTANinstall{install/macros/latex/contrib/luacolor.tds.zip}]
% \end{description}
% \emph{TDS} refers to the standard ``A Directory Structure
% for \TeX\ Files'' (\CTANpkg{tds}). Directories
% with \xfile{texmf} in their name are usually organized this way.
%
% \subsection{Bundle installation}
%
% \paragraph{Unpacking.} Unpack the \xfile{luacolor.tds.zip} in the
% TDS tree (also known as \xfile{texmf} tree) of your choice.
% Example (linux):
% \begin{quote}
%   |unzip luacolor.tds.zip -d ~/texmf|
% \end{quote}
%
% \paragraph{Script installation.}
% Check the directory \xfile{TDS:scripts/luacolor/} for
% scripts that need further installation steps.
%
% \subsection{Package installation}
%
% \paragraph{Unpacking.} The \xfile{.dtx} file is a self-extracting
% \docstrip\ archive. The files are extracted by running the
% \xfile{.dtx} through \plainTeX:
% \begin{quote}
%   \verb|tex luacolor.dtx|
% \end{quote}
%
% \paragraph{TDS.} Now the different files must be moved into
% the different directories in your installation TDS tree
% (also known as \xfile{texmf} tree):
% \begin{quote}
% \def\t{^^A
% \begin{tabular}{@{}>{\ttfamily}l@{ $\rightarrow$ }>{\ttfamily}l@{}}
%   luacolor.sty & tex/latex/luacolor/luacolor.sty\\
%   luacolor.lua & scripts/luacolor/luacolor.lua\\
%   luacolor.pdf & doc/latex/luacolor/luacolor.pdf\\
%   luacolor.dtx & source/latex/luacolor/luacolor.dtx\\
% \end{tabular}^^A
% }^^A
% \sbox0{\t}^^A
% \ifdim\wd0>\linewidth
%   \begingroup
%     \advance\linewidth by\leftmargin
%     \advance\linewidth by\rightmargin
%   \edef\x{\endgroup
%     \def\noexpand\lw{\the\linewidth}^^A
%   }\x
%   \def\lwbox{^^A
%     \leavevmode
%     \hbox to \linewidth{^^A
%       \kern-\leftmargin\relax
%       \hss
%       \usebox0
%       \hss
%       \kern-\rightmargin\relax
%     }^^A
%   }^^A
%   \ifdim\wd0>\lw
%     \sbox0{\small\t}^^A
%     \ifdim\wd0>\linewidth
%       \ifdim\wd0>\lw
%         \sbox0{\footnotesize\t}^^A
%         \ifdim\wd0>\linewidth
%           \ifdim\wd0>\lw
%             \sbox0{\scriptsize\t}^^A
%             \ifdim\wd0>\linewidth
%               \ifdim\wd0>\lw
%                 \sbox0{\tiny\t}^^A
%                 \ifdim\wd0>\linewidth
%                   \lwbox
%                 \else
%                   \usebox0
%                 \fi
%               \else
%                 \lwbox
%               \fi
%             \else
%               \usebox0
%             \fi
%           \else
%             \lwbox
%           \fi
%         \else
%           \usebox0
%         \fi
%       \else
%         \lwbox
%       \fi
%     \else
%       \usebox0
%     \fi
%   \else
%     \lwbox
%   \fi
% \else
%   \usebox0
% \fi
% \end{quote}
% If you have a \xfile{docstrip.cfg} that configures and enables \docstrip's
% TDS installing feature, then some files can already be in the right
% place, see the documentation of \docstrip.
%
% \subsection{Refresh file name databases}
%
% If your \TeX~distribution
% (\TeX\,Live, \mikTeX, \dots) relies on file name databases, you must refresh
% these. For example, \TeX\,Live\ users run \verb|texhash| or
% \verb|mktexlsr|.
%
% \subsection{Some details for the interested}
%
% \paragraph{Unpacking with \LaTeX.}
% The \xfile{.dtx} chooses its action depending on the format:
% \begin{description}
% \item[\plainTeX:] Run \docstrip\ and extract the files.
% \item[\LaTeX:] Generate the documentation.
% \end{description}
% If you insist on using \LaTeX\ for \docstrip\ (really,
% \docstrip\ does not need \LaTeX), then inform the autodetect routine
% about your intention:
% \begin{quote}
%   \verb|latex \let\install=y\input{luacolor.dtx}|
% \end{quote}
% Do not forget to quote the argument according to the demands
% of your shell.
%
% \paragraph{Generating the documentation.}
% You can use both the \xfile{.dtx} or the \xfile{.drv} to generate
% the documentation. The process can be configured by the
% configuration file \xfile{ltxdoc.cfg}. For instance, put this
% line into this file, if you want to have A4 as paper format:
% \begin{quote}
%   \verb|\PassOptionsToClass{a4paper}{article}|
% \end{quote}
% An example follows how to generate the
% documentation with pdf\LaTeX:
% \begin{quote}
%\begin{verbatim}
%pdflatex luacolor.dtx
%makeindex -s gind.ist luacolor.idx
%pdflatex luacolor.dtx
%makeindex -s gind.ist luacolor.idx
%pdflatex luacolor.dtx
%\end{verbatim}
% \end{quote}
%
% \begin{History}
%   \begin{Version}{2007/12/12 v1.0}
%   \item
%     First public version.
%   \end{Version}
%   \begin{Version}{2009/04/10 v1.1}
%   \item
%     Fixes for changed syntax of \cs{directlua} in \LuaTeX\ 0.36.
%   \end{Version}
%   \begin{Version}{2010/03/09 v1.2}
%   \item
%     Adaptation for package \xpackage{luatex} 2010/03/09 v0.4.
%   \end{Version}
%   \begin{Version}{2010/12/13 v1.3}
%   \item
%     Support for \cs{pdfxform} added.
%   \item
%     Loaded package \xpackage{luatexbase-attr} recognized.
%   \item
%     Update for \hologo{LuaTeX}: `list' fields renamed to `head' in v0.65.0.
%   \end{Version}
%   \begin{Version}{2011/03/29 v1.4}
%   \item
%     Avoid whatsit insertion if option \xoption{monochrome} is used
%     (thanks Manuel P\'egouri\'e-Gonnard).
%   \end{Version}
%   \begin{Version}{2011/04/22 v1.5}
%   \item
%     Bug fix by Manuel P\'egouri\'e-Gonnard: A typo prevented the
%     detection of whatsits and applying color changes for
%     \cs{pdfliteral} and \cs{special} nodes that might contain
%     typesetting material.
%   \item
%     Bug fix by Manuel P\'egouri\'e-Gonnard: Now colors are also
%     applied to leader boxes.
%   \item
%     Unnecessary color settings are removed for leaders boxes,
%     if after the leader box the color has not changed.
%     The costs are a little runtime, leader boxes are processed twice.
%   \item
%     Additional whatsits that are colored: \texttt{pdf\_refximage}.
%   \item
%     Workaround for bug with \texttt{node.insert\_before}
%     removed for the version after \hologo{LuaTeX} 0.65, because
%     bug was fixed in 0.27. (Thanks Manuel P\'egouri\'e-Gonnard.)
%   \end{Version}
%   \begin{Version}{2011/04/23 v1.6}
%   \item
%     Bug fix for nested leader boxes.
%   \item
%     Bug fix for leader boxes that change color, but
%     are not set because of missing place.
%   \item
%     Version check for Lua module added.
%   \end{Version}
%   \begin{Version}{2011/10/22 v1.7}
%   \item
%     Lua functions \texttt{getattribute} and \texttt{getvalue}
%     added to tell other external Lua functions
%     the attribute register number for coloring.
%   \end{Version}
%   \begin{Version}{2011/11/01 v1.8}
%   \item
%     Use of \texttt{node.subtype} instead of magic numbers.
%   \end{Version}
%   \begin{Version}{2016/05/13 v1.9}
%   \item
%     More use of \texttt{node.subtype} instead of magic numbers.
%   \item luatex 85 updates
%   \end{Version}
%   \begin{Version}{2016/05/16 v1.10}
%   \item
%     Documentation updates.
%   \end{Version}
%   \begin{Version}{2018/11/22 v1.11}
%   \item
%     handle issue 43.
%   \item
%     removed pre-0.65 stuff
%   \end{Version}
%   \begin{Version}{2019/07/25 v1.12}
%   \item
%     removed uses of module function, see PR70
%   \end{Version}
%   \begin{Version}{2019/11/29 v1.13}
%   \item
%     Documentation updates.
%   \item  Use \xpackage{iftex} directly.
%   \end{Version}
%   \begin{Version}{2020-02-22 v1.14}
%   \item Drop use of \textsf{iftex} \textsf{ltxcmds} and \textsf{infwarerr}.
%   \item Assume \textsf{ltluatex} preloaded into format (true since 2015).
%   \item Patch \verb|\saveboxresource| rather than \verb|\pdfxform| (keep old name as alias).
%   \item Grab the number via Lua so that a \verb|\immediate| prefix still works with
%         \verb|\saveboxresource|\slash \verb|\pdfxform|.
%   \item Added handler for the color feature of luaotfload      
%   \end{Version}
%   \begin{Version}{2020-02-24 v1.15}
%   \item Grab all possible arguments for \verb|\saveboxresource|\slash \verb|\pdfxform|
%   \end{Version}
%   \begin{Version}{2020-04-04 v1.16}
%   \item Reset color after \verb|pdf_colorstack| whatsits.
%   \end{Version}
%   \begin{Version}{2021-02-17 v1.17}
%   \item Use \LaTeXe's new \verb|pre_shipout_filter| callback if it's available to
%         allow coloring background and foreground layer material
%   \end{Version}
%   \begin{Version}{2023-08-18 v1.18}
%   \item added \verb|\reset@color| to \verb|\set@color| for \verb|\mathcolor|, issue 7.
%   \end{Version}
% \end{History}
%
% \PrintIndex
%
% \Finale
\endinput
|
% \end{quote}
% Do not forget to quote the argument according to the demands
% of your shell.
%
% \paragraph{Generating the documentation.}
% You can use both the \xfile{.dtx} or the \xfile{.drv} to generate
% the documentation. The process can be configured by the
% configuration file \xfile{ltxdoc.cfg}. For instance, put this
% line into this file, if you want to have A4 as paper format:
% \begin{quote}
%   \verb|\PassOptionsToClass{a4paper}{article}|
% \end{quote}
% An example follows how to generate the
% documentation with pdf\LaTeX:
% \begin{quote}
%\begin{verbatim}
%pdflatex luacolor.dtx
%makeindex -s gind.ist luacolor.idx
%pdflatex luacolor.dtx
%makeindex -s gind.ist luacolor.idx
%pdflatex luacolor.dtx
%\end{verbatim}
% \end{quote}
%
% \begin{History}
%   \begin{Version}{2007/12/12 v1.0}
%   \item
%     First public version.
%   \end{Version}
%   \begin{Version}{2009/04/10 v1.1}
%   \item
%     Fixes for changed syntax of \cs{directlua} in \LuaTeX\ 0.36.
%   \end{Version}
%   \begin{Version}{2010/03/09 v1.2}
%   \item
%     Adaptation for package \xpackage{luatex} 2010/03/09 v0.4.
%   \end{Version}
%   \begin{Version}{2010/12/13 v1.3}
%   \item
%     Support for \cs{pdfxform} added.
%   \item
%     Loaded package \xpackage{luatexbase-attr} recognized.
%   \item
%     Update for \hologo{LuaTeX}: `list' fields renamed to `head' in v0.65.0.
%   \end{Version}
%   \begin{Version}{2011/03/29 v1.4}
%   \item
%     Avoid whatsit insertion if option \xoption{monochrome} is used
%     (thanks Manuel P\'egouri\'e-Gonnard).
%   \end{Version}
%   \begin{Version}{2011/04/22 v1.5}
%   \item
%     Bug fix by Manuel P\'egouri\'e-Gonnard: A typo prevented the
%     detection of whatsits and applying color changes for
%     \cs{pdfliteral} and \cs{special} nodes that might contain
%     typesetting material.
%   \item
%     Bug fix by Manuel P\'egouri\'e-Gonnard: Now colors are also
%     applied to leader boxes.
%   \item
%     Unnecessary color settings are removed for leaders boxes,
%     if after the leader box the color has not changed.
%     The costs are a little runtime, leader boxes are processed twice.
%   \item
%     Additional whatsits that are colored: \texttt{pdf\_refximage}.
%   \item
%     Workaround for bug with \texttt{node.insert\_before}
%     removed for the version after \hologo{LuaTeX} 0.65, because
%     bug was fixed in 0.27. (Thanks Manuel P\'egouri\'e-Gonnard.)
%   \end{Version}
%   \begin{Version}{2011/04/23 v1.6}
%   \item
%     Bug fix for nested leader boxes.
%   \item
%     Bug fix for leader boxes that change color, but
%     are not set because of missing place.
%   \item
%     Version check for Lua module added.
%   \end{Version}
%   \begin{Version}{2011/10/22 v1.7}
%   \item
%     Lua functions \texttt{getattribute} and \texttt{getvalue}
%     added to tell other external Lua functions
%     the attribute register number for coloring.
%   \end{Version}
%   \begin{Version}{2011/11/01 v1.8}
%   \item
%     Use of \texttt{node.subtype} instead of magic numbers.
%   \end{Version}
%   \begin{Version}{2016/05/13 v1.9}
%   \item
%     More use of \texttt{node.subtype} instead of magic numbers.
%   \item luatex 85 updates
%   \end{Version}
%   \begin{Version}{2016/05/16 v1.10}
%   \item
%     Documentation updates.
%   \end{Version}
%   \begin{Version}{2018/11/22 v1.11}
%   \item
%     handle issue 43.
%   \item
%     removed pre-0.65 stuff
%   \end{Version}
%   \begin{Version}{2019/07/25 v1.12}
%   \item
%     removed uses of module function, see PR70
%   \end{Version}
%   \begin{Version}{2019/11/29 v1.13}
%   \item
%     Documentation updates.
%   \item  Use \xpackage{iftex} directly.
%   \end{Version}
%   \begin{Version}{2020-02-22 v1.14}
%   \item Drop use of \textsf{iftex} \textsf{ltxcmds} and \textsf{infwarerr}.
%   \item Assume \textsf{ltluatex} preloaded into format (true since 2015).
%   \item Patch \verb|\saveboxresource| rather than \verb|\pdfxform| (keep old name as alias).
%   \item Grab the number via Lua so that a \verb|\immediate| prefix still works with
%         \verb|\saveboxresource|\slash \verb|\pdfxform|.
%   \item Added handler for the color feature of luaotfload      
%   \end{Version}
%   \begin{Version}{2020-02-24 v1.15}
%   \item Grab all possible arguments for \verb|\saveboxresource|\slash \verb|\pdfxform|
%   \end{Version}
%   \begin{Version}{2020-04-04 v1.16}
%   \item Reset color after \verb|pdf_colorstack| whatsits.
%   \end{Version}
%   \begin{Version}{2021-02-17 v1.17}
%   \item Use \LaTeXe's new \verb|pre_shipout_filter| callback if it's available to
%         allow coloring background and foreground layer material
%   \end{Version}
%   \begin{Version}{2023-08-18 v1.18}
%   \item added \verb|\reset@color| to \verb|\set@color| for \verb|\mathcolor|, issue 7.
%   \end{Version}
% \end{History}
%
% \PrintIndex
%
% \Finale
\endinput
|
% \end{quote}
% Do not forget to quote the argument according to the demands
% of your shell.
%
% \paragraph{Generating the documentation.}
% You can use both the \xfile{.dtx} or the \xfile{.drv} to generate
% the documentation. The process can be configured by the
% configuration file \xfile{ltxdoc.cfg}. For instance, put this
% line into this file, if you want to have A4 as paper format:
% \begin{quote}
%   \verb|\PassOptionsToClass{a4paper}{article}|
% \end{quote}
% An example follows how to generate the
% documentation with pdf\LaTeX:
% \begin{quote}
%\begin{verbatim}
%pdflatex luacolor.dtx
%makeindex -s gind.ist luacolor.idx
%pdflatex luacolor.dtx
%makeindex -s gind.ist luacolor.idx
%pdflatex luacolor.dtx
%\end{verbatim}
% \end{quote}
%
% \begin{History}
%   \begin{Version}{2007/12/12 v1.0}
%   \item
%     First public version.
%   \end{Version}
%   \begin{Version}{2009/04/10 v1.1}
%   \item
%     Fixes for changed syntax of \cs{directlua} in \LuaTeX\ 0.36.
%   \end{Version}
%   \begin{Version}{2010/03/09 v1.2}
%   \item
%     Adaptation for package \xpackage{luatex} 2010/03/09 v0.4.
%   \end{Version}
%   \begin{Version}{2010/12/13 v1.3}
%   \item
%     Support for \cs{pdfxform} added.
%   \item
%     Loaded package \xpackage{luatexbase-attr} recognized.
%   \item
%     Update for \hologo{LuaTeX}: `list' fields renamed to `head' in v0.65.0.
%   \end{Version}
%   \begin{Version}{2011/03/29 v1.4}
%   \item
%     Avoid whatsit insertion if option \xoption{monochrome} is used
%     (thanks Manuel P\'egouri\'e-Gonnard).
%   \end{Version}
%   \begin{Version}{2011/04/22 v1.5}
%   \item
%     Bug fix by Manuel P\'egouri\'e-Gonnard: A typo prevented the
%     detection of whatsits and applying color changes for
%     \cs{pdfliteral} and \cs{special} nodes that might contain
%     typesetting material.
%   \item
%     Bug fix by Manuel P\'egouri\'e-Gonnard: Now colors are also
%     applied to leader boxes.
%   \item
%     Unnecessary color settings are removed for leaders boxes,
%     if after the leader box the color has not changed.
%     The costs are a little runtime, leader boxes are processed twice.
%   \item
%     Additional whatsits that are colored: \texttt{pdf\_refximage}.
%   \item
%     Workaround for bug with \texttt{node.insert\_before}
%     removed for the version after \hologo{LuaTeX} 0.65, because
%     bug was fixed in 0.27. (Thanks Manuel P\'egouri\'e-Gonnard.)
%   \end{Version}
%   \begin{Version}{2011/04/23 v1.6}
%   \item
%     Bug fix for nested leader boxes.
%   \item
%     Bug fix for leader boxes that change color, but
%     are not set because of missing place.
%   \item
%     Version check for Lua module added.
%   \end{Version}
%   \begin{Version}{2011/10/22 v1.7}
%   \item
%     Lua functions \texttt{getattribute} and \texttt{getvalue}
%     added to tell other external Lua functions
%     the attribute register number for coloring.
%   \end{Version}
%   \begin{Version}{2011/11/01 v1.8}
%   \item
%     Use of \texttt{node.subtype} instead of magic numbers.
%   \end{Version}
%   \begin{Version}{2016/05/13 v1.9}
%   \item
%     More use of \texttt{node.subtype} instead of magic numbers.
%   \item luatex 85 updates
%   \end{Version}
%   \begin{Version}{2016/05/16 v1.10}
%   \item
%     Documentation updates.
%   \end{Version}
%   \begin{Version}{2018/11/22 v1.11}
%   \item
%     handle issue 43.
%   \item
%     removed pre-0.65 stuff
%   \end{Version}
%   \begin{Version}{2019/07/25 v1.12}
%   \item
%     removed uses of module function, see PR70
%   \end{Version}
%   \begin{Version}{2019/11/29 v1.13}
%   \item
%     Documentation updates.
%   \item  Use \xpackage{iftex} directly.
%   \end{Version}
%   \begin{Version}{2020-02-22 v1.14}
%   \item Drop use of \textsf{iftex} \textsf{ltxcmds} and \textsf{infwarerr}.
%   \item Assume \textsf{ltluatex} preloaded into format (true since 2015).
%   \item Patch \verb|\saveboxresource| rather than \verb|\pdfxform| (keep old name as alias).
%   \item Grab the number via Lua so that a \verb|\immediate| prefix still works with
%         \verb|\saveboxresource|\slash \verb|\pdfxform|.
%   \item Added handler for the color feature of luaotfload      
%   \end{Version}
%   \begin{Version}{2020-02-24 v1.15}
%   \item Grab all possible arguments for \verb|\saveboxresource|\slash \verb|\pdfxform|
%   \end{Version}
%   \begin{Version}{2020-04-04 v1.16}
%   \item Reset color after \verb|pdf_colorstack| whatsits.
%   \end{Version}
%   \begin{Version}{2021-02-17 v1.17}
%   \item Use \LaTeXe's new \verb|pre_shipout_filter| callback if it's available to
%         allow coloring background and foreground layer material
%   \end{Version}
%   \begin{Version}{2023-08-18 v1.18}
%   \item added \verb|\reset@color| to \verb|\set@color| for \verb|\mathcolor|, issue 7.
%   \end{Version}
% \end{History}
%
% \PrintIndex
%
% \Finale
\endinput
|
% \end{quote}
% Do not forget to quote the argument according to the demands
% of your shell.
%
% \paragraph{Generating the documentation.}
% You can use both the \xfile{.dtx} or the \xfile{.drv} to generate
% the documentation. The process can be configured by the
% configuration file \xfile{ltxdoc.cfg}. For instance, put this
% line into this file, if you want to have A4 as paper format:
% \begin{quote}
%   \verb|\PassOptionsToClass{a4paper}{article}|
% \end{quote}
% An example follows how to generate the
% documentation with pdf\LaTeX:
% \begin{quote}
%\begin{verbatim}
%pdflatex luacolor.dtx
%makeindex -s gind.ist luacolor.idx
%pdflatex luacolor.dtx
%makeindex -s gind.ist luacolor.idx
%pdflatex luacolor.dtx
%\end{verbatim}
% \end{quote}
%
% \begin{History}
%   \begin{Version}{2007/12/12 v1.0}
%   \item
%     First public version.
%   \end{Version}
%   \begin{Version}{2009/04/10 v1.1}
%   \item
%     Fixes for changed syntax of \cs{directlua} in \LuaTeX\ 0.36.
%   \end{Version}
%   \begin{Version}{2010/03/09 v1.2}
%   \item
%     Adaptation for package \xpackage{luatex} 2010/03/09 v0.4.
%   \end{Version}
%   \begin{Version}{2010/12/13 v1.3}
%   \item
%     Support for \cs{pdfxform} added.
%   \item
%     Loaded package \xpackage{luatexbase-attr} recognized.
%   \item
%     Update for \hologo{LuaTeX}: `list' fields renamed to `head' in v0.65.0.
%   \end{Version}
%   \begin{Version}{2011/03/29 v1.4}
%   \item
%     Avoid whatsit insertion if option \xoption{monochrome} is used
%     (thanks Manuel P\'egouri\'e-Gonnard).
%   \end{Version}
%   \begin{Version}{2011/04/22 v1.5}
%   \item
%     Bug fix by Manuel P\'egouri\'e-Gonnard: A typo prevented the
%     detection of whatsits and applying color changes for
%     \cs{pdfliteral} and \cs{special} nodes that might contain
%     typesetting material.
%   \item
%     Bug fix by Manuel P\'egouri\'e-Gonnard: Now colors are also
%     applied to leader boxes.
%   \item
%     Unnecessary color settings are removed for leaders boxes,
%     if after the leader box the color has not changed.
%     The costs are a little runtime, leader boxes are processed twice.
%   \item
%     Additional whatsits that are colored: \texttt{pdf\_refximage}.
%   \item
%     Workaround for bug with \texttt{node.insert\_before}
%     removed for the version after \hologo{LuaTeX} 0.65, because
%     bug was fixed in 0.27. (Thanks Manuel P\'egouri\'e-Gonnard.)
%   \end{Version}
%   \begin{Version}{2011/04/23 v1.6}
%   \item
%     Bug fix for nested leader boxes.
%   \item
%     Bug fix for leader boxes that change color, but
%     are not set because of missing place.
%   \item
%     Version check for Lua module added.
%   \end{Version}
%   \begin{Version}{2011/10/22 v1.7}
%   \item
%     Lua functions \texttt{getattribute} and \texttt{getvalue}
%     added to tell other external Lua functions
%     the attribute register number for coloring.
%   \end{Version}
%   \begin{Version}{2011/11/01 v1.8}
%   \item
%     Use of \texttt{node.subtype} instead of magic numbers.
%   \end{Version}
%   \begin{Version}{2016/05/13 v1.9}
%   \item
%     More use of \texttt{node.subtype} instead of magic numbers.
%   \item luatex 85 updates
%   \end{Version}
%   \begin{Version}{2016/05/16 v1.10}
%   \item
%     Documentation updates.
%   \end{Version}
%   \begin{Version}{2018/11/22 v1.11}
%   \item
%     handle issue 43.
%   \item
%     removed pre-0.65 stuff
%   \end{Version}
%   \begin{Version}{2019/07/25 v1.12}
%   \item
%     removed uses of module function, see PR70
%   \end{Version}
%   \begin{Version}{2019/11/29 v1.13}
%   \item
%     Documentation updates.
%   \item  Use \xpackage{iftex} directly.
%   \end{Version}
%   \begin{Version}{2020-02-22 v1.14}
%   \item Drop use of \textsf{iftex} \textsf{ltxcmds} and \textsf{infwarerr}.
%   \item Assume \textsf{ltluatex} preloaded into format (true since 2015).
%   \item Patch \verb|\saveboxresource| rather than \verb|\pdfxform| (keep old name as alias).
%   \item Grab the number via Lua so that a \verb|\immediate| prefix still works with
%         \verb|\saveboxresource|\slash \verb|\pdfxform|.
%   \item Added handler for the color feature of luaotfload      
%   \end{Version}
%   \begin{Version}{2020-02-24 v1.15}
%   \item Grab all possible arguments for \verb|\saveboxresource|\slash \verb|\pdfxform|
%   \end{Version}
%   \begin{Version}{2020-04-04 v1.16}
%   \item Reset color after \verb|pdf_colorstack| whatsits.
%   \end{Version}
%   \begin{Version}{2021-02-17 v1.17}
%   \item Use \LaTeXe's new \verb|pre_shipout_filter| callback if it's available to
%         allow coloring background and foreground layer material
%   \end{Version}
%   \begin{Version}{2023-08-18 v1.18}
%   \item added \verb|\reset@color| to \verb|\set@color| for \verb|\mathcolor|, issue 7.
%   \end{Version}
% \end{History}
%
% \PrintIndex
%
% \Finale
\endinput
