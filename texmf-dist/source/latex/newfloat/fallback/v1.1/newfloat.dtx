% \iffalse meta-comment
% 
% This is file `newfloat.dtx'.
% 
% Copyright (C) 2011-2021 Axel Sommerfeldt (axel.sommerfeldt@f-m.fm)
% 
% --------------------------------------------------------------------------
% 
% This work may be distributed and/or modified under the
% conditions of the LaTeX Project Public License, either version 1.3
% of this license or (at your option) any later version.
% The latest version of this license is in
%   http://www.latex-project.org/lppl.txt
% and version 1.3 or later is part of all distributions of LaTeX
% version 2003/12/01 or later.
% 
% This work has the LPPL maintenance status "unmaintained".
% 
% This work consists of the files newfloat.ins, newfloat.dtx,
% and the derived file newfloat.sty.
% 
% \fi
% \CheckSum{1634}
%
% \iffalse
%<*driver>
\NeedsTeXFormat{LaTeX2e}[1994/12/01]
\ProvidesFile{newfloat.drv}[2021/02/27 v1.1 This package offers a command for defining new floating environments]
\hbadness=9999 \newcount\hbadness \hfuzz=48pt % Make TeX shut up.
%
\documentclass{ltxdoc}
\setlength\parindent{0pt}
\setlength\parskip{\smallskipamount}
%
\usepackage{ifpdf}
\ifpdf
  \usepackage{mathptmx,courier}
  \usepackage[scaled=0.90]{helvet}
% \addtolength\marginparwidth{15pt}
\fi
%
\usepackage[bottom]{footmisc}
%
\usepackage{hypdoc}
\ifpdf\usepackage{hypdestopt}\fi
\hypersetup{pdfkeywords={LaTeX, package, newfloat},pdfstartpage={},pdfstartview={}}
%
\usepackage{newfloat}
%
\DeclareRobustCommand*\eTeX{\texorpdfstring
  {\leavevmode\hbox{$\varepsilon$}-\TeX}%
  {e-TeX}}
\DeclareRobustCommand*\AmS{\texorpdfstring
  {{\protect\usefont{OMS}{cmsy}{m}{n}A\kern-.1667em\lower.5ex\hbox{M}\kern-.125emS}}%
  {AMS}}
\DeclareRobustCommand*\KOMAScript{\texorpdfstring
  {\textsf{K\kern.05em O\kern.05em M\kern.05em A\kern.1em-\kern.1em Script}}%
  {KOMA-Script}}
\DeclareRobustCommand*\NTG{NTG}
\DeclareRobustCommand*\SmF{SMF}
%
\begin{document}
  \DocInput{newfloat.dtx}
\end{document}
%</driver>
% \fi
%
% \newcommand*\purerm[1]{{\upshape\mdseries\rmfamily #1}}
% \newcommand*\puresf[1]{{\upshape\mdseries\sffamily #1}}
% \newcommand*\purett[1]{{\upshape\mdseries\ttfamily #1}}
% \let\class\puresf \let\package\puresf
% \let\env\purett \let\opt\purett
%
% \def\thispackage{the \package{newfloat} package}
% \def\Thispackage{The \package{newfloat} package}
% 
% \newcommand*\csmarg[1]{\texttt{\char`\{#1\char`\}}}
% \newcommand*\csoarg[1]{\texttt{\char`\[#1\char`\]}}
% \newcommand*\version[2][]{\textit{v#2}}
%
% \setlength\leftmargini{1.5em}
%
% \GetFileInfo{newfloat.drv}
% \let\docdate\filedate
% \let\docversion\fileversion
% \GetFileInfo{newfloat.sty}
%
% \title{\texorpdfstring{\Thispackage\thanks{%^^A
%          This package has version number \docversion.}}%^^A
%        {The newfloat package}}
% \author{Axel Sommerfeldt\\
%         \url{https://gitlab.com/axelsommerfeldt/newfloat}}
% \date{\docdate}
% \maketitle
% 
% \begin{abstract}
% \centering
% This package offers the command
% |\Declare|\-|Floating|\-|Environment| for defining new floating environments
% which behave like |figure| and |table|.
% \end{abstract}
% 
% \tableofcontents
%
% \clearpage
% \section{Loading the package}
%
% First of all you need to include this package into your document with
% \begin{quote}
%   |\usepackage|\oarg{options}|{newfloat}|
% \end{quote}
% where \meta{options} are one or more of
% \begin{quote}\begin{tabular}{r@{\,}l}
%   |within=| & \meta{``within'' counter} \emph{or} \texttt{none}\quad\\
%   |chapterlistsgap=| & \meta{value}~\footnotemark[1]
% \end{tabular}\end{quote}
%
% The ``within'' counter specifies the counter which will be used to reset
% the counter of the floating environments |figure| and |table| .
% (Furthermore this setting will be used as default setting for
% |\Declare|\-|Floating|\-|Environment|.)
%
% So for example |within=chapter| will give you a numbering scheme
% \meta{chapter}.\meta{x} for floating environments,
% while |within=section| will give you a numbering scheme
% \meta{chapter}.\meta{section}.\meta{x}, or \meta{section}.\meta{x}
% if the document class does not offer |\chapter|.
% |within=none| will result in a continuous numbering throughout the document,
% i.e.~the numbering scheme will be simply \meta{x}.
%
% The option |chapterlistsgap=|\meta{value} sets the amount of the
% vertical gap inserted into the ``List of Figure'', ``List of Tables'',
% and all lists created with |\Declare|\-|Floating|\-|Environment|
% when a new chapter will be started.
% The default value is |10pt|.
% (This option will only be available if the document class used offer
% the usage of chapters, e.g.~the |book| or |report| document class.)
%
% \pagebreak[3]
% Both options can be changed later on, too, by using the command
% \begin{quote}
% |\newfloatsetup|\meta{options}\quad,
% \end{quote}
%
% \pagebreak[3]
% \section{\cs{DeclareFloatingEnvironment}}
%
% After loading \thispackage\ you can define your own floating environments with
% \begin{quote}
% \iffalse
% \setlength{\leftmargin}{6cm}
% \fi
%  |\DeclareFloatingEnvironment|\oarg{options}\marg{type}%
% \iffalse
%   \oarg{name}\oarg{list name}
% \fi
% \end{quote}
% where \meta{options} are one or more of
% \begin{quote}\begin{tabular}{r@{\,}l}
%   |fileext=| & \meta{file extension}\\
%   |listname=| & \meta{list name}\\
%   |name=| & \meta{prosa name}\\
%   |placement=| & \meta{combination of \,\purett{htbp}}\\
%   |within=| & \meta{``within'' counter} \emph{or} \texttt{none}\\
%   |chapterlistsgaps=| & \texttt{on} \emph{or} \texttt{off}~\footnotemark[1]
% \end{tabular}\end{quote}
% \footnotetext[1]{Please note that although \thispackage\ tries hard to offer this
% option it may not be available or is working incorrectly in cooperation with
% your document class since there is no standard interface which we could use
% for that purpose. Therefore we have to try to patch the \cs{chapter} command,
% and if this fails, we try to patch \cs{addtocontents} (in an unusual way) instead.}
%
% If no \meta{options} are given,
% ``lo\meta{type}'' will be used as \meta{file extension} for the list,
% ``List of \meta{name}s'' as \meta{list name}.
% ``\meta{name}'' as \meta{name} (but with the first letter capitalized),
% ``|tbp|'' as \meta{placement} specifier,
% and ``|chapter|'' resp. ``|none|'' as \meta{``within'' counter},
% i.e., the counter which resets the numbering.
%
% The default value of the |chapterlistsgaps=| option depends on the
% ``within'' setting, it is set to |on| if |chapter| or |section| is selected,
% otherwise it is set to |off|.
% (This option will only be available if the document class used offer
% the usage of chapters, e.g.~the |book| or |report| document class.)
%
% The list will be typeset using the command |\list|\-|of|\meta{type}|s|
% resp.~|\list|\-|of|\meta{type}|es|,
% analogous to |\list|\-|of|\-|figures| and |\list|\-|of|\-|tables|.
%
% If the \package{fltpage} package is loaded, an environment called
% |FP|\meta{type} will be defined additionally, same for |sideways|\meta{type}
% (\package{rotating} package), |SC|\meta{type} (\package{sidecap} package),
% and |wrap|\meta{type} (\package{wrapfig} package).
% 
% \bigskip
% 
% \pagebreak[3]
% So for example
% \begin{quote}
%   |\DeclareFloatingEnvironment{diagram}|
% \end{quote}
% will define a new floating environment called |diagram|,
% the list will be stored in a file with the extension |lodiagram|, the name
% (used for the caption) will be ``Diagram'' and the list name
% ``List of Diagrams''.
% The list could be typeset with |\list|\-|of|\-|diagrams|.
% Dependent on which packages are loaded, the environments |FP|\-|diagram|,
% |sideways|\-|diagram|, |SC|\-|diagram|, and |wrap|\-|diagram| will be defined additionally.
%
% \pagebreak[3]
% Another example:
% \begin{quote}
%   |\DeclareFloatingEnvironment[|\\
%   |  fileext=lox,|\\
%   |  listname={List of Matrixes},|\\
%   |  name=Matrix,|\\
%   |  placement=p,|\\
%   |  within=section,|\\
%   |  chapterlistsgaps=off|\\
%   |  ]{matrixx}|
% \end{quote}
% will define a new floating environment called |matrixx| with the given settings.
% Please note that names which contain spaces needs to be enclosed in curly braces.
%
% \pagebreak[3]
% \section{\cs{SetupFloatingEnvironment}}
%
% While |\DeclareFloatingEnvironment| will create new floating environments,
% \begin{quote}
% |\SetupFloatingEnvironment|\marg{floating environment}\marg{options}
% \end{quote}
% will change the settings of existing ones, i.e. either |figure| or |table|,
% or a one created with |\Declare|\-|Floating|\-|Environment|,
% or a one created with |\new|\-|float| offered by the \package{float} package,
% or a one created with |\new|\-|float| offered by the \package{memoir} document class,
% or a one created with |\Declare|\-|New|\-|Float|\-|Type| offered by the \package{floatrow} package,
% or \ldots
%
% The \meta{options} are the same as the options for
% |\Declare|\-|Floating|\-|Environment|,
% but one should avoid changing the file extension
% of existing floating environments,
% i.e.~using the |fileext=| option within |\Setup|\-|Floating|\-|Environment|
% is usually a very bad idea.
%
% \bigskip
%
% \pagebreak[3]
% An example:
% \begin{quote}
%   |\SetupFloatingEnvironment{lstlisting}{chapterlistsgaps=off}|
% \end{quote}
% will switch off the chapter lists gaps for |lstlisting| environments
% offered by the \package{listings} package.
%
% \pagebreak[3]
% \section{\cs{ForEachFloatingEnvironment}}
%
% \begin{quote}
%  |\ForEachFloatingEnvironment|\marg{code with \#1}
% \end{quote}
% will execute the given \meta{code} for all known floating environments,
% and for ones defined with |\Declare|\-|Floating|\-|Environment| later on.
%
% \pagebreak[3]
% So for example the \package{subcaption} packages uses
% \begin{quote}
%  |\ForEachFloatingEnvironment{\DeclareCaptionSubType{#1}}|
% \end{quote}
% for initializing itself for all floating environments which are
% known to \thispackage.
%
% \medskip
%
% There is also a starred variant |\For|\-|Each|\-|Floating|\-|Environment*|
% which will execute the given code for already existing floating environments
% only, i.e.~no hook will be placed inside |\Declare|\-|Floating|\-|Environment|.
%
% \pagebreak[3]
% An example:
% \begin{quote}
%  |\ForEachFloatingEnvironment*{\typeout{#1}}|
% \end{quote}
% will typeout the names of all already known floating environments to the terminal
% and log file.
%
% \pagebreak[3]
% \section{\cs{PrepareListOf}}
%
% Unfortunately there is no standardized way of typesetting and customizing
% lists of floating environments, i.e.~every document class offers it's own stuff here,
% and there are plenty of \LaTeXe\ packages for this purpose.
%
% \pagebreak[3]
% For that reason the \thispackage\ simply typesets the lists of newly
% defined floating environments as ``List of Figure'' internally (using |\list|\-|of|\-|figure|)
% but using a different list (the right one, of course) and using a different list name,
% the one you have specified with either |\Declare|\-|Floating|\-|Environment| or
% |\Setup|\-|Floating|\-|Environment|.\footnote{Please note that this is not the
% fact when using a \KOMAScript\ document class, not \cs{listoffigure} but \cs{listoftoc}
% will be used here, and therefore no problems should occur.}
%
% \pagebreak[3]
% While this usually results in the same layout as the ``List of Figure'' and
% is sufficient for most cases, there are some cases where this might
% lead to an unwanted side effect.
%
% \pagebreak[3]
% Therefore the \thispackage\ offers
% \begin{quote}
%  |\PrepareListOf|\marg{floating environment}\marg{code}
% \end{quote}
% which will execute the given \meta{code} right before actually typesetting the list
% with |\list|\-|of|\-|figures| internally.
%
% \pagebreak[3]
% One example:
% \begin{quote}
%   |\PrepareListOf{diagram}{%|\\
%   |  \renewcommand{\cftfigpresnum}{Diagram~}}|
% \end{quote}
% will change the part typeset in front of the diagram number within the
% ``List of Diagrams'' when using the \package{tocloft} package~\cite{tocloft}.
% Without the code above you will get the part defined for figures instead.
%
% \pagebreak[3]
% However, when dealing with the \package{tocloft} package it's a better
% idea not to use |\Prepare|\-|List|\-|Of| but the customization facilities of that package
% instead, e.g.:
% \begin{quote}
%   |\documentclass{book}|\\
%   \ldots\\
%   |\usepackage{newfloat,tocloft}|\\
%   |\newlistof{diagram}{lod}{List of Diagrams}|\\
%   |\DeclareFloatingEnvironment[name=Diagram,fileext=lod]{diagram}|\\
%   |      % Note: Due of a bug in tocloft the file extension|\\
%   |      % must be repeated at \DeclareFloatingEnvironment,|\\
%   |      % otherwise you will get an empty list.|\\
%   ||\\
%   |% Customize `List of Figures'|\\
%   |\renewcommand{\cftfigpresnum}{Figure~}|\\
%   |\setlength{\cftfignumwidth}{2cm}|\\
%   ||\\
%   |% Customize `List of Diagrams'|\\
%   |\renewcommand{\cftdiagrampresnum}{Diagram~}|\\
%   |\setlength{\cftdiagramnumwidth}{2cm}|\\
%   ||\\
%   \ldots\\
%   ||\\
%   |\begin{document}|\\
%   \ldots\\
%   |\listoffigures|\\
%   |\listofdiagrams|\\
%   \ldots\\
%   |\end{document}|
% \end{quote}
%
% \iffalse
% \pagebreak[3]
% \section{For \LaTeX{} package authors}
%
% |\For|\-|Each|\-|Floating|\-|Environment| without \package{newfloat} package:
% \begin{quote}
%   |\newcommand\mypackage@foreachfloat[1]{%|\\
%   |  \@ifundefined{ForEachFloatingEnvironment}%|\\
%   |    {\def\@elt##1{#1}%|\\
%   |      \@ifundefined{c@figure}\@gobble\@elt{figure}%|\\
%   |      \@ifundefined{c@table}\@gobble\@elt{table}%|\\
%   |      \let\@elt\relax|\\
%   |      \newfloat@addtohook{#1}}%|\\
%   |    {\ForEachFloatingEnvironment{#1}}}|\\
%   |\providecommand\newfloat@addtohook[1]{%|\\
%   |  \toks@=\expandafter{\newfloat@hook{##1}#1}%|\\
%   |  \edef\@tempa{\def\noexpand\newfloat@hook####1{\the\toks@}}%|\\
%   |  \@tempa}|\\
%   |\providecommand*\newfloat@hook[1]{}|
% \end{quote}
%
% \pagebreak[3]
% Adding environment to list of floating environments:
% \begin{quote}
%   |\SetupFloatingEnvironment|\marg{environment}|{}|
% \end{quote}
% or (without \package{newfloat} package loaded):
% \begin{quote}
%   |\providecommand\newfloat@list{}|\\
%   |\@cons\newfloat@list{|\marg{environment}|}|
% \end{quote}
% (Please note the doubled curly braces around the environment name.)
% \fi
%
% \iffalse
% --------------------------------------------------------------------------- %
% \fi
%
% \StopEventually{%^^A
% \pagebreak[4]
% \begin{thebibliography}{9}
%
%   \bibitem{memoir}
%   Peter Wilson:\\
%   \href{http://www.ctan.org/pkg/memoir}%
%        {\emph{The Memoir Class for Configurable Typesetting}},\\
%   2011/03/06
%
%   \bibitem{NTGclasses}
%   Victor Eijkhout:\\
%   \href{http://www.ctan.org/pkg/ntgclass}%
%        {\emph{An introduction to the Dutch \LaTeX\ document classes}},\\
%   \mbox{3~September~1989}
%
%   \bibitem{KOMAScript}
%   Markus Kohm \& Jens-Uwe-Morawski:\\
%   \href{http://www.ctan.org/pkg/koma-script}%
%        {\emph{KOMA-Script -- a versatile \LaTeXe\ bundle}},\\
%   2012-07-22
%
%   \bibitem{float}
%   Anselm Lingnau:\\
%   \href{http://www.ctan.org/pkg/float}%
%        {\emph{An Improved Environment for Floats}},\\
%   2001/11/08
%
%   \bibitem{fltpage}
%   Sebastian Gross:\\
%   \href{http://www.ctan.org/pkg/fltpage}%
%        {\emph{Welcome to the beta test of fltpage package!}},\\
%   1998/11/13
%
%   \bibitem{rotating}
%   Sebastian Rahtz and Leonor Barroca:\\
%   \href{http://www.ctan.org/pkg/rotating}%
%        {\emph{A style option for rotated objects in \LaTeX}},\\
%   1997/09/26
%
%   \bibitem{sidecap}
%   Rolf Niepraschk \& Hubert G\"a\ss lein:\\
%   \href{http://www.ctan.org/pkg/sidecap}%
%        {\emph{The sidecap package}},\\
%   2003/06/06
%
%   \bibitem{tocloft}
%   Peter Wilson, Herries Press, Will Robertson:\\
%   \href{http://www.ctan.org/pkg/tocloft}%
%        {\emph{The tocloft package}},\\
%   2010/10/13
%
%   \bibitem{wrapfig}
%   Donald Arseneau:\\
%   \href{http://www.ctan.org/pkg/wrapfig}%
%        {\emph{WRAPFIG.STY ver 3.6}},\\
%   2003/01/31
%
% \end{thebibliography}
% }
%
% \iffalse
% --------------------------------------------------------------------------- %
% \fi
%
% \DoNotIndex{\\,\_,\ ,\@@par}
% \DoNotIndex{\@bsphack}
% \DoNotIndex{\@car,\@cdr,\@classoptionslist,\@cons,\@currext,\@currname}
% \DoNotIndex{\@ehc,\@ehd,\@empty,\@esphack,\@expandtwoargs}
% \DoNotIndex{\@for,\@firstofone,\@firstoftwo}
% \DoNotIndex{\@gobble,\@gobbletwo,\@gobblethree,\@gobblefour,\@hangfrom}
% \DoNotIndex{\@ifnextchar,\@ifpackagelater,\@ifpackageloaded}
% \DoNotIndex{\@ifstar,\@ifundefined,\@latex@error,\@namedef,\@nameuse}
% \DoNotIndex{\@onlypreamble,\@parboxrestore,\@plus,\@ptionlist}
% \DoNotIndex{\@removeelement,\@restorepar,\@secondoftwo,\@setpar}
% \DoNotIndex{\@tempa,\@tempboxa,\@tempdima,\@tempdimb,\@tempdimc,\@tempb,\@tempc}
% \DoNotIndex{\@testopt}
% \DoNotIndex{\@undefined,\@unprocessedoptions,\@unusedoptionlist}
% \DoNotIndex{\p@,\z@}
% \DoNotIndex{\active,\addtocounter,\addtolength,\advance,\aftergroup}
% \DoNotIndex{\baselineskip,\begin,\begingroup,\bfseries,\box}
% \DoNotIndex{\catcode,\centering,\changes,\csname,\def,\divide,\do,\downarrow}
% \DoNotIndex{\edef,\else,\empty,\end,\endcsname,\endgraf,\endgroup,\expandafter}
% \DoNotIndex{\fi,\footnotesize,\global}
% \DoNotIndex{\hangindent,\hbox,\hfil,\hsize,\hskip,\hspace,\hss}
% \DoNotIndex{\ifcase,\ifdim,\ifnum,\ifodd,\ifvoid,\ifvmode}
% \DoNotIndex{\ifx,\ignorespaces,\itshape}
% \DoNotIndex{\Large,\large,\leavevmode,\leftmargini,\leftskip,\let,\linewidth}
% \DoNotIndex{\llap,\long,\m@ne,\margin,\mdseries,\message}
% \DoNotIndex{\newcommand,\newdimen,\newlength,\newline,\newif,\newsavebox}
% \DoNotIndex{\next,\nobreak,\nobreakspace,\noexpand,\noindent,\numberline}
% \DoNotIndex{\normalcolor,\normalfont,\normalsize,\or,\par,\parbox,\parfillskip}
% \DoNotIndex{\parindent,\parskip,\prevdepth,\protect,\protected@edef,\protected@write}
% \DoNotIndex{\providecommand,\quad}
% \DoNotIndex{\raggedleft,\raggedright,\relax,\renewcommand,\RequirePackage}
% \DoNotIndex{\rightskip,\rmfamily}
% \DoNotIndex{\sbox,\scriptsize,\scshape,\setbox,\setlength,\sffamily,\slshape}
% \DoNotIndex{\small,\string,\space,\strut}
% \DoNotIndex{\textheight,\the,\toks@,\typeout,\ttfamily}
% \DoNotIndex{\unvbox,\uparrow,\upshape,\usebox,\usepackage}
% \DoNotIndex{\value,\vbox,\vsize,\vskip,\wd,\width,\z@skip}
% \DoNotIndex{\AtBeginDocument,\AtEndOfPackage,\CurrentOption,\DeclareOption}
% \DoNotIndex{\ExecuteOptions,\GenericWarning,\IfFileExists,\InputIfFileExists}
% \DoNotIndex{\NeedsTeXFormat,\MessageBreak}
% \DoNotIndex{\PackageError,\PackageInfo,\PackageWarning,\PackageWarningNoLine}
% \DoNotIndex{\PassOptionsToPackage,\ProcessOptions,\ProvidesPackage}
%
% \iffalse
% --------------------------------------------------------------------------- %
% \fi
%
% \setlength{\parskip}{0pt plus 1pt}
% \changes{v1.0}{2011/10/30}{First release as \package{newfloat} package,
%       outsourced from the \package{caption3} package and revised}
% \changes{v1.1}{2013/04/27}{Adapted to the \package{tocloft} package}
% \changes{v1.1}{2013/04/27}{New command \cs{PrepareListOf} added}
% \changes{v1.1d}{2019/03/16}{Adapted to the \package{dgruyter} package}
%
% \iffalse
% --------------------------------------------------------------------------- %
% \fi
%
% \clearpage
% \section{The Implementation}
% \iffalse
%<*package>
% \fi
%
% \pagebreak[3]
% \subsection{Identification}
%
% We need at least \LaTeX2e\ version 1994/12/01.
%    \begin{macrocode}
\NeedsTeXFormat{LaTeX2e}[1994/12/01]
%    \end{macrocode}
%
% Bypass the release declarations in case the \LaTeX\ kernel doesn’t know how to deal with them
% (as suggested by \url{https://www.latex-project.org/publications/2018-FMi-TUB-tb122mitt-version-rollback.pdf}).
%    \begin{macrocode}
\providecommand\DeclareRelease[3]{}
\providecommand\DeclareCurrentRelease[2]{}
%    \end{macrocode}
%
% Declare all supported releases.
%    \begin{macrocode}
\DeclareRelease{v1.0}{2011-10-30}{newfloat_v1.0.sty}
\DeclareCurrentRelease{v1.1}{2013-05-12} % 1st CTAN release of v1.1
%    \end{macrocode}
%
% Identify the current version of the package.
%    \begin{macrocode}
\ProvidesPackage{newfloat}[2021/02/27 v1.1p Defining new floating environments (AR)]
%    \end{macrocode}
%
% \begin{macro}{\newfloat@Info}
% \begin{macro}{\newfloat@InfoNoLine}
%  |\newfloat@Info|\marg{message}
%    \begin{macrocode}
\newcommand*\newfloat@Info[1]{%
  \PackageInfo{newfloat}{#1}}
\newcommand*\newfloat@InfoNoLine[1]{%
  \newfloat@Info{#1\@gobble}}
%    \end{macrocode}
% \end{macro}
% \end{macro}
%
% \begin{macro}{\newfloat@Warning}
% \begin{macro}{\newfloat@WarningNoLine}
%  |\newfloat@Warning|\marg{message}
%    \begin{macrocode}
\newcommand*\newfloat@Warning[1]{%
  \PackageWarning{newfloat}{#1}}
\newcommand*\newfloat@WarningNoLine[1]{%
  \newfloat@Warning{#1\@gobble}}
%    \end{macrocode}
% \end{macro}
% \end{macro}
%
% \begin{macro}{\newfloat@Error}
%  |\newfloat@Error|\marg{message}
%    \begin{macrocode}
\newcommand*\newfloat@Error[1]{%
  \PackageError{newfloat}{#1}\newfloat@eh}
%    \end{macrocode}
%    \begin{macrocode}
\newcommand*\newfloat@eh{%
  If you do not understand this error, please take a closer look\MessageBreak
  at the documentation of the `newfloat' package.\MessageBreak\@ehc}
%    \end{macrocode}
% \end{macro}
%
% \pagebreak[3]
% \subsection{Using the keyval package}
%
% We need the \package{keyval} package for option handling, so we load it here.
%    \begin{macrocode}
\RequirePackage{keyval}[1997/11/10]
%    \end{macrocode}
%
% \pagebreak[3]
% \subsection{Helper macros}
%
% \begin{macro}{\newfloat@def}
% \changes{v1.1}{2013/04/27}{This macro added}
% |\newfloat@def|\marg{macro name}\marg{definition}
%    \begin{macrocode}
\newcommand*\newfloat@def[2]{%
  \newfloat@ifundefined{#1}{%
    \@namedef{#1}{#2}}}
%    \end{macrocode}
% \end{macro}
%
% \begin{macro}{\newfloat@let}
% \changes{v1.1}{2013/04/27}{This macro added}
% |\newfloat@let|\marg{macro name}\marg{macro}
%    \begin{macrocode}
\newcommand*\newfloat@let[2]{%
  \newfloat@ifundefined{#1}{%
    \expandafter\let\csname #1\endcsname#2}}
%    \end{macrocode}
% \end{macro}
%
% \begin{macro}{\newfloat@ifundefined}
% \changes{v1.1}{2013/04/27}{This macro added}
% |\newfloat@ifundefined|\marg{macro name}\marg{code}
%    \begin{macrocode}
\newcommand*\newfloat@ifundefined[2]{%
  \@ifundefined{#1}{#2}{%
    \newfloat@Info{%
      \expandafter\string\csname#1\endcsname\space is already defined}}}
%    \end{macrocode}
% \end{macro}
%
% \pagebreak[3]
% \subsection{\cs{DeclareFloatingEnvironment}}
%
% \begin{macro}{\DeclareFloatingEnvironment}
% \changes{v1.0a}{2011/11/16}{Bugfix: Undefines \cs{fst@}\meta{environment} now}
% \changes{v1.1}{2013/04/27}{Revised, only defines stuff yet undefined now}
% \changes{v1.1}{2013/11/03}{Uses \cs{newfloat@capitalize} now}
% \changes{v1.1}{2015/09/16}{Support of \cs{autodot} (offered by the \KOMAScript\ classes) added}
% \changes{v1.1}{2018/03/04}{Usage of \cs{ifdefined} replaced by \cs{@ifundefined} to improve compatibility with the \package{listings} package}
% \changes{v1.1b}{2018/12/26}{Bugfix: Missing curly braces added}
% \changes{v1.1c}{2019/01/10}{Bugfix: Support of \cs{autodot} fixed}
% \changes{v1.1k}{2019/09/01}{Adapted to the \class{beamer} document class}
%  |\DeclareFloatingEnvironment|\oarg{options}\marg{environment}\oarg{name}\oarg{list name}
%    \begin{macrocode}
\newcommand*\DeclareFloatingEnvironment{%
  \@testopt\@DeclareFloatingEnvironment{}}
\@onlypreamble\DeclareFloatingEnvironment
%    \end{macrocode}
%    \begin{macrocode}
\def\@DeclareFloatingEnvironment[#1]#2{%
  \newfloat@Info{New float `#2' with options `#1'}%
%    \end{macrocode}
% Define a counter with the same name as the floating environment
%    \begin{macrocode}
  \newfloat@ifundefined{c@#2}{\newcounter{#2}}%
%    \end{macrocode}
% Set |\ftype@|\meta{type} which contains the type number for floats of type \meta{type}\par
% (See also \url{http://tex.stackexchange.com/q/32359/2574})
%    \begin{macrocode}
  \@ifundefined{c@float@type}%     from float package
    {\@ifundefined{c@newflo@tctr}% from memoir document class
       {\ifdefined\newfloat@ftype \else
          \newcount\newfloat@ftype
          \newfloat@ftype=8\relax
        \fi
        \expandafter\xdef\csname ftype@#2\endcsname{\the\newfloat@ftype}%
        \advance\newfloat@ftype\newfloat@ftype}%
       {\expandafter\edef\csname ftype@#2\endcsname{\the\c@newflo@tctr}%
        \advance\c@newflo@tctr \c@newflo@tctr}}%
    {\expandafter\edef\csname ftype@#2\endcsname{\the\value{float@type}}%
     \addtocounter{float@type}{\value{float@type}}}%
%    \end{macrocode}
%    \begin{macrocode}
  \newfloat@Info{float type `#2'=\@nameuse{ftype@#2}}%
%    \end{macrocode}
% Define |\fnum@|\meta{type}, a macro to generate the figure number for a caption
%    \begin{macrocode}
  \newfloat@def{fnum@#2}%
   {\@nameuse{#2name}\nobreakspace\@nameuse{the#2}%
    \ifdefined\autodot\expandafter\autodot\fi}%
%    \end{macrocode}
% Define |\|\meta{type}|name| used by |\fnum@|\meta{type} as \meta{type},
% but with first letter capitalized
%    \begin{macrocode}
  \newfloat@capitalize\newfloat@Type{#2}%
  \newfloat@let{#2name}{\newfloat@Type}%
  \newfloat@def{fleg#2}{\@nameuse{#2name}}% legend naming (memoir)
%    \end{macrocode}
% Legend name in ToC (memoir document class)
%    \begin{macrocode}
  \newfloat@ifundefined{flegtoc#2}{\@namedef{flegtoc#2}##1{}}%
%    \end{macrocode}
% Define the floating environment\par
% If the \class{tufte} document class is loaded,
% |\@tufte@float| and |\end@tufte@float| will be used instead of
% |\@float|, |\end@float|, |\@dblfloat|, and |\end@dblfloat|.\par
% If the \class{beamer} document class is loaded,
% |\beamer@float| and |\end@beamer@float| will be used.
%    \begin{macrocode}
  \ifcsname @tufte@float\endcsname
    \newenvironment{#2}[1][htbp]%
      {\begin{@tufte@float}[{##1}]{#2}{}}%
      {\end{@tufte@float}}%
    \newenvironment{#2*}[1][htbp]%
      {\begin{@tufte@float}[{##1}]{#2}{star}}%
      {\end{@tufte@float}}%
  \else
    \@ifclassloaded{beamer}{%
      \newenvironment{#2}[1][]{\beamer@float{#2}}{\end@beamer@float}%
    }{%
      \newenvironment{#2}{\@float{#2}}{\end@float}%
      \newenvironment{#2*}{\@dblfloat{#2}}{\end@dblfloat}%
    }%
  \fi
%    \end{macrocode}
% Define the listing command |\listof|\meta{type}(|e|)|s|
% \changes{v1.1}{2013/04/27}{Definition of list hook added}
%    \begin{macrocode}
  \newfloat@def{listof#2}{\newfloat@listof{#2}}%
  \newfloat@def{listof#2s}{\@nameuse{listof#2}}%
  \newfloat@def{listof#2es}{\@nameuse{listof#2s}}%
  \newfloat@def{newfloat@listof#2@hook}{}%
  \ifdefined\l@figure
    \newfloat@let{l@#2}{\l@figure}%
  \else
    \newfloat@def{l@#2}{\@dottedtocline{1}{1.5em}{2.3em}}%
  \fi
  \edef\newfloat@tempa{List of \newfloat@Type s}%
  \newfloat@let{list#2name}{\newfloat@tempa}%
%    \end{macrocode}
% We undefine |\fst@|\meta{environment} so the new environment
% will not be recognized as defined by the \package{float} package.
% (|\fst@|\meta{type} use to hold the float style, see \package{float} package
% implementation for details.)
%    \begin{macrocode}
  \expandafter\let\csname fst@#2\endcsname\@undefined
%    \end{macrocode}
% Set default parameters
%    \begin{macrocode}
  \newfloat@ifundefined{fps@#2}{\newfloat@setplacement{#2}{tbp}}%
  \newfloat@ifundefined{ext@#2}{\newfloat@setfileext{#2}{lo#2}}%
%    \end{macrocode}
% Apply given options
%    \begin{macrocode}
  \newfloat@setoptions*{#2}{#1}%
%    \end{macrocode}
% Announce the new floating environment to other packages
%    \begin{macrocode}
  \@expandtwoargs\newfloat@announce{#2}{\@nameuse{ext@#2}}%
%    \end{macrocode}
% Apply the last two optional arguments for setting names
%    \begin{macrocode}
  \@ifnextchar[\newfloat@DFE@setname\relax}
\@onlypreamble\@DeclareFloatingEnvironment
%    \end{macrocode}
%    \begin{macrocode}
\def\newfloat@DFE@setname[#1]{%
  \KV@@newfloat@name{#1}%
  \@ifnextchar[\newfloat@DFE@setlistname\relax}
\@onlypreamble\newfloat@DFE@setname
%    \end{macrocode}
%    \begin{macrocode}
\def\newfloat@DFE@setlistname[#1]{%
  \KV@@newfloat@listname{#1}}
\@onlypreamble\newfloat@DFE@setlistname
%    \end{macrocode}
% \end{macro}
%
% \begin{macro}{\newfloat@capitalize}
% \changes{v1.1}{2013/11/03}{Code extracted from \cs{DeclareFloatingEnvironment}}
% \changes{v1.1}{2013/11/03}{Uses \cs{newfloat@tempa} instead of \cs{@tempa} now}
% |\newfloat@capitalize|\marg{command}\marg{name}\\
% Defines \meta{command} to \meta{name}, but with first letter capitalized.
%    \begin{macrocode}
\newcommand*\newfloat@capitalize[2]{%
  \edef\newfloat@tempa{\gdef\noexpand#1{\@car#2\@nil}}%
  \uppercase\expandafter{\newfloat@tempa}%
  \edef\newfloat@tempa{%
    \noexpand\g@addto@macro\noexpand#1{\@cdr#2\@nil}}%
  \newfloat@tempa}
%    \end{macrocode}
% \end{macro}
%
% \begin{macro}{\newfloat@listof}
% \changes{v1.1}{2013/04/27}{Revised}
% \changes{v1.1g}{2019/08/28}{Starred variant added}
% \changes{v1.1n}{2021/02/27}{Uses \cs{newfloat@ext} now instead of \cs{ext@xxx}}
% |\newfloat@listof|\marg{float type}\\
% typesets the list, using |\listoffigure| with patched list name and file extension.
%    \begin{macrocode}
\newcommand*\newfloat@listof[1]{%
  \@ifstar
    {\newfloat@list@of{#1}*}%
    {\newfloat@list@of{#1}\relax}}
%    \end{macrocode}
%    \begin{macrocode}
\newcommand*\newfloat@list@of[1]{%
  \@expandtwoargs\newfloat@list@of@{#1}{\newfloat@ext{#1}}}
%    \end{macrocode}
%    \begin{macrocode}
\newcommand*\newfloat@list@of@[3]{%
  \begingroup
    \expandafter\let\expandafter\listfigurename\csname list#1name\endcsname
    \def\ext@figure{#2}%
    \let\newfloat@starttoc\@starttoc
    \def\@starttoc##1{\newfloat@starttoc{#2}}%
    \let\newfloat@listoftoc\listoftoc
    \def\listoftoc##1{\newfloat@listoftoc{#2}}%
    \@nameuse{newfloat@listof#1@hook}%
    \listoffigures#3%
  \endgroup}
%    \end{macrocode}
% \end{macro}
%
% \begin{macro}{\newfloat@setoptions}
% |\newfloat@setoptions*|\marg{environment}\marg{options}
%    \begin{macrocode}
\newcommand*\newfloat@setoptions{%
  \@ifstar
    {\newfloat@@setoptions\@firstofone}%
    {\newfloat@@setoptions\@gobble}}
%    \end{macrocode}
%    \begin{macrocode}
\newcommand*\newfloat@@setoptions[3]{%
  \let\newfloat@within@value\@undefined
  \let\newfloat@chapterlistsgaps@value\@undefined
  #1{\KV@@newfloat@within\newfloat@within@default}% set default value for new floats
  \def\newfloat@type{#2}%
  \setkeys{@newfloat}{#3}%
  \ifx\newfloat@within@value\@undefined \else
    \newfloat@setoption{within}\newfloat@within@value
  \fi
  \ifx\newfloat@chapterlistsgaps@value\@undefined \else
    \newfloat@setoption{chapterlistsgaps}\newfloat@chapterlistsgaps@value
  \fi}
%    \end{macrocode}
% \end{macro}
%
% \begin{macro}{\newfloat@within@default}
%  The default `within' value.
%  This one will be used if no option |within=|\meta{counter} is given.
%    \begin{macrocode}
\newcommand*\newfloat@within@default{%
  \ifcsname c@chapter\endcsname chapter\else none\fi}
\@onlypreamble\newfloat@within@default
%    \end{macrocode}
% \end{macro}
%
% \begin{macro}{\newfloat@setoption}
% \changes{v1.1}{2015/09/15}{\cs{@expandtwoargs} replaced by own expansion so the option value do not get expanded anymore}
% \changes{v1.1m}{2020/10/11}{Usage of \cs{caption@tempa} replaced by \cs{newfloat@tempa}}
% |\newfloat@setoption|\marg{option name}{options}
%    \begin{macrocode}
\newcommand*\newfloat@setoption[1]{%
  \edef\newfloat@tempa{\noexpand\@nameuse{newfloat@set#1}{\newfloat@type}}%
  \newfloat@tempa}
%    \end{macrocode}
% \end{macro}
%
% The available \meta{options} are:
% |fileext=|\meta{file extension},
% |listname=|\meta{list name},
% |name=|\meta{prosa name},
% |placement=|\meta{htbp},
% |within=|\meta{none,chapter,section}, and
% |without|.
%
% \begin{macro}{\newfloat@setfileext}
% |\newfloat@setfileext|\marg{environment}\marg{file extension}
%    \begin{macrocode}
\newcommand*\newfloat@setfileext[2]{%
  \@namedef{ext@#1}{#2}}
%    \end{macrocode}
%    \begin{macrocode}
\define@key{@newfloat}{fileext}{%
  \newfloat@setoption{fileext}{#1}}
%    \end{macrocode}
% \end{macro}
%
% \begin{macro}{\newfloat@setlistname}
% |\newfloat@setlistname|\marg{environment}\marg{list name}
%    \begin{macrocode}
\newcommand*\newfloat@setlistname[2]{%
  \@namedef{list#1name}{#2}}
%    \end{macrocode}
%    \begin{macrocode}
\define@key{@newfloat}{listname}{%
  \newfloat@setoption{listname}{#1}}
%    \end{macrocode}
% \end{macro}
%
% \begin{macro}{\newfloat@setname}
% |\newfloat@setname|\marg{environment}\marg{name}
%    \begin{macrocode}
\newcommand*\newfloat@setname[2]{%
  \newfloat@@setname{#1}{#2}%
  \begingroup
    \ifcsname languagename\endcsname
      \ifcsname captions\languagename\endcsname
        \expandafter\g@addto@macro\csname captions\languagename\endcsname
          {\newfloat@@setname{#1}{#2}}%
      \fi
    \fi
  \endgroup}
%%\AtBeginDocument{\let\newfloat@setname\newfloat@@setname}
%    \end{macrocode}
%    \begin{macrocode}
\newcommand*\newfloat@@setname[2]{%
  \@namedef{#1name}{#2}}
%    \end{macrocode}
%    \begin{macrocode}
\define@key{@newfloat}{name}{%
  \newfloat@setoption{name}{#1}}
%    \end{macrocode}
% \end{macro}
%
% \begin{macro}{\newfloat@setplacement}
% |\newfloat@setplacement|\marg{environment}\marg{float placement}
%    \begin{macrocode}
\newcommand*\newfloat@setplacement[2]{%
  \@namedef{fps@#1}{#2}}
%    \end{macrocode}
%    \begin{macrocode}
\define@key{@newfloat}{placement}{%
  \newfloat@setoption{placement}{#1}}
%    \end{macrocode}
% \end{macro}
%
% \begin{macro}{\newfloat@setwithin}
% |\newfloat@setwithin|\marg{environment}\marg{counter}\par
% setup the counter for working ``within'' a given counter.
% Furthermore the chapters lists gap will be switched on
% (if counter $=$ |chapter|) or off (otherwise).
%    \begin{macrocode}
\newcommand*\newfloat@setwithin[2]{%
  \ifcsname c@chapter\endcsname
    \@removefromreset{#1}{chapter}%
  \fi
  \@removefromreset{#1}{section}%
%    \end{macrocode}
%    \begin{macrocode}
  \edef\@tempa{#2}%
  \ifx\@tempa\@empty
    \def\@tempa{none}%
  \fi
  \def\@tempb{none}%
  \ifx\@tempa\@tempb
    \ifcsname c@chapter\endcsname
      \@chapterlistsgap@off{#1}%
    \fi
    \newfloat@@setwithin{#1}{}{}%
  \else
    \def\@tempb{chapter}%
    \ifx\@tempa\@tempb
      \@addtoreset{#1}{chapter}%
      \@chapterlistsgap@on{#1}%
      \newfloat@@setwithin{#1}{\ifnum\c@chapter>\z@ \thechapter.\fi}{\theHchapter.}%
    \else
      \def\@tempb{section}%
      \ifx\@tempa\@tempb
        \@addtoreset{#1}{section}%
        \ifcsname c@chapter\endcsname
          \@addtoreset{#1}{chapter}%
          \@chapterlistsgap@on{#1}%
          \newfloat@@setwithin{#1}{\thesection.}{\theHsection.}%
        \else
          \newfloat@@setwithin{#1}{\ifnum\c@section>\z@ \thesection.\fi}{\theHsection.}%
        \fi
      \else
        \newfloat@Error{Invalid value `#2' for option `within'}%
      \fi
    \fi
  \fi}
%    \end{macrocode}
% Note: Since the \package{hyperref} package with version $<$ 6.82k defines
% |\the|\-|H|\-|figure| and |\the|\-|H|\-|table| with |\new|\-|command|,
% |\newfloat@@setwithin| will yield to an error message when using with an
% old version of \package{hyperref},
% e.g. ``\texttt{LaTeX Error: Command \cs{theHfigure} already defined}''.
%    \begin{macrocode}
\newcommand*\newfloat@@setwithin[3]{%
  \global\@namedef{the#1}{#2\arabic{#1}}%
  \global\@namedef{theH#1}{#3\arabic{#1}}}
%    \end{macrocode}
%    \begin{macrocode}
\define@key{@newfloat}{within}{%
  \def\newfloat@within@value{#1}}
%    \end{macrocode}
% \end{macro}
%
% \begin{macro}{\newfloat@setwithout}
% |\newfloat@setwithout|\marg{environment}
%    \begin{macrocode}
\newcommand*\newfloat@setwithout[1]{%
  \newfloat@setwithin{#1}{none}}
%    \end{macrocode}
%    \begin{macrocode}
\define@key{@newfloat}{without}[]{%
  \def\newfloat@within@value{none}}
%    \end{macrocode}
% \end{macro}
%
% \begin{macro}{\newfloat@setchapterlistsgaps}
% |\newfloat@setchapterlistsgaps|\marg{environment}\marg{on/off}
%    \begin{macrocode}
\newcommand*\newfloat@setchapterlistsgaps[2]{%
  \edef\@tempa{#2}%
  \newfloat@setchapterlistsgaps@{off}{off}{#1}%
  \newfloat@setchapterlistsgaps@{false}{off}{#1}%
  \newfloat@setchapterlistsgaps@{on}{on}{#1}%
  \newfloat@setchapterlistsgaps@{true}{on}{#1}%
  \ifx\@tempa\newfloat@setchapterlistsgaps@ \else
    \newfloat@Error{Invalid value `#2' for option `chapterlistsgaps'}%
  \fi}
%    \end{macrocode}
%    \begin{macrocode}
\newcommand*\newfloat@setchapterlistsgaps@[3]{%
  \def\@tempb{#1}%
  \ifx\@tempa\@tempb
    \newfloat@chapterlistsgap@{#2}{#3}%
    \let\@tempa\newfloat@setchapterlistsgaps@
  \fi}
%    \end{macrocode}
%    \begin{macrocode}
\newcommand*\newfloat@chapterlistsgap@[1]{%
  \@ifundefined{@chapterlistsgap@#1}%
    {\@gobble}%
    {\@nameuse{@chapterlistsgap@#1}}}
%    \end{macrocode}
%    \begin{macrocode}
\define@key{@newfloat}{chapterlistsgaps}{%
  \def\newfloat@chapterlistsgaps@value{#1}}
%    \end{macrocode}
% \end{macro}
%
% \begin{macro}{\@removefromreset}
% This code was taken from the \package{remreset} package
% which is part of the `carlisle' package bundle.
% (Copyright 1997 David Carlisle)
%    \begin{macrocode}
\providecommand*\@removefromreset[2]{{%
  \expandafter\let\csname c@#1\endcsname\@removefromreset
  \def\@elt##1{%
    \expandafter\ifx\csname c@##1\endcsname\@removefromreset
    \else
      \noexpand\@elt{##1}%
    \fi}%
  \expandafter\xdef\csname cl@#2\endcsname{%
    \csname cl@#2\endcsname}}}
%    \end{macrocode}
% \end{macro}
%
% \begin{macro}{\newfloat@announce}
% \changes{v1.1n}{2021/02/27}{Defines \cs{newfloat@env@xxx} now instead of \cs{newfloat@ext@xxx}}
% |\newfloat@announce|\marg{environment name}\marg{list file extension}
%    \begin{macrocode}
\newcommand*\newfloat@announce[2]{%
%    \end{macrocode}
%    \begin{macrocode}
  \@cons\newfloat@list{{#1}}%
  \@cons\newfloat@@list{{#1}}%
%    \end{macrocode}
%    \begin{macrocode}
  \newfloat@ifundefined{newfloat@env@#2}{%
    \@namedef{newfloat@env@#2}{#1}%
%    \end{macrocode}
% Support of \package{memoir} document class
%    \begin{macrocode}
    \ifcsname c@lofdepth\endcsname
      \newfloat@ifundefined{c@#2depth}{%
        \newcounter{#2depth}%
        \setcounter{#2depth}{1}}%
    \fi
%    \end{macrocode}
% Support of \package{tocbasic} package
%    \begin{macrocode}
    \ifcsname addtotoclist\endcsname
      \addtotoclist[float]{#2}%
      \newfloat@def{listof#2name}{\@nameuse{list#1name}}%
    \fi
%    \end{macrocode}
%    \begin{macrocode}
  }%
%    \end{macrocode}
% Support of \package{titletoc} package
%    \begin{macrocode}
  \ifcsname contentsuse\endcsname
    \contentsuse{#1}{#2}%
  \fi
%    \end{macrocode}
%    \begin{macrocode}
  \newfloat@hook{#1}}
\@onlypreamble\newfloat@announce
%    \end{macrocode}
% \end{macro}
%
% \begin{macro}{\newfloat@ext}
% \changes{v1.1n}{2021/02/27}{This macro added}
% |\newfloat@ext|\marg{environment}
% expands to the file extension used for the given \meta{environment}.
%    \begin{macrocode}
\newcommand*\newfloat@ext[1]{\@nameuse{ext@#1}}
%    \end{macrocode}
% \end{macro}
%
% \begin{macro}{\newfloat@@list}
% |\newfloat@@list| is an \cs{@elt}-list containing the floating environments defined
% with |\Declare|\-|Floating|\-|Environment| only.
%    \begin{macrocode}
\newcommand*\newfloat@@list{}
%    \end{macrocode}
% \end{macro}
%
% \pagebreak[3]
% \subsection{\cs{SetupFloatingEnvironment}}
%
% \begin{macro}{\SetupFloatingEnvironment}
% |\SetupFloatingEnvironment|\marg{environment}\marg{options}
%    \begin{macrocode}
\newcommand*\SetupFloatingEnvironment[1]{%
  \newfloat@addtolist{#1}%
  \newfloat@setoptions{#1}}
%    \end{macrocode}
% \end{macro}
%
% \pagebreak[3]
% \subsection{\cs{ForEachFloatingEnvironment}}
%
% \begin{macro}{\ForEachFloatingEnvironment}
% |\ForEachFloatingEnvironment|\marg{code with \#1}\par
% will execute the given code for each (known) floating environment.
% The starred variant will only work for already existing environments,
% i.e.~no hook will be placed inside |\Declare|\-|Floating|\-|Environment|.
%    \begin{macrocode}
\newcommand*\ForEachFloatingEnvironment{%
  \@ifstar
    {\newfloat@foreach\@gobble}%
    {\newfloat@foreach\@iden}}
%    \end{macrocode}
%    \begin{macrocode}
\newcommand\newfloat@foreach[2]{%
  \def\@elt##1{#2}%
  \newfloat@list
  \let\@elt\relax
  #1{\newfloat@addtohook{#2}}}
%    \end{macrocode}
% \end{macro}
%
% \begin{macro}{\newfloat@addtohook}
% |\newfloat@addtohook|\marg{code with \#1}\par
% adds code to the hook called in |\Declare|\-|Floating|\-|Environment|.\par
% Note: We use |\provide|\-|command| instead of |\new|\-|command| here since the
% exact same code is defined for |\For|\-|Each|\-|Caption|\-|Type| offered by
% the \package{caption3} package.
%    \begin{macrocode}
\providecommand\newfloat@addtohook[1]{%
  \toks@=\expandafter{\newfloat@hook{##1}#1}%
  \edef\@tempa{\def\noexpand\newfloat@hook####1{\the\toks@}}%
  \@tempa}
%    \end{macrocode}
% \end{macro}
%
% \begin{macro}{\newfloat@hook}
% Hook used in |\Declare|\-|Floating|\-|Environment|.
%    \begin{macrocode}
\providecommand*\newfloat@hook[1]{}
%    \end{macrocode}
% \end{macro}
%
% \pagebreak[3]
% \subsection{\cs{PrepareListOf}}
%
% \begin{macro}{\PrepareListOf}
% \changes{v1.1}{2013/04/27}{This macro added}
% |\PrepareListOf|\marg{float type}\marg{code}\\
% adds \meta{code} to the one which prepares typesetting the list,
% just before |\listoffigures|.
%    \begin{macrocode}
\newcommand\PrepareListOf[1]{%
  \expandafter\g@addto@macro\csname newfloat@listof#1@hook\endcsname}
\@onlypreamble\PrepareListOf
%    \end{macrocode}
% \end{macro}
%
% \pagebreak[3]
% \subsection{The list of floating environments}
%
% \begin{macro}{\newfloat@list}
% |\newfloat@list| is an \cs{@elt}-list containing the already existising floating
% environments as well the ones defined with |\Declare|\-|Floating|\-|Environment|.
%    \begin{macrocode}
\newcommand*\newfloat@list{}
%    \end{macrocode}
% \end{macro}
%
% \begin{macro}{\newfloat@addtolist}
% \changes{v1.0b}{2012/05/18}{Bugfix: Definition of \cs{@ifchapterlistsgap@}\meta{environment} added}
% \changes{v1.1n}{2021/02/27}{Defines \cs{newfloat@env@xxx} now instead of \cs{newfloat@ext@xxx}}
% |\newfloat@addtolist|\marg{environment}
% adds an environment to the list of floating environments.
%    \begin{macrocode}
\newcommand*\newfloat@addtolist[1]{%
  \newfloat@ifinlist{#1}{}{%
    \ifcsname ext@#1\endcsname
      \@cons\newfloat@list{{#1}}%
      \@namedef{newfloat@env@\@nameuse{ext@#1}}{#1}%
      \newfloat@let{@ifchapterlistsgap@#1}{\@iden}%
    \else
      \newfloat@Error{`#1' does not seem to be a floating environment}%
    \fi}}
%    \end{macrocode}
% \end{macro}
%
% \begin{macro}{\newfloat@ifinlist}
% |\newfloat@ifinlist|\marg{environment}\marg{yes code}\marg{no code}
% tests if an environment is an element of the list of floating environments.
%    \begin{macrocode}
\newcommand*\newfloat@ifinlist[1]{%
  \let\next\@secondoftwo
  \begingroup
    \expandafter\let\csname c@#1\endcsname\newfloat@ifinlist
    \def\@elt##1{%
      \expandafter\ifx\csname c@##1\endcsname\newfloat@ifinlist
        \global\let\next\@firstoftwo
      \fi}%
    \newfloat@list
  \endgroup
  \next}
%    \end{macrocode}
% \end{macro}
%
% Add |figure| and |table| to the list of floating environments.
%    \begin{macrocode}
\ifcsname ext@figure\endcsname
  \newfloat@addtolist{figure}
\fi
\ifcsname ext@table\endcsname
  \newfloat@addtolist{table}
\fi
%    \end{macrocode}
%
% \pagebreak[3]
% \subsection{Chapter lists gaps}
%
%    \begin{macrocode}
\ifcsname @chapter\endcsname
%    \end{macrocode}
%
% \begin{macro}{\@chapterlistsgap}
% The amount of the chapter lists gap, the default one is |10pt|.
% (This command is already defined in \KOMAScript.)
%    \begin{macrocode}
  \providecommand*\@chapterlistsgap{10\p@}%
%    \end{macrocode}
% If the \package{dgruyter} package is loaded, the default gap is |13pt|.
%    \begin{macrocode}
  \@ifpackageloaded{dgruyter}{\renewcommand*\@chapterlistsgap{13\p@}}{}%
%    \end{macrocode}
% \end{macro}
%
% \begin{macro}{\@addchapterlistsgap}
% |\@addchapterlistsgap|\marg{float type}\marg{file extension}\par
% will add the chapter lists gap for the given float type.
%    \begin{macrocode}
  \providecommand*\@addchapterlistsgap[2]{%
    \@nameuse{@ifchapterlistsgap@#1}{% if switched on
      \@@addchapterlistsgap{#1}{#2}}}
%    \end{macrocode}
%    \begin{macrocode}
  \providecommand*\@@addchapterlistsgap[2]{%
    \@ifundefined{@addchapterlistsgap@#2}{% only once per extension
      \@namedef{@addchapterlistsgap@#2}{#1}%
      \@@@addchapterlistsgap{#2}}{}}
%    \end{macrocode}
%    \begin{macrocode}
  \providecommand*\@@@addchapterlistsgap[1]{%
    \ifdim \@chapterlistsgap>\z@
      \addtocontents{#1}{\protect\addvspace{\@chapterlistsgap}}%
    \fi}
%    \end{macrocode}
% \end{macro}
%
% \begin{macro}{\@addchapterlistsgaps}
% \changes{v1.0b}{2012/05/18}{Bugfix: Group added to make \cs{@addchapterlistsgap} work correctly}
% \changes{v1.1n}{2021/02/27}{Uses \cs{newfloat@ext} now instead of \cs{ext@xxx}}
% |\@addchapterlistsgaps|\par
% will add the chapter lists gaps for all floating environments in |\newfloat@list|.
%    \begin{macrocode}
  \providecommand*\@addchapterlistsgaps{%
    \begingroup
      \def\@elt##1{%
        \@expandtwoargs\@addchapterlistsgap{##1}{\newfloat@ext{##1}}}%
      \newfloat@list
    \endgroup}
%    \end{macrocode}
% \end{macro}
%
% \begin{macro}{\@chapterlistsgap@off}
% \changes{v1.1n}{2021/02/27}{Uses \cs{newfloat@ext} now instead of \cs{ext@xxx}}
% |\@chapterlistsgap@off|\marg{float type}\par
% switches the chapter lists gap off for the given float type.
% Since \KOMAScript{} (still) supports |\float@exts| we need to handle
% this locally, too, even if |\unsettoc| is offered by the \package{tocbasic} package.
% (Otherwise our handling could be moved into the |\else| branch.)
%    \begin{macrocode}
  \providecommand*\@chapterlistsgap@off[1]{%
    \expandafter\let\csname @ifchapterlistsgap@#1\endcsname\@gobble
    \ifcsname unsettoc\endcsname
      \@expandtwoargs\unsettoc{\newfloat@ext{#1}}{chapteratlist}%
    \fi}
%    \end{macrocode}
% \end{macro}
%
% \begin{macro}{\@chapterlistsgap@on}
% \changes{v1.1n}{2021/02/27}{Uses \cs{newfloat@ext} now instead of \cs{ext@xxx}}
% |\@chapterlistsgap@off|\marg{float type}\par
% switches the chapter lists gap on for the given float type.
%    \begin{macrocode}
  \providecommand*\@chapterlistsgap@on[1]{%
    \expandafter\let\csname @ifchapterlistsgap@#1\endcsname\@iden
    \ifcsname setuptoc\endcsname
      \@expandtwoargs\setuptoc{\newfloat@ext{#1}}{chapteratlist}%
    \fi}
%    \end{macrocode}
% \end{macro}
%
%    \begin{macrocode}
\fi
%    \end{macrocode}
%
% \pagebreak[3]
% \subsection{Global options}
%
% \begin{macro}{chapterlistsgap=}
% The |chapterlistsgap=| option sets the vertical skip added to each list when
% starting a new chapter.
%    \begin{macrocode}
\define@key{newfloat}{chapterlistsgap}{%
  \renewcommand*\@chapterlistsgap{#1}}
%    \end{macrocode}
% \end{macro}
%
% \begin{macro}{within=}
% The |within=| option redefines the default value and modifies all
% existing floating environments.
%    \begin{macrocode}
\define@key{newfloat}{within}{%
  \def\newfloat@within@default{#1}% set new default value
  \def\@elt##1{\newfloat@setwithin{##1}{#1}}%
  \newfloat@list
  \let\@elt\relax}
%    \end{macrocode}
%    \begin{macrocode}
\define@key{newfloat}{without}[]{%
  \KV@newfloat@within{none}}
%    \end{macrocode}
% \end{macro}
%
% \begin{macro}{figurename=}
% \begin{macro}{listfigurename=}
% \begin{macro}{figurewithin=}
% We define these options not only for |figure| but for all existing floating environments.
%    \begin{macrocode}
\def\@elt#1{%
  \define@key{newfloat}{#1name}{%
    \newfloat@setname{#1}{##1}}%
  \define@key{newfloat}{list#1name}{%
    \newfloat@setname{list#1}{##1}}%
  \define@key{newfloat}{#1within}{%
    \newfloat@setwithin{#1}{##1}}%
  \define@key{newfloat}{#1without}[]{%
    \newfloat@setwithout{#1}}%
}%
\newfloat@list
\let\@elt\relax
%    \end{macrocode}
% \end{macro}
% \end{macro}
% \end{macro}
%
% \begin{macro}{planb=}
% \changes{v1.0b}{2012/05/18}{This option added}
% This option switches the `Plan B' on or off. (default$=$true)
%    \begin{macrocode}
\define@key{newfloat}{planb}[true]{%
  \def\@tempa{#1}%
  \def\@tempb{false}%
  \ifx\@tempa\@tempb
    \let\newfloat@ifplanb\@gobble
  \else
    \def\@tempb{true}%
    \ifx\@tempa\@tempb
      \let\newfloat@ifplanb\@iden
    \else
      \newfloat@Error{Invalid value `#1' for option `planb'}%
    \fi
  \fi}
%    \end{macrocode}
% \end{macro}
% \begin{macro}{planb-filext=}
% \changes{v1.0b}{2012/05/19}{This option added}
% This option sets the file extension used for `Plan B'.
% (default$=$use the 1st one which seem to fit)
%    \begin{macrocode}
\define@key{newfloat}{planb-fileext}{%
  \newfloat@Info{Setting Plan B file extension to `#1'}
  \xdef\newfloat@addtocontents@ext{#1}}
%    \end{macrocode}
% \end{macro}
%
% Process the package options:
% We use |\set|\-|keys| here instead of |\Process|\-|Options|.
%    \begin{macrocode}
\let\@tempc\relax
\@expandtwoargs\setkeys{newfloat}{planb,\@ptionlist{\@currname.\@currext}}%
\AtEndOfPackage{\let\@unprocessedoptions\relax}
%    \end{macrocode}
%
% \begin{macro}{\newfloatsetup}
% |\newfloatsetup|\marg{options}
% sets global options after loading the package.
%    \begin{macrocode}
\newcommand*\newfloatsetup{\setkeys{newfloat}}
%    \end{macrocode}
% \end{macro}
%
% \pagebreak[3]
% \subsection{Patching \cs{chapter}}
%
% \begin{macro}{\newfloat@replace@chapter}
% \changes{v1.1f}{2019/08/25}{Parameter \meta{class name} added to macro}
% \changes{v1.1i}{2019/08/31}{Local definition of \cs{if@chapnum} added}
% \changes{v1.1o}{2021/02/27}{Support of the \package{algorithm2e} package added}
%   |\newfloat@replace@chapter|\marg{class name}\marg{original code}\marg{replacement code}\par
%   tries to patch |\@chapter| so |\@add|\-|chapter|\-|lists|\-|gaps| will be supported.
%   It checks for |\Hy@org@chapter|, too, since the original code will be stored here
%   if the \package{hyperref} package was loaded.
%    \begin{macrocode}
\newcommand\newfloat@replace@chapter[3]{%
  \begingroup
    \let\if@twocolumn\iffalse
    \let\if@mainmatter\iffalse
    \let\if@thema\iffalse
    \let\if@chapnum\iffalse
    \def\@tempa[##1]##2{#2}%
    \newfloat@replace@chapter@\@chapter{original}{#1}{#3}%
    \newfloat@replace@chapter@\Hy@org@chapter{hyperref}{#1}{#3}%
    \newfloat@replace@chapter@\algocf@original@chapter{algorithm2e}{#1}{#3}%
  \endgroup}
%    \end{macrocode}
%    \begin{macrocode}
\newcommand\newfloat@replace@chapter@[4]{%
  \ifx\@tempa#1%
    \newfloat@InfoNoLine{Document class `#3' detected (#2)}%
    \global\def#1[##1]##2{#4}%
    \global\let\newfloat@replace@chapter\newfloat@dont@replace@chapter
    \global\let\newfloat@replace@chapter@\@gobblefour
  \fi}
%    \end{macrocode}
%    \begin{macrocode}
\newcommand\newfloat@dont@replace@chapter[3]{}
%    \end{macrocode}
%    \begin{macrocode}
\ifcsname @chapter\endcsname \else
  \let\newfloat@replace@chapter\newfloat@dont@replace@chapter
\fi
%    \end{macrocode}
% \end{macro}
%
% \pagebreak[3]
% \subsubsection{Standard \LaTeX{} document classes}
%
%    \begin{macrocode}
% report.cls [2005/09/16 v1.4f Standard LaTeX document class]
\newfloat@replace@chapter{report}{%
  \ifnum \c@secnumdepth >\m@ne
    \refstepcounter{chapter}%
    \typeout{\@chapapp\space\thechapter.}%
    \addcontentsline{toc}{chapter}%
      {\protect\numberline{\thechapter}#1}%
  \else
    \addcontentsline{toc}{chapter}{#1}%
  \fi
  \chaptermark{#1}%
  \addtocontents{lof}{\protect\addvspace{10\p@}}%
  \addtocontents{lot}{\protect\addvspace{10\p@}}%
  \if@twocolumn
    \@topnewpage[\@makechapterhead{#2}]%
  \else
    \@makechapterhead{#2}%
    \@afterheading
  \fi
}{%
  \ifnum \c@secnumdepth >\m@ne
    \refstepcounter{chapter}%
    \typeout{\@chapapp\space\thechapter.}%
    \addcontentsline{toc}{chapter}%
      {\protect\numberline{\thechapter}#1}%
  \else
    \addcontentsline{toc}{chapter}{#1}%
  \fi
  \chaptermark{#1}%
  \@addchapterlistsgaps
  \if@twocolumn
    \@topnewpage[\@makechapterhead{#2}]%
  \else
    \@makechapterhead{#2}%
    \@afterheading
  \fi}
%    \end{macrocode}
%
%    \begin{macrocode}
% book.cls [2005/09/16 v1.4f Standard LaTeX document class]
\newfloat@replace@chapter{book}{%
  \ifnum \c@secnumdepth >\m@ne
    \if@mainmatter
      \refstepcounter{chapter}%
      \typeout{\@chapapp\space\thechapter.}%
      \addcontentsline{toc}{chapter}%
        {\protect\numberline{\thechapter}#1}%
    \else
      \addcontentsline{toc}{chapter}{#1}%
    \fi
  \else
    \addcontentsline{toc}{chapter}{#1}%
  \fi
  \chaptermark{#1}%
  \addtocontents{lof}{\protect\addvspace{10\p@}}%
  \addtocontents{lot}{\protect\addvspace{10\p@}}%
  \if@twocolumn
    \@topnewpage[\@makechapterhead{#2}]%
  \else
    \@makechapterhead{#2}%
    \@afterheading
  \fi
}{%
  \ifnum \c@secnumdepth >\m@ne
    \if@mainmatter
      \refstepcounter{chapter}%
      \typeout{\@chapapp\space\thechapter.}%
      \addcontentsline{toc}{chapter}%
        {\protect\numberline{\thechapter}#1}%
    \else
      \addcontentsline{toc}{chapter}{#1}%
    \fi
  \else
    \addcontentsline{toc}{chapter}{#1}%
  \fi
  \chaptermark{#1}%
  \@addchapterlistsgaps
  \if@twocolumn
    \@topnewpage[\@makechapterhead{#2}]%
  \else
    \@makechapterhead{#2}%
    \@afterheading
  \fi}
%    \end{macrocode}
%
% \pagebreak[3]
% \subsubsection{\AmS{} \& \SmF{} document classes}
%
%    \begin{macrocode}
% amsbook.cls [2004/08/06 v2.20]
% smfbook.cls [1999/11/15 v1.2f Classe LaTeX pour les monographies editees par la SMF]
\newfloat@replace@chapter{amsbook}{%
  \refstepcounter{chapter}%
  \ifnum\c@secnumdepth<\z@ \let\@secnumber\@empty
  \else \let\@secnumber\thechapter \fi
  \typeout{\chaptername\space\@secnumber}%
  \def\@toclevel{0}%
  \ifx\chaptername\appendixname \@tocwriteb\tocappendix{chapter}{#2}%
  \else \@tocwriteb\tocchapter{chapter}{#2}\fi
  \chaptermark{#1}%
  \addtocontents{lof}{\protect\addvspace{10\p@}}%
  \addtocontents{lot}{\protect\addvspace{10\p@}}%
  \@makechapterhead{#2}\@afterheading
}{%
  \refstepcounter{chapter}%
  \ifnum\c@secnumdepth<\z@ \let\@secnumber\@empty
  \else \let\@secnumber\thechapter \fi
  \typeout{\chaptername\space\@secnumber}%
  \def\@toclevel{0}%
  \ifx\chaptername\appendixname \@tocwriteb\tocappendix{chapter}{#2}%
  \else \@tocwriteb\tocchapter{chapter}{#2}\fi
  \chaptermark{#1}%
  \@addchapterlistsgaps
  \@makechapterhead{#2}\@afterheading}
%    \end{macrocode}
%
% \pagebreak[3]
% \subsubsection{\KOMAScript{} document classes}
%
% If a \KOMAScript{} document class or the \package{tocbasic} package is used
% we don't need to patch anything.
% Instead we use |\setup|\-|toc| and |\un|\-|set|\-|toc| to setup the
% chapters gap in |\@chapter|\-|lists|\-|gap@on| and |\@chapter|\-|lists|\-|gap@off|.
%    \begin{macrocode}
\@ifpackageloaded{tocbasic}{%
  \let\newfloat@replace@chapter\newfloat@dont@replace@chapter}{}
%    \end{macrocode}
%
% \pagebreak[3]
% \subsubsection{memoir document classes}
%
% If the \package{memoir} document class is used,
% replacing |\insert|\-|chapter|\-|space| by |\@add|\-|chapter|\-|lists|\-|gaps| is sufficient.
%    \begin{macrocode}
\ifcsname insertchapterspace\endcsname
  \renewcommand*\insertchapterspace{\@addchapterlistsgaps}
  \let\newfloat@replace@chapter\newfloat@dont@replace@chapter
\fi
%    \end{macrocode}
%
% \pagebreak[3]
% \subsubsection{NTG document classes}
%
%    \begin{macrocode}
% rapport1/3.cls [2004/06/07 v2.1a NTG LaTeX document class]
\newfloat@replace@chapter{rapport}{%
  \ifnum \c@secnumdepth >\m@ne
    \refstepcounter{chapter}%
    \typeout{\@chapapp\space\thechapter.}%
    \addcontentsline{toc}{chapter}%
      {\protect\numberline{\thechapter}\toc@font0 #1}%
  \else
    \addcontentsline{toc}{chapter}{\toc@font0 #1}%
  \fi
  \chaptermark{#1}%
  \addtocontents{lof}{\protect\addvspace{10\p@}}%
  \addtocontents{lot}{\protect\addvspace{10\p@}}%
  \if@twocolumn
    \@topnewpage[\@makechapterhead{#2}]%
  \else
    \@makechapterhead{#2}%
    \@afterheading
  \fi
}{%
  \ifnum \c@secnumdepth >\m@ne
    \refstepcounter{chapter}%
    \typeout{\@chapapp\space\thechapter.}%
    \addcontentsline{toc}{chapter}%
      {\protect\numberline{\thechapter}\toc@font0 #1}%
  \else
    \addcontentsline{toc}{chapter}{\toc@font0 #1}%
  \fi
  \chaptermark{#1}%
  \@addchapterlistsgaps
  \if@twocolumn
    \@topnewpage[\@makechapterhead{#2}]%
  \else
    \@makechapterhead{#2}%
    \@afterheading
  \fi}
%    \end{macrocode}
%    \begin{macrocode}
% boek(3).cls [2004/06/07 v2.1a NTG LaTeX document class]
\newfloat@replace@chapter{boek}{%
  \ifnum \c@secnumdepth >\m@ne
    \if@mainmatter
      \refstepcounter{chapter}%
      \typeout{\@chapapp\space\thechapter.}%
      \addcontentsline{toc}{chapter}%
        {\protect\numberline{\thechapter}\toc@font0 #1}%
    \else
      \addcontentsline{toc}{chapter}{\toc@font0 #1}%
    \fi
  \else
    \addcontentsline{toc}{chapter}{\toc@font0 #1}%
  \fi
  \chaptermark{#1}%
  \addtocontents{lof}{\protect\addvspace{10\p@}}%
  \addtocontents{lot}{\protect\addvspace{10\p@}}%
  \if@twocolumn
    \@topnewpage[\@makechapterhead{#2}]%
  \else
    \@makechapterhead{#2}%
    \@afterheading
  \fi
}{%
  \ifnum \c@secnumdepth >\m@ne
    \if@mainmatter
      \refstepcounter{chapter}%
      \typeout{\@chapapp\space\thechapter.}%
      \addcontentsline{toc}{chapter}%
        {\protect\numberline{\thechapter}\toc@font0 #1}%
    \else
      \addcontentsline{toc}{chapter}{\toc@font0 #1}%
    \fi
  \else
    \addcontentsline{toc}{chapter}{\toc@font0 #1}%
  \fi
  \chaptermark{#1}%
  \@addchapterlistsgaps
  \if@twocolumn
    \@topnewpage[\@makechapterhead{#2}]%
  \else
    \@makechapterhead{#2}%
    \@afterheading
  \fi}
%    \end{macrocode}
%
% \pagebreak[3]
% \subsubsection{The thesis document class}
%
%    \begin{macrocode}
% thesis.cls [1996/25/01 1.0g LaTeX document class (wm).]
\newfloat@replace@chapter{thesis}{%
  \ifnum \c@secnumdepth >\m@ne
    \if@mainmatter
      \refstepcounter{chapter}%
      \typeout{\chaptername\space\thechapter.}
      \if@thema
        \ifx\@shortauthor\@empty
          \addcontentsline{toc}{chapter}{%
          \protect\numberline{\thechapter.}#1}%
        \else
          \addcontentsline{toc}{chapter}{%
          \protect\numberline{\thechapter.}%
          \@shortauthor\hfill\mbox{}\vskip\normallineskip #1}%
        \fi
      \else
        \addcontentsline{toc}{chapter}{%
        \protect\numberline{\thechapter.}#1}%
      \fi
    \else
      \addcontentsline{toc}{chapter}{#1}
    \fi
  \else
    \addcontentsline{toc}{chapter}{#1}
  \fi
  \chaptermark{#1}
  \addtocontents{lof}{\protect\addvspace{10pt}}
  \addtocontents{lot}{\protect\addvspace{10pt}}
  \if@twocolumn
    \@topnewpage[\@makechapterhead{#2}]
  \else
    \@makechapterhead{#2}
    \@afterheading
  \fi
}{%
  \ifnum \c@secnumdepth >\m@ne
    \if@mainmatter
      \refstepcounter{chapter}%
      \typeout{\chaptername\space\thechapter.}%
      \if@thema
        \ifx\@shortauthor\@empty
          \addcontentsline{toc}{chapter}{%
          \protect\numberline{\thechapter.}#1}%
        \else
          \addcontentsline{toc}{chapter}{%
          \protect\numberline{\thechapter.}%
          \@shortauthor\hfill\mbox{}\vskip\normallineskip #1}%
        \fi
      \else
        \addcontentsline{toc}{chapter}{%
        \protect\numberline{\thechapter.}#1}%
      \fi
    \else
      \addcontentsline{toc}{chapter}{#1}%
    \fi
  \else
    \addcontentsline{toc}{chapter}{#1}%
  \fi
  \chaptermark{#1}%
  \@addchapterlistsgaps
  \if@twocolumn
    \@topnewpage[\@makechapterhead{#2}]%
  \else
    \@makechapterhead{#2}%
    \@afterheading
  \fi}
%    \end{macrocode}
%
% \pagebreak[3]
% \subsubsection{The Springer Verlag document classes}
% \changes{v1.1f}{2019/08/25}{Support of \package{svmono} and \package{svmult} document classes added}
%
%    \begin{macrocode}
% svmono.cls [2018/06/25 v5.6 Springer Verlag global LaTeX document class for monographs]
% svmult.cls [2018/01/11 v5.5 Springer Verlag global LaTeX document class for multi authored books]
\newfloat@replace@chapter{svmono}{%
  \if@chapnum  % war mal \ifnum \c@secnumdepth >\m@ne
    \refstepcounter{chapter}%
    \if@mainmatter
      \typeout{\@chapapp\space\thechapter.}%
      \addcontentsline{toc}{chapter}{\protect
        \numberline{\thechapter\thechapterend}#1}%
    \else
      \addcontentsline{toc}{chapter}{#1}%
    \fi
  \else
    \addcontentsline{toc}{chapter}{#1}%
  \fi
  \chaptermark{#1}%
  \addtocontents{lof}{\protect\addvspace{10\p@}}%
  \addtocontents{lot}{\protect\addvspace{10\p@}}%
  \if@twocolumn
    \@topnewpage[\@makechapterhead{#2}]%
  \else
    \@makechapterhead{#2}%
    \@afterheading
  \fi
}{%
  \if@chapnum  % war mal \ifnum \c@secnumdepth >\m@ne
    \refstepcounter{chapter}%
    \if@mainmatter
      \typeout{\@chapapp\space\thechapter.}%
      \addcontentsline{toc}{chapter}{\protect
        \numberline{\thechapter\thechapterend}#1}%
    \else
      \addcontentsline{toc}{chapter}{#1}%
    \fi
  \else
    \addcontentsline{toc}{chapter}{#1}%
  \fi
  \chaptermark{#1}%
  \@addchapterlistsgaps
  \if@twocolumn
    \@topnewpage[\@makechapterhead{#2}]%
  \else
    \@makechapterhead{#2}%
    \@afterheading
  \fi}
%    \end{macrocode}
%
% \pagebreak[3]
% \subsubsection{The dgruyter package}
% \changes{v1.1j}{2019/08/31}{Extended support of \package{dgruyter} package added}
%
%    \begin{macrocode}
% dgruyter.sty [2017/09/19 v2.00 De Gruyter layout]
\newfloat@replace@chapter{dgruyter}{%
  \let\thesection\theinchapsection
  \let\theequation\theinchapequation
  \let\thefigure\theinchapfigure
  \let\thetable\theinchaptable
  \ifx\bookDOI\@undefined
  \else
    \@tempcnta\numexpr\componentd@i+\@ne\relax
    \xdef\componentd@i{\the\@tempcnta}%
    \xdef\@DOI{%
      https://doi.org/\bookDOI
      -\ifx\thepage\thefrontmatterpage2\else\ifnum\@tempcnta<100 0\fi\fi
      \ifnum\@tempcnta<10 0\fi\componentd@i}%
  \fi
  \ifnum \c@secnumdepth >\m@ne
    \if@mainmatter
      \refstepcounter{chapter}%
      \typeout{\@chapapp\space\thechapter.}%
      \addcontentsline{toc}{chapter}{\protect\numberline{\thechapter}#1}%
    \else
      \addcontentsline{toc}{schapter}{#1}%
    \fi
  \else
    \addcontentsline{toc}{schapter}{#1}%
  \fi
  \chaptermark{#1}%
  \addtocontents{lof}{\protect\addvspace{13\p@}}%
  \addtocontents{lot}{\protect\addvspace{13\p@}}%
  \if@twocolumn
    \@topnewpage[\@makechapterhead{#2}]%
  \else
    \@makechapterhead{#2}%
    \@afterheading
  \fi
}{%
  \let\thesection\theinchapsection
  \let\theequation\theinchapequation
  \let\thefigure\theinchapfigure
  \let\thetable\theinchaptable
  \ifx\bookDOI\@undefined
  \else
    \@tempcnta\numexpr\componentd@i+\@ne\relax
    \xdef\componentd@i{\the\@tempcnta}%
    \xdef\@DOI{%
      https://doi.org/\bookDOI
      -\ifx\thepage\thefrontmatterpage2\else\ifnum\@tempcnta<100 0\fi\fi
      \ifnum\@tempcnta<10 0\fi\componentd@i}%
  \fi
  \ifnum \c@secnumdepth >\m@ne
    \if@mainmatter
      \refstepcounter{chapter}%
      \typeout{\@chapapp\space\thechapter.}%
      \addcontentsline{toc}{chapter}{\protect\numberline{\thechapter}#1}%
    \else
      \addcontentsline{toc}{schapter}{#1}%
    \fi
  \else
    \addcontentsline{toc}{schapter}{#1}%
  \fi
  \chaptermark{#1}%
  \@addchapterlistsgaps
  \if@twocolumn
    \@topnewpage[\@makechapterhead{#2}]%
  \else
    \@makechapterhead{#2}%
    \@afterheading
  \fi}
%    \end{macrocode}
%
% \pagebreak[3]
% \subsubsection{Compatibility warning}
%
%    \begin{macrocode}
\ifx\newfloat@replace@chapter\newfloat@dont@replace@chapter \else
%    \end{macrocode}
% If we were not able to patch |\@chapter| an info message is issued
% since we may not be able to support chapter lists gaps then.
%    \begin{macrocode}
  \newfloat@InfoNoLine{%
    Unsupported document class, or \noexpand\@chapter\MessageBreak
    was already redefined by another package}
  \newfloat@InfoNoLine{\string\@chapter\space=\space\meaning\@chapter}
  \newfloat@InfoNoLine{\string\Hy@org@chapter\space=\space\meaning\Hy@org@chapter}
  \newfloat@InfoNoLine{\string\algocf@original@chapter\space=\space\meaning\algocf@original@chapter}
%    \end{macrocode}
% Plan B: We patch |\add|\-|to|\-|contents| instead, and keep our fingers crossed
% in the hope that this redefinition actually will fit.
% (Note: Since \package{biblatex} patches |\add|\-|to|\-|contents| as well we do
% this using |\At||Begin||Document|.)
% \changes{v1.1f}{2019/08/25}{Plan B will now be done using \cs{AtBeginDocument}}
%    \begin{macrocode}
  \newfloat@ifplanb{\AtBeginDocument{%
    \newfloat@InfoNoLine{Trying Plan B..}%
    \let\newfloat@addtocontents@ORI\addtocontents
    \long\def\addtocontents#1#2{%
      \newfloat@addtocontents{#1}{#2}#2\addvspace\newfloat@nil}%
    \long\def\newfloat@addtocontents#1#2#3\addvspace#4\newfloat@nil{%
      \def\newfloat@tempa{#4}%
      \ifx\newfloat@tempa\@empty
        \newfloat@addtocontents@ORI{#1}{#2}%
      \else
        \ifx\newfloat@addtocontents@ext\@undefined
          \newfloat@Info{Setting Plan B file extension to `#1'...}%
          \xdef\newfloat@addtocontents@ext{#1}%
        \fi
        \edef\newfloat@tempa{#1}%
        \ifx\newfloat@tempa\newfloat@addtocontents@ext
          \begingroup
            \let\addtocontents\newfloat@addtocontents@ORI
            \@addchapterlistsgaps
          \endgroup
        \fi
      \fi}}}
%    \end{macrocode}
%    \begin{macrocode}
\fi
%    \end{macrocode}
%
% \pagebreak[3]
% \subsection{Support of document classes}
%
% \subsubsection{The beamer document class}
%
% \begin{macro}{\beamer@float}
% \changes{v1.1k}{2019/09/01}{This macro added}
% Unfortunately the \class{beamer} document class does not offer
% |\beamer@float| and |\end@beamer@float| (yet), so we have to define them
% for ourself.
% (This code was taken from beamerbaselocalstructure.sty [2019/07/23 v3.56] and adapted.)
%    \begin{macrocode}
\ifcsname beamer@float\endcsname \else
%    \end{macrocode}
%    \begin{macrocode}
  \def\beamer@float#1{%
    \ifcsname caption@settype\endcsname
      \caption@settype{#1}%
    \else
      \def\@captype{#1}%
    \fi
    \par\nobreak\begin{center}\nobreak}
%    \end{macrocode}
%    \begin{macrocode}
  \def\end@beamer@float{%
    \par\nobreak\end{center}}
%    \end{macrocode}
%    \begin{macrocode}
\fi
%    \end{macrocode}
% \end{macro}
%
% \pagebreak[3]
% \subsection{Support of other packages}
%
% \begin{macro}{\newfloat@IfPackageLoaded}
% \changes{v1.1e}{2019/03/17}{This macro revised}
% \changes{v1.1h}{2019/08/31}{This macro revised again and renamed from \cs{newfloat@ForEachNew} to \cs{newfloat@IfPackageLoaded}}
% |\newfloat@IfPackageLoaded|\marg{package name}\marg{command}\marg{code}
% will execute the given code for every floating environment defined with
% |\Declare|\-|Floating|\-|Environment| if \meta{command} is not defined yet.
% (If \meta{command} is already defined a warning will be issued.)
% This will be tried |\At|\-|Begin|\-|Document| additionally so the affected package could
% be loaded after \thispackage.
%    \begin{macrocode}
\newcommand\newfloat@IfPackageLoaded[3]{%
  \@ifpackageloaded{#1}{%
    \newfloat@@IfPackageLoaded{#1}{#2}{#3}%
  }{%
    \AtBeginDocument{%
      \@ifpackageloaded{#1}{%
        \newfloat@@IfPackageLoaded{#1}{#2}{#3}%
      }{}}%
  }}
\@onlypreamble\newfloat@IfPackageLoaded
%    \end{macrocode}
%    \begin{macrocode}
\newcommand\newfloat@@IfPackageLoaded[3]{%
  \newfloat@InfoNoLine{`#1' package detected}%
  % Execute code for every previous \DeclareFloatingEnvironment
  \def\@elt##1{\newfloat@@@IfPackageLoaded{#1}{#2}{#3}}%
  \newfloat@@list
  \let\@elt\relax
  % Execute code for every future \DeclareFloatingEnvironment
  \newfloat@addtohook{\newfloat@@@IfPackageLoaded{#1}{#2}{#3}}}
\@onlypreamble\newfloat@@IfPackageLoaded
%    \end{macrocode}
%    \begin{macrocode}
\newcommand\newfloat@@@IfPackageLoaded[3]{%
  \ifcsname#2\endcsname
    \newfloat@WarningNoLine{%
      Command \@backslashchar#2 already defined.\MessageBreak
      (`#1' package support)}%
  \else
    #3%
  \fi}
\@onlypreamble\newfloat@@@IfPackageLoaded
%    \end{macrocode}
% \end{macro}
%
% \pagebreak[3]
% \subsubsection{\package{dgruyter}}
%
% The De Gruyter layout package (for preparing books for Walter de Gruyter GmbH)
% is defining |\plist@figure| and |\plist@table|, and expects this to be
% defined for other floating environments as well.
% (This will be used as prefix for the ``List of''.)
%    \begin{macrocode}
\newfloat@IfPackageLoaded{dgruyter}{plist@#1}{%
  \expandafter\newcommand\csname plist@#1\endcsname{%
    \@nameuse{#1name}\space}}
%    \end{macrocode}
%
% \pagebreak[3]
% \subsubsection{\package{float}}
%
% If the \package{float} package is used we fill up |\float@exts| with our file
% extensions, too. Since this list will be used for inserting chapters gaps
% we only add the ones which are configured for chapters gaps on.\par
% Note: This has to be done |\At|\-|Begin|\-|Document| so the settings for
% chapter gaps could be changed by the user within the preamble of the document.
%    \begin{macrocode}
\AtBeginDocument{%
  \ifcsname float@exts\endcsname
    \newfloat@@IfPackageLoaded{float}{newfloat@undefined}{%
      \@nameuse{@ifchapterlistsgap@#1}{% if switched on
        \let\float@do=\relax
        \edef\@tempa{%
          \noexpand\float@exts{\the\float@exts\float@do{\@nameuse{ext@#1}}}}%
        \@tempa}}%
  \fi}
%    \end{macrocode}
%
% \pagebreak[3]
% \subsubsection{\package{fltpage}}
%
% We define a FP-variant of new floating environments here.
%    \begin{macrocode}
\newfloat@IfPackageLoaded{fltpage}{FP#1}{%
  \newcounter{FP@#1C}%
  \newenvironment{FP#1}{\FP@floatBegin{#1}}{\FP@floatEnd}}
%    \end{macrocode}
%
% \pagebreak[3]
% \subsubsection{\package{listings}}
%
% \begin{macro}{\ext@lstlisting}
% Since the \package{listings} package do not define |\ext@lst|\-|listing|
% but we needed it when |Setup|\-|Floating|\-|Environment{lst|\-|listing}{|\ldots|}|
% will be done by the end user, we define it here.
%    \begin{macrocode}
\providecommand*\ext@lstlisting{lol}%
%    \end{macrocode}
% \end{macro}
%
% \pagebreak[3]
% \subsubsection{\package{rotating}}
%
% We define a sideways-variant of new floating environments here.
%    \begin{macrocode}
\newfloat@IfPackageLoaded{rotating}{sideways#1}{%
  \newenvironment{sideways#1}{\@rotfloat{#1}}{\end@rotfloat}%
  \newenvironment{sideways#1*}{\@rotdblfloat{#1}}{\end@rotdblfloat}}
%    \end{macrocode}
%
% \pagebreak[3]
% \subsubsection{\package{sidecap}}
%
% We define a SC-variant of new floating environments here.
% \begin{macro}{\SC@float}
% \changes{v1.1b}{2018/12/26}{Bugfix: Missing curly braces added}
%    \begin{macrocode}
\newcommand*\newfloat@For@SC[2]{%
  \def#1{b}% = \sidecaptionvpos{#2}{b} (v1.6)
  \newenvironment{SC#2}%
    {\SC@float[{#1}]{#2}}{\endSC@float}%
  \newenvironment{SC#2*}%
    {\SC@dblfloat[{#1}]{#2}}{\endSC@dblfloat}}
\@onlypreamble\newfloat@For@SC
%    \end{macrocode}
%    \begin{macrocode}
\newfloat@IfPackageLoaded{sidecap}{SC#1}{%
  \expandafter\newfloat@For@SC\csname SC@#1@vpos\endcsname{#1}}
%    \end{macrocode}
% \end{macro}
%
% \pagebreak[3]
% \subsubsection{\package{wrapfig}}
%
% We define a wrap-variant of new floating environments here.
%    \begin{macrocode}
\newfloat@IfPackageLoaded{wrapfig}{wrap#1}{%
  \newenvironment{wrap#1}{\wrapfloat{#1}}{\endwrapfloat}}
%    \end{macrocode}
%
% \iffalse
%</package>
% \fi
%
% \iffalse
% --------------------------------------------------------------------------- %
% \fi
%
% \Finale
%
\endinput

