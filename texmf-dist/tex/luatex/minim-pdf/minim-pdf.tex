
\ifdefined\minimpdfloaded
    \message{(skipped)}
    \expandafter\endinput\fi
\chardef\minimpdfloaded = \catcode`\:
\catcode`\: = 11

\input minim-alloc
\input minim-hooks

% Abbreviations used throughout this document:
%   se   structure element
%   mci  marked content item

\newcount \writedocumentstructure
\newcount \writehyphensandspaces
\newcount \strictstructuretagging

\writedocumentstructure = 0
\writehyphensandspaces  = 1
\strictstructuretagging = 0

% 1 the attributes

% By three attributes do we determine the document structure.
%
%   One for marking the current structure element:
%   - is the index of the lua-side se table
%   - assignments are always local
\newattribute \tagging:current:se  \tagging:current:se = 1
%
%   One for ordering children of structure elements:
%   - assignments are always global
%   - should generally increase monotously
%   - should change (increase) at every se boundary
\newattribute \tagging:element:order \tagging:element:order = 1
%
%   One for marking the status:
%   - assignments are generally local
%   - if set, disables tagging and content marking in pre_shipout
%   - values >10 indicate artifacts
\newattribute \tagging:current:status \tagging:current:status = \unset
%
%   A fourth attribute keeps track of the current language
%   - assignments are local
%   - sets language of structure elements
\newattribute \tagging:current:language \tagging:current:language = \language

\directlua { require('minim-pdf') }

% 1 artifacts and content items

% \stoptagging ... \starttagging
%     disables marking structure elements
\newif\iftagging:enabled \tagging:enabledtrue
\protected\def\starttagging{\tagging:enabledtrue
    \ifnum0<\tagging:current:status
        \tagging:current:status\unset\fi}
\protected\def\stoptagging{\tagging:enabledfalse
    \ifnum\unset=\tagging:current:status
        \tagging:current:status1\relax\fi}

% \markartifact {Layout} {...}
% \startartifact {Pagination /Subtype/Header} ... \stopartifact
\long\def\markartifact#1#2{\startartifact{#1}#2\stopartifact}
\protected\def\startartifact{\begingroup\tagging:enabledfalse
    \tagging:current:status=\tagging:current:artifact
    \global\advance\tagging:current:artifact1\relax
    \tagging:mci:incr\tagging:art:markstart}
\protected\def\stopartifact{\tagging:art:markstop\tagging:mci:incr\endgroup}
\newcount\tagging:current:artifact \tagging:current:artifact = 11

% \startcontentitem ... \stopcontentitem
\protected\def\startcontentitem{\iftagging:enabled\tagging:mci:incr\tagging:mci:markstart\fi}
\protected\def\stopcontentitem{\iftagging:enabled\tagging:mci:markstop\tagging:mci:incr\fi}

% \startsinglecontentitem ... \stopsinglecontentitem
\protected\def\startsinglecontentitem{\begingroup \startcontentitem\stoptagging}
\protected\def\stopsinglecontentitem{\endgroup\stopcontentitem}

% \ensurecontentitem
\protected\def\ensurecontentitem{\iftagging:enabled\tagging:mci:content\fi}
\protected\def\tagging:mci:incr{\global\advance\tagging:element:order1\relax}

% 1 structure elements

% \savecurrentelement ... \continueelement
% \savecurrentelementto\name ... \continueelementfrom\name
\protected\def\savecurrentelementto#1{\global\chardef#1\tagging:current:se}
\protected\def\continueelementfrom#1{\tagging:current:se#1\tagging:mci:incr}
\def\savecurrentelement{\savecurrentelementto\tagging:saved:se}
\def\continueelement{\continueelementfrom\tagging:saved:se}

% \markelement {Tag} {...}
% \startelement {Tag} ... \stopelement {Tag}
% \ensurestopelement {Tag}
\long\protected\def\markelement#1#2{\iftagging:enabled
    \startelement{#1}#2\stopelement{#1}\else#2\fi}
\protected\def\startelement{\iftagging:enabled \tagging:mci:incr
    \expandafter\tagging:startelement\else
    \expandafter\tagging:ignore\fi}
\protected\def\stopelement{\iftagging:enabled \tagging:mci:incr
    \expandafter\tagging:stopelement\else
    \expandafter\tagging:ignore\fi}
\protected\def\ensurestopelement{\iftagging:enabled \tagging:mci:incr
    % n.b. this can cause problems if you skip levels
    \expandafter\tagging:ensurestopelement\else
    \expandafter\tagging:ignore\fi}
\def\tagging:ignore#1#{\ignore}

% \addstructuretype [options] Existing Alias
\protected\def\addstructuretype{\withoptions[]\addstructure:type}
\def\addstructure:type[#1]{\struct:addalias{return {#1}}}

% \setalttext {...}
\protected\def\setalttext{\iftagging:enabled
    \expandafter\tagging:alt\else
    \expandafter\tagging:ignore\fi}
% \setactualtext {...}
\protected\def\setactualtext{\iftagging:enabled
    \expandafter\tagging:actual\else
    \expandafter\tagging:ignore\fi}

% \tagattribute Owner Key /verbatim_value
\protected\def\tagattribute{\iftagging:enabled
    \expandafter\tagging:attribute\else
    \expandafter\tagging:ignore:attribute\fi}

% \settagid {name}
\protected\def\settagid{\iftagging:enabled
    \expandafter\tagging:tagid\else
    \expandafter\tagging:ignore\fi}

% \showstructure
\newtoks\tagging:showtoks
\protected\def\showstructure{%
    \tagging:getshowstructure
    \showthe\tagging:showtoks}

% 1 auto-marking paragraphs

% \nextpartag {H}
% \nextpartag {}    % no tag inserted for next paragraph
% \markparagraphs(true|false)
\newif\ifmarkparagraphs \markparagraphstrue
\newtoks\nextpartag \nextpartag{P}
\toksapp\minim:ateverypar{\iftagging:enabled \ifmarkparagraphs
    \expandafter\ifx\expandafter\relax\the\nextpartag\relax\else
        \startelement{\the\nextpartag}\fi
    \nextpartag{P}\fi \fi}

% 1 tagging tables

% \marktable \halign
%       {\marktablerow\marktablecell#&\marktablecell#\cr
%    \marktableheader
%       Header & cells \cr
%    \marktablebody
%       ... & ... \cr ...
%    \marktablefooter
%       ... & ... \cr ... }
%
% lots of variables and constants
\newcount\tagging:tablec % global; only used for cel ids
\newcount\tagging:tablerow
\newcount\tagging:tablecol
\newcount\tagging:tablecolspan \tagging:tablecolspan 1
\newcount\tagging:tablerowspan \tagging:tablerowspan 1
\newcount\tagging:i
\def\tagging:TH{{TH}\markcolumnhead}
\def\tagging:TD{{TD}}
% initialising variables
\def\marktable{\startelement{Table}%
    \global\advance\tagging:tablec1
    \tagging:tablerow0
    \tagging:tablecol0
    \def\tagging:tablecolheaders{}%
    \def\tagging:tablerowheaders{}%
    \let\tagging:cell\tagging:TD}
% the header
\def\marktableheader{%
    \noalign{\global\let\tagging:cell\tagging:TH
        \startelement{THead}%
        \savecurrentelementto\tagging:tpart}}
% the body
\def\marktablebody{%
    \noalign{\global\let\tagging:cell\tagging:TD
        \startelement{TBody}%
        \savecurrentelementto\tagging:tpart}}
% the footer
\def\marktablefooter{%
    \noalign{\startelement{TFoot}%
        \savecurrentelementto\tagging:tpart}}
% rows
\def\marktablerow{%
    \global\advance\tagging:tablerow1
    \global\tagging:tablecol0
    \unless\ifdefined\tagging:tpart
        \startelement{TBody}%
        \savecurrentelementto\tagging:tpart\fi
    \continueelementfrom\tagging:tpart
    \startelement{TR}%
    \savecurrentelementto\tagging:row}
% cells
\def\marktablecell{%
    \global\advance\tagging:tablecol1
    \continueelementfrom\tagging:row
    \expandafter\startelement\tagging:cell%
    \tagging:assignheaders}

% spans
\def\markcolumnspan#1{%
    \tagattribute Table ColSpan #1\relax
    \tagging:tablecolspan#1\relax
    % adjust the column number
    \global\advance\tagging:tablecol\numexpr#1-1\relax}
\def\markrowspan#1{%
    % note that the row number is not adjusted: \marktablerow must be present 
    % somewhere on every row to increment it.
    \tagattribute Table RowSpan #1\relax
    \tagging:tablerowspan#1\relax}

% headers
\def\markcolumnhead{%
    \tagging:setcelid
    \tagattribute Table Scope /Column\relax
    \tagging:i=1 \loop
        \xdef\tagging:tablecolheaders{\tagging:tablecolheaders
            {\the\numexpr\tagging:tablecol-\tagging:tablecolspan+\tagging:i}{\tagging:thecelid}}%
         \advance\tagging:i1 \unless\ifnum\tagging:i>\tagging:tablecolspan\repeat}
\def\markrowhead{%
    \tagging:setcelid
    \tagattribute Table Scope /Row\relax
    \tagging:i=0 \loop
        \xdef\tagging:tablerowheaders{\tagging:tablerowheaders
            {\the\numexpr\tagging:tablerow+\tagging:i}{\tagging:thecelid}}%
         \advance\tagging:i1 \ifnum\tagging:i<\tagging:tablerowspan\repeat}
\def\tagging:setcelid{%
    % assign an identifier to the cell structure element
    \edef\tagging:thecelid{t\the\tagging:tablec r\the\tagging:tablerow c\the\tagging:tablecol}%
    \expandafter\settagid\expandafter{\tagging:thecelid}}

% assign cells to their headers
\def\tagging:assignheaders{%
    \def\tagging:tmp{[}%
    \expandafter\tagging:assignheaders:do\expandafter\tagging:tablecol
        \tagging:tablecolheaders{}{}%
    \expandafter\tagging:assignheaders:do\expandafter\tagging:tablerow
        \tagging:tablerowheaders{}{}%
    \ifx\tagging:tmp\tagging:noheaders\else
        \tagattribute Table Headers {\tagging:tmp\space]}\fi}
\def\tagging:assignheaders:do#1#2#3{%
    \ifx\tagging:noheaders#3\tagging:noheaders
        \expandafter\tagging:assignheaders:done
    \else
        \ifnum#1=#2\relax
            % no cell is its own header
            \edef\tagging:cmp:a{\tagging:thecelid}%
            \edef\tagging:cmp:b{#3}%
            \unless\ifx\tagging:cmp:a\tagging:cmp:b
                \edef\tagging:tmp{\tagging:tmp\space\pdfstring{#3}}%
            \fi
        \fi
    \fi\tagging:assignheaders:do{#1}}
\def\tagging:noheaders{[}%]
\def\tagging:assignheaders:done#1#2{}
\def\tagging:thecelid{}

% \automarktable \halign ... { ... }
\def\automarktable#1#{\marktable
    \def\tagging:table{\tagging:mktrow{#1}}%
    \afterassignment\tagging:table
    \let\tagging:tmp= }% eats a bgroup
\def\tagging:mktrow#1#2&#3\cr{\iftrue
    \tagging:mktcell{#1\bgroup % \halign ...
        \marktablerow\marktablecell#2}\fi
        #3&\tagging:mktrow&\cr}
\def\tagging:mktcell#1#2\fi#3&{\fi
    % the next ifx is true on the last (artificial) template cell
    \ifx\tagging:mktrow#3\tagging:mkthdr{#1}\fi
    % the next ifx detects a double && (repeated template)
    \ifx\tagging:mktrow#3\tagging:mktrow
        \tagging:mktcell{#1&}\else
        \tagging:mktcell{#1&\marktablecell#3}\fi}
\def\tagging:mkthdr#1#2\cr#3\cr{\fi#1\cr
    \marktableheader#3\cr\marktablebody}

% 1 tagging helper macros

% \marktableaslist \halign ... { ... }
\def\marktableaslist#1#{\startelement{L}%
    \def\tagging:table{\tagging:mktlist{#1}}%
    \afterassignment\tagging:table
    \let\tagging:tmp= }% eats a bgroup
\def\tagging:mktlist#1#2&{#1\bgroup
    \startelement{LI}\savecurrentelementto\tagging:li
    \startelement{Lbl}#2&\continueelementfrom\tagging:li
    \startelement{LBody}\savecurrentelementto\tagging:lbody}

% \marktocentry {destination}{label}{title}{filler}{page}
\def\marktocentry#1#2#3#4#5{%
    \ifx\marktocentry#1\marktocentry
        \def\marktocentry:link##1{##1}\else
        \def\marktocentry:link##1{\hyperlink
            alt{\csname minim:str:page\endcsname#5}
            dest{#1}##1\endlink}\fi
    \markelement{TOCI}{\nextpartag{}\quitvmode
        \ifx\marktocentry#2\marktocentry\else
            \markelement{Lbl}{\marktocentry:link{#2}}\fi
        \markelement{Reference}{\marktocentry:link{#3%
            \ifx\marktocentry#4\marktocentry\else
                \markartifact{Layout}{#4}\fi#5}}}}
\def\minim:str:page{Page }

% \marknoteref {*}
\def\marknoteref#1{%
    \startelement ref{\newdestinationname}{Reference}%
    {\stopformulatagging#1}\stopelement{Reference}}
% \marknotelbl {*}
\def\marknotelbl#1{%
    \startelement id{\lastdestinationname}{Note}%
    \aftergroup\tagging:ensurestopNote
    \markelement{Lbl}{{\stopformulatagging#1}}}
\def\tagging:ensurestopNote{\ensurestopelement{Note}}

% \sectionstructure { subsection, section, chapter }
\def\sectionstructure{%
    \newtoks\tag:sect:resets \tag:sect:resets{}%
    \splitcommalist\tag:sect:a}
\def\tag:sect:a#1{\expandafter\tag:sect:b
    \expandafter{\Uchar\uccode`#1}{#1}}
\def\tag:sect:b#1#2{% Section section
    \addstructuretype Sect #1\relax
    \expandafter\newcount\csname #2nr\endcsname
    \expandafter\edef\csname mark#2\endcsname{%
        \noexpand\withoptions[]%
        \expandafter\noexpand\csname #2:domark\endcsname}%
    \expandafter\edef\csname #2:domark\endcsname[##1]{%
        \unexpanded{\ifhmode\errhelp{You should place section markers
            in the vertical list, before the header.}\errmessage
            {Section marker in horizontal list}\fi}%
        \global\advance\expandafter\noexpand
            \csname #2nr\endcsname1
        \noexpand\the\expandafter\noexpand
            \csname #2:resets\endcsname
        \startelement\noexpand\ifx=##1=\noexpand\else
            title{##1}\noexpand\fi{#1}%
        \nextpartag{H}\ignorespaces}%
    \expandafter\newtoks\csname #2:resets\endcsname
    \csname #2:resets\endcsname \tag:sect:resets
    \etoksapp\tag:sect:resets{%
        \noexpand\ensurestopelement{#1}%
        \global\expandafter\noexpand\csname#2nr\endcsname0 }}

% tagging the plain output routine
\begingroup\catcode`\@=11
\gdef\autotagplainoutput{%
    \let \vfootnot = \tag:vfootnote
    \begingroup
        % these should not be expanded
        \tag:mkrelax
            \ifhmode \ifvmode \ifnum \ifdim \ifodd \ifeven
            \ifcondition \unless \else \fi \next \tmp
            \strut \strutbox \footstrut
            \the \number \empty \folio
            \line \leftline \rightline \centerline
            \nointerlineskip \offinterlineskip
            \textindent \narrower \hang \llap \rlap
            \enspace \thinspace \hidewidth
            \quad \qquad \enskip
            \@sf \fo@t \footins
        \tag:mkrelax
        % redefine head- and footlines
        \def\footline{\taggedfootline}%
        \def\headline{\taggedheadline}%
        \xdef\makeheadline{\makeheadline}%
        \xdef\makefootline{\makefootline}%
        \xdef\footnoterule{\markartifact{Layout}\footnoterule}
        % redefine vfootnote
        \def\textindent##1{\noexpand\nextpartag{}%
            \noexpand\marknotelbl{\noexpand\textindent{##1}}%
            \noexpand\startelement{P}}%
        \xdef\vfootnote##1{\vfootnote{##1}}%
        % redefine footnote
        \let\vfootnote=\relax \let\textindent=\relax
        \def\fi##1\@sf{\noexpand\fi\noexpand\marknoteref{##1}\@sf}
        \xdef\footnote##1{\footnote{##1}}%
    \endgroup}
\endgroup
\def\tag:mkrelax#1{\ifx\tag:mkrelax#1\tag:else
    \let#1=\relax\expandafter\tag:mkrelax\tag:fi}
\let\tag:else=\else \let\tag:fi=\fi
\newtoks \taggedheadline \newtoks \taggedfootline
\taggedheadline{\tagtoks:nonempty{Pagination /Subtype/Header}\headline}
\taggedfootline{\tagtoks:nonempty{Pagination /Subtype/Footer}\footline}
\def\tagtoks:nonempty#1#2{%
    \edef\tagtoks:tmp{\the#2}%
    \ifx\tagtoks:tmp\tagtoks:hfil
        \the#2\else
        \markartifact{#1}{\the#2}\fi}
\def\tagtoks:hfil{\hfil}


% 1 tagging formulas

% \autotagformulas
% \stopformulatagging ... \startformulatagging
% \formulafilename
\newif \iftagging:indisplay
\newif \iftagging:toplevelmath \tagging:toplevelmathtrue
\newif \iftagging:formulaenabled
\newif \iftagging:tagthisformula \tagging:tagthisformulatrue
\let\startformulatagging = \tagging:formulaenabledtrue
\let\stopformulatagging = \tagging:formulaenabledfalse
\newcount \tagging:formulanr
\def\formulafilename{unnumbered-equation-\the\tagging:formulanr}
\def\autotagformulas{\startformulatagging
    \everymath\expandafter{\the\everymath
        \tagging:formula\tagging:startformula}%
    \everydisplay\expandafter{\the\everydisplay \tagging:indisplaytrue
        \tagging:formula\tagging:startdisplay}}
\def\tagging:formula#1{%
    \iftagging:enabled \iftagging:formulaenabled \iftagging:toplevelmath
        \tagging:toplevelmathfalse #1\fi\fi\fi}
\def\tagging:startformula\fi\fi\fi#1${\tagging:makeformula{#1}{$}}
\def\tagging:startdisplay\fi\fi\fi#1$${\tagging:makeformula{#1}{$$}}
\def\tagging:makeformula#1#2{\fi\fi\fi
    \tagging:tagformula{#1}%
    \scantextokens{#1}#2}
\def\tagging:tagformula#1{%
    \iftagging:formulaenabled
        \global\advance\tagging:formulanr1\relax
        \startelement attr Layout Placement \iftagging:indisplay/Block\else/Inline
            % some html converters ignore Placement attributes
            attr CSS-2.00 display (inline)\fi{Formula}%
        \formulasource:set{#1}\fi}
\def\tagging:formulasource#1{%
    $\iftagging:indisplay$\fi
    \unexpanded{#1}%
    $\iftagging:indisplay$\fi}

% \includeformulasources {alttext,actualtext,attachment}
\def\formulasource:set#1{%
    \begingroup
        % strip \setalttext, \setactualtext
        \def\setalttext##1{}\def\setactualtext##1{}%
        \xdef\formulasource:thesource{#1}%
    \endgroup
    \expandafter\splitcommalist
        \expandafter\formulasource:do
            \expandafter{\the\includeformulasources}}
\def\formulasource:do#1{%
    \expandafter\expandafter \csname formulasource:#1\endcsname
        \expandafter{\formulasource:thesource}}
\def\formulasource:actualtext#1{%
    \setactualtext{\tagging:formulasource{#1}}}
\def\formulasource:alttext#1{%
    \setalttext{\tagging:formulasource{#1}}}
\def\formulasource:attachment#1{%
    \ifnum3=\pdfaconformancelevel
        \embedfile mimetype text/x-tex
            relation Source desc {Equation source}
            name {\formulafilename.tex}
            string {\tagging:formulasource{#1}}\fi}
\def\formulasource:none#1{}
\newtoks\includeformulasources
\includeformulasources{actualtext,attachment}

% for compatibility with earlier versions (DO NOT USE)
\def\formulasource:none#1{}
\def\formulasource:both#1{%
    \formulasource:actualtext{#1}%
    \formulasource:attachment{#1}}

% 1 hyperlinks

% provided by the lua module:
%    \newdestinationname
%    \lastdestinationname

% \nameddestination {name}
\def\nameddestination{\ifhmode\expandafter\linkdest:h\else\expandafter\linkdest:v\fi}
\def\linkdest:h#1{\vadjust pre{\linkdest:v{#1}}}
\def\linkdest:v#1{\pdfextension dest name {#1} xyz\nobreak}

% \hyperlink <args> ... \endlink
%   args: alt {...} attr {...} <type>
%   type: dest {name} | url {url} | next | prev | first | last
\protected\def\endlink{\pdfextension endlink\ensurestopelement{Link}\relax}
\protected\def\startlink{\startelement{Link}\pdfextension startlink}
\protected\def\hyperlink{\quitvmode\startlink\hyper:makelink}

% \hyperlinkstyle { ... }
\def\minim:linkattr{\the\hyperlinkstyle\the\minim:pdfalinkattr}
\newtoks\hyperlinkstyle
\newtoks\minim:pdfalinkattr
\hyperlinkstyle{/Border [0 0 0]}

% 1 languages

% provided by the lua module:
%    \setdocumentlanguage {name or code}
%    \setlanguagecode {name} {code}

% \newnamedlanguage {name} {lhm} {rhm}
\def\newnamedlanguage#1#2#3{%
    \expandafter\newlanguage\csname lang@#1\endcsname
    \expandafter\chardef\csname lhm@#1\endcsname=#2\relax
    \expandafter\chardef\csname rhm@#1\endcsname=#3\relax
    \csname lu@texhyphen@loaded@\the\csname lang@#1\endcsname\endcsname}

% \newnameddialect {language} {dialect}
\def\newnameddialect#1#2{%
    \expandafter\chardef\csname lang@#2\endcsname\csname lang@#1\endcsname
    \expandafter\chardef\csname lhm@#2\endcsname\csname lhm@#1\endcsname
    \expandafter\chardef\csname rhm@#2\endcsname\csname rhm@#1\endcsname}

% and provide several dummy languages
\ifcsname lang@nohyph\endcsname \else
    \newnamedlanguage {nohyph} 1 1 \fi
\ifcsname lang@nolang\endcsname \else
    \newnameddialect {nohyph} {nolang} \fi
\ifcsname lang@uncoded\endcsname \else
    \newnameddialect {nohyph} {uncoded} \fi
\ifcsname lang@undetermined\endcsname \else
    \newnameddialect {nohyph} {undetermined} \fi

% 1 embedded files

% provided by the lua module
%    \embedfile <options>

% \setembeddedfilesmodate { yyyy-mm-dd }
\newtoks\setembeddedfilesmoddate
    \setembeddedfilesmoddate{}
\newtoks\embeddedfiles:moddate
\def\minim:makedefaultmoddate{%
    \expandafter\edef\expandafter
        \minim:tmp\expandafter{\the\setembeddedfilesmoddate}%
    \embeddedfiles:moddate\expandafter{\minim:tmp}}

% 1 declarations of pdf/a conformance

\newcount \pdfaconformancelevel
\pdfaconformancelevel = 0
\newcount \pdfuaconformancelevel
\pdfuaconformancelevel = 0

% \pdfalevel 2b
\def\pdfalevel#1#2{%
    \minim:pdfalinkattr{ /F 4}% hyperlinks must be printed
    \global\pdfaconformancelevel=#1\relax
    \ifcsname minim:pdfa:#1#2\endcsname \lastnamedcs\else
        \errmessage{Unknown pdf/a standard pdf/a-#1}\fi}

\def\minim:pdfasettings#1#2#3{%
    \pdfvariable minorversion #1\relax
    \minim:default:rgb:profile
    \if#2A\writedocumentstructure1\fi
    \input minim-xmp
    \pdfvariable omitcidset 1
    \setmetadata pdfaid:conformance{#2}%
    \setmetadata pdfaid:part{#3}}

\expandafter\def\csname minim:pdfa:1a\endcsname{\minim:pdfasettings 4A1}
\expandafter\def\csname minim:pdfa:1b\endcsname{\minim:pdfasettings 4B1}
\expandafter\def\csname minim:pdfa:2a\endcsname{\minim:pdfasettings 7A2}
\expandafter\def\csname minim:pdfa:2b\endcsname{\minim:pdfasettings 7B2}
\expandafter\def\csname minim:pdfa:2u\endcsname{\minim:pdfasettings 7U2}
\expandafter\def\csname minim:pdfa:3a\endcsname{\minim:pdfasettings 7A3}
\expandafter\def\csname minim:pdfa:3b\endcsname{\minim:pdfasettings 7B3}
\expandafter\def\csname minim:pdfa:3u\endcsname{\minim:pdfasettings 7U3}

% \pdfualavel 1
\def\pdfualevel{\input minim-xmp
    \setmetadata pdfuaid:part {1}
    \tagging:enablepdfua
    \global\pdfuaconformancelevel= }

% 

\catcode`\: = \minimpdfloaded

