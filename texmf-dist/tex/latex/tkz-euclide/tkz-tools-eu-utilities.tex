% tkz-tools-eu-utilities
% Copyright 2024  Alain Matthes
% This work may be distributed and/or modified under the
% conditions of the LaTeX Project Public License, either version 1.3
% of this license or (at your option) any later version.
% The latest version of this license is in
%   http://www.latex-project.org/lppl.txt
% and version 1.3 or later is part of all distributions of LaTeX
% version 2005/12/01 or later.
% This work has the LPPL maintenance status “maintained”.
% The Current Maintainer of this work is Alain Matthes.

\def\fileversion{5.06c}
\def\filedate{2024/02/04} 
\typeout{2024/02/04 5.06c  tkz-tools-eu-utilities.tex}  
\makeatletter
\pgfkeys{/tkzClip/.cd, 
space/.store in    = {\tkz@CLI@space},
space    = {}} 

\def\tkzClip{\pgfutil@ifnextchar[{\tkz@Clip}{\tkz@Clip[]}} 
\def\tkz@Clip[#1]{%

\pgfqkeys{/tkzClip}{#1} 
\ifx\tkzutil@empty\tkz@CLI@space
    \clip (\tkz@xa,\tkz@ya) rectangle (\tkz@xb,\tkz@yb);
 \else
 \edef\tkz@xxa{\fpeval{\tkz@xa-\tkz@CLI@space}}
 \edef\tkz@yya{\fpeval{\tkz@ya-\tkz@CLI@space}}
 \edef\tkz@xxb{\fpeval{\tkz@xb+\tkz@CLI@space}} % correction bug 2020
 \edef\tkz@yyb{\fpeval{\tkz@yb+\tkz@CLI@space}}
 \clip (\tkz@xxa,\tkz@yya) rectangle (\tkz@xxb,\tkz@yyb);
\fi
} 

% for testing results
\def\extractxy#1{%
\pgfextractx{\pgf@x}{\pgfpointanchor{#1}{center}}
\pgfextracty{\pgf@y}{\pgfpointanchor{#1}{center}}
} 
%<-------------- some tools -------------------------------------------------> 
\def\tkzutil@empty{}
\def\tkzutil@firstofone#1{#1}
\def\tkzutil@firstoftwo#1#2{#1}
\def\tkzutil@secondoftwo#1#2{#2}
%<--------------------------------------------------------------------------–>
\long\def\tkzutil@ifundefined#1{%
  \expandafter\ifx\csname#1\endcsname\relax
    \expandafter\tkzutil@firstoftwo
  \else
    \expandafter\tkzutil@secondoftwo
  \fi} 
%<--------------------------------------------------------------------------–>
\long\def\iftkznodedefined#1#2#3{%
    \tkzutil@ifundefined{pgf@sh@ns@#1}{#3}{#2}%
}
%<--------------------------------------------------------------------------–>
\gdef\tkzActivOff{%
   \edef\tkzTwoPtCode{\the\catcode`\:} 
   \edef\tkzPtExCode{\the\catcode`\!} 
   \edef\tkzPtVirCode{\the\catcode`\;} 
   \catcode`\:=12 \catcode`\!=12 \catcode`\;=12
   }%
\gdef\tkzActivOn{%
   \catcode`\:=\tkzTwoPtCode\relax
   \catcode`\!=\tkzPtExCode\relax
   \catcode`\;=\tkzPtVirCode\relax
}% 
%<--------------------------------------------------------------------------–>
%              Pour savoir le nombre de décimales d'un nombre
% le nombre dans #1 et le résultat dans \c@pgfmath@countb
%<--------------------------------------------------------------------------–>
\newcommand\tkz@getdecimal[1]{\expandafter\@getdecimal#1.\@nil}
\def\@getdecimal#1.#2\@nil{%
  \ifx\empty#2\empty
    % Si #2 est vide, c'est qu'il n'y avait pas de point
    % dans la chaîne initiale
    \c@pgfmath@countb0 %
   \gdef\tkz@decpart{}%
  \else
    % sinon c'est la chaîne qui suit le point décimal...
    \CountToken{#2}%
    % ... y compris le point en trop ajouté par \tkz@getdecimal  :-)
    \advance\c@pgfmath@countb by-1 %
    \@@getdecimal#2\@nil
  \fi
}
\def\@@getdecimal#1.\@nil{\gdef\tkz@decpart{#1}}
%<---------------------------------------------------------–>
% code from JCC modifi
\newcommand\CountToken[1]{%
 \c@pgfmath@countb0 %
  \expandafter\C@untToken#1\@nil
%modifi ajout du expandafter
}
\newcommand\C@untToken{%
  \afterassignment\C@untT@ken\let\CurrT@ken=
}
\newcommand\C@untT@ken{%
 \ifx\CurrT@ken\@nil \else
   \advance\c@pgfmath@countb by1 %
   \expandafter\C@untToken
 \fi
}
%<---------------------------------------------------------–>
\def\tkz@recuplast(#1,#2){\def\tkz@last{#1}}
\def\tkz@stop{\tkz@stop}
% end code from JCC    
%<---------------------------------------------------------–>
%            Tools
%<---------------------------------------------------------–>
%<–––––––– code from TeX in Practice ––––––––>
\def\SubStringConditional #1#2{%
    TT\fi
    \edef\@MainString{#1}%
    \edef\@SubStringConditionalTemp{{#1}{#2}}%
    \expandafter\@SubStringConditional\@SubStringConditionalTemp
}
\def\@SubStringConditional #1#2{% 
    \def\@TestSubS ##1#2##2\\{% 
        \def\@TestTemp{##1}% 
    }% 
    \@TestSubS #1#2\\
    \ifx\@MainString\@TestTemp
        \tkz@TestSubStringfalse
    \else
        \tkz@TestSubStringtrue
    \fi
    \iftkz@TestSubString
}
\def\RecursionMacroEnd #1#2#3{% 
    #1\relax
        \def\@RecursionMacroEndNext{#2}% 
    \else
        \def\@RecursionMacroEndNext{#3}% 
    \fi
    \@RecursionMacroEndNext
}
\def\ReplaceSubStrings #1#2#3#4{%
    \def\@ReplaceResult{#1}%
    \edef\@ReplaceMain{#2}%
    \edef\@ReplaceSub{#3}%
    \edef\@ReplaceSubRep{#4}%
    \@ReplaceSubStrings
}
\def\@ReplaceSubStrings{% 
    \RecursionMacroEnd
        {\if\SubStringConditional{\@ReplaceMain}{\@ReplaceSub}}%
      {\@ReplaceSubStringsDo}{\expandafter\edef\@ReplaceResult{\@ReplaceMain}}%
}
\def\@ReplaceSubStringsDoX{%
    \def\@ReplaceSubStringsDoA ##1%
}%
\def\@ReplaceSubStringsDo{% 
    \expandafter\@ReplaceSubStringsDoX \@ReplaceSub
                                    ##2\@EndReplaceSubStrings{%
        \edef\@ReplaceMain{##1\@ReplaceSubRep ##2}%
    }%
    \expandafter\@ReplaceSubStringsDoA\@ReplaceMain
        \@EndReplaceSubStrings
    \@ReplaceSubStrings
}
%<---------------------------------------------------------–>
\def\DisabledNumprint{%
      \let\tkz@numprint\numprint
      \let\numprint\relax
      }
\def\EnabledNumprint{\let\numprint\tkz@numprint} 
%<---------------------------------------------------------–>

%<---------------------------------------------------------–>
  \def\tkzHelpGrid{%
   \draw[help lines] (current bounding box.south west) grid
          (current bounding box.north east);
  }
   
\makeatother
\endinput