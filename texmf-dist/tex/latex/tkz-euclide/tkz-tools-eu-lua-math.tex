% tkz-tools-lua-math.tex
% Copyright 2024  Alain Matthes
% This work may be distributed and/or modified under the
% conditions of the LaTeX Project Public License, either version 1.3
% of this license or (at your option) any later version.
% The latest version of this license is in
%   http://www.latex-project.org/lppl.txt
% and version 1.3 or later is part of all distributions of LaTeX
% version 2005/12/01 or later.
% This work has the LPPL maintenance status “maintained”.
% The Current Maintainer of this work is Alain Matthes.

\def\fileversion{5.06c}
\def\filedate{2024/02/04} 
\typeout{2024/02/04 5.06c tkz-tools-lua-math.tex}     
\makeatletter
%<-------------------------------------------------------------------------->
%<-------------------------------------------------------------------------->
%                           Lengths
%<-------------------------------------------------------------------------->
%<-------------------------------------------------------------------------->

\begin{luacode*} 
  function normalize(angleA,angleB)       
      if angleA > 0 then
       if angleA > angleB then
          angleA = angleA - 360
          end
      else
     if angleA > angleB then
          angleB = angleB + 360
        end
      end
      return angleA, angleB
  end
  
  function math.angle(x1, y1, x2, y2)
      local a = math.deg(math.atan(y2 - y1, x2 - x1))
      if a < 0 then
          return a + 360
      else
          return a
      end
  end
  
  function tkzop(...)
  inf = math.huge
  return ...
end
 
function tkzround(nb, ND)
  local p = 10^(ND or 0)
   return math.floor(nb * p + 0.5) / p
end   
\end{luacode*}

\def\tkz@Dec#1{%
     \directlua{tex.print(string.format('\@percentchar.6f',#1))}
     }
\def\tkz@Op#1{\directlua{tex.sprint(tostring(tkzop(#1)))}}
\def\tkz@Log#1{\directlua{tex.sprint(math.log(#1))}}
\def\tkz@Exp#1{\directlua{tex.sprint(math.exp(#1))}}
\def\tkz@Sqrt#1{\directlua{tex.sprint(math.sqrt(#1))}}
\def\tkz@Abs#1{\directlua{tex.sprint(math.abs(#1))}}
\def\tkz@Pi{\directlua{tex.sprint(math.pi)}}
\def\tkz@Cos#1{\directlua{tex.sprint(math.cos(#1))}}
\def\tkz@Sin#1{\directlua{tex.sprint(tostring(math.sin(#1)))}}
\def\tkz@Tan#1{\directlua{tex.sprint(math.tan(#1))}}
\def\tkz@Rad#1{\directlua{tex.sprint(math.rad(#1))}}
\def\tkz@Acos#1{\directlua{tex.sprint(math.acos(#1))}}
\def\tkz@Asin#1{\directlua{tex.sprint(math.asin(#1))}}
\def\tkz@Atan#1{\directlua{tex.sprint(math.atan(#1))}}
\def\tkz@Round#1#2{\directlua{tex.sprint(tostring(tkzround(#1,#2)))}}
\def\tkz@Angle#1#2#3#4{\directlua{tex.sprint(math.angle(#1,#2,#3,#4))}}
\def\tkz@Ceil#1{\directlua{tex.sprint(math.ceil(#1))}}
\def\tkz@Floor#1{\directlua{tex.sprint(math.floor(#1))}}
\def\tkz@Huge{\directlua{tex.sprint(math.huge)}}
\def\tkz@Max#1{\directlua{tex.sprint(math.max(#1))}}
\def\tkz@Min#1{\directlua{tex.sprint(math.min(#1))}}
\def\tkz@Random#1{\directlua{tex.sprint(math.random(#1))}}
\def\tkz@veclen#1#2{%
  \directlua{%
    tex.print(string.format('\@percentchar.6f',math.sqrt((#1)^2+(#2)^2)))%
  }%
}
\let\tkzSqrt\tkz@Sqrt
\let\tkzPi\tkz@Pi
\let\tkzExp\tkz@Exp
\let\tkzLog\tkz@Log
\let\tkzSin\tkz@Sin
\let\tkzCos\tkz@Cos

%  \tkzpointnormalised    normalise un point A-->A' tq ||v(OA')=1||
% example
% \tkzpointnormalised{%
% \pgfpointdiff{\pgfpointanchor{A}{center}}
%              {\pgfpointanchor{B}{center}}}

% or
% \pgf@x=1 cm
% \pgf@y=12 cm 
% \tkzpointnormalised{}
%<--------------------------------------------------------------------------
\def\tkzpointnormalised#1{%
\pgf@process{#1}%
\pgf@xa=\pgf@x%
\pgf@ya=\pgf@y%
\edef\tkz@temp@xa{\strip@pt\pgf@xa}%
\edef\tkz@temp@ya{\strip@pt\pgf@ya}%
\edef\tkz@den{\tkz@veclen{\tkz@temp@xa}{\tkz@temp@ya}}
\edef\tkz@coordx{\tkz@Op{\tkz@temp@xa/\tkz@den}}
\edef\tkz@coordx{\tkz@Dec{\tkz@Round{\tkz@coordx}{5}}}
\edef\tkz@coordy{\tkz@Op{\tkz@temp@ya/\tkz@den}}
\edef\tkz@coordy{\tkz@Dec{\tkz@Round{\tkz@coordy}{5}}}
\pgf@x = \tkz@coordx pt
\pgf@y = \tkz@coordy pt
}
%\def\tkz@Dec#1{\directlua{tex.print(string.format('\@percentchar.12f',#1))}}
%<-------------------------------------------------------------------------->
% restaure and save length
\def\tkz@save@length{\global\let\tkz@temp@length\tkzLengthResult}%
\def\tkz@restore@length{\global\let\tkzLengthResult\tkz@temp@length }% 
%<-------------------------------------------------------------------------->
%    \tkzCalcLength      Distance entre deux points en pt ou en cm  avec xfp 
% \veclen mais avec fp 
%  option cm le résultat est en cm sinon en pt with cm=false
%<-------------------------------------------------------------------------->
\pgfkeys{tkzcalclen/.cd,
       cm/.is if         = tkzLengthIncm,
       cm/.default       = true,
       cm                = true}   

\def\tkzCalcLength{\pgfutil@ifnextchar[{\tkz@CalcLength}{\tkz@CalcLength[]}}  
\def\tkz@CalcLength[#1](#2,#3){%
\pgfqkeys{/tkzcalclen}{#1}%   
\begingroup
\tkz@@CalcLength(#2,#3){tkzLengthResult}
\iftkzLengthIncm 
   \edef\tkz@xfpMathLen{\tkz@Dec{\tkz@Round{\tkzLengthResult/28.45274}{6}}}
   \global\let\tkzLengthResult\tkz@xfpMathLen  
\fi 
\endgroup
}%

\def\tkz@@CalcLength(#1,#2)#3{%
\pgfpointdiff{\pgfpointanchor{#1}{center}}%
             {\pgfpointanchor{#2}{center}}%
\edef\tkz@xa{\strip@pt\pgf@x}%
\edef\tkz@ya{\strip@pt\pgf@y}%
\edef\tkz@xfpMathLen{\tkz@veclen{\tkz@xa}{\tkz@ya}}
\global\expandafter\edef\csname #3\endcsname{\tkz@xfpMathLen}
}

\def\tkz@@CalcLengthcm(#1,#2)#3{%
\pgfpointdiff{\pgfpointanchor{#1}{center}}%
             {\pgfpointanchor{#2}{center}}%
\edef\tkz@xa{\strip@pt\pgf@x}%
\edef\tkz@ya{\strip@pt\pgf@y}%
\edef\tkz@xfpMathLen{\tkz@veclen{\tkz@xa}{\tkz@ya}}
\edef\tkz@xfpMathLen{\tkz@Dec{\tkz@Round{\tkz@xfpMathLen/28.45274}{6}}}
\global\expandafter\edef\csname #3\endcsname{\tkz@xfpMathLen}
}
\def\tkz@@CalcLengthb(#1,#2)#3{%
\pgfpointdiff{\pgfpointanchor{#1}{center}}%
             {\pgfpointanchor{#2}{center}}%
\edef\tkz@xfpMathLen{\fpeval{sqrt((\pgf@x)^2+(\pgf@y)^2)}}
\edef\tkz@xfpMathLen{\fpeval{round(\tkz@xfpMathLen,6)}}
\global\expandafter\edef\csname #3\endcsname{\tkz@xfpMathLen}
}
%<-------------------------------------------------------------------------->
\def\tkzGetLength#1{%
\global\expandafter\edef\csname #1\endcsname{\tkzLengthResult}}  
%<-------------------------------------------------------------------------->
%     \tkzpttocm  passage de pt   cm div par 28.45274
%<-------------------------------------------------------------------------->
\def\tkzpttocm(#1)#2{%
\begingroup  
  \edef\tkz@mathresult{\tkz@Round{#1/28.45274}{6}}
 \global\expandafter\edef\csname #2\endcsname{\tkz@mathresult}%
\endgroup
}%
%<-------------------------------------------------------------------------->
%     \tkzcmtopt  passage de cm   pt mul par 28.45274
%<--------------------------------------------------------------------------
\def\tkzcmtopt(#1)#2{%
\begingroup 
  \edef\tkz@mathresult{\tkz@Round{#1*28.45274}{6}}
  \global\expandafter\edef\csname #2\endcsname{\tkz@mathresult}% 
\endgroup  
}% 
%<---------------------------------------------------------–>
 \def\tkzGetResult#1{%
   \global\expandafter\edef\csname #1\endcsname{\tkzMathResult}}
%<---------------------------------------------------------–>
%  Schrodinger's cat idea 03/01/20
\tikzset{veclen/.code={%
\pgfmathdeclarefunction*{veclen}{2}{%
\begingroup%
    \pgfmath@x##1pt\relax%
    \pgfmath@y##2pt\relax%
    \pgf@xa=\pgf@x%
    \pgf@ya=\pgf@y%
    \edef\tkz@temp@xa{\strip@pt\pgf@xa}% 
    \edef\tkz@temp@ya{\strip@pt\pgf@ya}%
    \edef\tkz@xfpMathLen{\tkz@veclen{\tkz@temp@xa}{\tkz@temp@ya}}%
    \pgfmath@returnone\tkz@xfpMathLen pt%
\endgroup%
}}}%
%<---------------------------------------------------------–>
\def\tkzSwapPoints(#1,#2){
   \pgfnodealias{tkzPointTmp}{#2}
   \pgfnodealias{#2}{#1}
   \pgfnodealias{#1}{tkzPointTmp}}
%<---------------------------------------------------------–>
\def\tkzPermute(#1,#2,#3){
   \tkzURotateWithNodes(#1,#3,#2)(#3)  \tkzGetPoint{tkzpt}
   \tkzURotateWithNodes(#1,#2,#3)(#2)  \tkzGetPoint{#2}
   \tkzSwapPoints(tkzpt,#3)
}
%<---------------------------------------------------------–>
\def\tkzDotProduct(#1,#2,#3){%
\begingroup
\pgfextractx{\pgf@x}{\pgfpointanchor{#1}{center}}%
\pgfextracty{\pgf@y}{\pgfpointanchor{#1}{center}}% 
\edef\tkzax{\strip@pt\pgf@x}%
\edef\tkzay{\strip@pt\pgf@y}%
\pgfextractx{\pgf@x}{\pgfpointanchor{#2}{center}}%
\pgfextracty{\pgf@y}{\pgfpointanchor{#2}{center}}% 
\edef\tkzbx{\strip@pt\pgf@x}%
\edef\tkzby{\strip@pt\pgf@y}%
\pgfextractx{\pgf@x}{\pgfpointanchor{#3}{center}}%
\pgfextracty{\pgf@y}{\pgfpointanchor{#3}{center}}% 
\edef\tkzcx{\strip@pt\pgf@x}%
\edef\tkzcy{\strip@pt\pgf@y}%
\edef\tkz@tmp{\tkz@Dec{\tkz@Round{((\tkzbx-(\tkzax))*(\tkzcx-(\tkzax))+(\tkzby-(\tkzay))*(\tkzcy-(\tkzay)))/809.55841}{6}}}
\global\let\tkzMathResult\tkz@tmp
\endgroup
}

% #1,#2 and #3 aligned
\def\tkzIsLinear(#1,#2,#3){%
\begingroup
\tkz@@CalcLengthcm(#1,#2){tkz@la}
\tkz@@CalcLengthcm(#1,#3){tkz@lb}
\tkzDotProduct(#1,#2,#3)
\edef\tkzResult{\tkz@Dec{\tkz@Abs{\tkzMathResult}-(\tkz@la)*(\tkz@lb)}}
\ifdim \tkzResult pt < 0.01 pt\relax%
\global\tkzLineartrue
\else
\global\tkzLinearfalse
\fi
\endgroup
}
%<---------------------------------------------------------–>
% syntax : vec(#2,#1) ortho vec(#3,#1)
\def\tkzIsOrtho(#1,#2,#3){%
\begingroup
\tkzDotProduct(#1,#2,#3)
\edef\tkzResult{\tkz@Dec{\tkz@Abs{\tkzMathResult}}}
\ifdim \tkzResult pt < 1 pt\relax%
\global\tkzOrthotrue
\else
\global\tkzOrthofalse
\fi
\endgroup
}
%<---------------------------------------------------------–>
% \tkzPowerCircle(M)(O,A) --> OM^2-OA^2
\def\tkzPowerCircle(#1)(#2,#3){%     
\begingroup 
\tkz@@CalcLengthcm(#2,#3){tkz@ra}
\tkz@@CalcLengthcm(#1,#2){tkz@om}
\gdef\tkzMathResult{\tkz@Dec{(\tkz@om)^2-(\tkz@ra)^2}}
\endgroup
}
%<---------------------------------------------------------–>
\def\tkzDefRadicalAxis(#1,#2)(#3,#4){%
\begingroup
\tkz@@CalcLengthcm(#1,#3){tkz@d}
\tkz@@CalcLengthcm(#1,#2){tkz@ra}
\tkz@@CalcLengthcm(#3,#4){tkz@rb}
\edef\tkzMathResult{\tkz@Dec{\tkz@d-(\tkz@ra+\tkz@rb)}}
\edef\tkzMathResultb{\tkz@Dec{\tkz@Abs{(\tkz@d-(\tkz@ra+\tkz@rb))}}}
\edef\tkzMathResultc{\tkz@Dec{\tkz@Abs{\tkz@d-\tkz@Abs{(\tkz@ra-(\tkz@rb))}}}}
\ifdim \tkzMathResultc pt < 0.1 pt\relax%
 \tkzURotateAngle(#2,90)(#3) \tkzGetPoint{tkzFirstPointResult}
  \tkzURotateAngle(#2,-90)(#3) \tkzGetPoint{tkzSecondPointResult}
\else
\ifdim \tkzMathResultb pt < 0.1 pt\relax%
 \tkzURotateAngle(#2,90)(#3) \tkzGetPoint{tkzFirstPointResult}
  \tkzURotateAngle(#2,-90)(#3) \tkzGetPoint{tkzSecondPointResult}
  \else
\ifdim \tkzMathResult pt > 1 pt\relax%
  \tkzURotateAngle(#1,60)(#3)    \tkzGetPoint{tkz@aux}
  \tkzInterCC(#1,#2)(tkz@aux,#1) \tkzGetPoints{tkz@pta}{tkz@ptb}
  \tkzInterCC(#3,#4)(tkz@aux,#1) \tkzGetPoints{tkz@ptc}{tkz@ptd}
  \tkzInterLL(tkz@pta,tkz@ptb)(tkz@ptc,tkz@ptd) \tkzGetPoint{tkz@pta}
  \tkzUProjection(#1,#3)(tkz@pta)  \tkzGetPoint{tkz@ptb}
  \pgfnodealias{tkzSecondPointResult}{tkz@ptb}
  \pgfnodealias{tkzFirstPointResult}{tkz@pta}
\else
\tkzInterCCR(#1,\tkz@ra)(#3,\tkz@rb){tkzFirstPointResult}{tkzSecondPointResult}
\fi
\fi
\fi
\endgroup
} 

\makeatother
\endinput