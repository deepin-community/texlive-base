% tkz-obj-lua-circles.tex
% Copyright 2024  Alain Matthes
% This work may be distributed and/or modified under the
% conditions of the LaTeX Project Public License, either version 1.3
% of this license or (at your option) any later version.
% The latest version of this license is in
%   http://www.latex-project.org/lppl.txt
% and version 1.3 or later is part of all distributions of LaTeX
% version 2005/12/01 or later.
% This work has the LPPL maintenance status “maintained”.
% The Current Maintainer of this work is Alain Matthes.

\def\fileversion{5.06c}
\def\filedate{2024/02/04} 
\typeout{2024/02/04 5.06c tkz-obj-lua-circles.tex} 
\makeatletter
%<--------------------------------------------------------------------------–>
%            tkzCircle center and one point 
%<--------------------------------------------------------------------------–>
% attention radius circle is defined by center and a point on the circle
% R defined by center and the value of the radius
% no need to define a circle with R tikz uses this method.
% through instead of radius
\def\tkz@numc{0}
\pgfkeys{/tkzcircle/.cd,
      R/.code                                   = \def\tkz@numc{0},
      diameter/.code                            = \def\tkz@numc{1},
      circum/.code                              = \def\tkz@numc{2},
      in/.code                                  = \def\tkz@numc{3},
      ex/.code                                  = \def\tkz@numc{4},
      euler/.code                               = \def\tkz@numc{5},
      nine/.code                                = \def\tkz@numc{5},
      apollonius/.code                          = \def\tkz@numc{6},
      spieker/.code                             = \def\tkz@numc{7},
      orthogonal from/.code args                = {#1}{\gdef\tkz@numc{8}
                                                   \def\tkz@ptfrom{#1}}, 
      orthogonal through/.code args             = {#1 and #2}{\gdef\tkz@numc{9}
                                                   \def\tkz@ptone{#1}
                                                   \def\tkz@pttwo{#2}},
      K/.code                                   = \def\tkz@koeff{#1},
      K                                         =  1,
      circum
      } 
\def\tkzDefCircle{\pgfutil@ifnextchar[{\tkz@DefCircle}{\tkz@DefCircle[]}}
\def\tkz@DefCircle[#1](#2){% 
\begingroup  
\pgfqkeys{/tkzcircle}{#1}  
\ifcase\tkz@numc%
  \tkzDefCircleR(#2)
  \or% 1 
  \tkzDefCircleD(#2)  
  \or% 2
  \tkzDefCircumCircle(#2) 
  \or% 3
  \tkzDefInCircle(#2) 
  \or% 4
  \tkzDefExCircle(#2) 
  \or% 5
  \tkzDefEulerCircle(#2)
  \or% 6
  \tkzDefApolloniusCircle(#2) 
  \or% 7
  \tkzDefSpiekerCircle(#2)
   \or% 8
  \tkzDefOrthogonalCircle(#2,\tkz@ptfrom)
   \or% 9
  \tkzDefOrthoThroughCircle(#2,\tkz@ptone,\tkz@pttwo)
     \fi    
\endgroup
}
%for compatibility
%<--------------------------------------------------------------------------–>
%                     R
%<--------------------------------------------------------------------------–>
\def\tkzDefCircleR(#1,#2){% 
\begingroup 
   \edef\tkzLengthResult{\fpeval{round(#2,5)}}
   \global\let\tkzLengthResult\tkzLengthResult
   \path (#1)--++(\tkzLengthResult,0) coordinate (tkzSecondPointResult);
   \tkzRenamePoint(tkzSecondPointResult){tkzPointResult}
\endgroup
} 

%<--------------------------------------------------------------------------–>
%              Diameter Circle
%<--------------------------------------------------------------------------–>
\def\tkzDefCircleD(#1,#2){% 
\begingroup
  \tkzDefMidPoint(#1,#2)
  \pgfnodealias{tkzFirstPointResult}{tkzPointResult}
  \pgfnodealias{tkzSecondPointResult}{#2}
  \tkz@@CalcLengthcm(#1,tkzPointResult){tkzLengthResult}
\endgroup
} 
%<--------------------------------------------------------------------------–>
%              Circum Circle
%<--------------------------------------------------------------------------–>
\def\tkzDefCircumCircle(#1,#2,#3){%
\begingroup
  \tkzCircumCenter(#1,#2,#3)
  \tkzRenamePoint(tkzPointResult){tkzFirstPointResult}
  \tkzRenamePoint(#1){tkzSecondPointResult}
  \tkz@@CalcLengthcm(#1,tkzPointResult){tkzLengthResult}%3.06 add [cm]
\endgroup
} 
%<--------------------------------------------------------------------------–>
%              In(scribe) Circle
%<--------------------------------------------------------------------------–>
\def\tkzDefInCircle(#1,#2,#3){%
\begingroup
  \tkzInCenter(#1,#2,#3) 
  \pgfnodealias{tkz@incenter}{tkzPointResult}
  \tkzUProjection(#1,#3)(tkz@incenter) 
  \pgfnodealias{tkzSecondPointResult}{tkzPointResult}
  \tkz@@CalcLengthcm(tkzPointResult,tkz@incenter){tkzLengthResult}
  \pgfnodealias{tkzPointResult}{tkz@incenter}
  \pgfnodealias{tkzFirstPointResult}{tkz@incenter}
\endgroup
}  
%<--------------------------------------------------------------------------–>
%              Ex(scribe) Circle
%<--------------------------------------------------------------------------–>
\def\tkzDefExCircle(#1,#2,#3){%
\begingroup
  \tkzExCenter(#1,#2,#3) 
  \pgfnodealias{tkz@excenter}{tkzPointResult}
  \tkzUProjection(#1,#3)(tkz@excenter)
  \pgfnodealias{tkzSecondPointResult}{tkzPointResult}
  \tkz@@CalcLengthcm(tkzPointResult,tkz@excenter){tkzLengthResult}% for tkzGetLength
  \pgfnodealias{tkzPointResult}{tkz@excenter}
  \pgfnodealias{tkzFirstPointResult}{tkz@excenter}  
\endgroup
}
%<--------------------------------------------------------------------------–>
%              Radius Ex Circle
%<--------------------------------------------------------------------------–>
\def\tkzDefExRadius(#1,#2,#3){%
\begingroup
   \tkzExCenter(#1,#2,#3)
   \tkzUProjection(#1,#3)(tkzPointResult)
\endgroup
}    
%<--------------------------------------------------------------------------–>
% The nine-point circle, also called Euler's circle or the Feuerbach circle  
% best way Ma,Mb,Mc circum center  2020
%<--------------------------------------------------------------------------–>
\def\tkzDefEulerCircle(#1,#2,#3){%
\begingroup
 \tkzDefMidPoint(#1,#2) \pgfnodealias{tkz@e1}{tkzPointResult}
 \tkzDefMidPoint(#2,#3) \pgfnodealias{tkz@e2}{tkzPointResult}
 \tkzDefMidPoint(#1,#3) \pgfnodealias{tkz@e3}{tkzPointResult}
 \tkzCircumCenter(tkz@e1,tkz@e2,tkz@e3) 
  \pgfnodealias{tkzFirstPointResult}{tkzPointResult}  
 \tkzRenamePoint(tkz@e1){tkzSecondPointResult}
 \tkz@@CalcLengthcm(tkzPointResult,tkz@e1){tkzLengthResult}
\endgroup
} 
\let\tkzDefNinePointsCircle\tkzEulerCircle%
\let\tkzFeuerBachCircle\tkzEulerCircle%  

\def\tkzDefEulerRadius(#1,#2,#3){%
\begingroup
   \tkzEulerCenter(#1,#2,#3)
   \pgfnodealias{eur@pta}{tkzPointResult}   
   \tkzDefMidPoint(#1,#2)  
   \tkz@@CalcLengthcm(eur@pta,tkzPointResult){tkzLengthResult}
\endgroup
} 
%<--------------------------------------------------------------------------–>
%                     Apollonius circle
%<--------------------------------------------------------------------------–>
\def\tkzDefApolloniusCircle(#1,#2){%
\begingroup
  \tkzDefBarycentricPointTwo(#1,1,#2,\tkz@koeff)   
  \pgfnodealias{apo@pta}{tkzPointResult} 
  \pgfnodealias{tkzSecondPointResult}{tkzPointResult}  
  \tkzDefBarycentricPointTwo(#1,1,#2,-\tkz@koeff)  
  \pgfnodealias{apo@ptb}{tkzPointResult} 
  \tkzDefMidPoint(apo@pta,apo@ptb)  
  \pgfnodealias{tkzFirstPointResult}{tkzPointResult} 
  \tkz@@CalcLengthcm(tkzFirstPointResult,apo@pta){tkzLengthResult}  
  \pgfnodealias{tkzThirdPointResult}{apo@ptb} 
\endgroup
}
%<--------------------------------------------------------------------------–>
\def\tkzDefOrthogonalCircle(#1,#2,#3){%   
\begingroup
   \tkzTgtFromP(#1,#2)(#3)
   \tkz@@CalcLengthcm(#1,tkzFirstPointResult){tkzLengthResult}
\endgroup
} 
%<--------------------------------------------------------------------------–>
\def\tkzDefOrthoThroughCircle(#1,#2,#3,#4){%  
\begingroup
  \tkz@@CalcLengthcm(#1,#3){tkz@lnb}%
  \edef\tkz@lnc{\fpeval{1/\tkz@lnb}}
  \tkzVecKNorm[\tkz@lnc](#1,#3) 
  \pgfnodealias{tkz@PointResult}{tkzPointResult}
  \tkzCircumCenter(tkz@PointResult,#3,#4)
  \tkz@@CalcLengthcm(tkzPointResult,#3){tkzLengthResult}
  \pgfnodealias{tkzFirstPointResult}{tkzPointResult}
  \pgfnodealias{tkzSecondPointResult}{#3}
\endgroup
} 
%<--------------------------------------------------------------------------–>
%              Spieker Circle
%<--------------------------------------------------------------------------–>
\def\tkzDefSpiekerCircle(#1,#2,#3){%
\begingroup
  \tkzSpiekerCenter(#1,#2,#3)
  \pgfnodealias{tkzFirstPointResult}{tkzPointResult}
  \tkzUProjection(tkz@m1,tkz@m2)(tkzPointResult)
  \pgfnodealias{tkzSecondPointResult}{tkzPointResult}
  \tkz@@CalcLength(tkzSecondPointResult,tkzFirstPointResult){tkzLengthResult}
\endgroup
} 
%<--------------------------------------------------------------------------–>

%<--------------------------------------------------------------------------–>
%              End Def Circle
%<--------------------------------------------------------------------------–>  
\makeatother 
\endinput