% tkz-tool-eu-angles.tex
% Copyright 2024  Alain Matthes
% This work may be distributed and/or modified under the
% conditions of the LaTeX Project Public License, either version 1.3
% of this license or (at your option) any later version.
% The latest version of this license is in
%   http://www.latex-project.org/lppl.txt
% and version 1.3 or later is part of all distributions of LaTeX
% version 2005/12/01 or later.
% This work has the LPPL maintenance status “maintained”.
% The Current Maintainer of this work is Alain Matthes.

\def\fileversion{5.06c}
\def\filedate{2024/02/04} 
\typeout{2024/02/04 5.06c tkz-tool-eu-angles.tex} 
\makeatletter
%<--------------------------------------------------------------------------–>
%                    tkzSetUpArc  
%<--------------------------------------------------------------------------–>
\pgfkeys{/tkzsetuparc/.cd,
   color/.store in        =    \tkz@arc@color,
   line width/.store in   =    \tkz@arc@lw,
   style/.store in        =    \tkz@arc@style,
   line width             =    \tkz@euc@linewidth,
   color                  =    \tkz@euc@linecolor,
   style                  =    \tkz@euc@linestyle,
   /tkzsetuparc/.search also = {/tikz,/tkzcompass},
 } 
\def\tkzSetUpArc{\pgfutil@ifnextchar[{\tkz@SetUpArc}{\tkz@SetUpArc[]}}
\def\tkz@SetUpArc[#1]{%
\pgfqkeys{/tkzsetuparc}{#1}
\tikzset{arc style/.style={color          = \tkz@arc@color,
                           line width     = \tkz@arc@lw,
                           style          = \tkz@arc@style
                           }}
}% end setup 
%<------------------------------  Arcs -------------------------------------–
% options : delta 
% \def\tkz@delta{0}
% \tikzset{arc style/.style={#1}}
% \pgfkeys{/tikz/.cd,delta/.code={\def\tkz@delta{#1}}}
\newif\iftkz@reverse
\gdef\tkz@numa{0}
\pgfkeys{/tkzdrawarc/.cd,
      type/.is choice,
      type/towards/.code               = \def\tkz@numa{0},
      type/rotate/.code                = \def\tkz@numa{1},
      type/angles/.code                = \def\tkz@numa{2}, 
      type/R/.code                     = \def\tkz@numa{3},
      type/R with nodes/.code          = \def\tkz@numa{4},
      towards/.style                   = {type=towards},
      rotate/.style                    = {type=rotate},
      R/.style                         = {type=R},
      angles/.style                    = {type=angles},
      R with nodes/.style              = {type=R with nodes},
      diameter/.code                   = {},
      arc/.code                        = {},
      size/.code                       = {},
      mark/.code                       = {},
      mkpos/.code                      = {},
      mksize/.code                     = {},
      mkcolor/.code                    = {},
      type/.default                    =  towards,
      delta/.store in                  = \tkz@delta,
      delta                            = 0,
      reverse/.is if                   = tkz@reverse,
      reverse/.default                 = true,
      reverse                          = false,
      /tkzdrawarc/.search also         = {/tikz}
}  
\def\tkzDrawArc{\pgfutil@ifnextchar[{\tkz@DrawArc}{\tkz@DrawArc[]}}
\def\tkz@DrawArc[#1](#2,#3)(#4){% 
\begingroup    
\pgfqkeys{/tkzdrawarc}{#1} 
\ifcase\tkz@numa%
   \tkzDrawArcTowards[#1](#2,#3)(#4)  
\or% 1
   \tkzDrawArcRotate[#1](#2,#3)(#4)  
\or% 2
   \tkzDrawArcAngles[#1](#2,#3)(#4) 
\or% 3
   \tkzDrawArcRAngles[#1](#2,#3)(#4)
\or% 4
   \tkzDrawArcR[#1](#2,#3)(#4)  
\fi    
\endgroup
}
%<--------------------------------------------------------------------------–>
%  ARC    nodes        #2 center #3 first point last point #4
% delta un peu plus à chaque extrémité 
% example : \tkzDrawArc(A,B)(C)      
%<--------------------------------------------------------------------------–>
\def\tkzDrawArcTowards{\pgfutil@ifnextchar[{\tkz@DrawArcN}{\tkz@DrawArcN[]}} 
\def\tkz@DrawArcN[#1](#2,#3)(#4){%
\begingroup
  \tkz@@CalcLengthcm(#2,#3){tkz@radius}
  \tkzFindSlopeAngle(#2,#3)\tkzGetAngle{tkz@FirstAngle}  
  \tkzFindSlopeAngle(#2,#4)\tkzGetAngle{tkz@SecondAngle}
  \tkz@DrawArcRAngles[#1](#2,\tkz@radius)(\tkz@FirstAngle,\tkz@SecondAngle) 
\endgroup
} 
%<--------------------------------------------------------------------------–>
%    nodes                 #2 center #3 first point rotate #4 with Angle
% delta un peu plus à chaque extrémité 
% tkzDrawArcRotate(O,A)(60)
%<--------------------------------------------------------------------------–>
\def\tkzDrawArcRotate{\pgfutil@ifnextchar[{\tkz@DrawArcRotate}{%
                                           \tkz@DrawArcRotate[]}}  
\def\tkz@DrawArcRotate[#1](#2,#3)(#4){%
\begingroup
  \tkz@@CalcLength(#2,#3){tkz@radius} 
  \tkzFindSlopeAngle(#2,#3)   \tkzGetAngle{tkz@FirstA}
  \pgfmathadd{\tkz@FirstA}{#4}
  \edef\tkz@SecondA{\pgfmathresult}
    \pgfmathgreaterthan{#4}{0}
  \ifdim\pgfmathresult pt=1 pt\relax%
    \tkz@DrawArcRAngles[#1](#2,\tkz@radius pt)(\tkz@FirstA,\tkz@SecondA)
  \else
    \tkz@DrawArcRAngles[#1](#2,\tkz@radius pt)(\tkz@SecondA,\tkz@FirstA)
  \fi
  \endgroup
}   
%<--------------------------------------------------------------------------–>
%  deux angles 
% \tkzDrawArcAngles(O,A)(0,60)
%<--------------------------------------------------------------------------–>
\def\tkzDrawArcAngles{\pgfutil@ifnextchar[{\tkz@DrawArcAngles}{%
                                           \tkz@DrawArcAngles[]}}  
\def\tkz@DrawArcAngles[#1](#2,#3)(#4,#5){%
\begingroup
    \tkz@@CalcLength(#2,#3){tkzLengthResult}
    \tkz@DrawArcRAngles[#1](#2,\tkzLengthResult pt)(#4,#5)      
\endgroup
}   
%<--------------------------------------------------------------------------–>
%    Degree      #2 center #4 - #3 radius from #5 (degree) to #6(degree) 
%<--------------------------------------------------------------------------–>
\def\tkzDrawArcRwithNodes{\pgfutil@ifnextchar[{\tkz@DrawArcRwithNodes}{%
                                            \tkz@DrawArcRwithNodes[]}}  
\def\tkz@DrawArcRwithNodes[#1](#2,#3,#4)(#5,#6){%
\begingroup
  \tkz@@CalcLength(#3,#4){tkzLengthResult}  
  \tkzFindSlopeAngle(#2,#5)\tkzGetAngle{tkz@FirstAngle}  
  \tkzFindSlopeAngle(#2,#6)\tkzGetAngle{tkz@SecondAngle}
  \tkz@DrawArcRAngles[#1](#2,\tkzLengthResult)(\tkz@FirstAngle,\tkz@SecondAngle)
\endgroup   
} 
%<--------------------------------------------------------------------------–>
%    Nodes R  #2 center #3 radius en cm  from #4(node) to #5(node)
%  \tkzDrawArcR(O,2)(A,B)
%<--------------------------------------------------------------------------–>
\def\tkzDrawArcR{\pgfutil@ifnextchar[{\tkz@DrawArcR}{\tkz@DrawArcR[]}} 
\def\tkz@DrawArcR[#1](#2,#3)(#4,#5){%
\begingroup
   \tkzFindSlopeAngle(#2,#4)\tkzGetAngle{tkz@FirstAngle}  
   \tkzFindSlopeAngle(#2,#5)\tkzGetAngle{tkz@SecondAngle} 
   \tkz@DrawArcRAngles[#1](#2,#3)(\tkz@FirstAngle,\tkz@SecondAngle)  
\endgroup   
}
%<--------------------------------------------------------------------------–>
%<--------------------------------------------------------------------------–>
% #1 center #2 radius #4 first angle (degree) #5 second angle  (degree)  
% angles  0 .. 180 or -180 .. 0
%<--------------------------------------------------------------------------–>
% example : \tkzDrawArc(A,2)(30,90)    
\def\tkzDrawArcRAngles{\pgfutil@ifnextchar[{\tkz@DrawArcRAngles}{%
                                            \tkz@DrawArcRAngles[]}} 
\def\tkz@DrawArcRAngles[#1](#2,#3)(#4,#5){% 
 \begingroup
 \tkzNormalizeAngle(#4,#5)    
 \pgfmathsubtract{\tkz@FirstAngle}{\tkz@delta}
 \edef\tkz@FirstAngle{\pgfmathresult}%
 \pgfmathadd{\tkz@SecondAngle}{\tkz@delta}
 \edef\tkz@SecondAngle{\pgfmathresult} 
 \iftkz@reverse 
   \let\tkztemp\tkz@FirstAngle 
   \let\tkz@FirstAngle\tkz@SecondAngle
   \let\tkz@SecondAngle\tkztemp
  \fi
  \draw[shift = {(#2)},arc style,/tkzdrawarc/.cd,#1]%
  (\tkz@FirstAngle:#3) arc (\tkz@FirstAngle:\tkz@SecondAngle:#3);
\endgroup  
}
%<--------------------------------------------------------------------------–>
\pgfkeys{
  /@tkzmarkarc/.cd,
     pos/.store in       = \tkz@mkarcpos,
     color/.store in     = \tkz@mkcolor,
     mark/.store in      = \tkz@markarcseg,
     size/.store in      = \tkz@mkarcsize,
     color               = teal!20,
     size                = 4pt,
     pos                 = .5,
     mark                = none,
    /@tkzmarkarc/.search also={/tikz},
}
\def\tkzMarkArc{\pgfutil@ifnextchar[{\tkz@MarkArc}{\tkz@MarkArc[]}}   
\def\tkz@MarkArc[#1](#2,#3,#4){% 
\begingroup
 \pgfqkeys{/@tkzmarkarc}{#1} \def\tkz@mymarkarc{\pgfsetplotmarksize{\tkz@mkarcsize}\pgfuseplotmark{\tkz@markarcseg}}
\tkz@@CalcLength(#2,#3){tkz@radius}
\tkzFindSlopeAngle(#2,#3)\tkzGetAngle{tkz@FirstAngle}  
\tkzFindSlopeAngle(#2,#4)\tkzGetAngle{tkz@SecondAngle}
\tkzNormalizeAngle(\tkz@FirstAngle,\tkz@SecondAngle)
 \pgfmathsubtract{\tkz@FirstAngle}{\tkz@delta}
 \edef\tkz@FirstAngle{\pgfmathresult}%
 \pgfmathadd{\tkz@SecondAngle}{\tkz@delta}
 \edef\tkz@SecondAngle{\pgfmathresult} 
\begin{scope}[decoration={markings,mark=at position \tkz@mkarcpos with {\tkz@mymarkarc}}]     
\path[shift = {(#2)},\tkz@mkcolor,/@tkzmarkarc/.cd,#1,postaction={decorate}]%
(\tkz@FirstAngle:\tkz@radius pt) arc (\tkz@FirstAngle:\tkz@SecondAngle:\tkz@radius pt);
\end{scope}
\endgroup 
}

\def\tkzLabelArc{\pgfutil@ifnextchar[{\tkz@LabelArc}{\tkz@LabelArc[]}}
\def\tkz@LabelArc[#1](#2,#3,#4)#5{%
\begingroup
\pgfqkeys{/@tkzmarkarc}{#1}
\tkz@@CalcLength(#2,#3){tkz@radius}
\tkzFindSlopeAngle(#2,#3)\tkzGetAngle{tkz@FirstAngle}  
\tkzFindSlopeAngle(#2,#4)\tkzGetAngle{tkz@SecondAngle}
\tkzNormalizeAngle(\tkz@FirstAngle,\tkz@SecondAngle)
\pgfmathsubtract{\tkz@FirstAngle}{\tkz@delta}
\edef\tkz@FirstAngle{\pgfmathresult}%
\pgfmathadd{\tkz@SecondAngle}{\tkz@delta}
\edef\tkz@SecondAngle{\pgfmathresult} 
\begin{scope}[decoration={markings,mark=at position \tkz@mkarcpos with \node{#5};}]
 \path[shift = {(#2)},/@tkzmarkarc/.cd,#1,postaction={decorate}]%
  (\tkz@FirstAngle:\tkz@radius pt) arc (\tkz@FirstAngle:\tkz@SecondAngle:\tkz@radius pt); \end{scope}
\endgroup 
}
%<--------------------------------------------------------------------------->
%                       tkzDrawArcRAN                                        >
%<--------------------------------------------------------------------------->
\def\tkzDrawArcRAN[#1](#2,#3)(#4,#5){% 
\begingroup 
    \draw[shift = {(#2)},/tkzdrawarc/.cd,#1] (#4:#3) arc (#4:#5:#3); 
\endgroup  
}
%<--------------- tkzPathArcRAN ----------------------------------
\def\tkzPathArcRAN[#1](#2,#3)(#4,#5){% 
\begingroup     
    \path[shift = {(#2)},/tkzdrawarc/.cd,#1] (#4:#3) arc (#4:#5:#3); 
\endgroup  
}
%<--------------------------------------------------------------------------->
%       Mark an angle modi 3.03 suppression de \tkz@mksize,
%<--------------------------------------------------------------------------->
%                \tkzMarkAngle(B, A, C)
%
% Marque d'angle
% arc de cercle (simple/double/triple) et marque d'églité.
%
% Par défaut: 
%                 arc       = simple
%                 size     = 1 (rayon de l'arc)
%                 style traits pleins
%                 mkpos ?  position: 0.5 (position de la marque)
%                 mark rien du tout (ignoré si type est utilisé)
%
% Paramètres (optionnels)
%             arc     : l, ll, lll
%             mksize  : 1
%             gap     : 3pt
%             dist ?
%             style   : type de traits
%             position: 0.5
%             mark    : none  , |, ||,|||, z, s, x, o, oo mais tous les 
%  % symboles de tikz sont permis
%<------------------------- Label on angle -------------------------------->
\def\tkz@arcsimple{l} 
\def\tkz@arcdouble{ll}
\def\tkz@arctriple{lll} 

%\tikzset{lbstyle/.style = {color=\tkz@mainlinecolor}}

\pgfkeys{/tkzmkangle/.cd,
      arc/.estore in          = \tkz@arc,
      size/.store in          = \tkz@size,
      mark/.store in          = \tkz@markang,
      mkpos/.estore in        = \tkz@mkpos,
      mksize/.store in        = \tkz@mksize,
      mkcolor/.store in       = \tkz@mkcolor,
      arc                     = l,
      size                    = 1,
      mark                    = none,
      mksize                  = 4pt,
      mkcolor                 = black,
      mkpos                   = 0.5,
  /tkzmkangle/.search also={/tikz}   
}                        
\def\tkzMarkAngle{\pgfutil@ifnextchar[{\tkz@MarkAngle}{\tkz@MarkAngle[]}}   
\def\tkz@MarkAngle[#1](#2,#3,#4){%
\begingroup
\pgfqkeys{/tkzmkangle}{#1}
\tkzFindSlopeAngle(#3,#2)\tkzGetAngle{tkz@dirOne}
\tkzFindSlopeAngle(#3,#4)\tkzGetAngle{tkz@dirTwo}
\tkzNormalizeAngle(\tkz@dirOne,\tkz@dirTwo)

\def\tkz@mymark{\pgfsetplotmarksize{\tkz@mksize}\pgfuseplotmark{\tkz@markang}}
% draw the arcs
\begin{scope}[veclen]
  \begin{scope}[decoration= {markings,
                          mark=at position \tkz@mkpos with {\tkz@mymark}}]
    \ifx\tkz@arc\tkz@arcsimple
      \tkzDrawArcRAN[#1,fill=none]%
      (#3,\tkz@size)(\tkz@FirstAngle,\tkz@SecondAngle)%
      \tkzPathArcRAN[\tkz@mkcolor,postaction={decorate}]%
      (#3,\tkz@size)(\tkz@FirstAngle,\tkz@SecondAngle)
    \else
      \ifx\tkz@arc\tkz@arcdouble     
       \tkzDrawArcRAN[#1,fill=none](#3,\tkz@size cm-2.5\pgflinewidth)%
              (\tkz@FirstAngle,\tkz@SecondAngle) 
       \tkzPathArcRAN[\tkz@mkcolor,postaction={decorate}]%
             (#3,\tkz@size)(\tkz@FirstAngle,\tkz@SecondAngle)%
       \tkzDrawArcRAN[#1,fill=none](#3,\tkz@size cm+2.5\pgflinewidth)%
                       (\tkz@FirstAngle,\tkz@SecondAngle)% 
      \else 
         \ifx\tkz@arc\tkz@arctriple
            \tkzDrawArcRAN[#1,fill=none]%
            (#3,\tkz@size)(\tkz@FirstAngle,\tkz@SecondAngle)% 
            \tkzPathArcRAN[\tkz@mkcolor,postaction={decorate}]%
            (#3,\tkz@size)(\tkz@FirstAngle,\tkz@SecondAngle)%             
            \tkzDrawArcRAN[#1,fill=none](#3,\tkz@size cm-2.5\pgflinewidth)%
                      (\tkz@FirstAngle,\tkz@SecondAngle)%
            \tkzDrawArcRAN[#1,fill=none](#3,\tkz@size cm+2.5\pgflinewidth)%
                      (\tkz@FirstAngle,\tkz@SecondAngle)%   
         \fi
       \fi
    \fi
  \end{scope}
\end{scope}

\endgroup
}
%<--------------------------------------------------------------------------->
% multiple
\def\tkz@multiMA#1 #2\@nil{%
 \protected@edef\tkz@temp{
   \noexpand \tkzMarkAngle[\tkz@optma](#1)}\tkz@temp%    
   \def\tkz@nextArg{#2}%
   \ifx\tkzutil@empty\tkz@nextArg
     \let\next\@gobble
   \fi
   \next#2\@nil
}
%<--------------------------------------------------------------------------->
\def\tkzMarkAngles{\pgfutil@ifnextchar[{\tkz@MarkAngles}{\tkz@MarkAngles[]}}   
\def\tkz@MarkAngles[#1](#2){% 
\xdef\tkz@optma{#1} 
  \begingroup
   \let\next\tkz@multiMA
   \next#2 \@nil %    
\endgroup 
} 
% fin de \tkzMarkAngle
%<--------------------------------------------------------------------------->
%                        Pic Angle
%<--------------------------------------------------------------------------->
\def\tkzPicAngle{\pgfutil@ifnextchar[{\tkz@PicAngle}{\tkz@PicAngle[]}}   
\def\tkz@PicAngle[#1](#2,#3,#4){% 
\begingroup  
\draw pic [#1]{angle=#2--#3--#4};
\endgroup 
}
\def\tkzPicRightAngle{\pgfutil@ifnextchar[{\tkz@PicRightAngle}{\tkz@PicRightAngle[]}}   
\def\tkz@PicRightAngle[#1](#2,#3,#4){% 
\begingroup  
\draw pic [#1]{right angle=#2--#3--#4};
\endgroup 
}
%<--------------------------------------------------------------------------->
%                        FillAngle
%<--------------------------------------------------------------------------->
\pgfkeys{/tkzFill/.cd,
    size/.store in        = \tkz@size,
    size                  = 1,
    /tkzFill/.search also = {/tikz}
  } 

\def\tkzFillAngle{\pgfutil@ifnextchar[{\tkz@FillAngle}{\tkz@FillAngle[]}}   
\def\tkz@FillAngle[#1](#2,#3,#4){% 
\begingroup
\pgfqkeys{/tkzFill}{#1}   
  \tkzFindSlopeAngle(#3,#2)\tkzGetAngle{tkz@dirOne}   
  \tkzFindSlopeAngle(#3,#4)\tkzGetAngle{tkz@dirTwo}   
  \tkzNormalizeAngle(\tkz@dirOne,\tkz@dirTwo)
    \path[shift  = {(#3)},fill=\tkz@euc@linecolor!10,/tkzFill/.cd,#1]%
          (#3) -- (\tkz@FirstAngle:\tkz@size) arc%
          (\tkz@FirstAngle:\tkz@SecondAngle:\tkz@size)--cycle;
\endgroup 
}
%<--------------------------------------------------------------------------->
% multiple
\def\tkz@multiFA#1 #2\@nil{\protected@edef\tkz@temp{%
   \noexpand \tkzFillAngle[\tkz@optfa](#1)}%
   \tkz@temp%    
   \def\tkz@nextArg{#2}%
   \ifx\tkzutil@empty\tkz@nextArg
     \let\next\@gobble
   \fi
   \next#2\@nil
}
%<--------------------------------------------------------------------------->
\def\tkzFillAngles{\pgfutil@ifnextchar[{\tkz@MFillAngles}{\tkz@MFillAngles[]}}   
\def\tkz@MFillAngles[#1](#2){% 
\xdef\tkz@optfa{#1} 
\begingroup
   \let\next\tkz@multiFA
   \next#2 \@nil %    
\endgroup 
} 
%<------------------------- Label on angle -------------------------------->
\pgfkeys{/tkzlabelangle/.cd,
    dist/.store in                  = \tkzlabeldist, 
    angle/.store in                 = \tkzlabelangle, 
    dist                            = 1,
    angle                           = {},
  /tkzlabelangle/.search also={/tikz}   
}  
\def\tkzLabelAngle{\pgfutil@ifnextchar[{\tkz@LabelAngle}{\tkz@LabelAngle[]}}
\def\tkz@LabelAngle[#1](#2,#3,#4)#5{%2016 added normalize from karu's code
\begingroup
  \pgfqkeys{/tkzlabelangle}{#1}
  \ifx\tkzutil@empty\tkzlabelangle% no value so calc angle of bisector
      \tkzFindSlopeAngle(#3,#2)\tkzGetAngle{tkz@dirOne}
      \tkzFindSlopeAngle(#3,#4)\tkzGetAngle{tkz@dirTwo}
      \tkzNormalizeAngle(\tkz@dirOne,\tkz@dirTwo)
      \edef\tkzlabelangle{\fpeval{(\tkz@SecondAngle+\tkz@FirstAngle)/2}}
  \fi
  \path  (#3) --+(\tkzlabelangle:\tkzlabeldist) node[label angle style,%
  /tkzlabelangle/.cd,#1] {#5};
\endgroup
}
%<--------------------------------------------------------------------------->
%<--------------------------------------------------------------------------->
% multiple labels
\def\tkz@multiLBA#1 #2\@nil{%
 \protected@edef\tkz@temp{
   \noexpand \tkzLabelAngle[\tkz@optlba](#1){\tkz@labelangle}}\tkz@temp%
   \def\tkz@nextArg{#2}%
   \ifx\tkzutil@empty\tkz@nextArg
     \let\next\@gobble
   \fi
   \next#2\@nil
}
% %<---------------------------------------------------------------------
\def\tkzLabelAngles{\pgfutil@ifnextchar[{\tkz@LabelAngles}{%
                                         \tkz@LabelAngles[]}}
\def\tkz@LabelAngles[#1](#2)#3{% 
 \def\tkz@optlba{#1}
 \def\tkz@labelangle{#3} 
   \begingroup
      \let\next\tkz@multiLBA
      \next#2 \@nil %
 \endgroup
}
%<-----------------------------------------------------------------------
%                     Symbole droites perpendiculaires      
%<---------------------------------------------------------------------- 
\pgfkeys{/tkzrightangle/.cd,
         norm tkz code/.is if            = normtkzcode@execute,
         german/.code                    = \normtkzcode@executefalse,
         size/.store in                  = \tkz@ra@size,
         dotsize/.store in               = \tkz@dotsize,
         size                            = .25,
         dotsize                         = 3 pt,
  /tkzrightangle/.search also={/tikz},
}
\newcommand*{\tkzMarkRightAngle}[1][]{\tkz@MarkRightAngle[#1]}
\def\tkz@MarkRightAngle[#1](#2,#3,#4){% 
\begingroup
\normtkzcode@executetrue
\pgfqkeys{/tkzrightangle}{#1} 
    \tkzpointnormalised{\pgfpointdiff{\pgfpointanchor{#3}{center}}{%
                                      \pgfpointanchor{#2}{center}}}
    \tkz@ax=\pgf@x\relax%
    \tkz@ay=\pgf@y\relax%
    \tkzpointnormalised{\pgfpointdiff{\pgfpointanchor{#3}{center}}{%
                                      \pgfpointanchor{#4}{center}}}
    \tkz@bx=\pgf@x\relax%
    \tkz@by=\pgf@y\relax%
  \ifnormtkzcode@execute%
      \draw[/tkzrightangle/.cd,#1]
    (#3) --++( 28.45274*\tkz@ra@size\tkz@ax , 28.45274*\tkz@ra@size\tkz@ay)
         --++( 28.45274*\tkz@ra@size\tkz@bx , 28.45274*\tkz@ra@size\tkz@by)
         --++(-28.45274*\tkz@ra@size\tkz@ax ,-28.45274*\tkz@ra@size\tkz@ay)
         -- (#3);
  \else%
    % find angles
      \tkzFindSlopeAngle(#3,#2)\tkzGetAngle{tkz@dirOne}
      \tkzFindSlopeAngle(#3,#4)\tkzGetAngle{tkz@dirTwo}
      \tkzNormalizeAngle(\tkz@dirOne,\tkz@dirTwo)
      \edef\tkzCircleAngle{\fpeval{(\tkz@SecondAngle+\tkz@FirstAngle)/2}}
      \tkzDrawArcRAN[/tkzrightangle/.cd,#1](#3,\tkz@ra@size)%
            (\tkz@FirstAngle,\tkz@SecondAngle)
      \path (#3) --+ (\tkzCircleAngle:\tkz@ra@size/2)
      node[fill,circle,minimum size=\tkz@dotsize,
           inner sep=0 pt,/tkzrightangle/.cd,#1] {};
\fi%
\endgroup
}
%<--------------------------------------------------------------------------->
\def\tkz@multiRA#1 #2\@nil{% 
\protected@edef\tkz@temp{
  \noexpand \tkzMarkRightAngle[\tkz@optRA](#1)}\tkz@temp% 
\def\tkz@nextArg{#2}%
   \ifx\tkzutil@empty\tkz@nextArg
      \let\next\@gobble
   \fi
\next#2\@nil
}
%<--------------------------------------------------------------------------->
\def\tkzMarkRightAngles{\pgfutil@ifnextchar[{\tkz@RightAngles}{%
                                         \tkz@RightAngles[]}} 
\def\tkz@RightAngles[#1](#2){% 
\xdef\tkz@optRA{#1} 
\begingroup
    \let\next\tkz@multiRA
    \next#2 \@nil %    
\endgroup 
}    
%<--------------------------------------------------------------------------->
% tkzdefMidArc center and two points
\def\tkzDefMidArc(#1,#2,#3){%
\begingroup
\tkz@@CalcLength(#1,#2){tkz@radius}
\tkzFindSlopeAngle(#1,#2)\tkzGetAngle{tkz@FirstAngle}
\tkzFindSlopeAngle(#1,#3)\tkzGetAngle{tkz@SecondAngle}
\tkzNormalizeAngle(\tkz@FirstAngle,\tkz@SecondAngle)
 \path[shift = {(#1)}](\tkz@FirstAngle:\tkz@radius pt) arc
           (\tkz@FirstAngle:\tkz@SecondAngle:\tkz@radius pt)  coordinate[midway] (tkzPointResult);
\endgroup
}
\makeatother

\endinput