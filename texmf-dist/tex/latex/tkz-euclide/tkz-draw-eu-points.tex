 % tkz-draw-points.tex
% Copyright 2024  Alain Matthes
% This work may be distributed and/or modified under the
% conditions of the LaTeX Project Public License, either version 1.3
% of this license or (at your option) any later version.
% The latest version of this license is in
%   http://www.latex-project.org/lppl.txt
% and version 1.3 or later is part of all distributions of LaTeX
% version 2005/12/01 or later.
% This work has the LPPL maintenance status “maintained”. 
% The Current Maintainer of this work is Alain Matthes.

\def\fileversion{5.10c}
\def\filedate{2024/04/19} 
\typeout{2024/04/19 5.10c  tkz-draw-eu-points.tex} 
\makeatletter
%<--------------------------------------------------------------------------->
%                    tkzSetUpPoint  définit la forme d'un point
%<--------------------------------------------------------------------------->
\pgfkeys{/tkzsetuppt/.cd,
      size/.store in       =  \tkz@pt@size,
      color/.store in      =  \tkz@pt@color,
      fill/.store in       =  \tkz@pt@fill, 
      shape/.store in      =  \tkz@pt@shape,
      size                 =  \tkz@euc@pointsize,
      color                =  \tkz@euc@pointcolor,
      fill                 =  \tkz@euc@pointfillcolor,
      shape                =  \tkz@euc@pointshape,
   /tkzsetuppt/.search also  = {/tikz},
 } 
\def\tkzSetUpPoint{\pgfutil@ifnextchar[{\tkz@SetUpPoint}{%
                                        \tkz@SetUpPoint[]}}
\def\tkz@SetUpPoint[#1]{%
\pgfqkeys{/tkzsetuppt}{#1}
% redefine point style with new values
\tikzset{point style/.style={draw         = \tkz@pt@color,
                             inner sep    = 0pt,
                             shape        = \tkz@pt@shape,
                             minimum size = \tkz@pt@size,
                             fill         = \tkz@pt@fill}}
}% end setup
%<--------------------------------------------------------------------------->
%                       Draw Point   
%<--------------------------------------------------------------------------->
\pgfkeys{/tkzdrawpt/.cd, 
  size/.code   = {\tikzset{point style/.append style={minimum size = #1}}},
  size         = \tkz@euc@pointsize,
  /tkzdrawpt/.search also  = {/tikz},
}
%<--------------------------------------------------------------------------    
\def\tkzDrawPoint{\pgfutil@ifnextchar[{\tkz@DrawPoint}{\tkz@DrawPoint[]}} 
\def\tkz@DrawPoint[#1](#2){%
\begingroup 
    \pgfqkeys{/tkzdrawpt}{#1}    
    \node[point style,/tkzdrawpt/.cd,#1] at (#2) {};%2016
\endgroup
}
%<---------------------------------------------------------------------------> 
\def\tkzDrawPoints{\pgfutil@ifnextchar[{\tkz@drawpts}{\tkz@drawpts[]}} 
%<---------------------------------------------------------------------------> 
\def\tkz@drawpts[#1](#2){%
\begingroup
   \pgfqkeys{/tkzdrawpt}{#1}        
   \foreach \point in {#2}{\node[point style,/tkzdrawpt/.cd,#1] at (\point) {};} %2016
\endgroup 
}
%<-------------------------------------------------------------------------->
%                   tkzLabelPoint          Affichage des LABELS pour un point
%<-------------------------------------------------------------------------->  
\def\tkzLabelPoint{\pgfutil@ifnextchar[{\tkz@LabelPoint}{\tkz@LabelPoint[]}} 
\def\tkz@LabelPoint[#1](#2)#3{%
      \node[label style,#1] at (#2) {#3};}%
%<--------------------------------------------------------------------------->

\def\tkzLabelPoints{\pgfutil@ifnextchar[{\tkz@LabelPoints}{\tkz@LabelPoints[]}}% 
\def\tkz@LabelPoints[#1](#2){%
 \foreach \point in {#2}{
      \node[label style,#1] at (\point) {$\point$};}
}%
%<--------------------------------------------------------------------------->
%<--------------------------------------------------------------------------->
%
%<--------------------------------------------------------------------------->
\def\tkzSetUpLabel{\pgfutil@ifnextchar[{\tkz@SetUpLabel}{%
                                        \tkz@SetUpLabel[]}}
\def\tkz@SetUpLabel[#1]{%
\tikzset{label style/.style={#1}}
}% end setup
%<--------------------------------------------------------------------------->

\pgfkeys{/tkzautolab/.cd,
    center/.store in           = \tkz@center,
    dist/.store in             = \tkz@dist,
    dist                       = 0.15,
   /tkzautolab/.search also   = {/tikz},
}
\def\tkzAutoLabelPoints{\pgfutil@ifnextchar[{\tkz@AutoLabelPoints}{\tkz@AutoLabelPoints[]}}% 
\def\tkz@AutoLabelPoints[#1](#2){%
\begingroup 
\pgfqkeys{/tkzautolab}{#1} 
 \foreach \point in {#2}{
 \path (\tkz@center) -- ($ (\point) + \tkz@dist*($(\point)-(\tkz@center)$) $) node[/tkzautolab/.cd,label style,#1]{$\point$};}
\endgroup
}%
%<--------------------------------------------------------------------------->
%                                 PointShowCoord  
%<--------------------------------------------------------------------------->
\pgfkeys{/tkzprcoord/.cd,
          xlabel/.store in    = \tkz@xlabel,
          ylabel/.store in    = \tkz@ylabel, 
          xstyle/.code        = {\tikzset{xcoord style/.append style={#1}}},
          ystyle/.code        = {\tikzset{ycoord style/.append style={#1}}},
          noxdraw/.is if      = tkz@coord@noxdraw,
          noxdraw/.default    = true,
          noydraw/.is if      = tkz@coord@noydraw,
          noydraw/.default    = true,
          xlabel              = {},
          ylabel              = {},
          xstyle              = {},
          ystyle              = {},
          noxdraw             = false,
          noydraw             = false,    
  /tkzprcoord/.search also   = {/tikz},
}
\def\tkzPointShowCoord{\pgfutil@ifnextchar[{\tkz@PointShowCoord}{\tkz@PointShowCoord[]}}     
\def\tkz@PointShowCoord[#1](#2){%
\begingroup 
\pgfqkeys{/tkzprcoord}{#1} 
% 2019 for showcoord
 \iftkznodedefined{tkz@xline}{}{%
 \path (0,0) --(1,0) node(tkz@xline){};
 \path (0,0) --(0,1) node(tkz@yline){};
 }
\iftkz@coord@noxdraw\else\draw[arrow coord style,/tkzprcoord/.cd,#1] (#2)--(#2 |- tkz@xline);\fi 
\iftkz@coord@noydraw\else\draw[arrow coord style,/tkzprcoord/.cd,#1] (#2)--(#2 -| tkz@yline);\fi   
\ifx\tkzutil@empty\tkz@xlabel
\else
\path (#2)--(#2 |- tkz@xline)
  node[xcoord style] {\tkz@xlabel};
\fi
\ifx\tkzutil@empty\tkz@ylabel
\else
 \path (#2)--(#2 -| tkz@yline)
   node[ycoord style] {\tkz@ylabel};
\fi    
\endgroup
}
\let\tkzShowPointCoord\tkzPointShowCoord
\makeatother  
\endinput