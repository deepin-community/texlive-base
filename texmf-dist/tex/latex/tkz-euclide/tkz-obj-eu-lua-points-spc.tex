% tkz-obj-lua-points.tex
% Copyright 2024  Alain Matthes
% This work may be distributed and/or modified under the
% conditions of the LaTeX Project Public License, either version 1.3
% of this license or (at your option) any later version.
% The latest version of this license is in
%   http://www.latex-project.org/lppl.txt
% and version 1.3 or later is part of all distributions of LaTeX
% version 2005/12/01 or later.
% This work has the LPPL maintenance status “maintained”.
% The Current Maintainer of this work is Alain Matthes.

\def\fileversion{5.10c}
\def\filedate{2024/01/23} 
\typeout{2024/01/23 5.10c tkz-obj-lua-points-spc.tex} 
\makeatletter  
%add ExCenter
%<--------------------------------------------------------------------------–>
%                   Specific points
%<--------------------------------------------------------------------------–>
%                              barycentre
%<--------------------------------------------------------------------------–>
\def\tkzDefBarycentricPoint(#1){%
\begingroup
 \path[coordinate]  (barycentric cs:#1) coordinate (tkzPointResult); 
\endgroup
}
\let\tkzDefBCPoint\tkzDefBarycentricPoint

\def\tkzDefBarycentricPointTwo(#1,#2,#3,#4){%
\begingroup
\pgfextractx{\pgf@x}{\pgfpointanchor{#1}{center}}
\pgfextracty{\pgf@y}{\pgfpointanchor{#1}{center}}
\tkz@ax\pgf@x %
\tkz@ay\pgf@y %
\edef\tkzcax{\strip@pt\tkz@ax}%
\edef\tkzcay{\strip@pt\tkz@ay}%
\pgfextractx{\pgf@x}{\pgfpointanchor{#3}{center}}
\pgfextracty{\pgf@y}{\pgfpointanchor{#3}{center}}
\tkz@bx\pgf@x %
\tkz@by\pgf@y %
\edef\tkzcbx{\strip@pt\tkz@bx}%
\edef\tkzcby{\strip@pt\tkz@by}%
\edef\tkztpmx{\tkz@Dec{(\tkzcax*(#2) + \tkzcbx*(#4))/(#2 + #4 )}}
\edef\tkztpmy{\tkz@Dec{(\tkzcay*(#2) + \tkzcby*(#2))/(#2 + #4 )}}
\pgfcoordinate{tkzPointResult}{\pgfqpoint{\tkztpmx pt}{\tkztpmy pt}}
\endgroup
}

%<--------------------------------------------------------------------------–>
%                      milieu  de deux points
%<--------------------------------------------------------------------------–>
% possible   \coordinate (#3) at ($(#1)!0.5!(#2)$);
%<--------------------------------------------------------------------------–>
% \def\tkzDefMidPoint(#1,#2){%
% \begingroup
%  \path (#1) -- (#2) coordinate[pos=.5](tkzPointResult);
% \endgroup
% }
\def\tkzDefMidPoint(#1,#2){%
\begingroup
   \pgf@process{\pgfpointanchor{#1}{center}}%
   \tkz@ax\pgf@x%
   \tkz@ay\pgf@y%
   \pgf@process{\pgfpointanchor{#2}{center}}%
   \tkz@bx\pgf@x%
   \tkz@by\pgf@y%
   \advance\tkz@bx by\tkz@ax\relax%
   \advance\tkz@by by\tkz@ay\relax%
   \divide\tkz@bx by2\relax%
   \divide\tkz@by by2\relax
   \pgfcoordinate{tkzPointResult}{\pgfqpoint{\tkz@bx}{\tkz@by}}
\endgroup
}
\def\tkz@DefMidPoint(#1,#2,#3,#4){%
\begingroup
   \tkz@ax#1%
   \tkz@ay#2%
   \tkz@bx#3%
   \tkz@by#4%
   \advance\tkz@bx by\tkz@ax\relax%
   \advance\tkz@by by\tkz@ay\relax%
   \divide\tkz@bx by2\relax%
   \divide\tkz@by by2\relax
   \pgfcoordinate{tkzPointResult}{\pgfqpoint{\tkz@bx}{\tkz@by}}
\endgroup
}

%<--------------------------------------------------------------------------–>
\def\tkz@simicenter{0}
\pgfkeys{/tkzsimicenter/.cd,
      ext/.code       = \def\tkz@simicenter{0},
      int/.code       = \def\tkz@simicenter{1},
      ext
}%

\def\tkzDefSimilitudeCenter{\pgfutil@ifnextchar[{\tkz@DefSimilitudeCenter}%
                                                {\tkz@DefSimilitudeCenter[]}}     
\def\tkz@DefSimilitudeCenter[#1](#2,#3)(#4,#5){%
\pgfqkeys{/tkzsimicenter}{#1} 
\begingroup  
\ifcase\tkz@simicenter%
  \tkzDefExtSimilitudeCenter[#1](#2,#3)(#4,#5)
  \or% 1 
  \tkzDefIntSimilitudeCenter[#1](#2,#3)(#4,#5) 
\fi
\endgroup
}
%<--------------------------------------------------------------------------–>
%                    Internal Similitude center
%  Two circles have two similitude centers namely the internal center of
%   similitude Si and the external similitude center Se.
%<--------------------------------------------------------------------------–>
\def\tkz@numhomo{0}
\pgfkeys{
/tkzSimilitudeCenter/.cd,
 node/.code          = \def\tkz@numhomo{0},
 R/.code             = \def\tkz@numhomo{1},
 node,
 /tkzSimilitudeCenter/.unknown/.code   = {\let\searchname=\pgfkeyscurrentname
 \pgfkeysalso{\searchname/.try=#1, /tikz/\searchname/.retry=#1}}
}
\def\tkzDefIntSimilitudeCenter{\pgfutil@ifnextchar[{\tkz@DefIntSimilitudeCenter}{\tkz@DefIntSimilitudeCenter[]}}
\def\tkz@DefIntSimilitudeCenter[#1](#2,#3)(#4,#5){%
\begingroup
\pgfqkeys{/tkzSimilitudeCenter}{#1}
\ifcase\tkz@numhomo%
    \tkz@@CalcLengthcm(#2,#3){tkz@rt}%
    \tkz@@CalcLengthcm(#4,#5){tkz@rf}%
  \or% 1
   \def\tkz@rt{#3}%
   \def\tkz@rf{#5}%
\fi
\pgfinterruptboundingbox 
\pgfextractx{\pgf@x}{\pgfpointanchor{#2}{center}}
\pgfextracty{\pgf@y}{\pgfpointanchor{#2}{center}}
\tkz@ax\pgf@x %
\tkz@ay\pgf@y %
\edef\tkzcax{\strip@pt\tkz@ax}%
\edef\tkzcay{\strip@pt\tkz@ay}%
\pgfextractx{\pgf@x}{\pgfpointanchor{#4}{center}}
\pgfextracty{\pgf@y}{\pgfpointanchor{#4}{center}}
\tkz@bx\pgf@x %
\tkz@by\pgf@y %
\edef\tkzcbx{\strip@pt\tkz@bx}%
\edef\tkzcby{\strip@pt\tkz@by}%
\edef\tkztpmx{\tkz@Dec{(\tkzcax*\tkz@rf + \tkzcbx*\tkz@rt)/(\tkz@rf + \tkz@rt )}}
\edef\tkztpmy{\tkz@Dec{(\tkzcay*\tkz@rf + \tkzcby*\tkz@rt)/(\tkz@rf + \tkz@rt )}}
  \pgfcoordinate{tkzPointResult}{\pgfqpoint{\tkztpmx pt}{\tkztpmy pt}}
\endpgfinterruptboundingbox 
\endgroup
}
\let\tkzDefIntHomotheticCenter\tkzDefIntSimilitudeCenter
%<--------------------------------------------------------------------------–>
%                    External Similitude center
%<--------------------------------------------------------------------------–>
\def\tkzDefExtSimilitudeCenter{\pgfutil@ifnextchar[{\tkz@DefExtSimilitudeCenter}{%
\tkz@DefExtSimilitudeCenter[]}}

\def\tkz@DefExtSimilitudeCenter[#1](#2,#3)(#4,#5){%
\begingroup
\pgfqkeys{/tkzSimilitudeCenter}{#1}
\ifcase\tkz@numhomo%
  \tkz@@CalcLengthcm(#2,#3){tkz@rt}%
  \tkz@@CalcLengthcm(#4,#5){tkz@rf}%
   \or% 1
  \def\tkz@rt{#3}%
  \def\tkz@rf{#5}%
\fi
\pgfinterruptboundingbox 
\pgfextractx{\pgf@x}{\pgfpointanchor{#2}{center}}
\pgfextracty{\pgf@y}{\pgfpointanchor{#2}{center}}
\tkz@ax\pgf@x %
\tkz@ay\pgf@y %
\edef\tkzcax{\strip@pt\tkz@ax}%
\edef\tkzcay{\strip@pt\tkz@ay}%
\pgfextractx{\pgf@x}{\pgfpointanchor{#4}{center}}
\pgfextracty{\pgf@y}{\pgfpointanchor{#4}{center}}
\tkz@bx\pgf@x %
\tkz@by\pgf@y %
\edef\tkzcbx{\strip@pt\tkz@bx}%
\edef\tkzcby{\strip@pt\tkz@by}%
\edef\tkztpmx{\tkz@Dec{(\tkzcax*(-\tkz@rf) + \tkzcbx*\tkz@rt)/(\tkz@rt - (\tkz@rf))}}
\edef\tkztpmy{\tkz@Dec{(\tkzcay*(-\tkz@rf) + \tkzcby*\tkz@rt)/(\tkz@rt - (\tkz@rf))}}
  \pgfcoordinate{tkzPointResult}{\pgfqpoint{\tkztpmx pt}{\tkztpmy pt}}
   \endpgfinterruptboundingbox 
\endgroup
}

\let\tkzDefExtHomotheticCenter\tkzDefExtSimilitudeCenter
%<--------------------------------------------------------------------------–>
%        Harmonic Division
%<--------------------------------------------------------------------------–>
%  A , B , C ,D  CA/CB = DA/DB
%<--------------------------------------------------------------------------–>
\def\tkz@numdha{0}
\pgfkeys{/tkzharmonic/.cd,
      ext/.code       = \def\tkz@numdha{0},
      int/.code       = \def\tkz@numdha{1},
      both/.code      = \def\tkz@numdha{2},
      both,
}%

\def\tkzDivHarmonic{\pgfutil@ifnextchar[{\tkz@DivHarmonic}{\tkz@DivHarmonic[]}}     
\def\tkz@DivHarmonic[#1](#2){%
\begingroup 
\pgfqkeys{/tkzharmonic}{#1}  
  \ifcase\tkz@numdha%
   \tkzDefDivHarmonicExt(#2)
 \or%
    \tkzDefDivHarmonicInt(#2)
  \or%
    \tkzDefDivHarmonicBoth(#2)
 \fi
\endgroup
}

\def\tkzDefDivHarmonicExt(#1,#2,#3){%
\begingroup
\pgfinterruptboundingbox 
  \tkz@VecKOrth[](#1,#2)  \tkzGetPoint{tkz@px}
  \tkzDefMidPoint(tkz@px,#2)          \tkzGetPoint{tkz@py}
  \tkzInterLL(tkz@px,#3)(#1,tkz@py)   \tkzGetPoint{tkz@pz}
  \tkzInterLL(#2,tkz@pz)(#1,tkz@px)   \tkzGetPoint{tkz@px}
  \tkzInterLL(tkz@py,tkz@px)(#1,#2)   \tkzGetPoint{tkzPointResult}
\endpgfinterruptboundingbox 
\endgroup
}

\def\tkzDefDivHarmonicInt(#1,#2,#3){%
\begingroup
\pgfinterruptboundingbox 
  \tkz@VecKOrth[1](#1,#2)  \tkzGetPoint{tkz@px}
  \tkzDefMidPoint(tkz@px,#2)          \tkzGetPoint{tkz@py}
  \tkzInterLL(tkz@py,#3)(#1,tkz@px)   \tkzGetPoint{tkz@pz}
  \tkzInterLL(#2,tkz@pz)(#1,tkz@py)   \tkzGetPoint{tkz@py}
  \tkzInterLL(tkz@py,tkz@px)(#1,#2)   \tkzGetPoint{tkzPointResult}
\endpgfinterruptboundingbox 
\endgroup
}

\def\tkzDefDivHarmonicBoth(#1,#2,#3){%
\begingroup
\edef\tkz@k{\fpeval{#3}}
    \path[coordinate]  (barycentric cs:#1=1,#2={\tkz@k}) coordinate (tkzFirstPointResult);
    \path[coordinate]  (barycentric cs:#1=1,#2={-\tkz@k}) coordinate (tkzSecondPointResult);
\endgroup
}

\let\tkzDefHarmonic\tkzDivHarmonic
%<--------------------------------------------------------------------------–> 
%                   golden ratio
%<--------------------------------------------------------------------------–>
\def\tkzDefGoldenRatio(#1,#2){%
\begingroup
\tkzDefPointWith[linear,K=\tkzInvPhi](#1,#2)
\endgroup
}
%<--------------------------------------------------------------------------–> 
%                    triangle center
%<--------------------------------------------------------------------------–>

\def\tkz@numtc{0}
\pgfkeys{/tkzDefTriangleCenter/.cd,
   ortho/.code        = \def\tkz@numtc{0},
   orthic/.code       = \def\tkz@numtc{0},
   centroid/.code     = \def\tkz@numtc{1},
   median/.code       = \def\tkz@numtc{1},
   circum/.code       = \def\tkz@numtc{2},
   in/.code           = \def\tkz@numtc{3},
   ex/.code           = \def\tkz@numtc{4},
   euler/.code        = \def\tkz@numtc{5},
   symmedian/.code    = \def\tkz@numtc{6},
   lemoine/.code      = \def\tkz@numtc{6},
   grebe/.code        = \def\tkz@numtc{6},
   spieker/.code      = \def\tkz@numtc{7},
   gergonne/.code     = \def\tkz@numtc{8},
   nagel/.code        = \def\tkz@numtc{9}, 
   mittenpunkt/.code  = \def\tkz@numtc{10}, 
   feuerbach/.code    = \def\tkz@numtc{11},
   circum       
}     
\def\tkzDefTriangleCenter{\pgfutil@ifnextchar[{\tkz@DefTriangleCenter}{\tkz@DefTriangleCenter[]}}
\def\tkz@DefTriangleCenter[#1](#2){% 
\begingroup  
\pgfqkeys{/tkzDefTriangleCenter}{#1}  
\ifcase\tkz@numtc%
  \tkzOrthoCenter(#2)
  \or% 1 
  \tkzCentroid(#2)  
  \or% 2
  \tkzCircumCenter(#2) 
  \or% 3
  \tkzInCenter(#2) 
  \or% 4
  \tkzExCenter(#2) 
  \or% 5
  \tkzEulerCenter(#2)
  \or% 6
  \tkzSymmedianCenter(#2) 
  \or% 7
  \tkzSpiekerCenter(#2) 
  \or% 8
  \tkzGergonneCenter(#2) 
  \or%9
  \tkzNagelCenter(#2)
  \or%10
  \tkzMittenpunktCenter(#2) 
  \or%11
  \tkzFeuerbachCenter(#2) 
    \fi    
\endgroup
}
%<--------------------------------------------------------------------------–> 
%                    OrthoCenter
%<--------------------------------------------------------------------------–>
\def\tkzOrthoCenter(#1,#2,#3){%
\begingroup
  \pgfinterruptboundingbox 
   \tkzUProjection(#1,#2)(#3)
   \pgfnodealias{ort@pta}{tkzPointResult}
   \tkzUProjection(#1,#3)(#2)
   \pgfnodealias{ort@ptb}{tkzPointResult}
   \tkzInterLL(#2,ort@ptb)(#3,ort@pta)
   \endpgfinterruptboundingbox 
\endgroup
}
\let\tkzDefOrthoCenter\tkzOrthoCenter
%<--------------------------------------------------------------------------–>
%                      GravityCenter modif 3.03
%<--------------------------------------------------------------------------–>
\def\tkzCentroid(#1,#2,#3){%
\begingroup 
   \pgf@process{\pgfpointanchor{#1}{center}}%
   \tkz@ax\pgf@x%
   \tkz@ay\pgf@y%
   \pgf@process{\pgfpointanchor{#2}{center}}%
   \tkz@bx\pgf@x%
   \tkz@by\pgf@y%
   \pgf@process{\pgfpointanchor{#3}{center}}%
   \tkz@cx\pgf@x%
   \tkz@cy\pgf@y%
   \edef\tkzcax{\strip@pt\tkz@ax}%
   \edef\tkzcay{\strip@pt\tkz@ay}%
   \edef\tkzcbx{\strip@pt\tkz@bx}%
   \edef\tkzcby{\strip@pt\tkz@by}%
   \edef\tkzccx{\strip@pt\tkz@cx}%
   \edef\tkzccy{\strip@pt\tkz@cy}%
   \edef\tkztpmx{\tkz@Dec{(\tkzcax + \tkzcbx + \tkzccx)/3}}
   \edef\tkztpmy{\tkz@Dec{(\tkzcay + \tkzcby + \tkzccy)/3}}
   \pgfinterruptboundingbox 
   \pgfcoordinate{tkzPointResult}{\pgfqpoint{\tkztpmx pt}{\tkztpmy pt}}
   \endpgfinterruptboundingbox   
\endgroup
}
\let\tkzBaryCenter\tkzCentroid

%<--------------------------------------------------------------------------–>
%                      CircumCenter
%<--------------------------------------------------------------------------–>
\def\tkzCircumCenter(#1,#2,#3){%
\begingroup
\pgfinterruptboundingbox 
 \tkzDefMediatorLine(#1,#2)
 \pgf@process{\pgfpointanchor{tkzFirstPointResult}{center}}%
 \tkz@ax\pgf@x%
 \tkz@ay\pgf@y%
 \pgf@process{\pgfpointanchor{tkzSecondPointResult}{center}}%
 \tkz@bx\pgf@x%
 \tkz@by\pgf@y%
 \tkzDefMediatorLine(#1,#3)
 \pgf@process{\pgfpointanchor{tkzFirstPointResult}{center}}%
 \tkz@cx\pgf@x%
 \tkz@cy\pgf@y%
 \pgf@process{\pgfpointanchor{tkzSecondPointResult}{center}}%
 \tkz@dx\pgf@x%
 \tkz@dy\pgf@y%
\tkzInterLLxy(\tkz@ax,\tkz@ay,\tkz@bx,\tkz@by)(\tkz@cx,\tkz@cy,\tkz@dx,\tkz@dy)%
\endpgfinterruptboundingbox
\endgroup
} 
\let\tkzDefCircumCenter\tkzCircumCenter
%<--------------------------------------------------------------------------–>
%                     InCenter
%<--------------------------------------------------------------------------–>
\def\tkzInCenter(#1,#2,#3){%
\begingroup
\pgfinterruptboundingbox 
\tkzDefBisectorLine(#3,#1,#2)
\pgf@process{\pgfpointanchor{tkzPointResult}{center}}%
\tkz@bx\pgf@x%
\tkz@by\pgf@y%
\tkzDefBisectorLine(#3,#2,#1) 
\pgf@process{\pgfpointanchor{tkzPointResult}{center}}%
\tkz@dx\pgf@x%
\tkz@dy\pgf@y%
\pgf@process{\pgfpointanchor{#1}{center}}%
\tkz@ax\pgf@x%
\tkz@ay\pgf@y%
\pgf@process{\pgfpointanchor{#2}{center}}%
\tkz@cx\pgf@x%
\tkz@cy\pgf@y%
\tkzInterLLxy(\tkz@ax,\tkz@ay,\tkz@bx,\tkz@by)(\tkz@cx,\tkz@cy,\tkz@dx,\tkz@dy)%
\endpgfinterruptboundingbox
\endgroup
} 
\let\tkzDefInCenter\tkzInCenter
%<--------------------------------------------------------------------------–>
%                     ExCenter
%<--------------------------------------------------------------------------–>
\def\tkzExCenter(#1,#2,#3){%
\begingroup
\pgfinterruptboundingbox 
\tkzDefBisectorOutLine(#2,#1,#3)
\pgf@process{\pgfpointanchor{tkzPointResult}{center}}%
\tkz@bx\pgf@x%
\tkz@by\pgf@y%
\tkzDefBisectorOutLine(#2,#3,#1) 
\pgf@process{\pgfpointanchor{tkzPointResult}{center}}%
\tkz@dx\pgf@x%
\tkz@dy\pgf@y%
\pgf@process{\pgfpointanchor{#1}{center}}%
\tkz@ax\pgf@x%
\tkz@ay\pgf@y%
\pgf@process{\pgfpointanchor{#3}{center}}%
\tkz@cx\pgf@x%
\tkz@cy\pgf@y%
\tkzInterLLxy(\tkz@ax,\tkz@ay,\tkz@bx,\tkz@by)(\tkz@cx,\tkz@cy,\tkz@dx,\tkz@dy)%
\endpgfinterruptboundingbox
\endgroup
} 
\let\tkzDefExCenter\tkzExCenter
%<--------------------------------------------------------------------------–>
%                     EulerCenter neuf points
%<--------------------------------------------------------------------------–>
\def\tkzEulerCenter(#1,#2,#3){% 
% mileu de orthocentre et centre cercle circonscrit
% passe par les midpoints par les pieds des hauteurs
\begingroup
\pgfinterruptboundingbox 
\tkzDefMidPoint(#1,#2)
\pgfnodealias{eu@mic}{tkzPointResult}
\tkzDefMidPoint(#1,#3)
\pgfnodealias{eu@mib}{tkzPointResult}
\tkzDefMidPoint(#2,#3)
\pgfnodealias{eu@mia}{tkzPointResult}
\tkzCircumCenter(eu@mia,eu@mib,eu@mic)
\endpgfinterruptboundingbox
\endgroup
}
\let\tkzNinePointCenter\tkzEulerCenter
\let\tkzDefEulerCenter\tkzEulerCenter
%<--------------------------------------------------------------------------–>
%Symmedian center Lemoine point Grebe point K
%<--------------------------------------------------------------------------–>
\def\tkzSymmedianCenter(#1,#2,#3){%
\begingroup
\pgfinterruptboundingbox    
\tkzDefMidPoint(#2,#3)
\pgfnodealias{eu@mic}{tkzPointResult}
\tkzDefMidPoint(#1,#3)
\pgfnodealias{eu@mib}{tkzPointResult}
\tkzUProjection(#2,#3)(#1)
\pgfnodealias{ort@pta}{tkzPointResult}
\tkzDefMidPoint(#1,ort@pta)
\pgfnodealias{eu@mid}{tkzPointResult}
\tkzUProjection(#1,#3)(#2)
\pgfnodealias{ort@ptb}{tkzPointResult}
\tkzDefMidPoint(#2,ort@ptb)
\pgfnodealias{eu@mie}{tkzPointResult}
\tkzInterLL(eu@mic,eu@mid)(eu@mib,eu@mie)
\endpgfinterruptboundingbox
\endgroup
}
\let\tkzLemoinePoint\tkzSymmedianCenter
\let\tkzGrebePoint\tkzSymmedianCenter
\let\tkzDefLemoinePoint\tkzLemoinePoint
%<--------------------------------------------------------------------------–> 
%                   Spieker center
%<--------------------------------------------------------------------------–>
\def\tkzSpiekerCenter(#1,#2,#3){%
\begingroup
% we need to get the midpoints
\pgfcoordinate{tkz@m3}{%
  \pgfpointscale{0.5}{%
  \pgfpointadd{\pgfpointanchor{#1}{center}}%
              {\pgfpointanchor{#2}{center}}}}%
\pgfcoordinate{tkz@m2}{%
  \pgfpointscale{0.5}{%
  \pgfpointadd{\pgfpointanchor{#1}{center}}%
              {\pgfpointanchor{#3}{center}}}}%
\pgfcoordinate{tkz@m1}{%
   \pgfpointscale{0.5}{%
   \pgfpointadd{\pgfpointanchor{#2}{center}}%
               {\pgfpointanchor{#3}{center}}}}%
\tkzInCenter(tkz@m1,tkz@m2,tkz@m3)
\endgroup
} 
\let\tkzDefSpiekerCenter\tkzSpiekerCenter 
%<--------------------------------------------------------------------------–> 
%                    Gergonne center Ge
%<--------------------------------------------------------------------------–>
\def\tkzGergonneCenter(#1,#2,#3){%  
\begingroup
\pgfinterruptboundingbox 
\tkzInCenter(#1,#2,#3)
\pgfnodealias{tkz@ptin}{tkzPointResult}
\tkzUProjection(#2,#3)(tkz@ptin)
\pgfnodealias{tkz@oca}{tkzPointResult}
\tkzUProjection(#1,#3)(tkz@ptin)
\pgfnodealias{tkz@ocb}{tkzPointResult}
\tkzInterLL(#1,tkz@oca)(#2,tkz@ocb)
\endpgfinterruptboundingbox
\endgroup
}
\let\tkzDefGergonneCenter\tkzGergonneCenter 
%<--------------------------------------------------------------------------–> 
%                    Nagel center Na
%<--------------------------------------------------------------------------–>
%  INa = 3 IG. Nagel point % correction 02/02/20
\def\tkzNagelCenter(#1,#2,#3){%  
\begingroup
\pgfinterruptboundingbox 
\tkzDefExcentralTriangle(#1,#2,#3){tkz@a,tkz@b,tkz@c}
\tkzUProjection(#2,#3)(tkz@a)
\pgfnodealias{tkz@tgta}{tkzPointResult}
\tkzUProjection(#1,#2)(tkz@c)
\pgfnodealias{tkz@tgtc}{tkzPointResult}
\tkzInterLL(#1,tkz@tgta)(#3,tkz@tgtc)
\endpgfinterruptboundingbox
\endgroup
}
\let\tkzDefNagelCenter\tkzNagelCenter 
%<--------------------------------------------------------------------------–>
%  Mittenpunkt
%<--------------------------------------------------------------------------–>
\def\tkzMittenpunktCenter(#1,#2,#3){%  
\begingroup
\pgfinterruptboundingbox 
\tkzExCenter(#2,#3,#1)
\pgfnodealias{tkz@a}{tkzPointResult}
\tkzExCenter(#3,#1,#2)
\pgfnodealias{tkz@b}{tkzPointResult}
\pgfcoordinate{tkz@ma}{%
\pgfpointscale{0.5}{%
\pgfpointadd{\pgfpointanchor{#1}{center}}{\pgfpointanchor{#2}{center}}}}%
\pgfcoordinate{tkz@mb}{%
\pgfpointscale{0.5}{%
\pgfpointadd{\pgfpointanchor{#2}{center}}{\pgfpointanchor{#3}{center}}}}%
\tkzInterLL(tkz@a,tkz@ma)(tkz@b,tkz@mb)
\endpgfinterruptboundingbox
\endgroup
}
\let\tkzDefMittenpunktCenter\tkzMittenpunktCenter 
\let\tkzDefMiddlespoint\tkzMittenpunktCenter     
%<--------------------------------------------------------------------------–> 
%                   Feuerbach point
%<--------------------------------------------------------------------------–>
\def\tkzFeuerbachCenter(#1,#2,#3){%  
\begingroup
\pgfinterruptboundingbox 
\tkzEulerCenter(#1,#2,#3)
\pgfnodealias{tkz@euler}{tkzPointResult}
\tkzInCenter(#1,#2,#3) 
\pgfnodealias{tkz@in}{tkzPointResult}
\tkzUProjection(#2,#3)(tkzPointResult)
\tkzInterLC(tkz@in,tkz@euler)(tkz@in,tkzPointResult)\tkzGetFirstPoint{tkz@fe}
\tkzRenamePoint(tkz@fe){tkzPointResult}
\endpgfinterruptboundingbox  
\endgroup
}
\let\tkzDefFeuerbachCenter\tkzFeuerbachCenter 
%<--------------------------------------------------------------------------–>
%                     Orthogonal center
%<--------------------------------------------------------------------------–>
\def\tkzOrthogonalCenter(#1,#2){%
\begingroup
\pgfinterruptboundingbox 
\tkz@VecK[\tkz@koeff/(1+\tkz@koeff)](#1,#2)
\pgfnodealias{tkzFirstPointResult}{tkzPointResult}
\tkz@VecK[\tkz@koeff/(\tkz@koeff-1)](#1,#2)
\pgfnodealias{tkzSecondPointResult}{tkzPointResult}
\tkzDefMidPoint(tkzFirstPointResult,tkzSecondPointResult) 
\endpgfinterruptboundingbox 
\endgroup
}
%<--------------------------------------------------------------------------–>
%                  End Triangle center
%<--------------------------------------------------------------------------–>
%<--------------------------------------------------------------------------–>
%                  Projection  center of excircles
%<--------------------------------------------------------------------------–>
\def\tkzDefProjExcenter{\pgfutil@ifnextchar[{%
    \tkz@DefProjExcenter}{%
    \tkz@DefProjExcenter[]}
    }
\def\tkz@DefProjExcenter[#1](#2,#3,#4)(#5)#6{
\begingroup  
  \SetUpPTTR{#1}
   \foreach \name  [count=\i] in {#5} {%
      \global\expandafter\edef\csname tkz@pt\i\endcsname{\name}
        }
   \foreach \name  [count=\i] in {#6} {%
      \global\expandafter\edef\csname tkz@ppt\i\endcsname{\name}
        }
\tkzDefPointBy[projection=onto #3--#4 ](\tkz@pttr@name \csname tkz@pt1\endcsname) 
\pgfnodealias{\csname tkz@ppt1\endcsname\csname tkz@pt1\endcsname}{tkzPointResult}
\tkzDefPointBy[projection=onto #3--#4 ](\tkz@pttr@name \csname tkz@pt2\endcsname) 
\pgfnodealias{\csname tkz@ppt1\endcsname\csname tkz@pt2\endcsname}{tkzPointResult}
\tkzDefPointBy[projection=onto #3--#4 ](\tkz@pttr@name \csname tkz@pt3\endcsname) 
\pgfnodealias{\csname tkz@ppt1\endcsname\csname tkz@pt3\endcsname}{tkzPointResult}
\tkzDefPointBy[projection=onto #2--#4 ](\tkz@pttr@name \csname tkz@pt1\endcsname) 
\pgfnodealias{\csname tkz@ppt2\endcsname\csname tkz@pt1\endcsname}{tkzPointResult}
\tkzDefPointBy[projection=onto #2--#4 ](\tkz@pttr@name \csname tkz@pt2\endcsname) 
\pgfnodealias{\csname tkz@ppt2\endcsname\csname tkz@pt2\endcsname}{tkzPointResult}
\tkzDefPointBy[projection=onto #2--#4 ](\tkz@pttr@name \csname tkz@pt3\endcsname) 
\pgfnodealias{\csname tkz@ppt2\endcsname\csname tkz@pt3\endcsname}{tkzPointResult}
\tkzDefPointBy[projection=onto #3--#2 ](\tkz@pttr@name \csname tkz@pt1\endcsname) 
\pgfnodealias{\csname tkz@ppt3\endcsname\csname tkz@pt1\endcsname}{tkzPointResult}
\tkzDefPointBy[projection=onto #3--#2 ](\tkz@pttr@name \csname tkz@pt2\endcsname) 
\pgfnodealias{\csname tkz@ppt3\endcsname\csname tkz@pt2\endcsname}{tkzPointResult}
\tkzDefPointBy[projection=onto #3--#2 ](\tkz@pttr@name \csname tkz@pt3\endcsname) 
\pgfnodealias{\csname tkz@ppt3\endcsname\csname tkz@pt3\endcsname}{tkzPointResult}
\endgroup 
} 
%<--------------------------------------------------------------------------–>
%              Point on circle
%<--------------------------------------------------------------------------–>
\def\tkz@numptcirc{0}
\pgfkeys{/tkzptcircle/.cd,
through/.code  args = {center #1 angle #2 point #3}        { \def\tkz@center{#1}%
                                                             \def\tkz@angle{#2}%
                                                             \def\tkz@through{#3}%
                                                             \def\tkz@numptcirc{0}},
R/.code args = {center #1 angle #2 radius #3}              { \def\tkz@center{#1}%
                                                             \def\tkz@angle{#2}%
                                                             \def\tkz@radius{#3}%
                                                             \def\tkz@numptcirc{1}},
through in rad/.code  args = {center #1 angle #2 point #3} { \def\tkz@center{#1}%
                                                             \def\tkz@angle{#2}%
                                                             \def\tkz@through{#3}%
                                                             \def\tkz@numptcirc{2}},
R in rad/.code args = {center #1 angle #2 radius #3}       { \def\tkz@center{#1}%
                                                             \def\tkz@angle{#2}%
                                                             \def\tkz@radius{#3}%
                                                             \def\tkz@numptcirc{3}}                                                 
}
                                                  
\def\tkzDefPointOnCircle{\pgfutil@ifnextchar[{\tkz@DefPointOnCircle}{%
                                              \tkz@DefPointOnCircle[]}}     
\def\tkz@DefPointOnCircle[#1]{%
\begingroup 
\pgfqkeys{/tkzptcircle}{#1}
 \ifcase\tkz@numptcirc%
    \tkz@@CalcLengthcm(\tkz@center,\tkz@through){tkz@radius}
     \or% 1
    \relax%
    \or% 2
    \pgfmathparse{\tkz@angle\space r}
    \let\tkz@angle\pgfmathresult
    \tkz@@CalcLengthcm(\tkz@center,\tkz@through){tkz@radius}
    \or% 3
    \pgfmathparse{\tkz@angle\space r}
    \let\tkz@angle\pgfmathresult
\fi
\path (\tkz@center) --++(\tkz@angle:\tkz@radius) coordinate(tkzPointResult);
\endgroup
} 
%<--------------------------------------------------------------------------–>
%              Point on line
%<--------------------------------------------------------------------------–>
\def\tkzDefPointOnLine{\pgfutil@ifnextchar[{\tkz@DefPointOnLine}{\tkz@DefPointOnLine[]}}     
\def\tkz@DefPointOnLine[#1](#2,#3){%
\begingroup 
\path (#2) to [#1] coordinate (tkzPointResult)  (#3);
\endgroup
} 

\makeatother
\endinput