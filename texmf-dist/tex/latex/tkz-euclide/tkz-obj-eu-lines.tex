% tkz-obj-eu-lines.tex
% Copyright 2024  Alain Matthes
% This work may be distributed and/or modified under the
% conditions of the LaTeX Project Public License, either version 1.3
% of this license or (at your option) any later version.
% The latest version of this license is in
%   http://www.latex-project.org/lppl.txt
% and version 1.3 or later is part of all distributions of LaTeX
% version 2005/12/01 or later.
% This work has the LPPL maintenance status “maintained”.
% The Current Maintainer of this work is Alain Matthes.

\def\fileversion{5.10c}
\def\filedate{2024/04/19} 
\typeout{2024/04/19 5.10c tkz-obj-eu-lines.tex}   
\makeatletter
%<--------------------------------------------------------------------------–>
%                          les lignes
%<--------------------------------------------------------------------------–>
\def\tkz@numline{0}
\pgfkeys{/tkzdefline/.cd,
  mediator/.code           =  \def\tkz@numline{0},
  perpendicular/.code args =  {through #1} {\def\tkz@numline{1}%
                                            \def\tkz@through{#1}},
  orthogonal/.code args    =  {through #1} {\def\tkz@numline{1}%
                                            \def\tkz@through{#1}}, 
  parallel/.code args      =  {through #1}{\def\tkz@numline{2}%
                                           \def\tkz@through{#1}},   
  bisector/.code           =  \def\tkz@numline{3},
  bisector out/.code       =  \def\tkz@numline{4},
  symmedian/.code          =  \def\tkz@numline{5},
  altitude/.code           =  \def\tkz@numline{6},
  euler/.code              =  \def\tkz@numline{7},
  tangent from/.code       =  \def\tkz@numline{8} \def\tkz@ptfrom{#1},
  tangent at/.code         =  \def\tkz@numline{9} \def\tkz@ptat{#1},
  K/.code                  =  \def\tkz@koeff{#1},
  K                        =  1,
  normed/.is if            =  tkz@line@normed,
  normed/.default          =  true,
  normed                   =  false,
  mediator
} 
\def\tkzDefLine{\pgfutil@ifnextchar[{\tkz@DefLine}{\tkz@DefLine[]}}
\def\tkz@DefLine[#1](#2){% 
\begingroup
%\pgfkeys{/tkzdefline/.cd,K=1}  
\pgfqkeys{/tkzdefline}{#1}  
\ifcase\tkz@numline%
 % first case 0
 \tkzDefMediatorLine(#2)  
  \or% 1
  \tkzDefOrthLine[\tkz@koeff](#2)
  \or% 2
   \tkzDefLineLL(#2)
  \or% 3
  \tkzDefBisectorLine(#2)
  \or% 4
  \tkzDefBisectorOutLine(#2)
  \or% 5
  \tkzDefSymmedianLine(#2)
    \or% 6
  \tkzDefAltitudeLine(#2)
  \or% 7
   \tkzDefEulerLine(#2)
   \or% 8
   \tkzTgtFromP(#2)(\tkz@ptfrom)
 \or% 9
   \tkzTgtAt(#2)(\tkz@ptat)
  \fi    
\endgroup
}
%<--------------------------------------------------------------------------–>
%                            tkzLineLL    revoir out !!
%<--------------------------------------------------------------------------–>
\def\tkzDefLineLL{\pgfutil@ifnextchar[{\tkz@DefLineLL}{%
                                         \tkz@DefLineLL[1]}} 
\def\tkz@DefLineLL[#1](#2,#3){% 
\begingroup
   \pgfpointdiff{\pgfpointanchor{#2}{center}}{\pgfpointanchor{#3}{center}}%
   \pgf@xa=\pgf@x\relax%%
   \pgf@ya=\pgf@y\relax%%
   \pgfinterruptboundingbox 
   \path[coordinate](\tkz@through)--++(\tkz@koeff\pgf@xa,\tkz@koeff\pgf@ya)%
         coordinate (tkzPointResult);
   \endpgfinterruptboundingbox
   \iftkz@line@normed
      \tkzVecKNorm(\tkz@through,tkzPointResult)
   \fi
\endgroup}% 
%<--------------------------------------------------------------------------–>
%                        tkzOrthLine 
%<--------------------------------------------------------------------------–>
\def\tkzDefOrthLine{\pgfutil@ifnextchar[{\tkz@DefOrthLine}{%
                                         \tkz@DefOrthLine[1]}} 
\def\tkz@DefOrthLine[#1](#2,#3){% 
\begingroup
 \tkzDefPointWith(#2,#3) 
  \pgfnodealias{tkz@OLtmp}{tkzPointResult}
  \tkz@VecKCoLinear[\tkz@koeff](#2,tkz@OLtmp,\tkz@through)
  \iftkz@line@normed
    \pgfinterruptboundingbox 
       \tkzVecKNorm(\tkz@through,tkzPointResult)
    \endpgfinterruptboundingbox 
  \fi
\endgroup}% 
%<--------------------------------------------------------------------------–>
%                            tkzMediatorLine
%<--------------------------------------------------------------------------–>
\def\tkzDefMediatorLine{\pgfutil@ifnextchar[{\tkz@DefMediatorLine}{%
                                             \tkz@DefMediatorLine[1]}} 

\def\tkz@DefMediatorLine[#1](#2,#3){% new 2020 
\begingroup
   \tkzDefEquilateral(#2,#3) 
   \pgfnodealias{tkzFirstPointResult}{tkzPointResult}
   \tkzDefEquilateral(#3,#2)
   \pgfnodealias{tkzSecondPointResult}{tkzPointResult} 
   \ifx\tkz@koeff=1 %
   \else
     \tkzDefMidPoint(#2,#3)
     \pgfnodealias{tkz@mid}{tkzPointResult}
     \tkz@VecK[\tkz@koeff](tkz@mid,tkzFirstPointResult)
     \pgfnodealias{tkzFirstPointResult}{tkzPointResult}
     \tkz@VecK[\tkz@koeff](tkz@mid,tkzSecondPointResult)
     \pgfnodealias{tkzSecondPointResult}{tkzPointResult}
   \fi
   \iftkz@line@normed 
      \tkzDefMidPoint(#2,#3)
      \pgfnodealias{tkz@mid}{tkzPointResult}
     % \pgfinterruptboundingbox
         \tkzVecKNorm[\tkz@koeff](tkz@mid,tkzFirstPointResult)
         \pgfnodealias{tkzFirstPointResult}{tkzPointResult}
         \tkzVecKNorm[\tkz@koeff](tkz@mid,tkzSecondPointResult)
         \pgfnodealias{tkzSecondPointResult}{tkzPointResult}
    %  \endpgfinterruptboundingbox
   \fi
\endgroup}
% autre possibilité

%<--------------------------------------------------------------------------–>
%              BisectorLine  
%<--------------------------------------------------------------------------–>
\def\tkzDefBisectorLine(#1,#2,#3){% 
\begingroup
  \pgfinterruptboundingbox 
  \tkzDuplicateSegment(#2,#1)(#2,#3)
  \pgfnodealias{bi@tmp}{tkzPointResult}  
  \tkzDefMidPoint(bi@tmp,#1)
%  \tkzDefEquilateral(bi@tmp,#1)
  \ifx\tkz@koeff=1 %
  \else
    \tkzVecK[\tkz@koeff](#2,tkzPointResult)
  \fi
  \iftkz@line@normed
     \tkzVecKNorm[\tkz@koeff](#2,tkzPointResult)
    \fi
    \endpgfinterruptboundingbox 
\endgroup
}
%<--------------------------------------------------------------------------–>
%              Out BisectorLine
%<--------------------------------------------------------------------------–>
\def\tkzDefBisectorOutLine(#1,#2,#3){% 
\begingroup
\pgfinterruptboundingbox 
   \tkzDuplicateLength(#2,#1)(#2,#3)
   \pgfnodealias{out@tmp}{tkzPointResult}
   \tkzDefMidPoint(#1,out@tmp) 
   \pgfnodealias{out@pt1}{tkzPointResult}
   \tkzURotateAngle(#2,90)(out@pt1)
   \ifx\tkz@koeff=1 %
   \else
     \tkzVecK[\tkz@koeff](#2,tkzPointResult)
   \fi
   \iftkz@line@normed
     \tkzVecKNorm[\tkz@koeff](#2,tkzPointResult)
    \fi
\endpgfinterruptboundingbox 
\endgroup
} 
%<--------------------------------------------------------------------------–>
%              Symmedian line
%<--------------------------------------------------------------------------–>
\def\tkzDefSymmedianLine(#1,#2,#3){% 
\begingroup
  \tkzDefBisectorLine(#1,#2,#3)
  \pgfnodealias{sym@pt1}{tkzPointResult}
  \tkzDefMidPoint(#1,#3)
  \pgfnodealias{sym@pt2}{tkzPointResult}
  \tkzUSymOrth(#2,sym@pt1)(sym@pt2)
   \ifx\tkz@koeff=1 %
   \else
     \tkzVecK[\tkz@koeff](#2,tkzPointResult)
   \fi
  \iftkz@line@normed
    \tkzVecKNorm[\tkz@koeff](#2,tkzPointResult)
  \fi
\endgroup
}
%<--------------------------------------------------------------------------–>
%              Altitude Line
%<--------------------------------------------------------------------------–>
\def\tkzDefAltitudeLine(#1,#2,#3){%
\begingroup
   \pgfinterruptboundingbox
   \tkzUProjection(#1,#3)(#2)
   \pgfnodealias{ort@pta}{tkzPointResult}
   \endpgfinterruptboundingbox  
\endgroup
}
%<--------------------------------------------------------------------------–>
%              Euler Line
%<--------------------------------------------------------------------------–>
\def\tkzDefEulerLine(#1,#2,#3){%
\begingroup
  \pgfinterruptboundingbox
  \tkzOrthoCenter(#1,#2,#3)  
  \pgfnodealias{euler@pt1}{tkzPointResult}
   \tkzEulerCenter(#1,#2,#3)  
   \pgfnodealias{euler@pt2}{tkzPointResult}
   \pgfnodealias{tkzSecondPointResult}{euler@pt2}
   \pgfnodealias{tkzFirstPointResult}{euler@pt1}
   \endpgfinterruptboundingbox  
\endgroup
}
%<--------------------------------------------------------------------------–>
%    tangente à cercle passant par un point donné
%<--------------------------------------------------------------------------–>
   
\def\tkzTgtAt(#1)(#2){%
\begingroup
     \tkz@VecKOrthNorm[-1](#2,#1)  
 \endgroup
}
%<-------------------------------------------------------------------------–> 
\def\tkzTgtFromP(#1,#2)(#3){%
\begingroup
    \tkz@@CalcLengthcm(#1,#2){tkz@radone}
    \tkzDefMidPoint(#1,#3) 
    \tkz@@CalcLengthcm(tkzPointResult,#1){tkz@radtwo}
    \tkzInterCCR(#1,\tkz@radone)(tkzPointResult,\tkz@radtwo){%
    tkzFirstPointResult}{tkzSecondPointResult}%
    \tkzFindAngle(#3,tkzFirstPointResult,#1)   \tkzGetAngle{tkz@an}
    \ifdim\tkz@an pt<180 pt\relax%
    \else
    \pgfnodealias{tkzPointTmp}{tkzSecondPointResult}
    \pgfnodealias{tkzSecondPointResult}{tkzFirstPointResult}
    \pgfnodealias{tkzFirstPointResult}{tkzPointTmp}
    \fi
\endgroup
} 
%<-------------------------------------------------------------------------–>  
\def\tkzTgtFromPR(#1,#2)(#3){%
\begingroup
    \tkzDefMidPoint(#1,#3)
    \tkz@@CalcLengthcm(tkzPointResult,#1){tkzLengthResult}
    \tkzInterCCR(#1,#2)(tkzPointResult,\tkzLengthResult){%
    tkzFirstPointResult}{%
    tkzSecondPointResult}%
    \tkzFindAngle(#3,tkzFirstPointResult,#1)   \tkzGetAngle{tkz@an}
    \ifdim\tkz@an pt<180 pt\relax%
    \else
    \pgfnodealias{tkzPointTmp}{tkzSecondPointResult}
    \pgfnodealias{tkzSecondPointResult}{tkzFirstPointResult}
    \pgfnodealias{tkzFirstPointResult}{tkzPointTmp}
    \fi
\endgroup
}
%<--------------------------------------------------------------------------–>
\def\tkz@numtang{0}
\pgfkeys{/tkz@tang/.cd,
  at/.code          = {\def\tkz@numtang{0}\def\tkz@ptat{#1}},
  from/.code        = {\def\tkz@numtang{1}\def\tkz@ptfrom{#1}},
  from with R/.code = {\def\tkz@numtang{2}\def\tkz@ptfrom{#1}}
  }
%<--------------------------------------------------------------------------–>
\def\tkzDefTangent{\pgfutil@ifnextchar[{\tkz@Tangent}{\tkz@Tangent[]}}

\def\tkz@Tangent[#1](#2){%
\begingroup
\pgfkeys{tkz@tang/.cd}
\pgfqkeys{/tkz@tang}{#1}
\ifcase\tkz@numtang
    \tkzTgtAt(#2)(\tkz@ptat)
\or
   \tkzTgtFromP(#2)(\tkz@ptfrom)
 \or
   \tkzTgtFromPR(#2)(\tkz@ptfrom)
\fi 
\endgroup
}   
\makeatother
\endinput