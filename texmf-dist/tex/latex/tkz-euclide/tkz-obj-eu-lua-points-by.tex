% tkz-tools-lua-points-by.tex
% Copyright 2024  Alain Matthes
% This work may be distributed and/or modified under the
% conditions of the LaTeX Project Public License, either version 1.3
% of this license or (at your option) any later version.
% The latest version of this license is in
%   http://www.latex-project.org/lppl.txt
% and version 1.3 or later is part of all distributions of LaTeX
% version 2005/12/01 or later.
% This work has the LPPL maintenance status “maintained”.
% The Current Maintainer of this work is Alain Matthes.

\def\fileversion{5.06c}
\def\filedate{2024/02/04} 
\typeout{2024/02/04 5.06c tkz-tools-lua-points-by.tex}  
\makeatletter
%<--------------------------------------------------------------------------–>
%                        Transformations Géométriques
%<--------------------------------------------------------------------------–>
\def\tkz@numtrsf{0}
\pgfkeys{/tkzDefPointBy/.cd,
 translation/.code args={from #1 to #2}{               \def\tkzfrom{#1}%
                                                       \def\tkzto{#2}%
                                                       \def\tkz@numtrsf{0}},
 homothety/.code args={center #1 ratio #2}{            \def\tkzcenter{#1}%
                                                       \def\tkzratio{#2}%
                                                       \def\tkz@numtrsf{1}},
 reflection/.code args={over #1--#2}{                  \def\tkzdeb{#1}%
                                                       \def\tkzfin{#2}%
                                                       \def\tkz@numtrsf{2}},
 symmetry/.code args={center #1}{                      \def\tkzcenter{#1}%
                                                       \def\tkz@numtrsf{3}},
 projection/.code args={onto #1--#2}{                  \def\tkzdeb{#1}%
                                                       \def\tkzfin{#2}%  
                                                       \def\tkz@numtrsf{4}}, 
 rotation/.code args={center #1 angle #2}{             \def\tkzcenter{#1}%
                                                       \def\tkzangle{#2}%
                                                       \def\tkz@numtrsf{5}},
 rotation in rad/.code args={center #1 angle #2}{      \def\tkzcenter{#1}%
                                                       \def\tkzangle{#2}%
                                                       \def\tkz@numtrsf{6}},
 inversion/.code args={center #1 through #2}{          \def\tkzcenter{#1}%
                                                       \def\tkzpoint{#2}%
                                                       \def\tkz@numtrsf{7}},
 inversion negative/.code args={center #1 through #2}{ \def\tkzcenter{#1}%
                                                       \def\tkzpoint{#2}%
                                                       \def\tkz@numtrsf{8}},
 rotation with nodes/.code  args={center #1 from #2 to #3}{  \def\tkzcenter{#1}%
                                                            \def\tkzfrom{#2}%
                                                            \def\tkzto{#3}%
                                                            \def\tkz@numtrsf{9}}
}
%<--------------------------------------------------------------------------–>
\def\tkzDefPointBy{\pgfutil@ifnextchar[{\tkz@DefPointBy}{\tkz@DefPointBy[]}}
\def\tkz@DefPointBy[#1](#2){% 
\begingroup 
\pgfqkeys{/tkzDefPointBy}{#1}    
 \ifcase\tkz@numtrsf%
%  % first case 0   
   \tkzUTranslation(\tkzfrom,\tkzto)(#2) 
\or% 1
   \tkzUHomo(\tkzcenter,\tkzratio)(#2)
\or% 2
   \tkzUSymOrth(\tkzdeb,\tkzfin)(#2) 
\or% 3
   \tkzUCSym(\tkzcenter)(#2)
\or% 4
   \tkzUProjection(\tkzdeb,\tkzfin)(#2)  
\or% 5  
   \tkzURotateAngle(\tkzcenter,\tkzangle)(#2)
\or% 6 
   \tkzURotateInRad(\tkzcenter,\tkzangle)(#2)
\or% 7
   \tkzUInversePoint(\tkzcenter,\tkzpoint)(#2)     
\or% 8
   \tkzUInverseNegativePoint(\tkzcenter,\tkzpoint)(#2)   
\or% 9
   \tkzURotateWithNodes(\tkzcenter,\tkzfrom,\tkzto)(#2)
\fi   
\endgroup
}
%<--------------------------------------------------------------------------–>
\def\tkzDefPointsBy{\pgfutil@ifnextchar[{\tkz@DefPointsBy}{\tkz@DefPointsBy[]}}
\def\tkz@DefPointsBy[#1](#2)#3{% 
\begingroup
\pgfqkeys{/tkzDefPointBy}{#1}  
\ifcase\tkz@numtrsf%
 % first case 0
   \tkzTranslation(\tkzfrom,\tkzto)(#2){#3}  
 \or% 1
   \tkzHomo(\tkzcenter,\tkzratio)(#2){#3}
 \or% 2
   \tkzSymOrth(\tkzdeb,\tkzfin)(#2){#3} 
 \or% 3
   \tkzCSym(\tkzcenter)(#2){#3}
 \or% 4
   \tkzProjection(\tkzdeb,\tkzfin)(#2){#3}
 \or% 5
   \tkzRotateAngle(\tkzcenter,\tkzangle)(#2){#3}
 \or% 6
   \tkzRotateInRad(\tkzcenter,\tkzangle)(#2){#3}
 \or% 7
   \tkzInversePoint(\tkzcenter,\tkzpoint)(#2){#3}    
 \or% 8
   \tkzInverseNegativePoint(\tkzcenter,\tkzpoint)(#2){#3}
   \or% 9
 \tkzRotateWithNodes(\tkzcenter,\tkzfrom,\tkzto)(#2){#3}
\fi    
\endgroup
} 

%<--------------------------------------------------------------------------–>

\def\ExtractPoint#1,#2\@nil{% 
\xdef\tkz@LastList{#2}
\xdef\tkz@FirstPoint{#1}  
} 
\def\FirstPointInList#1{% 
\edef\tkz@templist{#1,}
\expandafter\ExtractPoint\tkz@templist\@nil
}
%<--------------------------------------------------------------------------–>
%  Translation par rapport à un point
%<--------------------------------------------------------------------------–>
\def\tkzTranslation(#1,#2)(#3)#4{%
\begingroup 
\gdef\tkz@LastList{#4}
\foreach\PT in {#3}{%
   \FirstPointInList\tkz@LastList
    \ifx\tkz@FirstPoint\tkzutil@empty  
      \def\tkz@pointtsf{\PT '}
    \else
      \def\tkz@pointtsf{\tkz@FirstPoint}
    \fi 
   \tkz@VecCoLinear(#1,#2,\PT) 
    \pgfnodealias{\tkz@pointtsf}{tkzPointResult}       
   }  
\endgroup
}
%<--------------------------------------------------------------------------–>
\def\tkzUTranslation(#1,#2)(#3){%
\begingroup 
  \tkz@VecCoLinear(#1,#2,#3)% 
\endgroup
}  

%<--------------------------------------------------------------------------–>
%  Symétrie par rapport à un point Homo with (-1) 
% #2 le centre #3 l'antécédent 
%<--------------------------------------------------------------------------–>
\def\tkzCSym(#1)(#2)#3{%
\begingroup
\gdef\tkz@LastList{#3}
 \foreach\PointCS in {#2}{%
   \FirstPointInList\tkz@LastList
   \ifx\tkz@FirstPoint\tkzutil@empty  
      \xdef\tkz@pointtsf{\PointCS '}
   \else
      \xdef\tkz@pointtsf{\tkz@FirstPoint}
   \fi 
     \pgfpointdiff{\pgfpointanchor{#1}{center}}%
                  {\pgfpointanchor{\PointCS}{center}}%
     \tkz@ax=\pgf@x%
     \tkz@ay=\pgf@y%
 \pgfinterruptboundingbox
     \path(#1)--++(-\tkz@ax,-\tkz@ay)coordinate (\tkz@pointtsf); 
  \endpgfinterruptboundingbox
}       
\endgroup 
}
%<--------------------------------------------------------------------------–>
\def\tkzUCSym(#1)(#2){%
\begingroup
    \pgfpointdiff{\pgfpointanchor{#1}{center}}%
                 {\pgfpointanchor{#2}{center}}%
    \tkz@ax=\pgf@x%
    \tkz@ay=\pgf@y%
   \path(#1)--++(-\tkz@ax,-\tkz@ay)coordinate (tkzPointResult); 
\endgroup 
} 
%<--------------------------------------------------------------------------–>
%  Symétrie orthogonale par rapport à une droite
%<--------------------------------------------------------------------------–> 
\def\tkzSymOrth(#1,#2)(#3)#4{%
\begingroup 
\gdef\tkz@LastList{#4}
    \foreach\PointSO in {#3}{%
      \FirstPointInList\tkz@LastList
      \ifx\tkz@FirstPoint\tkzutil@empty  
         \def\tkz@pointtsf{\PointSO '}
      \else
         \def\tkz@pointtsf{\tkz@FirstPoint}
      \fi 
  \tkzUSymOrth(#1,#2)(\PointSO)    
  \pgfnodealias{\tkz@pointtsf}{tkzPointResult} 
  }    
\endgroup
}
%<--------------------------------------------------------------------------–>
\def\tkzUSymOrth(#1,#2)(#3){%
\begingroup
   \pgfpointdiff{\pgfpointanchor{#1}{center}}%
                {\pgfpointanchor{#2}{center}}%
   \tkz@ax =\pgf@y%
   \tkz@ay =\pgf@x%  
   \pgfinterruptboundingbox
   \path[coordinate] (#3)--++(-\tkz@ax,\tkz@ay) coordinate  (tkz@point);
     \endpgfinterruptboundingbox
   \tkzInterLL(#1,#2)(#3,tkz@point)
   \pgfnodealias{tkzPointofSym}{tkzPointResult}
   \tkz@VecK[2](#3,tkzPointofSym)
\endgroup
}

%<--------------------------------------------------------------------------–>
%  Projection orthogonale sur une droite
%<--------------------------------------------------------------------------–>
\def\tkzProjection(#1,#2)(#3)#4{%
\begingroup
\gdef\tkz@LastList{#4}
    \foreach\PointPJ in {#3}{%
      \FirstPointInList\tkz@LastList
      \ifx\tkz@FirstPoint\tkzutil@empty
         \def\tkz@pointtsf{\PointPJ '}
      \else
         \def\tkz@pointtsf{\tkz@FirstPoint}
      \fi
  \tkzUProjection(#1,#2)(\PointPJ)
  \pgfnodealias{\tkz@pointtsf}{tkzPointResult}
  }
\endgroup
}
%<--------------------------------------------------------------------------–>
\def\tkzUProjection(#1,#2)(#3){%
\begingroup 
  \pgfpointdiff{\pgfpointanchor{#1}{center}}%
               {\pgfpointanchor{#2}{center}}%
  \tkz@ax =\pgf@y%
  \tkz@ay =\pgf@x%
  \pgfinterruptboundingbox
    \path[coordinate](#3)--++(-\tkz@ax,\tkz@ay) coordinate (tkz@point);
    \tkzInterLL(#1,#2)(#3,tkz@point)% définit tkzPointResult 
  \endpgfinterruptboundingbox
\endgroup
}
%<--------------------------------------------------------------------------–>
\def\tkz@Projection(#1,#2)(#3)#4{%
\begingroup 
  \pgfpointdiff{\pgfpointanchor{#1}{center}}%
               {\pgfpointanchor{#2}{center}}%
  \tkz@ax =\pgf@y%
  \tkz@ay =\pgf@x%
   \pgfinterruptboundingbox
  \path[coordinate](#3)--++(-\tkz@ax,\tkz@ay) coordinate (tkz@point);
   \endpgfinterruptboundingbox
  \tkz@InterLL(#1,#2)(#3,tkz@point){#4}% définit tkzPointResult 
\endgroup
}
%<--------------------------------------------------------------------------–>
%  Homothétie par rapport à un point
%<--------------------------------------------------------------------------–>
\def\tkzHomo(#1,#2)(#3)#4{%
\begingroup 
\gdef\tkz@LastList{#4}
\foreach\PointHO in {#3}{%
  \FirstPointInList\tkz@LastList
  \ifx\tkz@FirstPoint\tkzutil@empty  
     \xdef\tkz@pointtsf{\PointHO '}
  \else
     \xdef\tkz@pointtsf{\tkz@FirstPoint}
  \fi 
   \pgfpointdiff{\pgfpointanchor{#1}{center}}%
                {\pgfpointanchor{\PointHO}{center}}%
  \pgf@xa=\pgf@x%
  \pgf@ya=\pgf@y% 
  \pgfmathparse{#2}
  \edef\tkz@coeff{\pgfmathresult}%
  \pgfinterruptboundingbox
  \path[coordinate](#1)--++(\tkz@coeff\pgf@xa,\tkz@coeff\pgf@ya)%
        coordinate(\tkz@pointtsf);
 \endpgfinterruptboundingbox
  }  
\endgroup
}
%<--------------------------------------------------------------------------–>
\def\tkzUHomo(#1,#2)(#3){%
\begingroup 
 \pgfpointdiff{\pgfpointanchor{#1}{center}}%
              {\pgfpointanchor{#3}{center}}%
 \pgf@xa=\pgf@x%
 \pgf@ya=\pgf@y% 
 \pgfmathparse{#2}
 \edef\tkz@coeff{\pgfmathresult}%
 \pgfinterruptboundingbox
 \path[coordinate](#1)--++(\tkz@coeff\pgf@xa,\tkz@coeff\pgf@ya)%
      coordinate(tkzPointResult);
        \endpgfinterruptboundingbox
\endgroup
}
%<--------------------------------------------------------------------------–>
%                 rotation  en degré
%<--------------------------------------------------------------------------–>
\def\tkzRotateAngle(#1,#2)(#3)#4{%
\begingroup
\gdef\tkz@LastList{#4}
\foreach\PointRot in {#3}{%
  \FirstPointInList\tkz@LastList
  \ifx\tkz@FirstPoint\tkzutil@empty  
     \def\tkz@pointtsf{\PointRot '}
  \else
     \def\tkz@pointtsf{\tkz@FirstPoint}
  \fi 
    \tkz@@extractxy{\PointRot}
    \tkz@ax\pgf@x%
    \tkz@ay\pgf@y%
    \tkz@@extractxy{#1}
    \tkz@bx\pgf@x%
    \tkz@by\pgf@y%
    \tkzmathrotatepointaround{\pgfpoint{\tkz@ax}{\tkz@ay}}%
                             {\pgfpoint{\tkz@bx}{\tkz@by}}%
                             {#2}
    \tkz@bx\pgf@x%
    \tkz@by\pgf@y%
    \pgfinterruptboundingbox
    \path[coordinate](\tkz@bx,\tkz@by)coordinate(\tkz@pointtsf);% 
    \endpgfinterruptboundingbox
}   
  \endgroup
}
%<--------------------------------------------------------------------------–>
\def\tkzURotateAngle(#1,#2)(#3){%
\begingroup 
  \pgf@process{\pgfpointanchor{#3}{center}}%
    \tkz@ax\pgf@x%
    \tkz@ay\pgf@y%
    \pgf@process{\pgfpointanchor{#1}{center}}%
    \tkz@bx\pgf@x%
    \tkz@by\pgf@y%
    \tkzmathrotatepointaround{\pgfpoint{\tkz@ax}{\tkz@ay}}%
                             {\pgfpoint{\tkz@bx}{\tkz@by}}%
                             {#2}
    \tkz@bx\pgf@x%
    \tkz@by\pgf@y%
    \pgfinterruptboundingbox
      \path (\tkz@bx,\tkz@by) coordinate (tkzPointResult);% 
    \endpgfinterruptboundingbox     
\endgroup
}   
%<--------------------------------------------------------------------------–>
% %                 rotation  en radian
% %<--------------------------------------------------------------------------–>
\def\tkzRotateInRad(#1,#2)(#3)#4{%
\begingroup
\gdef\tkz@LastList{#4}
    \foreach\PointRot in {#3}{%
      \FirstPointInList\tkz@LastList
      \ifx\tkz@FirstPoint\tkzutil@empty  
         \xdef\tkz@pointtsf{\PointRot '}
      \else
         \xdef\tkz@pointtsf{\tkz@FirstPoint}
      \fi 
   \pgfmathparse{#2 r}
   \let\tkz@Angle\pgfmathresult
    \tkz@@extractxy{\PointRot}
    \tkz@ax\pgf@x%
    \tkz@ay\pgf@y%
    \tkz@@extractxy{#1}
    \tkz@bx\pgf@x%
    \tkz@by\pgf@y%
    \tkzmathrotatepointaround{\pgfpoint{\tkz@ax}{\tkz@ay}}%
                             {\pgfpoint{\tkz@bx}{\tkz@by}}%
                             {\tkz@Angle}
    \tkz@bx\pgf@x%
    \tkz@by\pgf@y%
      \pgfinterruptboundingbox
    \path[coordinate](\tkz@bx,\tkz@by)coordinate(\tkz@pointtsf); 
      \endpgfinterruptboundingbox 
}  
\endgroup 
}
%<--------------------------------------------------------------------------–>
\def\tkzURotateInRad(#1,#2)(#3){%
\begingroup
   \pgfmathparse{#2 r}
   \let\tkz@Angle\pgfmathresult
    \tkz@@extractxy{#3}
    \tkz@ax\pgf@x%
    \tkz@ay\pgf@y%
    \tkz@@extractxy{#1}
    \tkz@bx\pgf@x%
    \tkz@by\pgf@y%
    \tkzmathrotatepointaround{\pgfpoint{\tkz@ax}{\tkz@ay}}%
                             {\pgfpoint{\tkz@bx}{\tkz@by}}%
                             {\tkz@Angle}
    \tkz@bx\pgf@x%
    \tkz@by\pgf@y%
      \pgfinterruptboundingbox
    \path[coordinate](\tkz@bx,\tkz@by)coordinate(tkzPointResult); 
      \endpgfinterruptboundingbox 
\endgroup 
}
%<--------------------------------------------------------------------------–>
%                   Inverse of a point 
%<--------------------------------------------------------------------------–>
\def\tkzInversePoint(#1,#2)(#3)#4{%
\begingroup 
\gdef\tkz@LastList{#4}
    \foreach\PointIP in {#3}{%
      \FirstPointInList\tkz@LastList
      \ifx\tkz@FirstPoint\tkzutil@empty  
         \xdef\tkz@pointtsf{\PointIP '}
      \else
         \xdef\tkz@pointtsf{\tkz@FirstPoint}
      \fi 
    \tkz@@CalcLengthcm(#1,#2){tkz@lna}
    \tkz@@CalcLengthcm(#1,\PointIP){tkz@lnb}
    \edef\tkz@lnc{\fpeval{\tkz@lna/\tkz@lnb*\tkz@lna}}
    \tkzVecKNorm[\tkz@lnc](#1,\PointIP) 
    \pgfnodealias{\tkz@pointtsf}{tkzPointResult}
    }  
\endgroup
} 
\def\tkzUInversePoint(#1,#2)(#3){%  
\begingroup  
   \tkz@@CalcLengthcm(#1,#2){tkz@lna}% 
   \tkz@@CalcLengthcm(#1,#3){tkz@lnb}% 
   \edef\tkz@lnc{\fpeval{\tkz@lna/\tkz@lnb*\tkz@lna}}
   \tkzVecKNorm[\tkz@lnc](#1,#3) 
\endgroup
} 
% possible 
% \tkzDefLine[tangent from =#3](#1,#2) 
% \tkzTgtFromP(#1,#2)(#3) 
% \tkzInterLL(tkzFirstPointResult,tkzSecondPointResult)(#1,#2) 
%<--------------------------------------------------------------------------–>
%                   Inverse negative of a point 
%<--------------------------------------------------------------------------–>
\def\tkzInverseNegativePoint(#1,#2)(#3)#4{%
\begingroup 
\gdef\tkz@LastList{#4}
    \foreach\PointIP in {#3}{%
      \FirstPointInList\tkz@LastList
      \ifx\tkz@FirstPoint\tkzutil@empty  
         \xdef\tkz@pointtsf{\PointIP '}
      \else
         \xdef\tkz@pointtsf{\tkz@FirstPoint}
      \fi 
   \tkz@@CalcLengthcm(#1,#2){tkz@lna}
   \tkz@@CalcLengthcm(#1,\PointIP){tkz@lnb}
   \edef\tkz@lnc{\fpeval{\tkz@lna/\tkz@lnb*\tkz@lna}}
   \tkzVecKNorm[\tkz@lnc](#1,\PointIP) 
   \tkzUCSym(#1)(tkzPointResult)
   \pgfnodealias{\tkz@pointtsf}{tkzPointResult}
   }  
\endgroup
} 
%<--------------------------------------------------------------------------–>
\def\tkzUInverseNegativePoint(#1,#2)(#3){%  
\begingroup  
   \tkz@@CalcLengthcm(#1,#2){tkz@lna}% 
   \tkz@@CalcLengthcm(#1,#3){tkz@lnb}% 
   \edef\tkz@lnc{\fpeval{\tkz@lna/\tkz@lnb*\tkz@lna}}
   \tkzVecKNorm[\tkz@lnc](#1,#3) 
   \tkzUCSym(#1)(tkzPointResult)
\endgroup
}
%<--------------------------------------------------------------------------–>
%<--------------- rotate with nodes                 ------------------------–>
%<--------------------------------------------------------------------------–>
\def\tkzRotateWithNodes(#1,#2,#3)(#4)#5{%
\begingroup
\gdef\tkz@LastList{#5}
 \foreach\PointRotWN in {#4}{%
   \FirstPointInList\tkz@LastList
   \ifx\tkz@FirstPoint\tkzutil@empty  
      \def\tkz@pointtsf{\PointRotWN '}
   \else
      \def\tkz@pointtsf{\tkz@FirstPoint}
   \fi 
   \tkzFindAngle(#2,#1,#3)
   \tkz@@extractxy{\PointRotWN}
   \tkz@ax\pgf@x%
   \tkz@ay\pgf@y%
   \tkz@@extractxy{#1}
   \tkz@bx\pgf@x%
   \tkz@by\pgf@y%
   \tkzmathrotatepointaround{\pgfpoint{\tkz@ax}{\tkz@ay}}%
                        {\pgfpoint{\tkz@bx}{\tkz@by}}%
                        {\tkzAngleResult}
   \tkz@bx\pgf@x%
   \tkz@by\pgf@y%
   \pgfinterruptboundingbox
   \path[coordinate](\tkz@bx,\tkz@by) coordinate (\tkz@pointtsf);% 
   \endpgfinterruptboundingbox
}   
\endgroup
}
%<--------------------------------------------------------------------------–>
\def\tkzURotateWithNodes(#1,#2,#3)(#4){%
\begingroup 
  \tkzFindAngle(#2,#1,#3)
  \pgf@process{\pgfpointanchor{#4}{center}}%
  \tkz@ax\pgf@x%
  \tkz@ay\pgf@y%
  \pgf@process{\pgfpointanchor{#1}{center}}%
  \tkz@bx\pgf@x%
  \tkz@by\pgf@y%
  \tkzmathrotatepointaround{\pgfpoint{\tkz@ax}{\tkz@ay}}%
                           {\pgfpoint{\tkz@bx}{\tkz@by}}%
                           {\tkzAngleResult}
  \tkz@bx\pgf@x%
  \tkz@by\pgf@y%
  \pgfinterruptboundingbox
  \path (\tkz@bx,\tkz@by) coordinate (tkzPointResult);%
  \endpgfinterruptboundingbox  
\endgroup
}   
%<--------------------------------------------------------------------------–>
%                   Fin des transformations
%<--------------------------------------------------------------------------–>
\makeatother
\endinput