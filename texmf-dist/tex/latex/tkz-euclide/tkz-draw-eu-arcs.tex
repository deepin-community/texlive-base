% tkz-tool-eu-arcs.tex
% Copyright 2024  Alain Matthes
% This work may be distributed and/or modified under the
% conditions of the LaTeX Project Public License, either version 1.3
% of this license or (at your option) any later version.
% The latest version of this license is in
%   http://www.latex-project.org/lppl.txt
% and version 1.3 or later is part of all distributions of LaTeX
% version 2005/12/01 or later.
% This work has the LPPL maintenance status “maintained”.
% The Current Maintainer of this work is Alain Matthes.

\def\fileversion{5.10c}
\def\filedate{2024/04/19} 
\typeout{2024/04/19 5.10c tkz-tool-eu-arcs.tex} 
\makeatletter
%<--------------------------------------------------------------------------–>
%                    tkzSetUpArc  
%<--------------------------------------------------------------------------–>
\pgfkeys{/tkzsetuparc/.cd,
   color/.store in        =    \tkz@arc@color,
   line width/.store in   =    \tkz@arc@lw,
   style/.store in        =    \tkz@arc@style,
   line width             =    \tkz@euc@linewidth,
   color                  =    \tkz@euc@linecolor,
   style                  =    \tkz@euc@linestyle,
   /tkzsetuparc/.search also = {/tikz,/tkzcompass},
 } 
\def\tkzSetUpArc{\pgfutil@ifnextchar[{\tkz@SetUpArc}{\tkz@SetUpArc[]}}
\def\tkz@SetUpArc[#1]{%
\pgfqkeys{/tkzsetuparc}{#1}
\tikzset{arc style/.style={color          = \tkz@arc@color,
                           line width     = \tkz@arc@lw,
                           style          = \tkz@arc@style
                           }}
}% end setup 
%<------------------------------  Arcs -------------------------------------–
% options : delta 
% \def\tkz@delta{0}
% \tikzset{arc style/.style={#1}}
% \pgfkeys{/tikz/.cd,delta/.code={\def\tkz@delta{#1}}}
\newif\iftkz@reverse
\gdef\tkz@numa{0}
\pgfkeys{/tkzdrawarc/.cd,
      type/.is choice,
      type/towards/.code               = \def\tkz@numa{0},
      type/rotate/.code                = \def\tkz@numa{1},
      type/angles/.code                = \def\tkz@numa{2}, 
      type/R/.code                     = \def\tkz@numa{3},
      type/R with nodes/.code          = \def\tkz@numa{4},
      towards/.style                   = {type=towards},
      rotate/.style                    = {type=rotate},
      R/.style                         = {type=R},
      angles/.style                    = {type=angles},
      R with nodes/.style              = {type=R with nodes},
      diameter/.code                   = {},
      arc/.code                        = {},
      size/.code                       = {},
      mark/.code                       = {},
      mkpos/.code                      = {},
      mksize/.code                     = {},
      mkcolor/.code                    = {},
      type/.default                    =  towards,
      delta/.store in                  = \tkz@delta,
      delta                            = 0,
      reverse/.is if                   = tkz@reverse,
      reverse/.default                 = true,
      reverse                          = false,
      /tkzdrawarc/.search also         = {/tikz}
}  
\def\tkzDrawArc{\pgfutil@ifnextchar[{\tkz@DrawArc}{\tkz@DrawArc[]}}
\def\tkz@DrawArc[#1](#2,#3)(#4){% 
\begingroup    
\pgfqkeys{/tkzdrawarc}{#1} 
\ifcase\tkz@numa%
   \tkzDrawArcTowards[#1](#2,#3)(#4)  
\or% 1
   \tkzDrawArcRotate[#1](#2,#3)(#4)  
\or% 2
   \tkzDrawArcAngles[#1](#2,#3)(#4) 
\or% 3
   \tkzDrawArcRAngles[#1](#2,#3)(#4)
\or% 4
   \tkzDrawArcR[#1](#2,#3)(#4)  
\fi    
\endgroup
}
%<--------------------------------------------------------------------------–>
%  ARC    nodes        #2 center #3 first point last point #4
% delta un peu plus à chaque extrémité 
% example : \tkzDrawArc(A,B)(C)      
%<--------------------------------------------------------------------------–>
\def\tkzDrawArcTowards{\pgfutil@ifnextchar[{\tkz@DrawArcN}{\tkz@DrawArcN[]}} 
\def\tkz@DrawArcN[#1](#2,#3)(#4){%
\begingroup
  \tkz@@CalcLengthcm(#2,#3){tkz@radius}
  \tkzFindSlopeAngle(#2,#3)\tkzGetAngle{tkz@FirstAngle}  
  \tkzFindSlopeAngle(#2,#4)\tkzGetAngle{tkz@SecondAngle}
  \tkz@DrawArcRAngles[#1](#2,\tkz@radius)(\tkz@FirstAngle,\tkz@SecondAngle) 
\endgroup
} 
%<--------------------------------------------------------------------------–>
%    nodes                 #2 center #3 first point rotate #4 with Angle
% delta un peu plus à chaque extrémité 
% tkzDrawArcRotate(O,A)(60)
%<--------------------------------------------------------------------------–>
\def\tkzDrawArcRotate{\pgfutil@ifnextchar[{\tkz@DrawArcRotate}{%
                                           \tkz@DrawArcRotate[]}}  
\def\tkz@DrawArcRotate[#1](#2,#3)(#4){%
\begingroup
  \tkz@@CalcLength(#2,#3){tkz@radius} 
  \tkzFindSlopeAngle(#2,#3)   \tkzGetAngle{tkz@FirstA}
  \pgfmathadd{\tkz@FirstA}{#4}
  \edef\tkz@SecondA{\pgfmathresult}
    \pgfmathgreaterthan{#4}{0}
  \ifdim\pgfmathresult pt=1 pt\relax%
    \tkz@DrawArcRAngles[#1](#2,\tkz@radius pt)(\tkz@FirstA,\tkz@SecondA)
  \else
    \tkz@DrawArcRAngles[#1](#2,\tkz@radius pt)(\tkz@SecondA,\tkz@FirstA)
  \fi
  \endgroup
}   
%<--------------------------------------------------------------------------–>
%  deux angles 
% \tkzDrawArcAngles(O,A)(0,60)
%<--------------------------------------------------------------------------–>
\def\tkzDrawArcAngles{\pgfutil@ifnextchar[{\tkz@DrawArcAngles}{%
                                           \tkz@DrawArcAngles[]}}  
\def\tkz@DrawArcAngles[#1](#2,#3)(#4,#5){%
\begingroup
    \tkz@@CalcLength(#2,#3){tkzLengthResult}
    \tkz@DrawArcRAngles[#1](#2,\tkzLengthResult pt)(#4,#5)      
\endgroup
}   
%<--------------------------------------------------------------------------–>
%    Degree      #2 center #4 - #3 radius from #5 (degree) to #6(degree) 
%<--------------------------------------------------------------------------–>
\def\tkzDrawArcRwithNodes{\pgfutil@ifnextchar[{\tkz@DrawArcRwithNodes}{%
                                            \tkz@DrawArcRwithNodes[]}}  
\def\tkz@DrawArcRwithNodes[#1](#2,#3,#4)(#5,#6){%
\begingroup
  \tkz@@CalcLength(#3,#4){tkzLengthResult}  
  \tkzFindSlopeAngle(#2,#5)\tkzGetAngle{tkz@FirstAngle}  
  \tkzFindSlopeAngle(#2,#6)\tkzGetAngle{tkz@SecondAngle}
  \tkz@DrawArcRAngles[#1](#2,\tkzLengthResult)(\tkz@FirstAngle,\tkz@SecondAngle)
\endgroup   
} 
%<--------------------------------------------------------------------------–>
%    Nodes R  #2 center #3 radius en cm  from #4(node) to #5(node)
%  \tkzDrawArcR(O,2)(A,B)
%<--------------------------------------------------------------------------–>
\def\tkzDrawArcR{\pgfutil@ifnextchar[{\tkz@DrawArcR}{\tkz@DrawArcR[]}} 
\def\tkz@DrawArcR[#1](#2,#3)(#4,#5){%
\begingroup
   \tkzFindSlopeAngle(#2,#4)\tkzGetAngle{tkz@FirstAngle}  
   \tkzFindSlopeAngle(#2,#5)\tkzGetAngle{tkz@SecondAngle} 
   \tkz@DrawArcRAngles[#1](#2,#3)(\tkz@FirstAngle,\tkz@SecondAngle)  
\endgroup   
}
%<--------------------------------------------------------------------------–>
%<--------------------------------------------------------------------------–>
% #1 center #2 radius #4 first angle (degree) #5 second angle  (degree)  
% angles  0 .. 180 or -180 .. 0
%<--------------------------------------------------------------------------–>
% example : \tkzDrawArc(A,2)(30,90)    
\def\tkzDrawArcRAngles{\pgfutil@ifnextchar[{\tkz@DrawArcRAngles}{%
                                            \tkz@DrawArcRAngles[]}} 
\def\tkz@DrawArcRAngles[#1](#2,#3)(#4,#5){% 
 \begingroup
 \tkzNormalizeAngle(#4,#5)    
 \pgfmathsubtract{\tkz@FirstAngle}{\tkz@delta}
 \edef\tkz@FirstAngle{\pgfmathresult}%
 \pgfmathadd{\tkz@SecondAngle}{\tkz@delta}
 \edef\tkz@SecondAngle{\pgfmathresult} 
 \iftkz@reverse 
   \let\tkztemp\tkz@FirstAngle 
   \let\tkz@FirstAngle\tkz@SecondAngle
   \let\tkz@SecondAngle\tkztemp
  \fi
  \draw[shift = {(#2)},arc style,/tkzdrawarc/.cd,#1]%
  (\tkz@FirstAngle:#3) arc (\tkz@FirstAngle:\tkz@SecondAngle:#3);
\endgroup  
}

%<--------------------------------------------------------------------------->
%                       tkzDrawArcRAN                                        >
%<--------------------------------------------------------------------------->
\def\tkzDrawArcRAN[#1](#2,#3)(#4,#5){% 
\begingroup 
    \draw[shift = {(#2)},/tkzdrawarc/.cd,#1] (#4:#3) arc (#4:#5:#3); 
\endgroup  
}
%<--------------- tkzPathArcRAN ----------------------------------
\def\tkzPathArcRAN[#1](#2,#3)(#4,#5){% 
\begingroup     
    \path[shift = {(#2)},/tkzdrawarc/.cd,#1] (#4:#3) arc (#4:#5:#3); 
\endgroup  
}
%<--------------------------------------------------------------------------->
%%     tkzdefMidArc center and two points
\def\tkzDefMidArc(#1,#2,#3){%
\begingroup
\tkz@@CalcLength(#1,#2){tkz@radius}
\tkzFindSlopeAngle(#1,#2)\tkzGetAngle{tkz@FirstAngle}
\tkzFindSlopeAngle(#1,#3)\tkzGetAngle{tkz@SecondAngle}
\tkzNormalizeAngle(\tkz@FirstAngle,\tkz@SecondAngle)
 \path[shift = {(#1)}](\tkz@FirstAngle:\tkz@radius pt) arc
           (\tkz@FirstAngle:\tkz@SecondAngle:\tkz@radius pt)  coordinate[midway] (tkzPointResult);
\endgroup
}
%<--------------------------------------------------------------------------–>
\pgfkeys{
  /@tkzmarkarc/.cd,
     pos/.store in       = \tkz@mkarcpos,
     color/.store in     = \tkz@mkcolor,
     mark/.store in      = \tkz@markarcseg,
     size/.store in      = \tkz@mkarcsize,
     color               = teal!20,
     size                = 4pt,
     pos                 = .5,
     mark                = none,
    /@tkzmarkarc/.search also={/tikz},
}
\def\tkzMarkArc{\pgfutil@ifnextchar[{\tkz@MarkArc}{\tkz@MarkArc[]}}   
\def\tkz@MarkArc[#1](#2,#3,#4){% 
\begingroup
 \pgfqkeys{/@tkzmarkarc}{#1} \def\tkz@mymarkarc{\pgfsetplotmarksize{\tkz@mkarcsize}\pgfuseplotmark{\tkz@markarcseg}}
\tkz@@CalcLength(#2,#3){tkz@radius}
\tkzFindSlopeAngle(#2,#3)\tkzGetAngle{tkz@FirstAngle}  
\tkzFindSlopeAngle(#2,#4)\tkzGetAngle{tkz@SecondAngle}
\tkzNormalizeAngle(\tkz@FirstAngle,\tkz@SecondAngle)
 \pgfmathsubtract{\tkz@FirstAngle}{\tkz@delta}
 \edef\tkz@FirstAngle{\pgfmathresult}%
 \pgfmathadd{\tkz@SecondAngle}{\tkz@delta}
 \edef\tkz@SecondAngle{\pgfmathresult} 
\begin{scope}[decoration={markings,mark=at position \tkz@mkarcpos with {\tkz@mymarkarc}}]     
\path[shift = {(#2)},\tkz@mkcolor,/@tkzmarkarc/.cd,#1,postaction={decorate}]%
(\tkz@FirstAngle:\tkz@radius pt) arc (\tkz@FirstAngle:\tkz@SecondAngle:\tkz@radius pt);
\end{scope}
\endgroup 
}

\def\tkzLabelArc{\pgfutil@ifnextchar[{\tkz@LabelArc}{\tkz@LabelArc[]}}
\def\tkz@LabelArc[#1](#2,#3,#4)#5{%
\begingroup
\pgfqkeys{/@tkzmarkarc}{#1}
\tkz@@CalcLength(#2,#3){tkz@radius}
\tkz@@CalcLength(#2,#3){tkz@radius}
\tkzFindSlopeAngle(#2,#3)\tkzGetAngle{tkz@FirstAngle}  
\tkzFindSlopeAngle(#2,#4)\tkzGetAngle{tkz@SecondAngle}
\pgfmathsubtract{\tkz@FirstAngle}{\tkz@delta}
\edef\tkz@FirstAngle{\pgfmathresult}%
\pgfmathadd{\tkz@SecondAngle}{\tkz@delta}
\edef\tkz@SecondAngle{\pgfmathresult} 
\begin{scope}[decoration={markings,mark=at position \tkz@mkarcpos with \node{#5};}]
 \path[shift = {(#2)},/@tkzmarkarc/.cd,#1,postaction={decorate}]%
  (\tkz@FirstAngle:\tkz@radius pt) arc (\tkz@FirstAngle:\tkz@SecondAngle:\tkz@radius pt); \end{scope}
\endgroup 
}
\makeatother

\endinput