%% The document of unimath-plain-xetex
%% ******************************************************
%% * This work may be distributed and/or modified under *
%% * the conditions of the LaTeX Project Public License *
%% *     http://www.latex-project.org/lppl.txt          *
%% * either version 1.3c of this license or any later   *
%% * version.                                           *
%% ******************************************************
\overfullrule0pt
\def\mainfontname{Latin Modern Roman}
\def\sansfontname{Latin Modern Sans}
\def\monofontname{Latin Modern Mono}
\def\mathfontname{Latin Modern Math}
\def\textfontopt{mapping=tex-text,}
\input unimath-plain-xetex

\unicodeprimesoff
\betweenprimeskip=-1mu\relax

\def\xetex{X\kern-.125em\lower.5ex\hbox{\char"018E}\kern-.1667em%
  T\kern-.1667em\lower.5ex\hbox{E}\kern-.125emX}
\let\XeTeX\xetex
\let\tex\TeX
\def\latex{L\kern-.24em{\setbox0=\hbox{T}\vbox to \ht0{\hbox{\sevenrm A}\vss}}%
  \kern-.12em\TeX}
\def\umpx{unimath-plain-\xetex}
\def\biggskip{\vskip24pt plus 8pt minus 4pt}
\def\pkg#1{{\tensf #1}}
\font\hugesans="\sansfontname" at 35pt
\font\Hugesansbf="\sansfontname/B" at 72pt
\font\sevenlmmath="Latin Modern Math:script=math,+ssty=0" at 7pt
\makefontcmdcompatible{ten}

\font\manfnt="manfnt" at 10pt
\def\dbend{{\manfnt\char127}}
\def\ddanger{\noindent\hangindent=32pt\hangafter=-2
  \hbox to 0pt{\hss\dbend\kern1pt\dbend\kern3pt}}
\def\ttverbatim{\begingroup
  \catcode`\\=12 \catcode`\{=12 \catcode`\}=12 \catcode`\$=12
  \catcode`\&=12 \catcode`\#=12 \catcode`\%=12 \catcode`\~=12
  \catcode`\_=12 \catcode`\^=12 \catcode`\ =12 \obeylines \tentt}
\outer\def\begintt{$$\let\par=\endgraf \ttverbatim \parskip=0pt
  \catcode`\|=0 \rightskip-2pc \ttfinish}
{\catcode`\|=0 |catcode`|\=12 % | is temporary escape character
  |obeylines % end of line is active
  |gdef|ttfinish#1^^M#2\endtt{#1|vbox{#2}|endgroup$$}}
\def\charhex{\char"}
\catcode`\"=\active
{\obeylines \gdef"{\ttverbatim \spaceskip.5em \let^^M=\  \let"=\endgroup}}
\def\<#1>{$\langle${\it#1\/}$\rangle$}

\long\def\scalehbox#1#2#3{%
  \leavevmode
  \setbox0\hbox{{#3}}%
  \setbox1\hbox{%
    \special{pdf:btrans}%
    \special{x:scale #1 #2}%
    \hbox to 0pt{\copy0\hss}%
    \special{pdf:etrans}%
  }%
  \ht1#2\ht0
  \dp1#2\dp0
  \hbox to#1\wd0{\box1\kern#1\wd0\hss}%
}
\long\def\colorspec#1#2#3#4{\special{color push rgb #1 #2 #3}#4
  \special{color pop}}
\def\green{\colorspec{0.08}{0.52}{0.08}}

\newdimen\paperwd
\newdimen\paperht
\newdimen\hmarginwd
\newdimen\vmarginht
\hsize=12cm
\vsize=16cm
\hmarginwd=3cm
\vmarginht=3cm
\hoffset=\dimexpr\hmarginwd-1in\relax
\voffset=\dimexpr\vmarginht-1in\relax
\paperwd=\dimexpr2\hmarginwd+\hsize\relax
\paperht=\dimexpr2\vmarginht+\vsize\relax
\special{papersize=\the\paperwd,\the\paperht}
\headline={\ifnum\pageno=1\hss\else\hss\vbox to 0pt{\vss\hbox to 0pt{\kern28pt%
  \colorspec{0.72}{0.88}{0.72}{\Hugesansbf\folio}\hss}\kern4pt}\fi}
\footline={\hss}
\long\def\protectedwrite#1#2{\edef\wrt{\write#1{#2}}\wrt}

\newcount\secno
\secno=0
\newwrite\tocout
\edef\tocname{\jobname.toc}
\openout\tocout=\tocname
\newread\tocin
\openin\tocin=\tocname
\long\outer\def\section#1{\par\bigskip\penalty-1000%
  \noindent \advance\secno by 1\relax
  {\green{%
    \twelvebf\hbox to 0pt{\hss\twentybf\the\secno\kern12pt}#1%
  }}%
  \protectedwrite\tocout{\the\secno\quad 
    #1\hskip.5em\leaders\hbox to 1.2em{.\hss}\hfill}%
  \write\tocout{\hbox to .5em{\hss\the\pageno}\par}%
  \par\medskip\penalty1000\relax}
\parindent=2pc
\parskip=3pt plus 2pt minus 1pt\relax

% title
\centerline{\twentyrm\green{The
  \scalehbox{0.5}{0.5}{\special{pdf:code q 1 Tr}%
    {\hugesans\umpx}%
  \special{pdf:code Q}} package}}
\biggskip

\centerline{Zhang Tingxuan}
\medskip

\centerline{2024/10/08\quad Version 0.2c}
\biggskip

\begingroup\parskip0pt
\centerline{\bf Abstract}
\smallskip
\leftskip=4pc\rightskip=4pc\parindent=1.5pc\relax
The {\sf\umpx} package provides OpenType math font support in
{\it plain \tex\/} format. The {\sf\umpx} package needs \xetex.\par

% TOC
\bigskip
\centerline{\bf Contents}
\smallskip
\parindent=0pt\relax
\ifeof\tocin\else\closein\tocin\input\tocname\fi
\endgroup\bigskip


\section{How to use this package?}
Please notice again that you're using {\it plain\/} format but not 
\latex\ format. If you are using \latex\ format, please use 
\pkg{unicode-math} package instead.

In your document, write
\begintt
\input unimath-plain-xetex
\endtt
Then compile your document with "xetex", you can get OpenType math support 
in your document. The package will set the math font ``Latin Modern Math'' 
with ``Latin Modern'' text fonts in default. To change the font, you can 
define some names before loading the package. For example,
\begintt
\def\mainfontname{TeX Gyre Termes}
\def\sansfontname{TeX Gyre Heros}
\def\monofontname{TeX Gyre Cursors}
\def\mathfontname{TeX Gyre Termes Math}
\input unimath-plain-xetex
\endtt
Your text fonts will be set in the first 3 lines and your math font will 
be set in the fourth line.

Currently, the package supports only font family names to use, if 
you want to use the file names, you can revise the code in 
"unimath-plain-xetex.tex".



\section{Text font commands}
The package provides text font commands in the format of 
$$\hbox{"\"\<pt-size>\<family>\<series>\<shape>}$$
such as "\tensfbfit", "\twelvebf", etc. The packages provides 
font commands in size of 5\,pt, 7\,pt, 9\,pt, 10\,pt, 12\,pt and 20\,pt, 
such as "\fiverm", "\sevensf", "\twelvett", etc.

Take ten point as an example,
$$\begingroup\let\tentt\seventt%
  \vbox{\openup-1pt\halign{#\hfil&\ \ #\hfil&\ \ #\hfil\cr 
    \green{\sevenrm(Serif)} & {\sevenrm Upright} & {\sevenit Italic} \cr
    {\sevenrm Medium}& "\tenrm"    & "\tenit"     \cr
    {\sevenbf Bold}  & "\tenbf"    & "\tenbfit"   \cr}}\ \vrule\ \ 
  \vbox{\openup-1pt\halign{#\hfil&\ \ #\hfil&\ \ #\hfil\cr 
    \green{\sevensf(Sans)}  & {\sevensf Upright}  & {\sevensfit Italic} \cr
    {\sevensf Medium} & "\tensf"   & "\tensfit"   \cr
    {\sevensfbf Bold} & "\tensfbf" & "\tensfbfit" \cr}}\ \vrule\ \ 
  \vbox{\openup-1pt\halign{#\hfil&\ \ #\hfil&\ \ #\hfil\cr 
    \green{\seventt(Mono)}  & {\seventt Upright}  & {\seventtit Italic} \cr
    {\seventt Medium} & "\tentt"   & "\tenttit"   \cr
    {\seventtbf Bold} & "\tenttbf" & "\tenttbfit" \cr}}
\endgroup$$
The font commands can be used as those provided in "plain.tex", for example,
"{\tenbfit ABC}" yields {\tenbfit ABC}.

But if you would't like to remember that many commands, you can write
$$\hbox{\tentt"\makefontcmdcompatible"\{\<pt-size>\}}$$
after loading the package. For example, if you write
\begintt
\makefontcmdcompatible{ten}
\endtt
the \<family>-\<series>-\<shape> order of "\ten"\<some> commands 
can be write randomly: writing "\tenbfsfit" is the same as "\tensfbfit".

You can also get more text font commands through the "\genfontcmd" command:
$$\hbox{\tentt"\genfontcmd"\{\<pt-size>\}\{\<dimension>\}}$$
For example, 
\begintt
\genfontcmd{fortyfour}{44pt}
\genfontcmd{verytiny}{2bp}
\endtt
will make commands like "\fortyfourrm" and "\verytinysfbfit" available.



\section{Math font commands}
You can input math formulae just like using traditional plain \tex. But 
OpenType math font is loaded. For example, "$a{\mbf0}={\mbf0}$" yields 
$a{\mbf0}={\mbf0}$. Available math font commands are listed below:
\begintt
\mrm, \mbf, \mit, \msf, \mtt, 
\mbfit, \msfbf, \msfit, \msfbfit,
\cal, \calbf, \bb, \bbit, \frak, \frakbf
\endtt
The \<family>-\<series>-\<shape> order of these commands allows being random; 
"cal" can be replaced by "scr".

\ddanger{\bf Please notice,} that the commands in the first and second 
line of the chart above contain an extra ``"m"'' in the beginning of commands. 
For example, the first ``"m"'' in the ``"\mrm"'' command in the first line. 
If you didn't write the ``"m"'' in these commands, the font selected by the 
commands would be no more math font, but text font.

The ``math font'' we said here means the font selected by "\mathfontname", 
"\mathalphafontname" and so on; the ``text font'' means the font selected by 
"\mainfontname", "\sansfontname" and "\monofontname". For example,
$$\vbox{\openup3pt\halign{#\hfil&\qquad#\hfil\cr
{\tenit Input}&{\tenit yields}\cr
"$\mit   abcdefg \mbfit   hijklmn$" & 
  $\mit   abcdefg \mbfit   hijklmn$\cr
"$\it    abcdefg \bfit    hijklmn$" & 
  $\it    abcdefg \bfit    hijklmn$\cr
"$\msfit abcdefg \msfbfit hijklmn$" & 
  $\msfit abcdefg \msfbfit hijklmn$\cr
"$\sfit  abcdefg \sfbfit  hijklmn$" & 
  $\sfit  abcdefg \sfbfit  hijklmn$\cr}}$$
A group of symbols in math font can be regarded as separate symbols, and
a group of symbols in text font can be regarded as a whole.

This package uses "unicode-math-symbols.tex" to generate math symbol commands, 
the source file can be found in \pkg{unicode-math} (\latex) package. To find 
all of the math symbol commands, you can execute
\begintt
texdoc unimath-symbols
\endtt
in Terminal.

You can also input Unicode math characters in your document's source file. 
For example, 
$$\vbox{\openup3pt\halign{#\hfil&\qquad#\hfil\cr
{\tenit Input}&{\tenit yields}\cr
"$ ∑_{i=0}^∞ ∫_a^b ρ_i \, {\rm d} τ' $" & 
  $∑_{i=0}^∞ ∫_a^b ρ_i\,{\rm d}τ'$\cr
"$$∑_{i=0}^∞ ∫_a^b ρ_i \, {\rm d} τ'$$" & 
  $\displaystyle ∑_{i=0}^∞ ∫_a^b ρ_i\,{\rm d}τ'$\cr}}$$



\section{Primes ($'$)}
The package changed ``"'"'' command in math mode, which yields prime(s) 
in superscript. After loading this package, the primes produced by ``"'"'' 
will be turned into Unicode primes:
$$\vbox{\openup2pt\halign{#\hfil\quad&\quad#\hfil\quad&\quad#\hfil\cr
{\tenit Input} & {\tenit yields} & {\tenit Unicode slot}\cr
"$'$"     & $^{\hbox{\sevenlmmath\charhex2032}}$ & U+2032 \cr
"$''$"    & $^{\hbox{\sevenlmmath\charhex2033}}$ & U+2033 \cr
"$'''$"   & $^{\hbox{\sevenlmmath\charhex2034}}$ & U+2034 \cr
"$''''$"  & $^{\hbox{\sevenlmmath\charhex2057}}$ & U+2057 \cr
"$'''''$" & $^{\hbox{\sevenlmmath\charhex2032
  \hskip-.15em\charhex2032
  \hskip-.15em\charhex2032
  \hskip-.15em\charhex2032
  \hskip-.15em\charhex2032}}$
  & $(\hbox{U+2032})\times5$\cr
$\vdots$ & $\vdots$ & $\vdots$ \cr
$(\hbox{"'"})\times N$ & $(^{\hbox{\sevenlmmath\charhex2032}})\times N$ &
$(\hbox{U+2032})\times N$\hbox to 20pt{\quad$(N>4)$\hss} \cr
}}$$
However, some OpenType math fonts don't contain some of the characters above. 
For example, there is no U+2032, U+2033 or U+2034 in Erewhon Math. 
When using such fonts, you can ``turn off'' the Unicode primes easily 
by inputing the following line {\it after} loading the package:
\begintt
\unicodeprimesoff
\endtt
Then when you input ``"'"'' $n$ times you will get $n$ primes, 
and each prime's Unicode character slot is U+2032, even when $n\leq 4$.

If the primes contain more than one single prime encoded U+2032, 
a negative math skip will inserted between every two single primes. 
This math skip is defined as "\betweenprimeskip" and its default value 
is "-2.7mu", which is proper for Latin Modern Math. To change the value 
of "\betweenprimeskip" is just like change any math skip, for example,
\begintt
\betweenprimeskip=-1mu
\endtt
This line should also be written {\it after} loading the package.

\section{Using multiple math fonts}
You can use more than one OpenType math fonts in math mode. 
This is the method to set multiple math fonts:
$$\vbox{%
  \hbox{"\def\mathalphafontname{"\<font family name>"}"}%
  \hbox{"\def\mathdelimiterfontname{"\<font family name>"}"}%
  \hbox{"\def\mathordfontname{"\<font family name>"}"}%
  \hbox{"\def\mathopfontname{"\<font family name>"}"}%
  \hbox{"\def\mathbinfontname{"\<font family name>"}"}%
  \hbox{"\def\mathaccentfontname{"\<font family name>"}"}%
}$$
Such definitions should be written {\tenit before} "\input"ing the package. 
For example, 
\begintt
\def\mathfontname{XITS Math}
\def\mathalphafontname{TeX Gyre Pagella Math}
\input unimath-plain-xetex
\endtt
Then your math font will be set as XITS Math and the font of variable family 
(numbers, Latin and Greek letters) will be set as \TeX\ Gyre Pagella Math.

"\mathalphafontname" will influence the font of numbers ("0"--"9"), Latin 
letters ("A"--"Z", "a"--"z") and Greek letters ("\alpha" and so on). 

"\mathdelimiterfontname" will influence the font of math delimiters, 
for example, extensible open symbols, close symbols and ordinary symbols 
like "\{", "\}", "\vert" and so on. 
Horizontal delimiters (or very long accents) are also influenced by 
"\mathdelimiterfontname".

"\mathordfontname" will influence the font of non-delimiter ordinary 
symbols, punctuations and ``"!"''. For example, ``","'', ``":"'' and 
"\colon"'s font will be influenced by it.

"\mathopfontname" will influence the font of large operators like "\sum".

"\mathbinfontname" will influence the font of binary operators and relations 
like ``"+"'' and ``"="''.

"\mathaccentfontname" will influence the font of math accents like "\dot".



\closeout\tocout
\bye
