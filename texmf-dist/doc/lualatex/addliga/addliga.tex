% !TEX TS-program = lualatex
\documentclass{scrartcl}
\setkomafont{disposition}{\bfseries}
\KOMAoptions{fontsize=12pt}
\usepackage[letterpaper]{geometry}
\usepackage{addliga}
\usepackage{fontspec}
\setmainfont[Numbers={OldStyle}]{Domitian}
\setmonofont[Scale=1.05]{CMU Typewriter Text Variable Width}
\usepackage[protrusion=true,expansion=true]{microtype}
\usepackage{realscripts}
\usepackage[main=english,,latin.classical,nil]{babel}
\newcommand{\laterm}[1]{\foreignlanguage{latin}{\textit{#1}}}
\deffootnote[1.5em]{1.5em}{1em}{\makebox[1.5em][l]{\thefootnotemark.}}
\usepackage[pdfusetitle,pdfa,hidelinks]{hyperref}\hypersetup{pdfencoding=unicode}
\usepackage{hologo}
\title{\normalfont\texttt{addliga}\\\large (version 1.0)}
%\date{}
\author{Daniel~Benjamin Miller\\\large(\texttt{dbmiller@dbmiller.org})}
\begin{document}
\maketitle
The \texttt{fontspec}\footnote{\url{https://ctan.org/pkg/fontspec}} package enables use of OpenType and TrueType fonts in \hologo{LuaLaTeX} and \hologo{XeLaTeX}.  Though legacy TrueType fonts often include at least a partial set of ligatures, these, unlike those of OpenType fonts, are not accessible through \texttt{fontspec}'s usual mechanisms. The \texttt{addliga} package is designed to allow easy use of basic TrueType font ligatures in \hologo{LuaLaTeX} by adding an OpenType \texttt{liga} lookup table to the loaded font. (\laterm{Nota bene}: not all fonts will contain ligature support; \texttt{addliga} only adds support for existing ligatures.) The ligatures supported are ff, fi, ffi, fl and ffl. This package has been contributed to the public domain.
\section*{Basic Usage}
\begin{verbatim}
\documentclass{…}
\usepackage{addliga}
\usepackage{fontspec}
\setmainfont{…}
\begin{document}
…
\end{document}
\end{verbatim}
\end{document}