% !TEX TS-program = LuaLaTeX
\documentclass[french,PubliClass=book,ColorTheme=Steel,FontSize=11pt,
StylisticSet={parallelslant,leqslant}]{tango}
\renewcommand{\arraystretch}{1.3}
\usepackage{tikz}
\usetikzlibrary{calc,through,arrows}
\tgosmartlists
\AtBeginDocument{%
\RenewCommandCopy{\setminus}{\smallsetminus}
}
\NewDocumentCommand\N{}{\ensuremath{\symbb{N}}}
\NewDocumentCommand\Z{}{\ensuremath{\symbb{Z}}}
\NewDocumentCommand\Q{}{\ensuremath{\symbb{Q}}}
\NewDocumentCommand\R{}{\ensuremath{\symbb{R}}}
\NewDocumentCommand\C{}{\ensuremath{\symbb{C}}}
\NewDocumentCommand\Pt{m}{\ensuremath{\symfrak{P}(#1)}}
\NewDocumentCommand\rond{}{\mathrel{\circ}}
\NewDocumentCommand\mesalg{m}{\overline{#1}}
\NewDocumentCommand\Vector{m}{\overrightarrow{#1}}
\DeclareMathOperator{\cl}{cl}
\DeclareMathOperator{\sgn}{sgn}
\newcounter{prop}
\newstatement{prop}{prop}{Propriété}
\newfixedcaption{\figcaption}{figure}
%%
\includeonly{rel-bin,recurrence,polynome-deg2,euler,puissance-cercle}
%\includeonly{polynome-deg2}
%%
%\tangoSmartLists
\begin{document}
\frontmatter
\tableofcontents
\mainmatter
%\tgosetup{ColorTheme=Framboise}
\part{Algèbre}
% !TEX TS-program = LuaLaTeX
%\documentclass[ColorTheme=Red]{dino-iPhone}

\tgotitle{Relations binaires}
\tgoshorttoc
\section*{Avertissement}
{\itshape
Ce chapitre de la \emph{Mathematica dinosaurorum} a un statut un peu particulier, dans la mesure où il regroupe des notions somme toute assez basiques et que je ne tiens pas à présenter en introduction à d'autres chapitres. Il s'agit donc d'un texte destiné à servir de référence, et pas nécessairement à être lu d'une traite. Les exemples y sont donc peu nombreux, de même que les théorèmes, car tout ceci figure ou figurera dans les chapitres qui font référence à celui-ci.

}

\section{Généralités}
\subsection{Produit cartésien}
Deux ensembles $E$ et $F$ étant donnés, on appelle produit cartésien de $E$ et $F$ l'ensemble noté~$E\times F$ des couples $(x,y)$ où $x\in E$ et $y\in F$ :
\[z\in E\times F \iff \exists\, x \in E, \exists\, y\in F,\;\text{tels que}\; z=(x,y).\]

Dans le cas où $E=F$, on peut noter $E^2$ au lieu de $E\times E$. Ce cas particulier est important : par exemple, l'ensemble des couples des coordonnées des points du plan rapporté à un repère donné est~$\mathbb{R}^2$. On comprend alors que l'ordre de chaque couple est important et qu'il faut distinguer $(x,y)$ et~$(y,x)$, dans le cas de l'exemple qui vient d'être donné, mais aussi d'une façon générale.

Cette notion se généralise à plus de deux ensembles : par exemple~$\R^3$ est l'ensemble~\mbox{$\R\times\R\times\R$} des triplets ordonnés de réels.

\subsection{Parties d'un ensemble}
On dit qu'un ensemble $X$ est une partie d'un ensemble $E$ si tout élément de $X$ est un élément de $E$. On dit aussi que~$X$ est un sous-ensemble de $E$ ou que $X$ est inclus dans $E$, ce que l'on note $X \subset E$ :
\[X\subset E \iff \forall x,\; (x\in X\implies x\in E).\]

L'ensemble vide $\emptyset$ et l'ensemble $E$ lui-même sont des parties de $E$, et l'on note~$\Pt{E}$ l'ensemble de toutes les parties de $E$.
\[X\in \Pt{E}\iff X\subset E.\]

\begin{alert}
La propriété ci-dessus exprime le fait qu'une inclusion dans $E$ se traduit par une appartenance à $\Pt{E}$. Si $A$ et $B$ sont des parties de $E$, ce sont des \emph{éléments} de $\Pt{E}$, de même que~$\emptyset$ et~$E$. En revanche $\{A\}$, qui est un ensemble admettant pour seul élément la partie $A$ de $E$, est une partie de~$\Pt{E}$ de même que $\{\emptyset\}$ et $\{\emptyset, A, B,E\}$. En tant que parties de $\Pt{E}$, ce sont aussi des éléments de $\Pt{\Pt{E}}$! On notera aussi que l'ensemble vide~$\emptyset$, qui ne contient aucun élément, ne doit pas être confondu avec $\{\emptyset\}$, qui est un ensemble qui contient l'ensemble vide pour seul élément : ces deux ensembles sont des parties de $\Pt{E}$, distinctes l'une de l'autre. Enfin, il apparaît que $\emptyset$ est à la fois un élément et une partie de $\Pt{E}$…
\end{alert}

\begin{example}[Exemples]
\begin{enumerate}
\item Si $E=\{a,b\}$, $\Pt{E}=\{\emptyset,\{a\},\{b\},E\}$.
\item Si $E=\emptyset$, $\Pt{E}=\{\emptyset\}$, $\Pt{\Pt{E}}=\{\emptyset,\{\emptyset\}\}$ et \[\Pt{\Pt{\Pt{E}}}=\{\emptyset,\{\emptyset\},\{\{\emptyset\}\},\{\emptyset,\{\emptyset\}\}\}.\]
\end{enumerate}
\end{example}

\subsubsection{Opérations dans l'ensemble des parties}
On définit l'union (notée $\cup$) et l'intersection (notée $\cap$) de deux parties de $E$ de la façon suivante :

\smallskip\par
\begin{itemize}
\item\textit{Union :} $x\in A\cup B \iff ((x\in A)\;\text{ou}\;(x\in B))$.
\item\textit{Intersection :} $x\in A\cap B \iff ((x\in A)\;\text{et}\;(x\in B))$.
\end{itemize}
\smallskip\par

À propos de l'union, on notera qu'en mathématique le mot \frquote{ou} a \emph{toujours} un sens inclusif, c'est-à-dire \frquote{l'un ou l'autre ou les deux}. L'union et l'intersection de deux parties de $E$ sont elles-mêmes des parties de $E$.

De même on définit le complémentaire d'une partie $A$ de $E$ comme étant la partie de $E$ constituée des éléments de $E$ qui n'appartiennent pas à~$A$. Ce complémentaire est noté $E\smallsetminus A$ ou~$\complement_{E} A$. Lorsque qu'il n'y a aucune ambiguïté concernant l'ensemble $E$, on peut même le noter~$\bar{A}$.
\smallskip\par
\begin{itemize}
\item\textit{Complémentaire :} $x\in \complement_E A \iff ((x\in E)\;\text{et}\;(x\notin A))$.
\end{itemize}

\smallskip
Enfin on définit également la \emph{différence ensembliste} de deux parties $A$ et $B$ de $E$, notée~$A\smallsetminus B$ comme étant l'ensemble des éléments de $A$ qui ne sont pas éléments de $B$.
\smallskip\par
\begin{itemize}
\item\textit{Différence ensembliste :} $x\in A\smallsetminus B \iff ((x\in A)\;\text{et}\;(x\notin B))$.
\end{itemize}

\subsubsection{Partition d'un ensemble}
Soit un ensemble $E$  et $\symcal{C}$ une ensemble de parties de $E$ (c'est-à-dire un sous-ensemble ou encore une partie de $\Pt{E}$), on dit que $\symcal{C}$ est une partition de $E$ si les deux conditions suivantes sont vérifiées :
\begin{enumerate}
\item Tout élément de $\symcal{C}$ est non vide.
\item Tout élément de $E$ appartient à  un élément de $\symcal{C}$ et un seul.\label{cond2}
\end{enumerate}

\begin{remark}On notera que la condition~\ref{cond2} ci-dessus peut s'expliciter en disant que les parties qui constituent $\symcal{C}$ sont deux à deux disjointes (c'est-à-dire d'intersection vide) et que leur réunion est égale à$E$.
\end{remark}

\subsection{Relation binaire entre deux ensembles}\label{relbin}
Une \emph{relation binaire} $\symcal{R}$ entre deux ensembles $E$ et $F$ est définie par la donnée d'une partie~$\symcal{G}$ du produit cartésien~$E\times F$ (on a donc $\symcal{G}\in\Pt{E\times F}$). Cette partie, qui est donc un ensemble de couples, est le \emph{graphe} de la relation : par définition, un couple~$(x,y)$ de~$E\times F$ est lié par la relation~$\symcal{R}$ si~$(x,y)\in \symcal{G}$. Donner une relation binaire revient donc à donner son graphe. Si la relation est notée $\symcal{R}$, on écrit $x\mathrel{\symcal{R}}y$ ou $\symcal{R}(x,y)$ pour exprimer que~$x$ et~$y$, dans cet ordre, sont liés par la relation $\symcal{R}$. On a donc pour une relation~$\symcal{R}$ dont le graphe est $\symcal{G}$:
\[\forall (x,y)\in E\times F, \; x\mathrel{\symcal{R}}y\iff(x,y)\in \symcal{G}.\]

\begin{example}[Exemples]
\begin{enumerate}
\item L'égalité est une relation binaire qui peut être définie sur tout ensemble $E$. Son graphe est ce que l'on nomme la \emph{diagonale} de $E$, c'est à dire l'ensemble des couples $(x,y)$ de~$E^2$ tels que~$x=y$.

%\item L'ordre strict dans l'ensemble des réels est une relation binaire notée~$<$ (on aurait également pu considérer un ordre large ou l'ordre inverse). Par définition, $x<y$ traduit le fait que $y-x$ est un réel strictement positif :
%\[\forall (x,y) \in \R^2, \; x<y \iff (y-x)\in \R^{*}_+.\]
%Le graphe de cette relation peut être représenté par l'ensemble des points du plan repéré dont l'ordonnée est strictement supérieure  à l'abscisse, c'est à dire les points situés strictement au dessus de la droite d'équation $y=x$ (première bissectrice du repère).
%\label{ord-strict}
%On notera que la \emph{règle des signes}, reposant sur la \emph{partition} de $\R$ en trois ensembles qui sont $\R^*_-$, le singleton $\{0\}$ et $\R^*_+$, préexiste à la définition d'un ordre dans $\R$.

\item Le relation $p$ définie entre $\R$ et $\R_+$ par 
\[\forall (x,y)\in \R\times \R_+,\; p(x,y)\iff y=x^2.\]
Le graphe de $p$ peut être représenté, dans le demi-plan repéré des points d'ordonnée positive, par la parabole d'équation~$y=x^2$. On notera que cette relation est en réalité une fonction et que la notation $y=p(x)$ aurait été plus habituelle; nous y reviendrons. 
\end{enumerate}
\end{example}

% Dans le cas d'une
%relation fonctionnelle (voir  \S~\ref{Rfonc} ci-dessous), pour laquelle un seul élément y de $F$ est en relation avec un élément $x$ donné dans $E$, on écrit en général $y=\symcal{R}(x)$. 
%

\section{Relations d'équivalence}
\subsection{Définition}
On dit qu'une relation $\symcal{R}$, définie sur $E$ (on se place donc dans le cas où $E=F$) est une relation d'équivalence si elle possède les trois propriétés suivantes :
\begin{enumerate}
\item Elle est \emph{réflexive}, c'est-à-dire que tout élément de $E$ est en relation avec lui même :
\[\forall x \in E,\; x\mathrel{\symcal{R}}x.\]
\item Elle est \emph{symétrique}:
\[\forall (x,y)\in E\times E,\; (x\mathrel{\symcal{R}}y \implies y\mathrel{\symcal{R}}x).\]
\item Elle est \emph{transitive} :
%\[
%\forall (x,y,z)\in E^3,\quad \left.\begin{aligned}
%x&\mathrel{\symcal{R}}y\\
%&\text{et}\\
%y&\mathrel{\symcal{R}}z
%\end{aligned}\right\}
%\implies x\mathrel{\symcal{R}}z
%\]
\[
\forall (x,y,z)\in E^3,\; \bigl((x\mathrel{\symcal{R}}y\;\text{ et }\;y\mathrel{\symcal{R}}z)\implies x\mathrel{\symcal{R}}z\bigr).
\]
\end{enumerate}

\begin{example}[Exemples]
\begin{enumerate}
\item L'égalité est une relation d'équivalence sur tout ensemble, c'est une conséquence directe de la définition.
\item Si l'on désigne par $\symfrak{D}$ l'ensemble des droites du plan, le parallélisme est une relation d'équivalence sur $\symfrak{D}$ : c'est immédiat également,  on peut noter au passage que la transitivité traduit le fait que \frquote{deux droites parallèles à une même troisième sont parallèles entre elles}.
\item Pour un ensemble $E$ donné, la relation dont le graphe est $E\times E$ est une relation d'équivalence. Tous les couples d'éléments de $E$ sont donc liés par cette relation.
\item La relation $\equiv$ définie dans l'ensemble $\Z$ des entiers par
\[\forall (m,n)\in\Z\times\Z,\; m\equiv n \iff (m-n) \text{ est un entier pair.}\]
est une relation d'équivalence.

Rappelons qu'un élément $p$ de $\Z$ est un entier pair s'il est divisible par~$2$, c'est-à-dire s'il existe $k$ appartenant à $\Z$ tel que~$p=2k$ et montrons que la relation~$\equiv$ remplit bien les conditions requises :
\begin{enumerate}
\item \emph{Réflexivité. — }Soit $m\in\Z$ un entier quelconque : $m-m=0=2\times0$. Donc~$m-m$ est pair. La relation $\equiv$ est réflexive.
\item \emph{Symétrie. — }Soit $m$ et $n$ deux entiers tels que $m\equiv n$. Alors il existe un entier~$k$ tel que~$m-n=2k$. Il suffit alors d'écrire $n-m=2\times(-k)$ pour constater que $n\equiv m$. La relation~$\equiv$ est symétrique.
\item\emph{Transitivité. — }Soit $(m,n,p)$ dans $\Z^3$ tels que $m\equiv n$ et $n\equiv p$. Il existe alors deux entiers~$k$ et~$k'$ tels que \mbox{$m-n=2k$} et $n-p=2k'$. Il suffit de remarquer que 
\[m-p=(m-n)+(n-p)=2(k+k'), \text{ où $(k+k')\in\Z$}\] 
pour établir que $m\equiv p$. La relation $\equiv$ est donc transitive.
\end{enumerate}
\end{enumerate}
\end{example}
%\subsection{Exemples}

\subsection{Classes d'équivalence}
Soit un ensemble $E$ et une relation d'équivalence $\symcal{R}$ sur$E$. On dit qu'une partie~$A$ de~$E$ est une classe d'équivalence modulo la relation~$\symcal{R}$ s'il existe un élément $x$ de~$E$ tel que~$A$ soit l'ensemble des éléments~$y$ de~$E$ pour lesquels la relation~$x\mathrel{\symcal{R}}y$ est vérifiée. On note alors~$A=\cl_{\symcal{R}} (x)$ ou~$A=\cl (x)$ s'il n'y a pas d'ambiguité.
\[y\in\cl_{\symcal{R}} (x)\iff x\mathrel{\symcal{R}}y.\]

\begin{thm}
Soit un ensemble $E$ et une relation d'équivalence $\symcal{R}$ sur $E$. L'ensemble des classes d'équivalence modulo $\symcal{R}$ constitue une partition de $E$.
\end{thm}

\begin{proof}
Tout d'abord la réflexivité de la relation $\symcal{R}$ entraîne que pour tout~$x$ appartenant à~$E$,~$x$ appartient à~$\cl(x)$. Par conséquent, tout $x\in E$ appartient à une classe et toute classe est non vide. Il nous reste à prouver que tout élément $x$ de $E$ appartient à une seule classe, c'est à dire que s'il existe un élément $y$ de $E$ tel que $x\in \cl(y)$, alors $\cl(x)=\cl(y)$.

Nous allons procéder par double inclusion.
\begin{itemize}
\item Soit donc $y\in E$ tel que $x\in\cl(y)$. On sait que $y\mathrel{\symcal{R}}x$, donc,
pour tout~$z\in \cl(x)$, on a à la fois $y\mathrel{\symcal{R}}x$ et $x\mathrel{\symcal{R}}z$ : par transitivité on obtient~$y\mathrel{\symcal{R}}z$, soit~$z\in\cl(y)$ :
\[\forall z\in E, (z\in\cl(x)\implies z\in\cl(y), \text{ donc }\cl (x)\subset\cl (y).\]
 \item Par symétrie de la relation d'équivalence $\symcal{R}$, on a aussi~$x\mathrel{\symcal{R}}y$. On peut alors refaire le même raisonnement que ci-dessus : 
 pour tout $z$ appartenant à $\cl(y)$ on a  à la fois~$x\mathrel{\symcal{R}}y$ et~$y\mathrel{\symcal{R}}z$, 
 donc par transitivité $x\mathrel{\symcal{R}}z$, soit~\mbox{$z\in\cl(x)$} :
\[\forall z\in E, (z\in\cl(y))\implies z\in\cl(x), \text{ donc }\cl (y)\subset\cl (x).\]
\item Finalement $\cl(x)=\cl(y)$.\qedhere
\end{itemize}
\end{proof}

\begin{remark}
Réciproquement, toute partition $\symcal{C}$ d'un ensemble $E$ est l'ensemble des classes d'une relation d'équivalence. Il suffit en effet de considérer la relation $\sim$ telle que pour tout~$(x,y)$ dans $E^2$, on ait \textit{$x\sim y$ si $x$  appartient au même élément de $\symcal{C}$ que~$y$}, laquelle répond à la question.
\end{remark}
%\end{document}
%\subsubsection{Notion de structure quotient}

\section{Relations d'ordre}
\subsection{Définition}
On dit qu'une relation $\preccurlyeq$, définie sur $E$ (on se place donc ici encore dans le cas où~$E=F$) est une relation d'ordre si elle possède les trois propriétés suivantes :
\begin{enumerate}
\item Elle est \emph{réflexive}, c'est-à-dire que tout élément de $E$ est en relation avec lui même :
\[\forall x \in E,\; x\preccurlyeq x.\]
\item Elle est \emph{antisymétrique}:
\[\forall (x,y)\in E\times E,\; \bigl((x\preccurlyeq y)\text{ et }(y\preccurlyeq x)\implies x=y\bigr).\]
\item Elle est \emph{transitive} :
%\[
%\forall (x,y,z)\in E^3,\quad \left.\begin{aligned}
%x&\mathrel{\symcal{R}}y\\
%&\text{et}\\
%y&\mathrel{\symcal{R}}z
%\end{aligned}\right\}
%\implies x\mathrel{\symcal{R}}z
%\]
\[
\forall (x,y,z)\in E^3,\; \bigl((x\preccurlyeq y\;\text{ et }\;y\preccurlyeq z)\implies x\preccurlyeq z\bigr).
\]
\end{enumerate}

\begin{remark}
La démonstration du résultat suivant est immédiate.

Une relation d'ordre $\preccurlyeq$ sur un ensemble $E$ étant donnée, la relation~$\succcurlyeq$, définie par 
\[\forall (x,y) \in E^2, x\succcurlyeq y \iff y\preccurlyeq x\,,\] 
est une relation d'ordre sur $\R$ nommée \emph{ordre inverse} de $\preccurlyeq$.
\end{remark}

%\begin{exemple}[Exemples]
%\begin{enumerate}
%\item La relation d'ordre strict sur l'ensemble $\R$ des réels, telle que nous l'avons rencontrée dans l'exemple~\ref{ord-strict} du paragraphe~\ref{relbin} n'est pas une relation d'ordre, car elle n'est pas réflexive. En revanche l'ordre large correspondant, défini par 
%\[\forall (x,y) \in \R^2, \; x\leq y \iff (y-x)\in \R_+\,,\]
%est bien une relation d'ordre. En effet :
%\begin{itemize}
%\item Pour tout $x$ réel, $x-x=0$ et $0\in \R_+$, ce qui assure que la relation~$\leq$ est \emph{réflexive}.
%\item Pour tout couple de réels $(x,y)$, si $x\leq y$ et $y\leq x$ alors $x=y$. En effet, en effet, $y\leq x$ signifie que $(x-y)\in \R_+$. Le réel $(y-x)$ se retrouve donc appartenir à la fois à $\R_+$ (car $x\leq y$) et à $\R_-$ (en tant qu'opposé de $(x-y)$) ; on en déduit que $y-x=0$ donc que $x=y$. La relation~$\leq$ est donc \emph{antisymétrique}.
%\item Soit $(x,y,z)\in\R^3$, on a $z-x=(z-y)+(y-x)$. On en déduit que si $x\leq y$ et $y\leq z$, $z-x$ est élément de $\R_+$ et tant que somme de deux éléments de $\R_+$, ce qui revient à dire que $x\leq z$. Le relation~$\leq$ est donc \emph{transitive}.
%\end{itemize}
%\end{enumerate}
%\end{exemple}

\subsection{Ordre dans les ensembles de nombres}
Les relations usuelles $<$ et $>$, définies sur les ensembles de nombres, \emph{ne sont pas} des relations d'ordre, il est facile de l'établir : ces relations \emph{ne sont pas} réflexives. Il est en revanche plus délicat de démontrer que les relations~$\leq$ et~$\geq$ sont des relations d'ordre, et même de tout simplement définir ces relations.
 On pourrait songer à définir l'ordre dans $\R$ en disant que~$x\leq y$ si $(y-x)\in \R_{+}$… Mais alors il resterait à définir $\R_{+}$ sans utiliser l'ordre, ce qui ne semble nullement évident. C'est pour cette raison que nous nous contenterons d'affirmer que la relation $\leq$, telle que chacun la connaît, est bien une relation d'ordre sur $\R$ (ainsi d'ailleurs que sur $\N$, $\Z$, $\symbb{D}$ et $\Q$).
 
% 
\subsection{Un peu de vocabulaire}
On se donne dans ce paragraphe un ensemble $\symcal{E}$ sur lequel est définie une relation d'ordre~$\preccurlyeq$.
\subsubsection{Ordre total}
 On dit que la relation $\preccurlyeq$ définit un ordre total sur $\symcal{E}$ si pour tout couple~$(x,y)$ d'éléments de~$\symcal{E}$, la proposition 
 \XSmartphoneCommand{$\bigl((x\preccurlyeq y)\text{ ou }(y\preccurlyeq x)\bigr)$}
 \SmartphoneCommand{%
 \[
 \bigl((x\preccurlyeq y)\text{ ou }(y\preccurlyeq x)\bigr)
 \]%
 }
  est vraie. 

\subsubsection{Majorants, minorants \& C\up{ie}}
Soit maintenant $\symcal{A}$ une partie de $\symcal{E}$.

\begin{itemize}
\item On dit qu'un élément $M$ de $\symcal{E}$ est un majorant de $\symcal{A}$, si $\forall x\in \symcal{A},\, x\preccurlyeq M$.
\item On dit qu'un élément $m$ de $\symcal{E}$ est un minorant de $\symcal{A}$, si $\forall x\in \symcal{A},\, m\preccurlyeq x$.
\item On dit que $\symcal{A}$ admet $A$ pour plus grand élément si $A\in \symcal{A}$ et si $A$ est un majorant de $\symcal{A}$.
\item On dit que $\symcal{A}$ admet $a$ pour plus petit élément si $a\in \symcal{A}$ et si $a$ est un minorant de~$\symcal{A}$.
\end{itemize}



\begin{remark}
Dans un ensemble muni d'une relation d'ordre, l'existence d'un majorant, d'un minorant, d'un plus petit ou d'un plus grand élément n'est nullement assurée. On a en revanche le résutat suivant.
\end{remark}

\begin{thm}
Soit un ensemble $\symcal{E}$ sur lequel est définie une relation d'ordre $\preccurlyeq$ et soit $\symcal{A}$ une partie de~$\symcal{E}$. Si~$\symcal{A}$ admet une plus grand élément (respectivement un plus petit élément) pour la relation $\preccurlyeq$, ce plus grand élément (respectivement ce plus petit élément) est unique.
\end{thm}
\begin{proof}
Nous proposons la démonstration dans le cas du plus petit élément, celle de l'autre cas est identique. Supposons que~$a$ et~$a'$ soient des plus petits éléments de $\symcal{A}$. D'après la définition d'un plus petit élément, on sait que pour tout~$x$ de~$\symcal{A}$, $a\preccurlyeq x$ et en particulier~\mbox{$a\preccurlyeq a'$}. Et de la même façon, on sait que pour tout $x$ de $\symcal{A}$, $a'\preccurlyeq x$ et en particulier~\mbox{$a'\preccurlyeq a$}. L'antisymétrie de la relation $\preccurlyeq$ nous assure alors que $a=a'$, d'où l'unicité du plus petit élément.
\end{proof}

Supposons maintenant que $\symcal{A}$ admette $A$ pour plus grand élément et soit $\symcal{M}$ l'ensemble de tous les majorants de $\symcal{A}$. Il va de soi que $\symcal{A}\cap\symcal{M}=\{A\}$. L'élément $A$ apparaît ainsi également comme le plus petit élément de $\symcal{M}$. De la même façon, si $\symcal{A}$ admet un plus petit élément, ce dernier est le plus grand des minorants de $\symcal{A}$.

\begin{alert}
Dans le cas où $\symcal{A}$ ne possède pas de plus grand élément, il n'est nullement assuré que~$\symcal{M}$ possède un plus petit élément, ni même d'ailleurs que $\symcal{M}$ soit non vide. Néanmoins il est possible que $\symcal{M}$ possède un plus petit élément, et ceci que $\symcal{A}$ possède ou non un plus grand élément. Ce plus petit élément de $\symcal{M}$ est alors par définition la \emph{borne supérieure} de~$\symcal{A}$ dans l'ensemble $\symcal{E}$ ordonné par $\preccurlyeq$. 
\end{alert}

 \section{Applications, fonctions}\label{Rfonc}
%%%Fonction caractéristique, cardinal de $\symcal{P}(E)$ ou $\symfrak{P}(E)$
\subsection{Définition}
On dit qu'une relation binaire $f$ entre deux ensembles $E$ et~$F$ est fonctionnelle ou qu'elle est une \emph{fonction} définie sur~$E$ et à valeurs dans $F$ ou encore une \emph{application} de~$E$ dans $F$ si tout élément $x$ de $E$ est en relation avec un élément~$y$ de~$F$ et un seul. Cet élément $y$ est appelé \emph{image} de~$x$ par~$f$ et noté $f(x)$ (autrement dit on écrit plus volontiers~$y=f(x)$ que~$x \mathrel{f}y$). L'ensemble $E$ est l'ensemble de départ de la fonction et l'ensemble $F$ son ensemble d'arrivée. 

Une application est donc déterminée par la donnée de son ensemble de départ, de son ensemble d'arrivée et d'un procédé décrivant une correspondance fonctionnelle entre ces deux ensembles. On note souvent :

\begin{align*}&f\::\;E\to F\;,\quad x\longmapsto f(x)\,.\\
\intertext{Par exemple :}
&f\::\;\left]-1,1\right[\to\R\;,\quad x \longmapsto \frac{1}{x^2-1}\;,\\
\intertext{ou}
&g\::\;\Z\to\{0,1\}\;,\quad p\longmapsto
\begin{cases} 0&\text{ si $p$ est pair,}\\
1&\text{ sinon.}
\end{cases}
\end{align*}

Il n'y a pas de distinction entre fonction et application, mais le contexte peut introduire certaines nuances. En effet, lorsque $E$ est une partie de l'ensemble $\R$ des nombres réels, $f(x)$ est souvent donné par une expression, sa valeur étant le résultat d'un calcul. On emploie alors plus volontiers le vocable \emph{fonction} et dans ce cas l'ensemble de départ peut ne pas être précisé et correspond alors à l'ensemble des valeurs de $x$ pour lesquelles le calcul de $f(x)$ est possible. On parle alors plutôt d'ensemble de définition que d'ensemble de départ. 

\begin{example}[Exemples]
\begin{enumerate}
\item Soit la fonction $f$ de la variable réelle $x$, telle que 
\[f(x)=\frac{x}{x^2-4}\;.
\]
Déterminons l'ensemble de définition $\symcal{D}_f$ de la fonction $f$. On a 
\begin{align*}
x\in \symcal{D}_f &\iff x^2-4\neq 0\\
&\iff x\neq -2 \text{ et } x\neq 2.
\end{align*}

Donc $\symcal{D}_f= \R\smallsetminus\{-2,2\}$.
\item Soit la fonction $g$ telle que $g(x)=\sqrt{1-x^2}$. Déterminons l'ensemble de définition de $g$. On a
\[
x\in \symcal{D}_g \iff 1-x^2\geq 0.
\]

Il nous faut donc résoudre l'inéquation $1-x^2\geq0$. Le polynôme $1-x^2$ admet les nombres~$-1$ et~$1$ pour racines et le coefficient de $x^2$ est négatif :~\mbox{$1-x^2$} est donc positif dans l'intervalle des racines (on  pourra consulter le volet consacré aux polynômes du second degré).

Donc $\symcal{D}_g= \left[-1,1\right]$.
\end{enumerate}
\end{example}

\subsection{Antécédents, surjections, injections \& bijections}
\subsubsection{Notion d'antécédent}
Soit une application $f$ de $E$ dans $F$ et soit~$b$ élément de~$F$. On dit qu'un élément~$a$ de~$E$ est un \emph{antécédent} de $b$ si $f(a)=b$, autrement dit si~$b$ est l'image de~$a$ par $f$.
Alors que la définition d'une fonction entraîne que l'image d'un quelconque élément de $E$ existe et est unique, dans le cas d'une recherche d'antécédent, rien de ce genre n'est assuré.

Considérons par exemple la fonction \emph{carré} définie par
\[f\::\;\R\to\R\;,\quad x\longmapsto f(x)=x^2 \]
et recherchons les antécédents par $f$ d'un élément~$y$ de~$\R$, c.-à-d. les solutions de l'équation d'inconnue~$x$ :~\mbox{$x^2=y$}.
\begin{itemize}
\item \textit{Si $y>0$} : $y$ admet deux antécédents qui sont $\sqrt{y}$ et $-\sqrt{y}$.
\item \textit{Si $y=0$} : le seul antécédent de $0$ est $0$.
\item \textit{Si $y<0$} : $y$ ne possède aucun antécédent.
\end{itemize}
\begin{remark}
Il faut remarquer ici que la connaissance précise des ensembles de départ et d'arrivée est indispensable à la recherche des antécédents. Considérons la fonction
\[g\::\;\R_+\to\R_+\;,\quad x\longmapsto g(x)=x^2. \]
Les expressions de $g(x)$ et de $f(x)$ sont parfaitement identiques, mais, contrairement à ce que nous venons d'étudier pour la fonction $f$,  dans le cas de la fonction $g$ tout élément~$y$ de l'ensemble d'arrivée possède un antécédent unique qui est $\sqrt{y}$.
\end{remark}
\subsubsection{Surjections, injections et bijections}
\begin{remark}[Surjections]
On dit qu'une application $f$ de $E$ dans $F$ est \emph{surjective} (ou encore que~$f$ est une \emph{surjection}) si tout élément de $F$ possède au moins un antécédent dans~$E$.
\end{remark}
\begin{remark}[Injections]
On dit qu'une application $f$ de $E$ dans $F$ est \emph{injective} (ou encore que $f$ est une \emph{injection}) ou encore si  deux éléments distincts quelconques de $E$, ont des images distinctes. De façon plus formelle,~\mbox{$f\::\;E\to F$} est injective si
\begin{gather}\forall(x,y)\in E\times E, \;\bigl(x\neq y\implies f(x)\neq f(y)\bigr).\label{condinj1}\\
\intertext{Ou, par contraposition, $f\::\;E\to F$ est injective si}
\forall(x,y)\in E\times E, \;\bigl(f(x)= f(y)\implies x=y\bigr).\label{condinj2}
\end{gather}

On peut dire également que $f$ est injective si et seulement si tout élément de $F$ possède au plus un antécédent dans~$E$.
\end{remark}

\begin{remark}[Bijections]
On dit qu'une application $f$ de $E$ dans $F$ est \emph{bijective} (ou encore que $f$ est une \emph{bijection}) si tout élément de $F$ possède un antécédent dans $E$ et un seul, autrement dit si $f$ est à la fois surjective et injective.

Si l'application $f\::\;E\to F$ est bijective, l'application qui à tout élément de $F$ associe son unique antécédent dans $E$ est évidemment bijective : c'est la \emph{bijection réciproque} de~$f$, notée~$f^{-1}$.
\end{remark}
\begin{example}[Exemples]
\begin{enumerate}
\item \label{phi} Soit l'application 
\[
\phi\::\;\R\smallsetminus\{1\}\to\R\,,\quad x\longmapsto \frac{2x}{x-1}\;.
\]

Donnons-nous un réel $y$ quelconque et recherchons ses antécédents, qui sont les solutions dans $\R\smallsetminus\{1\}$ de l'équation $\phi(x)=y$, que nous résolvons ici :
\begin{align*}
\frac{2x}{x-1}=y &\iff 2x=y(x-1)\text{ et } x\neq 1\,,\\
&\iff x(2-y)=-y\text{ et } x\neq 1\,,\\
&\iff x(y-2)=y\text{ et } x\neq 1\,.
\end{align*}
\begin{itemize}
\item\textit{Si $y=2$}, l'équation est équivalente à $0=2$ et n'a donc pas de solution. Le réel $2$ n'a pas d'antécédent par $\phi$.
\item\textit{Si $y\neq 2$}, l'équation admet \[x=\frac{y}{y-2}\] pour unique solution (il est clair en effet que $y/(y-2)$ ne peut être égal à $1$).
\end{itemize}


En résumé, le réel $2$ ne possède aucun antécédent et les réels différents de $2$ possèdent un antécédent et un seul. L'application $\phi$ est donc injective, mais n'est pas surjective.
\item Soit maintenant l'application
\[
\psi\::\;\R\smallsetminus\{1\}\to\R\smallsetminus\{2\}\,,\quad x\longmapsto \frac{2x}{x-1}\;.
\]

On notera que les applications $\phi$ et $\psi$ ne diffèrent que par leur ensemble d'arrivée. Cependant, l'étude faite ci-dessus en \ref{phi} prouve que $\psi$ est bijective et que sa bijection réciproque est 
\[
\psi^{-1}\::\;\R\smallsetminus\{2\}\to\R\smallsetminus\{1\}\,,\quad x\longmapsto \frac{x}{x-2}\;.
\]
\item L'application $\theta$ définie par 
\[
\theta\::\;\R\to\R\,,\quad x\longmapsto x^2-1
\]
n'est ni injective ni surjective. En effet $\theta(x)$ est nul si et seulement si $x^2=1$, c.-à-d si et seulement si $x=-1$ ou $x=1$ : le réel $0$ possède donc deux antécedents pour $\theta$. Par ailleurs, pour tout réel~$x$,~$x^2\geq0$, donc $\theta(x)\geq-1$ : le réel $-2$, par exemple, ne possède aucun antécédent par $\theta$.
\end{enumerate}
\end{example}
\subsection{Composée de deux applications}

Soit deux applications $f\::\;E\to F$ et $g\::\;F\to G$, on appelle \emph{application composée de~$f$ par~$g$} et on note $g\rond f$ l'application de $E$ dans $G$ définie par :
\[g\rond f\::\; E\to G\,,\quad x\longmapsto g\rond f(x) = g\bigl(f(x)\bigr).\]

Notons qu'il est important pour définir $g\rond f$ que l'ensemble d'arrivée de $f$ coïncide avec l'ensemble de départ de~$g$.

\begin{example}
Soit 
\[
f\::\;\R^*\to\R\,,\; x\mapsto \frac1x\;\text{ et }\;g\::\:\R\to\R\,,\; x\mapsto x^2+1\,.
\]
Déterminons $g\rond f$ qui est une application de $\R^*$ dans $\R$.

On a pour tout $x\in\R^*$,
\begin{align*}
g\rond f(x) &= g\bigl(f(x)\bigr),\\
&={\bigl(f(x)\bigr)}^2+1,\\
&={\left(\frac1x\right)}^2+1,\\
\intertext{ou, en réduisant au même dénominateur,}
g\rond f(x)&= \frac{1+x^2}{x^2}\;.
\end{align*}
\begin{remark}
On notera que la même lettre $x$ intervient dans les définitions de~$f$ et~$g$. Il s'agit bien sûr d'une lettre muette, qui pourrait dans chaque cas être remplacée par n'importe quelle autre lettre (en évitant tout de même les lettres $f$ et $g$!).
\end{remark}
\end{example}

\begin{thm}Soit deux applications $f\::\;E\to F$ et $g\::\;F\to G$. Si $f$ et $g$ sont des bijections, alors l'application composée $g\rond f$ est une bijection dont la bijection réciproque est $f^{-1}\rond g^{-1}$. 
\end{thm}

\begin{proof}
Soit deux bijections $f\::\;E\to F$ et $g\::\;F\to G$. % et un élément $a$ de $E$. 
%Si $a$ est un antécédent de $b$ par $g\rond f$, on a $g\rond f(a)=c$, soit $g\bigl(f(a)\bigr)=c$. 
\begin{enumerate}
\item\label{gofsurj} Donnons nous un élément $c$ quelconque de $G$. Les applications~$g$ et~$f$ étant surjectives, $c$ possède un antécédent~$b$ par $g$  et $b$ possède lui-même un antécédent $a$ par $f$. On a alors 
\[g\rond f(a) =g\bigl(f(a)\bigr)=g(b)=c.\]
L'élément $c$ admet $a$ pour antécédent par $g\rond f$ qui est donc  est donc application surjective.
\item Supposons maintenant qu'il existe un élément $c$ de $G$ et $a$ et $a'$ éléments de $E$ tels que~$g\rond f(a)=g\rond f(a')$. Alors  $g\bigl(f(a)\bigr)=g\bigl(f(a')\bigr)$ et puisque $g$ est injective on a, d'après la proposition \eqref{condinj2} p.~\pageref{condinj2},  $f(a) = f(a')$; l'application $f$ étant elle-même injective, on obtient finalement~$a=a'$. On a donc prouvé que 
\[\forall(a,a')\in E\times E, \;\bigl(g\rond f(a)= g\rond f(a')\implies a=a'\bigr),\]
autrement dit que $g\rond f$ est injective.
\item L'application $g\rond f$ est donc surjective et injective : elle est bijective. On remarque alors en employant les notations du point \ref{gofsurj} ci-dessus \mbox{que $a=f^{-1}(b)$} et $b=g^{-1}(c)$, \mbox{d'où $a=f^{-1}\bigl(g^{-1}(c)\bigr)$}, ce qui achève la démonstration en prouvant que 
\[{\bigl(g\rond f\bigr)}^{-1}=f^{-1}\rond g^{-1}.\qedhere\]
\end{enumerate}
\end{proof}

\begin{remark}
La réciproque de ce théorème est fausse. Considérons par exemple les applications suivantes :
\[
f\::\;\R_+\to\R\,,\; x\mapsto \sqrt{x}\;\text{ et }\;g\::\;\R\to\R_+\,,\; x\mapsto x^2.
\]
Pour tout réel $x$ positif, on a $g\bigl(f(x)\bigr)=\bigl(\sqrt{x}\bigr)^2=x$, par définition de la racine carrée. Par conséquent, on a 
\[g\rond f \::\;\R_+\to\R_+\,,\quad x\longmapsto x,\]
qui est à l'évidence une bijection. Cependant, $f$ n'est pas surjective (les nombres strictement négatifs n'ont pas d'antécédent) et $g$ n'est pas injective (deux nombres opposés non nuls ont la même image).
\end{remark}

On a néanmoins le théorème suivant:
\begin{thm}
Soit deux applications $f\::\;E\to F$ et $g\::\;F\to G$. Si $g\rond f$ est injective, $f$ est injective. Si $g\rond f$ est surjective, $g$ est surjective.
\end{thm}

\begin{proof}
\begin{itemize}
\item Supposons  $g\rond f$ injective et soit $x$ et $y$ éléments de $E$ tels que~\mbox{$f(x)=f(y)$}. Montrons que nécessairement $x=y$, ce qui établira que~$f$ est injective.

Puisque $f(x)=f(y)$, $g\bigl(f(x)\bigr)=g\bigl(f(y)\bigr)$, soit $g\rond f(x)=g\rond f(y)$ : on en déduit que~$x=y$, car $g\rond f$ est injective par hypothèse. Par conséquent~$f$ est injective.
\item Supposons  $g\rond f$ surjective et soit $z$ un élément de $G$. L'application~$g\rond f$ étant surjective, il existe un élément $x$ de $E$ tel que $g\rond f(x)=z$, soit~$g\bigl(f(x)\bigr)=z$. L'élément~$z$ admet donc $f(x)$ pour antécédent par $g$, ce qui prouve que~$g$ est surjective. \qedhere
\end{itemize}
\end{proof}
\endinput
% !TEX TS-program = LuaLaTeX

\renewcommand{\theprop}{(\ensuremath{\fnsymbol{prop}})}

\tgotitle{Raisonnement par récurrence}
%
\section{Une propriété des entiers naturels}
La propriété énoncée dans cette section est parfois présentée comme un axiome. Il n'entre toutefois pas dans nos intentions de construire une ou des axiomatiques, ne serait-ce que parce que cette notion est délicate. On se contentera donc d'un point de vue naïf, selon lequel un axiome est un énoncé dont on admet qu'il est vrai. Le lecteur doit toutefois savoir que les énoncés de la propriété~\ref{propppe} et du théorème~\ref{threc} ont des rôles en quelque sorte interchangeables : dans les axiomatiques classiques permettant de fonder l'arithmétique, prendre l'un comme axiome permet de démontrer l'autre comme théorème. Pour cette raison, le lecteur pourra trouver ailleurs des présentations proposant un axiome (ou parfois un principe) de \emph{récurrence}, la propriété~\ref{propppe} devenant alors un théorème.
\par
\vspace{0.5\baselineskip}\par
Nous supposons donc connu l'ensemble $\N$ des entiers naturels, muni notamment d'une relation d'ordre notée ${≤}$ et nous admettons que cet ensemble  possède la propriété suivante:
%
\begin{prop}
Toute partie non vide de $\N$ possède un plus petit élément.\label{propppe}
\end{prop}

On traduit souvent cette propriété en disant que \N{} est \emph{bien ordonné}.
%x
%\par\noindent{\color{gray}\rule{\linewidth}{5pt}}\par\addvspace{-3pt}\par\nobreak

%Cette propriété peut être considérée comme un des axiomes qui permettent de décrire l'ensemble \N{} des entiers naturels. Elle peut également se démontrer à partir d'un autre système d'axiomes, de portée équivalente (axiomes de Peano, notamment) ou de portée plus générale (Zermelo-Fraenkel, par exemple).

%\par\noindent\hrulefill
\section{Récurrence}
\subsection{Introduction}
\label{secexplerec}
La propriété \ref{propppe} permet de formaliser une méthode consistant à établir la validité d'une assertion \frquote{de proche en proche}. Fixons-nous par exemple un réel $a$ strictement positif et, pour tout entier naturel $n$, considérons la propriété $P(n)$
\[(1+a)^n\geq 1+na.\]

Nous nous proposons de démontrer que cette propriété est vraie pour tout entier naturel~$n$.

Prouvons dans un premier temps que $P(0)$ est vraie (\mbox{c.-à-d.} que si $n=0$ alors $P(n)$ est vraie) : puisqu'il est vrai que $(1+a)^0=1$ et $1+0\times a=1$, il est établi que $(1+a)^0\geq1+0\times a$, donc que $P(0)$ est vraie.

Donnons-nous alors un entier naturel $n$ et supposons que $P(n)$ est vraie. Prouvons alors que nécessairement $P(n+1)$ est vraie :
\begin{align*}
(1+a)^n&\geq1+na,\\
\intertext{soit, en multipliant les deux membres par $1+a$, qui est strictement positif,}
(1+a)^{n+1}&\geq(1+a)(1+na),\\
\intertext{d'où}
(1+a)^{n+1}&\geq 1+(n+1)a+na^2,\\
\intertext{et enfin}
(1+a)^{n+1}&\geq 1+(n+1)a,\quad \text{car $na^2\geq0$.}
\end{align*}

Nous venons d'établir que, pour tout entier naturel $n$, si $P(n)$ est vraie alors $P(n+1)$ est vraie. En termes plus formels :
\begin{equation}
\forall n\in\N,\; P(n)\implies P(n+1).
\label{eqhered}
\end{equation}

Nous notons maintenant $\symcal{A}$ l'ensemble des valeurs de l'entier naturel $n$ pour lesquelles $P(n)$ \emph{est fausse}. Nous allons prouver \emph{par l'absurde} que $\symcal{A}$ est vide, autrement dit que $P(n)$ est vraie pour tout entier naturel $n$. 

Supposons donc que $\symcal{A}$ est non vide. En vertu de la propriété \ref{propppe}, $\symcal{A}$ possède un plus petit élément que nous notons $m$. Puisque $P(0)$ est vraie, on peut affirmer que $0\notin\symcal{A}$ donc que $m>0$, ce qui nous assure que $m-1\in\N$. Le fait que $m-1<m$ nous prouve maintenant que $m-1\notin\symcal{A}$, puisque $m$ est le plus petit élément de $\symcal{A}$. Alors $P(m-1)$ est vraie, ce qui entraîne, d'après la proposition \eqref{eqhered}, que $P(m)$ est vraie également. L'entier $m$ étant élément de $\symcal{A}$ on aboutit à une contradiction. Finalement $\symcal{A}$ est vide et la propriété $P(n)$ est vraie pour tout entier $n$.

\begin{remark}
On traduit la proposition \eqref{eqhered} en disant que la propriété $P$ est \emph{héréditaire} à partir du rang $0$.
\end{remark}

\subsection{Un théorème}
Nous souhaitons maintenant nous éviter le recours direct à la propriété \ref{propppe}, et en particulier un raisonnement par l'absurde nécessitant l'introduction un peu lourde d'une partie telle que $\symcal{A}$. Nous souhaitons également pouvoir facilement établir qu'une propriété est vraie non pas sur \N{} tout entier, mais plutôt pour tout entier supérieur à un entier $n_0$ donné. À cet effet nous formulons maintenant le \emph{théorème de récurrence}, dont la démonstration ne sera pas très différente de celle qui a été faite sur l'exemple de la section \ref{secexplerec}.
\begin{thm}
Soit un entier $n_0$ fixé et une propriété $P(n)$ dont l'énoncé dépend d'un entier $n$. Si $P(n_0)$ est vraie et si pour tout entier $n$ supérieur à $n_0$ l'implication
\[P(n)\implies P(n+1)\]
est vraie, alors la propriété $P(n)$ est vraie pour tout entier $n$ supérieur à $n_0$.
\label{threc}
\end{thm}
\begin{proof}Considérons une propriété $P(n)$ vérifiant les hypothèses de notre théorème et montrons alors que $P(n)$ est vraie pour tout entier $n\geq n_0$. 
Nous noterons $\symcal{A}$ l'ensemble des entiers naturels $k$ pour lesquels que $P(n_0+k)$ est fausse. Le fait que $P(n_0)$ est vrai se traduit par 
\[
0\notin\symcal{A}
\]
 et plus généralement, pour $n≥n_0$, en remarquant que $n$ peut s'écrire $n_0+(n-n_0)$,
  \frquote{$P(n)$ est vrai} se traduit par $(n-n_0)\notin\symcal{A}$. 
 En posant $k=n-n_0$, l'hypothèse 
 \begin{align}
 \forall n≥n_0,\; P(n)&\implies P(n+1)\notag\\
 \intertext{nous assure donc que}
 \forall k\in\N,\;k\notin\symcal{A}&\implies(k+1)\notin\symcal{A}. \label{eqheredA}
 \end{align}
 
Prouver que la propriété $P(n)$ est vraie pour tout entier $n$ supérieur à $n_0$ revient à prouver que $\symcal{A}$ est vide. Raisonnons par l'absurde et supposons que $\symcal{A}$ est non vide : en vertu de la propriété \ref{propppe}, $\symcal{A}$ possède alors un plus petit élément que nous notons~$m$. Nous savons que $m>0$, puisque $0\notin\symcal{A}$, et cela nous assure que $m-1\in\N$. Le fait que $m-1<m$ nous prouve maintenant que $m-1\notin\symcal{A}$, puisque $m$ est le plus petit élément de $\symcal{A}$. Alors d'après la proposition \eqref{eqheredA}, $m\notin\symcal{A}$, ce qui est en contradiction avec le fait que $m$ est le plus petit élément de $\symcal{A}$. Il est alors établi que la propriété $P(n)$ est vraie pour tout $n≥n_0$.
 \end{proof}
%
\begin{remark}
Un raisonnement par récurrence consiste donc à appliquer le théorème \ref{threc} pour démontrer qu'une propriété $P(n)$ est vraie pour tout entier $n$ supérieur à un certain entier $n_0$. Il se fait en trois étapes qui doivent, pour la clarté du propos, être suffisamment mises en évidence.
\begin{enumerate}
\item \textit{Initialisation :} on établit que $P(n_0)$ est vraie.
\item \textit{Preuve du caractère héréditaire de la propriété :} on démontre que 
\[
\forall n≥n_0,\; P(n)\implies P(n+1).
\]
\item \textit{Conclusion :} on a ainsi établi par récurrence que $P(n)$ est vraie pour tout $n≥n_0$.
\end{enumerate}
\end{remark}

\subsection{Applications et exemples}

\subsubsection{Une somme}
On se propose de démontrer par récurrence que la somme $S_n$ des $n$ premiers entiers non nuls est égale à $n(n+1)/2$. Par définition on a donc
\[
S_n=\sum_{k=1}^n k\;\text{ ou si l'on préfère }\;  S_n=1+2+\cdots+n-1+n.
\]
On veut démontrer par récurrence que la propriété $P(n)$, définie par 
\[
S_n=\frac{n(n+1)}{2}\;,
\]
est vraie pour tout $n≥1$.
\begin{enumerate}
\item \textit{Initialisation :} on a $S_1=1$ et pour $n=1$, $n(n+1)/2=2/2=1$. Donc $P(1)$ est vraie.
\item \textit{Hérédité :} donnons nous un entier $n≥1$ et supposons que $P(n)$ est vraie. 
Montrons alors que $P(n+1)$ est vraie c'est à dire que $S_{n+1}=((n+1)(n+2))/2$.
On sait que $S_{n+1}=S_{n}+n+1$, donc
\begin{align*}
S_{n+1}&=\frac{n(n+1)}{2}+n+1,\\
&=\frac{n(n+1)+2(n+1)}{2}\;,\\
&=\frac{(n+1)(n+2)}{2}\:.
\end{align*}
On vient d'établir que $\forall n≥1,\;P(n)\implies P(n+1)$. La propriété $P(n)$ est donc héréditaire à partir du rang $1$.
\item \textit{Conclusion :} la propriété $P(n)$ est vraie pour tout $n≥1$.
\[
\forall n≥1,\;S_n=\frac{n(n+1)}{2}\:.
\]
\end{enumerate}

\subsubsection{Une autre somme}
On se propose cette fois de démontrer par récurrence que la somme $T_n$ des carrés des $n$ premiers entiers non nuls est égale à $\bigl(n(n+1)(2n+1)\bigr)/6$. Par définition on a donc
\[
T_n=\sum_{k=1}^n k^2\;\text{ ou si l'on préfère }\;  T_n=1^2+2^2+\cdots+(n-1)^2+n^2.
\]
On veut démontrer par récurrence que la propriété $P(n)$, définie par
\[
T_n=\frac{n(n+1)(2n+1)}{6}\;,
\]
est vraie pour tout $n≥1$.
\begin{enumerate}
\item \textit{Initialisation :} on a $T_1=1$ et pour $n=1$, 
\[\frac{n(n+1)(2n+1)}{6}=\frac{2\times3}{6}=1.\]
 Donc $P(1)$ est vraie.
\item \textit{Hérédité :} donnons nous un entier $n≥1$ et supposons que $P(n)$ est vraie. 
Montrons alors que $P(n+1)$ est vraie c'est à dire que 
\begin{align*}
T_{n+1}&=\frac{(n+1)\bigl((n+1)+1)(2(n+1)+1\bigr)}{6}\;,\\
&=\frac{(n+1)(n+2)(2n+3)}{6}\:.
\end{align*}
On sait que $T_{n+1}=T_{n}+(n+1)^2$, donc
\begin{align*}
T_{n+1}&=\frac{n(n+1)(2n+1)}{6}+(n+1)^2,\\
&=\frac{n(n+1)(2n+1)+6(n+1)(n+1)}{6}\;,\\
\intertext{soit, en factorisant $n+1$}
T_{n+1}&=\frac{(n+1)\bigl(n(2n+1)+6(n+1)\bigr)}{6}\;,\\
&=\frac{(n+1)(2n^2+n+6n+6)}{6}\;.\\
\intertext{On remarque alors que $n+6n=7n$, que l'on peut aussi écrire $3n+4n$ d'où}
T_{n+1}&=\frac{(n+1)(2n^2+3n+4n+6)}{6}\;,\\
&=\frac{(n+1)\bigl(n(2n+3)+2(2n+3)\bigr)}{6}\;,\\
&=\frac{(n+1)(n+2)(2n+3)}{6}\:.
\end{align*}
On vient d'établir que $\forall n≥1,\;P(n)\implies P(n+1)$. La propriété $P(n)$ est donc héréditaire à partir du rang $1$.
\item \textit{Conclusion :} la propriété $P(n)$ est vraie pour tout $n≥1$.
\[
\forall n≥1,\;T_n=\frac{n(n+1)(2n+1)}{6}\:.
\]
\end{enumerate}

\subsubsection{Divisibilité}
Démontrer par récurrence que, pour tout entier naturel $n$, $n^3+2n$ est un multiple de $3$.
On veut démontrer par récurrence que la propriété $P(n)$, définie par
\[
\exists\, k\in\N \text{ tel que }n^3+2n=3k,
\]
est vraie pour tout $n≥0$.
\begin{enumerate}
\item \textit{Initialisation :} Si $n=0$, $n^3+2n=0$ qui s'écrit aussi $3\times0$. Donc $P(0)$ est vraie.
\item \textit{Hérédité :} Donnons nous un entier naturel $n$ et supposons que $P(n)$ est vraie. Montrons alors que $P(n+1)$ est vraie, c.-à-d. que $(n+1)^3+2(n+1)$ est divisible par~$3$.
\begin{align*}
(n+1)^3+2(n+1)&=n^3+3n^2+3n+1+2n+2,\\
&=(n^3+2n)+3(n^2+n+1).
\end{align*}
L'hypothèse de récurrence nous dit que $P(n)$ est vraie, donc qu'il existe un entier $k$ tel que \mbox{$n^3+2n=3k$}. Alors
\[(n+1)^3+2(n+1)=3(k+n^2+n+1),\]
Ce qui prouve que $(n+1)^3+2(n+1)$ est un multiple de 3.

On vient d'établir que $\forall n\in\N,\;P(n)\implies P(n+1)$. La propriété $P(n)$ est donc héréditaire à partir du rang $0$.
\item \textit{Conclusion :} la propriété $P(n)$ est vraie pour tout $n\in\N$. 
\[
\forall n\in\N,\; n^3+2n \text{ est divisible par $3$.}
\]
\end{enumerate}

\endinput

\renewcommand{\arraystretch}{1.3}
%\renewcommand*{\DinoThmLogo}{\textcolor{ColorOne}{\oldpilcrowfour}\ignorespaces}
%\renewcommand*{\DinoThmLogo}{\hspace*{\DinoTitleIndent-3pt}\ignorespaces}

%\textcolor{ColorOne}{\oldpilcrowfour}\ignorespaces

%%%%%% quelques figures, le début arrive ensuite
\newcommand{\paraone}{
\begin{tikzpicture}[domain=-5:5,samples=100,x=6mm,y=6mm]
		\clip (-2,-3.5) rectangle (3.5,3);
	\draw[very thin,color=gray,xstep=0.5,ystep=0.5] (-2,-2.5) grid (3.5,3);
	\draw [->](-2,0) -- (3.5,0) node[above=15mm,left=4mm] {$\symcal{C}_P$};
	\draw[->] (-1.9,-2.5) -- (-1.9,3) node[above] {};
	\draw[semithick,domain=-2:3.5,label=left:$\symcal{C}_P$] plot (\x,{0.5(\x+1)*(\x-2)});
	\draw [fill=white] (-1,0) circle (1.2pt) node[below=2mm,left] {$\beta$};
\draw [fill=white] (2,0) circle (1.2pt) node[below=2mm,right] {$\alpha$};
\node[below] at (.75,-2.5){$\Delta>0\;\text{ et }\; a>0$};	
	\end{tikzpicture}
}
\newcommand{\paratwo}{	
	\begin{tikzpicture}[domain=-5:5,samples=100,x=6mm,y=6mm]
		\clip (-2,-4) rectangle (3.5,2.5);
	\draw[very thin,color=gray,xstep=0.5,ystep=0.5] (-2,-3) grid (3.5,2.5);
	\draw [->](-2,0) -- (3.5,0) node[above=12mm,left=8mm] {$\symcal{C}_P$};
	\draw[->] (-1.9,-3) -- (-1.9,2.5) node[above] {};
	\draw[semithick,domain=-2:3.5,label=left:$\symcal{C}_P$] plot (\x,{-(\x+1)*(\x-2)});
	\draw [fill=white] (-1,0) circle (1.2pt) node[above=2mm,left] {$\alpha$};
\draw [fill=white] (2,0) circle (1.2pt) node[above=2mm,right] {$\beta$};
\draw[fill=white,white] (-2,-4) rectangle (3.5,-3);
\node[below] at (.75,-3){$\Delta>0\;\text{ et }\; a<0$};		
	\end{tikzpicture}	
}

\newcommand{\parathree}{	
\begin{tikzpicture}[domain=-5:5,samples=100,x=6mm,y=6mm]
		\clip (-2,-1.5) rectangle (3.5,5);
	\draw[very thin,color=gray,xstep=0.5,ystep=0.5] (-2,-.5) grid (3.5,5);
	\draw [->](-2,0) -- (3.5,0) node[above=25mm,left=6mm] {$\symcal{C}_P$};
	\draw[->] (-1.9,-.5) -- (-1.9,5) node[above] {};
	\draw[semithick,domain=-2:3.5] plot (\x,{(\x+1)*(\x-2)+2.25});
	\draw [fill=white] (0.5,0) circle (1.2pt) node[above=1mm] {$\frac{-b}{2a}$};
\node[below] at (.75,-0.5){$\Delta=0\;\text{ et }\; a>0$};	
	\end{tikzpicture}
}

\newcommand{\parafour}{	
\begin{tikzpicture}[domain=-5:5,samples=100,x=6mm,y=6mm]
		\clip (-2,-6) rectangle (3.5,0.5);
	\draw[very thin,color=gray,xstep=0.5,ystep=0.5] (-2,-5) grid (3.5,.5);
	\draw [->](-2,0) -- (3.5,0) node[below=25mm,left=6mm] {$\symcal{C}_P$};
	\draw[->] (-1.9,-5) -- (-1.9,.5) node[above] {};
	\draw[semithick,domain=-2:3.5] plot (\x,{-(\x+1)*(\x-2)-2.25});
	\draw [fill=white] (0.5,0) circle (1.2pt) node[below=1mm] {$\frac{-b}{2a}$};
\draw[fill=white,white] (-2,-6) rectangle (3.5,-5);	
\node[below] at (.75,-5){$\Delta=0\;\text{ et }\; a<0$};	
	\end{tikzpicture}
}

\newcommand{\parafive}{
\begin{tikzpicture}[domain=-5:5,samples=100,x=6mm,y=6mm]
		\clip (-2,-1.5) rectangle (3.5,5);
	\draw[very thin,color=gray,xstep=0.5,ystep=0.5] (-2,-.5) grid (3.5,5);
	\draw [->](-2,0) -- (3.5,0) node[above=25mm,left=6mm] {$\symcal{C}_P$};
	\draw[->] (-1.9,-.5) -- (-1.9,5) node[above] {};
	\draw[semithick,domain=-2:3.5] plot (\x,{(\x+1)*(\x-2)+2.75});
\node[below] at (.75,-.5){$\Delta<0\;\text{ et }\; a>0$};		
	\end{tikzpicture}
}

\newcommand{\parasix}{	
	\begin{tikzpicture}[domain=-5:5,samples=100,x=6mm,y=6mm]
		\clip (-2,-6) rectangle (3.5,0.5);
	\draw[very thin,color=gray,xstep=0.5,ystep=0.5] (-2,-5) grid (3.5,.5);
	\draw [->](-2,0) -- (3.5,0) node[below=25mm,left=6mm] {$\symcal{C}_P$};
	\draw[->] (-1.9,-5) -- (-1.9,.5) node[above] {};
	\draw[semithick,domain=-2:3.5] plot (\x,{-(\x+1)*(\x-2)-2.75});
\draw[fill=white,white] (-2,-6) rectangle (3.5,-5);	
\node[below] at (.75,-5){$\Delta<0\;\text{ et }\; a<0$};	
	\end{tikzpicture}
}
%%%%
\tgotitle{Polynômes du second degré}
\tgoshorttoc
\section{Généralités}
\subsection{Fonction polynôme du second degré}
On dit qu'une fonction $P$, définie sur l'ensemble $\R$ des nombres réels et à valeurs dans $\R$, est une fonction polynôme du second degré s'il existe un réel $a$ non nul ainsi que deux réels $b$ et $c$ tels que
\[\forall x\in\R,\; P(x)=ax^2+bx+c.\]
On convient alors de dire que $ax^2+bx+c$ est un polynôme du second degré.

\begin{remark}
Cette définition n'entraîne évidemment pas qu'une fonction polynôme de degré 2 donnée ne puisse pas s'écrire sous une forme différente. Par exemple :
\begin{itemize}
\item Cas de $P : \R\rightarrow\R,\; x\mapsto x^2-3x+2$ : $\forall\,x\in\R$, $P(x)=(x-1)(x-2)$.
\item  Cas de $Q : \R\rightarrow\R,\; x\mapsto x^2-1$ : $\forall\,x\in\R$,
 \[Q(x)=\frac{x^4-1}{x^2+1}.\]
 \item Cas de $R : \R\rightarrow\R,\; x\mapsto x^2-2x+5$ : $\forall\,x\in\R$, $R(x)=\left(\left({x-1}\right)^2\right)+4$.
\end{itemize}
\end{remark}
\subsection{Écriture unique}
Il existe cependant un certaine unicité de l'écriture d'un polynôme de degré~2, qui concerne plus précisément ses \emph{coefficients} $a$, $b$ et $c$.
\begin{thm}
Soit $(a,a')\in\R^*\times\R^*$, $(b,b')\in\R\times\R$ et $(c,c')\in\R\times\R$ tels que
\[\forall x\in\R,\; ax^2+bx+c=a'x^2+b'x+c',\]
alors $a=a'$, $b=b'$ et $c=c'$. 
\label{thunicite}
\end{thm}
\begin{proof}
L'égalité entre les deux expressions étant vraie pour tout $x\in\R$, elle l'est en particulier pour $x=0$. On en déduit que $c=c'$ et par conséquent
\begin{align*}
\forall x\in\R,\; ax^2+bx&=a'x^2+b'x,\\
\forall x\in\R,\; x(ax+b)&=x(a'x+b'),\\
\intertext{Soit, en simplifiant par $x\neq0$}
\forall x\in\R^*,\;ax+b&=a'x+b'.
\end{align*}

En donnant alors successivement à $x$ les valeurs $1$ et $2$, on obtient :
\[\left\{\begin{aligned}
(a-a')+(b-b')&=0\\
2(a-a')+(b-b')&=0
\end{aligned}\right.\;,\]
dont la résolution amène à $a-a'=0$ et $b-b'=0$, c.-à-d. : $a=a'$ et $b=b'$.
\end{proof}

\section{Forme canonique}
\subsection{Méthode de complétion au carré}\label{sec2-1}
Fixons nous un réel $m$. On rappelle que, pour tout $x\in\R$, $(x+m)^2=x^2+2mx+m^2$. Ou encore :
\begin{equation}
\forall x\in\R, \; x^2+2mx=\left(x+m\right)^2-m^2.
\end{equation}

Dans le cas de $P : \R\rightarrow\R, x\mapsto x^2+2mx+p$, on remarque que l'on peut alors écrire 
\begin{equation}
x^2+2mx+p=\left(x+m\right)^2-m^2+p.\label{eqfcreduite}
\end{equation}

Cette remarque fonde la \emph{méthode de complétion au carré}.

\begin{example}[Exemples]
\begin{enumerate}
\item  \label{explefactor} 
On souhaite résoudre l'équation $x^2-6x+7=0$. On remarque que \[x^2-6x+7=(x-3)^2-9+7\] et on se ramène ainsi à
\begin{align*}
(x-3)^2-2&=0,\\
(x-3)^2-\bigl(\sqrt{2}\bigr)^2&=0,\\
\intertext{puis en employant une identité remarquable}
\bigl(x-3+\sqrt{2}\bigr)\bigl(x-3-\sqrt{2}\bigr)&=0.
\end{align*}
D'où l'on déduit que l'ensemble des solutions de $x^2-6x+7=0$ est 
\[
\symcal{S}=\bigl\{3-\sqrt{2}, 3+\sqrt{2}\bigr\}.
\]
\item Démontrer que $x^2+x+1$ est strictement positif pour tout $x\in\R$.
On écrit 
\begin{align*}
x^2+x+1&=x^2+2\times\frac12x+1,\\
&=\left(x+\frac12\right)^2-\frac14+1,\\
&=\left(x+\frac12\right)^2+\frac34\,.
\end{align*}

On en déduit que, pour tout $x\in\R$,
\[x^2+x+1≥\frac34\:.\]
On a donc finalement : $\forall x\in\R,\; x^2+x+1>0$.
\end{enumerate}
\end{example}
\subsection{Cas général}
En factorisant $a$ dans  $P(x)$, on obtient le théorème suivant :
\begin{thm}
Soit $P$, la fonction polynôme de degré 2 définie sur $\R$ par $P(x)=ax^2+bx+c$. On a 
\begin{equation}
\forall x\in\R,\;P(x)=a\left(\left(x+\frac{b}{2a}\right)^2-\frac{\Delta}{4a^2}\vphantom{\frac{b^2}{4a^2}}\right),\label{eqformcan}
\end{equation}
où $\Delta$, que l'on nomme \emph{discriminant} du polynôme $P(x)$, vaut $\Delta=b^2-4ac$.
\label{thformcan}
\end{thm}
\begin{proof}
On commence par factoriser $a$ dans $P(x)$ :
\[P(x)=a\left(x^2+\frac{b}{a}x+\frac{c}{a}\right),\]
puis on utilise l'égalité \eqref{eqfcreduite} avec $m=b/(2a)$ et $p=c/a$. On obtient 
\[P(x)=a\left(\left(x+\frac{b}{2a}\right)^2-\frac{b^2}{4a^2}+\frac{c}{a}\right).
\]
%D'où le résultat annoncé.
Une réduction au même dénominateur amène alors au résultat annoncé.
\end{proof}

\subsubsection{Discriminant réduit}
Dans le cas où $b$ se met agréablement sous la forme $2b'$ (par exemple, lorsque $b$ est un entier pair) on peut remarquer que $b/2a=b'/a$ et que $\Delta=4b'^2-4ac$.
 On est alors amené à poser $\Delta'=b'^2-ac$ et l'on obtient
\begin{equation}
\forall x\in\R,\;P(x)=a\left(\left(x+\frac{b'}{a}\right)^2-\frac{\Delta'}{a^2}\vphantom{\frac{b^2}{4a^2}}\right).
\end{equation}
Le nombre $\Delta'$ est souvent nommé \emph{discriminant réduit}

\subsubsection{Extrémums}
La forme canonique de la fonction $P$ met en évidence l'existence d'un extrémum dont la nature dépend du signe de $a$.
\begin{thm}
Soit $P$, la fonction polynôme de degré 2 définie sur $\R$ par $P(x)=ax^2+bx+c$. 
\XSmartphoneCommand{\vspace{\baselineskip}}

La fonction $P$ admet en\SmartphoneCommand{\\} $-\dfrac{b}{2a}\left\{\begin{aligned} 
\text{un \emph{minimum} égal à $\smash{-\dfrac{\Delta}{4a}}$}\; &\text{si $a>0$,}\\[2.5ex]
\text{un \emph{maximum} égal à $\smash{-\dfrac{\Delta}{4a}}$}\; &\text{si $a<0$.}
\end{aligned}\right.$
\vspace{1ex}\par
\label{thextrem}
\end{thm}

\begin{proof}
On voit tout d'abord que 
\[\forall x\in\R,\;\left(x+\frac{b}{2a}\right)^2≥0,
\]
donc que
\[
\forall x\in\R,\;\left(x+\frac{b}{2a}\right)^2-\frac{\Delta}{4a^2}≥-\frac{\Delta}{4a^2},
\]
l'égalité n'étant atteinte que si le carré est nul, donc si $x=-b/(2a)$. On fait alors apparaître $P(x)$ (compte tenu de l'égalité \eqref{eqformcan} du théorème \ref{thformcan}) en multipliant chaque membre de cette inégalité par $a\neq0$, ce qui change ou non son sens, selon que a est positif ou négatif. Il vient alors :
\begin{itemize}
\item si $a>0$, 
\[
\forall x\in\R,\; P(x)≥-\frac{\Delta}{4a}\;;
\]
\item et si $a<0$, 
\[
\forall x\in\R,\; P(x)≤-\frac{\Delta}{4a}\,.
\]
\end{itemize}
On sait de plus que dans les deux cas, l'égalité est atteinte si et seulement si $x=-b/(2a)$.
\end{proof}

\section{Factorisation}
\subsection{Théorème de factorisation}
La méthode vue dans l'exemple \ref{explefactor} de la section \ref{sec2-1} se généralise, pourvu que $\Delta$ soit positif. Précisément, on a le théorème 
\begin{thm}\label{thfactor}%
Soit $P$, la fonction polynôme de degré 2 définie sur $\R$ par $P(x)=ax^2+bx+c$.  Si le discriminant $\Delta$ de $P$ est positif ($\Delta≥0$), alors
\begin{align*}
\forall x \in\R,\;P(x)&=a(x-\alpha)(x-\beta)\\
\text{où}\quad\alpha=\frac{-b+\sqrt{\Delta}}{2a}&\text{ et }\beta=\frac{-b-\sqrt{\Delta}}{2a}.
\end{align*}
\end{thm} 
\begin{remark}
Le cas où $\Delta=0$ est particulier, puisque l'on a alors 
\[
\alpha=\beta=\frac{-b}{2a}\quad\text{soit :}\quad\forall x\in\R,\;P(x)=a\left(x-\frac{-b}{2a}\right)^2.
\]
\end{remark}

\begin{proof}
On remarque que si $\Delta≥0$, on peut écrire
\[
\frac{\Delta}{4a^2}=\left(\frac{\sqrt{\Delta}}{2a}\right)^2.
\] 

En reportant dans l'égalité \eqref{eqformcan} du théorème \ref{thformcan}, on obtient
\[
\forall x\in\R,\;P(x)=a\left(\left(x+\frac{b}{2a}\right)^2-\left(\frac{\sqrt{\Delta}}{2a}\right)^2\right).
\]

On obtient alors le résultat annoncé en factorisant la différence de deux carrés grâce à l'identité remarquable $A^2-B^2=(A+B)(A-B)$.
\end{proof}

\subsection{Racines}
Soit $P$ une fonction polynôme du second degré. 
On dit que $P$ est factorisable par $(x-x_0)$ s'il existe $m\in\R^*$ et $p\in\R$ tels que
\begin{equation}\forall x\in\R,\;P(x)=(x-x_0)(mx+p).\label{eqfactorx0}\end{equation}
On dit qu'un réel $x_0$ est une racine de $P$ si $P(x_0)=0$ : une racine de $P$ n'est donc rien d'autre qu'une solution de l'équation $P(x)=0$.

Nous pouvons maintenant formuler le théorème qui suit :
\begin{thm}
Soit $P$ une fonction polynôme du second degré et $x_0\in\R$. Le réel $x_0$ est une racine de $P$ si et seulement si $P$ est factorisable par $(x-x_0)$.
\end{thm}

\begin{proof}
Tout d'abord, si $P$ est factorisable par $x_0$, il suffit de calculer $P(x_0)$ en utilisant l'égalité \eqref{eqfactorx0} pour constater que $P(x_0)=0$, donc que $x_0$ est une racine de $P$.

Réciproquement supposons que $x_0$ est une racine de $P$, donc que $P(x_0)=0$. Soit alors $a$, $b$ et $c$ les coefficients de $P(x)$ : on a 
\[ax_0^2+bx_0+c=P(x_0)=0,\]
 d'où $c=-ax_0^2-bx_0$. Et en substituant cette écriture de $c$ dans l'expression de $P(x)$ :
\begin{align*}
P(x)&=ax^2+bx+c\\
&= ax^2+bx-ax_0^2-bx_0,\\
&= a(x^2-x_0^2)+b(x-x_0),\\
&= a(x-x_0)(x+x_0)+b(x-x_0),\\
&=(x-x_0)(ax+ax_0+b).
\end{align*}

On en arrive à $P(x)=(x-x_0)(mx+p)$, en prenant $m=a$ et $p=ax_0+b$.
\end{proof}

\subsubsection{Racines et équation du second degré}
Nous faisons ici le point sur les racines de $P$, autrement dit sur la résolution de l'équation 
$P(x)=0$.


\begin{itemize}
\item  \textit{Si $\Delta<0$,} nous reprenons le théorème \ref{thextrem} : dans ce cas, la fonction $P$ possède un minimum strictement positif si $a>0$ ou un maximum strictement négatif si $a<0$. Dans chaque cas, si $\Delta<0$, l'équation $P(x)=0$ ne possède pas de solution, autrement dit $P$ n'admet aucune racine.

\item  \textit{Si maintenant $\Delta=0$,} la remarque qui suit le théorème \ref{thfactor} prouve que l'équation $P(x)=0$ équivaut à
\[
\left(x-\frac{-b}{2a}\right)^2=0
\]
 et admet $-b/(2a)$ pour seule solution, ou encore que ce nombre est la seule racine de $P$. Dans la mesure où le facteur $x-(-b/(2a))$ intervient par son carré, on parle alors de \emph{racine double}.
 
 \item  \textit{Enfin si $\Delta>0$,} d'après le théorème \ref{thfactor}, $P(x)=0$ équivaut à $(x-\alpha)(x-\beta)=0$.
%\DinoXiPadminiCommand{$(x-\alpha)(x-\beta)=0$.}
%\DinoiPadminiCommand{\[(x-\alpha)(x-\beta)=0.\]}
 Cette équation admet donc deux solutions, ou encore $P$ admet deux racines qui sont
 \[
 \alpha=\frac{-b+\sqrt{\Delta}}{2a}\quad\text{et}\quad\beta=\frac{-b-\sqrt{\Delta}}{2a}.
 \]
\end{itemize}


\begin{remark}
Dons le cas où l'on a posé $b=2b'$ et utilisé le discriminant réduit $\Delta'$, il est facile de vérifier que les racines s'expriment comme suit :
\[
\alpha=\frac{-b'+\sqrt{\Delta'}}{a}\quad\text{et}\quad\beta=\frac{-b'-\sqrt{\Delta'}}{a}.
\]
\end{remark}

\begin{example}[Exemples]
\begin{enumerate}
\item Résoudre l'équation $3x^2-5x+1=0$ : calculons $\Delta=25-12=13$. L'équation a deux solutions qui sont les racines de $3x^2-5x+1$ :
\[
\alpha=\frac{5+\sqrt{13}}{6}\quad\text{et}\quad\beta=\frac{5-\sqrt{13}}{6}.
\]
L'ensemble des solutions est $\symcal{S}=\{\alpha,\beta\}$.
\item Factoriser si possible $-2x^2+3x+2$ : on calcule $\Delta=9+16=5^2$. Il existe deux racines qui sont $2$ et $-1/2$ ; d'où
\[
\forall x\in\R,\;-2x^2+3x+2=-2(x-2)\left(x+\frac12\right).
\]
\item Factoriser si possible $A(x)=2x^2-8$ : utilisons  l'identité remarquable 
\[
a^2-b^2=(a+b)(a-b).
\]
\[\forall x\in\R,\;2x^2-8=2(x^2-4)=2(x-2)(x+2).\]
Dans un cas de ce genre, il est bien sûr inutile de calculer $\Delta$. En revanche, l'existence de deux racines distinctes nous prouve que $\Delta>0$.
\item Résoudre l'équation $3x^2+2=0$. La fonction polynôme $x\mapsto3x^2+2$ admet un minimum strictement positif égal à $2$. Il n'y a aucune solution, l'ensemble des solutions est $\symcal{S}=\emptyset$.
\item Factoriser $9x^2-6x+1=0$. On pourrait envisager de calculer $\Delta$, mais en observant l'expression, on reconnait un membre d'identité remarquable :
\[\forall x\in\R,\;9x^2-6x+1=(3x-1)^2.\]
\item Résoudre $x^2+2x=0$. La factorisation est immédiate :
\[x^2+2x=0\iff x(x+2)=0\iff x=0\text{ ou }x+2=0.\]
L'ensemble des solutions de cette équation est donc $\symcal{S}=\{0,-2\}$.
\end{enumerate}
\end{example}
%
%

\subsection{Problème de signe}
Nous rappelons ici que la fonction \emph{signe} est définie sur $\R$ comme suit :
\[\forall x \in \R,\; \sgn(x)=\begin{cases}
-1 &\text{si $x<0$,}\\
0 &\text{si $x=0$,}\\
1 &\text{si $x>0$.}
\end{cases}
\]
Cette définition est beaucoup plus pertinente que l'usage des symboles ${+}$ et ${-}$, ne serait-ce que parce qu'elle rend exacte et claire une formulation comme \emph{le signe du produit est le produit des signes}.
\begin{thm}
Soit $P$, la fonction polynôme de degré 2 définie sur $\R$ par $P(x)=ax^2+bx+c$. 
\begin{itemize}
\item  Si $\Delta<0$,  le signe de $P(x)$ est égal à celui de $a$ en tout $x$ réel.
\item  Si $\Delta=0$, le signe de $P(x)$ est égal au signe de $a$ en tout $x$ réel différent de $-b/(2a)$ et s'annule en $-b/(2a)$.
\item  Si $\Delta>0$, le signe de $P(x)$ est égal au signe de $-a$ en tout $x$ intérieur à l'intervalle des racines, au signe de $a$ en tout $x$ extérieur à l'intervalle des racines et s'annule aux racines. 
\end{itemize}
\label{thsgn}
\end{thm}

Ces résultats s'interprètent sur la représentation graphique $\symcal{C}_P$ de $P$, selon les signes de $\Delta$ et $a$ (voir figure \ref{figracines}).
\afterpage{\clearpage\begin{figure}[H]\centering
\AfourCommand{%
\noindent\hfill\paraone\hfill\parathree\hfill\parafive\hfill\null
\par
\noindent\hfill\paratwo\hfill\parafour\hfill\parasix\hfill\null
}
\BigTabletCommand{%
\noindent\hfill\paraone\hfill\parathree\hfill\parafive\hfill\null
\par
\noindent\hfill\paratwo\hfill\parafour\hfill\parasix\hfill\null
}
\TabletCommand{%
\noindent\hfill\paraone\hfill\parathree\hfill\parafive\hfill\null
\par
\noindent\hfill\paratwo\hfill\parafour\hfill\parasix\hfill\null
}
\SmartphoneCommand{%
\noindent\hfill\paraone\hfill\paratwo\hfill\null
\par
\noindent\hfill\parathree\hfill\parafour\hfill\null
\par
\noindent\hfill\parafive\hfill\parasix\hfill\null
}
\SmallTabletCommand{%
\vspace*{1.5cm}\par
\noindent\hfill\paraone\hfill\paratwo\hfill\null
\par
\noindent\hfill\parathree\hfill\parafour\hfill\null
\par
\noindent\hfill\parafive\hfill\parasix\hfill\null
}
\figcaption{}\label{figracines}
\end{figure}
}

\begin{proof}
Examinons successivement les trois cas du théorème.
\begin{itemize}
\item  Si $\Delta<0$, le résultat provient d'une remarque déjà faite : dans ce cas, la fonction $P$ possède un minimum strictement positif si $a>0$ ou un maximum strictement négatif si $a<0$. 
\item  Si $\Delta=0$, on a 
\[
P(x)=a\left(x+\frac{b}{2a}\right)^2.
\]
% La fonction $P$ admet donc $0$ pour minimum lorsque $a>0$ ou pour maximum lorsque $a>0$. Dans les deux cas cet extrémum n'est atteint qu'une fois en $-b/(2a)$.
Le résultat annoncé se déduit du tableau de signes
\[\begin{array}{c|ccccc}
\hline
x&-\infty&&-b/(2a)&&\infty\\
\hline
\sgn\bigl(\bigl(x+\frac{b}{2a}\bigr)^2\bigr)&&1&0&1&\\
\hline
\sgn(P(x))&&\sgn(a)&0&\sgn(a)&\\
\hline
\end{array}\]
\item  Si $\Delta>0$, on constate, en se référant à l'expression des racines $\alpha$ et $\beta$ dans théorème \ref{thfactor}, que $\alpha$ est la plus grande des deux racines si $a>0$ et la plus petite si $a<0$. En posant
\begin{align*}\alpha'=\min(\alpha,\beta)&\text{ et }\beta'=\max(\alpha,\beta),\\
\intertext{on obtient}
P(x)=a(x-\alpha')(x-\beta'),\quad&\text{ avec $\alpha'<\beta'$}.
\end{align*}
On établit alors le tableau de signes
{\SmartphoneCommand{\small}
\[\begin{array}{c|ccccccc}
\hline
x&-\infty&&\alpha'&&\beta'&&\infty\\
\hline
\sgn(x-\alpha')&&-1&0&1&1&1&\\
\hline
\sgn(x-\beta')&&-1&-1&-1&0&1&\\
\hline
\sgn(P(x))&&\sgn(a)&0&\sgn(-a)&0&\sgn(a)&\\
\hline
\end{array}
\]}
Le résultat annoncé s'en déduit.\qedhere
\end{itemize}
\end{proof}

\begin{example}[Exemples]
\begin{enumerate}
\item Résoudre $x^2-x-6≤0$ : le discriminant vaut $\Delta=1+24=25$. Il existe deux racines qui sont $-2$ et $3$. On veut que $x^2-x-6$ soit négatif donc nul ou du signe de~$-a$ : cela se produit aux racines ou à l'intérieur de l'intervalle des racines. L'ensemble des solutions est $\symcal{S}=[-2,3]$.

On aurait pu également, une fois les racines calculées, utiliser la forme factorisée du polynôme $P(x)=x^2-x-6$. Ainsi $P(x)=(x+2)(x-3)$, d'où le tableau de signes :
\[\begin{array}{c|ccccccc}
\hline
x&-\infty&&\quad-2\quad&&\quad3\quad&&\infty\\
\hline
\sgn(x+2)&&-1&0&1&1&1&\\
\hline
\sgn(x-3)&&-1&-1&-1&0&1&\\
\hline
\sgn(P(x))&&1&0&-1&0&1&\\
\hline
\end{array}
\]
On retrouve la solution donnée par la première méthode.

\item Résoudre $(2x-6)^2>0$ : ce carré est positif pour tout $x\in\R$ et s'annule si et seulement si $x=3$. Par conséquent, l'ensemble des solutions de cette inéquation est $\symcal{S}=\R\smallsetminus\{3\}$. Si l'on avait eu la très mauvaise idée de développer pour calculer le discriminant, on aurait trouvé $\Delta=0$ et une racine double égale à $3$.
\item Résoudre $3x^2-4x-1≥0$ : c'est l'occasion d'utiliser le discriminant réduit.

On trouve $\Delta'=4+3=7$.
Il existe deux racines :
\[
\alpha=\frac{2+\sqrt{7}}{3}\quad\text{et}\quad\beta=\frac{2-\sqrt{7}}{3}.
\]
On veut que $3x^2-4x-1$ soit positif, donc nul ou du signe de $a$ : cela se produit aux racines ou à l'extérieur de l'intervalle des racines. L'ensemble des solutions est donc
\[
\symcal{S}=\left]-\infty, \beta\right] \cup \left[\alpha, \infty\right[.
\]
\item Résoudre $5x-2x^2>0$. La factorisation se fait à vue : 
\[
5x-2x^2=-2x\left(x-\frac52\right),
\]
 il y a donc deux racines : $\alpha=0$ et $\beta=5/2$.

On peut ensuite soit faire un tableau de signes, soit utiliser le théorème \ref{thsgn}, ce que nous ferons ici.
On veut que $5x-2x^2$ soit strictement positif,  donc du signe de $-a$ : cela se produit si et seulement si $x$ est à l'intérieur de l'intervalle des racines. L'ensemble des solutions est donc 
$\symcal{S}=\left]0,\, 5/2\right[$.
\item Résoudre $x^2-x+1>0$ : on trouve $\Delta=-3$. Il n'y a pas de racines, donc $x^2-x+1$ admet pour signe $1$ pour tout $x\in\R$. Tout $x$ réel est donc solution : l'ensemble des solutions est $\symcal{S}=\R$. 
\end{enumerate}
\end{example}
\section{Somme et produit des racines}
\subsection{Expression de la somme et du produit des racines}
On suppose ici que $\Delta≥0$, c'est-à-dire que le polynôme $P$ deux racines $\alpha$ et $\beta$, confondues en une racine double au cas où $\Delta=0$. On a le théorème suivant:
\begin{thm}\label{thsomprod}%
Soit $P$, la fonction polynôme de degré 2 définie sur $\R$ par $P(x)=ax^2+bx+c$. Si le discriminant $\Delta$ de $P$ est positif, la somme et le produit  des racines de $P$ sont données par
\[
\alpha+\beta=-\frac{b}{a} \quad\text{et}\quad \alpha\beta=\frac{c}{a}\:.
\]
\end{thm}

\begin{remark}
On sait que si $\Delta=0$, on a $\alpha=\beta=-b/(2a)$. On retrouve bien 
\[\alpha+\beta=2\times\left(-\frac{b}{2a}\right)=-\frac{b}{a}\;;\]
on remarque également que, puisque $\Delta=0$, on a $b^2=4ac$ et par conséquent
\[
\alpha\beta=\left(-\frac{b}{2a}\right)^2=\frac{b^2}{4a^2}=\frac{4ac}{4a^2}=\frac{c}{a}\:.
\]
\end{remark}

\begin{proof}On utilise  la forme factorisée de $P(x)$ :
\begin{align*}
\forall x\in\R,\;P(x)&=a(x-\alpha)(x-\beta),\\
&=a(x^2-\alpha x-\beta x +\alpha\beta),\\
&=a\left(x^2-(\alpha+\beta)x+\alpha\beta\right),\\
&=ax^2-a(\alpha+\beta)x +a\alpha\beta.
\end{align*}

D'après le théorème \ref{thunicite}, on peut identifier les coefficients : $-a(\alpha+\beta)=b$ et $a\alpha\beta=c$. Le résultat annoncé en découle immédiatement.
\end{proof}
\subsubsection{Racine apparente ou connue}
On dit que $\alpha$ est une \frquote{racine apparente} (on dit également \frquote{évidente}) du polynôme $P$ lorsqu'un simple calcul à vue permet de montrer que $P(\alpha)=0$. On parle de racine connue lorsqu'un calcul antérieur a permis d'établir cette égalité. Dans une telle situation, il est facile de déterminer $\beta$, l'autre racine de $P$, grâce à la somme ou au produit :
\[
 \beta=-\frac{b}{a}-\alpha \quad \text{ou (dans le cas où $\alpha\neq0$)}\quad\beta=\frac{c}{a\alpha}.
\]
\begin{example}[Exemples]
\begin{enumerate}
\item Factoriser le polynôme $P$, défini sur $\R$ par $P(x)=3x^2+9x+6$. Un calcul immédiat montre que $P(-1)=0$ : le réel $-1$ est donc racine de $P$, tandis que l'autre racine $\beta$ vérifie~\mbox{$-1\times\beta=6/3=2$}. D'où $\beta=-2$ et la factorisation : \[P(x)=3(x+1)(x+2).\]
\item On pose $Q(x)=2x^2-13x+15$. Calculer $Q(5)$ et résoudre l'équation $Q(x)=0$. On trouve $Q(5)=0$ : le réel $5$ est donc racine de $Q$. L'autre racine vérifie $5+\beta=6{,}5$, d'où $\beta=1{,}5$ : l'ensemble des solutions de l'équation proposée est \[\symcal{S}=\left\{5, \frac32\right\}\,.\]

\end{enumerate}
\end{example}

\subsection{Trouver deux nombres connaissant leur somme et leur produit}

\begin{thm}
Deux réels $s$ et $p$ étant donnés, il existe deux nombres réels $\alpha$ et $\beta$ dont la somme vaut $s$ et le produit $p$ si et seulement si le nombre $\Delta=s^2-4p$ est positif. Dans ce cas, les  nombres $\alpha$ et $\beta$ sont les racines de la fonction polynôme de degré $2$ définie par  $P(x)=x^2-sx+p$, de discriminant $\Delta$.
\end{thm}
\begin{proof}
\begin{itemize}
\item Tout d'abord, si $s^2-4p≥0$, la fonction  polynôme $P$, de discriminant $\Delta=s^2-4p$ admet deux racines dont la somme est $s$ et le produit $p$ d'après le théorème \ref{thsomprod}.

\item Réciproquement, s'il existe deux nombres $\alpha$ et $\beta$ tels que 
\[ 
\alpha+\beta=s \quad\text{et}\quad \alpha\beta=p,
\]
on a, d'après la première équation, $\beta=s-\alpha$, puis en substituant dans la seconde 
\[\alpha(s-\alpha)=p\quad\text{soit}\quad\alpha^2-s\alpha+p=0,\]
c'est-à-dire $P(\alpha)=0$.

Il est donc nécessaire que $\alpha$ soit une racine de $P$, ce qui entraîne que le discriminant $s^2-4p$ soit positif. Enfin, la relation $\beta=s-\alpha$ entraîne que $\beta$ est alors la seconde racine de $P$.\qedhere
\end{itemize}
\end{proof}
\begin{example}[Exemples]
\begin{enumerate}
\item Trouver deux nombres dont la somme vaut $1$ et le produit $-12$. Ces deux nombres doivent être racines de $x^2+x-12$ dont le discriminant est $49$ et qui admet $-3$ et $4$ pour racines. Les deux nombres cherchés sont donc $-3$ et $4$.
\item Trouver deux nombres dont la somme vaut $6$ et le produit $9$. On forme le polynôme $x^2-6x+9$, égal à $(x-3)^2$, dont le discriminant est nul et qui admet $3$ pour racine double : les nombres $\alpha$ et $\beta$ dont le produit vaut $9$ et la somme $6$ sont tous deux égaux à $3$.
\item Trouver deux nombres dont la somme vaut $s=5$ et le produit $p=8$. On calcule $s^2-4p=-7$ : le problème n'a pas de solution (tout comme le polynôme $x^2-5x+8$ n'admet pas de racines). 
\end{enumerate}
\end{example}
\endinput


%\tgosetup{ColorTheme=Sapin}
\part{Géométrie}

\tgotitle{Droite et cercle d'Euler}
\tgoshorttoc
\section{Cercle circonscrit à un triangle}
\subsection{Un résultat préliminaire}
On sait depuis le collège  que la droite qui joint les milieux de deux côtés d'un triangle est parallèle au troisième côté. Nous précisons ce résultat à l'aide d'un théorème qui nous sera utile par la suite (voir figure \ref{figmilieux}).

\begin{figure}[ht]
\centering
\begin{tikzpicture}
\coordinate[label=left:$B$](B) at (0,0);
\coordinate[label=right:$C$](C) at (4,0);
\coordinate (T) at ($(B)!1!38:(C)$);	
\coordinate (S) at ($(C)!1!105:(B)$);	
\coordinate[label=above:$A$](A) at at (intersection of C--S and B--T);
\draw (A)--(B);
\draw (A)--(C);
\coordinate [label=right:$B'$] (B') at ($(A)!0.5!(C)$);		
\coordinate [label=left:$C'$] (C') at ($(A)!0.5!(B)$);
\draw[>=stealth,->,semithick](B)--(C);
\draw[>=stealth,->,semithick](C')--(B');
\foreach \point in {A,B,C,C',B'}
\draw[black,fill=white](\point) circle (1.2pt);
\end{tikzpicture}
\figcaption{}\label{figmilieux}
\end{figure}

\begin{thm}
Soit un triangle $ABC$ et soit $B'$ et $C'$ les milieux respectifs de $[AC]$ et $[AB]$. Alors
les droites $(BC)$ et $(C'B')$ sont parallèles et  $\overrightarrow{BC}=2\overrightarrow{C'B'}$.
\label{thmilieux}
\end{thm}

\begin{proof}
La relation de Chasles permet d'écrire
\begin{align*}
\overrightarrow{BC}&=\overrightarrow{BA}+\overrightarrow{AC},\\
\intertext{puis, en utilisant une caractérisation vectorielle du milieu,}
\overrightarrow{BC}&=2\overrightarrow{C'A}+2\overrightarrow{AB'},\\
\overrightarrow{BC}&=2\left(\overrightarrow{C'A}+\overrightarrow{AB'}\right),\\
\overrightarrow{BC}&=2\overrightarrow{C'B'}.\qedhere
\end{align*}
\end{proof}

\subsection{Médiatrices des côtés d'un triangle}
%%%%% Version 1
 La médiatrice d'un segment est par définition la droite perpendiculaire à ce segment en son milieu. Nous admettrons que la médiatrice d'un segment est aussi l'ensemble des points qui sont équidistants des extrémités de ce segment (voir figure \ref{figcerccir}). Nous formulons alors le théorème suivant :
%%%%%%
%%%%%%
%%%%%%
%%%%% Version 2
%La médiatrice d'un segment est par définition l'ensemble des points équidistants des extrémités de ce segment. Il est clair que le milieu du segment appartient à la médiatrice. On a même :
% %
%\begin{thm}
%La médiatrice d'un segment est la droite perpendiculaire à ce segment en son milieu.
%\end{thm}\label{caracmed}
%
%\begin{proof}
%Soit un segment $[AB]$ et $I$ son milieu. Notons $\symcal{E}$ la médiatrice de $[AB]$ et $\symcal{D}$ la droite perpendiculaire à $[AB]$ en $I$. Montrons que $\symcal{E}=\symcal{D}$ par double inclusion.
%Soit un point $M$ appartenant à la médiatrice $\symcal{E}$ de $[AB]$. On a $IA=IB$ et $MA=MB$ par définition de la médiatrice. Les triangles $MIA$ et $MIB$ sont donc superposables. Si $\alpha$ et $\beta$ sont les mesures respectives de $\widehat{MIA}$ et  $\widehat{MIB}$, on en déduit que $\alpha=\beta$. L'angle $\widehat{AIB}$ et plat de mesure $\alpha+\beta$ : 
%\[\left.\begin{aligned} \alpha=\beta&\\\alpha+\beta=180\degre\end{aligned}\right\}\implies \alpha=\beta=90\degre.\]
%Le point $M$ appartient donc à la droite $\symcal{D}$ : $\symcal{E}\subset\symcal{D}$.
%
%Réciproquement, supposons que $M\in\symcal{D}$. Considérons la réflexion $s$ d'axe $\symcal{D}$ : on a $s(A)=B$ et $s(M)=M$ (car l'ensemble des points invariants de $s$ est la droite $\symcal{D}$). Par conservation des distances, on a $MA=s(M)s(A)=MB$. Donc $M$ appartient à $\symcal{E}$ : $\symcal{E}\subset\symcal{D}$.
%
%On a donc bien $\symcal{E}=\symcal{D}$.
%\end{proof}
%
%

\begin{thm}
Les médiatrices des trois côtés d'un triangle sont concourantes en un point équidistant des trois sommets et centre de l'unique cercle passant par ces derniers.
\end{thm}

\vspace{1ex}

\begin{figure}[ht]
\centering
\begin{tikzpicture}
\coordinate[label=left:$B$](B) at (0,0);
\coordinate[label=right:$C$](C) at (4,0);
\coordinate (T) at ($(B)!1!38:(C)$);	
\coordinate (S) at ($(C)!1!105:(B)$);	
\coordinate[label=above:$A$](A) at at (intersection of C--S and B--T);
\draw (A)--(B)--(C)--cycle;
\coordinate [label=below:$A'$] (A') at ($(B)!0.5!(C)$);
\coordinate [label=right:$B'$] (B') at ($(A)!0.5!(C)$);		
\coordinate [label=left:$C'$] (C') at ($(A)!0.5!(B)$);
\coordinate (X) at ($(C')!1!90:(B)$);	
\coordinate (Y) at ($(B')!1!90:(C)$);	
\coordinate [label=above:$O${\rule[-1mm]{0mm}{1mm}}](O) at (intersection of C'--X and B'--Y);
\draw(C')--(O);
\draw (B')--(O);
\draw[dotted,semithick](A')--(O);
\node [draw] at (O) [circle through={(C)}] {};
\draw[gray](O)--(A);
\draw[gray](O)--(B);
\draw[gray](O)--(C);
\foreach \point in {A,B,C,C',A',B',O}
\draw[black,fill=white](\point) circle (1.2pt);	
\end{tikzpicture}
\figcaption{}\label{figcerccir}
\end{figure}



\begin{proof}
Reprenons la figure \ref{figmilieux} et nommons $A'$ le milieu de $[BC]$. Les médatrices des côtés $[AB]$ et $[AC]$ sont sécantes en un point $O$. (On rappelle qu'un triangle est la donnée de trois points non alignés, autrement dit, un triangle aplati ne peut être considéré comme un triangle : cela assure que les deux médiatrices sont bien sécantes.) Le point $O$ se trouve à la fois sur la médiatrice de $[AB]$ et sur la médiatrice de $[AC]$ : il est donc à la fois équidistant de $A$ et $B$ ($OA=OB$) et équidistant de $A$ et $C$ ($OA=OC$). On a donc finalement $OB=OC$, ce qui assure que la médiatrice de $[BC]$ passe par $O$. Les trois médiatrices sont donc concourantes en $O$, point équidistant des trois sommets du triangle et par conséquent centre d'un cercle passant par ces sommets. Pour établir l'unicité d'un tel cercle,  il suffit de remarquer que le centre d'un cercle passant par les sommets est nécessairement équidistant des sommets donc point de concours des médiatrices.
\end{proof}

Le cercle en question est le \emph{cercle circonscrit} au triangle. 

\subsection{Un cas particulier}
Le cas du cercle circonscrit au triangle rectangle est bien connu (voir figure \ref{figtrec}).

\begin{figure}[ht]
\centering
\begin{tikzpicture}
\coordinate[label=left:$B$](B) at (0,0);
\coordinate[label=right:$C$](C) at (4,0);
\coordinate (T) at ($(B)!1!30:(C)$);	
\coordinate (S) at ($(C)!1!120:(B)$);	
\coordinate[label=above:$A$](A) at at (intersection of C--S and B--T);
\draw (A)--(B)--(C)--cycle;
\coordinate [label=below:$A'$] (A') at ($(B)!0.5!(C)$);
\coordinate [label=left:$B'${\rule{1.3mm}{0mm}}] (B') at ($(A)!0.5!(C)$);		
\coordinate [label=right:{\rule{1.3mm}{0mm}}$C'$] (C') at ($(A)!0.5!(B)$);
\coordinate (X) at ($(C')!1!90:(B)$);	
\coordinate (Y) at ($(B')!1!90:(C)$);	
\coordinate(O) at (intersection of C'--X and B'--Y);
\draw(C')--(O);
\draw (B')--(O);
\draw[dotted,semithick](A')--(O);
\node [draw] at (O) [circle through={(C)}] {};
\draw[gray](O)--(A);
\draw[gray](O)--(B);
\draw[gray](O)--(C);
\foreach \point in {A,B,C,C',A',B',O}
\draw[black,fill=white](\point) circle (1.2pt);
\end{tikzpicture}
\figcaption{}\label{figtrec}
\end{figure}


\begin{thm}
Si un triangle est rectangle, alors le centre de son cercle circonscrit est le milieu de l'hypoténuse qui est donc un diamètre du cercle. Réciproquement si le cercle circonscrit à un triangle admet l'un des côtés comme diamètre, alors ce triangle est rectangle et admet le côté en question pour hypoténuse.
\label{diamtrrec}
\end{thm}

\begin{proof}
Supposons le triangle $ABC$ rectangle en $A$. La droite $(AB)$ est alors perpendiculaire à la droite $(AC)$.
D'après le théorème \ref{thmilieux} on sait que $\overrightarrow{AB}=2\overrightarrow{B'A'}$ et en particulier que les droites $(AB)$ et $(B'A')$ sont parallèles. On en déduit que les droites $(B'A')$ et $(AC)$ sont perpendiculaires:
\[\left.\begin{aligned} &(AB)\parallel(A'B')\\
\text{et}&\\
&(AB)\perp(AC)\end{aligned}\right\}\implies (A'B')\perp(AC).
\]

La droite $(B'A')$, perpendiculaire au segment $[AC]$ en son milieu, en est la médiatrice. De même, $(C'A')$ est la médiatrice du segment $[AB]$. Le point $A'$, milieu de $[BC]$, est commun à ces deux médiatrices : il est donc le centre du cercle circonscrit au triangle.



Réciproquement, supposons que le milieu $A'$ de $[BC]$ soit le centre du cercle circonscrit. Ce point est alors équidistant de $A$ et $C$, donc est un point de la médiatrice du segment $[AC]$; il en est de même pour $B'$, milieu de ce segment. La droite $(A'B')$ est donc la médiatrice du segment $[AC]$ et lui est  perpendiculaire. Cette droite est aussi parallèle à $(AB)$, en vertu du théorème \ref{thmilieux}. De la même façon que précédemment, on en déduit que les droites $(AB)$ et $(AC)$ sont perpendiculaires. Le triangle $ABC$ est donc rectangle en $A$.
\end{proof}

\begin{remark} La situation représentée sur la figure \ref{figcerccir} est celle d'un triangle acutangle. On comprend assez rapidement (mais ceci ne constitue pas une démonstration) que le centre du cercle circonscrit est intérieur au triangle si les trois angles sont aigus et extérieur dans le cas où l'un des angles est obtus (voir figure \ref{figtrobtu}). Le cas du triangle rectangle constitue en quelque sorte une situation limite…
\end{remark}


\begin{figure}[ht]
\centering
\begin{tikzpicture}
\coordinate[label=left:$B$](B) at (0,0);
\coordinate[label=right:$C$](C) at (4,0);
\coordinate (T) at ($(B)!1!25:(C)$);	
\coordinate (S) at ($(C)!1!140:(B)$);	
\coordinate[label=above:$A$](A) at at (intersection of C--S and B--T);
\draw (A)--(B)--(C)--cycle;
\coordinate [label=above:$A'$] (A') at ($(B)!0.5!(C)$);
\coordinate [label=left:$B'${\rule{1.3mm}{0mm}}] (B') at ($(A)!0.5!(C)$);		
\coordinate [label=right:{\rule{1.3mm}{0mm}}$C'$] (C') at ($(A)!0.5!(B)$);
\coordinate (X) at ($(C')!1!90:(B)$);	
\coordinate (Y) at ($(B')!1!90:(C)$);	
\coordinate [label=below:$O${\rule[-1mm]{0mm}{1mm}}](O) at (intersection of C'--X and B'--Y);
\draw(C')--(O);
\draw (B')--(O);
\draw[dotted,semithick](A')--(O);
\node [draw] at (O) [circle through={(C)}] {};
\draw[gray](O)--(A);
\draw[gray](O)--(B);
\draw[gray](O)--(C);
\foreach \point in {A,B,C,C',A',B',O}
\draw[black,fill=white](\point) circle (1.2pt);	
\end{tikzpicture}
\figcaption{}\label{figtrobtu}
\end{figure}



\section{Centre de gravité d'un triangle}
Il est connu que les médianes d'un triangle sont concourantes en un point situé aux deux tiers de chacune d'elles à partir du sommet. C'est à cette propriété que nous allons maintenant nous intéresser, sous un angle vectoriel (voir figure \ref{figmedianes}).


\begin{thm}\label{thgravit}%
Soit un triangle $ABC$ et soit $A'$, $B'$ et $C'$ les milieux respectifs des côtés $[BC]$, $[AC]$ et $[AB]$.
Les médianes de ce triangle sont concourantes en un point $G$, nommé centre de gravité du triangle et qui est l'unique point du plan à vérifier les relations vectorielles suivantes :
\begin{gather}
\overrightarrow{GA}+\overrightarrow{GB}+\overrightarrow{GC}=\vec{0}\,,\label{eqisobar}\\
\overrightarrow{AG}=\frac23\overrightarrow{AA'}\,,\quad\overrightarrow{BG}=\frac23\overrightarrow{BB'}\quad\text{et}\quad\overrightarrow{CG}=\frac23\overrightarrow{CC'}\,.\label{eqGmediane}
\end{gather}%
\end{thm}

\begin{figure}[ht]
\centering
\begin{tikzpicture}
\coordinate[label=below:$B$](B) at (0,0);
\coordinate[label=below:$C$](C) at (4,0);
\coordinate (T) at ($(B)!1!36:(C)$);	
\coordinate (S) at ($(C)!1!107:(B)$);	
\coordinate[label=above:$A$](A) at at (intersection of C--S and B--T);
\draw (A)--(B)--(C)--cycle;
\coordinate [label=below:$A'$] (A') at ($(B)!0.5!(C)$);
\coordinate [label=right:$B'$](B') at ($(A)!0.5!(C)$);		
\coordinate[label=left:$C'$](C') at ($(A)!0.5!(B)$);
\coordinate (X) at ($(C')!1!90:(B)$);	
\coordinate (Y) at ($(B')!1!90:(C)$);	
\coordinate (O) at (intersection of C'--X and B'--Y);
\coordinate[label=above:$G${\rule[-1mm]{0mm}{1mm}}] (G) at ($(A')!0.333!(A)$);
\coordinate (H) at ($(O)!3!(G)$);
\draw (A)--(A');
\draw[dotted,semithick] (B)--(B');
\draw[dotted,semithick] (C)--(C');
\foreach \point in {A,B,C,A',G,C',B'}
\draw[black,fill=white](\point) circle (1.2pt);
\end{tikzpicture}
\figcaption{}\label{figmedianes}
\end{figure}
	


\begin{proof}
Pour tout point $M$ du plan, on considère le vecteur 
\[
\overrightarrow{v(M)}=\overrightarrow{MA}+\overrightarrow{MB}+\overrightarrow{MC}.
\]

 Nous nous proposons de prouver qu'il existe un point $M$ et un seul pour lequel ce vecteur est nul : cet unique point $M$ sera le point $G$ que nous cherchons. À cet effet nous allons transformer le vecteur $\overrightarrow{v(M)}$, en faisant notamment intervenir un des milieux, ici $A'$, grâce à la relation de Chasles. Nous rappelons que ce milieu est caractérisé par $\overrightarrow{A'B}+\overrightarrow{A'C}=\vec{0}$.
%\DinoXiPhoneCommand{%
\begin{align*}
\overrightarrow{v(M)}&=
(\overrightarrow{MA'}+\overrightarrow{A'A})+
(\overrightarrow{MA'}+\overrightarrow{A'B})+
(\overrightarrow{MA'}+\overrightarrow{A'C}),\\
&=
3\overrightarrow{MA'}+\overrightarrow{A'A}+\underbrace{\overrightarrow{A'B}+\overrightarrow{A'C}}_{\vec{0}}.
\end{align*}

Nous transformons alors le vecteur $\overrightarrow{MA'}$ en faisant intervenir le point $A$ grâce à la relation de Chasles :
\begin{align*}
\overrightarrow{v(M)}&=3(\overrightarrow{MA}+\overrightarrow{AA'})+\overrightarrow{A'A},\\
\intertext{puis, en développant et en remarquant que $\overrightarrow{A'A}=-\overrightarrow{AA'}$,}
\overrightarrow{v(M)}&=3\overrightarrow{MA}+2\overrightarrow{AA'}.
\end{align*}%}
%\DinoiPhoneCommand{%
%\begin{multline*}
%\overrightarrow{MA}+\overrightarrow{MB}+\overrightarrow{MC}=
%(\overrightarrow{MA'}+\overrightarrow{A'A})\\
%+(\overrightarrow{MA'}+\overrightarrow{A'B})+
%(\overrightarrow{MA'}+\overrightarrow{A'C}),\end{multline*}
%soit
%\begin{align*}
%\overrightarrow{MA}+\overrightarrow{MB}+\overrightarrow{MC}&=
%3\overrightarrow{MA'}+\overrightarrow{A'A}+\underbrace{\overrightarrow{A'B}+\overrightarrow{A'C}}_{\vec{0}},\\
%&=3\overrightarrow{MA}+3\overrightarrow{AA'}+\overrightarrow{A'A},\\
%&=3\overrightarrow{MA}+2\overrightarrow{AA'}.
%\end{align*}}
En tenant compte de $\overrightarrow{v(M)}=\overrightarrow{MA}+\overrightarrow{MB}+\overrightarrow{MC}$,
  \begin{align}\overrightarrow{MA}+\overrightarrow{MB}+\overrightarrow{MC}=\vec{0}
 &\iff 3\overrightarrow{MA}+2\overrightarrow{AA'}=\vec{0},\notag\\
 &\iff 3\overrightarrow{AM}=2\overrightarrow{AA'},\notag\\
  &\iff\overrightarrow{AM}=\dfrac23\overrightarrow{AA'}.\label{caraccg}\end{align}

La relation \eqref{caraccg} définit l'unique point $M$ vérifiant $\overrightarrow{MA}+\overrightarrow{MB}+\overrightarrow{MC}=\vec{0}$. Ce point est par définition le centre de gravité du triangle $ABC$ et on le note traditionnellement $G$. Le calcul vectoriel ci-dessus aurait tout aussi bien pu être mené en utilisant $B'$ ou $C'$ plutôt que $A'$. On en déduit les trois relations annoncées en \eqref{eqGmediane}.
\end{proof}

\section{Orthocentre et droite d'Euler}
\subsection{Orthocentre d'un triangle}
\begin{thm}
Les hauteurs issues des trois sommets de tout triangle sont concourantes en un point $H$ nommé orthocentre de ce triangle.
\end{thm}

\begin{figure}[ht]
\centering
\begin{tikzpicture}
\coordinate[label=below:$B$](B) at (0,0);
\coordinate[label=below:$C$](C) at (5,0);
\coordinate (T) at ($(B)!1!36:(C)$);	
\coordinate (S) at ($(C)!1!107:(B)$);	
\coordinate[label=above:$A$](A) at at (intersection of C--S and B--T);
\draw (A)--(B)--(C)--cycle;
\coordinate [label=below:$A'$] (A') at ($(B)!0.5!(C)$);
\coordinate  (B') at ($(A)!0.5!(C)$);		
\coordinate (C') at ($(A)!0.5!(B)$);
\coordinate (X) at ($(C')!1!90:(B)$);	
\coordinate (Y) at ($(B')!1!90:(C)$);	
\coordinate [label=above:$O$](O) at (intersection of C'--X and B'--Y);
\coordinate[label=above:$G${\rule[-1mm]{0mm}{1mm}}] (G) at ($(A')!0.333!(A)$);
\coordinate[label=below:$H$] (H) at ($(O)!3!(G)$);
\draw (A')--(O)--(H)--(A);
\draw (A)--(A');
\foreach \point in {A,B,C,A',O,G,H}
\draw[black,fill=white](\point) circle (1.2pt);
\end{tikzpicture}
\figcaption{}\label{figeuler}
\end{figure}
	

\begin{proof}
Nous en restons aux notations utilisées précédemment.
Si le triangle $ABC$ est équilatéral, la hauteur,  la médiane et la médiatrice relatives à chaque côté sont confondues, les points $O$, centre du cercle circonscrit, et $G$, centre de gravité de $ABC$ le sont également. Ce point unique est évidemment aussi point de concours des hauteurs. Nous supposerons désormais que le triangle $ABC$ n'est pas équilatéral. 

La démonstration repose alors sur la remarque qui suit.
\begin{remark}[Remarque préliminaire]
La hauteur issue de $A$ et la médiatrice de $[BC]$ sont deux droites parallèles, car perpendiculaires à une même troisième.
L'observation de la figure \ref{figeuler} nous incite alors à prolonger la droite $(OG)$ et à fabriquer ainsi une configuration de Thales. Comme nous savons, de par la position de $G$ aux deux tiers de la médiane $[AA']$, que $\overrightarrow{GA}=-2\overrightarrow{GA'}$ nous allons introduire un point $H$ défini  par $\overrightarrow{GH}=-2\overrightarrow{GO}$ pour ensuite ensuite établir (en exploitant de façon vectorielle notre configuration de Thales) que les vecteurs $ \overrightarrow{AH}$ et  $\overrightarrow{OA'}$ sont colinéaires, ce qui établira, sauf cas particulier, que la médiatrice $(OA')$ et la droite $(AH)$ sont parallèles donc que $(AH)$ est la hauteur issue de $A$.
\end{remark}

Soit donc le point $H$ défini par $\overrightarrow{GH}=-2\overrightarrow{GO}$.
De l'égalité \eqref{eqGmediane}, nous déduisons que $3\overrightarrow{AG}=2\overrightarrow{AA'}$, puis que $\overrightarrow{AG}=2\overrightarrow{GA'}$. Par conséquent :
\[\begin{split} \overrightarrow{AH}&=\overrightarrow{AG}+\overrightarrow{GH},\\
&=2\overrightarrow{GA'}+2\overrightarrow{OG},\\
&=2(\overrightarrow{OG}+\overrightarrow{GA'}),\\
&=2\overrightarrow{OA'}.
\end{split}
\]
Les vecteurs $\overrightarrow{OA'}$ et $\overrightarrow{AH}$ sont donc colinéaires. 

\begin{alert}
Si le triangle $ABC$ est rectangle, les points $O$ et $A'$ (et par conséquent les points $A$ et $H$) sont confondus. On ne peut parler des droites $(OA')$ et $(AH)$ : ce cas sera examiné plus tard.
\end{alert}
Supposons provisoirement que le triangle $ABC$ \emph{n'est pas} rectangle : le point $O$, centre du cercle circonscrit n'est pas confondu avec $A'$ (cf. théorème \ref{diamtrrec}) ou encore $\overrightarrow{OA'}\neq\vec{0}$. On peut dire alors que la droite $(AH)$ est parallèle à la droite $(OA')$, elle même perpendiculaire au côté $[BC]$, en tant que médiatrice. La droite $(AH)$ est donc perpendiculaire à $[BC]$ : c'est la hauteur issue de $A$.


Le raisonnement qui vient d'être fait tient toujours si l'on remplace les points $A$ et $A'$ par $B$ et $B'$ ou par $C$ et $C'$. Le point $H$ quant à lui est fixe (il ne dépend que de $O$ et $G$) : il appartient donc à chacune des trois hauteurs. On a donc démontré que les trois hauteurs sont concourantes en $H$ : ce point $H$ est \emph{l'orthocentre} du triangle $ABC$.
\begin{alert}
Si maintenant le triangle est rectangle, par exemple en $A$ : les hauteurs issues de $B$ et $C$ sont respectivement $(BA)$ et $(CA)$ : les hauteurs sont concourantes en $A$. L'orthocentre $H$ est alors confondu avec $A$ de même que le centre du cercle circonscrit  $O$ est confondu avec $A'$. La relation $\overrightarrow{GH} = -2\overrightarrow{GO}$ est toujours vérifiée.\qedhere
\end{alert}
\end{proof}
\subsection{Droite d'Euler}



Nous avons en fait montré un peu plus que ce que nous avions annoncé. 
La relation vectorielle liant $O$, $G$ et $H$ montre que si le triangle $ABC$ n'est pas équilatéral ces trois points sont alignés. La droite support est la \emph{droite d'Euler}. Résumons-nous maintenant :
\begin{thm}
Le centre $O$ du cercle circonscrit, le centre de gravité $G$ et l'orthocentre $H$ d'un triangle $ABC$ vérifient la relation vectorielle 
\begin{equation}\overrightarrow{GH}=-2\overrightarrow{GO}\,.\label{eulerposition}\end{equation}
Ces trois points sont confondus si $ABC$ est équilatéral. Dans le cas contraire, ils sont alignés sur une droite nommée droite d'Euler. 
\label{eulerprecis}
\end{thm}

\section{Cercle d'Euler}
\subsection{Neuf points remarquables}
Le cercle d'Euler est également connu comme cercle des neuf points ou encore cercle de Feuerbach (voir figure \ref{figcercleuler}).

\begin{figure}[ht]
\centering
\begin{tikzpicture}
\coordinate[label=below:$B$](B) at (0,0);
\coordinate[label=below:$C$](C) at (5,0);
\coordinate (T) at ($(B)!1!61:(C)$);	
\coordinate (S) at ($(C)!1!130:(B)$);	
\coordinate[label=above:$A$](A) at at (intersection of C--S and B--T);
\draw (A)--(B)--(C)--cycle;
\coordinate [label=below:$A'$] (A') at ($(B)!0.5!(C)$);
\coordinate [label=right:$B'${\rule[-1mm]{0mm}{1mm}}] (B') at ($(A)!0.5!(C)$);		
\coordinate [label=left:$C'$] (C') at ($(A)!0.5!(B)$);
\coordinate (X) at ($(C')!1!90:(B)$);	
\coordinate (Y) at ($(B')!1!90:(C)$);	
\coordinate(O) at (intersection of C'--X and B'--Y);
\coordinate (G) at ($(A')!0.333!(A)$);
\coordinate[label=below:$H$] (H) at ($(O)!3!(G)$);
\draw[dashed] (A)--(H);
\draw[dashed] (B)--(H);
\draw[dashed] (C)--(H);
\coordinate[label=below:$D$](D) at at (intersection of A--H and B--C);
\coordinate[label=right:$E${\rule[-1mm]{0mm}{1mm}}](E) at at (intersection of B--H and A--C);
\coordinate[label=left:$F$](F) at at (intersection of C--H and B--A);
\coordinate [label=left:$A''$] (A'') at ($(A)!0.5!(H)$);
\coordinate [label=below:$B''$] (B'') at ($(B)!0.5!(H)$);
\coordinate [label=below:$C''$] (C'') at ($(C)!0.5!(H)$);
\draw (A')--(C')--(A'')--(C'')--cycle ;
\draw (B')--(C')--(B'')--(C'')--cycle ;
\coordinate (O') at ($(O)!1.5!(G)$);
\node [draw,dotted,semithick] at (O') [circle through={(A')}] {\rule{2.2cm}{0mm}$\symcal{C}$};
\foreach \point in {A,B,C,B',C',A',H,D,E,F,A'',B'',C''}
\draw[black,fill=white](\point) circle (1.2pt);
\end{tikzpicture}
\figcaption{}\label{figcercleuler}
\end{figure}
	

\begin{thm}
Pour tout triangle, les pieds des hauteurs, les milieux des côtés et les milieux des segments joignant l'orthocentre aux sommets sont situés sur un même cercle.
\end{thm}
\begin{remark}
Suivant les configurations, certains de ces points peuvent être confondus.
\end{remark}
\begin{proof} Soit un triangle $ABC$ d'orthocentre $H$, $A'$, $B'$ et $C'$ les milieux respectifs des côtés $[BC]$, $[CA]$ et $[AB]$, $D$, $E$ et $F$ les pieds des hauteurs issues respectivement de $A$, $B$ et $C$ et $A''$, $B''$ et $C''$ les milieux respectifs des segments $[AH]$, $[BH]$ et $[CH]$.

En vertu du théorème \ref{thmilieux}, appliqué successivement aux triangles $ABC$ et $HBC$
\XSmartphoneCommand{%
\begin{equation}
\left.\begin{aligned}
\overrightarrow{C'B'}=\frac12\overrightarrow{BC}&\\
\overrightarrow{B''C''}=\frac12\overrightarrow{BC}&	
\end{aligned}\right\}\implies \overrightarrow{C'B'}=\overrightarrow{B''C''}\text{ et }(C'B')\parallel(B''C'')\parallel(BC).\label{parallelBC}
\end{equation}}
\SmartphoneCommand{
\begin{equation}
\left.\begin{aligned}
\overrightarrow{C'B'}=\frac12\overrightarrow{BC}&\\
\overrightarrow{B''C''}=\frac12\overrightarrow{BC}&	
\end{aligned}\right\}\implies \left\{\begin{aligned}&\overrightarrow{C'B'}=\overrightarrow{B''C''}\\ \text{et}&\\(&C'B')\parallel(B''C'')\parallel(BC).\end{aligned}\right.\label{parallelBC}
\end{equation}}

De même, en appliquant ce même théorème aux triangles $BAH$ et $CAH$
\XSmartphoneCommand{%
\begin{equation}
\left.\begin{aligned}
\overrightarrow{C'B''}=\frac12\overrightarrow{AH}&\\
\overrightarrow{B'C''}=\frac12\overrightarrow{AH}&	
\end{aligned}\right\}\implies \overrightarrow{C'B''}=\overrightarrow{B'C''}\text{ et }(C'B'')\parallel(B'C'')\parallel(AH).\label{parallelAH}
\end{equation}}
\SmartphoneCommand{%
\begin{equation}
\left.\begin{aligned}
\overrightarrow{C'B''}=\frac12\overrightarrow{AH}&\\
\overrightarrow{B'C''}=\frac12\overrightarrow{AH}&	
\end{aligned}\right\}\implies\left\{\begin{aligned}&\overrightarrow{C'B''}=\overrightarrow{B'C''}\\ \text{et}&\\(&C'B'')\parallel(B'C'')\parallel(AH).\end{aligned}\right.\label{parallelAH}
\end{equation}}

À ce stade nous avons établi que $B'C'B''C''$ est un parallélogramme. Nous savons également que $(AH)$, en tant que hauteur du triangle, est perpendiculaire à $(BC)$. Grâce à cette remarque et aux relations de parallélisme avec $(BC)$ et $(AH)$ établies ci-dessus dans les implications \eqref{parallelBC} et \eqref{parallelAH}, nous pouvons affirmer que $B'C'B''C''$ est un rectangle. Ce rectangle est inscriptible dans le cercle  admettant $[B'B'']$ et $[C'C'']$ pour diamètres. 

On montre de la même façon que $A'C'A''C''$ est un rectangle inscriptible dans le même cercle  de diamètres $[C'C'']$ et $[A'A'']$. Les six points $A'$, $A''$, $B'$, $B''$, $C'$ et $C''$ appartiennent donc au cercle $\symcal{C}$ de diamètre, par exemple, $[A'A'']$.

Considérons alors le point $D$ : si le triangle $ABC$ est isocèle en $A$, il est confondu avec $A'$ et appartient donc au cercle $\symcal{C}$. Sinon $D$ et $A'$ sont distincts et le triangle $ADA'$ est rectangle en $D$ : d'après le théorème \ref{diamtrrec}, $D$ appartient au cercle $\symcal{C}$ de diamètre $[AA']$. On montre de la même façon que les points $E$ et $F$ appartiennent  à $\symcal{C}$.
\end{proof}
\begin{remark} Le cercle d'Euler est le cercle circonscrit au triangle des milieux $A'B'C'$.
\end{remark}
%
\subsection{Centre du cercle d'Euler}
Nous allons préciser la remarque précédente (voir figure \ref{centreeuler}).
\begin{thm}
Soit un triangle $ABC$ et soit $A'$, $B'$ et $C'$ les milieux respectifs de $[BC]$, $[CA]$ et $[AB]$.
Soit $O$ le centre du cercle circonscrit à $ABC$, $G$ son centre de gravité et $H$ son orthocentre :
\begin{itemize}
\item le point $O$ est l'orthocentre du triangle $A'B'C'$ ;
\item le point $G$ est le centre de gravité du triangle $A'B'C'$ ;
\item le centre $O'$ du cercle circonscrit au triangle $A'B'C'$ appartient à la droite d'Euler et vérifie la relation vectorielle
\[\overrightarrow{GO'}=-\frac12\overrightarrow{GO}.\]
\end{itemize} 
\end{thm}

\begin{figure}[ht]
\centering
\begin{tikzpicture}
\coordinate[label=below:$B$](B) at (0,0);
\coordinate[label=below:$C$](C) at (5,0);
\coordinate (T) at ($(B)!1!32:(C)$);	
\coordinate (S) at ($(C)!1!100:(B)$);	
\coordinate[label=above:$A$](A) at at (intersection of C--S and B--T);
\draw (A)--(B)--(C)--cycle;
\coordinate [label=below:$A'$] (A') at ($(B)!0.5!(C)$);
\coordinate [label=right:$B'${\rule[-1mm]{0mm}{1mm}}] (B') at ($(A)!0.5!(C)$);		
\coordinate [label=left:$C'${\rule[-1mm]{0mm}{1mm}}] (C') at ($(A)!0.5!(B)$);
\coordinate (X) at ($(C')!1!90:(B)$);	
\coordinate (Y) at ($(B')!1!90:(C)$);	
\coordinate[label=above:$O$] (O) at (intersection of C'--X and B'--Y);
\coordinate [label=above:$G$]  (G) at ($(A')!0.333!(A)$);
\coordinate[label=below:$H$] (H) at ($(O)!3!(G)$);
\draw (A)--(B)--(C)--cycle ;
\coordinate[label=above:$O'$]  (O') at ($(O)!1.5!(G)$);
\draw [dotted,semithick](A')--(B')--(C')--cycle;
\draw (O)--(H);
\foreach \point in {A,B,C,B',C',A',O,O',G,H}
\draw[black,fill=white](\point) circle (1.2pt);	
\end{tikzpicture}
\figcaption{}\label{centreeuler}
\end{figure}


%%\DinoiPadAirCommand{\pagebreak}

\begin{proof}
\begin{itemize}
\item La droite $(A'O)$, médiatrice de $[BC]$, est perpendiculaire à la droite $(BC)$, donc également à la droite $(C'B')$ parallèle à $(BC)$ d'après le théorème \ref{thmilieux}. Dans le triangle $A'B'C'$, le point $O$ appartient donc à la hauteur issue de $A'$. On montre de la même façon que $O$ appartient à la hauteur issue de $B'$. Le point $O$ est donc l'orthocentre de $A'B'C'$.
\item Grâce à la relation \eqref{eqGmediane} du théorème \ref{thgravit}, on obtient
\[\overrightarrow{GA'}=-\frac12\overrightarrow{GA},\quad \overrightarrow{GB'}=-\frac12\overrightarrow{GC}\quad\text{et}\quad\overrightarrow{GC'}=-\frac12\overrightarrow{GC}.\]
On en déduit
\begin{align*}
\overrightarrow{GA'}+\overrightarrow{GB'}+\overrightarrow{GC'}&=-\frac12\Bigl(\,\underbrace{\overrightarrow{GA}+\overrightarrow{GB}+\overrightarrow{GC}}_{\vec{0}}\,\Bigr),\\
\overrightarrow{GA'}+\overrightarrow{GB'}+\overrightarrow{GC'}&=\vec{0},
\end{align*}
qui prouve que $G$ est le centre de gravité de $A'B'C'$.
\item Si l'on applique la relation \eqref{eulerposition} du théorème \ref{eulerprecis} au triangle $A'B'C'$, on obtient $\overrightarrow{GO}=-2\overrightarrow{GO'}$. On en déduit la relation
\[\overrightarrow{GO'}=-\frac12\overrightarrow{GO}.\qedhere\]\end{itemize}
\end{proof}

\begin{remark}
La configuration des  points  $(O, O', G, H)$  est remarquable. On dit qu'ils sont en \emph{division harmonique} ; cette notion fera l'objet d'une prochaine parution de la \emph{Mathematica dinosaurorum}.
\end{remark}
\endinput

\tgotitle{Puissance d'un point par rapport à un cercle}

\section{Mesure algébrique}
On définit la mesure algébrique d'un bipoint~$(A,B)$ dans un repère de la droite~$(AB)$ comme la différence des abscisses $x_B-x_A$ (voir figure~\ref{figmesalg}). Dans le contexte euclidien qui nous intéresse ici, les repères des droites sont normés : la mesure algébrique ne dépend que de l'orientation de la droite (elle est définie \frquote{au signe près}). On notera cependant  que le produit ou le quotient de deux mesures algébriques sont quant à eux indépendants du choix du repère, puisqu'en cas de changement d'orientation les signes des deux quantités considérées changent simultanément. 

\begin{figure}[ht]
\centering
\begin{tikzpicture}
\coordinate[label=above:$A$] (A) at (-1.3,0);
\coordinate [label=below:$x_A$] (xA) at (-1.3,0);
\coordinate[label=above:$B$] (B) at (1.7,0);
\coordinate [label=below:$x_B$] (xB) at (1.7,0);
\coordinate (Z) at ($(A)!1.1!(B)$);
\coordinate (T) at ($(B)!1.1!(A)$);
\draw (Z)--(T);
\node[below] at ($(A)!0.5!(B)$){$\mesalg{AB}=x_B-x_A$};
\foreach \point in {A,B}
\draw[black,fill=white](\point) circle (1.2pt);	
\end{tikzpicture}
\figcaption{}\label{figmesalg}
\end{figure}

De la définition qui vient d'être donnée, on déduit immédiatement que :
\begin{itemize}
\item Pour tout point $A$, $\mesalg{AA}=0$.
\item Pour tout couple de points $A$ et $B$, $\mesalg{BA}$ et $\mesalg{AB}$ sont deux nombres opposés.
\end{itemize} 
Plus généralement, on peut énoncer le
\begin{thm}\label{thchasles}%
Soit $A$, $B$ et $C$ trois points donnés sur une droite dans un ordre quelconque. On a l'égalité suivante (relation de Chasles pour les mesures algébriques) 
\[\mesalg{AC}=\mesalg{AB}+\mesalg{BC}.\]
\end{thm}

\begin{proof} La vérification est immédiate.
\begin{align*}\mesalg{AB}+\mesalg{BC}&=(x_B-x_A)+(x_C-x_B),\\
&=x_C-x_A,\\
&=\mesalg{AC}.\qedhere
\end{align*}
\end{proof}

Nous donnons maintenant deux compléments sur la notion de mesure algébrique, qui faciliteront la compréhension de la suite.

\begin{remark}[Lien avec le produit d'un vecteur par un réel]
La notion de mesure algébrique est fortement liée à celle de produit d'un vecteur par un réel. Si $O$ et $A$ sont deux points distincts, $\lambda$ un réel et $M$ un point quelconque du plan, on a l'équivalence
\[
(\Vector{OM}=\lambda\Vector{OA})\iff(M\in (OA) \text{ et }\mesalg{OM}=\lambda\mesalg{OA}).
\]
Il s'en déduit que le milieu $I$ d'un segment $[AB]$ peut être caractérisé par des relations entre mesures algébriques, qu'il est parfois intéressant de substituer aux égalités vectorielles correspondantes :
\begin{align}
I \text{ est le milieu de }[AB]&\iff \mesalg{AI}=\mesalg{IB}=\frac12\mesalg{AB},\\
&\iff\mesalg{IA}+\mesalg{IB}=0.\label{eqcaracmilieu}
\end{align}
\end{remark}

\begin{remark}[Lien avec le produit scalaire]
La mesure algébrique peut également intervenir dans l'expression du cosinus d'un angle et du produit scalaire. Précisément, si $A$ et $B$ sont deux points tous deux distincts du point $O$, le produit scalaire $\Vector{OA}\cdot\Vector{OB}$ peut être défini par 
\begin{equation}\Vector{OA}\cdot\Vector{OB}=OA\times OB\times \cos(\widehat{AOB}).\label{pscal1}
\end{equation}
%\[
%\Vector{OA}\cdot\Vector{OB}=OA\times OB\times \rho([OA),[OB)),
%\]
%où $\rho([OA),[OB))$ désigne le rapport de projection orthogonale de la demi-droite $[OB)$ sur la demi-droite $[OA)$. 

On désigne par $H$ le projeté orthogonal de $B$ sur la \emph{droite} $(OA)$.
La définition~\eqref{pscal1} permet d'affirmer que le produit scalaire $\Vector{OA}\cdot\Vector{OB}$ est de même signe que $\cos(\widehat{AOB})$, soit positif, si l'angle $\widehat{AOB}$ est aigu et négatif si cet angle est obtus. Le point $H$ appartient dans ce  dernier cas à la demi-droite opposée à $[OA)$ (voir figure~\ref{figpscal0}). On a donc :
\begin{equation*}
\cos(\widehat{AOB})=
\begin{cases}
\hphantom{-}OH/OB&\text{si $\widehat{AOB}$ est aigu},\\
-OH/OB&\text{si $\widehat{AOB}$ est obtus}.\\
\end{cases}
\end{equation*}

%\[\cos(\widehat{AOB})=\frac{\mesalg{OH}}{\mesalg{OB}}.\]
En reportant dans~\eqref{pscal1} et en simplifiant par $OB$, on obtient une autre expression du produit scalaire :
\begin{equation*}
\Vector{OA}\cdot\Vector{OB}=
\begin{cases}
\hphantom{-}OA\times OH&\text{si $\widehat{AOB}$ est aigu},\\
-OA\times OH&\text{si $\widehat{AOB}$ est obtus}.
\end{cases}%\label{pscal0}
\end{equation*}



On remarque alors que les demi-droites $[OA)$ et $[OH)$ sont de même sens si $\widehat{AOB}$ est aigu, et de sens contraire si $\widehat{AOB}$ est obtus. D'où une expression du produit scalaire utilisant les mesures algébriques :
\begin{equation}
\Vector{OA}\cdot\Vector{OB}=\mesalg{OA}\times\mesalg{OH}.\label{pscal0}
\end{equation}

%\dinofig{\figcaption{}\label{figpscal0}}
\XSmartphoneCommand{
\begin{figure}[ht]
\hfill
\begin{tikzpicture}
\coordinate[label=left:$O$] (O) at (0,0);
\coordinate(X) at (1,0);
\coordinate[label=below:$B$](B) at ($(O)!3!350:(X)$);
\coordinate(Y) at ($(O)!0.5!(B)$);
\coordinate[label=above:$H$](H) at ($(Y)!1!50:(B)$);
\coordinate[label=right:$A$](A) at ($(O)!1.3!(H)$);
\draw[semithick,->] (O)--(B);
\draw[semithick,->] (O)--(A);
\draw[dashed](B)--(H);
\foreach \point in {O,H}
\draw[black,fill=white](\point) circle (1.2pt);	
\end{tikzpicture}
\hfill\hfill
\begin{tikzpicture}
\coordinate[label=below:$O$] (O) at (0,0);
\coordinate(X) at (1,0);
\coordinate[label=above:$B$](B) at ($(O)!2.9!160:(X)$);
\coordinate(Y) at ($(O)!0.5!(B)$);
\coordinate[label=below:$H$](H) at ($(Y)!1!50:(B)$);
\coordinate(Z) at ($(O)!1.2!(H)$);
\coordinate[label=above:$A$](A) at ($(O)!-0.6!(H)$);
\draw[semithick,->] (O)--(B);
\draw[semithick,->] (O)--(A);
\draw(O)--(Z);
\draw[dashed](B)--(H);
\foreach \point in {O,H}
\draw[black,fill=white](\point) circle (1.2pt);	
\end{tikzpicture}
\hfill\null
\figcaption{}\label{figpscal0}
\end{figure}
}
\SmartphoneCommand{
\begin{figure}[ht]
\centering
\begin{tikzpicture}
\coordinate[label=left:$O$] (O) at (0,0);
\coordinate(X) at (1,0);
\coordinate[label=below:$B$](B) at ($(O)!3!350:(X)$);
\coordinate(Y) at ($(O)!0.5!(B)$);
\coordinate[label=above:$H$](H) at ($(Y)!1!50:(B)$);
\coordinate[label=right:$A$](A) at ($(O)!1.3!(H)$);
\draw[semithick,->] (O)--(B);
\draw[semithick,->] (O)--(A);
\draw[dashed](B)--(H);
\foreach \point in {O,H}
\draw[black,fill=white](\point) circle (1.2pt);	
\end{tikzpicture}
\\[1ex]
\begin{tikzpicture}
\coordinate[label=below:$O$] (O) at (0,0);
\coordinate(X) at (1,0);
\coordinate[label=above:$B$](B) at ($(O)!2.9!160:(X)$);
\coordinate(Y) at ($(O)!0.5!(B)$);
\coordinate[label=below:$H$](H) at ($(Y)!1!50:(B)$);
\coordinate(Z) at ($(O)!1.2!(H)$);
\coordinate[label=above:$A$](A) at ($(O)!-0.6!(H)$);
\draw[semithick,->] (O)--(B);
\draw[semithick,->] (O)--(A);
\draw(O)--(Z);
\draw[dashed](B)--(H);
\foreach \point in {O,H}
\draw[black,fill=white](\point) circle (1.2pt);	
\end{tikzpicture}
\figcaption{}\label{figpscal0}
\end{figure}
}
\begin{example}[Remarques]
\begin{enumerate}
\item L'angle noté $\widehat{AOB}$ est un angle \emph{non orienté} de demi-droites ; autrement dit, il n'y a pas lieu de distinguer  $\widehat{AOB}$ et $\widehat{BOA}$. Il en découle que le produit scalaire est commutatif (le terme \frquote{symétrique} serait sans doute plus adapté).

On aurait donc pu tout aussi bien utiliser le projeté de $A$ sur $(OB)$. 
\item L'indépendance du cosinus par rapport à l'ordre des deux demi droites $[OA)$ et $[OB)$ provient de ce que $\cos(\widehat{AOB})$ est égal au rapport de projection orthogonale, noté $\rho([OA),[OB))$, de  la demi-droite $[OB)$ sur la demi-droite $[OA)$. On sait en effet (c'est en général considéré comme un des axiomes de la géométrie plane) que
\[\rho([OA),[OB))=\rho([OB),[OA)).\]

 \item Lorsque les points $A$ et $B$ sont confondus, on a 
 \begin{align*}
 \Vector{OA}\cdot\Vector{OB}&=\Vector{OA}\cdot\Vector{OA},\\
 &=OA^2\times \cos(\widehat{AOA}).
 \end{align*}
 L'angle $\widehat{AOA}$ est nul : on a donc $\cos(\widehat{AOA})=1$ d'où finalement 
 \begin{equation}
\Vector{OA}\cdot\Vector{OA}=\bigl({\Vector{OA}}\bigr)^2=OA^2.\label{eqcarres}
\end{equation}
\end{enumerate}
\end{example}
\end{remark}

\section{Point et cercle}
\subsection{Une définition}
Étant donné un cercle $\symcal{C}$, de centre $O$ et de rayon $r$, et un point $P$ quelconque, la puissance du point $P$ par rapport au cercle $\symcal{C}$ est par définition le nombre 
\[\symcal{C}(P)=OP^2-r^2 \text{ (voir figure~\ref{figdefppc})}.\]

\begin{figure}[ht]
\centering
\begin{tikzpicture}
\coordinate[label=below:$O$] (O) at (2,0);
\coordinate(X) at (3,0);
\node[draw,label=above:$\symcal{C}$] (c) at (O) [circle through={(X)}]{};
\coordinate[label=right:$P$] (P) at ($(O)!3.2!15:(X)$);
\coordinate[label=right:$A$](A) at ($(O)!1!312:(X)$);
\draw[semithick](P)--(O)--(A);
\node[right] at (X){$\symcal{C}(P)=OP^2-OA^2$};
\node[right=1.3cm,below=1mm] at (X){$\text{pour tout }A\in\symcal{C}$};
\foreach \point in {P,O,A}
\draw[black,fill=white](\point) circle (1.2pt);	
\end{tikzpicture}
\figcaption{}\label{figdefppc}
\end{figure}

Par conséquent, la puissance d'un point $P$ par rapport à un cercle est :
\begin{itemize}
\item strictement positive si le point $P$ est extérieur au cercle;
\item nulle si le point $P$ appartient au cercle;
\item strictement négative et au minimum égale à $-r^2$ si le point $P$ est intérieur au cercle.
\end{itemize}

\subsection{Avec une tangente}\label{subsectangente}
On se place ici dans le cas où le point $P$ est extérieur au cercle. Il est alors possible de s'intéresser aux tangentes issues de $P$.
\begin{thm}
Soit un cercle $\symcal{C}$ de centre $O$, un point $P$ extérieur au cercle et $T$ le point de contact de l'une des tangentes à $\symcal{C}$ issues de $P$. La puissance de $P$ par rapport à $\symcal{C}$ vaut
\[\symcal{C}(P)=PT^2.\]
\end{thm}


\begin{proof}
On pourra se référer à la figure \ref{figtangente}. La droite $(PT)$ étant tangente à $\symcal{C}$, le triangle $OTP$ est rectangle en $T$. D'après le théorème de Pythagore, on a 
\[\symcal{C}(P)=OP^2-OT^2=PT^2.\qedhere\]
\end{proof}

\begin{figure}[ht]
\centering
\begin{tikzpicture}
\coordinate[label=left:$O$] (O) at (2,0);
\coordinate(X) at (3.5,0);
\node[draw,label=above:$\symcal{C}$] (c) at (O) [circle through={(X)}]{};
\coordinate[label=right:$P$] (P) at ($(O)!1.7!345:(X)$);
\coordinate(Y) at ($(O)!0.5!(P)$);
\node[draw,dashed] (x) at (Y) [circle through={(P)}]{};
\coordinate[label=above:$T$](T) at  (intersection 2 of c and x);
\draw (P)--(T)--(O)--cycle;
\foreach \point in {P,T,O}
\draw[black,fill=white](\point) circle (1.2pt);	
\end{tikzpicture}
\figcaption{}\label{figtangente}
\end{figure}

\subsection{Avec une sécante}\label{subsecsecante}
Le point $P$ est cette fois-ci quelconque. On considère alors une droite sécante au cercle et passant par $P$ (voir la figure \ref{figsecante}).
\begin{thm}
Soit un cercle $\symcal{C}$ de centre $O$, un point $P$ quelconque et une droite passant par $P$ sécante au cercle en $A$ et $B$. La puissance de $P$ par rapport à $\symcal{C}$ vaut alors
\[\symcal{C}(P)=\mesalg{PA}\times\mesalg{PB}.\]
\end{thm}

\begin{figure}[ht]
\centering
\begin{tikzpicture}
\coordinate[label=below:$O$] (O) at (2,0);
\coordinate(X) at (3.5,0);
\node[draw,label=above:$\symcal{C}$] (c) at (O) [circle through={(X)}]{};
\coordinate[label=left:$A$] (A) at ($(O)!1!155:(X)$);
\coordinate(A') at ($(A)!2!(O)$);
\coordinate[label=right:$P$] (P) at ($(O)!1.8!35:(X)$);
\coordinate(Z) at ($(A')!0.5!(P)$);
\node (x) at (Z) [circle through={(P)}]{};
\coordinate[label=above:$B$](B) at  (intersection 2 of c and x);
\coordinate[label=above:$I$](I) at ($(A)!0.5!(B)$);
\draw (P)--(B)--(I)--(A);
\draw(A)--(O)--(B);
\draw(P)--(O)--(I);
\foreach \point in {A,P,B,O,I}
\draw[black,fill=white](\point) circle (1.2pt);	
\end{tikzpicture}
\figcaption{}\label{figsecante}
\end{figure}

\begin{proof}
Soit $I$ le milieu de $[AB]$ : $AOB$ étant isocèle en $O$, la droite $(OI)$ est la médiatrice de $[AB]$. Les triangles $OIP$ et $OIA$ sont rectangles en $I$. On a donc $OP^2=OI^2+IP^2$ et $OA^2=OI^2+IA^2$.

 Par conséquent,
\begin{align*}
\symcal{C}(P)&=PO^2-OA^2,\\
&=(OI^2+IP^2)-(OI^2+IA^2),\\
\intertext{soit en réduisant,}
\symcal{C}(P)&=IP^2-IA^2,\\
\intertext{puis en utilisant une identité remarquable et les égalités~\eqref{eqcarres} et~\eqref{eqcaracmilieu}}
\symcal{C}(P)&=\mesalg{PI}^2-\mesalg{IA}^2,\\
&=(\mesalg{PI}+\mesalg{IA})(\mesalg{PI}-\mesalg{IA}),\\
&=(\mesalg{PI}+\mesalg{IA})(\mesalg{PI}+\mesalg{IB}),\\
\intertext{et enfin, grâce au théorème~\ref{thchasles}}
\symcal{C}(P)&=\mesalg{PA}\times\mesalg{PB}.\qedhere
\end{align*}

\end{proof}

\begin{example}[Remarques]
\begin{enumerate}
\item Il est remarquable que la quantité $\mesalg{PA}\times\mesalg{PB}$ ne dépende pas de la sécante choisie : autrement dit, si deux sécantes passant par $P$ coupent le cercle respectivement en $A$ et $B$ et en $C$ et $D$, alors
\[\mesalg{PA}\times\mesalg{PB}=\mesalg{PC}\times\mesalg{PD}.\]
\item Si $P$ est intérieur au cercle, les demi-droites $[PA)$ et $[PB)$ sont de sens contraires : on retrouve le fait que $\symcal{C}(P)$ est dans ce cas négatif.
\item Une tangente issue de $P$, comme on l'a évoqué en \ref{subsectangente}, peut être considérée comme une position limite d'une sécante pivotant autour de $P$, les points $A$ et $B$ étant alors confondus en $T$ : on aura bien de ce point de vue $\mesalg{PA}\times\mesalg{PB}=PT^2$.
\end{enumerate}
\end{example} 

\subsection{Utilisation du produit scalaire}
On utilise ici la relation~\eqref{pscal0} pour donner une autre expression de la puissance d'un point par rapport à un cercle (voir figure~\ref{figpscal2}).
\begin{thm}
Soit un cercle $\symcal{C}$ de centre $O$, un point $P$ quelconque, et deux points de $\symcal{C}$ diamétralement opposés $A_1$ et $A_2$. La puissance de $P$ par rapport à $\symcal{C}$ vaut alors
\[\symcal{C}(P)=\Vector{PA_1}\cdot\Vector{PA_2}.\]
\end{thm}

\begin{figure}[ht]
\centering
\begin{tikzpicture}
\coordinate[label=below:$O$] (O) at (2,0);
\coordinate(X) at (3.5,0);
\node[draw,label=above:$\symcal{C}$] (c) at (O) [circle through={(X)}]{};
\coordinate[label=left:$A_1$] (A) at ($(O)!1!160:(X)$);
\coordinate[label=right:$A_2$] (A') at ($(A)!2!(O)$);
\coordinate[label=right:$P$] (P) at ($(O)!2.2!32:(X)$);
\coordinate(Z) at ($(A')!0.5!(P)$);
\node (x) at (Z) [circle through={(P)}]{};
\coordinate[label=above:$B$](B) at  (intersection 2 of c and x);
\draw [semithick,->](P)--(A);
\draw[semithick,->](P)--(A');
\draw(A')--(B);
\foreach \point in {P,B,O}
\draw[black,fill=white](\point) circle (1.2pt);	
\end{tikzpicture}
\figcaption{}\label{figpscal2}
\end{figure}

\begin{proof}%\sloppy
Tout d'abord, les droites $(PA_1)$ et $(PA_2)$ ne peuvent être toutes les deux tangentes à $\symcal{C}$ : en effet, deux tangentes en des points diamétralement opposés sont parallèles disjointes et ne peuvent avoir le point $P$ en commun. Quitte à permuter $A_1$ et $A_2$, il n'est donc pas restrictif de supposer que la droite $(PA_1)$ est sécante (et non tangente) à $\symcal{C}$.

Soit alors $B$ la seconde intersection de la droite $(PA_1)$ avec le cercle. Le segment $[A_1A_2]$ étant un diamètre du cercle, le point $B$ est le projeté orthogonal de $A_2$ sur $(PA_1)$. On a donc d'après \ref{subsectangente}
\begin{align*}
\symcal{C}(P)&=\mesalg{PA_1}\times\mesalg{PB},\\
&=\Vector{PA_1}\cdot\Vector{PA_2}.\qedhere
\end{align*}
\end{proof}


\begin{remark}
On se persuadera facilement, en observant la figure \ref{figpscal3}, que le raisonnement précédent est indépendant de la position de $P$ par rapport au cercle.
\end{remark}
%\dinofig[-0.5]{\figcaption{}\label{figpscal3}}

\begin{figure}[ht]
\centering
\begin{tikzpicture}
\coordinate[label=below:$O$] (O) at (2,0);
\coordinate(X) at (3.5,0);
\node[draw,label=above:$\symcal{C}$] (c) at (O) [circle through={(X)}]{};
\coordinate[label=left:$A_1$] (A) at ($(O)!1!160:(X)$);
\coordinate[label=right:$A_2$] (A') at ($(A)!2!(O)$);
\coordinate[label=above:$P$] (P) at ($(O)!0.6!70:(X)$);
\coordinate(Z) at ($(A')!0.5!(P)$);
\node (x) at (Z) [circle through={(P)}]{};
\coordinate[label=above:$B$](B) at  (intersection 2 of c and x);
\draw [semithick,->](P)--(A);
\draw[semithick,->](P)--(A');
\draw(P)--(B);
\draw(A')--(B);
\foreach \point in {P,B,O}
\draw[black,fill=white](\point) circle (1.2pt);	
\end{tikzpicture}
\figcaption{}\label{figpscal3}
\end{figure}

\subsection{Dans un repère orthonormal}
On se place ici dans un repère orthonormal et l'on considère un cercle $\symcal{C}$ de centre $\Omega(a,b)$ et de rayon $r$. On s'intéresse alors à l'expression de la puissance d'un point $M(x,y)$ quelconque par rapport au cercle $\symcal{C}$.
\begin{thm}\label{threp}
Soit dans le plan rapporté à un repère orthonormal un cercle $\symcal{C}$ de centre $\Omega(a,b)$ et de rayon $r$. Soit un point $M(x,y)$. La puissance de $M$ par rapport à $\symcal{C}$ est 
\[\symcal{C}(M) = x^2+y^2-2ax-2by+a^2+b^2-r^2.\]
\end{thm}

\begin{proof}
On a 
\begin{align*}\symcal{C}(M)&=\Omega M^2-r^2\\
&={(x-a)}^2+{(y-b)}^2-r^2,\\
&=x^2-2ax+a^2+y^2-2by+b^2-r^2.\qedhere
\end{align*}
\end{proof}

De plus, en remarquant que $M\in\symcal{C}$ si et seulement si la puissance de $M$ par rapport à $\symcal{C}$ est nulle, on établit le corollaire suivant :
%\DinoAfourNBCommand{\par\pagebreak}
\begin{coro*}
Dans le plan rapporté à un repère orthonormal, le cercle $\symcal{C}$ de centre $\Omega(a,b)$ et de rayon $r$ admet pour équation
\[x^2+y^2-2ax-2by+c=0,\]
où $c=a^2+b^2-r^2$.
\end{coro*}

\section{Cocyclicité}
On dit que quatre points distincts sont cocycliques s'ils appartiennent à un même cercle (un tel cercle est évidemment unique : ce que l'on affirme en disant que les points sont cocycliques est que cercle circonscrit au triangle formé par trois quelconques de ces points passe par le quatrième). On se propose d'établir le théorème suivant (voir figure~\ref{figcocy}).

\begin{thm} Soit quatre points distincts $A$, $B$, $C$ et $D$ tels que les droites $(AB)$ et $(CD)$ soient sécantes en $P$. Les points $A$, $B$, $C$, $D$ sont cocycliques si et seulement si 
\[\mesalg{PA}\times\mesalg{PB}=\mesalg{PC}\times\mesalg{PD}.\]
\end{thm}

%\dinofig{\figcaption{}\label{figcocy}}
\begin{figure}[ht]
\centering
\begin{tikzpicture}
\coordinate (O) at (2,0);
\coordinate(X) at (4,0);
\node[draw,label=above:$\symcal{C}$] (c) at (O) [circle through={(X)}]{};
\coordinate[label=left:$A$] (A) at ($(O)!1!140:(X)$);
\coordinate[label=below:$B$] (B) at ($(O)!1!310:(X)$);
\coordinate[label=left:$C$] (C) at ($(O)!1!210:(X)$);
\coordinate[label=above:$D$] (D) at ($(O)!1!65:(X)$);
\coordinate[label=right:$\kern3pt P$](P) at  (intersection of A--B and C--D);
\draw(A)--(B);
\draw(C)--(D);
\foreach \point in {A,B,C,D,P}
\draw[black,fill=white](\point) circle (1.2pt);	
\end{tikzpicture}
\figcaption{}\label{figcocy}
\end{figure}

\begin{proof}
Il s'agit de démontrer une équivalence. Nous procéderons donc classiquement en deux temps : d'abord le sens direct, puis la réciproque.

Supposons pour commencer que les points $A$, $B$, $C$, $D$ soient sur un même cercle $\symcal{C}$. Alors les droites $(AB)$ et $(CD)$ étant des sécantes au cercle contenant chacune le point $P$, on a
\[\symcal{C}(P)=\mesalg{PA}\times\mesalg{PB}=\mesalg{PC}\times\mesalg{PD}.\]

Réciproquement, supposons que $\mesalg{PA}\times\mesalg{PB}=\mesalg{PC}\times\mesalg{PD}$. Soit alors $\symcal{C}\;$ le cercle circonscrit au triangle $ABC$ : il nous suffit de prouver que $D\in\symcal{C}$. La droite $(CD)$ coupe le cercle en $C$ et en un second point $M$ tel que
\begin{align*}\mesalg{PC}\times\mesalg{PM}&=\symcal{C}(P)\\
&=\mesalg{PA}\times\mesalg{PB}\\
&=\mesalg{PC}\times\mesalg{PD}, \text{ par hypothèse.}
\end{align*}
On en déduit que $\mesalg{PD}=\mesalg{PM}$, donc que $D=M$ ($P$ ne peut être confondu avec $C$, car alors il serait lui même confondu avec $A$ ou $B$).
\end{proof}


\section{Axe radical de deux cercles}

\subsection{Définition}
On s'intéresse ici à l'ensemble des  points qui ont même puissance par rapport à deux cercles non concentriques donnés.

\begin{thm}
Étant donné deux cercles non concentriques $\symcal{C}_1$ et $\symcal{C}_2$ de centres et de rayons respectifs $(O_1,r_1)$ et $(O_2,r_2)$, l'ensemble des points qui ont la même puissance par rapport à chacun des cercles est une droite $\Delta$ perpendiculaire à $(O_1O_2)$, nommée \emph{axe radical} des deux cercles.
\end{thm}

%\dinofig{\figcaption{}\label{figaxerad0}}
\begin{figure}[ht]
\centering
\begin{tikzpicture}
\clip (0.3,-1.3) rectangle (5.1,1.9);
\coordinate[label=below:$O_1$] (O1) at (1,0);
\coordinate(X1) at (1.6,0);
\node[draw,label=above:$\symcal{C_1}$] (c1) at (O1) [circle through={(X1)}]{};
\coordinate[label=below:$O_2$] (O2) at (3.8,0);
\coordinate[label=below:$I$] (I) at ($(O1)!0.5!(O2)$);
\coordinate(X2) at (5,0);
\node[draw,label=above:$\symcal{C_2}$] (c2) at (O2) [circle through={(X2)}]{};
\coordinate(Y1) at ($(O1)!-1.2!(X1)$);
\coordinate(Y2) at ($(O2)!1.1!(X2)$);
\draw (Y1)--(Y2);
\coordinate(Y) at (2,-1);
\coordinate(U) at (2,0.5);
\node (x) at (Y) [circle through={(U)}]{};
\coordinate(Z1) at  (intersection 1 of c1 and x);
\coordinate(T1) at  (intersection 2 of c1 and x);
\coordinate(Z2) at  (intersection 1 of c2 and x);
\coordinate(T2) at  (intersection 2 of c2 and x);
\coordinate[label=left:$M$](M) at (intersection of Z1--T1 and Z2--T2);
\coordinate(W1) at ($(O1)!0.5!(M)$);
\coordinate(W2) at ($(O2)!0.5!(M)$);
\node (w1) at (W1) [circle through={(M)}]{};
\node(w2) at (W2) [circle through={(M)}]{};
\coordinate[label=below:$H\kern13pt$](H) at  (intersection 1 of w1 and w2);
\coordinate[label=left:$\Delta$](V1) at ($(H)!1.5!(M)$);
\coordinate(V2) at ($(H)!-1.1!(M)$);
\draw[semithick](V1)--(V2);
\foreach \point in {O1,O2,I,M,H}
\draw[black,fill=white](\point) circle (1.2pt);	
\end{tikzpicture}
\figcaption{}\label{figaxerad0}
\end{figure}

\begin{remark}
Si les deux cercles sont concentriques et distincts, aucun point $M$ du plan n'est tel que $\symcal{C}_1(M)=\symcal{C}_2(M)$, c'est une conséquence immédiate de la définition de la puissance d'un point par rapport à un cercle.
\end{remark}

\begin{proof}
Soit un point $M$ du plan et $H$ son projeté orthogonal sur la droite $(O_1O_2)$. Nous supposons que $\symcal{C}_1(M)=\symcal{C}_2(M)$ ; nous allons prouver que le point $H$ occupe une position fixe (indépendante de $M$) sur $(O_1O_2)$, ce qui prouvera que $M$ appartient à la perpendiculaire en $H$ à cette droite. Pour nous éviter une réciproque fastidieuse, nous procéderons par équivalence. Désignons par $I$ le milieu de $[O_1O_2]$ et exprimons l'égalité des puissances (voir la figure~\ref{figaxerad0}).
\begin{align*}
\symcal{C}_1(M)=\symcal{C}_2(M) &\iff O_1M^2-r_1^2=O_2M^2-r_2^2,\\
\intertext{soit, en posant  $k=r_1^2-r_2^2$,}
\symcal{C}_1(M)=\symcal{C}_2(M) &\iff O_1M_1^2-O_2M2^2=k,\\
&\iff (\Vector{O_1M}+\Vector{O_2M})\cdot(\Vector{O_1M}-\Vector{O_2M})=k.
\end{align*}

On remarque alors que 
\begin{equation*}
\Vector{O_1M}+\Vector{O_2M}=\hspace{-0.5em}\underbrace{\Vector{O_1I}+\Vector{O_2I}}_{\substack{\vec{0}\text{, car } I\\\text{milieu de }[O_1O_2]}}\hspace{-0.5em}+\,2\Vector{IM}=2\Vector{IM}
\end{equation*}
et que
\begin{equation*}
\Vector{O_1M}-\Vector{O_2M}=\Vector{O_1M}+\Vector{MO_2}=\Vector{O_1O_2}.
\end{equation*}

On en déduit
\begin{align*}
\symcal{C}_1(M)=\symcal{C}_2(M) &\iff 2\Vector{IM}\cdot\Vector{O_1O2}=k,\\
&\iff 2(\underbrace{\Vector{IH}+\Vector{HM}}_{\Vector{IM}})\cdot\Vector{O_1O_2}=k,\\
&\iff 2\Vector{IH}\cdot\Vector{O_1O_2}+2\underbrace{\Vector{HM}\cdot\Vector{O_1O_2}}_{\substack{0 \text{, car}\\(HM)\perp(O_1O_2)}}=k.
\end{align*}

Puisque les points $I$, $H$, $O_1$ et $O_2$ sont sur la même droite, on peut  passer aux mesures algébriques, et on obtient finalement :
\begin{align}
\symcal{C}_1(M)=\symcal{C}_2(M)&\iff 2\mesalg{IH}\cdot\mesalg{O_1O_2}=k,\notag\\
&\iff\mesalg{IH}=\frac{r_1^2-r_2^2}{2\mesalg{O_1O_2}}.\label{eqposH}
\end{align}


On en déduit que $\symcal{C}_1(M)=\symcal{C}_2(M)$ si et seulement si le projeté orthogonal $H$ de $M$ sur $(O_1O_2)$ occupe la position (fixe) définie par la relation~\eqref{eqposH}. L'ensemble des points $M$ est donc la droite $\Delta$ perpendiculaire à $(O_1O_2)$ passant par $H$.
\end{proof}

\subsection{Détermination pratique}\label{subsecdetprat}
On s'intéresse ici à la position relative des cercles $\symcal{C_1}$ et $\symcal{C_2}$.
\subsubsection{Cercles sécants}
Supposons que les cercles $\symcal{C_1}$ et $\symcal{C_2}$ soient sécants en deux points $A$ et $B$ (voir la figure~\ref{figaxerad1}). On a 
\[\symcal{C}_1(A)=\symcal{C}_2(A)=0 \text{ et }\symcal{C}_1(B)=\symcal{C}_2(B)=0 .
\]
Les points $A$ et $B$ ont tous deux la même puissance par rapport à chacun des deux cercles : l'axe radical $\Delta$ de $\symcal{C}_1$ et $\symcal{C}_2$ est donc la droite $(AB)$.


%\dinofig{\figcaption{}\label{figaxerad1}}
\begin{figure}[ht]
\centering
\begin{tikzpicture}
\coordinate[label=below:$O_1$] (O1) at (0.8,0);
\coordinate(X1) at (2,0);
\node[draw,label=above:$\symcal{C_1}$] (c1) at (O1) [circle through={(X1)}]{};
\coordinate[label=below:$O_2$] (O2) at (3,0);
\coordinate(X2) at (4.8,0);
\node[draw,label=above:$\symcal{C_2}$] (c2) at (O2) [circle through={(X2)}]{};
\coordinate(Y1) at ($(O1)!-1.2!(X1)$);
\coordinate(Y2) at ($(O2)!1.1!(X2)$);
\draw (Y1)--(Y2);
\coordinate[label=right:$\kern2pt A$](A) at  (intersection 2 of c1 and c2);
\coordinate[label=right:$\kern2pt B$](B) at  (intersection 1 of c1 and c2);
\coordinate[label=below:$H\kern-13pt$](H) at  (intersection  of O1--O2 and A--B);
\coordinate[label=left:$\Delta$](V1) at ($(H)!1.7!(A)$);
\coordinate(V2) at ($(H)!1.3!(B)$);
\draw[semithick](V1)--(V2);
\foreach \point in {O1,O2,A,B,H}
\draw[black,fill=white](\point) circle (1.2pt);	
\end{tikzpicture}
\figcaption{}\label{figaxerad1}
\end{figure}

\subsubsection{Cercles tangents}
Supposons que les cercles $\symcal{C_1}$ et $\symcal{C_2}$ soient tangents (intérieurement ou extérieurement) en un point $T$ (voir la figure~\ref{figaxerad2}). On a 
$\symcal{C}_1(T)=\symcal{C}_2(T)=0$.
Le point $T$ a la même puissance par rapport à chacun des deux cercles : l'axe radical $\Delta$ de $\symcal{C}_1$ et $\symcal{C}_2$ est donc la droite perpendiculaire en $T$ à $(O_1O_2)$.

%\dinofig{\figcaption{}\label{figaxerad2}}
\begin{figure}[ht]
\centering
\begin{tikzpicture}
\coordinate[label=below:$O_1$] (O1) at (1.2,0);
\coordinate[label=below:$\kern10pt T$](T) at (3,0);
\node[draw,label=above:$\symcal{C_1}$] (c1) at (O1) [circle through={(T)}]{};
\coordinate[label=below:$O_2$] (O2) at (1.8,0);
%\coordinate(X2) at (4.8,0);
\node[draw,label=above:$\symcal{C_2}$] (c2) at (O2) [circle through={(T)}]{};
\coordinate(Y1) at ($(O1)!-1.3!(T)$);
\coordinate(Y2) at ($(O2)!1.3!(T)$);
\draw (Y1)--(Y2);
\coordinate[label=left:$\Delta$](V1) at ($(T)!1!270:(O1)$);
\coordinate(V2) at ($(T)!1!90:(O1)$);
\draw[semithick](V1)--(V2);
\foreach \point in {O1,O2,T}
\draw[black,fill=white](\point) circle (1.2pt);	
\end{tikzpicture}
\figcaption{}\label{figaxerad2}
\end{figure}


\subsubsection{Cas général}
Il est toujours possible de tracer un cercle auxiliaire $\Gamma$ sécant à $\symcal{C}_1$ en $A_1$ et $B_1$, à $\symcal{C}_2$ en $A_2$ et $B_2$, et  de telle sorte que les droites $(A_1B_1)$ et $(A_2B_2)$ soient elles-mêmes sécantes en un point $P$ (voir la figure~\ref{figaxerad3}). On est ainsi ramené à des constructions ans le cas de cercles sécants.

Le point $P$ appartient à l'axe radical de $(A_1B_1)$ de $\symcal{C}_1$ et $\Gamma$, donc $\symcal{C}_1(P)=\Gamma(P)$. De même, $\symcal{C}_2(P)=\Gamma(P)$ et par conséquent $\symcal{C}_1(P)=\symcal{C}_2(P)$. Le point $P$ appartient donc à l'axe radical $\Delta$ de $\symcal{C}_1$ et $\symcal{C}_2$ : $\Delta$  est  la perpendiculaire à $(O_1O_2)$ passant par $P$. 

\begin{figure}[ht]
\centering
\begin{tikzpicture}
\coordinate[label=below:$O_1$] (O1) at (0.4,0);
\coordinate(X1) at (1.5,0);
\node[draw,label=above:$\symcal{C_1}$] (c1) at (O1) [circle through={(X1)}]{};
\coordinate[label=below:$O_2$] (O2) at (3.6,0);
\coordinate(X2) at (5,0);
\node[draw,label=above:$\symcal{C_2}$] (c2) at (O2) [circle through={(X2)}]{};
\coordinate(Y1) at ($(O1)!-1.2!(X1)$);
\coordinate(Y2) at ($(O2)!1.1!(X2)$);
\draw (Y1)--(Y2);
\coordinate(Y) at (2,-1);
\coordinate(U) at (2,0.5);
\node[draw,label=below:$\kern10pt\Gamma$] (g) at (Y) [circle through={(U)}]{};
\coordinate[label=left:$A_1${\rule[-8pt]{0pt}{10pt}}](A1) at  (intersection 2 of c1 and g);
\coordinate[label=below:$B_1\kern10pt$](B1) at  (intersection 1 of c1 and g);
\coordinate[label=right: $A_2${\rule[-8pt]{0pt}{10pt}}](A2) at  (intersection 1 of c2 and g);
\coordinate[label=below:$\kern10pt B_2$](B2) at  (intersection 2 of c2 and g);
\coordinate[label=left:$P$](P) at (intersection of A1--B1 and A2--B2);
\draw (P)--(B1);
\draw (P)--(B2);
\coordinate(W1) at ($(O1)!0.5!(P)$);
\coordinate(W2) at ($(O2)!0.5!(P)$);
\node (w1) at (W1) [circle through={(P)}]{};
\node(w2) at (W2) [circle through={(P)}]{};
\coordinate[label=below:$H\kern13pt$](H) at  (intersection 1 of w1 and w2);
\coordinate[label=left:$\Delta$](V1) at ($(H)!1.5!(P)$);
\coordinate(V2) at ($(H)!-2.4!(P)$);
\draw[semithick](V1)--(V2);
\foreach \point in {O1,O2,P,H,A1,B1,A2,B2}
\draw[black,fill=white](\point) circle (1.2pt);	
\end{tikzpicture}
\figcaption{}\label{figaxerad3}
\end{figure}

\subsubsection{Dans un repère orthonormal}
Soit un cercle $\symcal{C}_1$ de centre $\Omega_1(a_1,b_1)$ et de rayon $r_2$, un cercle $\symcal{C}_2$ de centre $\Omega_2(a_2,b_2)$ et de rayon $r_2$. On suppose $\Omega_1\neq\Omega_2$ et on nomme $\Delta$ l'axe radical de ces deux cercles. D'après le théorème~\ref{threp} page \pageref{threp}, l'appartenance de $M(x,y)$ à $\Delta$ se traduit par
\begin{equation*}
x^2+y^2-2a_1x-2b_1y+c_1=x^2+y^2-2a_2x-2b_2y+c_2,
\end{equation*}
où $c_1=a_1^2+b_1^2-r_1^2$ et $c_2=a_2^2+b_2^2-r_2^2$.

\vspace\baselineskip\par Donc, après simplification :
\[P\in \Delta \iff (a_2-a_1)x+(b_2-b_1)y+(c_1-c_2)/2=0.\]
L'axe radical $\Delta$ admet pour équation 
\[(a_2-a_1)x+(b_2-b_1)y+(c_1-c_2)/2=0,\]
 où l'on constate bien que $\Delta$  admet $\Vector{\Omega_1\Omega_2}$ pour vecteur normal.
 
 \subsection{Centre radical}
 Lorsqu'au~\ref{subsecdetprat} nous avons, pour traiter le cas général, exploité l'idée d'un cercle auxiliaire, nous avons implicitement démontré le théorème suivant (voir la figure~\ref{figcentrerad}).
 \begin{thm}
 Soient trois cercles $\symcal{C}_1$, $\symcal{C}_2$ et $\Gamma$ dont les centres sont distincts et non alignés. Les axes radicaux $\Delta$ de $\symcal{C}_1$ et $\symcal{C}_2$, $\symcal{D}_1$ de $\symcal{C}_1$ et $\Gamma$ et finalement $\symcal{D}_2$ de $\symcal{C}_2$ et $\Gamma$ sont concourants en un point $P$ nommé \emph{centre radical} des trois cercles.
 \end{thm}
 
\newcommand{\figurecentrerad}{%
 \begin{figure}[ht]
\centering
\begin{tikzpicture}
\coordinate (O1) at (0.4,0);
\coordinate(X1) at (1.5,0);
\node[draw,label=above:$\symcal{C_1}$] (c1) at (O1) [circle through={(X1)}]{};
\coordinate (O2) at (3.6,0);
\coordinate(X2) at (5,0);
\node[draw,label=above:$\symcal{C_2}$] (c2) at (O2) [circle through={(X2)}]{};
\coordinate(Y) at (2,-1);
\coordinate(U) at (2,0.5);
\node[draw,label=below:$\kern10pt\Gamma$] (g) at (Y) [circle through={(U)}]{};
\coordinate(A1) at  (intersection 2 of c1 and g);
\coordinate(B1) at  (intersection 1 of c1 and g);
\coordinate(A2) at  (intersection 1 of c2 and g);
\coordinate(B2) at  (intersection 2 of c2 and g);
\coordinate[label=right:$\symcal{D}_1$](Y1) at ($(A1)!1.8!(B1)$);
\coordinate[label=left:$\symcal{D}_2$](Y2) at ($(A2)!1.5!(B2)$);
\coordinate[label=left:$P$](P) at (intersection of A1--B1 and A2--B2);
\draw (P)--(Y1);
\draw (P)--(Y2);
\coordinate(W1) at ($(O1)!0.5!(P)$);
\coordinate(W2) at ($(O2)!0.5!(P)$);
\node (w1) at (W1) [circle through={(P)}]{};
\node(w2) at (W2) [circle through={(P)}]{};
\coordinate(H) at  (intersection 1 of w1 and w2);
\coordinate[label=left:$\Delta$](V1) at ($(H)!1.5!(P)$);
\coordinate(V2) at ($(H)!-2.4!(P)$);
\draw(V1)--(V2);
\foreach \point in {P,A1,B1,A2,B2}
\draw[black,fill=white](\point) circle (1.2pt);	
\end{tikzpicture}
\figcaption{}\label{figcentrerad}
\end{figure}
}

%\DinoiPhoneCommand{%
%\afterpage{\clearpage
%\figurecentrerad
%\par
%\vspace{0.7cm}
%{
%\centering\LARGE\textcolor{ColorOne}\decoone

%}
%}}

%\DinoiPadStdCommand{\figurecentrerad}
%\DinoiPadAirCommand{\figurecentrerad}
%\DinoiPadProCommand{\figurecentrerad}
%%\DinoAfourNBCommand{\figurecentrerad}
%\DinoAfourColorCommand{\figurecentrerad}
%\DinoiPadminiCommand{\figurecentrerad}
%%\DinoSaisieCommand{\figurecentrerad}
%\DinoiPhoneCommand{\figurecentrerad}
\figurecentrerad

\begin{proof}
On définit $P$ comme intersection de $\symcal{D}_1$ et $\symcal{D}_2$ (le fait que les centres ne soient pas alignés garantit l'existence de $P$) :
\begin{align*}
P\in \symcal{D}_1&\implies \symcal{C}_1(P)=\Gamma(P)\text{ et }\\
P\in \symcal{D}_2&\implies \symcal{C}_2(P)=\Gamma(P),
\end{align*}
dont on déduit que $\symcal{C}_1(P)=\symcal{C}_2(P)$. Autrement dit $P\in \Delta$.
\end{proof}
%\DinoAfourColorCommand{%
%\par
%\vspace{1.5cm}
%{
%\centering\LARGE\textcolor{ColorOne}\decoone
%
%}
%}

\endinput
\backmatter
\end{document}