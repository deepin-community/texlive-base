
\tgotitle{Droite et cercle d'Euler}
\tgoshorttoc
\section{Cercle circonscrit à un triangle}
\subsection{Un résultat préliminaire}
On sait depuis le collège  que la droite qui joint les milieux de deux côtés d'un triangle est parallèle au troisième côté. Nous précisons ce résultat à l'aide d'un théorème qui nous sera utile par la suite (voir figure \ref{figmilieux}).

\begin{figure}[ht]
\centering
\begin{tikzpicture}
\coordinate[label=left:$B$](B) at (0,0);
\coordinate[label=right:$C$](C) at (4,0);
\coordinate (T) at ($(B)!1!38:(C)$);	
\coordinate (S) at ($(C)!1!105:(B)$);	
\coordinate[label=above:$A$](A) at at (intersection of C--S and B--T);
\draw (A)--(B);
\draw (A)--(C);
\coordinate [label=right:$B'$] (B') at ($(A)!0.5!(C)$);		
\coordinate [label=left:$C'$] (C') at ($(A)!0.5!(B)$);
\draw[>=stealth,->,semithick](B)--(C);
\draw[>=stealth,->,semithick](C')--(B');
\foreach \point in {A,B,C,C',B'}
\draw[black,fill=white](\point) circle (1.2pt);
\end{tikzpicture}
\figcaption{}\label{figmilieux}
\end{figure}

\begin{thm}
Soit un triangle $ABC$ et soit $B'$ et $C'$ les milieux respectifs de $[AC]$ et $[AB]$. Alors
les droites $(BC)$ et $(C'B')$ sont parallèles et  $\overrightarrow{BC}=2\overrightarrow{C'B'}$.
\label{thmilieux}
\end{thm}

\begin{proof}
La relation de Chasles permet d'écrire
\begin{align*}
\overrightarrow{BC}&=\overrightarrow{BA}+\overrightarrow{AC},\\
\intertext{puis, en utilisant une caractérisation vectorielle du milieu,}
\overrightarrow{BC}&=2\overrightarrow{C'A}+2\overrightarrow{AB'},\\
\overrightarrow{BC}&=2\left(\overrightarrow{C'A}+\overrightarrow{AB'}\right),\\
\overrightarrow{BC}&=2\overrightarrow{C'B'}.\qedhere
\end{align*}
\end{proof}

\subsection{Médiatrices des côtés d'un triangle}
%%%%% Version 1
 La médiatrice d'un segment est par définition la droite perpendiculaire à ce segment en son milieu. Nous admettrons que la médiatrice d'un segment est aussi l'ensemble des points qui sont équidistants des extrémités de ce segment (voir figure \ref{figcerccir}). Nous formulons alors le théorème suivant :
%%%%%%
%%%%%%
%%%%%%
%%%%% Version 2
%La médiatrice d'un segment est par définition l'ensemble des points équidistants des extrémités de ce segment. Il est clair que le milieu du segment appartient à la médiatrice. On a même :
% %
%\begin{thm}
%La médiatrice d'un segment est la droite perpendiculaire à ce segment en son milieu.
%\end{thm}\label{caracmed}
%
%\begin{proof}
%Soit un segment $[AB]$ et $I$ son milieu. Notons $\symcal{E}$ la médiatrice de $[AB]$ et $\symcal{D}$ la droite perpendiculaire à $[AB]$ en $I$. Montrons que $\symcal{E}=\symcal{D}$ par double inclusion.
%Soit un point $M$ appartenant à la médiatrice $\symcal{E}$ de $[AB]$. On a $IA=IB$ et $MA=MB$ par définition de la médiatrice. Les triangles $MIA$ et $MIB$ sont donc superposables. Si $\alpha$ et $\beta$ sont les mesures respectives de $\widehat{MIA}$ et  $\widehat{MIB}$, on en déduit que $\alpha=\beta$. L'angle $\widehat{AIB}$ et plat de mesure $\alpha+\beta$ : 
%\[\left.\begin{aligned} \alpha=\beta&\\\alpha+\beta=180\degre\end{aligned}\right\}\implies \alpha=\beta=90\degre.\]
%Le point $M$ appartient donc à la droite $\symcal{D}$ : $\symcal{E}\subset\symcal{D}$.
%
%Réciproquement, supposons que $M\in\symcal{D}$. Considérons la réflexion $s$ d'axe $\symcal{D}$ : on a $s(A)=B$ et $s(M)=M$ (car l'ensemble des points invariants de $s$ est la droite $\symcal{D}$). Par conservation des distances, on a $MA=s(M)s(A)=MB$. Donc $M$ appartient à $\symcal{E}$ : $\symcal{E}\subset\symcal{D}$.
%
%On a donc bien $\symcal{E}=\symcal{D}$.
%\end{proof}
%
%

\begin{thm}
Les médiatrices des trois côtés d'un triangle sont concourantes en un point équidistant des trois sommets et centre de l'unique cercle passant par ces derniers.
\end{thm}

\vspace{1ex}

\begin{figure}[ht]
\centering
\begin{tikzpicture}
\coordinate[label=left:$B$](B) at (0,0);
\coordinate[label=right:$C$](C) at (4,0);
\coordinate (T) at ($(B)!1!38:(C)$);	
\coordinate (S) at ($(C)!1!105:(B)$);	
\coordinate[label=above:$A$](A) at at (intersection of C--S and B--T);
\draw (A)--(B)--(C)--cycle;
\coordinate [label=below:$A'$] (A') at ($(B)!0.5!(C)$);
\coordinate [label=right:$B'$] (B') at ($(A)!0.5!(C)$);		
\coordinate [label=left:$C'$] (C') at ($(A)!0.5!(B)$);
\coordinate (X) at ($(C')!1!90:(B)$);	
\coordinate (Y) at ($(B')!1!90:(C)$);	
\coordinate [label=above:$O${\rule[-1mm]{0mm}{1mm}}](O) at (intersection of C'--X and B'--Y);
\draw(C')--(O);
\draw (B')--(O);
\draw[dotted,semithick](A')--(O);
\node [draw] at (O) [circle through={(C)}] {};
\draw[gray](O)--(A);
\draw[gray](O)--(B);
\draw[gray](O)--(C);
\foreach \point in {A,B,C,C',A',B',O}
\draw[black,fill=white](\point) circle (1.2pt);	
\end{tikzpicture}
\figcaption{}\label{figcerccir}
\end{figure}



\begin{proof}
Reprenons la figure \ref{figmilieux} et nommons $A'$ le milieu de $[BC]$. Les médatrices des côtés $[AB]$ et $[AC]$ sont sécantes en un point $O$. (On rappelle qu'un triangle est la donnée de trois points non alignés, autrement dit, un triangle aplati ne peut être considéré comme un triangle : cela assure que les deux médiatrices sont bien sécantes.) Le point $O$ se trouve à la fois sur la médiatrice de $[AB]$ et sur la médiatrice de $[AC]$ : il est donc à la fois équidistant de $A$ et $B$ ($OA=OB$) et équidistant de $A$ et $C$ ($OA=OC$). On a donc finalement $OB=OC$, ce qui assure que la médiatrice de $[BC]$ passe par $O$. Les trois médiatrices sont donc concourantes en $O$, point équidistant des trois sommets du triangle et par conséquent centre d'un cercle passant par ces sommets. Pour établir l'unicité d'un tel cercle,  il suffit de remarquer que le centre d'un cercle passant par les sommets est nécessairement équidistant des sommets donc point de concours des médiatrices.
\end{proof}

Le cercle en question est le \emph{cercle circonscrit} au triangle. 

\subsection{Un cas particulier}
Le cas du cercle circonscrit au triangle rectangle est bien connu (voir figure \ref{figtrec}).

\begin{figure}[ht]
\centering
\begin{tikzpicture}
\coordinate[label=left:$B$](B) at (0,0);
\coordinate[label=right:$C$](C) at (4,0);
\coordinate (T) at ($(B)!1!30:(C)$);	
\coordinate (S) at ($(C)!1!120:(B)$);	
\coordinate[label=above:$A$](A) at at (intersection of C--S and B--T);
\draw (A)--(B)--(C)--cycle;
\coordinate [label=below:$A'$] (A') at ($(B)!0.5!(C)$);
\coordinate [label=left:$B'${\rule{1.3mm}{0mm}}] (B') at ($(A)!0.5!(C)$);		
\coordinate [label=right:{\rule{1.3mm}{0mm}}$C'$] (C') at ($(A)!0.5!(B)$);
\coordinate (X) at ($(C')!1!90:(B)$);	
\coordinate (Y) at ($(B')!1!90:(C)$);	
\coordinate(O) at (intersection of C'--X and B'--Y);
\draw(C')--(O);
\draw (B')--(O);
\draw[dotted,semithick](A')--(O);
\node [draw] at (O) [circle through={(C)}] {};
\draw[gray](O)--(A);
\draw[gray](O)--(B);
\draw[gray](O)--(C);
\foreach \point in {A,B,C,C',A',B',O}
\draw[black,fill=white](\point) circle (1.2pt);
\end{tikzpicture}
\figcaption{}\label{figtrec}
\end{figure}


\begin{thm}
Si un triangle est rectangle, alors le centre de son cercle circonscrit est le milieu de l'hypoténuse qui est donc un diamètre du cercle. Réciproquement si le cercle circonscrit à un triangle admet l'un des côtés comme diamètre, alors ce triangle est rectangle et admet le côté en question pour hypoténuse.
\label{diamtrrec}
\end{thm}

\begin{proof}
Supposons le triangle $ABC$ rectangle en $A$. La droite $(AB)$ est alors perpendiculaire à la droite $(AC)$.
D'après le théorème \ref{thmilieux} on sait que $\overrightarrow{AB}=2\overrightarrow{B'A'}$ et en particulier que les droites $(AB)$ et $(B'A')$ sont parallèles. On en déduit que les droites $(B'A')$ et $(AC)$ sont perpendiculaires:
\[\left.\begin{aligned} &(AB)\parallel(A'B')\\
\text{et}&\\
&(AB)\perp(AC)\end{aligned}\right\}\implies (A'B')\perp(AC).
\]

La droite $(B'A')$, perpendiculaire au segment $[AC]$ en son milieu, en est la médiatrice. De même, $(C'A')$ est la médiatrice du segment $[AB]$. Le point $A'$, milieu de $[BC]$, est commun à ces deux médiatrices : il est donc le centre du cercle circonscrit au triangle.



Réciproquement, supposons que le milieu $A'$ de $[BC]$ soit le centre du cercle circonscrit. Ce point est alors équidistant de $A$ et $C$, donc est un point de la médiatrice du segment $[AC]$; il en est de même pour $B'$, milieu de ce segment. La droite $(A'B')$ est donc la médiatrice du segment $[AC]$ et lui est  perpendiculaire. Cette droite est aussi parallèle à $(AB)$, en vertu du théorème \ref{thmilieux}. De la même façon que précédemment, on en déduit que les droites $(AB)$ et $(AC)$ sont perpendiculaires. Le triangle $ABC$ est donc rectangle en $A$.
\end{proof}

\begin{remark} La situation représentée sur la figure \ref{figcerccir} est celle d'un triangle acutangle. On comprend assez rapidement (mais ceci ne constitue pas une démonstration) que le centre du cercle circonscrit est intérieur au triangle si les trois angles sont aigus et extérieur dans le cas où l'un des angles est obtus (voir figure \ref{figtrobtu}). Le cas du triangle rectangle constitue en quelque sorte une situation limite…
\end{remark}


\begin{figure}[ht]
\centering
\begin{tikzpicture}
\coordinate[label=left:$B$](B) at (0,0);
\coordinate[label=right:$C$](C) at (4,0);
\coordinate (T) at ($(B)!1!25:(C)$);	
\coordinate (S) at ($(C)!1!140:(B)$);	
\coordinate[label=above:$A$](A) at at (intersection of C--S and B--T);
\draw (A)--(B)--(C)--cycle;
\coordinate [label=above:$A'$] (A') at ($(B)!0.5!(C)$);
\coordinate [label=left:$B'${\rule{1.3mm}{0mm}}] (B') at ($(A)!0.5!(C)$);		
\coordinate [label=right:{\rule{1.3mm}{0mm}}$C'$] (C') at ($(A)!0.5!(B)$);
\coordinate (X) at ($(C')!1!90:(B)$);	
\coordinate (Y) at ($(B')!1!90:(C)$);	
\coordinate [label=below:$O${\rule[-1mm]{0mm}{1mm}}](O) at (intersection of C'--X and B'--Y);
\draw(C')--(O);
\draw (B')--(O);
\draw[dotted,semithick](A')--(O);
\node [draw] at (O) [circle through={(C)}] {};
\draw[gray](O)--(A);
\draw[gray](O)--(B);
\draw[gray](O)--(C);
\foreach \point in {A,B,C,C',A',B',O}
\draw[black,fill=white](\point) circle (1.2pt);	
\end{tikzpicture}
\figcaption{}\label{figtrobtu}
\end{figure}



\section{Centre de gravité d'un triangle}
Il est connu que les médianes d'un triangle sont concourantes en un point situé aux deux tiers de chacune d'elles à partir du sommet. C'est à cette propriété que nous allons maintenant nous intéresser, sous un angle vectoriel (voir figure \ref{figmedianes}).


\begin{thm}\label{thgravit}%
Soit un triangle $ABC$ et soit $A'$, $B'$ et $C'$ les milieux respectifs des côtés $[BC]$, $[AC]$ et $[AB]$.
Les médianes de ce triangle sont concourantes en un point $G$, nommé centre de gravité du triangle et qui est l'unique point du plan à vérifier les relations vectorielles suivantes :
\begin{gather}
\overrightarrow{GA}+\overrightarrow{GB}+\overrightarrow{GC}=\vec{0}\,,\label{eqisobar}\\
\overrightarrow{AG}=\frac23\overrightarrow{AA'}\,,\quad\overrightarrow{BG}=\frac23\overrightarrow{BB'}\quad\text{et}\quad\overrightarrow{CG}=\frac23\overrightarrow{CC'}\,.\label{eqGmediane}
\end{gather}%
\end{thm}

\begin{figure}[ht]
\centering
\begin{tikzpicture}
\coordinate[label=below:$B$](B) at (0,0);
\coordinate[label=below:$C$](C) at (4,0);
\coordinate (T) at ($(B)!1!36:(C)$);	
\coordinate (S) at ($(C)!1!107:(B)$);	
\coordinate[label=above:$A$](A) at at (intersection of C--S and B--T);
\draw (A)--(B)--(C)--cycle;
\coordinate [label=below:$A'$] (A') at ($(B)!0.5!(C)$);
\coordinate [label=right:$B'$](B') at ($(A)!0.5!(C)$);		
\coordinate[label=left:$C'$](C') at ($(A)!0.5!(B)$);
\coordinate (X) at ($(C')!1!90:(B)$);	
\coordinate (Y) at ($(B')!1!90:(C)$);	
\coordinate (O) at (intersection of C'--X and B'--Y);
\coordinate[label=above:$G${\rule[-1mm]{0mm}{1mm}}] (G) at ($(A')!0.333!(A)$);
\coordinate (H) at ($(O)!3!(G)$);
\draw (A)--(A');
\draw[dotted,semithick] (B)--(B');
\draw[dotted,semithick] (C)--(C');
\foreach \point in {A,B,C,A',G,C',B'}
\draw[black,fill=white](\point) circle (1.2pt);
\end{tikzpicture}
\figcaption{}\label{figmedianes}
\end{figure}
	


\begin{proof}
Pour tout point $M$ du plan, on considère le vecteur 
\[
\overrightarrow{v(M)}=\overrightarrow{MA}+\overrightarrow{MB}+\overrightarrow{MC}.
\]

 Nous nous proposons de prouver qu'il existe un point $M$ et un seul pour lequel ce vecteur est nul : cet unique point $M$ sera le point $G$ que nous cherchons. À cet effet nous allons transformer le vecteur $\overrightarrow{v(M)}$, en faisant notamment intervenir un des milieux, ici $A'$, grâce à la relation de Chasles. Nous rappelons que ce milieu est caractérisé par $\overrightarrow{A'B}+\overrightarrow{A'C}=\vec{0}$.
%\DinoXiPhoneCommand{%
\begin{align*}
\overrightarrow{v(M)}&=
(\overrightarrow{MA'}+\overrightarrow{A'A})+
(\overrightarrow{MA'}+\overrightarrow{A'B})+
(\overrightarrow{MA'}+\overrightarrow{A'C}),\\
&=
3\overrightarrow{MA'}+\overrightarrow{A'A}+\underbrace{\overrightarrow{A'B}+\overrightarrow{A'C}}_{\vec{0}}.
\end{align*}

Nous transformons alors le vecteur $\overrightarrow{MA'}$ en faisant intervenir le point $A$ grâce à la relation de Chasles :
\begin{align*}
\overrightarrow{v(M)}&=3(\overrightarrow{MA}+\overrightarrow{AA'})+\overrightarrow{A'A},\\
\intertext{puis, en développant et en remarquant que $\overrightarrow{A'A}=-\overrightarrow{AA'}$,}
\overrightarrow{v(M)}&=3\overrightarrow{MA}+2\overrightarrow{AA'}.
\end{align*}%}
%\DinoiPhoneCommand{%
%\begin{multline*}
%\overrightarrow{MA}+\overrightarrow{MB}+\overrightarrow{MC}=
%(\overrightarrow{MA'}+\overrightarrow{A'A})\\
%+(\overrightarrow{MA'}+\overrightarrow{A'B})+
%(\overrightarrow{MA'}+\overrightarrow{A'C}),\end{multline*}
%soit
%\begin{align*}
%\overrightarrow{MA}+\overrightarrow{MB}+\overrightarrow{MC}&=
%3\overrightarrow{MA'}+\overrightarrow{A'A}+\underbrace{\overrightarrow{A'B}+\overrightarrow{A'C}}_{\vec{0}},\\
%&=3\overrightarrow{MA}+3\overrightarrow{AA'}+\overrightarrow{A'A},\\
%&=3\overrightarrow{MA}+2\overrightarrow{AA'}.
%\end{align*}}
En tenant compte de $\overrightarrow{v(M)}=\overrightarrow{MA}+\overrightarrow{MB}+\overrightarrow{MC}$,
  \begin{align}\overrightarrow{MA}+\overrightarrow{MB}+\overrightarrow{MC}=\vec{0}
 &\iff 3\overrightarrow{MA}+2\overrightarrow{AA'}=\vec{0},\notag\\
 &\iff 3\overrightarrow{AM}=2\overrightarrow{AA'},\notag\\
  &\iff\overrightarrow{AM}=\dfrac23\overrightarrow{AA'}.\label{caraccg}\end{align}

La relation \eqref{caraccg} définit l'unique point $M$ vérifiant $\overrightarrow{MA}+\overrightarrow{MB}+\overrightarrow{MC}=\vec{0}$. Ce point est par définition le centre de gravité du triangle $ABC$ et on le note traditionnellement $G$. Le calcul vectoriel ci-dessus aurait tout aussi bien pu être mené en utilisant $B'$ ou $C'$ plutôt que $A'$. On en déduit les trois relations annoncées en \eqref{eqGmediane}.
\end{proof}

\section{Orthocentre et droite d'Euler}
\subsection{Orthocentre d'un triangle}
\begin{thm}
Les hauteurs issues des trois sommets de tout triangle sont concourantes en un point $H$ nommé orthocentre de ce triangle.
\end{thm}

\begin{figure}[ht]
\centering
\begin{tikzpicture}
\coordinate[label=below:$B$](B) at (0,0);
\coordinate[label=below:$C$](C) at (5,0);
\coordinate (T) at ($(B)!1!36:(C)$);	
\coordinate (S) at ($(C)!1!107:(B)$);	
\coordinate[label=above:$A$](A) at at (intersection of C--S and B--T);
\draw (A)--(B)--(C)--cycle;
\coordinate [label=below:$A'$] (A') at ($(B)!0.5!(C)$);
\coordinate  (B') at ($(A)!0.5!(C)$);		
\coordinate (C') at ($(A)!0.5!(B)$);
\coordinate (X) at ($(C')!1!90:(B)$);	
\coordinate (Y) at ($(B')!1!90:(C)$);	
\coordinate [label=above:$O$](O) at (intersection of C'--X and B'--Y);
\coordinate[label=above:$G${\rule[-1mm]{0mm}{1mm}}] (G) at ($(A')!0.333!(A)$);
\coordinate[label=below:$H$] (H) at ($(O)!3!(G)$);
\draw (A')--(O)--(H)--(A);
\draw (A)--(A');
\foreach \point in {A,B,C,A',O,G,H}
\draw[black,fill=white](\point) circle (1.2pt);
\end{tikzpicture}
\figcaption{}\label{figeuler}
\end{figure}
	

\begin{proof}
Nous en restons aux notations utilisées précédemment.
Si le triangle $ABC$ est équilatéral, la hauteur,  la médiane et la médiatrice relatives à chaque côté sont confondues, les points $O$, centre du cercle circonscrit, et $G$, centre de gravité de $ABC$ le sont également. Ce point unique est évidemment aussi point de concours des hauteurs. Nous supposerons désormais que le triangle $ABC$ n'est pas équilatéral. 

La démonstration repose alors sur la remarque qui suit.
\begin{remark}[Remarque préliminaire]
La hauteur issue de $A$ et la médiatrice de $[BC]$ sont deux droites parallèles, car perpendiculaires à une même troisième.
L'observation de la figure \ref{figeuler} nous incite alors à prolonger la droite $(OG)$ et à fabriquer ainsi une configuration de Thales. Comme nous savons, de par la position de $G$ aux deux tiers de la médiane $[AA']$, que $\overrightarrow{GA}=-2\overrightarrow{GA'}$ nous allons introduire un point $H$ défini  par $\overrightarrow{GH}=-2\overrightarrow{GO}$ pour ensuite ensuite établir (en exploitant de façon vectorielle notre configuration de Thales) que les vecteurs $ \overrightarrow{AH}$ et  $\overrightarrow{OA'}$ sont colinéaires, ce qui établira, sauf cas particulier, que la médiatrice $(OA')$ et la droite $(AH)$ sont parallèles donc que $(AH)$ est la hauteur issue de $A$.
\end{remark}

Soit donc le point $H$ défini par $\overrightarrow{GH}=-2\overrightarrow{GO}$.
De l'égalité \eqref{eqGmediane}, nous déduisons que $3\overrightarrow{AG}=2\overrightarrow{AA'}$, puis que $\overrightarrow{AG}=2\overrightarrow{GA'}$. Par conséquent :
\[\begin{split} \overrightarrow{AH}&=\overrightarrow{AG}+\overrightarrow{GH},\\
&=2\overrightarrow{GA'}+2\overrightarrow{OG},\\
&=2(\overrightarrow{OG}+\overrightarrow{GA'}),\\
&=2\overrightarrow{OA'}.
\end{split}
\]
Les vecteurs $\overrightarrow{OA'}$ et $\overrightarrow{AH}$ sont donc colinéaires. 

\begin{alert}
Si le triangle $ABC$ est rectangle, les points $O$ et $A'$ (et par conséquent les points $A$ et $H$) sont confondus. On ne peut parler des droites $(OA')$ et $(AH)$ : ce cas sera examiné plus tard.
\end{alert}
Supposons provisoirement que le triangle $ABC$ \emph{n'est pas} rectangle : le point $O$, centre du cercle circonscrit n'est pas confondu avec $A'$ (cf. théorème \ref{diamtrrec}) ou encore $\overrightarrow{OA'}\neq\vec{0}$. On peut dire alors que la droite $(AH)$ est parallèle à la droite $(OA')$, elle même perpendiculaire au côté $[BC]$, en tant que médiatrice. La droite $(AH)$ est donc perpendiculaire à $[BC]$ : c'est la hauteur issue de $A$.


Le raisonnement qui vient d'être fait tient toujours si l'on remplace les points $A$ et $A'$ par $B$ et $B'$ ou par $C$ et $C'$. Le point $H$ quant à lui est fixe (il ne dépend que de $O$ et $G$) : il appartient donc à chacune des trois hauteurs. On a donc démontré que les trois hauteurs sont concourantes en $H$ : ce point $H$ est \emph{l'orthocentre} du triangle $ABC$.
\begin{alert}
Si maintenant le triangle est rectangle, par exemple en $A$ : les hauteurs issues de $B$ et $C$ sont respectivement $(BA)$ et $(CA)$ : les hauteurs sont concourantes en $A$. L'orthocentre $H$ est alors confondu avec $A$ de même que le centre du cercle circonscrit  $O$ est confondu avec $A'$. La relation $\overrightarrow{GH} = -2\overrightarrow{GO}$ est toujours vérifiée.\qedhere
\end{alert}
\end{proof}
\subsection{Droite d'Euler}



Nous avons en fait montré un peu plus que ce que nous avions annoncé. 
La relation vectorielle liant $O$, $G$ et $H$ montre que si le triangle $ABC$ n'est pas équilatéral ces trois points sont alignés. La droite support est la \emph{droite d'Euler}. Résumons-nous maintenant :
\begin{thm}
Le centre $O$ du cercle circonscrit, le centre de gravité $G$ et l'orthocentre $H$ d'un triangle $ABC$ vérifient la relation vectorielle 
\begin{equation}\overrightarrow{GH}=-2\overrightarrow{GO}\,.\label{eulerposition}\end{equation}
Ces trois points sont confondus si $ABC$ est équilatéral. Dans le cas contraire, ils sont alignés sur une droite nommée droite d'Euler. 
\label{eulerprecis}
\end{thm}

\section{Cercle d'Euler}
\subsection{Neuf points remarquables}
Le cercle d'Euler est également connu comme cercle des neuf points ou encore cercle de Feuerbach (voir figure \ref{figcercleuler}).

\begin{figure}[ht]
\centering
\begin{tikzpicture}
\coordinate[label=below:$B$](B) at (0,0);
\coordinate[label=below:$C$](C) at (5,0);
\coordinate (T) at ($(B)!1!61:(C)$);	
\coordinate (S) at ($(C)!1!130:(B)$);	
\coordinate[label=above:$A$](A) at at (intersection of C--S and B--T);
\draw (A)--(B)--(C)--cycle;
\coordinate [label=below:$A'$] (A') at ($(B)!0.5!(C)$);
\coordinate [label=right:$B'${\rule[-1mm]{0mm}{1mm}}] (B') at ($(A)!0.5!(C)$);		
\coordinate [label=left:$C'$] (C') at ($(A)!0.5!(B)$);
\coordinate (X) at ($(C')!1!90:(B)$);	
\coordinate (Y) at ($(B')!1!90:(C)$);	
\coordinate(O) at (intersection of C'--X and B'--Y);
\coordinate (G) at ($(A')!0.333!(A)$);
\coordinate[label=below:$H$] (H) at ($(O)!3!(G)$);
\draw[dashed] (A)--(H);
\draw[dashed] (B)--(H);
\draw[dashed] (C)--(H);
\coordinate[label=below:$D$](D) at at (intersection of A--H and B--C);
\coordinate[label=right:$E${\rule[-1mm]{0mm}{1mm}}](E) at at (intersection of B--H and A--C);
\coordinate[label=left:$F$](F) at at (intersection of C--H and B--A);
\coordinate [label=left:$A''$] (A'') at ($(A)!0.5!(H)$);
\coordinate [label=below:$B''$] (B'') at ($(B)!0.5!(H)$);
\coordinate [label=below:$C''$] (C'') at ($(C)!0.5!(H)$);
\draw (A')--(C')--(A'')--(C'')--cycle ;
\draw (B')--(C')--(B'')--(C'')--cycle ;
\coordinate (O') at ($(O)!1.5!(G)$);
\node [draw,dotted,semithick] at (O') [circle through={(A')}] {\rule{2.2cm}{0mm}$\symcal{C}$};
\foreach \point in {A,B,C,B',C',A',H,D,E,F,A'',B'',C''}
\draw[black,fill=white](\point) circle (1.2pt);
\end{tikzpicture}
\figcaption{}\label{figcercleuler}
\end{figure}
	

\begin{thm}
Pour tout triangle, les pieds des hauteurs, les milieux des côtés et les milieux des segments joignant l'orthocentre aux sommets sont situés sur un même cercle.
\end{thm}
\begin{remark}
Suivant les configurations, certains de ces points peuvent être confondus.
\end{remark}
\begin{proof} Soit un triangle $ABC$ d'orthocentre $H$, $A'$, $B'$ et $C'$ les milieux respectifs des côtés $[BC]$, $[CA]$ et $[AB]$, $D$, $E$ et $F$ les pieds des hauteurs issues respectivement de $A$, $B$ et $C$ et $A''$, $B''$ et $C''$ les milieux respectifs des segments $[AH]$, $[BH]$ et $[CH]$.

En vertu du théorème \ref{thmilieux}, appliqué successivement aux triangles $ABC$ et $HBC$
\XSmartphoneCommand{%
\begin{equation}
\left.\begin{aligned}
\overrightarrow{C'B'}=\frac12\overrightarrow{BC}&\\
\overrightarrow{B''C''}=\frac12\overrightarrow{BC}&	
\end{aligned}\right\}\implies \overrightarrow{C'B'}=\overrightarrow{B''C''}\text{ et }(C'B')\parallel(B''C'')\parallel(BC).\label{parallelBC}
\end{equation}}
\SmartphoneCommand{
\begin{equation}
\left.\begin{aligned}
\overrightarrow{C'B'}=\frac12\overrightarrow{BC}&\\
\overrightarrow{B''C''}=\frac12\overrightarrow{BC}&	
\end{aligned}\right\}\implies \left\{\begin{aligned}&\overrightarrow{C'B'}=\overrightarrow{B''C''}\\ \text{et}&\\(&C'B')\parallel(B''C'')\parallel(BC).\end{aligned}\right.\label{parallelBC}
\end{equation}}

De même, en appliquant ce même théorème aux triangles $BAH$ et $CAH$
\XSmartphoneCommand{%
\begin{equation}
\left.\begin{aligned}
\overrightarrow{C'B''}=\frac12\overrightarrow{AH}&\\
\overrightarrow{B'C''}=\frac12\overrightarrow{AH}&	
\end{aligned}\right\}\implies \overrightarrow{C'B''}=\overrightarrow{B'C''}\text{ et }(C'B'')\parallel(B'C'')\parallel(AH).\label{parallelAH}
\end{equation}}
\SmartphoneCommand{%
\begin{equation}
\left.\begin{aligned}
\overrightarrow{C'B''}=\frac12\overrightarrow{AH}&\\
\overrightarrow{B'C''}=\frac12\overrightarrow{AH}&	
\end{aligned}\right\}\implies\left\{\begin{aligned}&\overrightarrow{C'B''}=\overrightarrow{B'C''}\\ \text{et}&\\(&C'B'')\parallel(B'C'')\parallel(AH).\end{aligned}\right.\label{parallelAH}
\end{equation}}

À ce stade nous avons établi que $B'C'B''C''$ est un parallélogramme. Nous savons également que $(AH)$, en tant que hauteur du triangle, est perpendiculaire à $(BC)$. Grâce à cette remarque et aux relations de parallélisme avec $(BC)$ et $(AH)$ établies ci-dessus dans les implications \eqref{parallelBC} et \eqref{parallelAH}, nous pouvons affirmer que $B'C'B''C''$ est un rectangle. Ce rectangle est inscriptible dans le cercle  admettant $[B'B'']$ et $[C'C'']$ pour diamètres. 

On montre de la même façon que $A'C'A''C''$ est un rectangle inscriptible dans le même cercle  de diamètres $[C'C'']$ et $[A'A'']$. Les six points $A'$, $A''$, $B'$, $B''$, $C'$ et $C''$ appartiennent donc au cercle $\symcal{C}$ de diamètre, par exemple, $[A'A'']$.

Considérons alors le point $D$ : si le triangle $ABC$ est isocèle en $A$, il est confondu avec $A'$ et appartient donc au cercle $\symcal{C}$. Sinon $D$ et $A'$ sont distincts et le triangle $ADA'$ est rectangle en $D$ : d'après le théorème \ref{diamtrrec}, $D$ appartient au cercle $\symcal{C}$ de diamètre $[AA']$. On montre de la même façon que les points $E$ et $F$ appartiennent  à $\symcal{C}$.
\end{proof}
\begin{remark} Le cercle d'Euler est le cercle circonscrit au triangle des milieux $A'B'C'$.
\end{remark}
%
\subsection{Centre du cercle d'Euler}
Nous allons préciser la remarque précédente (voir figure \ref{centreeuler}).
\begin{thm}
Soit un triangle $ABC$ et soit $A'$, $B'$ et $C'$ les milieux respectifs de $[BC]$, $[CA]$ et $[AB]$.
Soit $O$ le centre du cercle circonscrit à $ABC$, $G$ son centre de gravité et $H$ son orthocentre :
\begin{itemize}
\item le point $O$ est l'orthocentre du triangle $A'B'C'$ ;
\item le point $G$ est le centre de gravité du triangle $A'B'C'$ ;
\item le centre $O'$ du cercle circonscrit au triangle $A'B'C'$ appartient à la droite d'Euler et vérifie la relation vectorielle
\[\overrightarrow{GO'}=-\frac12\overrightarrow{GO}.\]
\end{itemize} 
\end{thm}

\begin{figure}[ht]
\centering
\begin{tikzpicture}
\coordinate[label=below:$B$](B) at (0,0);
\coordinate[label=below:$C$](C) at (5,0);
\coordinate (T) at ($(B)!1!32:(C)$);	
\coordinate (S) at ($(C)!1!100:(B)$);	
\coordinate[label=above:$A$](A) at at (intersection of C--S and B--T);
\draw (A)--(B)--(C)--cycle;
\coordinate [label=below:$A'$] (A') at ($(B)!0.5!(C)$);
\coordinate [label=right:$B'${\rule[-1mm]{0mm}{1mm}}] (B') at ($(A)!0.5!(C)$);		
\coordinate [label=left:$C'${\rule[-1mm]{0mm}{1mm}}] (C') at ($(A)!0.5!(B)$);
\coordinate (X) at ($(C')!1!90:(B)$);	
\coordinate (Y) at ($(B')!1!90:(C)$);	
\coordinate[label=above:$O$] (O) at (intersection of C'--X and B'--Y);
\coordinate [label=above:$G$]  (G) at ($(A')!0.333!(A)$);
\coordinate[label=below:$H$] (H) at ($(O)!3!(G)$);
\draw (A)--(B)--(C)--cycle ;
\coordinate[label=above:$O'$]  (O') at ($(O)!1.5!(G)$);
\draw [dotted,semithick](A')--(B')--(C')--cycle;
\draw (O)--(H);
\foreach \point in {A,B,C,B',C',A',O,O',G,H}
\draw[black,fill=white](\point) circle (1.2pt);	
\end{tikzpicture}
\figcaption{}\label{centreeuler}
\end{figure}


%%\DinoiPadAirCommand{\pagebreak}

\begin{proof}
\begin{itemize}
\item La droite $(A'O)$, médiatrice de $[BC]$, est perpendiculaire à la droite $(BC)$, donc également à la droite $(C'B')$ parallèle à $(BC)$ d'après le théorème \ref{thmilieux}. Dans le triangle $A'B'C'$, le point $O$ appartient donc à la hauteur issue de $A'$. On montre de la même façon que $O$ appartient à la hauteur issue de $B'$. Le point $O$ est donc l'orthocentre de $A'B'C'$.
\item Grâce à la relation \eqref{eqGmediane} du théorème \ref{thgravit}, on obtient
\[\overrightarrow{GA'}=-\frac12\overrightarrow{GA},\quad \overrightarrow{GB'}=-\frac12\overrightarrow{GC}\quad\text{et}\quad\overrightarrow{GC'}=-\frac12\overrightarrow{GC}.\]
On en déduit
\begin{align*}
\overrightarrow{GA'}+\overrightarrow{GB'}+\overrightarrow{GC'}&=-\frac12\Bigl(\,\underbrace{\overrightarrow{GA}+\overrightarrow{GB}+\overrightarrow{GC}}_{\vec{0}}\,\Bigr),\\
\overrightarrow{GA'}+\overrightarrow{GB'}+\overrightarrow{GC'}&=\vec{0},
\end{align*}
qui prouve que $G$ est le centre de gravité de $A'B'C'$.
\item Si l'on applique la relation \eqref{eulerposition} du théorème \ref{eulerprecis} au triangle $A'B'C'$, on obtient $\overrightarrow{GO}=-2\overrightarrow{GO'}$. On en déduit la relation
\[\overrightarrow{GO'}=-\frac12\overrightarrow{GO}.\qedhere\]\end{itemize}
\end{proof}

\begin{remark}
La configuration des  points  $(O, O', G, H)$  est remarquable. On dit qu'ils sont en \emph{division harmonique} ; cette notion fera l'objet d'une prochaine parution de la \emph{Mathematica dinosaurorum}.
\end{remark}
\endinput
