\tgotitle{Puissance d'un point par rapport à un cercle}

\section{Mesure algébrique}
On définit la mesure algébrique d'un bipoint~$(A,B)$ dans un repère de la droite~$(AB)$ comme la différence des abscisses $x_B-x_A$ (voir figure~\ref{figmesalg}). Dans le contexte euclidien qui nous intéresse ici, les repères des droites sont normés : la mesure algébrique ne dépend que de l'orientation de la droite (elle est définie \frquote{au signe près}). On notera cependant  que le produit ou le quotient de deux mesures algébriques sont quant à eux indépendants du choix du repère, puisqu'en cas de changement d'orientation les signes des deux quantités considérées changent simultanément. 

\begin{figure}[ht]
\centering
\begin{tikzpicture}
\coordinate[label=above:$A$] (A) at (-1.3,0);
\coordinate [label=below:$x_A$] (xA) at (-1.3,0);
\coordinate[label=above:$B$] (B) at (1.7,0);
\coordinate [label=below:$x_B$] (xB) at (1.7,0);
\coordinate (Z) at ($(A)!1.1!(B)$);
\coordinate (T) at ($(B)!1.1!(A)$);
\draw (Z)--(T);
\node[below] at ($(A)!0.5!(B)$){$\mesalg{AB}=x_B-x_A$};
\foreach \point in {A,B}
\draw[black,fill=white](\point) circle (1.2pt);	
\end{tikzpicture}
\figcaption{}\label{figmesalg}
\end{figure}

De la définition qui vient d'être donnée, on déduit immédiatement que :
\begin{itemize}
\item Pour tout point $A$, $\mesalg{AA}=0$.
\item Pour tout couple de points $A$ et $B$, $\mesalg{BA}$ et $\mesalg{AB}$ sont deux nombres opposés.
\end{itemize} 
Plus généralement, on peut énoncer le
\begin{thm}\label{thchasles}%
Soit $A$, $B$ et $C$ trois points donnés sur une droite dans un ordre quelconque. On a l'égalité suivante (relation de Chasles pour les mesures algébriques) 
\[\mesalg{AC}=\mesalg{AB}+\mesalg{BC}.\]
\end{thm}

\begin{proof} La vérification est immédiate.
\begin{align*}\mesalg{AB}+\mesalg{BC}&=(x_B-x_A)+(x_C-x_B),\\
&=x_C-x_A,\\
&=\mesalg{AC}.\qedhere
\end{align*}
\end{proof}

Nous donnons maintenant deux compléments sur la notion de mesure algébrique, qui faciliteront la compréhension de la suite.

\begin{remark}[Lien avec le produit d'un vecteur par un réel]
La notion de mesure algébrique est fortement liée à celle de produit d'un vecteur par un réel. Si $O$ et $A$ sont deux points distincts, $\lambda$ un réel et $M$ un point quelconque du plan, on a l'équivalence
\[
(\Vector{OM}=\lambda\Vector{OA})\iff(M\in (OA) \text{ et }\mesalg{OM}=\lambda\mesalg{OA}).
\]
Il s'en déduit que le milieu $I$ d'un segment $[AB]$ peut être caractérisé par des relations entre mesures algébriques, qu'il est parfois intéressant de substituer aux égalités vectorielles correspondantes :
\begin{align}
I \text{ est le milieu de }[AB]&\iff \mesalg{AI}=\mesalg{IB}=\frac12\mesalg{AB},\\
&\iff\mesalg{IA}+\mesalg{IB}=0.\label{eqcaracmilieu}
\end{align}
\end{remark}

\begin{remark}[Lien avec le produit scalaire]
La mesure algébrique peut également intervenir dans l'expression du cosinus d'un angle et du produit scalaire. Précisément, si $A$ et $B$ sont deux points tous deux distincts du point $O$, le produit scalaire $\Vector{OA}\cdot\Vector{OB}$ peut être défini par 
\begin{equation}\Vector{OA}\cdot\Vector{OB}=OA\times OB\times \cos(\widehat{AOB}).\label{pscal1}
\end{equation}
%\[
%\Vector{OA}\cdot\Vector{OB}=OA\times OB\times \rho([OA),[OB)),
%\]
%où $\rho([OA),[OB))$ désigne le rapport de projection orthogonale de la demi-droite $[OB)$ sur la demi-droite $[OA)$. 

On désigne par $H$ le projeté orthogonal de $B$ sur la \emph{droite} $(OA)$.
La définition~\eqref{pscal1} permet d'affirmer que le produit scalaire $\Vector{OA}\cdot\Vector{OB}$ est de même signe que $\cos(\widehat{AOB})$, soit positif, si l'angle $\widehat{AOB}$ est aigu et négatif si cet angle est obtus. Le point $H$ appartient dans ce  dernier cas à la demi-droite opposée à $[OA)$ (voir figure~\ref{figpscal0}). On a donc :
\begin{equation*}
\cos(\widehat{AOB})=
\begin{cases}
\hphantom{-}OH/OB&\text{si $\widehat{AOB}$ est aigu},\\
-OH/OB&\text{si $\widehat{AOB}$ est obtus}.\\
\end{cases}
\end{equation*}

%\[\cos(\widehat{AOB})=\frac{\mesalg{OH}}{\mesalg{OB}}.\]
En reportant dans~\eqref{pscal1} et en simplifiant par $OB$, on obtient une autre expression du produit scalaire :
\begin{equation*}
\Vector{OA}\cdot\Vector{OB}=
\begin{cases}
\hphantom{-}OA\times OH&\text{si $\widehat{AOB}$ est aigu},\\
-OA\times OH&\text{si $\widehat{AOB}$ est obtus}.
\end{cases}%\label{pscal0}
\end{equation*}



On remarque alors que les demi-droites $[OA)$ et $[OH)$ sont de même sens si $\widehat{AOB}$ est aigu, et de sens contraire si $\widehat{AOB}$ est obtus. D'où une expression du produit scalaire utilisant les mesures algébriques :
\begin{equation}
\Vector{OA}\cdot\Vector{OB}=\mesalg{OA}\times\mesalg{OH}.\label{pscal0}
\end{equation}

%\dinofig{\figcaption{}\label{figpscal0}}
\XSmartphoneCommand{
\begin{figure}[ht]
\hfill
\begin{tikzpicture}
\coordinate[label=left:$O$] (O) at (0,0);
\coordinate(X) at (1,0);
\coordinate[label=below:$B$](B) at ($(O)!3!350:(X)$);
\coordinate(Y) at ($(O)!0.5!(B)$);
\coordinate[label=above:$H$](H) at ($(Y)!1!50:(B)$);
\coordinate[label=right:$A$](A) at ($(O)!1.3!(H)$);
\draw[semithick,->] (O)--(B);
\draw[semithick,->] (O)--(A);
\draw[dashed](B)--(H);
\foreach \point in {O,H}
\draw[black,fill=white](\point) circle (1.2pt);	
\end{tikzpicture}
\hfill\hfill
\begin{tikzpicture}
\coordinate[label=below:$O$] (O) at (0,0);
\coordinate(X) at (1,0);
\coordinate[label=above:$B$](B) at ($(O)!2.9!160:(X)$);
\coordinate(Y) at ($(O)!0.5!(B)$);
\coordinate[label=below:$H$](H) at ($(Y)!1!50:(B)$);
\coordinate(Z) at ($(O)!1.2!(H)$);
\coordinate[label=above:$A$](A) at ($(O)!-0.6!(H)$);
\draw[semithick,->] (O)--(B);
\draw[semithick,->] (O)--(A);
\draw(O)--(Z);
\draw[dashed](B)--(H);
\foreach \point in {O,H}
\draw[black,fill=white](\point) circle (1.2pt);	
\end{tikzpicture}
\hfill\null
\figcaption{}\label{figpscal0}
\end{figure}
}
\SmartphoneCommand{
\begin{figure}[ht]
\centering
\begin{tikzpicture}
\coordinate[label=left:$O$] (O) at (0,0);
\coordinate(X) at (1,0);
\coordinate[label=below:$B$](B) at ($(O)!3!350:(X)$);
\coordinate(Y) at ($(O)!0.5!(B)$);
\coordinate[label=above:$H$](H) at ($(Y)!1!50:(B)$);
\coordinate[label=right:$A$](A) at ($(O)!1.3!(H)$);
\draw[semithick,->] (O)--(B);
\draw[semithick,->] (O)--(A);
\draw[dashed](B)--(H);
\foreach \point in {O,H}
\draw[black,fill=white](\point) circle (1.2pt);	
\end{tikzpicture}
\\[1ex]
\begin{tikzpicture}
\coordinate[label=below:$O$] (O) at (0,0);
\coordinate(X) at (1,0);
\coordinate[label=above:$B$](B) at ($(O)!2.9!160:(X)$);
\coordinate(Y) at ($(O)!0.5!(B)$);
\coordinate[label=below:$H$](H) at ($(Y)!1!50:(B)$);
\coordinate(Z) at ($(O)!1.2!(H)$);
\coordinate[label=above:$A$](A) at ($(O)!-0.6!(H)$);
\draw[semithick,->] (O)--(B);
\draw[semithick,->] (O)--(A);
\draw(O)--(Z);
\draw[dashed](B)--(H);
\foreach \point in {O,H}
\draw[black,fill=white](\point) circle (1.2pt);	
\end{tikzpicture}
\figcaption{}\label{figpscal0}
\end{figure}
}
\begin{example}[Remarques]
\begin{enumerate}
\item L'angle noté $\widehat{AOB}$ est un angle \emph{non orienté} de demi-droites ; autrement dit, il n'y a pas lieu de distinguer  $\widehat{AOB}$ et $\widehat{BOA}$. Il en découle que le produit scalaire est commutatif (le terme \frquote{symétrique} serait sans doute plus adapté).

On aurait donc pu tout aussi bien utiliser le projeté de $A$ sur $(OB)$. 
\item L'indépendance du cosinus par rapport à l'ordre des deux demi droites $[OA)$ et $[OB)$ provient de ce que $\cos(\widehat{AOB})$ est égal au rapport de projection orthogonale, noté $\rho([OA),[OB))$, de  la demi-droite $[OB)$ sur la demi-droite $[OA)$. On sait en effet (c'est en général considéré comme un des axiomes de la géométrie plane) que
\[\rho([OA),[OB))=\rho([OB),[OA)).\]

 \item Lorsque les points $A$ et $B$ sont confondus, on a 
 \begin{align*}
 \Vector{OA}\cdot\Vector{OB}&=\Vector{OA}\cdot\Vector{OA},\\
 &=OA^2\times \cos(\widehat{AOA}).
 \end{align*}
 L'angle $\widehat{AOA}$ est nul : on a donc $\cos(\widehat{AOA})=1$ d'où finalement 
 \begin{equation}
\Vector{OA}\cdot\Vector{OA}=\bigl({\Vector{OA}}\bigr)^2=OA^2.\label{eqcarres}
\end{equation}
\end{enumerate}
\end{example}
\end{remark}

\section{Point et cercle}
\subsection{Une définition}
Étant donné un cercle $\symcal{C}$, de centre $O$ et de rayon $r$, et un point $P$ quelconque, la puissance du point $P$ par rapport au cercle $\symcal{C}$ est par définition le nombre 
\[\symcal{C}(P)=OP^2-r^2 \text{ (voir figure~\ref{figdefppc})}.\]

\begin{figure}[ht]
\centering
\begin{tikzpicture}
\coordinate[label=below:$O$] (O) at (2,0);
\coordinate(X) at (3,0);
\node[draw,label=above:$\symcal{C}$] (c) at (O) [circle through={(X)}]{};
\coordinate[label=right:$P$] (P) at ($(O)!3.2!15:(X)$);
\coordinate[label=right:$A$](A) at ($(O)!1!312:(X)$);
\draw[semithick](P)--(O)--(A);
\node[right] at (X){$\symcal{C}(P)=OP^2-OA^2$};
\node[right=1.3cm,below=1mm] at (X){$\text{pour tout }A\in\symcal{C}$};
\foreach \point in {P,O,A}
\draw[black,fill=white](\point) circle (1.2pt);	
\end{tikzpicture}
\figcaption{}\label{figdefppc}
\end{figure}

Par conséquent, la puissance d'un point $P$ par rapport à un cercle est :
\begin{itemize}
\item strictement positive si le point $P$ est extérieur au cercle;
\item nulle si le point $P$ appartient au cercle;
\item strictement négative et au minimum égale à $-r^2$ si le point $P$ est intérieur au cercle.
\end{itemize}

\subsection{Avec une tangente}\label{subsectangente}
On se place ici dans le cas où le point $P$ est extérieur au cercle. Il est alors possible de s'intéresser aux tangentes issues de $P$.
\begin{thm}
Soit un cercle $\symcal{C}$ de centre $O$, un point $P$ extérieur au cercle et $T$ le point de contact de l'une des tangentes à $\symcal{C}$ issues de $P$. La puissance de $P$ par rapport à $\symcal{C}$ vaut
\[\symcal{C}(P)=PT^2.\]
\end{thm}


\begin{proof}
On pourra se référer à la figure \ref{figtangente}. La droite $(PT)$ étant tangente à $\symcal{C}$, le triangle $OTP$ est rectangle en $T$. D'après le théorème de Pythagore, on a 
\[\symcal{C}(P)=OP^2-OT^2=PT^2.\qedhere\]
\end{proof}

\begin{figure}[ht]
\centering
\begin{tikzpicture}
\coordinate[label=left:$O$] (O) at (2,0);
\coordinate(X) at (3.5,0);
\node[draw,label=above:$\symcal{C}$] (c) at (O) [circle through={(X)}]{};
\coordinate[label=right:$P$] (P) at ($(O)!1.7!345:(X)$);
\coordinate(Y) at ($(O)!0.5!(P)$);
\node[draw,dashed] (x) at (Y) [circle through={(P)}]{};
\coordinate[label=above:$T$](T) at  (intersection 2 of c and x);
\draw (P)--(T)--(O)--cycle;
\foreach \point in {P,T,O}
\draw[black,fill=white](\point) circle (1.2pt);	
\end{tikzpicture}
\figcaption{}\label{figtangente}
\end{figure}

\subsection{Avec une sécante}\label{subsecsecante}
Le point $P$ est cette fois-ci quelconque. On considère alors une droite sécante au cercle et passant par $P$ (voir la figure \ref{figsecante}).
\begin{thm}
Soit un cercle $\symcal{C}$ de centre $O$, un point $P$ quelconque et une droite passant par $P$ sécante au cercle en $A$ et $B$. La puissance de $P$ par rapport à $\symcal{C}$ vaut alors
\[\symcal{C}(P)=\mesalg{PA}\times\mesalg{PB}.\]
\end{thm}

\begin{figure}[ht]
\centering
\begin{tikzpicture}
\coordinate[label=below:$O$] (O) at (2,0);
\coordinate(X) at (3.5,0);
\node[draw,label=above:$\symcal{C}$] (c) at (O) [circle through={(X)}]{};
\coordinate[label=left:$A$] (A) at ($(O)!1!155:(X)$);
\coordinate(A') at ($(A)!2!(O)$);
\coordinate[label=right:$P$] (P) at ($(O)!1.8!35:(X)$);
\coordinate(Z) at ($(A')!0.5!(P)$);
\node (x) at (Z) [circle through={(P)}]{};
\coordinate[label=above:$B$](B) at  (intersection 2 of c and x);
\coordinate[label=above:$I$](I) at ($(A)!0.5!(B)$);
\draw (P)--(B)--(I)--(A);
\draw(A)--(O)--(B);
\draw(P)--(O)--(I);
\foreach \point in {A,P,B,O,I}
\draw[black,fill=white](\point) circle (1.2pt);	
\end{tikzpicture}
\figcaption{}\label{figsecante}
\end{figure}

\begin{proof}
Soit $I$ le milieu de $[AB]$ : $AOB$ étant isocèle en $O$, la droite $(OI)$ est la médiatrice de $[AB]$. Les triangles $OIP$ et $OIA$ sont rectangles en $I$. On a donc $OP^2=OI^2+IP^2$ et $OA^2=OI^2+IA^2$.

 Par conséquent,
\begin{align*}
\symcal{C}(P)&=PO^2-OA^2,\\
&=(OI^2+IP^2)-(OI^2+IA^2),\\
\intertext{soit en réduisant,}
\symcal{C}(P)&=IP^2-IA^2,\\
\intertext{puis en utilisant une identité remarquable et les égalités~\eqref{eqcarres} et~\eqref{eqcaracmilieu}}
\symcal{C}(P)&=\mesalg{PI}^2-\mesalg{IA}^2,\\
&=(\mesalg{PI}+\mesalg{IA})(\mesalg{PI}-\mesalg{IA}),\\
&=(\mesalg{PI}+\mesalg{IA})(\mesalg{PI}+\mesalg{IB}),\\
\intertext{et enfin, grâce au théorème~\ref{thchasles}}
\symcal{C}(P)&=\mesalg{PA}\times\mesalg{PB}.\qedhere
\end{align*}

\end{proof}

\begin{example}[Remarques]
\begin{enumerate}
\item Il est remarquable que la quantité $\mesalg{PA}\times\mesalg{PB}$ ne dépende pas de la sécante choisie : autrement dit, si deux sécantes passant par $P$ coupent le cercle respectivement en $A$ et $B$ et en $C$ et $D$, alors
\[\mesalg{PA}\times\mesalg{PB}=\mesalg{PC}\times\mesalg{PD}.\]
\item Si $P$ est intérieur au cercle, les demi-droites $[PA)$ et $[PB)$ sont de sens contraires : on retrouve le fait que $\symcal{C}(P)$ est dans ce cas négatif.
\item Une tangente issue de $P$, comme on l'a évoqué en \ref{subsectangente}, peut être considérée comme une position limite d'une sécante pivotant autour de $P$, les points $A$ et $B$ étant alors confondus en $T$ : on aura bien de ce point de vue $\mesalg{PA}\times\mesalg{PB}=PT^2$.
\end{enumerate}
\end{example} 

\subsection{Utilisation du produit scalaire}
On utilise ici la relation~\eqref{pscal0} pour donner une autre expression de la puissance d'un point par rapport à un cercle (voir figure~\ref{figpscal2}).
\begin{thm}
Soit un cercle $\symcal{C}$ de centre $O$, un point $P$ quelconque, et deux points de $\symcal{C}$ diamétralement opposés $A_1$ et $A_2$. La puissance de $P$ par rapport à $\symcal{C}$ vaut alors
\[\symcal{C}(P)=\Vector{PA_1}\cdot\Vector{PA_2}.\]
\end{thm}

\begin{figure}[ht]
\centering
\begin{tikzpicture}
\coordinate[label=below:$O$] (O) at (2,0);
\coordinate(X) at (3.5,0);
\node[draw,label=above:$\symcal{C}$] (c) at (O) [circle through={(X)}]{};
\coordinate[label=left:$A_1$] (A) at ($(O)!1!160:(X)$);
\coordinate[label=right:$A_2$] (A') at ($(A)!2!(O)$);
\coordinate[label=right:$P$] (P) at ($(O)!2.2!32:(X)$);
\coordinate(Z) at ($(A')!0.5!(P)$);
\node (x) at (Z) [circle through={(P)}]{};
\coordinate[label=above:$B$](B) at  (intersection 2 of c and x);
\draw [semithick,->](P)--(A);
\draw[semithick,->](P)--(A');
\draw(A')--(B);
\foreach \point in {P,B,O}
\draw[black,fill=white](\point) circle (1.2pt);	
\end{tikzpicture}
\figcaption{}\label{figpscal2}
\end{figure}

\begin{proof}%\sloppy
Tout d'abord, les droites $(PA_1)$ et $(PA_2)$ ne peuvent être toutes les deux tangentes à $\symcal{C}$ : en effet, deux tangentes en des points diamétralement opposés sont parallèles disjointes et ne peuvent avoir le point $P$ en commun. Quitte à permuter $A_1$ et $A_2$, il n'est donc pas restrictif de supposer que la droite $(PA_1)$ est sécante (et non tangente) à $\symcal{C}$.

Soit alors $B$ la seconde intersection de la droite $(PA_1)$ avec le cercle. Le segment $[A_1A_2]$ étant un diamètre du cercle, le point $B$ est le projeté orthogonal de $A_2$ sur $(PA_1)$. On a donc d'après \ref{subsectangente}
\begin{align*}
\symcal{C}(P)&=\mesalg{PA_1}\times\mesalg{PB},\\
&=\Vector{PA_1}\cdot\Vector{PA_2}.\qedhere
\end{align*}
\end{proof}


\begin{remark}
On se persuadera facilement, en observant la figure \ref{figpscal3}, que le raisonnement précédent est indépendant de la position de $P$ par rapport au cercle.
\end{remark}
%\dinofig[-0.5]{\figcaption{}\label{figpscal3}}

\begin{figure}[ht]
\centering
\begin{tikzpicture}
\coordinate[label=below:$O$] (O) at (2,0);
\coordinate(X) at (3.5,0);
\node[draw,label=above:$\symcal{C}$] (c) at (O) [circle through={(X)}]{};
\coordinate[label=left:$A_1$] (A) at ($(O)!1!160:(X)$);
\coordinate[label=right:$A_2$] (A') at ($(A)!2!(O)$);
\coordinate[label=above:$P$] (P) at ($(O)!0.6!70:(X)$);
\coordinate(Z) at ($(A')!0.5!(P)$);
\node (x) at (Z) [circle through={(P)}]{};
\coordinate[label=above:$B$](B) at  (intersection 2 of c and x);
\draw [semithick,->](P)--(A);
\draw[semithick,->](P)--(A');
\draw(P)--(B);
\draw(A')--(B);
\foreach \point in {P,B,O}
\draw[black,fill=white](\point) circle (1.2pt);	
\end{tikzpicture}
\figcaption{}\label{figpscal3}
\end{figure}

\subsection{Dans un repère orthonormal}
On se place ici dans un repère orthonormal et l'on considère un cercle $\symcal{C}$ de centre $\Omega(a,b)$ et de rayon $r$. On s'intéresse alors à l'expression de la puissance d'un point $M(x,y)$ quelconque par rapport au cercle $\symcal{C}$.
\begin{thm}\label{threp}
Soit dans le plan rapporté à un repère orthonormal un cercle $\symcal{C}$ de centre $\Omega(a,b)$ et de rayon $r$. Soit un point $M(x,y)$. La puissance de $M$ par rapport à $\symcal{C}$ est 
\[\symcal{C}(M) = x^2+y^2-2ax-2by+a^2+b^2-r^2.\]
\end{thm}

\begin{proof}
On a 
\begin{align*}\symcal{C}(M)&=\Omega M^2-r^2\\
&={(x-a)}^2+{(y-b)}^2-r^2,\\
&=x^2-2ax+a^2+y^2-2by+b^2-r^2.\qedhere
\end{align*}
\end{proof}

De plus, en remarquant que $M\in\symcal{C}$ si et seulement si la puissance de $M$ par rapport à $\symcal{C}$ est nulle, on établit le corollaire suivant :
%\DinoAfourNBCommand{\par\pagebreak}
\begin{coro*}
Dans le plan rapporté à un repère orthonormal, le cercle $\symcal{C}$ de centre $\Omega(a,b)$ et de rayon $r$ admet pour équation
\[x^2+y^2-2ax-2by+c=0,\]
où $c=a^2+b^2-r^2$.
\end{coro*}

\section{Cocyclicité}
On dit que quatre points distincts sont cocycliques s'ils appartiennent à un même cercle (un tel cercle est évidemment unique : ce que l'on affirme en disant que les points sont cocycliques est que cercle circonscrit au triangle formé par trois quelconques de ces points passe par le quatrième). On se propose d'établir le théorème suivant (voir figure~\ref{figcocy}).

\begin{thm} Soit quatre points distincts $A$, $B$, $C$ et $D$ tels que les droites $(AB)$ et $(CD)$ soient sécantes en $P$. Les points $A$, $B$, $C$, $D$ sont cocycliques si et seulement si 
\[\mesalg{PA}\times\mesalg{PB}=\mesalg{PC}\times\mesalg{PD}.\]
\end{thm}

%\dinofig{\figcaption{}\label{figcocy}}
\begin{figure}[ht]
\centering
\begin{tikzpicture}
\coordinate (O) at (2,0);
\coordinate(X) at (4,0);
\node[draw,label=above:$\symcal{C}$] (c) at (O) [circle through={(X)}]{};
\coordinate[label=left:$A$] (A) at ($(O)!1!140:(X)$);
\coordinate[label=below:$B$] (B) at ($(O)!1!310:(X)$);
\coordinate[label=left:$C$] (C) at ($(O)!1!210:(X)$);
\coordinate[label=above:$D$] (D) at ($(O)!1!65:(X)$);
\coordinate[label=right:$\kern3pt P$](P) at  (intersection of A--B and C--D);
\draw(A)--(B);
\draw(C)--(D);
\foreach \point in {A,B,C,D,P}
\draw[black,fill=white](\point) circle (1.2pt);	
\end{tikzpicture}
\figcaption{}\label{figcocy}
\end{figure}

\begin{proof}
Il s'agit de démontrer une équivalence. Nous procéderons donc classiquement en deux temps : d'abord le sens direct, puis la réciproque.

Supposons pour commencer que les points $A$, $B$, $C$, $D$ soient sur un même cercle $\symcal{C}$. Alors les droites $(AB)$ et $(CD)$ étant des sécantes au cercle contenant chacune le point $P$, on a
\[\symcal{C}(P)=\mesalg{PA}\times\mesalg{PB}=\mesalg{PC}\times\mesalg{PD}.\]

Réciproquement, supposons que $\mesalg{PA}\times\mesalg{PB}=\mesalg{PC}\times\mesalg{PD}$. Soit alors $\symcal{C}\;$ le cercle circonscrit au triangle $ABC$ : il nous suffit de prouver que $D\in\symcal{C}$. La droite $(CD)$ coupe le cercle en $C$ et en un second point $M$ tel que
\begin{align*}\mesalg{PC}\times\mesalg{PM}&=\symcal{C}(P)\\
&=\mesalg{PA}\times\mesalg{PB}\\
&=\mesalg{PC}\times\mesalg{PD}, \text{ par hypothèse.}
\end{align*}
On en déduit que $\mesalg{PD}=\mesalg{PM}$, donc que $D=M$ ($P$ ne peut être confondu avec $C$, car alors il serait lui même confondu avec $A$ ou $B$).
\end{proof}


\section{Axe radical de deux cercles}

\subsection{Définition}
On s'intéresse ici à l'ensemble des  points qui ont même puissance par rapport à deux cercles non concentriques donnés.

\begin{thm}
Étant donné deux cercles non concentriques $\symcal{C}_1$ et $\symcal{C}_2$ de centres et de rayons respectifs $(O_1,r_1)$ et $(O_2,r_2)$, l'ensemble des points qui ont la même puissance par rapport à chacun des cercles est une droite $\Delta$ perpendiculaire à $(O_1O_2)$, nommée \emph{axe radical} des deux cercles.
\end{thm}

%\dinofig{\figcaption{}\label{figaxerad0}}
\begin{figure}[ht]
\centering
\begin{tikzpicture}
\clip (0.3,-1.3) rectangle (5.1,1.9);
\coordinate[label=below:$O_1$] (O1) at (1,0);
\coordinate(X1) at (1.6,0);
\node[draw,label=above:$\symcal{C_1}$] (c1) at (O1) [circle through={(X1)}]{};
\coordinate[label=below:$O_2$] (O2) at (3.8,0);
\coordinate[label=below:$I$] (I) at ($(O1)!0.5!(O2)$);
\coordinate(X2) at (5,0);
\node[draw,label=above:$\symcal{C_2}$] (c2) at (O2) [circle through={(X2)}]{};
\coordinate(Y1) at ($(O1)!-1.2!(X1)$);
\coordinate(Y2) at ($(O2)!1.1!(X2)$);
\draw (Y1)--(Y2);
\coordinate(Y) at (2,-1);
\coordinate(U) at (2,0.5);
\node (x) at (Y) [circle through={(U)}]{};
\coordinate(Z1) at  (intersection 1 of c1 and x);
\coordinate(T1) at  (intersection 2 of c1 and x);
\coordinate(Z2) at  (intersection 1 of c2 and x);
\coordinate(T2) at  (intersection 2 of c2 and x);
\coordinate[label=left:$M$](M) at (intersection of Z1--T1 and Z2--T2);
\coordinate(W1) at ($(O1)!0.5!(M)$);
\coordinate(W2) at ($(O2)!0.5!(M)$);
\node (w1) at (W1) [circle through={(M)}]{};
\node(w2) at (W2) [circle through={(M)}]{};
\coordinate[label=below:$H\kern13pt$](H) at  (intersection 1 of w1 and w2);
\coordinate[label=left:$\Delta$](V1) at ($(H)!1.5!(M)$);
\coordinate(V2) at ($(H)!-1.1!(M)$);
\draw[semithick](V1)--(V2);
\foreach \point in {O1,O2,I,M,H}
\draw[black,fill=white](\point) circle (1.2pt);	
\end{tikzpicture}
\figcaption{}\label{figaxerad0}
\end{figure}

\begin{remark}
Si les deux cercles sont concentriques et distincts, aucun point $M$ du plan n'est tel que $\symcal{C}_1(M)=\symcal{C}_2(M)$, c'est une conséquence immédiate de la définition de la puissance d'un point par rapport à un cercle.
\end{remark}

\begin{proof}
Soit un point $M$ du plan et $H$ son projeté orthogonal sur la droite $(O_1O_2)$. Nous supposons que $\symcal{C}_1(M)=\symcal{C}_2(M)$ ; nous allons prouver que le point $H$ occupe une position fixe (indépendante de $M$) sur $(O_1O_2)$, ce qui prouvera que $M$ appartient à la perpendiculaire en $H$ à cette droite. Pour nous éviter une réciproque fastidieuse, nous procéderons par équivalence. Désignons par $I$ le milieu de $[O_1O_2]$ et exprimons l'égalité des puissances (voir la figure~\ref{figaxerad0}).
\begin{align*}
\symcal{C}_1(M)=\symcal{C}_2(M) &\iff O_1M^2-r_1^2=O_2M^2-r_2^2,\\
\intertext{soit, en posant  $k=r_1^2-r_2^2$,}
\symcal{C}_1(M)=\symcal{C}_2(M) &\iff O_1M_1^2-O_2M2^2=k,\\
&\iff (\Vector{O_1M}+\Vector{O_2M})\cdot(\Vector{O_1M}-\Vector{O_2M})=k.
\end{align*}

On remarque alors que 
\begin{equation*}
\Vector{O_1M}+\Vector{O_2M}=\hspace{-0.5em}\underbrace{\Vector{O_1I}+\Vector{O_2I}}_{\substack{\vec{0}\text{, car } I\\\text{milieu de }[O_1O_2]}}\hspace{-0.5em}+\,2\Vector{IM}=2\Vector{IM}
\end{equation*}
et que
\begin{equation*}
\Vector{O_1M}-\Vector{O_2M}=\Vector{O_1M}+\Vector{MO_2}=\Vector{O_1O_2}.
\end{equation*}

On en déduit
\begin{align*}
\symcal{C}_1(M)=\symcal{C}_2(M) &\iff 2\Vector{IM}\cdot\Vector{O_1O2}=k,\\
&\iff 2(\underbrace{\Vector{IH}+\Vector{HM}}_{\Vector{IM}})\cdot\Vector{O_1O_2}=k,\\
&\iff 2\Vector{IH}\cdot\Vector{O_1O_2}+2\underbrace{\Vector{HM}\cdot\Vector{O_1O_2}}_{\substack{0 \text{, car}\\(HM)\perp(O_1O_2)}}=k.
\end{align*}

Puisque les points $I$, $H$, $O_1$ et $O_2$ sont sur la même droite, on peut  passer aux mesures algébriques, et on obtient finalement :
\begin{align}
\symcal{C}_1(M)=\symcal{C}_2(M)&\iff 2\mesalg{IH}\cdot\mesalg{O_1O_2}=k,\notag\\
&\iff\mesalg{IH}=\frac{r_1^2-r_2^2}{2\mesalg{O_1O_2}}.\label{eqposH}
\end{align}


On en déduit que $\symcal{C}_1(M)=\symcal{C}_2(M)$ si et seulement si le projeté orthogonal $H$ de $M$ sur $(O_1O_2)$ occupe la position (fixe) définie par la relation~\eqref{eqposH}. L'ensemble des points $M$ est donc la droite $\Delta$ perpendiculaire à $(O_1O_2)$ passant par $H$.
\end{proof}

\subsection{Détermination pratique}\label{subsecdetprat}
On s'intéresse ici à la position relative des cercles $\symcal{C_1}$ et $\symcal{C_2}$.
\subsubsection{Cercles sécants}
Supposons que les cercles $\symcal{C_1}$ et $\symcal{C_2}$ soient sécants en deux points $A$ et $B$ (voir la figure~\ref{figaxerad1}). On a 
\[\symcal{C}_1(A)=\symcal{C}_2(A)=0 \text{ et }\symcal{C}_1(B)=\symcal{C}_2(B)=0 .
\]
Les points $A$ et $B$ ont tous deux la même puissance par rapport à chacun des deux cercles : l'axe radical $\Delta$ de $\symcal{C}_1$ et $\symcal{C}_2$ est donc la droite $(AB)$.


%\dinofig{\figcaption{}\label{figaxerad1}}
\begin{figure}[ht]
\centering
\begin{tikzpicture}
\coordinate[label=below:$O_1$] (O1) at (0.8,0);
\coordinate(X1) at (2,0);
\node[draw,label=above:$\symcal{C_1}$] (c1) at (O1) [circle through={(X1)}]{};
\coordinate[label=below:$O_2$] (O2) at (3,0);
\coordinate(X2) at (4.8,0);
\node[draw,label=above:$\symcal{C_2}$] (c2) at (O2) [circle through={(X2)}]{};
\coordinate(Y1) at ($(O1)!-1.2!(X1)$);
\coordinate(Y2) at ($(O2)!1.1!(X2)$);
\draw (Y1)--(Y2);
\coordinate[label=right:$\kern2pt A$](A) at  (intersection 2 of c1 and c2);
\coordinate[label=right:$\kern2pt B$](B) at  (intersection 1 of c1 and c2);
\coordinate[label=below:$H\kern-13pt$](H) at  (intersection  of O1--O2 and A--B);
\coordinate[label=left:$\Delta$](V1) at ($(H)!1.7!(A)$);
\coordinate(V2) at ($(H)!1.3!(B)$);
\draw[semithick](V1)--(V2);
\foreach \point in {O1,O2,A,B,H}
\draw[black,fill=white](\point) circle (1.2pt);	
\end{tikzpicture}
\figcaption{}\label{figaxerad1}
\end{figure}

\subsubsection{Cercles tangents}
Supposons que les cercles $\symcal{C_1}$ et $\symcal{C_2}$ soient tangents (intérieurement ou extérieurement) en un point $T$ (voir la figure~\ref{figaxerad2}). On a 
$\symcal{C}_1(T)=\symcal{C}_2(T)=0$.
Le point $T$ a la même puissance par rapport à chacun des deux cercles : l'axe radical $\Delta$ de $\symcal{C}_1$ et $\symcal{C}_2$ est donc la droite perpendiculaire en $T$ à $(O_1O_2)$.

%\dinofig{\figcaption{}\label{figaxerad2}}
\begin{figure}[ht]
\centering
\begin{tikzpicture}
\coordinate[label=below:$O_1$] (O1) at (1.2,0);
\coordinate[label=below:$\kern10pt T$](T) at (3,0);
\node[draw,label=above:$\symcal{C_1}$] (c1) at (O1) [circle through={(T)}]{};
\coordinate[label=below:$O_2$] (O2) at (1.8,0);
%\coordinate(X2) at (4.8,0);
\node[draw,label=above:$\symcal{C_2}$] (c2) at (O2) [circle through={(T)}]{};
\coordinate(Y1) at ($(O1)!-1.3!(T)$);
\coordinate(Y2) at ($(O2)!1.3!(T)$);
\draw (Y1)--(Y2);
\coordinate[label=left:$\Delta$](V1) at ($(T)!1!270:(O1)$);
\coordinate(V2) at ($(T)!1!90:(O1)$);
\draw[semithick](V1)--(V2);
\foreach \point in {O1,O2,T}
\draw[black,fill=white](\point) circle (1.2pt);	
\end{tikzpicture}
\figcaption{}\label{figaxerad2}
\end{figure}


\subsubsection{Cas général}
Il est toujours possible de tracer un cercle auxiliaire $\Gamma$ sécant à $\symcal{C}_1$ en $A_1$ et $B_1$, à $\symcal{C}_2$ en $A_2$ et $B_2$, et  de telle sorte que les droites $(A_1B_1)$ et $(A_2B_2)$ soient elles-mêmes sécantes en un point $P$ (voir la figure~\ref{figaxerad3}). On est ainsi ramené à des constructions ans le cas de cercles sécants.

Le point $P$ appartient à l'axe radical de $(A_1B_1)$ de $\symcal{C}_1$ et $\Gamma$, donc $\symcal{C}_1(P)=\Gamma(P)$. De même, $\symcal{C}_2(P)=\Gamma(P)$ et par conséquent $\symcal{C}_1(P)=\symcal{C}_2(P)$. Le point $P$ appartient donc à l'axe radical $\Delta$ de $\symcal{C}_1$ et $\symcal{C}_2$ : $\Delta$  est  la perpendiculaire à $(O_1O_2)$ passant par $P$. 

\begin{figure}[ht]
\centering
\begin{tikzpicture}
\coordinate[label=below:$O_1$] (O1) at (0.4,0);
\coordinate(X1) at (1.5,0);
\node[draw,label=above:$\symcal{C_1}$] (c1) at (O1) [circle through={(X1)}]{};
\coordinate[label=below:$O_2$] (O2) at (3.6,0);
\coordinate(X2) at (5,0);
\node[draw,label=above:$\symcal{C_2}$] (c2) at (O2) [circle through={(X2)}]{};
\coordinate(Y1) at ($(O1)!-1.2!(X1)$);
\coordinate(Y2) at ($(O2)!1.1!(X2)$);
\draw (Y1)--(Y2);
\coordinate(Y) at (2,-1);
\coordinate(U) at (2,0.5);
\node[draw,label=below:$\kern10pt\Gamma$] (g) at (Y) [circle through={(U)}]{};
\coordinate[label=left:$A_1${\rule[-8pt]{0pt}{10pt}}](A1) at  (intersection 2 of c1 and g);
\coordinate[label=below:$B_1\kern10pt$](B1) at  (intersection 1 of c1 and g);
\coordinate[label=right: $A_2${\rule[-8pt]{0pt}{10pt}}](A2) at  (intersection 1 of c2 and g);
\coordinate[label=below:$\kern10pt B_2$](B2) at  (intersection 2 of c2 and g);
\coordinate[label=left:$P$](P) at (intersection of A1--B1 and A2--B2);
\draw (P)--(B1);
\draw (P)--(B2);
\coordinate(W1) at ($(O1)!0.5!(P)$);
\coordinate(W2) at ($(O2)!0.5!(P)$);
\node (w1) at (W1) [circle through={(P)}]{};
\node(w2) at (W2) [circle through={(P)}]{};
\coordinate[label=below:$H\kern13pt$](H) at  (intersection 1 of w1 and w2);
\coordinate[label=left:$\Delta$](V1) at ($(H)!1.5!(P)$);
\coordinate(V2) at ($(H)!-2.4!(P)$);
\draw[semithick](V1)--(V2);
\foreach \point in {O1,O2,P,H,A1,B1,A2,B2}
\draw[black,fill=white](\point) circle (1.2pt);	
\end{tikzpicture}
\figcaption{}\label{figaxerad3}
\end{figure}

\subsubsection{Dans un repère orthonormal}
Soit un cercle $\symcal{C}_1$ de centre $\Omega_1(a_1,b_1)$ et de rayon $r_2$, un cercle $\symcal{C}_2$ de centre $\Omega_2(a_2,b_2)$ et de rayon $r_2$. On suppose $\Omega_1\neq\Omega_2$ et on nomme $\Delta$ l'axe radical de ces deux cercles. D'après le théorème~\ref{threp} page \pageref{threp}, l'appartenance de $M(x,y)$ à $\Delta$ se traduit par
\begin{equation*}
x^2+y^2-2a_1x-2b_1y+c_1=x^2+y^2-2a_2x-2b_2y+c_2,
\end{equation*}
où $c_1=a_1^2+b_1^2-r_1^2$ et $c_2=a_2^2+b_2^2-r_2^2$.

\vspace\baselineskip\par Donc, après simplification :
\[P\in \Delta \iff (a_2-a_1)x+(b_2-b_1)y+(c_1-c_2)/2=0.\]
L'axe radical $\Delta$ admet pour équation 
\[(a_2-a_1)x+(b_2-b_1)y+(c_1-c_2)/2=0,\]
 où l'on constate bien que $\Delta$  admet $\Vector{\Omega_1\Omega_2}$ pour vecteur normal.
 
 \subsection{Centre radical}
 Lorsqu'au~\ref{subsecdetprat} nous avons, pour traiter le cas général, exploité l'idée d'un cercle auxiliaire, nous avons implicitement démontré le théorème suivant (voir la figure~\ref{figcentrerad}).
 \begin{thm}
 Soient trois cercles $\symcal{C}_1$, $\symcal{C}_2$ et $\Gamma$ dont les centres sont distincts et non alignés. Les axes radicaux $\Delta$ de $\symcal{C}_1$ et $\symcal{C}_2$, $\symcal{D}_1$ de $\symcal{C}_1$ et $\Gamma$ et finalement $\symcal{D}_2$ de $\symcal{C}_2$ et $\Gamma$ sont concourants en un point $P$ nommé \emph{centre radical} des trois cercles.
 \end{thm}
 
\newcommand{\figurecentrerad}{%
 \begin{figure}[ht]
\centering
\begin{tikzpicture}
\coordinate (O1) at (0.4,0);
\coordinate(X1) at (1.5,0);
\node[draw,label=above:$\symcal{C_1}$] (c1) at (O1) [circle through={(X1)}]{};
\coordinate (O2) at (3.6,0);
\coordinate(X2) at (5,0);
\node[draw,label=above:$\symcal{C_2}$] (c2) at (O2) [circle through={(X2)}]{};
\coordinate(Y) at (2,-1);
\coordinate(U) at (2,0.5);
\node[draw,label=below:$\kern10pt\Gamma$] (g) at (Y) [circle through={(U)}]{};
\coordinate(A1) at  (intersection 2 of c1 and g);
\coordinate(B1) at  (intersection 1 of c1 and g);
\coordinate(A2) at  (intersection 1 of c2 and g);
\coordinate(B2) at  (intersection 2 of c2 and g);
\coordinate[label=right:$\symcal{D}_1$](Y1) at ($(A1)!1.8!(B1)$);
\coordinate[label=left:$\symcal{D}_2$](Y2) at ($(A2)!1.5!(B2)$);
\coordinate[label=left:$P$](P) at (intersection of A1--B1 and A2--B2);
\draw (P)--(Y1);
\draw (P)--(Y2);
\coordinate(W1) at ($(O1)!0.5!(P)$);
\coordinate(W2) at ($(O2)!0.5!(P)$);
\node (w1) at (W1) [circle through={(P)}]{};
\node(w2) at (W2) [circle through={(P)}]{};
\coordinate(H) at  (intersection 1 of w1 and w2);
\coordinate[label=left:$\Delta$](V1) at ($(H)!1.5!(P)$);
\coordinate(V2) at ($(H)!-2.4!(P)$);
\draw(V1)--(V2);
\foreach \point in {P,A1,B1,A2,B2}
\draw[black,fill=white](\point) circle (1.2pt);	
\end{tikzpicture}
\figcaption{}\label{figcentrerad}
\end{figure}
}

%\DinoiPhoneCommand{%
%\afterpage{\clearpage
%\figurecentrerad
%\par
%\vspace{0.7cm}
%{
%\centering\LARGE\textcolor{ColorOne}\decoone

%}
%}}

%\DinoiPadStdCommand{\figurecentrerad}
%\DinoiPadAirCommand{\figurecentrerad}
%\DinoiPadProCommand{\figurecentrerad}
%%\DinoAfourNBCommand{\figurecentrerad}
%\DinoAfourColorCommand{\figurecentrerad}
%\DinoiPadminiCommand{\figurecentrerad}
%%\DinoSaisieCommand{\figurecentrerad}
%\DinoiPhoneCommand{\figurecentrerad}
\figurecentrerad

\begin{proof}
On définit $P$ comme intersection de $\symcal{D}_1$ et $\symcal{D}_2$ (le fait que les centres ne soient pas alignés garantit l'existence de $P$) :
\begin{align*}
P\in \symcal{D}_1&\implies \symcal{C}_1(P)=\Gamma(P)\text{ et }\\
P\in \symcal{D}_2&\implies \symcal{C}_2(P)=\Gamma(P),
\end{align*}
dont on déduit que $\symcal{C}_1(P)=\symcal{C}_2(P)$. Autrement dit $P\in \Delta$.
\end{proof}
%\DinoAfourColorCommand{%
%\par
%\vspace{1.5cm}
%{
%\centering\LARGE\textcolor{ColorOne}\decoone
%
%}
%}

\endinput