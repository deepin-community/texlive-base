

\chapter[Ce qui advint aux deux voyageurs…]{Ce qui advint aux deux voyageurs avec deux filles,\\ deux singes, et les
sauvages nommés Oreillons}


\lettrine{C}{andide} et son valet furent au-delà des barrières, et personne ne
savait encore dans le camp la mort du jésuite allemand. Le vigilant
\bname{Cacambo} avait eu soin de remplir sa valise de pain, de chocolat, de
jambon, de fruits, et de quelques mesures de vin. Ils s’enfoncèrent
avec leurs chevaux andalous dans un pays inconnu où ils ne découvrirent
aucune route. Enfin une belle prairie entrecoupée de ruisseaux se
présenta devant eux. Nos deux voyageurs font repaître leurs montures.
\bname{Cacambo} propose à son maître de manger, et lui en donne l’exemple.
«Comment veux-tu, disait \bname{Candide}, que je mange du jambon, quand j’ai tué
le fils de M.~le baron, et que je me vois condamné à ne revoir la
belle \bname{Cunégonde} de ma vie? à quoi me servira de prolonger mes
misérables jours, puisque je dois les traîner loin d’elle dans les
remords et dans le désespoir? et que dira le Journal de Trévoux?»


En parlant ainsi, il ne laissa pas de manger. Le soleil se couchait.
Les deux égarés entendirent quelques petits cris qui paraissaient
poussés par des femmes. Ils ne savaient si ces cris étaient de douleur
ou de joie; mais ils se levèrent précipitamment avec cette inquiétude
et cette alarme que tout inspire dans un pays inconnu. Ces clameurs
partaient de deux filles toutes nues qui couraient légèrement au bord
de la prairie, tandis que deux singes les suivaient en leur mordant les
fesses. \bname{Candide} fut touché de pitié; il avait appris à tirer chez les
Bulgares, et il aurait abattu une noisette dans un buisson sans toucher
aux feuilles. Il prend son fusil espagnol à deux coups, tire, et tue
les deux singes. «Dieu soit loué, mon cher \bname{Cacambo}! j’ai délivré d’un
grand péril ces deux pauvres créatures: si j’ai commis un péché en
tuant un inquisiteur et un jésuite, je l’ai bien réparé en sauvant la
vie à deux filles. Ce sont peut-être deux demoiselles de condition, et
cette aventure nous peut procurer de très grands avantages dans le
pays.»


Il allait continuer, mais sa langue devint percluse quand il vit ces
deux filles embrasser tendrement les deux singes, fondre en larmes sur
leurs corps, et remplir l’air des cris les plus douloureux. «Je ne
m’attendais pas à tant de bonté d’âme», dit-il enfin à \bname{Cacambo}; lequel
lui répliqua: «Vous avez fait là un beau chef d’œuvre, mon maître; vous
avez tué les deux amants de ces demoiselles. — Leurs amants! serait-il
possible? vous vous moquez de moi, \bname{Cacambo}; le moyen de vous croire?
— Mon cher maître, repartit \bname{Cacambo}, vous êtes toujours étonné de tout;
pourquoi trouvez-vous si étrange que dans quelques pays il y ait des
singes qui obtiennent les bonnes grâces des dames? Ils sont des quarts
d’homme, comme je suis un quart d’Espagnol. — Hélas! reprit \bname{Candide}, je
me souviens d’avoir entendu dire à maître \bname{Pangloss} qu’autrefois pareils
accidents étaient arrivés, et que ces mélanges avaient produit des
égypans, des faunes, des satyres; que plusieurs grands personnages de
l’antiquité en avaient vu; mais je prenais cela pour des fables. — Vous
devez être convaincu à présent, dit \bname{Cacambo}, que c’est une vérité, et
vous voyez comment en usent les personnes qui n’ont pas reçu une
certaine éducation; tout ce que je crains, c’est que ces dames ne nous
fassent quelque méchante affaire.»

Ces réflexions solides engagèrent \bname{Candide} à quitter la prairie, et à
s’enfoncer dans un bois. Il y soupa avec \bname{Cacambo}; et tous deux, après
avoir maudit l’inquisiteur de Portugal, le gouverneur de \bname{Buenos} Aires,
et le baron, s’endormirent sur de la mousse. À leur réveil, ils
sentirent qu’ils ne pouvaient remuer; la raison en était que pendant la
nuit les Oreillons, habitants du pays, à qui les deux dames les avaient
dénoncés, les avaient garrottés avec des cordes d’écorces d’arbre.
Ils étaient entourés d’une cinquantaine d’Oreillons tout nus, armés de
flèches, de massues, et de haches de caillou: les uns faisaient bouillir
une grande chaudière; les autres préparaient des broches, et tous
criaient: «C’est un jésuite, c’est un jésuite! nous serons vengés, et
nous ferons bonne chère; mangeons du jésuite, mangeons du jésuite!»

«Je vous l’avais bien dit, mon cher maître, s’écria tristement \bname{Cacambo},
que ces deux filles nous joueraient d’un mauvais tour.» \bname{Candide}
apercevant la chaudière et les broches s’écria: «Nous allons
certainement être rôtis ou bouillis. Ah! que dirait maître \bname{Pangloss},
s’il voyait comme la pure nature est faite? Tout est bien; soit, mais
j’avoue qu’il est bien cruel, d’avoir perdu M\up{lle}~\bname{Cunégonde}, et
d’être mis à la broche par des Oreillons.» \bname{Cacambo} ne perdait jamais la
tête. «Ne désespérez de rien, dit-il au désolé \bname{Candide}; j’entends un peu
le jargon de ces peuples, je vais leur parler. — Ne manquez pas, dit
\bname{Candide}, de leur représenter quelle est l’inhumanité affreuse de faire
cuire des hommes, et combien cela est peu chrétien.»

«Messieurs, dit \bname{Cacambo}, vous comptez donc manger aujourd’hui un
jésuite : c’est très bien fait; rien n’est plus juste que de traiter
ainsi ses ennemis. En effet le droit naturel nous enseigne à tuer notre
prochain, et c’est ainsi qu’on en agit dans toute la terre. Si nous
n’usons pas du droit de le manger, c’est que nous avons d’ailleurs de
quoi faire bonne chère; mais vous n’avez pas les mêmes ressources que
nous: certainement il vaut mieux manger ses ennemis que 
d’abandonner
 aux corbeaux et aux corneilles le fruit de sa victoire. Mais,
messieurs, vous ne voudriez pas manger vos amis. Vous croyez aller
mettre un jésuite en broche, et c’est votre défenseur, c’est l’ennemi
de vos ennemis que vous allez rôtir. Pour moi, je suis né dans votre
pays; monsieur que vous voyez est mon maître, et bien loin d’être
jésuite, il vient de tuer un jésuite, il en porte les dépouilles; voilà
le sujet de votre méprise. Pour vérifier ce que je vous dis, prenez sa
robe, portez-la à la première barrière du royaume de Los Padres;
informez-vous si mon maître n’a pas tué un officier jésuite. Il vous
faudra peu de temps; vous pourrez toujours nous manger, si vous trouvez
que je vous ai menti. Mais, si je vous ai dit la vérité, vous
connaissez trop les principes du droit public, les mœurs, et les lois,
pour ne nous pas faire grâce.»

Les Oreillons trouvèrent ce discours très raisonnable; ils députèrent
deux notables pour aller en diligence s’informer de la vérité; les deux
députés s’acquittèrent de leur commission en gens d’esprit, et
revinrent bientôt apporter de bonnes nouvelles. Les Oreillons délièrent
leurs deux prisonniers, leur firent toutes sortes de civilités, leur
offrirent des filles, leur donnèrent des rafraîchissements, et les
reconduisirent jusqu’aux confins de leurs états, en criant avec
allégresse: «Il n’est point jésuite, il n’est point jésuite!»

\bname{Candide} ne se lassait point d’admirer le sujet de sa délivrance. «Quel
peuple! disait-il, quels hommes! quelles mœurs! si je n’avais pas eu
le bonheur de donner un grand coup d’épée au travers du corps du frère
de M\up{lle}~\bname{Cunégonde}, j’étais mangé sans rémission. Mais, après
tout, la pure nature est bonne, puisque ces gens-ci, au lieu de me
manger, m’ont fait mille honnêtetés, dès qu’ils ont su que je n’étais
pas jésuite.»



\chapter[Arrivée de \bname{Candide} et de son valet…]{Arrivée de \bname{Candide} et de son valet\\au pays d’Eldorado, et ce qu’ils y
virent}


\lettrine{Q}{uand} ils furent aux frontières des Oreillons: «Vous voyez, dit \bname{Cacambo}
à \bname{Candide}, que cet hémisphère-ci ne vaut pas mieux que l’autre;
croyez-moi, retournons en Europe par le plus court chemin. — Comment y
retourner, dit \bname{Candide}; et où aller? Si je vais dans mon pays, les
Bulgares et les Abares y égorgent tout; si je retourne en Portugal, j’y
suis brûlé; si nous restons dans ce pays-ci, nous risquons à tout
moment d’être mis en broche. Mais comment se résoudre à quitter la
partie du monde que M\up{lle}~\bname{Cunégonde} habite?

 — Tournons vers la Cayenne, dit \bname{Cacambo}, nous y trouverons des Français
qui vont par tout le monde; ils pourront nous aider. Dieu aura
peut-être pitié de nous.»

Il n’était pas facile d’aller à la Cayenne: ils savaient bien à peu
près de quel côté il fallait marcher; mais des montagnes, des fleuves,
des précipices, des brigands, des sauvages, étaient partout de
terribles obstacles. Leurs
chevaux
 moururent de fatigue; leurs
provisions furent consumées; ils se nourrirent un mois entier de fruits
sauvages, et se trouvèrent enfin auprès d’une petite rivière bordée de
cocotiers qui soutinrent leur vie et leurs espérances.

\bname{Cacambo}, qui donnait toujours d’aussi bons conseils que la vieille, dit
à \bname{Candide}: «Nous n’en pouvons plus, nous avons assez marché; j’aperçois
un canot vide sur le rivage, emplissons-le de cocos, jetons-nous dans
cette petite barque, laissons-nous aller au courant; une rivière mène
toujours à quelque endroit habité. Si nous ne trouvons pas des choses
agréables, nous trouverons du moins des choses nouvelles. — Allons, dit
\bname{Candide}, recommandons-nous à la \bname{Providence}.»


Ils voguèrent quelques lieues entre des bords, tantôt fleuris, tantôt
arides, tantôt unis, tantôt escarpés. La rivière s’élargissait
toujours; enfin elle se perdait sous une voûte de rochers épouvantables
qui s’élevaient jusqu’au ciel. Les deux voyageurs eurent la hardiesse
de  s’abandonner aux flots sous cette voûte. Le fleuve resserré en cet
endroit les porta avec une rapidité et un bruit horrible. Au bout de
vingt-quatre heures ils revirent le jour; mais leur canot se fracassa
contre les écueils; il fallut se traîner de rocher en rocher pendant
une lieue entière; enfin ils découvrirent un horizon immense, bordé de
montagnes inaccessibles. Le pays était cultivé pour le plaisir comme
pour le besoin; partout l’utile était agréable: les chemins étaient
couverts ou plutôt ornés de voitures d’une forme et d’une matière
brillante, portant des hommes et des femmes d’une beauté singulière,
traînés rapidement par de gros moutons rouges qui surpassaient en
vitesse les plus beaux chevaux d’\bname{Andalousie}, de \bname{Tétuan}, et de \bname{Méquinez}.




«Voilà pourtant, dit \bname{Candide}, un pays qui vaut mieux que la \bname{Westphalie}.»
Il mit pied à terre avec \bname{Cacambo} auprès du premier village qu’il
rencontra. Quelques enfants du village, couverts de brocarts d’or tout
déchirés, jouaient au palet à l’entrée du bourg; nos deux hommes de
l’autre monde s’amusèrent à les regarder: leurs palets étaient d’assez
larges pièces rondes, jaunes, rouges, vertes, qui jetaient un éclat
singulier. Il prit envie aux voyageurs d’en ramasser quelques uns;
c’était de l’or, c’était des émeraudes, des rubis, dont le moindre
aurait été le plus grand ornement du trône du Mogol. «Sans doute, dit
\bname{Cacambo}, ces enfants sont les fils du roi du pays qui jouent au petit
palet.» Le magister du village parut dans ce moment pour les faire
rentrer à l’école. «Voilà, dit \bname{Candide}, le précepteur de la famille
royale.»

Les petits gueux quittèrent aussitôt le jeu, en laissant à terre leurs
palets, et tout ce qui avait servi à leurs divertissements.\bname{Candide} les
ramasse, court au précepteur et les lui présente humblement, lui faisant
entendre par signes que leurs altesses royales avaient oublié leur or
et leurs pierreries. Le magister du village, en souriant, les jeta par
terre, regarda un moment la figure de \bname{Candide} avec beaucoup de
surprise, et continua son chemin.


Les voyageurs ne manquèrent pas de ramasser l’or, les rubis, et les
émeraudes. «Où sommes-nous? s’écria \bname{Candide}. Il faut que les enfants des
rois de ce pays soient bien élevés, puisqu’on leur apprend à mépriser
l’or et les pierreries.» \bname{Cacambo} était aussi surpris que \bname{Candide}. Ils
approchèrent enfin de la première maison du village; elle était bâtie
comme un palais d’Europe. Une foule de monde s’empressait à la porte,
et encore plus dans le logis; une musique très agréable se faisait
entendre, et une odeur délicieuse de cuisine se faisait sentir. \bname{Cacambo}
s’approcha de la porte, et entendit qu’on parlait péruvien; c’était sa
langue maternelle; car tout le monde sait que \bname{Cacambo} était né au
Tucuman, dans un village où l’on ne connaissait que cette langue. «Je
vous servirai d’interprète, dit-il à \bname{Candide}; entrons, c’est ici un
cabaret.»

Aussitôt deux garçons et deux filles de l’hôtellerie, vêtus de drap
d’or, et les cheveux renoués avec des rubans, les invitent à se mettre
à la table de l’hôte. On servit quatre potages garnis chacun de deux
perroquets, un contour bouilli qui pesait deux cents livres, deux
singes rôtis d’un goût excellent, trois cents colibris dans un plat, et
six cents oiseaux-mouches dans un autre; des ragoûts exquis, des
pâtisseries délicieuses; le tout dans des plats d’une espèce de cristal
de roche. Les garçons et les filles de l’hôtellerie versaient plusieurs
liqueurs faites de cannes de sucre.

Les convives étaient pour la plupart des marchands et des voituriers,
tous d’une politesse extrême, qui firent quelques questions à \bname{Cacambo}
avec la discrétion la plus circonspecte, et qui répondirent aux siennes
d’une manière à le satisfaire.

Quand le repas fut fini, \bname{Cacambo} crut, ainsi que \bname{Candide}, bien payer
son écot, en jetant sur la table de l’hôte deux de ces larges pièces
d’or qu’il avait ramassées; l’hôte et l’hôtesse éclatèrent de rire, et
se tinrent longtemps les côtés. Enfin ils se remirent. «Messieurs, dit
l’hôte, nous voyons bien que vous êtes des étrangers; nous ne sommes
pas accoutumés à en voir. Pardonnez-nous si nous nous sommes mis à rire
quand vous nous avez offert en paiement les cailloux de nos grands
chemins. Vous n’avez pas sans doute de la monnaie du pays, mais il
n’est pas nécessaire d’en avoir pour dîner ici. Toutes les hôtelleries
établies pour la commodité du commerce sont payées par le gouvernement.
Vous avez fait mauvaise chère ici, parce que c’est un pauvre village,
mais partout ailleurs vous serez reçus comme vous méritez de l’être.»
\bname{Cacambo} expliquait à \bname{Candide} tous les discours de l’hôte, et \bname{Candide}
les écoutait avec la même admiration et le même égarement que son ami
\bname{Cacambo} les rendait. «Quel est donc ce pays, disaient-ils l’un et
l’autre, inconnu à tout le reste de la terre, et où toute la nature est
d’une espèce si différente de la nôtre? C’est probablement le pays où
tout va bien; car il faut absolument qu’il y en ait un de cette espèce.
Et, quoi qu’en dît maître \bname{Pangloss}, je me suis souvent aperçu que tout
allait assez mal en \bname{Westphalie}.»


\chapter{Ce qu’ils virent dans le pays d’Eldorado}



\lettrine{C}{acambo} témoigna à son hôte toute sa curiosité; l’hôte lui dit: «Je suis
fort ignorant, et je m’en trouve bien; mais nous avons ici un vieillard
retiré de la cour qui est le plus savant homme du royaume, et le plus
communicatif.» \textls[5]{Aussitôt il mène \bname{Cacambo} chez le vieillard. \bname{Candide} ne
jouait plus que le second personnage, et accompagnait son valet. Ils
entrèrent dans une maison fort simple, car la porte n’était que
d’argent, et les lambris des appartements n’étaient que d’or, mais
travaillés avec tant de goût, que les plus riches lambris ne
l’effaçaient pas. L’antichambre n’était à la vérité incrustée que de
rubis et d’émeraudes; mais l’ordre dans lequel tout était arrangé
réparait bien cette extrême simplicité.

Le vieillard reçut les deux étrangers sur un sofa matelassé de plumes
de colibri, et leur fit présenter des liqueurs dans des vases de
diamant; après quoi il satisfit à leur curiosité en ces termes:


«Je suis âgé de cent soixante et douze ans, et j’ai appris de feu mon
père, écuyer du roi, les étonnantes révolutions du Pérou dont il avait
été témoin. Le royaume où nous sommes est l’ancienne patrie des Incas,
qui en sortirent très imprudemment pour aller subjuguer une partie du
monde, et qui furent enfin détruits par les Espagnols.

«Les princes de leur famille qui restèrent dans leur pays natal furent
plus sages; ils ordonnèrent, du consentement de la nation, qu’aucun
habitant ne sortirait jamais de notre petit royaume; et c’est ce qui
nous a conservé notre innocence et notre félicité. Les Espagnols ont eu
une connaissance confuse de ce pays, ils l’ont appelé El Dorado; et un
\bname{Anglais}, nommé le chevalier Raleigh, en a même approché il y a environ
cent années; mais, comme nous sommes entourés de rochers inabordables
et de précipices, nous avons toujours été jusqu’à présent à l’abri de
la rapacité des nations de l’Europe, qui ont une fureur inconcevable
pour les cailloux et pour la fange de notre terre, et qui, pour en
avoir, nous tueraient tous jusqu’au dernier.»

La conversation fut longue; elle roula sur la forme du gouvernement,
sur les mœurs, sur les femmes, sur les spectacles publics, sur les
arts. Enfin \bname{Candide}, qui avait
toujours
 du goût pour la métaphysique,
fit demander par \bname{Cacambo} si dans le pays il y avait une religion.

%\looseness=1
Le vieillard rougit un peu. «Comment donc! dit-il, en pouvez-vous
douter? Est-ce que vous nous prenez pour des ingrats?» \bname{Cacambo} demanda
humblement quelle était la religion d’Eldorado. Le vieillard rougit
encore: «Est-ce qu’il peut y avoir deux religions? dit-il. Nous avons,
je crois, la religion de tout le monde; nous adorons Dieu du soir
jusqu’au matin. — N’adorez vous qu’un seul Dieu? dit \bname{Cacambo}, qui servait
toujours d’interprète aux doutes de \bname{Candide}. — Apparemment, dit le
vieillard, qu’il n’y en a ni deux, ni trois, ni quatre. Je vous avoue
que les gens de votre monde font des questions bien singulières.»
\bname{Candide} ne se lassait pas de faire interroger ce bon vieillard; il
voulut savoir comment on priait Dieu dans Eldorado. «Nous ne le prions
point, dit le bon et respectable sage; nous n’avons rien à lui
demander, il nous a donné tout ce qu’il nous faut; nous le remercions
sans cesse.» \bname{Candide} eut la curiosité de voir des prêtres; il fit
demander où ils étaient. Le bon vieillard sourit. «Mes amis, dit-il,
nous sommes tous prêtres; le roi et tous les chefs de famille chantent
des
cantiques d’actions de grâces solennellement tous les matins, et
cinq ou six mille musiciens les accompagnent. 

— Quoi! vous n’avez point
de moines qui enseignent, qui disputent, qui gouvernent, qui cabalent,
et qui font brûler les gens qui ne sont pas de leur avis? — Il faudrait
que nous fussions fous, dit le vieillard; nous sommes tous ici du même
avis, et nous n’entendons pas ce que vous voulez dire avec vos moines.»
\bname{Candide} à tous ces discours demeurait en extase, et disait en lui-même:
«Ceci est bien différent de la \bname{Westphalie} et du château de monsieur le
baron: si notre ami \bname{Pangloss} avait vu Eldorado, il n’aurait plus dit
que le château de Thunder-ten-tronckh était ce qu’il y avait de mieux
sur la terre; il est certain qu’il faut voyager.»


Après cette longue conversation, le bon vieillard fit atteler un
carrosse à six moutons, et donna douze de ses domestiques aux deux
voyageurs pour les conduire à la cour. «Excusez-moi, leur dit-il, si mon
âge me prive de l’honneur de vous accompagner. Le roi vous recevra
d’une manière dont vous ne serez pas mécontents, et vous pardonnerez
sans doute aux usages du pays, s’il y en a quelques uns qui vous
déplaisent.»


\bname{Candide} et \bname{Cacambo} montent en carrosse; les six moutons volaient, et en
moins de quatre heures on arriva au palais du roi, situé à un bout de
la capitale. Le portail était de deux cent vingt pieds de haut, et de
cent de large; il est impossible d’exprimer quelle en était la matière.
On voit assez quelle supériorité prodigieuse elle devait avoir sur ces
cailloux et sur ce sable que nous nommons \emph{or} et {pierreries}.


Vingt belles filles de la garde reçurent \bname{Candide} et \bname{Cacambo} à la
descente du carrosse, les conduisirent aux bains, les vêtirent de robes
d’un tissu de duvet de colibri; après quoi les grands officiers et les
grandes officières de la couronne les
menèrent à l’appartement de Sa
Majesté au milieu de deux files, chacune de mille musiciens, selon
l’usage ordinaire. Quand ils approchèrent de la salle du trône, \bname{Cacambo}
demanda à un grand officier comment il fallait s’y prendre pour saluer
Sa Majesté: si on se jetait à genoux ou ventre à terre; si on mettait
les mains sur la tête ou sur le derrière; si on léchait la poussière de
la salle: en un mot, quelle était la cérémonie. «L’usage, dit le
grand-officier, est d’embrasser le roi et de le baiser des deux côtés.»
\bname{Candide} et \bname{Cacambo} sautèrent au cou de Sa Majesté, qui les reçut avec
toute la grâce imaginable, et qui les pria poliment à souper.


En attendant, on leur fit voir la ville, les édifices publics élevés
jusqu’aux nues, les marchés ornés de mille colonnes, les fontaines
d’eau pure, les fontaines d’eau rose, celles de liqueurs de canne de
sucre qui coulaient continuellement dans de grandes places pavées d’une
espèce de pierreries qui répandaient une odeur semblable à celle du
girofle et de la cannelle. \bname{Candide} demanda à voir la cour de justice,
le parlement; on lui dit qu’il n’y en avait point, et qu’on ne plaidait
jamais. Il s’informa s’il y avait des prisons, et on lui dit que non.
Ce qui le surprit davantage, et qui lui fit le plus de plaisir, ce fut
le palais des sciences, dans lequel il vit une galerie de deux mille
pas, toute pleine d’instruments de mathématiques et de physique.}


Après avoir parcouru toute l’après-dinée à peu près la millième partie
de la ville, on les ramena chez le roi. \bname{Candide} se mit à table entre Sa
\bname{Majesté}, son valet \bname{Cacambo}, et plusieurs dames. Jamais on ne fit
meilleure chère, et jamais on n’eut plus d’esprit à souper qu’en eut Sa
\bname{Majesté}. \bname{Cacambo} expliquait les bons mots du roi à \bname{Candide}, et quoique
traduits, ils paraissaient toujours des bons mots. De tout ce qui
étonnait \bname{Candide}, ce n’était pas ce qui l’étonna le moins.

Ils passèrent un mois dans cet hospice. \bname{Candide} ne cessait de dire à
\bname{Cacambo}: «Il est vrai, mon ami, encore une fois, que le château où je
suis né ne vaut pas le pays où nous sommes; mais enfin M\up{lle}~\bname{Cunégonde} n’y est pas, et vous avez sans doute quelque maîtresse en
Europe. Si nous restons ici, nous n’y serons que comme les autres; au
lieu que si nous retournons dans notre monde, seulement avec douze
moutons chargés de cailloux d’Eldorado, nous serons plus riches que
tous les rois ensemble, nous n’aurons plus d’inquisiteurs à craindre,
et nous pourrons aisément reprendre M\up{lle}~\bname{Cunégonde}.»

Ce discours plut à \bname{Cacambo}; on aime tant à courir, à se faire valoir
chez les siens, à faire parade de ce qu’on a vu dans ses voyages, que
les deux heureux résolurent de ne plus l’être, et de demander leur
congé à Sa Majesté.

«Vous faites une sottise, leur dit le roi: je sais bien que mon pays est
peu de chose; mais, quand on est passablement quelque part, il faut y
rester. Je n’ai pas assurément le droit de retenir des étrangers; c’est
une tyrannie qui n’est ni dans nos mœurs ni dans nos lois; tous les
hommes sont libres; partez quand vous voudrez, mais la sortie est bien
difficile. Il est impossible de remonter la rivière rapide sur laquelle
vous êtes arrivés par miracle, et qui court sous des voûtes de rochers.
Les montagnes qui entourent tout mon royaume ont dix mille pieds de
hauteur, et sont droites comme des murailles: elles occupent chacune en
largeur un espace de plus de dix lieues; on ne peut en descendre que
par des précipices. Cependant, puisque vous voulez absolument partir,
je vais donner ordre aux intendants des machines d’en faire une qui
puisse vous transporter commodément. Quand on vous aura conduits au
revers des montagnes, personne ne pourra vous accompagner; car mes
sujets ont fait vœu de ne jamais sortir de leur enceinte, et ils sont
trop sages pour rompre leur vœu. Demandez-moi d’ailleurs tout ce qu’il
vous plaira. — Nous ne demandons à Votre Majesté, dit \bname{Cacambo}, que
quelques moutons chargés de vivres, de cailloux, et de la boue du pays.»
Le roi rit: «Je ne conçois pas, dit-il, quel goût vos gens d’Europe ont
pour notre boue jaune: mais emportez-en tant que vous voudrez, et grand
bien vous fasse.»

Il donna l’ordre sur-le-champ à ses ingénieurs de faire une machine
pour guinder ces deux hommes extraordinaires hors du royaume. Trois
mille bons physiciens y travaillèrent; elle fut prête au bout de quinze
jours, et ne coûta pas plus de vingt millions de livres sterling,
monnaie du pays. On mit sur la machine \bname{Candide} et \bname{Cacambo}; il y avait
deux grands moutons rouges sellés et bridés pour leur servir de monture
quand ils auraient franchi les montagnes, vingt moutons de bât chargés
de vivres, trente qui portaient des présents de ce que le pays a de
plus curieux, et cinquante chargés d’or, de pierreries, et de diamants.
Le roi embrassa tendrement les deux vagabonds.

Ce fut un beau spectacle que leur départ, et la manière ingénieuse dont
ils furent hissés eux et leurs moutons au haut des montagnes. Les
physiciens prirent congé d’eux après les avoir mis en sûreté, et
\bname{Candide} n’eut plus d’autre désir et d’autre objet que d’aller présenter
ses moutons à M\up{lle}~\bname{Cunégonde}. «Nous avons, dit-il, de quoi payer
le gouverneur de \bname{Buenos} Aires, si M\up{lle}~\bname{Cunégonde} peut être mise
à prix. Marchons vers la Cayenne, embarquons-nous, et nous verrons
ensuite quel royaume nous pourrons acheter.»




\chapter[Ce qui leur arriva à Surinam…]{Ce qui leur arriva à Surinam, et comment \bname{Candide} fit connaissance avec
\bname{Martin}}


\lettrine{L}{a première} journée de nos deux voyageurs fut assez agréable. Ils
étaient encouragés par l’idée de se voir possesseurs de plus de trésors
que l’Asie, \bname{l’Europe}, et l’Afrique, n’en pouvaient rassembler. \bname{Candide}
transporté écrivit le nom de \bname{Cunégonde} sur les arbres. À la seconde
journée deux de leurs moutons s’enfoncèrent dans des marais, et y
furent abîmés avec leurs charges; deux autres moutons moururent de
fatigue quelques jours après; sept ou huit périrent ensuite de faim
dans un désert; d’autres tombèrent au bout de quelques jours dans des
précipices. Enfin, après cent jours de marche, il ne leur resta que
deux moutons. \bname{Candide} dit à \bname{Cacambo}: «Mon ami, vous voyez comme les
richesses de ce monde sont périssables; il n’y a rien de solide que la
vertu et le bonheur de revoir M\up{lle}~\bname{Cunégonde}. — Je l’avoue, dit
\bname{Cacambo}; mais il nous reste encore deux moutons avec plus de trésors
que n’en aura jamais le roi d’Espagne; et je vois bien de loin une
ville que je soupçonne être Surinam, appartenante aux Hollandais. Nous
sommes au bout de nos peines et au commencement de notre félicité.»

%\looseness=1
En approchant de la ville, ils rencontrèrent un nègre étendu par terre,
n’ayant plus que la moitié de son habit, c’est-à-dire d’un caleçon de
toile bleue; il manquait à ce pauvre homme la jambe gauche et la main
droite. «Eh! mon Dieu! lui dit \bname{Candide} en hollandais, que fais-tu là,
mon ami, dans l’état horrible où je te vois? — J’attends mon maître, M.~\bname{Vanderdendur}, le fameux négociant, répondit le nègre. — Est-ce M.~\bname{Vanderdendur}, dit \bname{Candide}, qui t’a traité ainsi? — Oui, monsieur, dit le
nègre, c’est l’usage. On nous donne un caleçon de toile pour tout
vêtement deux fois l’année. Quand nous travaillons aux sucreries, et
que la meule nous attrape le doigt, on nous coupe la main: quand nous
voulons nous enfuir, on nous coupe la jambe: je me suis trouvé dans les
deux cas. C’est à ce prix que vous mangez du sucre en Europe.
Cependant, lorsque ma mère me vendit dix écus patagons sur la côte de
Guinée, elle me disait: \frquote{\localleftbox{« }Mon cher enfant, bénis nos fétiches, adore-les
toujours, ils te feront vivre heureux; tu as l’honneur d’être esclave
de nos seigneurs les blancs, et tu fais par là la fortune de ton père
et de ta mère.}\localleftbox{} Hélas! je ne sais pas si j’ai fait leur fortune, mais
ils n’ont pas fait la mienne. Les chiens, les singes, et les
perroquets, sont mille fois moins malheureux que nous: les fétiches
hollandais qui m’ont converti me disent tous les dimanches que nous
sommes tous enfants d’Adam, blancs et noirs. Je ne suis pas
généalogiste; mais si ces prêcheurs disent vrai, nous sommes tous
cousins issus de germain. Or vous m’avouerez qu’on ne peut pas en user
avec ses parents d’une manière \linebreak plus horrible.


— Ô \bname{Pangloss}! s’écria \bname{Candide}, tu n’avais pas deviné cette abomination;
c’en est fait, il faudra qu’à la fin je renonce à ton optimisme.
— Qu’est-ce qu’optimisme? disait \bname{Cacambo}. — Hélas! dit \bname{Candide}, c’est la
rage de soutenir que tout est bien quand on est mal»; et il versait des
larmes en regardant son nègre; et en pleurant, il entra dans \bname{Surinam}.

La première chose dont ils s’informent, c’est s’il n’y a point au port
quelque vaisseau qu’on pût envoyer à \bname{Buenos} Aires. Celui à qui ils
s’adressèrent était justement un patron espagnol qui s’offrit à faire
avec eux un marché honnête. Il leur donna rendez-vous dans un cabaret.
\bname{Candide} et le fidèle \bname{Cacambo} allèrent l’y attendre avec leurs deux
moutons.


\bname{Candide}, qui avait le cœur sur les lèvres, conta à \bname{l’Espagnol} toutes
ses aventures, et lui avoua qu’il voulait enlever mademoiselle
\bname{Cunégonde}. «Je me garderai bien de vous passer à \bname{Buenos} Aires, dit le
patron: je serais pendu, et vous aussi; la belle \bname{Cunégonde} est la
maîtresse favorite de monseigneur.» Ce fut un coup de foudre pour
\bname{Candide}, il pleura longtemps; enfin il tira à part \bname{Cacambo}. «Voici, mon
cher ami, lui dit-il, ce qu’il faut que tu fasses. Nous avons chacun
dans nos poches pour cinq ou six millions de diamants, tu es plus
habile que moi; va prendre M\up{lle}~\bname{Cunégonde} à \bname{Buenos} Aires. Si le
gouverneur fait quelque difficulté, donne-lui un million: s’il ne se
rend pas, donne-lui-en deux; tu n’as point tué d’inquisiteur, on ne se
défiera point de toi. J’équiperai un autre vaisseau, j’irai t’attendre
à  \bname{Venise}: c’est un pays libre où l’on n’a rien à craindre ni des
Bulgares, ni des Abares, ni des Juifs, ni des inquisiteurs.» \bname{Cacambo}
applaudit, à cette sage résolution. Il était au désespoir de se séparer
d’un bon maître devenu son ami intime; mais le plaisir de lui être
utile l’emporta sur la douleur de le quitter. Ils s’embrassèrent en
versant des larmes: \bname{Candide} lui recommanda de ne point oublier la bonne
vieille. \bname{Cacambo} partit dès le jour même: c’était un très bon homme que
ce \bname{Cacambo}.

\bname{Candide} resta encore quelque temps à Surinam, et attendit qu’un autre
patron voulût le mener en Italie lui et les deux moutons qui lui
restaient. Il prit des domestiques, et acheta tout ce qui lui était
nécessaire pour un long voyage; enfin M.~\bname{Vanderdendur}, maître d’un gros
vaisseau, vint se présenter à lui. «Combien voulez-vous, demanda-t-il à
cet homme, pour me mener en droiture à  \bname{Venise}, moi, mes gens, mon
bagage, et les deux moutons que voilà?» Le patron s’accorda à dix mille
piastres: \bname{Candide} n’hésita pas.


«Oh! oh! dit à part soi le prudent Vanderdendur, cet étranger donne dix
mille piastres tout d’un coup! il faut qu’il soit bien riche.» Puis
revenant un moment après, il signifia qu’il ne pouvait partir à moins
de vingt mille. «Eh bien! vous les aurez, dit \bname{Candide}.

«Ouais, se dit tout bas le marchand, cet homme donne vingt mille
piastres aussi aisément que dix mille.» Il revint encore, et dit qu’il
ne pouvait le conduire à  \bname{Venise} à moins de trente mille piastres. «Vous
en aurez donc trente mille», répondit \bname{Candide}.

«Oh! oh! se dit encore le marchand hollandais, trente mille piastres ne
coûtent rien à cet homme-ci; sans doute les deux moutons portent des
trésors immenses; n’insistons pas davantage: faisons-nous d’abord payer
les trente mille piastres, et puis nous verrons.» \bname{Candide}, vendit deux
petits diamants, dont le moindre valait plus que tout l’argent que
demandait le patron. Il le paya d’avance. Les deux moutons furent
embarqués. \bname{Candide} suivait dans un petit bateau pour joindre le
vaisseau à la rade; le patron prend son temps, met à la voile, démarre;
le vent le favorise. \bname{Candide} éperdu et stupéfait le perd bientôt de
vue. «Hélas! cria-t-il, voilà un tour digne de l’Ancien Monde.» Il
retourne au rivage, abîmé dans la douleur; car enfin il avait perdu de
quoi faire la fortune de vingt monarques.

Il se transporte chez le juge hollandais; et, comme il était un peu
troublé, il frappe rudement à la porte; il entre, expose son aventure,
et crie un peu plus haut qu’il ne convenait. Le juge commença par lui
faire payer dix mille piastres pour le bruit qu’il avait fait: ensuite
il l’écouta patiemment, lui promit d’examiner son affaire sitôt que le
marchand serait revenu, et se fit payer dix mille autres piastres pour
les frais de l’audience.

Ce procédé acheva de désespérer \bname{Candide}; il avait à la vérité essuyé
des malheurs mille fois plus douloureux; mais le sang froid du juge, et
celui du patron dont il était volé, alluma sa bile, et le plongea dans
une noire 
mélancolie. La méchanceté des hommes se présentait à son
esprit dans toute sa laideur, il ne se nourrissait que d’idées tristes.
Enfin un vaisseau français étant sur le point de partir pour Bordeaux,
comme il n’avait plus de moutons chargés de diamants à embarquer, il
loua une chambre du vaisseau à juste prix, et fit signifier dans la
ville qu’il paierait le passage, la nourriture, et donnerait deux mille
piastres à un honnête homme qui voudrait faire le voyage avec lui, à
condition que cet homme serait le plus dégoûté de son état, et le plus
malheureux de la province.


Il se présenta une foule de prétendants qu’une flotte n’aurait pu
contenir. \bname{Candide}, voulant choisir entre les plus apparents, il
distingua une vingtaine de personnes qui lui paraissaient sociables, et
qui toutes prétendaient mériter la préférence. Il les assembla dans son
cabaret, et leur donna à souper, à condition que chacun ferait serment
de raconter fidèlement son histoire, promettant de choisir celui qui
lui paraîtrait le plus à plaindre et le plus mécontent de son état, à
plus juste titre, et de donner aux autres quelques gratifications.

La séance dura jusqu’à quatre heures du matin. \bname{Candide}, en écoutant
toutes leurs aventures, se ressouvenait de ce que lui avait dit la
vieille en allant à \bname{Buenos} Aires, et de la gageure qu’elle avait faite,
qu’il n’y avait personne sur le vaisseau à qui il ne fût arrivé de très
grands malheurs. Il songeait à \bname{Pangloss} à chaque aventure qu’on lui
contait. Ce \bname{Pangloss}, disait-il, serait bien embarrassé à démontrer son
système. Je voudrais qu’il fût ici. Certainement si tout va bien, c’est
dans Eldorado, et non pas dans le reste de la terre. Enfin il se
détermina en faveur d’un pauvre savant qui avait travaillé dix ans pour
les libraires à Amsterdam. Il jugea qu’il n’y avait point de métier au
monde dont on dût être plus dégoûté.

Ce savant, qui était d’ailleurs un bon homme, avait été volé par sa
femme, battu par son fils, et abandonné de sa fille, qui s’était fait
enlever par un Portugais. Il venait d’être privé d’un petit emploi
duquel il subsistait; et les prédicants de Surinam le persécutaient,
parce qu’ils le prenaient pour un socinien. Il faut avouer que les
autres étaient pour le moins aussi malheureux que lui; mais \bname{Candide}
espérait que le \bname{savant} le désennuierait dans le voyage. Tous ses autres
rivaux trouvèrent que \bname{Candide} leur faisait une grande injustice; mais il
les apaisa en leur donnant à chacun cent piastres.



\chapter{Ce qui arriva sur mer à \bname{Candide} et à \bname{Martin}}


\lettrine{L}{e vieux} savant, qui s’appelait \bname{Martin}, s’embarqua donc pour Bordeaux
avec \bname{Candide}. L’un et l’autre avaient beaucoup vu et beaucoup souffert;
et quand le vaisseau aurait dû faire voile de Surinam au Japon par le
cap de Bonne-Espérance, ils auraient eu de quoi s’entretenir du mal
moral et du mal physique pendant tout le voyage.


Cependant \bname{Candide} avait un grand avantage sur \bname{Martin}, c’est qu’il
espérait toujours revoir M\up{lle}~\bname{Cunégonde}, et que \bname{Martin} n’avait
rien à espérer; de plus il avait de l’or et des diamants; et, quoiqu’il
eût perdu cent gros moutons rouges chargés des plus grands trésors de
la terre, quoiqu’il eût toujours sur le cœur la friponnerie du patron
hollandais, cependant quand il songeait à ce qui lui restait dans ses
poches, et quand il parlait de \bname{Cunégonde}, surtout à la fin du repas, il
penchait alors pour le système de \bname{Pangloss}.


«Mais vous, monsieur \bname{Martin}, dit-il au savant, que pensez-vous de tout
cela? quelle est votre idée sur le mal moral et le mal physique?
— Monsieur, répondit \bname{Martin}, mes prêtres m’ont accusé d’être socinien;
mais la vérité du fait est que je suis manichéen. — Vous vous moquez
de moi, dit \bname{Candide}; il n’y a plus de manichéens dans le monde. — Il y a
moi, dit \bname{Martin}: je ne sais qu’y faire; mais je ne peux penser
autrement. — Il faut que vous ayez le diable au corps, dit \bname{Candide}. — Il se
mêle si fort des affaires de ce monde, dit \bname{Martin}, qu’il pourrait bien
être dans mon corps, comme partout ailleurs; mais je vous avoue qu’en
jetant la vue sur ce globe, ou plutôt sur ce globule, je pense que Dieu
l’a abandonné à quelque être malfaisant; j’en excepte toujours
Eldorado. Je n’ai guère vu de ville qui ne désirât la ruine de la ville
voisine, point de famille qui ne voulût exterminer quelque autre
famille. Partout les faibles ont en exécration les puissants devant
lesquels ils rampent, et les puissants les traitent comme des troupeaux
dont on vend la laine et la chair. Un million d’assassins enrégimentés,
courant d’un bout de \bname{l’Europe} à
 l’autre, exerce le meurtre et le
brigandage avec discipline pour gagner son pain, parce qu’il n’a pas de
métier plus honnête; et dans les villes qui paraissent jouir de la
paix, et où les arts fleurissent, les hommes sont dévorés de plus
d’envie, de soins, et d’inquiétudes, qu’une ville assiégée n’éprouve de
fléaux. Les chagrins secrets sont encore plus cruels que les misères
publiques. En un mot j’en ai tant vu et tant éprouvé, que \linebreak je suis
manichéen.



— Il y a pourtant du bon, répliquait \bname{Candide}. — Cela peut être, disait
\bname{Martin}; mais je ne le connais pas.»



Au milieu de cette dispute, on entendit un bruit de canon. Le bruit
redouble de moment en moment. Chacun prend sa lunette. On aperçoit deux
vaisseaux qui combattaient à la distance d’environ trois milles: le
vent les amena l’un et l’autre si près du vaisseau français, qu’on eut
le plaisir de voir le combat tout à son aise. Enfin l’un des deux
vaisseaux lâcha à l’autre une bordée si bas et si juste, qu’il le coula
à fond. \bname{Candide} et \bname{Martin} aperçurent distinctement une centaine
d’hommes sur le tillac du vaisseau qui s’enfonçait; ils levaient tous
les mains au ciel, et jetaient des clameurs effroyables; en un moment
tout fut englouti.

Eh bien! dit \bname{Martin}, voilà comme les hommes se traitent les uns les
autres. Il est vrai, dit \bname{Candide}, qu’il y a quelque chose de diabolique
dans cette affaire. En parlant ainsi, il aperçut je ne sais quoi d’un
rouge éclatant, qui nageait auprès de son vaisseau. On détacha la
chaloupe pour voir ce que ce pouvait être; c’était un de ses moutons.
\bname{Candide} eut plus de joie de retrouver ce mouton, qu’il n’avait été
affligé d’en perdre cent tous chargés de gros diamants d’Eldorado.

Le capitaine français aperçut bientôt que le capitaine du vaisseau
submergeant était espagnol, et que celui du vaisseau submergé était un
pirate hollandais; c’était celui-là même qui avait volé \bname{Candide}. Les
richesses immenses dont ce scélérat s’était emparé furent ensevelies
avec lui dans la mer, et il n’y eut qu’un mouton de sauvé. «Vous voyez,
dit \bname{Candide} à \bname{Martin}, que le crime est puni quelquefois; ce coquin de
patron hollandais a eu le sort qu’il méritait. — Oui, dit \bname{Martin}; mais
fallait-il que les passagers qui étaient sur son vaisseau périssent
aussi? Dieu a puni ce fripon, le diable a noyé \linebreak les autres.»

Cependant le vaisseau français et l’espagnol continuèrent leur route,
et \bname{Candide} continua ses conversations avec \bname{Martin}. Ils disputèrent
quinze jours de suite, et au bout de quinze jours ils étaient aussi
avancés que le premier. Mais enfin ils parlaient, ils se communiquaient
des idées, ils se consolaient. \bname{Candide} caressait son mouton. «Puisque je
t’ai retrouvé, dit-il, je pourrai bien retrouver \bname{Cunégonde}.»



\chapter[Candide et \bname{Martin} approchent des côtes…]{Candide et \bname{Martin} approchent des côtes\\ de France, et raisonnent.}


\lettrine{O}{n aperçut} enfin les côtes de France. «Avez-vous jamais été en France,
monsieur \bname{Martin}? dit \bname{Candide}. — Oui, dit \bname{Martin}, j’ai parcouru plusieurs
provinces. Il y en a où la moitié des habitants est folle, quelques
unes où l’on est trop rusé, d’autres où l’on est communément assez doux
et assez bête, d’autres où l’on fait le bel esprit; et, dans toutes, la
principale occupation est l’amour, la seconde, de médire; et la
troisième, de dire des sottises. — Mais, monsieur \bname{Martin}, avez-vous vu
Paris? — Oui, j’ai vu Paris; il tient de toutes ces espèces-là; c’est un
chaos, c’est une presse dans laquelle tout le monde cherche le plaisir,
et où presque personne ne le trouve, du moins à ce qu’il m’a paru. J’y
ai séjourné peu; j’y fus volé, en arrivant, de tout ce que j’avais, par
des filous, à la foire Saint-Germain; on me prit moi-même pour un
voleur, et je fus huit jours en prison; après quoi je me fis correcteur
d’imprimerie pour gagner de quoi retourner à pied en \bname{Hollande}. Je
connus la canaille écrivante, la canaille cabalante, et la canaille
convulsionnaire. On dit qu’il y a des gens fort polis dans cette
ville-là: je le veux croire.

— Pour moi, je n’ai nulle curiosité de voir la France, dit \bname{Candide}; vous
devinez aisément que quand on a passé un mois dans Eldorado, on ne se
soucie plus de rien voir sur la terre que M\up{lle}~\bname{Cunégonde}; je
vais l’attendre à  \bname{Venise}; nous 
traverserons
 la France pour aller en
Italie; ne m’accompagnerez-vous pas? — Très volontiers, dit \bname{Martin}: on
dit que  \bname{Venise} n’est bonne que pour les nobles Vénitiens, mais que
cependant on y reçoit très bien les étrangers quand ils ont beaucoup
d’argent; je n’en ai point; vous en avez, je vous suivrai partout. — À 
propos, dit \bname{Candide}, pensez-vous que la terre ait été originairement
une mer, comme on l’assure dans ce gros livre qui appartient au
capitaine du vaisseau? — Je n’en crois rien du tout, dit \bname{Martin}, non
plus que de toutes les rêveries qu’on nous débite depuis quelque temps.
— Mais à quelle fin ce monde a-t-il donc été formé? dit \bname{Candide}. — Pour
nous faire enrager, répondit \bname{Martin}. — N’êtes-vous pas bien étonné,
continua \bname{Candide}, de l’amour que ces deux filles du pays des Oreillons
avaient pour ces deux singes, et dont je vous ai conté l’aventure?
— Point du tout, dit \bname{Martin}, je ne vois pas ce que cette passion a
d’étrange; j’ai tant vu de choses extraordinaires, qu’il n’y a plus
rien d’extraordinaire pour moi. — Croyez-vous, dit \bname{Candide}, que les
hommes se soient toujours mutuellement massacrés comme ils font
aujourd’hui? qu’ils aient toujours été menteurs, fourbes, perfides,
ingrats, brigands, faibles, volages, lâches, envieux, gourmands,
ivrognes, avares, ambitieux, sanguinaires, calomniateurs, débauchés,
fanatiques, hypocrites, et sots? — Croyez-vous, dit \bname{Martin}, que les
éperviers aient toujours mangé des pigeons quand ils en ont trouvé?
— Oui, sans doute, dit \bname{Candide}. — Eh bien! dit \bname{Martin}, si les éperviers ont
toujours eu le même caractère, pourquoi voulez-vous que les hommes
aient changé le leur? — Oh! dit \bname{Candide}, il y a bien de la différence,
car le libre arbitre…» En raisonnant ainsi, ils arrivèrent à Bordeaux.



\chapter[Ce qui arriva en France…]{Ce qui arriva en France à Candide et à \bname{Martin}}



\lettrine{C}{andide} ne s’arrêta dans Bordeaux qu’autant de temps qu’il en fallait
pour vendre quelques cailloux du \bname{Dorado}, et pour s’accommoder d’une
bonne chaise à deux places; car il ne pouvait plus se passer de son
philosophe \bname{Martin};  il fut seulement très fâché de se séparer de son
mouton, qu’il laissa à l’académie des sciences de Bordeaux, laquelle
proposa pour le sujet du prix de cette année de trouver pourquoi la
laine de ce mouton était rouge; et le prix fut adjugé à un savant du
Nord, qui démontra par A, plus B, moins C divisé par Z, que le mouton
devait être rouge, et mourir de la clavelée.



Cependant tous les voyageurs que \bname{Candide} rencontra dans les cabarets de
la route lui disaient: «Nous allons à Paris.» Cet empressement général
lui donna enfin l’envie de voir cette capitale; ce n’était pas beaucoup
se détourner du chemin de  \bname{Venise}.

Il entra par le faubourg Saint-Marceau, et crut être dans le plus
vilain village de la \bname{Westphalie}.


À peine \bname{Candide} fut-il dans son auberge, qu’il fut attaqué d’une
maladie légère, causée par ses fatigues. Comme il avait au doigt un
diamant énorme, et qu’on avait aperçu dans son équipage une cassette
prodigieusement pesante, il eut aussitôt auprès de lui deux médecins
qu’il n’avait pas mandés, quelques amis intimes qui ne le quittèrent
pas, et deux dévotes qui faisaient chauffer ses bouillons. \bname{Martin}
disait: «Je me souviens d’avoir été malade aussi à Paris dans mon 
premier voyage; j’étais fort pauvre: aussi n’eus-je ni amis, ni
dévotes, ni médecins, et je guéris.»

Cependant, à force de médecines et de saignées, la maladie de \bname{Candide}
devint sérieuse. Un habitué du quartier vint avec douceur lui demander
un billet payable au porteur pour l’autre monde: \bname{Candide} n’en voulut
rien faire; les dévotes l’assurèrent que c’était une nouvelle mode:
\bname{Candide} répondit qu’il n’était point homme à la mode. \bname{Martin} voulut
jeter l’habitué par les fenêtres. Le clerc jura qu’on n’enterrerait
point \bname{Candide}. \bname{Martin} jura qu’il enterrerait le clerc, s’il continuait
à les importuner. La querelle s’échauffa: \bname{Martin} le prit par les
épaules, et le chassa rudement; ce qui causa un grand scandale, dont on
fit un procès-verbal.

\bname{Candide} guérit; et pendant sa convalescence il eut très bonne compagnie
à souper chez lui. On jouait gros jeu. \bname{Candide} était tout étonné que
jamais les as ne lui vinssent; et \bname{Martin} ne s’en étonnait pas.

Parmi ceux qui lui faisaient les honneurs de la ville, il y avait un
petit abbé périgourdin, l’un de ces gens empressés, toujours alertes,
toujours serviables, effrontés, caressants, accommodants, qui guettent
les étrangers à leur passage, leur content l’histoire scandaleuse de la
ville, et leur offrent des plaisirs à tout prix. Celui-ci mena d’abord
\bname{Candide} et \bname{Martin} à la comédie. On y jouait une tragédie nouvelle.
\bname{Candide} se trouva placé auprès de quelques beaux esprits. Cela ne
l’empêcha pas de pleurer à des scènes jouées parfaitement. Un des
raisonneurs qui étaient à ses côtés lui dit dans un entr’acte: «Vous
avez grand tort de pleurer, cette actrice est fort mauvaise; l’acteur
qui joue avec elle est plus mauvais acteur encore; la pièce est encore
plus mauvaise que les acteurs; l’auteur ne sait pas un mot d’arabe, et
cependant la scène est en \bname{Arabie}; et, de plus, c’est un homme qui ne
croit pas aux idées innées; je vous apporterai demain vingt 
brochures 
contre lui. — Monsieur, combien avez-vous de pièces de théâtre en
France?» dit \bname{Candide} à l’abbé; lequel répondit: «Cinq ou six mille. — C’est
beaucoup, dit \bname{Candide}: combien y en a-t-il de bonnes? —Quinze ou seize,
répliqua l’autre. — C’est beaucoup», dit \bname{Martin}.



\bname{Candide} fut très content d’une actrice qui faisait la reine Élisabeth,
dans une assez plate tragédie, que l’on joue quelquefois. «Cette
actrice, dit-il à \bname{Martin}, me plaît beaucoup; elle a un faux air de
M\up{lle}~\bname{Cunégonde}; je serais bien aise de la saluer.» L’abbé
périgourdin s’offrit à l’introduire chez elle. \bname{Candide}, élevé en
\bname{Allemagne}, demanda quelle était l’étiquette, et comment on traitait en
France les reines d’Angleterre. «Il faut distinguer, dit l’abbé: en
province, on les mène au cabaret; à Paris, on les respecte quand elles
sont belles, et on les jette à la voirie quand elles sont mortes. — Des
reines à la voirie! dit \bname{Candide}. — Oui vraiment, dit \bname{Martin}; monsieur
l’abbé a raison; j’étais à Paris quand M\up{lle}~Monime passa,
comme on dit, de cette vie à l’autre; on lui refusa, ce que ces gens-ci
appellent les \emph{honneurs de la sépulture}, c’est-à-dire de pourrir avec
tous les gueux du quartier dans un vilain cimetière; elle fut enterrée
toute seule de sa bande au coin de la rue de Bourgogne; ce qui dut lui
faire une peine extrême, car elle pensait très noblement. — Cela est bien
impoli, dit \bname{Candide}. — Que voulez-vous? dit \bname{Martin}; ces gens-ci sont
ainsi faits. Imaginez toutes les contradictions, toutes les
incompatibilités possibles, vous les verrez dans le gouvernement, dans
les tribunaux, dans les églises, dans les spectacles de cette drôle de
nation. — Est-il vrai qu’on rit toujours à Paris? dit \bname{Candide}. — Oui, dit
l’abbé, mais c’est en enrageant; car on s’y plaint de tout avec de
grands éclats de rire; même on y fait en riant les actions les plus
\linebreak détestables.



— Quel est, dit \bname{Candide}, ce gros cochon qui me disait tant de mal de la
pièce où j’ai tant pleuré, et des acteurs qui m’ont fait tant de
plaisir? — C’est un mal-vivant, répondit l’abbé, qui gagne sa vie à dire
du mal de toutes les pièces et de tous les livres; il hait quiconque
réussit, comme les eunuques haïssent les jouissants; c’est un de ces
serpents de la littérature qui se nourrissent de fange et de venin;
c’est un folliculaire. — Qu’appelez-vous folliculaire? dit \bname{Candide}.
— C’est, dit l’abbé, un faiseur de feuilles, un \emph{Fréron}.»

C’est ainsi que \bname{Candide}, \bname{Martin}, et le Périgourdin, raisonnaient sur
l’escalier, en voyant défiler le monde au sortir de la pièce. «Quoique
je sois très empressé de revoir M\up{lle}~\bname{Cunégonde}, dit \bname{Candide}, je
voudrais pourtant souper avec M\up{lle}~Clairon, car elle m’a paru
admirable.»

L’abbé n’était pas homme à approcher de M\up{lle}~\bname{Clairon}, qui ne
voyait que bonne compagnie. «Elle est engagée pour ce soir, dit-il; mais
j’aurai l’honneur de vous mener chez une dame de qualité, et là vous
connaîtrez Paris comme si vous y aviez été quatre ans.»

\bname{Candide}, qui était naturellement curieux, se laissa mener chez la dame,
au fond du faubourg Saint-Honoré; on y était occupé d’un pharaon; douze
tristes pontes tenaient chacun en main un petit livre de cartes,
registre cornu de leurs infortunes. Un profond silence régnait, la
pâleur était sur le front des pontes, l’inquiétude sur celui du
banquier; et la dame du logis, assise auprès de ce banquier
impitoyable, remarquait avec des yeux de lynx tous les parolis, tous
les sept  et le va de campagne, dont chaque joueur cornait ses cartes;
elle les faisait décorner avec une attention sévère, mais polie, et ne
se fâchait point, de peur de perdre ses pratiques. La dame se faisait
appeler la marquise de Parolignac. Sa fille, âgée de quinze ans, était
au nombre des pontes, et avertissait d’un clin d’œil des friponneries
de ces pauvres gens qui tâchaient de réparer les cruautés du sort.
L’abbé périgourdin, \bname{Candide}, et \bname{Martin}, entrèrent; personne ne se leva,
ni les salua, ni les regarda; tous étaient profondément occupés de
leurs cartes. «M\up{me} la baronne de Thunder-ten-tronckh était plus
civile», dit \bname{Candide}.

Cependant l’abbé s’approcha de l’oreille de la marquise, qui se leva à
moitié, honora \bname{Candide} d’un sourire gracieux, et \bname{Martin} d’un air de
tête tout-à-fait noble; elle fit donner un siège et un jeu de cartes à
\bname{Candide}, qui perdit cinquante mille francs en deux tailles: après quoi
on soupa très gaiement; et tout le monde était étonné que \bname{Candide} ne
fût pas ému de sa perte; les laquais disaient entre eux, dans leur
langage de laquais: «Il faut que ce soit quelque milord anglais.» Le
souper fut comme la plupart des soupers de Paris, d’abord du silence,
ensuite un bruit de paroles qu’on ne distingue point, puis des
plaisanteries dont la plupart sont insipides, de fausses nouvelles, de
mauvais raisonnements, un peu de politique, et beaucoup de médisance;
on parla même de livres nouveaux. «Avez-vous vu, dit l’abbé périgourdin,
le roman du sieur Gauchat, docteur en théologie? — Oui, répondit un des
convives, mais je n’ai pu l’achever. Nous avons une foule d’écrits
impertinents; mais tous ensemble n’approchent pas de l’impertinence de
\bname{Gauchat}, docteur en théologie; je suis si rassasié de cette
immensité de détestables livres qui nous inondent, que je me suis mis à
ponter au pharaon. — Et les \emph{\bname{Mélanges}} de l’archidiacre Trublet, qu’en
dites-vous? dit l’abbé. — Ah! dit M\up{me} de Parolignac, l’ennuyeux
mortel! comme il vous dit curieusement ce que tout le monde sait! comme
il discute pesamment ce qui ne vaut pas la peine d’être remarqué
légèrement! comme il s’approprie, sans esprit, l’esprit des autres!
comme il gâte ce qu’il pille! comme il me dégoûte! mais il ne me
dégoûtera plus; c’est assez d’avoir lu quelques pages de l’archidiacre.»


Il y avait à table un homme savant et de goût qui appuya ce que disait
la marquise. On parla ensuite de 
tragédies;
 la dame demanda pourquoi il
y avait des tragédies qu’on jouait quelquefois, et qu’on ne pouvait
lire. L’homme de goût expliqua très bien comment une pièce pouvait
avoir quelque intérêt, et n’avoir presque aucun mérite; il prouva en
peu de mots que ce n’était pas assez d’amener une ou deux de ces
situations qu’on trouve dans tous les romans, et qui séduisent toujours
les spectateurs; mais qu’il faut être neuf sans être bizarre, souvent
sublime et toujours naturel, connaître le cœur humain et le faire
parler; être grand poète, sans que jamais aucun personnage de la pièce
paraisse poète; savoir parfaitement sa langue, la parler avec pureté,
avec une harmonie continue, sans que jamais la rime coûte rien au sens.
«\bname{Quiconque}, ajouta-t-il, 
n’observe
 pas toutes ces règles, peut faire une
ou deux tragédies applaudies au théâtre, mais il ne sera jamais compté
au rang des bons écrivains; il y a très peu de bonnes tragédies: les
unes sont des idylles en dialogues bien écrits et bien rimés; les
autres, des raisonnements politiques qui endorment, ou des
amplifications qui rebutent; les autres, des rêves d’énergumène, en
style barbare, des propos interrompus, de longues apostrophes aux
dieux, parce qu’on ne sait point parler aux hommes, des maximes fausses,
des lieux communs ampoulés.»


\bname{Candide} écouta ce propos avec attention, et conçut une grande idée du
discoureur; et, comme la marquise avait eu soin de le placer à côté
d’elle, il s’approcha de son oreille, et prit la liberté de lui
demander qui était cet homme qui parlait si bien. «C’est un savant, dit
la dame, qui ne ponte point, et que l’abbé m’amène quelquefois à
souper; il se connaît parfaitement en tragédies et en livres, et il a
fait une tragédie sifflée, et un livre dont on n’a jamais vu hors de la
boutique de son libraire qu’un exemplaire qu’il m’a dédié. — Le grand
homme! dit \bname{Candide}, c’est un autre \bname{Pangloss}.»

Alors se tournant vers lui, il lui dit: «Monsieur, vous pensez, sans
doute, que tout est au mieux dans le monde physique et dans le moral,
et que rien ne pouvait être autrement? — Moi, monsieur, lui répondit le
savant, je ne pense rien de tout cela; je trouve que tout va de travers
chez nous; que personne ne sait ni quel est son rang, ni quelle est sa
charge, ni ce qu’il fait, ni ce qu’il doit faire, et qu’excepté le
souper, qui est assez gai, et où il paraît assez d’union, tout le reste
du temps se passe en querelles impertinentes; jansénistes contre
molinistes, gens du parlement contre gens d’église, gens de lettres
contre gens de lettres, courtisans contre courtisans, financiers contre
le peuple, femmes contre maris, parents contre parents; c’est une
guerre éternelle.»


\bname{Candide} lui répliqua: «J’ai vu pis; mais un sage, qui depuis a eu le
malheur d’être pendu, m’apprit que tout cela est à merveille; ce sont
des ombres à un beau tableau. — Votre pendu se moquait du monde, dit
\bname{Martin}; vos ombres sont des taches horribles. — Ce sont les hommes qui
font les taches, dit \bname{Candide}, et ils ne peuvent pas s’en dispenser. — Ce
n’est donc pas leur faute», dit \bname{Martin}. La plupart des pontes, qui
n’entendaient rien à ce langage, buvaient; et \bname{Martin} raisonna avec le
savant, et \bname{Candide} raconta une partie de ses aventures à la dame du
logis.

Après souper, la marquise mena \bname{Candide} dans son cabinet, et le fit
asseoir sur un canapé. «Eh bien! lui dit-elle, vous aimez donc toujours
éperdument M\up{lle}~\bname{Cunégonde} de Thunder-ten-tronckh? — Oui, madame,
répondit \bname{Candide}.» La marquise lui répliqua avec un souris tendre: «Vous
me répondez comme un jeune homme de \bname{Westphalie}; un Français m’aurait
dit: Il est vrai que j’ai aimé M\up{lle}~\bname{Cunégonde}; mais en vous
voyant, madame, je crains de ne la plus aimer. — Hélas! madame, dit
\bname{Candide}, je répondrai comme vous voudrez. — Votre passion pour elle, dit
la marquise, a commencé en ramassant son mouchoir; je veux que vous
ramassiez ma jarretière. — De tout mon cœur, dit \bname{Candide}; et il la
ramassa. — Mais je veux que vous me la remettiez», dit la dame; et \bname{Candide}
la lui remit. «Voyez-vous, dit la dame, vous êtes étranger; je fais
quelquefois languir mes amants de Paris quinze jours, mais je me rends
à vous dès la première nuit, parce qu’il faut faire les honneurs de son
pays à un jeune homme de \bname{Westphalie}.» La belle ayant aperçu deux énormes
diamants aux deux mains de son jeune étranger, les loua de si bonne
foi, que des doigts de \bname{Candide} ils passèrent aux doigts de la marquise.

\looseness=1
\bname{Candide}, en s’en retournant avec son abbé périgourdin, sentit quelques
remords d’avoir fait une infidélité à M\up{lle}~\bname{Cunégonde}. M. l’abbé
entra dans sa peine; il n’avait qu’une légère part aux cinquante mille
livres perdues au jeu par \bname{Candide}, et à la valeur des deux brillants
moitié donnés, moitié extorqués. Son dessein était de profiter, autant
qu’il le pourrait, des avantages que la connaissance de \bname{Candide} pouvait
lui procurer. Il lui parla beaucoup de \bname{Cunégonde}; et \bname{Candide} lui dit
qu’il demanderait bien pardon à cette belle de son infidélité, quand il
la verrait à \bname{ \bname{Venise}}.

Le Périgourdin redoublait de politesses et d’attentions, et prenait un
intérêt tendre à tout ce que \bname{Candide} disait, à tout ce qu’il faisait, à
tout ce qu’il voulait faire.

«Vous avez donc, monsieur, lui dit-il, un rendez-vous à \bname{Venise}? — Oui,
monsieur l’abbé, dit \bname{Candide}; il faut absolument que j’aille trouver
M\up{lle}~\bname{Cunégonde}.» Alors, engagé par le plaisir de parler de ce
qu’il aimait, il conta, selon son usage, une partie de ses aventures
avec cette illustre  \bname{Westphalienne}.


«Je crois, dit l’abbé, que M\up{lle}~\bname{Cunégonde} a bien de l’esprit, et
qu’elle écrit des lettres charmantes. — Je n’en ai jamais reçu, dit
\bname{Candide}; car, figurez-vous qu’ayant été chassé du château pour l’amour
d’elle, je ne pus lui écrire; que bientôt après j’appris qu’elle était
morte, qu’ensuite je la retrouvai, et que je la perdis, et que je lui
ai envoyé à deux mille cinq cents lieues d’ici un exprès dont j’attends
la réponse.»

L’abbé écoutait attentivement, et paraissait un peu rêveur. Il prit
bientôt congé des deux étrangers, après les avoir tendrement embrassés.
Le lendemain \bname{Candide} reçut à son réveil une lettre conçue en ces
termes:
\par\vspace{\baselineskip}\par
{\itshape Monsieur mon très cher amant, il y a huit jours que je suis malade en
cette ville; j’apprends que vous y êtes. Je volerais dans vos bras si
je pouvais remuer. J’ai su votre passage à Bordeaux; j’y ai laissé le
fidèle \bname{Cacambo} et la vieille, qui doivent bientôt me suivre. Le
gouverneur de \bname{Buenos} Aires a tout pris, mais il me reste votre cœur.
Venez; votre présence me rendra la vie ou me fera mourir de plaisir.

}
\par\vspace{\baselineskip}\par
Cette lettre charmante, cette lettre inespérée, transporta \bname{Candide}
d’une joie inexprimable; et la maladie de sa chère \bname{Cunégonde} l’accabla
de douleur. Partagé entre ces deux sentiments, il prend son or et ses
diamants, et se fait conduire avec \bname{Martin} à l’hôtel où 
mademoiselle
\bname{Cunégonde} demeurait. Il entre en tremblant d’émotion, son cœur
palpite, sa voix sanglote; il veut ouvrir les rideaux du lit; il veut
faire apporter de la lumière. «Gardez-vous-en bien, lui dit la suivante;
la lumière la tue»; et soudain elle referme le rideau. «Ma chère
\bname{Cunégonde}, dit \bname{Candide} en pleurant, comment vous portez-vous? si vous
ne pouvez me voir, parlez-moi du moins. — Elle ne peut parler», dit la
suivante, la dame alors tire du lit une main potelée que \bname{Candide} arrose
longtemps de ses larmes, et qu’il remplit ensuite de diamants, en
laissant un sac plein d’or sur le fauteuil.


Au milieu de ses transports arrive un exempt suivi de l’abbé
périgourdin et d’une escouade. «Voilà donc, dit-il, ces deux étrangers
suspects?» Il les fait incontinent saisir, et ordonne à ses braves de
les traîner en prison. «Ce n’est pas ainsi qu’on traite les voyageurs
dans Eldorado, dit \bname{Candide}. — Je suis plus manichéen que jamais, dit
\bname{Martin}. — Mais, monsieur, où nous menez-vous? dit \bname{Candide}. — Dans un cul de
basse-fosse», dit l’exempt.

\bname{Martin}, ayant repris son sang froid, jugea que la dame qui se
prétendait \bname{Cunégonde} était une friponne, M. l’abbé périgourdin un
fripon, qui avait abusé au plus vite de 
l'innocence
 de \bname{Candide}, et
l’exempt un autre fripon dont on pouvait aisément se débarrasser.


Plutôt que de s’exposer aux procédures de la justice, \bname{Candide}, éclairé
par son conseil, et d’ailleurs toujours impatient de revoir la
véritable \bname{Cunégonde}, propose à l’exempt trois petits diamants d’environ
trois mille pistoles chacun. — Ah! monsieur, lui dit l’homme au bâton
d’ivoire, eussiez-vous commis tous les crimes imaginables, vous êtes le
plus honnête homme du monde; trois diamants! chacun de trois mille
pistoles! Monsieur! je me ferais tuer pour vous, au lieu de vous mener
dans un cachot. On arrête tous les étrangers, mais laissez-moi faire;
j’ai un frère à Dieppe en Normandie; je vais vous y mener; et si vous
avez quelque diamant à lui donner, il aura soin de vous comme moi-même.

— Et pourquoi arrête-t-on tous les étrangers? dit \bname{Candide}. L’abbé
périgourdin prit alors la parole, et dit: «C’est parce qu’un gueux du
pays d’Atrébatie a entendu dire des sottises; cela seul lui a fait
commettre un parricide, non pas tel que celui de 1610 au mois de
mai, mais tel que celui de 1594 au mois de décembre, et tel que
plusieurs autres commis dans d’autres années et dans d’autres mois par
d’autres gueux qui avaient entendu dire des sottises.»



L’exempt alors expliqua de quoi il s’agissait. «Ah! les monstres!
s’écria \bname{Candide}; quoi! de telles horreurs chez un peuple qui danse et
qui chante! Ne pourrai-je sortir au plus vite de ce pays où des singes
agacent des tigres? J’ai vu des ours dans mon pays; je n’ai vu des
hommes que dans le Dorado. Au nom de Dieu, monsieur l’exempt, menez-moi
à  \bname{Venise}, où je dois attendre M\up{lle}~\bname{Cunégonde}. — Je ne peux vous
mener qu’en Basse-Normandie, dit le barigel.» Aussitôt il lui fait
ôter ses fers, dit qu’il s’est mépris, renvoie ses gens, emmène à
Dieppe \bname{Candide} et \bname{Martin}, et les laisse entre les mains de son frère.
Il y avait un petit vaisseau hollandais à la rade. Le Normand, à l’aide
de trois autres diamants, devenu le plus serviable des hommes, embarque
\bname{Candide} et ses gens dans le vaisseau qui allait faire voile pour
Portsmouth en Angleterre. Ce n’était pas le chemin de  \bname{Venise}; mais
\bname{Candide} croyait être délivré de l’enfer; et il comptait bien reprendre
la route de  \bname{Venise} à la première occasion.





\chapter[Candide et \bname{Martin} vont sur les côtes…]{Candide et \bname{Martin} vont sur les côtes d’Angleterre;\\ ce qu’ils y voient}


\lettrine[ante=«]{A}{h! \bname{Pangloss}!} \bname{Pangloss}! Ah! \bname{Martin}! \bname{Martin}! Ah! ma chère \bname{Cunégonde}!
qu’est-ce que ce monde-ci?» disait \bname{Candide} sur le vaisseau hollandais.
— Quelque chose de bien fou et de bien abominable, répondait \bname{Martin}. — Vous
connaissez l’Angleterre; y est-on aussi fou qu’en France? — C’est une
autre espèce de folie, dit \bname{Martin}. Vous savez que ces deux nations sont
en guerre pour quelques arpents de neige vers le Canada, et qu’elles
dépensent pour cette belle guerre beaucoup plus que tout le Canada ne
vaut. 
De vous dire précisément s’il y a plus de gens à lier dans un
pays que dans un autre, c’est ce que mes faibles lumières ne me
permettent pas; je sais seulement qu’en général les gens que nous
allons voir sont fort atrabilaires.»


En causant ainsi ils abordèrent à Portsmouth; une multitude de peuple
couvrait le rivage, et regardait attentivement un assez gros homme qui
était à genoux, les yeux bandés, sur le tillac d’un des vaisseaux de la
flotte; quatre soldats, postés vis-à-vis de cet homme, lui tirèrent
chacun trois balles dans le crâne, le plus paisiblement du monde; et
toute l’assemblée s’en retourna extrêmement satisfaite. «Qu’est-ce
donc que tout ceci? dit \bname{Candide}; et quel démon exerce partout son
empire?» Il demanda qui était ce gros homme qu’on venait de tuer en
cérémonie.
«C’est un amiral, lui répondit-on. — Et pourquoi tuer cet
amiral? — C’est, lui dit-on, parce qu’il n’a pas fait tuer assez de monde;
il a livré un combat à un amiral français, et on a trouvé qu’il n’était
pas assez près de lui. — Mais, dit \bname{Candide}, l’amiral français était aussi
loin de l’amiral anglais que celui-ci l’était de l’autre! — Cela est
incontestable, lui répliqua-t-on; mais dans ce pays-ci il est bon de
tuer de temps en temps un amiral pour encourager les autres.»


\bname{Candide} fut si étourdi et si choqué de ce qu’il voyait et de ce qu’il
entendait, qu’il ne voulut pas seulement mettre pied à terre, et qu’il
fit son marché avec le patron hollandais (dût-il le voler comme celui
de Surinam), pour le conduire sans délai à  \bname{Venise}.

Le patron fut prêt au bout de deux jours. On côtoya la France; on passa
à la vue de Lisbonne, et \bname{Candide} frémit. On entra dans le détroit et
dans la \bname{Méditerranée}, enfin on aborda à  \bname{Venise}. «Dieu soit loué! dit
\bname{Candide}, en embrassant \bname{Martin}; c’est ici que je reverrai la belle
\bname{Cunégonde}. Je compte sur \bname{Cacambo} comme sur moi-même. Tout est bien,
tout va bien, tout va le mieux qu’il soit possible.»


\chapter{De \bname{Paquette}, et de frère Giroflée}


\lettrine{D}{ès} qu’il fut à  \bname{Venise}, il fit chercher \bname{Cacambo} dans tous les cabarets,
dans tous les cafés, chez toutes les filles de joie, et ne le trouva
point. Il envoyait tous les jours à la découverte de tous les vaisseaux
et de toutes les barques: nulles nouvelles de \bname{Cacambo}. «Quoi! disait-il
à \bname{Martin}, j’ai eu le temps de passer de Surinam à Bordeaux, d’aller de
Bordeaux à Paris, de Paris à Dieppe, de Dieppe à Portsmouth, de côtoyer
le Portugal et l’Espagne, de traverser toute la Méditerranée, de passer
quelques mois à  \bname{Venise}; et la belle \bname{Cunégonde} n’est point venue! Je
n’ai rencontré au lieu d’elle qu’une drôlesse et un abbé périgourdin!
\bname{Cunégonde} est morte, sans doute; je n’ai plus qu’à mourir. Ah! il
valait mieux rester dans le paradis du Dorado que de revenir dans cette
maudite Europe. Que vous avez raison, mon cher \bname{Martin}! tout n’est
qu’illusion et calamité.»

Il tomba dans une mélancolie noire, et ne prit aucune part à l’opéra
\emph{alla moda}, ni aux autres divertissements du carnaval; pas une dame ne
lui donna la moindre 
tentation. \bname{Martin} lui dit: «Vous êtes bien simple,
en vérité, de vous figurer qu’un valet métis, qui a cinq ou six
millions dans ses poches, ira chercher votre maîtresse au bout du
monde, et vous l’amènera à  \bname{Venise}. Il la prendra pour lui, s’il la
trouve; s’il ne la trouve pas, il en prendra une autre: je vous
conseille d’oublier votre valet \bname{Cacambo} et votre maîtresse \bname{Cunégonde}.»
\bname{Martin} n’était pas consolant. La mélancolie de \bname{Candide} augmenta, et
\bname{Martin} ne cessait de lui prouver qu’il y avait peu de vertu et peu de
bonheur sur la terre; excepté peut-être dans Eldorado, où personne
ne pouvait aller.



En disputant sur cette matière importante, et en attendant \bname{Cunégonde},
\bname{Candide} aperçut un jeune théatin dans la place Saint-Marc, qui tenait
sous le bras une fille. Le théatin paraissait frais, potelé, vigoureux;
ses yeux étaient brillants, son air assuré, sa mine haute, sa démarche
fière. La fille était très jolie, et chantait; elle regardait
amoureusement son théatin, et de temps en temps lui pinçait ses grosses
joues. «Vous m’avouerez du moins, dit \bname{Candide} à \bname{Martin}, que ces gens-ci
sont heureux. Je n’ai trouvé jusqu’à présent dans toute la terre
habitable, excepté dans Eldorado, que des infortunés; mais pour cette
fille et ce théatin, je gage que ce sont des créatures très heureuses.
«Je gage que non, dit \bname{Martin}. — Il n’y a qu’à les prier à dîner, dit
\bname{Candide}, et vous verrez si je me trompe.»

Aussitôt il les aborde, il leur fait son compliment, et les invite à
venir à son hôtellerie manger des macaronis, des perdrix de Lombardie,
des œufs d’esturgeon, et à boire du vin de Montepulciano, du
lacryma-christi, du chypre, et du samos. La demoiselle rougit, le
théatin accepta la partie, et la fille le suivit en regardant \bname{Candide}
avec des yeux de surprise et de confusion, qui furent obscurcis de
quelques larmes. À peine fut-elle entrée dans la chambre de \bname{Candide},
qu’elle lui dit: «Eh quoi! monsieur \bname{Candide} ne reconnaît plus \bname{Paquette}!»
À ces mots \bname{Candide}, qui ne l’avait pas considérée jusque-là avec
attention, parce qu’il n’était occupé que de \bname{Cunégonde}, lui dit: «Hélas!
ma pauvre enfant, c’est donc vous qui avez mis le docteur \bname{Pangloss} dans
le bel état où je l’ai vu?


— Hélas! monsieur, c’est moi-même, dit \bname{Paquette}; je vois que vous êtes
instruit de tout. J’ai su les malheurs 
épouvantables
 arrivés à toute la
maison de M\up{me} la baronne et à la belle \bname{Cunégonde}. Je vous jure que
ma destinée n’a guère été moins triste. J’étais fort innocente quand
vous m’avez vue. Un cordelier, qui était mon confesseur, me séduisit
aisément. Les suites en furent affreuses; je fus obligée de sortir du
château quelque temps après que M.~le baron vous eut renvoyé à grands
coups de pied dans le derrière. Si un fameux médecin n’avait pas pris
pitié de moi, j’étais morte. Je fus quelque temps par reconnaissance la
maîtresse de ce médecin. Sa femme, qui était jalouse à la rage, me
battait tous les jours impitoyablement; c’était une furie. Ce médecin
était le plus laid de tous les hommes, et moi la plus malheureuse de
toutes les créatures d’être battue continuellement pour un homme que je
n’aimais pas. Vous savez, monsieur, combien il est dangereux pour une
femme acariâtre d’être l’épouse d’un médecin. Celui-ci, outré des
procédés de sa femme, lui donna un jour, pour la guérir d’un petit
rhume, une médecine si efficace, qu’elle en mourut en deux heures de
temps dans des convulsions horribles. Les parents de madame intentèrent
à monsieur un procès criminel; il prit la fuite, et moi je fus mise en
prison. Mon innocence ne m’aurait pas sauvée, si je n’avais été un peu
jolie. Le juge m’élargit à condition qu’il succéderait au médecin. Je
fus bientôt supplantée par une rivale, chassée sans récompense, et
obligée de continuer ce métier abominable qui vous paraît si plaisant à
vous autres hommes, et qui n’est pour nous qu’un abîme de misère.
J’allai exercer la profession à  \bname{Venise}. Ah! monsieur, si vous pouviez
vous imaginer ce que c’est que d’être obligée de caresser
indifféremment un vieux marchand, un avocat, un moine, un gondolier, un
abbé; d’être exposée à toutes les insultes, à toutes les avanies;
d’être souvent réduite à emprunter une jupe pour aller se la faire
lever par un homme dégoûtant; d’être volée par l’un de ce qu’on a gagné
avec l’autre; d’être rançonnée par les officiers de justice, et de
n’avoir en perspective qu’une vieillesse affreuse, un hôpital, et un
fumier, vous concluriez que je suis une des plus malheureuses créatures
du monde.»

\bname{Paquette} ouvrait ainsi son cœur au bon \bname{Candide}, dans un cabinet, en
présence de \bname{Martin}, qui disait à \bname{Candide}: «Vous voyez que j’ai déjà
gagné la moitié de la gageure.»


Frère Giroflée était resté dans la salle à manger, et buvait un coup en
attendant le dîner. «Mais, dit \bname{Candide} à \bname{Paquette}, vous aviez l’air si
gai, si content, quand je vous ai rencontrée; vous chantiez, vous
caressiez le théatin avec une complaisance naturelle; vous m’avez paru
aussi heureuse que vous prétendez être infortunée. — Ah! monsieur,
répondit \bname{Paquette}, c’est encore là une des misères du métier. J’ai été
hier volée et battue par un officier, et il faut aujourd’hui que je
paraisse de bonne humeur pour plaire à un moine.»

\bname{Candide} n’en voulut pas davantage; il avoua que \bname{Martin} avait raison. On
se mit à table avec \bname{Paquette} et le théatin; le repas fut assez amusant,
et sur la fin on se parla avec quelque confiance. «Mon père, dit \bname{Candide}
au moine, vous me paraissez jouir d’une destinée que tout le monde doit
envier; la fleur de la santé brille sur votre visage, votre physionomie
annonce le bonheur; vous avez une très jolie fille pour votre
récréation, et vous paraissez très content de votre état de théatin.


— Ma foi, monsieur, dit frère Giroflée, je voudrais que tous les théatins
fussent au fond de la mer. J’ai été tenté cent fois de mettre le feu au
couvent, et d’aller me faire turc. Mes parents me forcèrent, à l’âge de
quinze ans, d’endosser cette détestable robe, pour laisser plus de
fortune à un maudit frère aîné, que Dieu confonde! La jalousie, la
discorde, la rage, habitent dans le couvent. Il est vrai que j’ai
prêché quelques mauvais sermons qui m’ont valu un peu d’argent dont le
prieur me vole la moitié; le reste me sert à entretenir des filles:
mais quand je rentre le soir dans le monastère, je suis prêt à me
casser la tête contre les murs du dortoir; et tous mes confrères sont
dans le même cas.»

\bname{Martin} se tournant vers \bname{Candide} avec son sang froid ordinaire: «Eh bien!
lui dit-il, n’ai-je pas gagné la gageure tout entière?» \bname{Candide} donna
deux mille piastres à \bname{Paquette}, et mille piastres à frère Giroflée. «Je
vous réponds, dit-il, qu’avec cela ils seront heureux. — Je n’en crois
rien du tout, dit \bname{Martin}; vous les rendrez peut-être avec ces piastres
beaucoup plus malheureux encore. — Il en sera ce qui pourra, dit \bname{Candide}:
mais une chose me console, je vois qu’on retrouve souvent les gens
qu’on ne croyait jamais retrouver; il se pourra bien faire qu’ayant
rencontré mon mouton rouge et \bname{Paquette}, je rencontre aussi \bname{Cunégonde}.
— Je souhaite, dit \bname{Martin}, qu’elle fasse un jour votre bonheur; mais
c’est de quoi je doute fort. — Vous êtes bien dur, dit \bname{Candide}. — C’est que
j’ai vécu, dit \bname{Martin}. 

— Mais regardez ces gondoliers, dit \bname{Candide}: ne
chantent-ils pas sans cesse? — Vous ne les voyez pas dans leur ménage,
avec leurs femmes et leurs marmots d’enfants, dit \bname{Martin}. Le doge a ses
chagrins, les gondoliers ont les leurs. Il est vrai qu’à tout prendre
le sort d’un gondolier est préférable à celui d’un doge; mais je crois
la différence si médiocre, que cela ne vaut pas la peine d’être
examiné.

— On parle, dit \bname{Candide}, du sénateur \bname{Pococurante}, qui demeure dans ce
beau palais sur la Brenta, et qui reçoit assez bien les étrangers. On
prétend que c’est un homme qui n’a jamais eu de chagrin. — Je voudrais
voir une espèce si rare, dit \bname{Martin}.» \bname{Candide} aussitôt fit demander au
seigneur \bname{Pococurante} la permission de venir le voir le lendemain.



\chapter[Visite chez le seigneur \bname{Pococurante}…]{Visite chez le seigneur \bname{Pococurante}, noble vénitien}


\lettrine{C}{andide} et \bname{Martin} allèrent en gondole sur la Brenta, et arrivèrent au
palais du noble \bname{Pococurante}. Les jardins étaient bien entendus, et
ornés de belles statues de marbre; le palais d’une belle architecture.
Le maître du logis, homme de soixante ans, fort riche, reçut très
poliment les deux curieux, mais avec très peu d’empressement, ce qui
déconcerta \bname{Candide}, et ne déplut point à \bname{Martin}.


 D’abord deux filles jolies et proprement mises servirent du chocolat,
qu’elles firent très bien mousser. \bname{Candide} ne put s’empêcher de les
louer sur leur beauté, sur leur bonne grâce, et sur leur adresse. «Ce
sont d’assez bonnes créatures, dit le sénateur \bname{Pococurante}; je les fais
quelquefois coucher dans mon lit; car je suis bien las des dames de la
ville, de leurs coquetteries, de leurs jalousies, de leurs querelles,
de leurs humeurs, de leurs petitesses, de leur orgueil, de leurs
sottises, et des sonnets qu’il faut faire ou commander pour elles;
mais, après tout, ces deux filles commencent fort à m’ennuyer.»


\bname{Candide}, après le déjeuner, se promenant dans une longue galerie, fut
surpris de la beauté des tableaux. Il demanda de quel maître étaient
les deux premiers. «Ils sont de \bname{Raphaël}, dit le sénateur; je les achetai
fort cher par vanité, il y a quelques années; on dit que c’est ce qu’il
y a de plus beau en Italie, mais ils ne me plaisent point du tout: la
couleur en est très rembrunie, les figures ne sont pas assez arrondies,
et ne sortent point assez; les draperies ne ressemblent en rien à une
étoffe: en un mot, quoi qu’on en dise, je ne trouve point là une
imitation vraie de la nature. Je n’aimerai un tableau que quand je
croirai voir la nature elle-même: il n’y en a point de cette espèce.
J’ai beaucoup de tableaux, mais je ne les regarde plus.»

\bname{Pococurante}, en attendant le dîner, se fit donner un concerto. \bname{Candide}
trouva la musique délicieuse. «Ce bruit, dit \bname{Pococurante}, peut amuser
une demi-heure; mais s’il dure plus long-temps, il fatigue tout le
monde, quoique personne n’ose l’avouer. La musique aujourd’hui n’est
plus que l’art d’exécuter des choses difficiles, et ce qui n’est que
difficile ne plaît point à la longue.

«J’aimerais peut être mieux l’opéra, si on n’avait pas trouvé le secret
d’en faire un monstre qui me révolte. Ira voir qui voudra de mauvaises
tragédies en musique, où les scènes ne sont faites que pour amener très
mal à propos deux ou trois chansons ridicules qui font valoir le gosier
d’une actrice; se pâmera de plaisir qui voudra ou qui pourra, en voyant
un châtré fredonner le rôle de \bname{César} et de \bname{Caton}, et se promener d’un
air gauche sur des planches: pour moi, il y a long-temps que j’ai
renoncé à ces pauvretés qui font aujourd’hui la gloire de l’Italie, et
que des souverains paient si chèrement.» \bname{Candide} disputa un peu, mais
avec discrétion. \bname{Martin} fut entièrement de l’avis du sénateur.


On se mit à table; et, après un excellent dîner, on entra dans la
bibliothèque. \bname{Candide}, en voyant un Homère magnifiquement relié, loua
l’illustrissime sur son bon goût. «Voilà, dit-il, un livre qui faisait
les délices du grand \bname{Pangloss}, le meilleur philosophe de l’Allemagne.
— Il ne fait pas les miennes, dit froidement \bname{Pococurante}: on me fit
accroire autrefois que j’avais du plaisir en le lisant; mais cette
répétition continuelle de combats qui se ressemblent tous, ces dieux
qui agissent toujours pour ne rien faire de décisif, cette Hélène qui
est le sujet de la guerre, et qui à peine est une actrice de la pièce;
cette Troie qu’on assiège et qu’on ne prend point; tout cela me causait
le plus mortel ennui. J’ai demandé quelquefois à des savants s’ils
s’ennuyaient autant que moi à cette lecture: tous les gens sincères
m’ont avoué que le livre leur tombait des mains, mais qu’il fallait
toujours l’avoir dans sa bibliothèque, comme un monument de
l’antiquité, et comme ces médailles rouillées qui ne peuvent être de
commerce.

— Votre excellence ne pense pas ainsi de Virgile? dit \bname{Candide}. — Je
conviens, dit \bname{Pococurante}, que le second, le quatrième, et le sixième
livre de son Énéide, sont excellents; mais pour son pieux Énée, et le
fort Cloanthe, et l’ami Achates, et le petit Ascanius, et l’imbécile
roi \bname{Latinus}, et la bourgeoise Amata, et l’insipide Lavinia, je ne crois
pas qu’il y ait rien de si froid et de plus désagréable. J’aime mieux
le Tasse et les contes à dormir debout de l’Arioste.


— Oserais-je vous demander, monsieur, dit \bname{Candide}, si vous n’avez pas un
grand plaisir à lire Horace? — Il y a des maximes, dit \bname{Pococurante}, dont
un homme du monde peut faire son profit, et qui, étant resserrées dans
des vers énergiques, se gravent plus aisément dans la mémoire: mais je
me soucie fort peu de son voyage à Brindes, et de sa description d’un
mauvais dîner, et de la querelle de crocheteurs entre je ne sais quel
Pupilus dont les paroles, dit-il, \emph{étaient pleines de pus}, et un
autre dont les paroles \emph{étaient du vinaigre}. Je n’ai lu qu’avec un
extrême dégoût ses vers grossiers contre des vieilles et contre des
sorcières; et je ne vois pas quel mérite il peut y avoir à dire à son
ami Mæcenas que, s’il est mis par lui au rang des poètes lyriques, il
frappera les astres de son front sublime. Les sots admirent tout
dans un auteur estimé. Je ne lis que pour moi; je n’aime que ce qui est
à mon usage.» \bname{Candide}, qui avait été élevé à ne jamais juger de rien par
lui-même, était fort étonné de ce qu’il entendait; et \bname{Martin} trouvait
la façon de penser de \bname{Pococurante} assez raisonnable.


\looseness=1
«Oh! voici un Cicéron, dit \bname{Candide}: pour ce grand homme-là, je pense que
vous ne vous lassez point de le lire. 
%
— Je ne le lis jamais, répondit le
Vénitien. Que m’importe qu’il ait plaidé pour Rabirius ou pour
Cluentius? J’ai bien assez des procès que je juge; je me serais mieux
accommodé de ses œuvres philosophiques; mais quand j’ai vu qu’il
doutait de tout, j’ai conclu que j’en savais autant que lui, et que je
n’avais besoin de personne pour être ignorant.



— Ah! voilà quatre-vingts volumes de recueils d’une académie des
sciences, s’écria \bname{Martin}; il se peut qu’il y ait là du bon. — Il y en
aurait, dit \bname{Pococurante}, si un seul des auteurs de ces fatras avait
inventé seulement l’art de faire des épingles; mais il n’y a dans tous
ces livres que de vains systèmes, et pas une seule chose utile.

— Que de pièces de théâtre je vois là, dit \bname{Candide}, en italien, en
espagnol, en français! — Oui, dit le sénateur, il y en a trois mille, et
pas trois douzaines de bonnes. Pour ces recueils de sermons, qui tous
ensemble ne valent pas une page de Sénèque, et tous ces gros volumes de
théologie, vous pensez bien que je ne les ouvre jamais, ni moi ni
personne.»



\bname{Martin} aperçut des rayons chargés de livres anglais. «Je crois, dit-il,
qu’un républicain doit se plaire à la plupart de ces ouvrages écrits si
librement. — Oui, répondit \bname{Pococurante}, il est beau d’écrire ce qu’on
pense; c’est le privilège de l’homme. Dans toute notre Italie, on
n’écrit que ce qu’on ne pense pas; ceux qui habitent la patrie des
Césars et des \bname{Antonins} n’osent avoir une idée sans la permission d’un
jacobin. Je serais content de la liberté qui inspire les génies
anglais, si la passion et l’esprit de parti ne corrompaient pas tout ce
que cette précieuse liberté a d’estimable.»

\looseness=1
\bname{Candide} apercevant un Milton, lui demanda s’il ne regardait pas cet
auteur comme un grand homme. «Qui? dit \bname{Pococurante}, ce barbare, qui fait
un long commentaire du premier chapitre de la Genèse, en dix livres de
vers durs? ce grossier imitateur des Grecs, qui défigure la création,
et qui, tandis que Moïse représente l’Être éternel produisant le monde
par la parole, fait prendre un grand compas par le Messiah dans une
armoire du ciel pour tracer son ouvrage? Moi, j’estimerais celui qui a
gâté l’enfer et le diable du Tasse; qui déguise Lucifer tantôt en
crapaud, tantôt en pygmée; qui lui fait rebattre cent fois les mêmes
discours; qui le fait disputer sur la théologie; qui, en imitant
sérieusement l’invention comique des armes à feu de l’Arioste, fait
tirer le canon dans le ciel par les diables? Ni moi, ni personne en
Italie n’a pu se plaire à toutes ces tristes extravagances. Le mariage
du péché et de la mort et les couleuvres dont le péché accouche, font
vomir tout homme qui a le goût un peu délicat; et sa longue description
d’un hôpital n’est bonne que pour un fossoyeur. Ce poème obscur,
bizarre, et dégoûtant, fut méprisé à sa naissance; je le traite
aujourd’hui comme il fut traité dans sa patrie par les
contemporains. Au reste je dis ce que je pense, et je me soucie fort
peu que les autres pensent comme moi.» \bname{Candide} était affligé de ces
discours; il respectait Homère, il aimait un peu Milton. «Hélas! dit-il
tout bas à \bname{Martin}, j’ai bien peur que cet homme-ci n’ait un souverain
mépris pour nos poètes allemands. — Il n’y aurait pas grand mal à cela,
dit \bname{Martin}. — Oh! quel homme supérieur! disait encore \bname{Candide} entre ses
dents, quel grand génie que ce \bname{Pococurante}! rien ne peut lui plaire.



Après avoir fait ainsi la revue de tous les livres, ils descendirent
dans le jardin. \bname{Candide} en loua toutes les beautés. «Je ne sais rien de
si mauvais goût, dit le maître; nous n’avons ici que des colifichets:
mais je vais dès demain en faire planter un d’un dessin plus noble.»


Quand les deux curieux eurent pris congé de son excellence: «Or çà, dit
\bname{Candide} à \bname{Martin}, vous conviendrez que voilà le plus heureux de tous
les hommes, car il est au-dessus de tout ce qu’il possède. — Ne
voyez-vous pas, dit \bname{Martin}, qu’il est dégoûté de tout ce qu’il possède?
Platon a dit, il y a long-temps, que les meilleurs estomacs ne sont pas
ceux qui rebutent tous les aliments. — Mais, dit \bname{Candide}, n’y a-t-il pas
du plaisir à tout critiquer, à sentir des défauts où les autres hommes
croient voir des 
beautés? — C’est-à-dire, reprit \bname{Martin}, qu’il y a du
plaisir à n’avoir pas de plaisir? — Oh bien! dit \bname{Candide}, il n’y a donc
d’heureux que moi, quand je reverrai M\up{lle}~\bname{Cunégonde}. — C’est
toujours bien fait d’espérer, dit \bname{Martin}.

Cependant les jours, les semaines, s’écoulaient; \bname{Cacambo} ne revenait
point, et \bname{Candide} était si abîmé dans sa douleur, qu’il ne fit pas même
réflexion que \bname{Paquette} et frère Giroflée n’étaient pas venus seulement
le remercier.


\chapter[D’un souper que Candide et Martin firent…]{D’un souper que \bname{Candide} et \bname{Martin} firent\\avec six étrangers, et qui ils
étaient}


\lettrine{U}{n soir} que \bname{Candide}, suivi de \bname{Martin}, allait se mettre à table avec les
étrangers qui logeaient dans la même hôtellerie, un homme à visage
couleur de suie l’aborda par derrière, et, le prenant par le bras, lui
dit: «Soyez prêt à partir avec nous, n’y manquez pas.» Il se retourne, et
voit \bname{Cacambo}. Il n’y avait que la vue de \bname{Cunégonde} qui pût l’étonner et
lui plaire davantage. Il fut sur le point de devenir fou de joie. Il
embrasse son cher ami. «\bname{Cunégonde} est ici, sans doute? où est-elle?
Mène-moi vers elle, que je meure de joie avec elle. — \bname{Cunégonde} n’est
point ici, dit \bname{Cacambo}, elle est à Constantinople. — Ah ciel! à
Constantinople! mais fût-elle à la Chine, j’y vole, partons. — Nous
partirons après souper, reprit \bname{Cacambo}; je ne peux vous en dire
davantage; je suis esclave, mon maître m’attend; il faut que j’aille le
servir à table: ne dites mot, soupez, et tenez-vous prêt.»

\bname{Candide}, partagé entre la joie et la douleur, charmé d’avoir revu son
agent fidèle, étonné de le voir esclave, plein de l’idée de retrouver
sa maîtresse, le cœur agité, l’esprit bouleversé, se mit à table avec
\bname{Martin}, qui voyait de sang froid toutes ces aventures, et avec six
étrangers, qui étaient venus passer le carnaval à  \bname{Venise}.


\bname{Cacambo}, qui versait à boire à l’un de ces six étrangers, s’approcha de
l’oreille de son maître, sur la fin du repas, et lui dit: «Sire, Votre
Majesté partira quand elle voudra, le vaisseau est prêt.» Ayant dit ces
mots, il sortit. Les convives étonnés se regardaient sans proférer une
seule parole, lorsqu’un autre domestique s’approchant de son maître,
lui dit: «Sire, la chaise de Votre Majesté est à Padoue, et la barque
est prête.» Le maître fit un signe, et le domestique partit. Tous les
convives se regardèrent encore, et la surprise commune redoubla. Un
troisième valet s’approchant aussi d’un troisième étranger, lui dit:
«Sire, croyez-moi, Votre Majesté ne doit pas rester ici plus longtemps,
je vais tout préparer»; et aussitôt il disparut.

\bname{Candide} et \bname{Martin} ne doutèrent pas alors que ce ne fût une mascarade du
carnaval. Un quatrième domestique dit au quatrième maître: «Votre
Majesté partira quand elle voudra», et sortit comme les autres. Le
cinquième valet en dit autant au cinquième maître. Mais le sixième
valet parla différemment au sixième étranger qui était auprès de
\bname{Candide}; il lui dit: «Ma foi, sire, on ne veut plus faire crédit à Votre
Majesté ni à moi non plus, et nous pourrions bien être coffrés cette
nuit vous et moi; je vais pourvoir à mes affaires: adieu.»

Tous les domestiques ayant disparu, les six étrangers, \bname{Candide}, et
\bname{Martin}, demeurèrent dans un profond silence. Enfin \bname{Candide} le rompit:
«Messieurs, dit-il, voilà une singulière plaisanterie. Pourquoi
êtes-vous tous rois? pour moi, je vous avoue que ni moi ni \bname{Martin} nous
ne le sommes.»

Le maître de \bname{Cacambo} prit alors gravement la parole, et dit en italien:
«Je ne suis point plaisant, je m’appelle Achmet III; j’ai été
grand-sultan plusieurs années; je détrônai mon frère; mon neveu m’a
détrôné; on a coupé le cou à mes vizirs; j’achève ma vie dans le vieux
sérail; mon neveu le grand-sultan Mahmoud me permet de voyager
quelquefois pour ma santé; et je suis venu passer le carnaval à  \bname{Venise}.»

Un jeune homme qui était auprès d’Achmet parla après lui, et dit: «Je
m’appelle Ivan; j’ai été empereur de toutes les Russies; j’ai été
détrôné au berceau; mon père et ma mère ont été enfermés; on m’a élevé
en prison; j’ai quelquefois la permission de voyager, accompagné de
ceux qui me gardent; et je suis venu passer le carnaval à  \bname{Venise}.»



Le troisième dit: «Je suis Charles-Édouard, roi \bname{d’Angleterre};
 mon
père m’a cédé ses droits au royaume; j’ai combattu pour les soutenir;
on a arraché le cœur à huit cents de mes partisans, et on leur en a
battu les joues; j’ai été mis en prison; je vais à Rome faire une
visite au roi mon père, détrôné ainsi que moi et mon grand-père; et je
suis venu passer le carnaval à  \bname{Venise}.»



Le quatrième prit alors la parole et dit: «Je suis roi des Polaques; le
sort de la guerre m’a privé de mes états héréditaires; mon père a
éprouvé les mêmes revers; je me résigne à la Providence comme le sultan
Achmet, l’empereur Ivan, et le roi Charles-Édouard, à qui Dieu donne
une longue vie; et je suis venu passer le carnaval à  \bname{Venise}.»



Le cinquième dit: «Je suis aussi roi des Polaques; j’ai perdu mon
royaume deux fois; mais la Providence m’a donné un autre état dans
lequel j’ai fait plus de bien que tous les rois des Sarmates ensemble
n’en ont jamais pu faire sur les bords de la Vistule. Je me résigne
aussi à la Providence; et je suis venu passer le carnaval à  \bname{Venise}.»



Il restait au sixième monarque à parler. «Messieurs, dit-il, je ne suis
pas si grand seigneur que vous; mais enfin j’ai été roi tout comme un
autre; je suis Théodore; on m’a élu roi en Corse; on m’a appelé
\textit{Votre Majesté}, et à présent à peine m’appelle-t-on \textit{monsieur}; j’ai
fait frapper de la monnaie, et je ne possède pas un denier; j’ai eu
deux secrétaires d’état, et j’ai à peine un valet; je me suis vu sur un
trône, et j’ai longtemps été à Londres en prison sur la paille; j’ai
bien peur d’être traité de même ici, quoique je sois venu, comme Vos
Majestés, passer le carnaval à  \bname{Venise}.»


Les cinq autres rois écoutèrent ce discours avec une noble compassion.
Chacun d’eux donna vingt sequins au roi \bname{Théodore} pour avoir des habits
et des chemises; \bname{Candide} lui fit présent d’un diamant de deux mille
sequins. «Quel est donc, disaient les cinq rois, cet homme qui est en
état de donner cent fois autant que chacun de nous, et qui le donne?
Êtes-vous roi aussi, monsieur? — Non, messieurs, et n’en ai nulle envie.»

Dans l’instant qu’on sortait de table, il arriva dans la même
hôtellerie quatre altesses sérénissimes qui avaient aussi perdu leurs
états par le sort de la guerre, et qui venaient passer le reste du
carnaval à  \bname{Venise}; mais \bname{Candide} ne prit pas seulement garde à ces
nouveaux venus. Il n’était occupé que d’aller trouver sa chère
\bname{Cunégonde} à \bname{Constantinople}.



\chapter{Voyage de \bname{Candide} à Constantinople}


\lettrine{L}{e fidèle} \bname{Cacambo} avait déjà obtenu du patron turc qui allait
reconduire le sultan Achmet à Constantinople qu’il recevrait \bname{Candide} et
\bname{Martin} sur son bord. L’un et l’autre s’y rendirent après s’être
prosternés devant Sa misérable Hautesse. \bname{Candide}, chemin faisant, disait
à \bname{Martin}: «Voilà pourtant six rois détrônés avec qui nous avons soupé!
et encore dans ces six rois il y en a un à qui j’ai fait l’aumône.
Peut-être y a-t-il beaucoup d’autres princes plus infortunés. Pour moi,
je n’ai perdu que cent moutons, et je vole dans les bras de \bname{Cunégonde}.
Mon cher \bname{Martin}, encore une fois, \bname{Pangloss} avait raison, tout est bien.
— Je le souhaite, dit \bname{Martin}. — Mais, dit \bname{Candide}, voilà une aventure bien
peu vraisemblable que nous avons eue à  \bname{Venise}. On n’avait jamais vu ni
ouï conter que six rois détrônés soupassent ensemble au cabaret. — Cela
n’est pas plus extraordinaire, dit \bname{Martin}, que la plupart des choses
qui nous sont arrivées. Il est très commun que des rois soient
détrônés; et à l’égard de l’honneur que nous avons eu de souper avec
eux, c’est une bagatelle qui ne mérite pas notre attention.»
% Qu’importe avec qui l’on soupe, pourvu qu’on fasse bonne chère?»

À peine \bname{Candide} fut-il dans le vaisseau, qu’il sauta au cou de son
ancien valet, de son ami \bname{Cacambo}. «Eh bien! lui dit-il, que fait
\bname{Cunégonde}? est-elle toujours un prodige de beauté? m’aime-t-elle
toujours? comment se porte-t-elle? Tu lui as, sans doute, acheté un
palais à Constantinople?



— Mon cher maître, répondit \bname{Cacambo}, \bname{Cunégonde} lave les écuelles sur le
bord de la Propontide, chez un prince qui a très peu d’écuelles; elle
est esclave dans la maison d’un ancien souverain, nommé \bname{Ragotski}, à
qui le Grand-Turc donne trois écus par jour dans son asile; mais, ce
qui est bien plus triste, c’est qu’elle a perdu sa beauté, et qu’elle
est devenue horriblement laide. — Ah! belle ou laide, dit \bname{Candide}, je
suis honnête homme, et mon devoir est de l’aimer toujours. Mais comment
peut-elle être réduite à un état si abject avec les cinq ou six
millions que tu avais emportés? — Bon, dit \bname{Cacambo}, ne m’en a-t-il pas
fallu donner deux au senor don \bname{Fernando} d’Ibaraa, y Figueora, y
Mascarenès, y \bname{Lampourdos}, y Souza, gouverneur de \bname{Buenos} Aires, pour
avoir la permission de reprendre M\up{lle}~\bname{Cunégonde}? et un pirate
ne nous a-t-il pas bravement dépouillés de tout le reste? Ce pirate ne
nous a-t-il pas menés au cap de Matapan, à Milo, à Nicarie, à Samos, à
Petra, aux \bname{Dardanelles}, à Marmara, à \bname{Scutari}? \bname{Cunégonde} et la vieille
servent chez ce prince dont je vous ai parlé, et moi je suis esclave du
sultan détrôné. — Que d’épouvantables calamités enchaînées les unes aux
autres! dit \bname{Candide}. Mais, après tout, j’ai encore quelques diamants;
je délivrerai aisément \bname{Cunégonde}. C’est bien dommage qu’elle soit
devenue si laide.»



Ensuite, se tournant vers \bname{Martin}: «Que pensez-vous, dit-il, qui soit le
plus à plaindre, de l’empereur Achmet, de l’empereur Ivan, du roi
Charles-Édouard, ou de moi? — Je n’en sais rien, dit \bname{Martin}; il faudrait
que je fusse dans vos cœurs pour le savoir. — Ah! dit \bname{Candide}, si
\bname{Pangloss} était ici, il le saurait, et nous l’apprendrait. — Je ne sais,
dit \bname{Martin}, avec quelles balances votre \bname{Pangloss} aurait pu peser les
infortunes des hommes, et apprécier leurs douleurs. Tout ce que je
présume c’est qu’il y a des millions d’hommes sur la terre cent fois
plus à plaindre que le roi Charles-Édouard, l’empereur Ivan, et le
sultan Achmet. — Cela pourrait bien être», dit \bname{Candide}.

On arriva en peu de jours sur le canal de la mer Noire. \bname{Candide}
commença par racheter \bname{Cacambo} fort cher; et, sans perdre de temps, il
se jeta dans une galère, avec ses compagnons, pour aller sur le rivage
de la Propontide chercher \bname{Cunégonde}, quelque laide qu’elle pût être.


Il y avait dans la chiourme deux forçats qui ramaient fort mal, et à
qui le levanti patron appliquait de temps en temps quelques coups de
nerf de bœuf sur leurs épaules nues; \bname{Candide}, par un mouvement
naturel, les regarda plus attentivement que les autres galériens, et
s’approcha d’eux avec pitié. Quelques traits de leurs visages défigurés
lui parurent avoir un peu de ressemblance avec \bname{Pangloss} et avec ce
malheureux jésuite, ce baron, ce frère de M\up{lle}~\bname{Cunégonde}. Cette
idée l’émut et l’attrista. Il les considéra encore plus attentivement.
«En vérité, dit-il à \bname{Cacambo}, si je n’avais pas vu pendre maître
\bname{Pangloss}, et si je n’avais pas eu le malheur de tuer le baron, je
croirais que ce sont eux qui rament dans cette galère.»


Au nom du baron et de \bname{Pangloss} les deux forçats poussèrent un grand
cri, s’arrêtèrent sur leur banc, et laissèrent tomber leurs rames. Le
levanti patron accourait sur eux, et les coups de nerf de bœuf
\bname{redoublaient}. «Arrêtez! arrêtez! seigneur, s’écria \bname{Candide}; je vous
donnerai tant d’argent que vous voudrez. — Quoi! c’est \bname{Candide}! disait
l’un des forçats; — quoi! c’est \bname{Candide}! disait l’autre. — Est-ce un songe?
dit \bname{Candide}; veillé-je? suis-je dans cette galère? Est-ce là M.
le baron, que j’ai tué? est-ce là maître \bname{Pangloss}, que j’ai vu pendre?


— C’est nous-mêmes, c’est nous-mêmes, répondaient-ils. — Quoi! c’est là ce
grand philosophe? disait \bname{Martin}. — Eh! monsieur le levanti patron, dit
\bname{Candide}, combien voulez-vous d’argent pour la rançon de M. de
Thunder-ten-tronckh, un des premiers barons de l’empire, et de M.~\bname{Pangloss}, le plus profond métaphysicien d’Allemagne? — Chien de chrétien,
répondit le levanti patron, puisque ces deux chiens de forçats
chrétiens sont des barons et des métaphysiciens, ce qui est sans doute
une grande dignité dans leur pays, tu m’en donneras cinquante mille
sequins. — Vous les aurez, monsieur; ramenez-moi comme un éclair à
Constantinople, et vous serez payé sur-le-champ. Mais non, menez-moi
chez M\up{lle}~\bname{Cunégonde}.» Le levanti patron, sur la première offre
de \bname{Candide}, avait déjà tourné la proue vers la ville, et il faisait
ramer plus vite qu’un oiseau ne fend les airs.

\bname{Candide} embrassa cent fois le baron et \bname{Pangloss}. «Et comment ne vous
ai-je pas tué, mon cher baron? et mon cher \bname{Pangloss}, comment êtes-vous
en vie, après avoir été pendu? et pourquoi êtes-vous tous deux aux
galères en Turquie? — Est-il bien vrai que ma chère sœur soit dans ce
pays? disait le baron. — Oui, répondait \bname{Cacambo}. — Je revois donc mon cher
\bname{Candide}! s’écriait \bname{Pangloss}.» \bname{Candide} leur présentait \bname{Martin} et \bname{Cacambo}.
Ils s’embrassaient tous; ils parlaient tous à-la-fois. La galère
volait, ils étaient déjà dans le port. On fit venir un Juif, à qui
\bname{Candide} vendit pour cinquante mille sequins un diamant de la valeur de
cent mille, et qui lui jura par Abraham qu’il n’en pouvait donner
davantage. Il paya incontinent la rançon du baron et de \bname{Pangloss}.
Celui-ci se jeta aux pieds de son libérateur, et les baigna de larmes;
l’autre le remercia par un signe de tête, et lui promit de lui rendre
cet argent à la première occasion. «Mais est-il bien possible que ma
sœur soit en Turquie? disait-il. — Rien n’est si possible, reprit
\bname{Cacambo}, puisqu’elle écure la vaisselle chez un prince de \bname{Transylvanie}.»
On fit aussitôt venir deux Juifs; \bname{Candide} vendit encore des diamants;
et ils repartirent tous dans une autre galère pour aller délivrer
\bname{Cunégonde}.


\chapter[Ce qui arriva à Candide, à Cunégonde…]{Ce qui arriva à \bname{Candide}, à \bname{Cunégonde},\\à \bname{Pangloss}, à \bname{Martin}, etc.}


\lettrine[findent=1pt,ante=«]{P}{ardon}, encore une fois, dit \bname{Candide} au baron; pardon, mon Révérend
Père, de vous avoir donné un grand coup d’épée au travers du corps.
— N’en parlons plus, dit le baron; je fus un peu trop vif, je l’avoue;
mais puisque vous voulez savoir par quel hasard vous m’avez vu aux
galères, je vous dirai qu’après avoir été guéri de ma blessure par le
frère apothicaire du collège, je fus attaqué et enlevé par un parti
espagnol; on me mit en prison à \bname{Buenos} Aires dans le temps que ma sœur
venait d’en partir. Je demandai à retourner à Rome auprès du père
général. Je fus nommé pour aller servir d’aumônier à Constantinople
auprès de M. l’ambassadeur de France. Il n’y avait pas huit jours
que j’étais entré en fonction, quand je trouvai sur le soir un jeune
icoglan très bien fait. Il faisait fort chaud: le jeune homme voulut se
baigner; je pris cette occasion de me baigner aussi. Je ne savais pas
que ce fût un crime capital pour un chrétien d’être trouvé tout nu avec
un jeune musulman. Un cadi me fit donner cent coups de bâton sous la
plante des pieds, et me condamna aux galères. Je ne crois pas qu’on ait
fait une plus horrible injustice. Mais je voudrais bien savoir pourquoi
ma sœur est dans la cuisine d’un souverain de \bname{Transylvanie} réfugié
chez les Turcs.


— Mais vous, mon cher \bname{Pangloss}, dit \bname{Candide}, comment se peut-il que je
vous revoie? — Il est vrai, dit \bname{Pangloss}, que vous m’avez vu pendre; je
devais naturellement être brûlé mais vous vous souvenez qu’il plut à
verse lorsqu’on allait me cuire: l’orage fut si violent qu’on désespéra
d’allumer le feu; je fus pendu, parce qu’on ne put mieux faire: un
chirurgien acheta mon corps, m’emporta chez lui, et me disséqua. Il me
fit d’abord une incision cruciale depuis le nombril jusqu’à la
clavicule. On ne pouvait pas avoir été plus mal pendu que je l’avais
été. L’exécuteur des hautes œuvres de la sainte inquisition, lequel
était sous-diacre, brûlait à la vérité les gens à merveille, mais il
n’était pas accoutumé à pendre: la corde était mouillée et glissa mal,
elle fut mal nouée; enfin je respirais encore: l’incision cruciale me
fit jeter un si grand cri, que mon chirurgien tomba à la renverse; et
croyant qu’il disséquait le diable, il s’enfuit en mourant de peur, et
tomba encore sur l’escalier en fuyant. Sa femme accourut au bruit d’un
cabinet voisin: elle me vit sur la table étendu avec mon incision
cruciale; elle eut encore plus de peur que son mari, s’enfuit, et tomba
sur lui. Quand ils furent un peu revenus à eux, j’entendis la
chirurgienne qui disait au chirurgien: \frquote{\localleftbox{« }Mon bon, de quoi vous
avisez-vous aussi de disséquer un hérétique? ne savez-vous pas que le diable est toujours dans le corps de ces gens-là? je vais vite chercher un prêtre pour l’exorciser.}\localleftbox{}
Je frémis à ce propos, et je ramassai le
peu de forces qui me restaient pour crier: «Ayez pitié de moi!» Enfin le
barbier portugais s’enhardit: il recousit ma peau; sa femme même eut
soin de moi; je fus sur pied au bout de quinze jours. Le barbier me
trouva une condition, et me fit laquais d’un chevalier de Malte qui
allait à  \bname{Venise}: mais mon maître n’ayant pas de quoi me payer, je me
mis au service d’un marchand vénitien, et je le suivis à
Constantinople.


«Un jour il me prit fantaisie d’entrer dans une mosquée; il n’y avait
qu’un vieil iman et une jeune dévote très jolie qui disait ses
patenôtres; sa gorge était toute découverte: elle avait entre ses deux
tétons un beau bouquet de tulipes, de roses, d’anémones, de renoncules,
d’hyacinthes, et d’oreilles d’ours: elle laissa tomber son 
bouquet; je
le ramassai, et je le lui remis avec un empressement très respectueux.
Je fus si longtemps à le lui remettre, que l’iman se mit en colère, et
voyant que j’étais chrétien, il cria à l’aide. On me mena chez le cadi,
qui me fit donner cent coups de latte sous la plante des pieds, et
m’envoya aux galères. Je fus enchaîné précisément dans la même galère
et au même banc que M.~le baron. Il y avait dans cette galère
quatre jeunes gens de Marseille, cinq prêtres napolitains, et deux
moines de Corfou, qui nous dirent que de pareilles aventures arrivaient
tous les jours. M.~le baron prétendait qu’il avait essuyé une
plus grande injustice que moi: je prétendais, moi, qu’il était beaucoup
plus permis de remettre un bouquet sur la gorge d’une femme que d’être
tout nu avec un icoglan. Nous disputions sans cesse, et nous recevions
vingt coups de nerf de bœuf par jour, lorsque l’enchaînement des
événements de cet univers vous a conduit dans notre galère, et que vous
nous avez rachetés.

— Eh bien! mon cher \bname{Pangloss}, lui dit \bname{Candide}, quand vous avez été pendu,
disséqué, roué de coups, et que vous avez ramé aux galères, avez-vous
toujours pensé que tout allait le mieux du monde? — Je suis toujours de
mon premier sentiment, répondit \bname{Pangloss}; car enfin je suis philosophe;
il ne me convient pas de me dédire, Leibnitz ne pouvant pas avoir tort,
et l’harmonie préétablie étant d’ailleurs la plus belle chose du monde,
aussi bien que le plein et la matière subtile.»



\chapter{Comment \bname{Candide} retrouva \bname{Cunégonde} et la vieille}


\lettrine{P}{endant} que \bname{Candide}, le baron, \bname{Pangloss}, \bname{Martin}, et \bname{Cacambo}, contaient
leurs aventures, qu’ils raisonnaient sur les événements contingents ou
non contingents de cet univers, qu’ils disputaient sur les effets et
les causes, sur le mal moral et sur le mal physique, sur la liberté et
la nécessité, sur les consolations que l’on peut éprouver lorsqu’on est
aux galères en Turquie, ils abordèrent sur le rivage de la Propontide,
à la maison du prince de \bname{Transylvanie}. Les premiers objets qui se
présentèrent furent \bname{Cunégonde} et la vieille, qui étendaient des
serviettes sur des ficelles pour les faire sécher.

Le baron pâlit à cette vue. Le tendre amant \bname{Candide} en voyant sa belle
\bname{Cunégonde} rembrunie, les yeux éraillés, la gorge sèche, les joues
ridées, les bras rouges et écaillés, recula trois pas, saisi d’horreur,
et avança ensuite par bon procédé. Elle embrassa \bname{Candide} et son frère:
on embrassa la vieille: \bname{Candide} les racheta toutes deux.

Il y avait une petite métairie dans le voisinage; la vieille proposa à
\bname{Candide} de s’en accommoder, en attendant que toute la troupe eût une
meilleure destinée. \bname{Cunégonde} ne savait pas qu’elle était enlaidie,
personne ne l’en avait avertie: elle fit souvenir \bname{Candide} de ses
promesses avec un ton si absolu, que le bon \bname{Candide} n’osa pas la
refuser. Il signifia donc au baron qu’il allait se marier avec sa
sœur. «Je ne souffrirai jamais, dit le baron, une telle bassesse de sa
part, et une telle insolence de la vôtre; cette infamie ne me sera
jamais reprochée: les enfants de ma sœur ne pourraient entrer dans les
chapitres d’Allemagne. Non, jamais ma sœur n’épousera qu’un baron de
l’empire.» \bname{Cunégonde} se jeta à ses pieds, et les baigna de larmes; il
fut inflexible. «Maître fou, lui dit \bname{Candide}, je t’ai réchappé des
galères, j’ai payé ta rançon, j’ai payé celle de ta sœur; elle lavait
ici des écuelles, elle est laide, j’ai la bonté d’en faire ma femme; et
tu prétends encore t’y opposer! je te retuerais si j’en croyais ma
colère. — Tu peux me tuer encore, dit le baron, mais tu n’épouseras pas
ma sœur de mon vivant.»




\chapter{Conclusion}


\lettrine{C}{andide}, dans le fond de son cœur, n’avait aucune envie d’épouser
\bname{Cunégonde}; mais l’impertinence extrême du baron le déterminait à
conclure le mariage; et \bname{Cunégonde} le pressait si vivement qu’il ne
pouvait s’en dédire. Il consulta \bname{Pangloss}, \bname{Martin}, et le fidèle
\bname{Cacambo}. \bname{Pangloss} fit un beau mémoire par lequel il prouvait que le
baron n’avait nul droit sur sa sœur, et qu’elle pouvait, selon toutes
les lois de l’empire, épouser \bname{Candide} de la main gauche. \bname{Martin} conclut
à jeter le baron dans la mer; \bname{Cacambo} décida qu’il fallait le rendre au
levanti patron, et le remettre aux galères, après quoi on l’enverrait à
Rome au père général par le premier vaisseau. L’avis fut trouvé fort
bon; la vieille l’approuva; on n’en dit rien à sa sœur; la chose fut
exécutée pour quelque argent, et on eut le plaisir d’attraper un
jésuite, et de punir l’orgueil d’un baron allemand.


Il était tout naturel d’imaginer qu’après tant de désastres, \bname{Candide}
marié avec sa maîtresse, et vivant avec le philosophe \bname{Pangloss}, le
philosophe \bname{Martin}, le prudent \bname{Cacambo}, et la vieille, ayant d’ailleurs
rapporté tant de diamants de la patrie des anciens Incas, mènerait la
vie du monde la plus agréable; mais il fut tant friponné par les Juifs,
qu’il ne lui resta plus rien que sa petite métairie; sa femme devenant
tous les jours plus laide devint acariâtre et insupportable: la vieille
était infirme, et fut encore de plus mauvaise humeur que \bname{Cunégonde}.
\bname{Cacambo}, qui travaillait au jardin, et qui allait vendre des \bname{légumes} à
Constantinople, était excédé de travail, et maudissait sa destinée.
\bname{Pangloss} était au désespoir de ne pas briller dans quelque université
d’Allemagne. Pour \bname{Martin}, il était fermement persuadé qu’on est
également mal partout; il prenait les choses en patience. \bname{Candide},
\bname{Martin}, et \bname{Pangloss}, disputaient quelquefois de métaphysique et de
morale. On voyait souvent passer sous les fenêtres de la métairie des
bateaux chargés d’effendis, de bachas, de cadis, qu’on envoyait en exil
à Lemnos, à Mytilène, à Erzeroum: on voyait venir d’autres cadis,
d’autres bachas, d’autres effendis, qui prenaient la place des
expulsés, et qui étaient expulsés à leur tour: on voyait des têtes
proprement empaillées qu’on allait présenter à la sublime Porte. Ces
spectacles faisaient redoubler les dissertations; et quand on ne
disputait pas, l’ennui était si excessif, que la vieille osa un jour
leur dire: «Je voudrais savoir lequel est le pire, ou d’être violée cent
fois par des pirates nègres, d’avoir une fesse coupée, de passer par
les baguettes chez les \bname{Bulgares}, d’être fouetté et pendu dans un
auto-da-fé, d’être disséqué, de ramer en galère, d’éprouver enfin
toutes les misères par lesquelles nous avons tous passé, ou bien de
rester ici à ne rien faire? — C’est une grande question», dit \bname{Candide}.






Ce discours fit naître de nouvelles réflexions, et \bname{Martin} surtout
conclut que l’homme était né pour vivre dans les convulsions de
l’inquiétude, ou dans la léthargie de l’ennui. \bname{Candide} n’en convenait
pas, mais il n’assurait rien. \bname{Pangloss} avouait qu’il avait toujours
horriblement souffert; mais ayant soutenu une fois que tout allait à
merveille, il le soutenait toujours, et n’en croyait rien.

Une chose acheva de confirmer \bname{Martin} dans ses détestables principes, de
faire hésiter plus que jamais \bname{Candide} et d’embarrasser \bname{Pangloss}. C’est
qu’ils virent un jour aborder dans leur métairie \bname{Paquette} et le frère
Giroflée, qui étaient dans la plus extrême misère; ils avaient bien
vite mangé leurs trois mille piastres, s’étaient quittés, s’étaient
raccommodés, s’étaient brouillés, avaient été mis en 
prison; s’étaient
enfuis, et enfin frère Giroflée s’était fait turc. \bname{Paquette} continuait
son métier partout, et n’y gagnait plus rien. «Je l’avais bien prévu,
dit \bname{Martin} à \bname{Candide}, que vos présents seraient bientôt dissipés, et ne
les rendraient que plus misérables. Vous avez regorgé de millions de
piastres, vous et \bname{Cacambo}, et vous n’êtes pas plus heureux que frère
Giroflée et \bname{Paquette}. — Ah! ah! dit \bname{Pangloss} à \bname{Paquette}, le ciel vous
ramène donc ici parmi nous. Ma pauvre enfant! savez-vous bien que vous
m’avez coûté le bout du nez, un œil, et une oreille? Comme vous voilà
faite! Eh! qu’est-ce que ce monde!» Cette nouvelle aventure les engagea
à philosopher plus que jamais.

Il y avait dans le voisinage un derviche très fameux qui passait pour
le meilleur philosophe de la Turquie; ils allèrent le consulter;
\bname{Pangloss} porta la parole, et lui dit: «Maître, nous venons vous prier de
nous dire pourquoi un aussi étrange animal que l’homme a été formé.






— De quoi te mêles-tu? lui dit le derviche; est-ce là ton affaire? — Mais,
mon Révérend Père, dit \bname{Candide}, il y a horriblement de mal sur la
terre. — Qu’importe, dit le derviche, qu’il y ait du mal ou du bien?
quand Sa Hautesse envoie un vaisseau en Égypte, s’embarrasse-t-elle si
les souris qui sont dans le vaisseau sont à leur aise ou non? — Que
faut-il donc faire? dit \bname{Pangloss}. — Te taire, dit le derviche. — Je me
flattais, dit \bname{Pangloss}, de raisonner un peu avec vous des effets et des
causes, du meilleur des mondes possibles, de l’origine du mal, de la
nature de l’âme, et de l’harmonie préétablie.» Le derviche, à ces mots,
leur ferma la porte au nez.



Pendant cette conversation, la nouvelle s’était répandue qu’on venait
d’étrangler à Constantinople deux vizirs du banc et le muphti, et qu’on
avait empalé plusieurs de leurs amis. Cette catastrophe faisait partout
un grand bruit pendant quelques heures. \bname{Pangloss}, \bname{Candide}, et \bname{Martin},
en retournant à la petite métairie, rencontrèrent un bon vieillard qui
prenait le frais à sa porte sous un berceau d’orangers. \bname{Pangloss}, qui
était aussi curieux que raisonneur, lui demanda comment se nommait le
muphti qu’on venait d’étrangler. «Je n’en sais rien, répondit le
bonhomme, et je n’ai jamais su le nom d’aucun muphti ni d’aucun vizir.
J’ignore absolument l’aventure dont vous me parlez; je présume qu’en
général ceux qui se mêlent des affaires publiques périssent quelquefois
misérablement, et qu’ils le méritent; mais je ne m’informe jamais de ce
qu’on fait à Constantinople; je me contente d’y envoyer vendre les
fruits du jardin que je cultive.» Ayant dit ces mots, il fit entrer les
étrangers dans sa maison; ses deux filles et ses deux fils leur
présentèrent plusieurs sortes de sorbets qu’ils faisaient eux-mêmes, du
kaïmak piqué d’écorces de cédrat confit, des oranges, des citrons, des
limons, des ananas, des dattes, des pistaches, du café de Moka qui
n’était point mêlé avec le mauvais café de Batavia et des îles. Après
quoi les deux filles de ce bon musulman parfumèrent les barbes de
\bname{Candide}, de \bname{Pangloss}, et de \bname{Martin}.



«Vous devez avoir, dit \bname{Candide} au Turc, une vaste et magnifique terre?
— Je n’ai que vingt arpents, répondit le Turc; je les cultive avec mes
enfants; le travail éloigne de nous trois grands maux, l’ennui, le
vice, et le besoin.»


\bname{Candide} en retournant dans sa métairie fit de profondes réflexions sur
le discours du Turc. Il dit à \bname{Pangloss} et à \bname{Martin}: «Ce bon vieillard me
paraît s’être fait un sort bien préférable à celui des six rois avec
qui nous avons eu l’honneur de souper. — Les grandeurs, dit \bname{Pangloss},
sont fort dangereuses, selon le rapport de tous les philosophes; car
enfin Églon, roi des Moabites, fut assassiné par Aod; \bname{Absalon} fut pendu
par les cheveux et percé de trois dards; le roi \bname{Nadab}, fils de
\bname{Jéroboam}, fut tué par \bname{Baasa}; le roi Éla, par \bname{Zambri}; \bname{Ochosias}, par
\bname{Jéhu}; \bname{Athalie}, par \bname{Joïada}; les rois \bname{Joachim}, \bname{Jéchonias}, \bname{Sédécias},
furent esclaves. Vous savez comment périrent \bname{Crésus}, \bname{Astyage}, \bname{Darius},
\bname{Denys} de \bname{Syracuse}, \bname{Pyrrhus}, \bname{Persée}, \bname{Annibal}, \bname{Jugurtha}, \bname{Arioviste},
\bname{César}, \bname{Pompée}, \bname{Néron}, \bname{Othon}, \bname{Vitellius}, \bname{Domitien}, \bname{Richard}~II
\bname{d’Angleterre}, \bname{Édouard}~II, \bname{Henri}~VI, \bname{Richard}~III, \mbox{Marie Stuart}, Charles~I\up{er}, les trois \bname{Henri} de France, l’empereur \bname{Henri}~IV? Vous savez… — Je
sais aussi, dit \bname{Candide}, qu’il faut cultiver notre jardin. — Vous avez
raison, dit \bname{Pangloss}; car, quand l’homme fut mis dans le jardin d’Éden,
il y fut mis \emph{ut operaretur eum}, pour qu’il travaillât; ce qui prouve
que l’homme n’est pas né pour le repos. — Travaillons sans raisonner, dit
\bname{Martin}, c’est le seul moyen de rendre la vie supportable.

Toute la petite société entra dans ce louable dessein; chacun se mit à
exercer ses talents. La petite terre rapporta beaucoup. \bname{Cunégonde}
était, à la vérité, bien laide; mais elle devint une excellente
pâtissière; \bname{Paquette} broda; la vieille eut soin du linge. Il n’y eut
pas jusqu’à frère Giroflée qui ne rendît service; il fut un très bon
menuisier, et même devint honnête homme: et \bname{Pangloss} disait quelquefois
à \bname{Candide}: «Tous les événements sont enchaînés dans le meilleur des
mondes possibles; car enfin si vous n’aviez pas été chassé d’un beau
château à grands coups de pied dans le derrière pour l’amour de
M\up{lle}~\bname{Cunégonde}, si vous n’aviez pas été mis à l’inquisition, si
vous n’aviez pas couru l’Amérique à pied, si vous n’aviez pas donné un
bon coup d’épée au baron, si vous n’aviez pas perdu tous vos moutons du
bon pays d’Eldorado, vous ne mangeriez pas ici des cédrats confits et
des pistaches. — Cela est bien dit, répondit \bname{Candide}, mais il faut
cultiver notre jardin.»


\endinput



