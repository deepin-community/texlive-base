% !TEX TS-program = LuaLaTeX
\documentclass[french,no-indent,FontSize=11pt]{tango}
\usepackage{numprint}
%
\NewCommandCopy{\np}{\numprint}
%
\newcommand\headcontentL{}
\newcommand\headcontentC{}
\newcommand\headcontentR{}
\newcommand\footcontentL{}
\newcommand\footcontentC{}
\newcommand\footcontentR{}
\newlength\SWruleheight
\setlength\SWruleheight{0.4pt}
\NewDocumentCommand\headL{m}{\renewcommand\headcontentL%
{\sffamily\fontseries{l}\selectfont#1}}
%
\NewDocumentCommand\headC{m}{\renewcommand\headcontentC%
{\sffamily\fontseries{l}\selectfont#1}}
%
\NewDocumentCommand\headR{m}{\renewcommand\headcontentR%
{\sffamily\fontseries{l}\selectfont#1}}
%
\NewDocumentCommand\footL{m}{\renewcommand\footcontentL%
{\sffamily\fontseries{l}\selectfont#1}}
%
\NewDocumentCommand\footC{m}{\renewcommand\footcontentC%
{\sffamily\fontseries{l}\selectfont#1}}
%
%
\NewDocumentCommand\footR{m}{\renewcommand\footcontentR%
{\sffamily\fontseries{l}\selectfont#1}}
%
\newpagestyle{schoolwork}{%
\setheadrule{\SWruleheight}%
\setfootrule{\SWruleheight}%
\sethead{\headcontentL}%
{\headcontentC}%
{\headcontentR}%
\setfoot{\footcontentL}%
{\footcontentC}%
{\footcontentR}
}


\headL{Le 13 novembre 2024}
\headC{Entraînement au Devoir Commun \no 1}
\headR{Durée : 2 heures}
\footL{Lycée Jean-Baptiste \textsc{Colbert}}
\footC{Classe de 1\iere~3}
\footR{page~\thepage}
%
%
\pagestyle{schoolwork}
\geometry{top=2cm,outer=2cm,bottom=2cm,inner=2cm,headheight=18pt,headsep=25pt,footskip=34pt}

\begin{document}
 

\exo[Thème : suites — 4 points]


On considère la suite $\left(u_n\right)$ définie par $u_0 = \numprint{10000}$ et pour tout entier naturel $n$
\[u_{n+1} = 0{,}95u_n + 200.\]

\begin{enumerate}
\item Calculer $u_1$ et vérifier que $u_2 = \np{9415}$. 
\item 
	\begin{enumerate}
		\item Démontrer, à l'aide d'un raisonnement par récurrence, que pour tout entier naturel $n$
\[u_n >\np{4000}.\]
\item Démontrer que la suite $\left(u_n\right)$ est décroissante. 		 	\end{enumerate}
\item  Pour tout entier naturel $n$, on considère la suite $\left(v_n\right)$ définie par $v_n = u_n -  \np{4000}$.
	\begin{enumerate}
		\item Calculer $v_0$.
		\item Démontrer que la suite $\left(v_n\right)$ est géométrique de raison égale à $0{,}95$.
		\item En déduire que pour tout entier naturel $n$ :
\[u_n = \np{4000} + \np{6000} \times 0{,}95^n.\]
		\item Émettre une conjecture quant à la limite de la suite $\left(u_n\right)$ ? 
	\end{enumerate}
\item En 2020, une espèce animale comptait \np{10000} individus. L'évolution observée les années précédentes conduit à estimer qu'à partir de l'année 2021, cette population baissera de 5~\% chaque début d'année.

Pour ralentir cette baisse, il a été décidé de réintroduire $200$ individus à  la fin de chaque année, à partir de 2021.

Une responsable d'une association soutenant cette stratégie affirme que : \og l'espèce ne devrait pas s'éteindre, mais malheureusement, nous n'empêcherons pas une disparition de plus de la moitié de la population \fg.

Que pensez-vous de cette affirmation ? Justifier la réponse.
\end{enumerate}



\exo[Thème : probablités — 4 points]

Selon les autorités sanitaires d'un pays, 7~\% des habitants sont affectés par une certaine maladie.

Dans ce pays, un test est mis au point pour détecter cette maladie. Ce test a les caractéristiques suivantes :
\begin{itemize}
\item Pour les individus malades, le test donne un résultat négatif dans $20 \,\%$ des cas ;
\item Pour les individus sains, le test donne un résultat positif dans $1\,\%$ des cas.
\end{itemize}

Une personne est choisie au hasard dans la population et testée.

On considère les événements suivants :
\begin{itemize}
\item $M$ : \og la personne est malade \fg{} ;
\item $T$ : \og le test est positif \fg{}.
\end{itemize}

\begin{enumerate}
\item Calculer la probabilité de l'événement $M \cap T$. On pourra s'appuyer sur un arbre pondéré.
\item Démontrer que la probabilité que le test  de la personne choisie au hasard soit positif, est de $\np{0,0653}$.
\item On considère dans cette question que la personne choisie au hasard a eu un test positif.
Quelle est la probabilité qu'elle soit malade ? On arrondira le résultat à $10^{-2}$.
\item On choisit des personnes au hasard dans la population. La taille de la population de ce pays permet d'assimiler ce prélèvement à un tirage avec remise.

On note $X$ la variable aléatoire qui donne le nombre d'individus ayant un test positif parmi les 10 personnes. 
	\begin{enumerate}
		\item Préciser la nature et les paramètres de la loi de probabilité suivie par $X$.
		\item Déterminer la probabilité pour qu'exactement deux personnes aient un test positif. On arrondira le résultat à $10^{-2}$ près.
	\end{enumerate}	
\end{enumerate}

\exo[Thème : géométrie dans l'espace — 2 points]



Dans l'espace rapporté à un repère orthonormé $(O,\vec\imath,\vec\jmath,\vec k)$ , on considère les points
\[A(-1,-1,3),\qquad B(1,1,2),\qquad  C(1,-1,7).\]

On considère également la droite $\Delta$ passant par les points $D(-1,6,8)$ et $E(11,- 9,2)$.

Pour chacune des questions posées, quatre affirmations \textbf{dont une seule est correcte} sont proposées.
Entourer la bonne réponse directement sur l'énoncé.

Aucune justification n'est attendue.

Une réponse correcte apporte 0,5 point. Une réponse incorrecte ou une absence de réponse n'enlève pas de point.  


	\begin{enumerate}
		\item Une représentation paramétrique de la droite $\Delta$ est 
\medskip
			
\textbf{a.} $\left\{
		\begin{aligned}
		x&= -1 + 11t\\
		y&= 6 - 9t\\
		z&= 8 +2t\\
		\end{aligned} \right.~(t \in \symbb{R})$
		 \hfill
\textbf{b.} $\left\{
		\begin{aligned}
		x&= -1 + 4t\\
		y&= 6 - 5t\\
		z&= 8 -2t\\
		\end{aligned} \right.~(t \in \symbb{R})$
		\hfill 
\textbf{c.} $\left\{
		\begin{aligned}
		x&= 11 -t\\
		y&= - 9+6t\\
		z&= 2 +8t\\
		\end{aligned} \right.~(t \in \symbb{R})$
		\hfill
 \textbf{d.}
		$\left\{
		\begin{aligned}
		x&= 4 -t\\
		y&= -5 +6t\\
		z&= -2 + 8 t\\
		\end{aligned} \right.~(t \in \symbb{R})$\hfill\null

\bigskip		
		
		\item Lequel de ces points appartient-il à  la droite $\Delta$ ?

\medskip
		
		\textbf{a.} Le point $F(-3,1,6)$  
		\hfill \textbf{b.}  Le point $G (5,\:-1{,}5,\:5)$
		\hfill \textbf{c.}  Le point $H (10,-3,10)$ 
		\hfill \textbf{d.} Le point $K (12,-15,-6)$ 

\bigskip

		\item Pour savoir si les points $A$, $B$ et $C$ définissent un plan, il faut vérifier que $\overrightarrow{AB}$ et $\overrightarrow{AC}$ : 
		
\medskip
	
			\textbf{a.} sont colinéaires. 
			\hfill \textbf{b.} sont coplanaires. 
			\hfill \textbf{c.} ne sont pas colinéaires.  
			\hfill \textbf{d.} ne sont pas coplanaires.
			
\bigskip
	
		\item La droite $\Delta$ et le plan (ABC) :

\medskip	
		
		\begin{tabular}{ll}
			\textbf{a.} sont strictement parallèles.  
			&\textbf{b.} ont une infinité de points communs.\\[12pt]
			\textbf{c.} sont sécants en   $I(-13,21,14)$   
			&\textbf{d.} sont sécants en  $J(3,1,6)$ 
		\end{tabular}
		
		\bigskip
			
\item Les droites $\delta$ et $\delta'$ de représentations paramétriques respectives : 
			
			$ \delta : \left\{
		\begin{aligned}
		x&= -1 + 11k\\
		y&= 6 - 9k\\
		z&= 8 +2k\\
		\end{aligned} \right.~(k \in \symbb{R})$
		 \quad et\quad  $\delta' : \left\{
		\begin{aligned}
		x&= 11 -\ell\\
		y&= - 9+6\ell\\
		z&= 2 +8\ell\\
		\end{aligned} \right.~(\ell \in \symbb{R})$\quad sont : 
		
		\medskip
		
	\begin{tabular}{ll}	
	\textbf{a.} sécantes au point de coordonnées $(1,1,1)$. 
	& \textbf{b.} sécantes au point de coordonnées $(-12,15,6)$.\\[12pt] 
	\textbf{c.} sécantes en un autre point
	&\textbf{d.} non sécantes.  
	\end{tabular}
		
			
\end{enumerate}

\end{document}