\chapter[Comment \bname{Candide} fut élevé…]
{Comment \bname{Candide} fut élevé dans un\\ beau château,\\ et comment il fut
chassé d’icelui}


\lettrine{I}{l y avait} en \bname{Westphalie}, dans le château de M.~le baron de
Thunder-ten-tronckh, un jeune garçon à qui la nature avait donné les
mœurs les plus douces. Sa physionomie annonçait son âme. Il avait le
jugement assez droit, avec l’esprit le plus simple; c’est, je crois,
pour cette raison qu’on le nommait \bname{Candide}. Les anciens domestiques de
la maison soupçonnaient qu’il était fils de la sœur de M.~le
baron et d’un bon et honnête gentilhomme du voisinage, que cette
demoiselle ne voulut jamais épouser parce qu’il n’avait pu prouver que
soixante et onze quartiers, et que le reste de son arbre généalogique
avait été perdu par l’injure du temps.

M. le baron était un des plus puissants seigneurs de la
\bname{Westphalie}, car son château avait une porte et des fenêtres. Sa grande
salle même était ornée d’une tapisserie. Tous les chiens de ses
basses-cours composaient une meute dans le besoin; ses palefreniers
étaient ses piqueurs; le vicaire du village était son grand-aumônier.
Ils l’appelaient tous \bname{Monseigneur}, et ils riaient quand il faisait des
contes.

M\up{me} la baronne, qui pesait environ trois cent cinquante livres,
s’attirait par là une très grande considération, et faisait les honneurs
de la maison avec une dignité qui la rendait encore plus respectable.
Sa fille \bname{Cunégonde}, âgée de dix-sept ans, était haute en couleur,
fraîche, grasse, appétissante. Le fils du baron paraissait en tout
digne de son père. Le précepteur \bname{Pangloss} était l’oracle de la
maison, et le petit \bname{Candide} écoutait ses leçons avec toute la bonne foi
de son âge et de son caractère.




\bname{Pangloss} enseignait la métaphysico-théologo-cosmo\-lo\-ni\-go\-logie. Il
prouvait admirablement qu’il n’y a point d’effet sans cause, et que,
dans ce meilleur des mondes possibles, le château de monseigneur le
baron était le plus beau des châteaux, et madame la meilleure des
baronnes possibles.



\frquote{Il est démontré, disait-il, que les choses ne peuvent être autrement;
car tout étant fait pour une fin, tout est nécessairement pour la
meilleure fin. Remarquez bien que les nez ont été faits pour porter des
lunettes; aussi avons-nous des lunettes. Les jambes sont visiblement
instituées pour être chaussées, et nous avons des chausses. Les pierres
ont été formées pour être taillées et pour en faire des châteaux; aussi
monseigneur a un très beau château: le plus grand baron de la province
doit être le mieux logé; et les cochons étant faits pour être mangés,
nous mangeons du porc toute l’année: par conséquent, ceux qui ont
avancé que tout est bien ont dit une sottise; il fallait dire que tout
est au mieux.}



\bname{Candide} écoutait attentivement, et croyait innocemment; car il trouvait
M\up{lle}~\bname{Cunégonde} extrêmement belle, quoiqu’il ne prît jamais la
hardiesse de le lui dire. Il concluait qu’après le bonheur d’être né
baron de Thunder-ten-tronckh, le second degré de bonheur était d’être
M\up{lle}~\bname{Cunégonde}; le troisième, de la voir tous les jours; et le
quatrième, d’entendre maître \bname{Pangloss}, le plus grand philosophe de la
province, et par conséquent de toute la terre.


Un jour \bname{Cunégonde}, en se promenant auprès du château, dans le petit
bois qu’on appelait \emph{parc}, vit entre des broussailes le docteur
\bname{Pangloss} qui donnait une leçon de physique expérimentale à la femme de
chambre de sa mère, petite brune très jolie et très docile. Comme
M\up{lle}~\bname{Cunégonde} avait beaucoup de dispositions pour les sciences,
elle observa, sans souffler, les expériences réitérées dont elle fut
témoin; elle vit clairement la raison suffisante du docteur, les effets
et les causes, et s’en retourna tout agitée, toute pensive, toute
remplie du désir d’être savante, songeant qu’elle pourrait bien être la
raison suffisante du jeune \bname{Candide},\linebreak qui pouvait aussi être la sienne.

Elle rencontra \bname{Candide} en revenant au château, et rougit: \bname{Candide}
rougit aussi. Elle lui dit bonjour d’une voix entrecoupée; et \bname{Candide}
lui parla sans savoir ce qu’il disait. Le lendemain, après le dîner,
comme on sortait de table, \bname{Cunégonde} et \bname{Candide} se trouvèrent derrière
un paravent; \bname{Cunégonde} laissa tomber son mouchoir, \bname{Candide} le ramassa;
elle lui prit innocemment la main; le jeune homme baisa innocemment la
main de la jeune demoiselle avec une vivacité, une sensibilité, une
grâce toute particulière; leurs bouches se rencontrèrent, leurs yeux
s’enflammèrent, leurs genoux tremblèrent, leurs mains s’égarèrent. 
M.~le baron de Thunder-ten-tronckh passa auprès du paravent, et voyant
cette cause et cet effet, chassa \bname{Candide} du château à grands coups de
pied dans le derrière. \bname{Cunégonde} s’évanouit: elle fut souffletée par
M\up{me}~la baronne dès qu’elle fut revenue à elle-même; et tout fut
consterné dans le plus beau et le plus agréable des châteaux possibles.




\chapter{Ce que devint \bname{Candide} parmi les Bulgares}


\lettrine{C}{andide}, chassé du paradis terrestre, marcha longtemps sans savoir où,
pleurant, levant les yeux au ciel, les tournant souvent vers le plus
beau des châteaux qui renfermait la plus belle des baronnettes; il se
coucha sans souper au milieu des champs entre deux sillons; la neige
tombait à gros flocons. \bname{Candide}, tout transi, se traîna le lendemain
vers la ville voisine, qui s’appelle Valdberghoff-trarbk-dikdorff,
n’ayant point d’argent, mourant de faim et de lassitude. Il s’arrêta
tristement à la porte d’un cabaret. Deux hommes habillés de bleu le
remarquèrent: «Camarade, dit l’un, voilà un jeune homme très bien fait,
et qui a la taille requise.» Ils s’avancèrent vers \bname{Candide} et le
prièrent à dîner très civilement.  «Messieurs, leur dit \bname{Candide} avec une
modestie charmante, vous me faites beaucoup d’honneur, mais je n’ai pas
de quoi payer mon écot. — Ah! monsieur, lui dit un des bleus, les
personnes de votre figure et de votre mérite ne paient jamais rien:
n’avez-vous pas cinq pieds cinq pouces de haut? — Oui, messieurs, c’est
ma taille, dit-il en faisant la révérence. — Ah! monsieur, mettez-vous à
table; non seulement nous vous défraierons, mais nous ne souffrirons
jamais qu’un homme comme vous manque d’argent; les hommes ne sont faits
que pour se secourir les uns les autres. — Vous avez raison, dit \bname{Candide};
c’est ce que M. \bname{Pangloss} m’a toujours dit, et je vois bien que tout est
au mieux.» On le prie d’accepter quelques écus, il les prend et veut
faire son billet; on n’en veut point, on se met à table. «N’aimez-vous
pas tendrement?… — Oh! oui, répond-il, j’aime tendrement M\up{lle}~\bname{Cunégonde}. — Non, dit l’un de ces messieurs, nous vous demandons si vous
n’aimez pas tendrement le roi des Bulgares? — Point du tout, dit-il, car
je ne l’ai jamais vu. — Comment! c’est le plus charmant des rois, et il
faut boire à sa santé. — Oh! très volontiers, messieurs.» Et il boit. «C’en
est assez, lui dit-on, vous voilà l’appui, le soutien, le défenseur, le
héros des Bulgares; votre fortune est faite, et votre gloire est
assurée.» On lui met sur-le-champ les fers aux pieds, et on le mène au
régiment. On le fait tourner à droite, à gauche, hausser la baguette,
remettre la baguette, coucher en joue, tirer, doubler le pas, et on lui
donne trente coups de bâton; le lendemain, il fait l’exercice un peu
moins mal, et il ne reçoit que vingt coups; le surlendemain, on ne lui
en donne que dix, et il est regardé par ses camarades comme un prodige.


\bname{Candide}, tout stupéfait, ne démêlait pas encore trop bien comment il
était un héros. Il s’avisa un beau jour de printemps de s’aller
promener, marchant tout droit devant lui, croyant que c’était un
privilège de l’espèce humaine, comme de l’espèce animale, de se servir
de ses jambes à son plaisir. Il n’eut pas fait deux lieues que voilà
quatre autres héros de six pieds qui l’atteignent, qui le lient, qui le
mènent dans un cachot. On lui demanda juridiquement ce qu’il aimait le
mieux d’être fustigé trente-six fois par tout le régiment, ou de
recevoir à-la-fois douze balles de plomb dans la cervelle. Il eut beau
dire que les volontés sont libres, et qu’il ne voulait ni l’un ni
l’autre, il fallut faire un choix; il se détermina, en vertu du don de
Dieu qu’on nomme \textit{liberté}, à passer trente-six fois par les baguettes;
il essuya deux promenades. Le régiment était composé de deux mille
hommes; cela lui composa quatre mille coups de baguette, qui, depuis la
nuque du cou jusqu’au cul, lui découvrirent les muscles et les nerfs.
Comme on allait procéder à la troisième course, \bname{Candide}, n’en pouvant
plus, demanda en grâce qu’on voulût bien avoir la bonté de lui casser
la tête; il obtint cette faveur; on lui bande les yeux; on le fait
mettre à genoux. Le roi des Bulgares passe dans ce moment, s’informe du
crime du patient; et comme ce roi avait un grand génie, il comprit, par
tout ce qu’il apprit de \bname{Candide}, que c’était un jeune métaphysicien
fort ignorant des choses de ce monde, et il lui accorda sa grâce avec
une clémence qui sera louée dans tous les journaux et dans tous les
siècles. Un brave chirurgien guérit \bname{Candide} en trois semaines avec les
émollients enseignés par \bname{Dioscoride}. Il avait déjà un peu de peau et
pouvait marcher, quand le roi des Bulgares livra bataille au roi des
Abares.



\chapter[Comment \bname{Candide} se sauva…]{Comment \bname{Candide} se sauva d’entre\\ les Bulgares, et ce qu’il devint}


\lettrine{R}{ien} n’était si beau, si leste, si brillant, si bien ordonné que les
deux armées. Les trompettes, les fifres, les hautbois, les tambours,
les canons, formaient une harmonie telle qu’il n’y en eut jamais en
enfer. Les canons renversèrent d’abord à peu près six mille hommes de
chaque côté; ensuite la mousqueterie ôta du meilleur des mondes environ
neuf à dix mille coquins qui en infectaient la surface. La baïonnette
fut aussi la raison suffisante de la mort de quelques milliers
d’hommes. Le tout pouvait bien se monter à une trentaine de mille âmes.
\bname{Candide}, qui tremblait comme un philosophe, se cacha du mieux qu’il put
pendant cette boucherie héroïque.

Enfin, tandis que les deux rois faisaient chanter des \textit{Te Deum}, chacun
dans son camp, il prit le parti d’aller raisonner ailleurs des effets
et des causes. Il passa par-dessus des tas de morts et de mourants, et
gagna d’abord un village voisin; il était en cendres: c’était un
village abare que les Bulgares avaient brûlé, selon les lois du droit
public. Ici des vieillards criblés de coups regardaient mourir leurs
femmes égorgées, qui tenaient leurs enfants à leurs mamelles
sanglantes; là des filles éventrées après avoir assouvi les besoins
naturels de quelques héros, rendaient les derniers soupirs; d’autres à
demi brûlées criaient qu’on achevât de leur donner la mort. Des
cervelles étaient répandues sur la terre à côté de bras et de jambes
\bname{coupés}.

\bname{Candide} s’enfuit au plus vite dans un autre village: il appartenait à
des Bulgares, et les héros abares l’avaient traité de même. \bname{Candide},
toujours marchant sur des membres palpitants ou à travers des ruines,
arriva enfin hors du théâtre de la guerre, portant quelques petites
provisions dans son bissac, et n’oubliant jamais M\up{lle}~\bname{Cunégonde}. Ses provisions lui manquèrent quand il fut en \bname{Hollande}; mais
ayant entendu dire que tout le monde était riche dans ce pays-là, et
qu’on y était chrétien, il ne douta pas qu’on ne le traitât aussi bien
qu’il l’avait été dans le château de M.~le baron, avant qu’il en eût
été chassé pour les beaux yeux de M\up{lle}~\bname{Cunégonde}.

Il demanda l’aumône à plusieurs graves personnages, qui lui répondirent
tous que, s’il continuait à faire ce métier, on l’enfermerait dans une
maison de correction pour lui apprendre à vivre.

Il s’adressa ensuite à un homme qui venait de parler tout seul une
heure de suite sur la charité dans une grande assemblée. Cet orateur le
regardant de travers lui dit: «Que venez-vous faire ici? y êtes-vous
pour la bonne cause? — Il n’y a point d’effet sans cause, répondit
modestement \bname{Candide}; tout est enchaîné nécessairement et arrangé pour
le mieux. Il a fallu que je fusse chassé d’auprès de M\up{lle}~\bname{Cunégonde}, que j’aie passé par les baguettes, et il faut que je demande
mon pain, jusqu’à ce que je puisse en gagner; tout cela ne pouvait être
autrement. — Mon ami, lui dit l’orateur, croyez-vous que le pape soit
l’antechrist? — Je ne l’avais pas encore entendu dire, répondit \bname{Candide}:
mais qu’il le soit, ou qu’il ne le soit pas, je manque de pain. — Tu ne
mérites pas d’en manger, dit l’autre: va, coquin, va, misérable, ne
m’approche de ta vie.» La femme de l’orateur ayant mis la tête à la
fenêtre, et avisant un homme qui doutait que le pape fût antechrist,
lui répandit sur le chef un plein… Ô ciel! à quel excès se porte le
zèle de la religion dans les dames!

Un homme qui n’avait point été baptisé, un bon anabaptiste, nommé
Jacques, vit la manière cruelle et 
ignominieuse dont on traitait ainsi
un de ses frères, un être à deux pieds sans plumes, qui avait une âme;
il l’amena chez lui, le nettoya, lui donna du pain et de la bière, lui
fit présent de deux florins, et voulut même lui apprendre à travailler
dans ses manufactures aux étoffes de Perse qu’on fabrique en \bname{Hollande}.
\bname{Candide} se prosternant presque devant lui, s’écriait: «Maître \bname{Pangloss}
me l’avait bien dit que tout est au mieux dans ce monde, car je suis
infiniment plus touché de votre extrême générosité que de la dureté de
ce monsieur à manteau noir, et de madame son épouse.»

Le lendemain, en se promenant, il rencontra un gueux tout couvert de
pustules, les yeux morts, le bout du nez rongé, la bouche de travers,
les dents noires, et parlant de la gorge, tourmenté d’une toux
violente, et crachant une dent à chaque effort.





\chapter[Comment \bname{Candide} rencontra…]{Comment \bname{Candide} rencontra son ancien maître de\\ philosophie, le docteur
\bname{Pangloss}, et ce qui en advint}


\lettrine{C}{andide}, plus ému encore de compassion que d’horreur, donna à cet
épouvantable gueux les deux florins qu’il avait reçus de son honnête 
anabaptiste Jacques. \textls[-10]{Le fantôme le regarda fixement, versa des larmes,
et sauta à son cou. \bname{Candide} effrayé recule. «Hélas! dit le misérable à
l’autre misérable, ne reconnaissez-vous plus votre cher \bname{Pangloss}?
— Qu’entends-je? vous, mon cher maître! vous, dans cet état horrible!
Quel malheur vous est-il donc arrivé? Pourquoi n’êtes-vous plus dans le
plus beau des châteaux? qu’est devenue M\up{lle}~\bname{Cunégonde}, la perle
des filles, le chef-d’œuvre de la nature? — Je n’en peux plus, dit
\bname{Pangloss}.» Aussitôt \bname{Candide} le mena dans l’étable de l’anabaptiste, où
il lui fit manger un peu de pain; et quand \bname{Pangloss} fut refait: «Eh
bien! lui dit-il, \bname{Cunégonde}? — Elle est morte, reprit l’autre.»} \bname{Candide}
s’évanouit à ce mot: son ami rappela ses sens avec un peu de mauvais
vinaigre qui se trouva par hasard dans l’étable. \bname{Candide} rouvre les
yeux. «\bname{Cunégonde} est morte! Ah! meilleur des mondes, où êtes-vous? Mais
de quelle maladie est-elle morte? ne serait-ce point de m’avoir vu
chasser du beau château de monsieur son père à grands coups de pied?
— Non, dit \bname{Pangloss}, elle a été éventrée par des soldats bulgares, après
avoir été violée autant qu’on peut l’être; ils ont cassé la tête à
M. le baron qui voulait la défendre; M\up{me}~la baronne a été
coupée en morceaux; mon pauvre pupille traité précisément comme sa
sœur; et quant au château, il n’est pas resté pierre sur pierre, pas
une grange, pas un mouton, pas un canard, pas un arbre; mais nous avons
été bien vengés, car les Abares en ont fait autant dans une baronnie
voisine qui appartenait à un seigneur bulgare.»

À ce discours, \bname{Candide} s’évanouit encore; mais revenu à soi, et ayant
dit tout ce qu’il devait dire, il s’enquit de la cause et de l’effet,
et de la raison suffisante qui avait mis \bname{Pangloss} dans un si piteux
état. «Hélas! dit l’autre, c’est l’amour: l’amour, le consolateur du
genre humain, le conservateur de l’univers, l’âme de tous les êtres
sensibles, le tendre amour. — Hélas! dit \bname{Candide}, je l’ai connu cet
amour, ce souverain des cœurs, cette âme de notre âme; il ne m’a
jamais valu qu’un baiser et vingt coups de pied au cul. Comment cette
belle cause a-t-elle pu produire en vous un effet si abominable?»

\bname{Pangloss} répondit en ces termes: «Ô mon cher \bname{Candide}! vous avez connu
\bname{Paquette}, cette jolie suivante de notre auguste baronne: j’ai goûté
dans ses bras les délices du paradis, qui ont produit ces tourments
d’enfer dont vous me voyez dévoré; elle en était infectée, elle en est
peut-être morte. \bname{Paquette} tenait ce présent d’un cordelier très savant
qui avait remonté à la source, car il l’avait eu d’une vieille
comtesse, qui l’avait reçu d’un capitaine de cavalerie, qui le devait à
une marquise, qui le tenait d’un page, qui l’avait reçu d’un jésuite,
qui, étant novice, l’avait eu en droite ligne d’un des compagnons de
Christophe Colomb. Pour moi, je ne le donnerai à personne, car je me
meurs.


— Ô \bname{Pangloss}! s’écria \bname{Candide}, voilà une étrange généalogie! n’est-ce pas
le diable qui en fut la souche? — Point du tout, répliqua ce grand homme;
c’était une chose indispensable dans le meilleur des mondes, un
ingrédient nécessaire; car si \bname{Colomb} n’avait pas attrapé dans une île
de \mbox{l’Amérique} cette maladie qui empoisonne la source de la
génération,
 qui souvent même empêche la génération, et qui est
évidemment l’opposé du grand but de la nature, nous n’aurions ni le
chocolat ni la cochenille; il faut encore observer que
jusqu’aujourd’hui, dans notre continent, cette maladie nous est
particulière, comme la controverse. Les \bname{Turcs}, les \bname{Indiens}, les
\bname{Persans}, les \bname{Chinois}, les \bname{Siamois}, les \bname{Japonais}, ne la connaissent pas
encore; mais il y a une raison suffisante pour qu’ils la connaissent à
leur tour dans quelques siècles. En attendant elle a fait un
merveilleux progrès parmi nous, et surtout dans ces grandes armées
composées d’honnêtes stipendiaires bien élevés, qui décident du destin
des états; on peut assurer que, quand trente mille hommes combattent en
bataille rangée contre des troupes égales en nombre, il y a environ
vingt mille vérolés de chaque côté.

— Voilà qui est admirable, dit \bname{Candide}; mais il faut vous faire guérir.
— Et comment le puis-je? dit \bname{Pangloss}; je n’ai pas le sou, mon ami, et
dans toute l’étendue de ce globe on ne peut ni se faire saigner, ni
prendre un lavement sans payer, ou sans qu’il y ait quelqu’un qui paie
pour nous.»

\looseness=-1
Ce dernier discours détermina \bname{Candide}; il alla se jeter aux pieds de
son charitable anabaptiste Jacques, et lui fit une peinture si
touchante de l’état où son ami était réduit, que le bonhomme n’hésita
pas à recueillir le docteur \bname{Pangloss}; il le fit guérir à ses dépens.
\bname{Pangloss}, dans la cure, ne perdit qu’un œil et une oreille. Il
écrivait bien, et savait parfaitement l’arithmétique. L’anabaptiste
Jacques en fit son teneur de livres. Au bout de deux mois, étant obligé
d’aller à \bname{Lisbonne} pour les affaires de son commerce, il mena dans son
vaisseau ses deux philosophes. \bname{Pangloss} lui expliqua comment tout était
on ne peut mieux. Jacques n’était pas de cet avis. «Il faut bien,
disait-il, que les hommes aient un peu corrompu la nature, car ils ne
sont point nés loups, et ils sont devenus loups. Dieu ne leur a donné
ni canons de vingt-quatre, ni baïonnettes, et ils se sont fait des
baïonnettes et des canons pour se détruire. Je pourrais mettre en ligne
de compte les banqueroutes, et la justice qui s’empare des biens des
banqueroutiers pour en frustrer les créanciers. — Tout cela était
indispensable, répliquait le docteur borgne, et les malheurs
particuliers font le bien général; de sorte que plus il y a de malheurs
particuliers, et plus tout est bien.» Tandis qu’il raisonnait, l’air
s’obscurcit, les vents soufflèrent des quatre coins du monde, et le
vaisseau fut assailli de la plus horrible tempête, à la vue du port de
\bname{Lisbonne}.





\chapter[Tempête, naufrage, tremblement de terre…]{Tempête, naufrage, tremblement de terre,\\et ce qui advint du docteur
\bname{Pangloss}, de \bname{Candide},\\et de l’anabaptiste Jacques}


\lettrine[findent=-3pt,nindent=0.6em]{L}{a moitié} 
des passagers affaiblis, expirants de ces angoisses 
inconcevables que le roulis d’un vaisseau porte dans les nerfs et dans
toutes les humeurs du corps agitées en sens contraires, n’avait pas
même la force de s’inquiéter du danger. L’autre moitié jetait des cris
et faisait des prières; les voiles étaient déchirées, les mâts brisés,
le vaisseau entr’ouvert. Travaillait qui pouvait, personne ne
s’entendait, personne ne commandait. L’anabaptiste aidait un peu à la
manœuvre; il était sur le tillac; un matelot furieux le frappe
rudement et l’étend sur les planches; mais du coup qu’il lui donna, il
eut lui-même une si violente secousse, qu’il tomba hors du vaisseau, la
tête la première. Il restait suspendu et accroché à une partie de mât
rompu. Le bon Jacques court à son secours, l’aide à remonter, et de
l’effort qu’il fait, il est précipité dans la mer à la vue du matelot,
qui le laissa périr sans daigner seulement le regarder. \bname{Candide}
approche, voit son bienfaiteur qui reparaît un moment, et qui est
englouti pour jamais. Il veut se jeter après lui dans la mer: le
philosophe \bname{Pangloss} l’en empêche, en lui prouvant que la rade de
\bname{\bname{Lisbonne}} avait été formée exprès pour que cet anabaptiste s’y noyât.
Tandis qu’il le prouvait \textit{a priori}, le vaisseau s’entr’ouvre, tout
périt à la réserve de \bname{Pangloss}, de \bname{Candide}, et de ce brutal de 
matelot
qui avait noyé le vertueux anabaptiste; le coquin nagea heureusement
jusqu’au rivage, où \bname{Pangloss} et \bname{Candide} furent portés sur une planche.

Quand ils furent revenus un peu à eux, ils marchèrent vers \bname{Lisbonne}; il
leur restait quelque argent, avec lequel ils espéraient se sauver de la
faim après avoir échappé à la \bname{tempête}.


À peine ont-ils mis le pied dans la ville, en pleurant la mort de leur
bienfaiteur, qu’ils sentent la terre trembler sous leurs pas; la mer
s’élève en bouillonnant dans le port, et brise les vaisseaux qui sont à
l’ancre. Des tourbillons de flammes et de cendres couvrent les rues et
les places publiques; les maisons s’écroulent, les toits sont renversés
sur les fondements, et les fondements se dispersent; trente mille
habitants de tout âge et de tout sexe sont écrasés sous des ruines. Le
matelot disait en sifflant et en jurant: «Il y aura quelque chose à
gagner ici. — Quelle peut être la raison suffisante de ce phénomène?
disait \bname{Pangloss}. — Voici le dernier jour du monde!» s’écriait \bname{Candide}. Le matelot court incontinent au milieu des débris, affronte la mort pour
trouver de l’argent, en trouve, s’en empare, s’enivre, et ayant cuvé
son vin, achète les faveurs de la première fille de bonne volonté qu’il
rencontre sur les ruines des maisons détruites, et au milieu des
mourants et des morts. \bname{Pangloss} le tirait cependant par la manche: «Mon
ami, lui disait-il, cela n’est pas bien, vous manquez à la raison
universelle, vous prenez mal votre temps. — Tête et sang, répondit
l’autre, je suis matelot et né à Batavia; j’ai marché quatre fois sur
le crucifix dans quatre voyages au Japon; tu as bien trouvé ton
homme avec ta \linebreak raison universelle!»


Quelques éclats de pierre avaient blessé \bname{Candide}; il était étendu dans
la rue et couvert de débris. Il disait à \bname{Pangloss}: «Hélas! procure-moi
un peu de vin et d’huile; je me meurs. — Ce tremblement de terre n’est
pas une chose nouvelle, répondit \bname{Pangloss}; la ville de Lima éprouva les
mêmes secousses en Amérique l’année passée; mêmes causes, mêmes effets;
il y a certainement une traînée de soufre sous terre depuis Lima
jusqu’à \bname{Lisbonne}. — Rien n’est plus probable, dit \bname{Candide}; mais, pour
Dieu, un peu d’huile et de vin. — Comment probable? répliqua le
philosophe, je soutiens que la chose est démontrée.» \bname{Candide} perdit
connaissance, et \bname{Pangloss} lui apporta un peu d’eau d’une fontaine
voisine.

\textls[-10]{Le lendemain, ayant trouvé quelques provisions de bouche en se glissant
à travers des décombres,} ils réparèrent un peu leurs forces. Ensuite
ils travaillèrent comme les autres à soulager les habitants échappés à
la mort. Quelques citoyens, secourus par eux, leur donnèrent un aussi
bon dîner qu’on le pouvait dans un tel désastre: il est vrai que le
repas était triste; les convives arrosaient leur pain de leurs larmes;
mais \bname{Pangloss} les consola, en les assurant que les choses ne pouvaient
être autrement: «Car, dit-il, tout ceci est ce qu’il y a de mieux; car
s’il y a un volcan à \bname{Lisbonne}, il ne pouvait être ailleurs; car il est
impossible que les choses ne soient pas où elles sont, car tout est
bien.»


Un petit homme noir, familier de l’inquisition, lequel était à côté de
lui, prit poliment la parole et dit: «Apparemment que monsieur ne croit
pas au péché originel; car si tout est au mieux, il n’y a donc eu ni
chute ni punition. 

— Je demande très humblement pardon à votre excellence, répondit \bname{Pangloss}
encore plus poliment, car la chute de l’homme et la malédiction
entraient nécessairement dans le meilleur des mondes possibles.
— \bname{Monsieur} ne croit donc pas à la liberté? dit le familier. — Votre
excellence m’excusera, dit \bname{Pangloss}; la liberté peut subsister avec la
nécessité absolue; car il était nécessaire que nous fussions libres;
car enfin la volonté déterminée…» \bname{Pangloss} était au milieu de sa
phrase, quand le familier fit un signe de tête à son estafier qui lui
servait à boire du vin de Porto ou d’Oporto.





\chapter[Comment on fit un bel auto-da-fé…]{Comment on fit un bel auto-da-fé pour empêcher les\\tremblements de
terre, et comment \bname{Candide} fut fessé}


\lettrine{A}{près} le tremblement de terre qui avait détruit les trois quarts de
\bname{Lisbonne}, les sages du pays n’avaient pas trouvé un moyen plus efficace
pour prévenir une ruine totale que de donner au peuple un bel
auto-da-fé; il était décidé par l’université de Coïmbre que le
spectacle de quelques personnes brûlées à petit feu, en grande
cérémonie, est un secret infaillible pour empêcher la terre de
trembler.

On avait en conséquence saisi un Biscayen convaincu d’avoir épousé sa
commère, et deux Portugais qui en mangeant un poulet en avaient arraché
le lard: on vint lier après le dîner le docteur \bname{Pangloss} et son
disciple \bname{Candide}, l’un pour avoir parlé, et l’autre pour l’avoir écouté
avec un air d’approbation: tous deux furent menés séparément dans des
appartements d’une extrême fraîcheur, dans lesquels on n’était jamais
incommodé du soleil: huit jours après ils furent tous deux 
revêtus d’un
\emph{san-benito}, et on orna leurs têtes de mitres de papier: la mitre et le
san-benito de \bname{Candide} étaient peints de flammes renversées, et de
diables qui n’avaient ni queues ni griffes; mais les diables de
\bname{Pangloss} portaient griffes et queues, et les flammes étaient droites.
Ils marchèrent en procession ainsi vêtus, et entendirent un sermon très
pathétique, suivi d’une belle musique en faux-bourdon. \bname{Candide} fut
fessé en cadence, pendant qu’on chantait; le Biscayen et les deux
hommes qui n’avaient point voulu manger de lard furent 
brûlés, et
\bname{Pangloss} fut pendu, quoique ce ne soit pas la coutume. Le même jour la
terre trembla de nouveau avec un fracas épouvantable.

\bname{Candide} épouvanté, interdit, éperdu, tout sanglant, tout palpitant, se
disait à lui-même: «Si c’est ici le meilleur des mondes possibles, que
sont donc les autres? passe encore si je n’étais que fessé, je l’ai été
chez les Bulgares; mais, ô mon cher \bname{Pangloss}! le plus grand des
philosophes, faut-il vous avoir vu pendre, sans que je sache pourquoi!
ô mon cher anabaptiste! le meilleur des hommes, faut-il que vous ayez
été noyé dans le port! Ô mademoiselle~\bname{Cunégonde}! la perle des filles,
faut-il qu’on vous ait fendu le ventre!»

Il s’en retournait, se soutenant à peine, prêché, fessé, absous, et
béni, lorsqu’une vieille l’aborda, et lui dit: «Mon fils, prenez
courage, suivez-moi.»






\chapter[Comment une vieille prit soin de \bname{Candide}…]{Comment une vieille prit soin de \bname{Candide},\\et comment il retrouva ce
qu’il aimait}

\lettrine{C}{andide} ne prit point courage, mais il suivit la vieille dans une
masure: elle lui donna un pot de pommade pour se frotter, lui laissa à
manger et à boire; elle lui montra un petit lit assez propre; il y
avait auprès du lit un habit complet. «Mangez, buvez, dormez, lui
dit-elle, et que Notre-Dame d’Atocha, M\up{gr} saint Antoine de
Padoue, et M\up{gr} saint Jacques de Compostelle prennent soin de
vous! je reviendrai demain.» \bname{Candide}, toujours étonné de tout ce qu’il
avait vu, de tout ce qu’il avait souffert, et encore plus de la charité
de la vieille, voulut lui baiser la main. «Ce n’est pas ma main qu’il
faut baiser, dit la vieille; je reviendrai demain. Frottez-vous de
pommade, mangez et dormez.»



\bname{Candide}, malgré tant de malheurs, mangea et dormit. Le lendemain la
vieille lui apporte à déjeuner, visite son dos, le frotte elle-même
d’une autre pommade: elle lui apporte ensuite à dîner: elle revient sur
le soir et apporte à souper. Le surlendemain elle fit encore les mêmes
cérémonies. «Qui êtes-vous? lui disait toujours \bname{Candide}; qui vous a
inspiré tant de bonté? quelles grâces puis-je vous rendre?» La bonne
femme ne répondait jamais rien. Elle revint sur le soir, et n’apporta
point à souper: «Venez avec moi, dit-elle, et ne dites mot.» Elle le
prend sous le bras, et marche avec lui dans la campagne environ un
quart de mille: ils arrivent à une maison isolée, entourée de jardins
et de canaux. La vieille frappe à une petite porte. On ouvre; elle mène
\bname{Candide}, par un escalier dérobé, dans un cabinet doré, le laisse sur un
canapé de brocart, referme la porte, et s’en va. \bname{Candide} croyait rêver,
et regardait toute sa vie comme un songe funeste, et le moment présent
comme un songe agréable.

\textls[-5]
{La vieille reparut bientôt; elle soutenait avec peine une femme
tremblante, d’une taille majestueuse, brillante de pierreries, et
couverte d’un voile. «Ôtez ce voile», dit la vieille à \bname{Candide}. Le jeune
homme approche; il lève le voile d’une main timide. Quel moment! quelle
surprise! il croit voir M\up{lle}~\bname{Cunégonde}; il la voyait en effet,
c’était elle-même. La force lui manque, il ne peut proférer une parole,
il tombe à ses pieds. \bname{Cunégonde} tombe sur le canapé. La vieille les
accable d’eaux spiritueuses, ils reprennent leurs sens, ils se parlent:
ce sont d’abord des mots entrecoupés, des demandes et des réponses qui
se croisent, des soupirs, des larmes, des cris. La vieille leur
recommande de faire moins de bruit, et les laisse en liberté. «Quoi!
c’est vous, lui dit \bname{Candide}, vous vivez! Je vous retrouve en Portugal!
On ne vous a donc pas violée? on ne vous a point fendu le ventre, comme
le philosophe \bname{Pangloss} me l’avait assuré?» — Si fait, dit la belle
\bname{Cunégonde}; mais on ne meurt pas toujours de ces deux accidents. — Mais
votre père et votre mère ont-ils été tués? — Il n’est que trop vrai, dit
\bname{Cunégonde} en pleurant. — Et votre frère? — Mon frère a été tué aussi. — Et
pourquoi êtes-vous en Portugal? et comment avez-vous su que j’y étais?
et par quelle étrange aventure m’avez-vous fait conduire dans cette
maison? — Je vous dirai tout cela, répliqua la dame; mais il faut
auparavant que vous m’appreniez tout ce qui vous est arrivé depuis le
baiser innocent que vous me donnâtes, et les coups de pied que vous
reçûtes.»}



\bname{Candide} lui obéit avec un profond respect; et quoiqu’il fût interdit,
quoique sa voix fût faible et tremblante, quoique l’échine lui fît
encore un peu mal, il lui raconta de la manière la plus naïve tout ce
qu’il avait éprouvé depuis le moment de leur séparation. \bname{Cunégonde}
levait les yeux au ciel: elle donna des larmes à la mort du bon
anabaptiste et de \bname{\bname{Pangloss}}; après quoi elle parla en ces termes à
\bname{Candide}, qui ne perdait pas une parole, et qui la dévorait des yeux.






\chapter{Histoire de Cunégonde}


\lettrine[ante={«}]{J'}{étais} dans mon lit et je dormais profondément, quand il plut au ciel
d’envoyer les Bulgares dans notre beau château de Thunder-ten-tronckh;
ils égorgèrent mon père et mon frère, et coupèrent ma mère par
morceaux. Un grand Bulgare, haut de six pieds, voyant qu’à ce spectacle
j’avais perdu connaissance, se mit à me violer; cela me fit revenir, je
repris mes sens, je criai, je me débattis, je mordis, j’égratignai, je
voulais arracher les yeux à ce grand Bulgare, ne sachant pas que tout
ce qui arrivait dans le château de mon père était une chose d’usage: le
brutal me donna un coup de couteau dans le flanc gauche dont je porte
encore la marque. — Hélas! j’espère bien la voir, dit le naïf \bname{Candide}.
— Vous la verrez, dit \bname{Cunégonde}; mais continuons. — Continuez», dit \bname{Candide}.

{\parfillskip=1em
Elle reprit ainsi le fil de son histoire: «Un capitaine bulgare entra,
il me vit toute sanglante, et le soldat ne se dérangeait pas. Le
capitaine se mit en colère du peu de respect que lui témoignait, ce
brutal, et le tua sur mon corps. Ensuite il me fit panser, et m’emmena
prisonnière de guerre dans son quartier. Je blanchissais le peu de
chemises qu’il avait, je faisais sa cuisine; il me trouvait fort jolie,
il faut l’avouer; et je ne nierai pas qu’il ne fût très bien fait, et
qu’il n’eût la peau blanche et douce; d’ailleurs peu d’esprit, peu de
philosophie: on voyait bien qu’il n’avait pas été élevé par le docteur
\bname{Pangloss}. Au bout de trois mois, ayant perdu tout son argent, et
s’étant dégoûté de moi, il me vendit à un Juif nommé don Issachar, qui
trafiquait en \bname{Hollande} et en Portugal, et qui aimait passionnément les
femmes. Ce Juif s’attacha beaucoup à ma personne, mais il ne pouvait en
triompher; je lui ai mieux résisté qu’au soldat bulgare. Une personne
d’honneur peut être violée une fois, mais sa vertu s’en affermit. Le
Juif, pour m’apprivoiser, me mena dans cette maison de campagne que
vous voyez. J’avais cru jusque-là qu’il n’y avait rien sur la terre de
si beau que le château de Thunder-ten-tronckh; j’ai été détrompée.

}

«Le grand inquisiteur m’aperçut un jour à la messe; il me lorgna
beaucoup, et me fit dire qu’il avait à me parler pour des affaires
secrètes. Je fus conduite à son palais; je lui appris ma naissance; il
me représenta combien il était au-dessous de mon rang d’appartenir à un
Israélite. On proposa de sa part à don Issachar de me céder à
monseigneur. Don Issachar, qui est le banquier de la cour, et homme de
crédit, n’en voulut rien faire. L’inquisiteur le menaça d’un
auto-da-fé. Enfin mon Juif intimidé conclut un marché par lequel la
maison et moi leur appartiendraient à tous deux en commun; que le Juif
aurait pour lui les lundis, mercredis, et le jour du sabbat, et que
l’inquisiteur aurait les autres jours de la semaine. Il y a six mois
que cette convention subsiste. Ce n’a pas été sans querelles; car
souvent il a été indécis si la nuit du samedi au dimanche appartenait à
l’ancienne loi ou à la nouvelle. Pour moi, j’ai résisté jusqu’à présent
à toutes les deux; et je crois que c’est pour cette raison que j’ai
toujours été aimée.

«Enfin, pour détourner le fléau des tremblements de terre, et pour
intimider don Issachar, il plut à \bname{monseigneur} l’inquisiteur de célébrer
un auto-da-fé. Il me fit l’honneur de m’y inviter. Je fus très bien
placée; on servit aux dames des rafraîchissements entre la messe et
l’exécution. Je fus, à la vérité, saisie d’horreur en voyant brûler ces
deux Juifs et cet honnête Biscayen qui avait épousé sa commère: mais
quelle fut ma surprise, mon effroi, mon trouble, quand je vis dans un
san-benito,
 et sous une mitre, une figure qui ressemblait à celle de
\bname{Pangloss}! Je me 
frottai les yeux, je regardai attentivement, je le vis
pendre; je tombai en faiblesse. À peine reprenais-je mes sens, que je
vous vis dépouillé tout nu; ce fut là le comble de l’horreur, de la
consternation, de la douleur, du désespoir. Je vous dirai, avec vérité,
que votre peau est encore plus blanche, et d’un incarnat plus parfait
que celle de mon capitaine des Bulgares. Cette vue redoubla tous les
sentiments qui m’accablaient, qui me dévoraient. Je m’écriai, je voulus
dire, «Arrêtez, barbares!» mais la voix me manqua, et mes cris auraient
été inutiles. Quand vous eûtes été bien fessé: \frquote{\localleftbox{« }Comment se peut-il
faire, disais-je, que l’aimable \bname{Candide} et le sage \bname{Pangloss} se trouvent
à \bname{Lisbonne}, l’un pour recevoir cent coups de fouet, et l’autre pour
être pendu par l’ordre de \bname{Monseigneur} l’inquisiteur, dont je suis la
bien-aimée? \bname{Pangloss} m’a donc bien cruellement trompée, quand il me
disait que tout va le mieux du monde!}\localleftbox{}

«Agitée, éperdue, tantôt hors de moi-même, et tantôt prête de mourir de
faiblesse, j’avais la tête remplie du massacre de mon père, de ma mère,
de mon frère, de l’insolence de mon vilain soldat bulgare, du coup de
couteau qu’il me donna, de ma servitude, de mon métier de cuisinière,
de mon capitaine bulgare, de mon vilain don~\bname{Issachar}, de mon abominable
inquisiteur, de la pendaison du docteur \bname{Pangloss}, de ce grand \emph{Miserere}
en faux-bourdon pendant lequel on vous fessait, et surtout du baiser
que je vous avais donné derrière un paravent, le jour que je vous avais
vu pour la dernière fois. Je louai Dieu, qui vous ramenait à moi par
tant d’épreuves. Je recommandai à ma vieille d’avoir soin de vous, et
de vous amener ici dès qu’elle le pourrait. Elle a très bien exécuté ma
commission; j’ai goûté le plaisir inexprimable de vous revoir, de vous
entendre, de vous parler. Vous devez avoir une faim dévorante; j’ai
grand appétit; commençons par souper.»

Les voilà qui se mettent tous deux à table; et, après le souper, ils se
replacent sur ce beau canapé dont on a déjà parlé; ils y étaient quand
le signor don Issachar, l’un des maîtres de la maison, arriva. C’était
le jour du sabbat. Il venait jouir de ses droits, et expliquer son
tendre amour.





\chapter[Ce qui advint de Cunégonde, de Candide…]{Ce qui advint de \bname{Cunégonde}, de \bname{Candide},\\du grand inquisiteur, et d’un
Juif}


\lettrine{C}{et} Issachar était le plus colérique Hébreu qu’on eût vu dans Israël,
depuis la captivité en Babylone. «Quoi! dit-il, chienne de galiléenne,
ce n’est pas assez de M. l’inquisiteur? il faut que ce coquin
partage aussi avec moi?» En disant cela il tire un long poignard dont il
était toujours pourvu, et, ne croyant pas que son adverse partie eût
des armes, il se jette sur \bname{Candide}; mais notre bon Westphalien avait
reçu une belle épée de la vieille avec l’habit complet. Il tire son
épée, quoiqu’il eût les mœurs fort douces, et vous étend l’Israélite
roide mort sur le carreau, aux pieds de la belle \bname{Cunégonde}.

«Sainte Vierge! s’écria-t-elle, qu’allons-nous devenir? un homme tué
chez moi! si la justice vient, nous sommes perdus. — Si \bname{Pangloss} n’avait
pas été pendu, dit \bname{Candide}, il nous donnerait un bon conseil dans cette
extrémité, car c’était un grand philosophe. À son défaut, consultons la
vieille.» Elle était fort prudente, et commençait à dire son avis quand
une autre petite porte s’ouvrit. Il était une heure après minuit,
c’était le commencement du dimanche. Ce jour appartenait à monseigneur
l’inquisiteur. Il entre et voit le fessé \bname{Candide}, l’épée à la main, un
mort étendu par terre, \bname{Cunégonde} effarée, et la vieille donnant des
conseils.

Voici dans ce moment ce qui se passa dans l’âme de \bname{Candide}, et comment
il raisonna: «Si ce saint homme appelle du secours, il me fera
infailliblement brûler, il pourra en faire autant de \bname{Cunégonde}; il m’a
fait fouetter impitoyablement; il est mon rival; je suis en train de
tuer; il n’y a pas à balancer.» Ce raisonnement fut net et rapide; et,
sans donner le temps à l’inquisiteur de revenir de sa surprise, il le
perce d’outre en outre, et le jette à côté du Juif. «En voici bien d’une
autre, dit \bname{Cunégonde}; il n’y a plus de rémission; nous sommes
excommuniés, notre dernière heure est venue! Comment avez-vous fait,
vous qui êtes né si doux, pour tuer en deux minutes un Juif et un
prélat? — Ma belle demoiselle, répondit\bname{Candide}, quand on est amoureux,
jaloux, et fouetté par l’inquisition, on ne se connaît plus.»

La vieille prit alors la parole, et dit: «Il y a trois chevaux andalous
dans l’écurie, avec leurs selles et leurs brides, que le brave \bname{Candide}
les prépare; madame a des moyadors et des diamants, montons vite à
cheval, quoique je ne puisse me tenir que sur une fesse, et allons à
Cadix; il fait le plus beau temps du monde, et c’est un grand plaisir
de voyager pendant la fraîcheur de la nuit.»

Aussitôt \bname{Candide} selle les trois chevaux; \bname{Cunégonde}, la vieille, et
lui, font trente milles d’une traite. Pendant qu’ils s’éloignaient, la
Sainte-Hermandad arrive dans la maison, on enterre monseigneur dans une
belle église, on jette Issachar à la voirie.

\bname{Candide}, \bname{Cunégonde}, et la vieille, étaient déjà dans la petite ville
\bname{d’Avacéna}, au milieu des montagnes de la Sierra Morena; et ils
parlaient ainsi dans un cabaret.




\chapter[Dans quelle détresse Candide…]{Dans quelle détresse \bname{Candide}, \bname{Cunégonde},\\et la vieille, arrivent à
Cadix, et leur embarquement}


\lettrine[ante={«}]{Q}{ui a donc} pu me voler mes pistoles et mes diamants? disait en pleurant
\bname{Cunégonde}; de quoi vivrons-nous? comment ferons-nous? où trouver des
inquisiteurs et des Juifs qui m’en donnent d’autres? — Hélas! dit la
vieille, je soupçonne fort un révérend père cordelier, qui coucha hier
dans la même auberge que nous à \bname{Badajos}; Dieu me garde de faire un
jugement téméraire! mais il entra deux fois dans notre chambre, et il
partit longtemps avant nous. — Hélas! dit \bname{Candide}, le bon \bname{Pangloss}
m’avait souvent prouvé que les biens de la terre sont communs à tous
les hommes, que chacun y a un droit égal. Ce cordelier devait bien,
suivant ces principes, nous laisser de quoi achever notre voyage. Il ne
vous reste donc rien du tout, ma belle \bname{Cunégonde}? — Pas un maravédis,
dit-elle. — Quel parti prendre? dit \bname{Candide}. — Vendons un des chevaux, dit
la vieille; je monterai en croupe derrière mademoiselle, quoique je ne
puisse me tenir que sur une fesse, et nous arriverons à Cadix.»

Il y avait dans la même hôtellerie un prieur de bénédictins; il acheta
le cheval bon marché. \bname{Candide}, \bname{Cunégonde}, et la vieille, passèrent par
Lucena, par Chillas, par Lebrixa, et arrivèrent enfin à Cadix. On y
équipait une flotte, et on y assemblait des troupes pour mettre à la
raison les révérends pères jésuites du Paraguay, qu’on accusait d’avoir
fait révolter une de leurs hordes contre les rois d’Espagne et de
Portugal, auprès de la ville du Saint-Sacrement. \bname{Candide}, ayant
servi chez les Bulgares, fit l’exercice bulgarien devant le général de
la petite armée avec tant de grâce, de célérité, d’adresse, de fierté,
d’agilité, qu’on lui donna une compagnie d’infanterie à commander. Le
voilà capitaine; il s’embarque avec M\up{lle}~\bname{Cunégonde}, la vieille,
deux valets, et les deux chevaux andalous qui avaient appartenu à M. le
grand inquisiteur de Portugal.

\looseness=-1
Pendant toute la traversée ils raisonnèrent beaucoup sur la philosophie
du pauvre \bname{Pangloss}. «Nous allons dans un autre univers, disait \bname{Candide};
c’est dans celui-là, sans doute, que tout est bien: car il faut avouer
qu’on pourrait gémir un peu de ce qui se passe dans le nôtre en
physique et en morale. — Je vous aime de tout mon cœur, disait
\bname{Cunégonde}; mais j’ai encore l’âme tout effarouchée de ce que j’ai vu,
de ce que j’ai éprouvé. — Tout ira bien, répliquait \bname{Candide}; la mer de ce
nouveau monde vaut déjà mieux que les mers de notre Europe; elle est
plus calme, les vents plus constants. C’est certainement le
nouveau monde qui est le meilleur des univers possibles. — Dieu le
veuille! disait \bname{Cunégonde}: mais j’ai été si horriblement malheureuse
dans le mien, que mon cœur est presque fermé à l’espérance. — Vous vous
plaignez, leur dit la vieille; hélas! vous n’avez pas éprouvé des
infortunes telles que les miennes.» \bname{Cunégonde} se mit presque à rire, et
trouva cette bonne femme fort plaisante de prétendre être plus
malheureuse qu’elle. «Hélas! lui dit-elle, ma bonne, à moins que vous
n’ayez été violée par deux Bulgares, que vous n’ayez reçu deux coups de
couteau dans le ventre, qu’on n’ait démoli deux de vos châteaux, qu’on
n’ait égorgé à vos yeux deux mères et deux pères, et que vous n’ayez vu
deux de vos amants fouettés dans un auto-da-fé, je ne vois pas que vous
puissiez l’emporter sur moi; ajoutez que je suis née baronne avec
soixante et douze quartiers, et que j’ai été cuisinière. — Mademoiselle,
répondit la vieille, vous ne savez pas quelle est ma naissance; et si
je vous montrais mon derrière, vous ne parleriez pas comme vous faites,
et vous suspendriez votre jugement.» Ce discours fit naître une extrême
curiosité dans l’esprit de \bname{Cunégonde} et de \bname{Candide}. La vieille leur
parla en ces termes.\quad\null


\chapter{Histoire de la vieille}


\lettrine[ante=«]{J}{e n’ai} pas eu toujours les yeux éraillés et bordés d’écarlate; mon nez
n’a pas toujours touché à mon menton, et je n’ai pas toujours été
servante. Je suis la fille du pape \bname{Urbain}~X et de la princesse de
\bname{Palestrine}. 
On m’éleva jusqu’à quatorze ans dans un palais auquel
tous les châteaux de vos barons allemands n’auraient pas servi
d’écurie; et une de mes robes valait mieux que toutes les magnificences
de la \bname{Westphalie}. Je croissais en beauté, en grâces, en talents, au
milieu des plaisirs, des respects, et des espérances: j’inspirais déjà
de l’amour; ma gorge se formait; et quelle gorge! blanche, ferme,
taillée comme celle de la Vénus de \bname{Médicis}; et quels yeux! quelles
paupières! quels sourcils noirs! quelles flammes brillaient dans mes
deux prunelles, et effaçaient la scintillation des étoiles! comme me
disaient les poètes du quartier. Les femmes qui m’habillaient et qui me
déshabillaient tombaient en extase en me regardant par-devant et
par-derrière; et tous les hommes auraient voulu être à leur place.


«Je fus fiancée à un prince souverain de Massa-Carrara: quel prince!
aussi beau que moi, pétri de douceur et d’agréments, brillant d’esprit
et brûlant d’amour; je l’aimais comme on aime pour la première fois,
avec idolâtrie, avec emportement. Les noces furent préparées: c’était
une pompe, une magnificence inouïe; c’étaient des fêtes, des
carrousels, des opéra-buffa continuels; et toute l’Italie fit pour moi
des sonnets dont il n’y eut pas un seul de passable. Je touchais au 
moment 
de mon bonheur, quand une vieille marquise, qui avait été
maîtresse de mon prince, l’invita à prendre du chocolat chez elle; il
mourut en moins de deux heures avec des convulsions épouvantables; mais
ce n’est qu’une bagatelle. Ma mère au désespoir, et bien moins affligée
que moi, voulut s’arracher pour quelque temps à un séjour si funeste.
Elle avait une très belle terre auprès de Gaète: nous nous embarquâmes
sur une galère du pays, dorée comme l’autel de Saint-Pierre de Rome.
Voilà qu’un corsaire de Salé fond sur nous et nous aborde: nos soldats
se défendirent comme des soldats du pape; ils se mirent tous à genoux
en jetant leurs armes, et en demandant au corsaire une absolution \emph{in
articulo mortis}.

«Aussitôt on les dépouilla nus comme des singes, et ma mère aussi, nos
filles d’honneur aussi, et moi aussi. C’est une chose admirable que la
diligence avec laquelle ces messieurs déshabillent le monde; mais ce
qui me surprit davantage, c’est qu’ils nous mirent à tous le doigt dans
un endroit où nous autres femmes nous ne nous laissons mettre
d’ordinaire que des canules. Cette cérémonie me paraissait bien
étrange: voilà comme on juge de tout quand on n’est pas sorti de son
pays. J’appris bientôt que c’était pour voir si nous n’avions pas caché
là quelques diamants; c’est un usage établi de temps immémorial parmi
les nations policées qui courent sur mer. J’ai su que messieurs les
religieux chevaliers de Malte n’y manquent jamais quand ils prennent
des Turcs et des Turques; c’est une loi du droit des gens à laquelle on
n’a jamais dérogé.

«Je ne vous dirai point combien il est dur pour une jeune princesse
d’être menée esclave à Maroc avec sa mère: vous concevez assez tout ce
que nous eûmes à souffrir dans le vaisseau corsaire. Ma mère était
encore très belle: nos filles d’honneur, nos simples femmes de chambre
avaient plus de charmes qu’on n’en peut trouver dans toute l’Afrique.
Pour moi, j’étais ravissante, j’étais la beauté, la grâce même, et
j’étais pucelle: je ne le fus pas longtemps; cette fleur, qui avait
été réservée pour le beau prince de Massa-Carrara, me fut ravie par le
capitaine corsaire; c’était un nègre abominable, qui croyait encore me
faire beaucoup d’honneur. Certes il fallait que M\up{me} la princesse de
\bname{Palestrine} et moi fussions bien fortes pour résister à tout ce que nous
éprouvâmes jusqu’à notre arrivée à Maroc! Mais passons; ce sont des
choses si communes, qu’elles ne valent pas la peine qu’on en parle.

«Maroc nageait dans le sang quand nous arrivâmes. Cinquante fils de
l’empereur Muley-Ismaël avaient chacun leur parti; ce qui produisait
en effet cinquante guerres civiles, de noirs contre noirs, de noirs
contre basanés, de basanés contre basanés, de mulâtres contre mulâtres:
c’était un carnage continuel dans toute l’étendue de l’empire.



«À peine fûmes-nous débarquées, que des noirs d’une faction ennemie de
celle de mon corsaire se présentèrent pour lui enlever son butin. Nous
étions, après les diamants et l’or, ce qu’il avait de plus précieux. Je
fus témoin d’un combat tel que vous n’en voyez jamais dans vos climats
d’Europe. Les peuples septentrionaux n’ont pas le sang assez ardent;
ils n’ont pas la rage des femmes au point où elle est commune en
Afrique. Il semble que vos Européens aient du lait dans les veines;
c’est du vitriol, c’est du feu qui coule dans celles des habitants du
mont Atlas et des pays voisins. On combattit avec la fureur des lions,
des tigres, et des serpents de la contrée, pour savoir qui nous aurait.
Un Maure saisit ma mère par le bras droit, le lieutenant de mon
capitaine la retint par le bras gauche; un soldat maure la prit par une
jambe, un de nos pirates la tenait par l’autre. Nos filles se
trouvèrent presque toutes en un moment tirées ainsi à quatre soldats.
Mon capitaine me tenait cachée derrière lui; il avait le cimeterre au
poing, et tuait tout ce qui s’opposait à sa rage. Enfin je vis toutes
nos Italiennes et ma mère déchirées, coupées, massacrées par les
monstres qui se les disputaient. Les captifs, mes compagnons, ceux qui
les avaient pris, soldats, matelots, noirs, basanés, blancs, mulâtres,
et enfin mon capitaine, tout fut tué, et je demeurai mourante sur un
tas de morts. Des scènes pareilles se passaient, comme on sait, dans
l’étendue de plus de trois cents lieues, sans qu’on manquât aux cinq
prières par jour ordonnées par Mahomet.

«Je me débarrassai avec beaucoup de peine de la foule de tant de
cadavres sanglants entassés, et je me traînai sous un grand oranger au
bord d’un ruisseau voisin; j’y tombai d’effroi, de lassitude,
d’horreur, de désespoir, et de faim. Bientôt après, mes sens accablés se
livrèrent à un sommeil qui tenait plus de l’évanouissement que du
repos. J’étais dans cet état de faiblesse et d’insensibilité, entre la
mort et la vie, quand je me sentis pressée de quelque chose qui
s’agitait sur mon corps; j’ouvris les yeux, je vis un homme blanc et de
bonne mine qui soupirait, et qui disait entre ses dents: \emph{O che
sciagura d’essere senza coglioni!}




\chapter{Suite des malheurs de la vieille}


\lettrine[ante=«]{É}{tonnée} et ravie d’entendre la langue de ma patrie, et non moins
surprise des paroles que proférait cet homme, je lui répondis qu’il y
avait de plus grands malheurs que celui dont il se plaignait; je
l’instruisis en peu de mots des horreurs que j’avais essuyées, et je
retombai en faiblesse. Il m’emporta dans une maison voisine, me fit
mettre au lit, me fit donner à manger, me servit, me consola, me
flatta, me dit qu’il n’avait rien vu de si beau que moi, et que jamais
il n’avait tant regretté ce que personne ne pouvait lui rendre. \frquote{\localleftbox{« }Je suis
né à Naples, me dit-il; on y chaponne deux ou trois mille enfants tous
les ans; les uns en meurent, les autres acquièrent une voix plus belle
que celle des femmes, les autres vont gouverner des états. On me fit
cette opération avec un très grand succès, et j’ai été musicien de la
chapelle de M\up{me} la princesse de \bname{Palestrine}. — De ma mère! m’écriai-je.
— De votre mère! s’écria-t-il en pleurant. Quoi! vous seriez cette jeune
princesse que j’ai élevée jusqu’à l’âge de six ans, et qui promettait
déjà d’être aussi belle que vous êtes? — C’est moi-même; ma mère est à
quatre cents pas d’ici coupée en quartiers sous un tas de morts…}\localleftbox{}




«Je lui contai tout ce qui m’était arrivé; il me conta aussi ses
aventures, et m’apprit comment il avait été envoyé chez le roi de Maroc
par une puissance chrétienne, pour conclure avec ce monarque un traité
par lequel on lui fournirait de la poudre, des canons, et des
vaisseaux, pour l’aider à exterminer le commerce des autres chrétiens.
\frquote{\localleftbox{« }Ma mission est faite, dit cet honnête eunuque; je vais m’embarquer à
Ceuta, et je vous ramènerai en Italie. \emph{Ma che sciagura d’essere senza
coglioni!}}\localleftbox{}


«Je le remerciai avec des larmes d’attendrissement; et au lieu de me
mener en Italie, il me conduisit à Alger, et me vendit au dey de cette
province. À peine fus-je vendue, que cette peste qui a fait le tour de
l’Afrique, de l’Asie, de l’Europe, se déclara dans Alger avec fureur.
«Vous avez vu des tremblements de terre; mais, mademoiselle, avez-vous
jamais eu la peste? — Jamais, répondit la baronne.

%\looseness=1
— Si vous l’aviez eue, reprit la vieille, vous avoueriez qu’elle est bien
au-dessus d’un tremblement de terre. Elle est fort commune en Afrique;
j’en fus attaquée. Figurez-vous quelle situation pour la fille d’un
pape, âgée de quinze ans, qui en trois mois de temps avait éprouvé la
pauvreté, l’esclavage, avait été violée presque tous les jours, avait
vu couper sa mère en quatre, avait essuyé la faim et la guerre, et
mourait pestiférée dans Alger! Je n’en mourus pourtant pas; mais mon
eunuque et le dey, et presque tout le sérail \linebreak d’Alger périrent.



«Quand les premiers ravages de cette épouvantable peste furent passés,
on vendit les esclaves du dey. Un marchand m’acheta, et me mena à
Tunis; il me vendit à un autre marchand qui me revendit à Tripoli; de
Tripoli je fus revendue à Alexandrie, d’Alexandrie revendue à Smyrne;
de Smyrne à Constantinople. J’appartins enfin à un aga des janissaires,
qui fut bientôt commandé pour aller défendre Azof contre les Russes qui
l’assiégeaient.

«L’aga, qui était un très galant homme, mena avec lui tout son sérail,
et nous logea dans un petit fort sur les Palus-Méotides, gardé par deux
eunuques noirs et vingt soldats. On tua prodigieusement de Russes, mais
ils nous le rendirent bien: Azof fut mis à feu et à sang, et on ne
pardonna ni au sexe, ni à l’âge; il ne resta que notre petit fort; les
ennemis voulurent nous prendre par famine. Les vingt janissaires
avaient juré de ne se jamais rendre. Les extrémités de la faim où ils
furent réduits les contraignirent à manger nos deux eunuques, de peur
de violer leur serment. Au bout de quelques jours ils résolurent de
manger les femmes.


«Nous avions un iman très pieux et très compatissant, qui leur fit un
beau sermon par lequel il leur persuada de ne nous pas tuer
tout-à-fait. \frquote{\localleftbox{« }Coupez, dit-il, seulement une fesse à chacune de ces
dames, vous ferez très bonne chère; s’il faut y revenir, vous en aurez
encore autant dans quelques jours; le ciel vous saura gré d’une action
si charitable, et vous serez secourus.}\localleftbox{}

«Il avait beaucoup d’éloquence; il les persuada: on nous fit cette
horrible opération; l’iman nous appliqua le même baume qu’on met aux
enfants qu’on vient de circoncire: nous étions toutes à la mort.


«À peine les janissaires eurent-ils fait le repas que nous leur avions
fourni, que les Russes arrivent sur des bateaux plats; pas un
janissaire ne réchappa. Les Russes ne firent aucune attention à l’état
où nous étions. Il y a partout des chirurgiens français: un d’eux qui
était fort adroit prit soin de nous, il nous guérit; et je me
souviendrai toute ma vie, que quand mes plaies furent bien fermées, il
me fit des propositions. Au reste, il nous dit à toutes de nous
consoler; il nous assura que dans plusieurs sièges pareille chose était
arrivée, et que c’était la loi de la guerre.


«Dès que mes compagnes purent marcher, on les fit aller à Moscou;
j’échus en partage à un boyard qui me fit sa jardinière, et qui me
donnait vingt coups de fouet par jour; mais ce seigneur ayant été roué
au bout de deux ans avec une trentaine de boyards pour quelque
tracasserie de cour, je profitai de cette aventure; je m’enfuis; je
traversai toute la Russie; je fus longtemps servante de cabaret à
Riga, puis à Rostock, à Vismar, à Leipsick, à Cassel, à Utrecht, à
Leyde, à La Haye, à Rotterdam: j’ai vieilli dans la misère et dans
l’opprobre, n’ayant que la moitié d’un derrière, me souvenant toujours
que j’étais fille d’un pape; je voulus cent fois me tuer, mais j’aimais
encore la vie. Cette faiblesse ridicule est peut-être un de nos
penchants les plus funestes; car y a-t-il rien de plus sot que de
vouloir porter continuellement un fardeau qu’on veut toujours jeter par
terre; d’avoir son être en horreur, et de tenir à son être; enfin de
\bname{caresser} le serpent qui nous dévore, jusqu’à ce qu’il nous ait \linebreak mangé le
cœur?



«J’ai vu dans les pays que le sort m’a fait parcourir, et dans les
cabarets où j’ai servi, un nombre prodigieux de personnes qui avaient
leur existence en exécration; mais je n’en ai vu que douze qui aient
mis volontairement fin à leur misère, trois nègres, quatre Anglais,
quatre Genevois, et un professeur allemand nommé Robeck. J’ai fini
par être servante chez le Juif don Issachar; il me mit auprès de vous,
ma belle demoiselle; je me suis attachée à votre destinée, et j’ai été
plus occupée de vos aventures que des miennes. Je ne vous aurais même
jamais parlé de mes malheurs, si vous ne m’aviez pas un peu piquée, et
s’il n’était d’usage, dans un vaisseau, de conter des histoires pour se
désennuyer. Enfin, mademoiselle, j’ai de l’expérience, je connais le
monde; donnez-vous un plaisir, engagez chaque passager à vous conter
son histoire, et s’il s’en trouve un seul qui n’ait souvent maudit sa
vie, qui ne se soit souvent dit à lui-même qu’il était le plus
malheureux des hommes, jetez-moi dans la mer la tête la première.»




\chapter[Comment Candide fut obligé de se séparer…]{Comment \bname{Candide} fut obligé de se séparer\\de la belle \bname{Cunégonde} et de la
vieille}


\lettrine{L}{a belle} \bname{Cunégonde}, ayant entendu l’histoire de la vieille, lui fit
toutes les politesses qu’on devait à une personne de son rang et de son
mérite. Elle accepta la proposition; elle engagea tous les passagers,
l’un après l’autre, à lui conter leurs aventures. \bname{Candide} et elle
avouèrent que la vieille avait raison. C’est bien dommage, disait
\bname{Candide}, que le sage \bname{Pangloss} ait été pendu contre la coutume dans un
auto-da-fé; il nous dirait des choses admirables sur le mal physique et
sur le mal moral qui couvrent la terre et la mer, et je me sentirais
assez de force pour oser lui faire respectueusement quelques
objections.

À mesure que chacun racontait son histoire, le vaisseau avançait. On
aborda dans \bname{Buenos} Aires. \bname{Cunégonde}, le capitaine \bname{Candide}, et la
vieille, allèrent chez le gouverneur don \bname{Fernando} \bname{d’Ibaraa}, y Figueora,
y \bname{Mascarenes}, y \bname{Lampourdos}, y Souza. Ce seigneur avait une fierté
convenable à un homme qui portait tant de noms. Il parlait aux hommes
avec le dédain le plus noble, portant le nez si haut, élevant si
impitoyablement la voix, prenant un ton si imposant, affectant une
démarche si altière, que tous ceux qui le saluaient étaient tentés de
le battre. Il aimait les femmes à la fureur. \bname{Cunégonde} lui parut ce
qu’il avait jamais vu de plus beau. La première chose qu’il fit fut de
demander si elle n’était point la femme du capitaine. L’air dont il fit
cette question alarma \bname{Candide}: il n’osa pas dire qu’elle était sa
femme, parce qu’en effet elle ne l’était point; il n’osait pas dire que
c’était sa sœur, parce qu’elle ne l’était pas non plus; et quoique ce
mensonge officieux eût été autrefois très à la mode chez les
anciens, et qu’il pût être utile aux modernes, son âme était trop
pure pour trahir la vérité. «M\up{lle}~\bname{Cunégonde}, dit-il, doit me
faire l’honneur de m’épouser, et nous supplions Votre Excellence de
daigner faire notre noce.»

Don Fernando d’Ibaraa, y Figueora, y Mascarenes, y \bname{Lampourdos}, y Souza,
relevant sa moustache, sourit amèrement, et ordonna au \mbox{capitaine}
\bname{Candide} d’aller faire la revue de sa compagnie. \bname{Candide} obéit; le
gouverneur demeura avec M\up{lle}~\bname{Cunégonde}. Il lui déclara sa
passion, lui protesta que le lendemain il l’épouserait à la face de
l’Église, ou autrement, ainsi qu’il plairait à ses charmes. \bname{Cunégonde}
lui demanda un quart d’heure pour se recueillir, pour consulter la
vieille, et pour se déterminer.

La vieille dit à \bname{Cunégonde}: «Mademoiselle, vous avez soixante et douze
quartiers et pas une obole; il ne tient qu’à vous d’être la femme du
plus grand seigneur de l’Amérique méridionale, qui a une très belle
moustache; est-ce à vous de vous piquer d’une fidélité à toute épreuve?
Vous avez été violée par les Bulgares; un Juif et un inquisiteur ont eu
vos bonnes grâces: les malheurs donnent des droits. J’avoue que si
j’étais à votre place, je ne ferais aucun scrupule d’épouser M.~le gouverneur, et de faire la fortune de M.~le capitaine \bname{Candide}.»
Tandis que la vieille parlait avec toute la prudence que l’âge et
l’expérience donnent, on vit entrer dans le port un petit vaisseau; il
portait un alcade et des alguazils, et voici ce qui était arrivé.

La vieille avait très bien deviné que ce fut un cordelier à la grande
manche qui vola l’argent et les bijoux de \bname{Cunégonde} dans la ville de
\bname{Badajos}, lorsqu’elle fuyait en hâte avec \bname{Candide}. Ce moine voulut
vendre quelques unes des 
pierreries
à un joaillier. Le marchand les
reconnut pour celles du grand-inquisiteur. Le cordelier, avant d’être
pendu, avoua qu’il les avait volées: il indiqua les personnes, et la
route qu’elles prenaient. La fuite de \bname{Cunégonde} et de \bname{Candide} était
déjà connue. On les suivit à Cadix: on envoya, sans perdre de temps, un
vaisseau à leur poursuite. Le vaisseau était déjà dans le port de
\bname{Buenos} Aires. Le bruit se répandit qu’un alcade allait débarquer, et
qu’on poursuivait les meurtriers de monseigneur le grand inquisiteur.
La prudente vieille vit dans l’instant tout ce qui était à faire. «Vous
ne pouvez fuir, dit-elle à \bname{Cunégonde}, et vous n’avez rien à craindre;
ce n’est pas vous qui avez tué monseigneur, et d’ailleurs le
gouverneur, qui vous aime, ne souffrira pas qu’on vous maltraite;
demeurez.» Elle court sur-le-champ à \bname{Candide}: «Fuyez, dit-elle, ou dans
une heure vous allez être brûlé. Il n’y avait pas un moment à perdre»;
mais comment se séparer de \bname{Cunégonde}, et où se réfugier?




\chapter[Comment Candide et \bname{Cacambo} furent reçus…]{Comment \bname{Candide} et \bname{Cacambo} furent reçus\\chez les jésuites du Paraguay}


\lettrine{C}{andide} avait amené de Cadix un valet tel qu’on en trouve beaucoup sur
les côtes d’Espagne et dans les colonies. C’était un quart \bname{d’Espagnol},
né d’un métis dans le Tucuman; il avait été enfant de chœur,
sacristain, matelot, moine, facteur, soldat, laquais. Il s’appelait
\bname{Cacambo}, et aimait fort son maître, parce que son maître était un fort
bon homme. Il sella au plus vite les deux chevaux andalous. «Allons, mon
maître, suivons le conseil de la vieille, partons, et courons sans
regarder derrière nous.» \bname{Candide} versa des larmes: «Ô ma chère \bname{Cunégonde}! 
faut-il vous abandonner dans le temps que M. le gouverneur va
faire nos noces! \bname{Cunégonde} amenée de si loin, que deviendrez-vous? — Elle
deviendra ce qu’elle pourra, dit \bname{Cacambo}; les femmes ne sont jamais
embarrassées d’elles; Dieu y pourvoit; courons. — Où me mènes-tu? où
allons-nous? que ferons-nous sans \bname{Cunégonde}? disait \bname{Candide}. — Par saint
Jacques de Compostelle, dit \bname{Cacambo}, vous alliez faire la guerre aux
jésuites, allons la faire pour eux; je sais assez les chemins, je vous
mènerai dans leur royaume, ils seront charmés d’avoir un capitaine qui
fasse l’exercice à la bulgare; vous ferez une fortune prodigieuse;
quand on n’a pas son compte dans un monde, on le trouve dans un autre.
C’est un très grand plaisir de voir et de faire des choses nouvelles.


— Tu as donc été déjà dans le Paraguay? dit \bname{Candide}. — Eh vraiment oui! dit
\bname{Cacambo}; j’ai été cuistre dans le collège de l’Assomption, et je
connais le gouvernement de Los Padres comme je connais les rues de
Cadix. C’est une chose admirable que ce gouvernement. Le royaume a déjà
plus de trois cents lieues de diamètre; il est divisé en trente
provinces. Los Padres y ont tout, et les peuples rien; c’est le
chef-d’œuvre de la raison et de la justice. Pour moi, je ne vois rien
de si divin que Los Padres, qui font ici la guerre au roi d’Espagne et
au roi de Portugal, et qui en Europe confessent ces rois; qui tuent ici
des Espagnols, et qui à Madrid les envoient au ciel; cela me ravit;
avançons: vous allez être le plus heureux de tous les hommes. Quel
plaisir auront Los Padres, quand ils sauront qu’il leur vient un
capitaine qui sait l’exercice bulgare!»



Dès qu’ils furent arrivés à la première barrière, \bname{Cacambo} dit à la
garde avancée qu’un capitaine demandait à parler à monseigneur le
commandant. On alla avertir la grande garde. Un officier paraguain
courut aux pieds du commandant lui donner part de la nouvelle. \bname{Candide}
et \bname{Cacambo} furent d’abord désarmés; on se saisit de leurs deux chevaux
andalous. Les deux étrangers sont introduits au milieu de deux files de
soldats; le commandant était au bout, le bonnet à trois cornes en tête,
la robe retroussée, l’épée au côté, l’esponton à la main. Il fit un
signe; aussitôt vingt-quatre soldats entourent les deux nouveaux venus.
Un sergent leur dit qu’il faut attendre, que le commandant ne peut leur
parler, que le révérend père provincial ne permet pas qu’aucun Espagnol
ouvre la bouche qu’en sa présence, et demeure plus de trois heures dans
le pays. «Et où est le révérend père provincial? dit \bname{Cacambo}. — Il est à
la parade après avoir dit sa messe, répondit le sergent, et vous ne
pourrez baiser ses éperons que dans trois heures. — Mais, dit \bname{Cacambo},
M. le capitaine, qui meurt de faim comme moi, n’est point
espagnol, il est allemand; ne pourrions-nous point déjeuner en
attendant Sa Révérence?»

Le sergent alla sur-le-champ rendre compte de ce discours au
commandant. «Dieu soit béni! dit ce seigneur, puisqu’il est allemand, je
peux lui parler; qu’on le mène dans ma feuillée.» Aussitôt on conduit
\bname{Candide} dans un cabinet de verdure, orné d’une très jolie colonnade de
marbre vert et or, et de treillages qui renfermaient des perroquets,
des colibris, des oiseaux-mouches, des pintades, et tous les oiseaux
les plus rares. Un excellent déjeuner était préparé dans des vases
d’or; et tandis que les Paraguains mangèrent du maïs dans des écuelles
de bois, en plein champ, à l’ardeur du soleil, le révérend père
commandant entra dans la feuillée.

C’était un très beau jeune homme, le visage plein, assez blanc, haut en
couleur, le sourcil relevé, l’œil vif, l’oreille rouge, les lèvres
vermeilles, l’air fier, mais d’une fierté qui n’était ni celle d’un
Espagnol ni celle d’un jésuite. On rendit à \bname{Candide} et à \bname{Cacambo} leurs
armes, qu’on leur avait saisies, ainsi que les deux 
chevaux
 andalous;
\bname{Cacambo} leur fit manger l’avoine auprès de la feuillée, ayant toujours
l’œil sur eux, crainte de surprise.


\bname{Candide} baisa d’abord le bas de la robe du commandant, ensuite ils se
mirent à table. «Vous êtes donc allemand? lui dit le jésuite en cette
langue. — Oui, mon 
Révérend
Père», dit \bname{Candide}. L’un et l’autre, en
prononçant ces paroles, se regardaient avec une extrême surprise, et
une émotion dont ils n’étaient pas les maîtres. «Et de quel pays
d’Allemagne êtes-vous? dit le jésuite. — De la sale province de
\bname{Westphalie}, dit \bname{Candide}: je suis né dans le château de
Thunder-ten-tronckh. — Ô ciel! est-il possible! s’écria le commandant.
— Quel miracle! s’écria \bname{Candide}. — Serait-ce vous? dit le commandant. — Cela
n’est pas possible», dit \bname{Candide}. Ils se laissent tomber tous deux à la
renverse, ils s’embrassent, ils versent des ruisseaux de larmes. «Quoi!
serait-ce vous, mon Révérend Père? vous, le frère de la belle
\bname{Cunégonde}! vous qui fûtes tué par les Bulgares! vous le fils de
M. le baron! vous jésuite au Paraguay! Il faut avouer que ce
monde est une étrange chose. Ô \bname{Pangloss}! \bname{Pangloss}! que vous seriez aise
si vous n’aviez pas été pendu!»


Le commandant fit retirer les esclaves nègres et les \bname{Paraguains} qui
servaient à boire dans des gobelets de cristal de roche. Il remercia
Dieu et saint Ignace mille fois; il serrait \bname{Candide} entre ses bras,
leurs visages étaient baignés de pleurs. «Vous seriez bien plus étonné,
plus attendri, plus hors de vous-même, dit \bname{Candide}, si je vous disais
que M\up{lle}~\bname{Cunégonde}, votre sœur, que vous avez crue éventrée,
est pleine de santé. — Où? — Dans votre voisinage, chez M.~le gouverneur de
\bname{Buenos} \bname{Aires}; et je venais pour vous faire la guerre.» Chaque mot qu’ils
prononcèrent dans cette longue conversation accumulait prodige sur
prodige. Leur âme tout entière volait sur leur langue, était attentive
dans leurs oreilles, et étincelante dans leurs yeux. Comme ils étaient
Allemands, ils tinrent table longtemps, en attendant le révérend père
provincial; et le commandant parla ainsi à son cher \bname{Candide}.




\chapter[Comment \bname{Candide} tua le frère…]{Comment \bname{Candide} tua le frère de sa chère \bname{Cunégonde}}


\lettrine[ante=«]{J’}{aurai} toute ma vie présent à la mémoire le jour horrible où je vis
tuer mon père et ma mère, et violer ma sœur. \textls[-5]{Quand les Bulgares furent
retirés, on ne trouva point cette sœur adorable, et on mit dans une
charrette ma mère, mon père, et moi, deux servantes et trois petits
garçons égorgés, pour nous aller enterrer dans une chapelle de
jésuites, à deux lieues du château de mes pères. Un jésuite nous jeta
de l’eau bénite; elle était horriblement salée; il en entra quelques
gouttes dans mes yeux; le père s’aperçut que ma paupière faisait un
petit mouvement: il mit la main sur mon cœur, et le sentit palpiter;
je fus secouru, et au bout de trois semaines il n’y paraissait pas.
Vous savez, mon cher \bname{Candide}, que j’étais fort joli; je le devins
encore davantage; aussi le révérend père Croust, supérieur de la
maison, prit pour moi la plus tendre amitié; il me donna l’habit de
novice; quelque temps après je fus envoyé à Rome. Le père général avait
besoin d’une recrue de jeunes jésuites allemands. Les 
souverains
du
Paraguay reçoivent le moins qu’ils peuvent de jésuites espagnols; ils
aiment mieux les étrangers, dont ils se croient plus maîtres. Je fus
jugé propre par le révérend père général pour aller travailler dans
cette vigne.
Nous partîmes, un Polonais, un Tyrolien, et moi. Je fus
honoré, en arrivant, du sous-diaconat et d’une lieutenance: je suis
aujourd’hui colonel et prêtre. Nous recevrons vigoureusement les
troupes du roi d’Espagne; je vous réponds qu’elles seront excommuniées
et battues. La Providence vous envoie ici pour nous seconder. Mais
est-il bien vrai que ma chère sœur \bname{Cunégonde} soit dans le voisinage,
chez le gouverneur de \bname{Buenos} Aires?» \bname{Candide} l’assura par serment que
rien n’était plus vrai. Leurs larmes recommencèrent à couler.}


Le baron ne pouvait se lasser d’embrasser \bname{Candide}; il l’appelait son
frère, son sauveur. «Ah! peut-être, lui dit-il, nous pourrons ensemble,
mon cher \bname{Candide}, entrer en vainqueurs dans la ville, et reprendre ma
sœur \bname{Cunégonde}. — C’est tout ce que je souhaite, dit \bname{Candide}; car je
comptais l’épouser, et je l’espère encore. — Vous, insolent! répondit le
baron, vous auriez l’impudence d’épouser ma sœur qui a soixante et
douze quartiers! Je vous trouve bien effronté d’oser me parler d’un
dessein si téméraire!» \bname{Candide}, pétrifié d’un tel discours, lui
répondit: «Mon Révérend Père, tous les quartiers du monde n’y font rien;
j’ai tiré votre sœur des bras d’un Juif et d’un inquisiteur; elle m’a
assez d’obligations, elle veut m’épouser. Maître \bname{Pangloss} m’a toujours
dit que les hommes sont égaux; et assurément je l’épouserai. — C’est ce
que nous verrons, coquin!» dit le jésuite baron de Thunder-ten-tronckh;
et en même temps il lui donna un grand coup du plat de son épée sur le
visage. \bname{Candide} dans l’instant tire la sienne, et l’enfonce jusqu’à la
garde dans le ventre du baron jésuite; mais en la retirant toute
fumante, il se mit à pleurer: «Hélas! mon Dieu! dit-il, j’ai tué mon
ancien maître, mon ami, mon beau-frère; je suis le meilleur homme du
monde, et voilà déjà trois hommes que je tue; et dans ces trois il y a
deux prêtres.»

\bname{Cacambo}, qui faisait sentinelle à la porte de la feuillée, accourut. «Il
ne nous reste qu’à vendre cher notre vie, lui dit son maître; on va,
sans doute, entrer dans la feuillée; il faut mourir les armes à la
main.» \bname{Cacambo}, qui en avait bien vu d’autres, ne perdit point la tête;
il prit la robe de jésuite que portait le baron, la mit sur le corps de
\bname{Candide}, lui donna le bonnet carré du mort, et le fit monter à cheval.
Tout cela se fit en un clin d’œil. «Galopons, mon maître; tout le monde
vous prendra pour un jésuite qui va donner des ordres; et nous aurons
passé les frontières avant qu’on puisse courir après nous.» Il volait
déjà en prononçant ces paroles, et en criant en espagnol: «Place, place
pour le révérend père colonel!»

\endinput


