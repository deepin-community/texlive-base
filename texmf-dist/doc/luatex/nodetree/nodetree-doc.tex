%!TEX program = lualatex
\documentclass{ltxdoc}
\usepackage{paralist,fontspec,microtype}
\usepackage[
  colorlinks=true,
  linkcolor=red,
  filecolor=red,
  urlcolor=red,
]{hyperref}
\usepackage{nodetree-embed}
\mdfsetup{
  innerleftmargin=0.2em,
  innerrightmargin=0.2em,
}
\EnableCrossrefs
\CodelineIndex
\RecordChanges

% Improve ltxdoc's `|...|` feature by allowing breaks.
\usepackage{fvextra}
\fvset{breaklines=true,breakbefore=.-_}
\AtBeginDocument{\DefineShortVerb{\|}}

% We use 'pmboxdraw' to draw box elements in minted environments.

% https://tex.stackexchange.com/questions/281368/print-box-drawing-characters-with-pdflatex/355403#355403
\usepackage{pmboxdraw}
\begingroup
  \def\DeclareUnicodeCharacter#1{%
    \begingroup
      \lccode`\~="#1\relax
    \lowercase{\endgroup
      \global\catcode`~=\active
      \gdef~%
    }%
  }%
  \input{pmboxdrawenc.dfu}%
\endgroup

% https://tex.stackexchange.com/questions/108127/block-element-characters-pmboxdraw-are-shown-too-wide-in-verbatim-and-verbatim
\pmboxdrawsetup{
  Block/box={\texttt{0}},
}

\usepackage{minted}
\usemintedstyle{colorful}
\BeforeBeginEnvironment{minted}{\begin{mdframed}[backgroundcolor=gray!3]}
\AfterEndEnvironment{minted}{\end{mdframed}}
\setminted{
  breaklines=true,
  fontsize=\footnotesize,
}
\setmintedinline{
  fontsize=auto
}

\def\TmpLuaCodeInline#1{\texttt{\scantokens{\catcode`\_=12\relax#1}}}

\def\TmpSecRef#1{(\rightarrow\ \ref{#1})}

\def\TmpPageSecRef#1{
  Page
  \pageref{#1},
  Section
  \ref{#1}
}

\newcommand{\TmpExample}[2]{
\begin{NodetreeEmbedView}[fontsize=#2]
\input{examples/#1.nttex}
\end{NodetreeEmbedView}
}

\newcommand{\TmpVerbExample}[2]{
\VerbatimInput[firstline=4]{examples/#1.tex}
\TmpExample{#1}{#2}
}

\newcommand\TmpMacroName[1]{%
  \texorpdfstring{\cs{#1}}{\textbackslash #1}%
}

\DefineVerbatimEnvironment{code}{Verbatim}
{
  frame=single,
  fontsize=\footnotesize,
}

\newcommand{\TmpLuaFunction}[1]{
  \marginpar{%
    \raggedleft%
    \MacroFont%
    \texttt{%
      \scantokens{\catcode`\_=12\relax#1}%
    }%
  }%
}

\begin{document}

\providecommand*{\url}{\texttt}
\GetFileInfo{nodetree.dtx}
\title{The \textsf{nodetree} package}
\author{%
  Josef Friedrich\\%
  \url{josef@friedrich.rocks}\\%
  \href{https://github.com/Josef-Friedrich/nodetree}{github.com/Josef-Friedrich/nodetree}\\%
  with contributions by Werner Lemberg
}
\date{v2.4.0 from 2024/09/17}

\maketitle

\begin{NodetreeEmbedEnv}
nodetree
\end{NodetreeEmbedEnv}

\newpage

\tableofcontents

\newpage

%-----------------------------------------------------------------------
% Abstract
%-----------------------------------------------------------------------

\section{Abstract}

|nodetree| is a Lua\TeX{} development package for both plain \TeX{}
and \LaTeX{} that visualizes the structure of node lists while
compiling with the \TeX{} engine. It uses a visual representation of
node lists similar to the UNIX |tree| command’s output for folder
trees. The processed document isn’t changed.

The tree view can be emitted to the console, to a log file, or as a
\LaTeX{} input file. Its appearance is highly customizable; multiple
color and B/W themes together with various levels of verbosity are
provided.

Node lists are the main building blocks of the \TeX{} engine, which
Lua\TeX{} allows to inspect and modify. |nodetree| is inspired by a
\href{https://gist.github.com/pgundlach/556247} {gist from Patrick
  Gundlach}.

%-----------------------------------------------------------------------
% Usage
%-----------------------------------------------------------------------

\section{Usage}

The package |nodetree| has four usage scenarios.
It can be used as a standalone Lua module, as a plain Lua\TeX{}, a
Lua\LaTeX{} package or as package to embed nodetree views in a
Lua\LaTeX{} document.

%-----------------------------------------------------------------------
%
%-----------------------------------------------------------------------

\newpage

\subsection{As a plain Lua\TeX{} package}

Run |luatex luatex-test.tex| for example to list the nodes using
Lua\TeX{}.

\begin{minted}{latex}
%%
%% This is file `nodetree.tex',
%% generated with the docstrip utility.
%%
%% The original source files were:
%%
%% nodetree.dtx  (with options: `tex')
%% 
%% This is a generated file.
%% 
%% Copyright (C) 2016-2024 by Josef Friedrich <josef@friedrich.rocks>
%% ----------------------------------------------------------------------
%% This work may be distributed and/or modified under the conditions of
%% the LaTeX Project Public License, either version 1.3c of this license
%% or (at your option) any later version. The latest version of this
%% license is in:
%% 
%%   http://www.latex-project.org/lppl.txt
%% 
%% and version 1.3c or later is part of all distributions of LaTeX
%% version 2008/05/05 or later.
%% 
\directlua{
  nodetree = require('nodetree')
}
\def\NodetreeSetOption[#1]#2{
  \directlua{
    nodetree.set_option('#1', '#2')
  }
}
\let\nodetreeoption\NodetreeSetOption
\def\NodetreeResetOption#1{
  \NodetreeSetOption[#1]{%
    \directlua{
      tex.print(nodetree.get_default_option('#1'))
    }%
  }%
}
\def\NodetreeReset{
  \NodetreeResetOption{callback}
  \NodetreeResetOption{channel}
  \NodetreeResetOption{color}
  \NodetreeResetOption{decimalplaces}
  \NodetreeResetOption{unit}
  \NodetreeResetOption{verbosity}
}
\let\nodetreereset\NodetreeReset
\def\NodetreeRegisterCallback#1{
  \directlua{
    nodetree.set_option('callback', '#1')
    nodetree.register_callbacks()
  }
}
\let\nodetreeregister\NodetreeRegisterCallback
\def\NodetreeUnregisterCallback#1{
  \directlua{
    nodetree.set_option('callback', '#1')
    nodetree.unregister_callbacks()
  }
}
\let\nodetreeunregister\NodetreeUnregisterCallback
\endinput
%%
%% End of file `nodetree.tex'.

\NodetreeRegisterCallback{postline}

Lorem ipsum dolor.
\bye
\end{minted}

\subsubsection{Available macros}

\def\TmpTabularMacrosPlainTeX{
\cmd{\NodetreeRegisterCallback}\marg{callbacks} &
\TmpPageSecRef{sec:cmd:nodetree-register-callback} \\

\cmd{\NodetreeUnregisterCallback}\marg{callbacks} &
\TmpPageSecRef{sec:cmd:nodetree-unregister-callback} \\

\cmd{\NodetreeSetOption}\oarg{option}\marg{value} &
\TmpPageSecRef{sec:cmd:nodetree-set-option} \\

\cmd{\NodetreeResetOption}\marg{option} &
\TmpPageSecRef{sec:cmd:nodetree-reset-option} \\

\cmd{\NodetreeReset} &
\TmpPageSecRef{sec:cmd:nodetree-reset} \\
}

\begin{tabular}{ll}
\textbf{Macro name} &
\textbf{Reference} \\

\TmpTabularMacrosPlainTeX

\end{tabular}

\subsubsection{Available options}

\def\TmpTabularOptionsBase{\scantextokens{
|callback| &
\TmpPageSecRef{sec:option:callback} \\

|verbosity| &
\TmpPageSecRef{sec:option:verbosity} \\

|color| &
\TmpPageSecRef{sec:option:color} \\

|unit| &
\TmpPageSecRef{sec:option:unit} \\

|decimalplaces| &
\TmpPageSecRef{sec:option:decimalplaces} \\
}}

\begin{tabular}{ll}
\textbf{Option name} &
\textbf{Reference} \\

\TmpTabularOptionsBase

|channel| &
\TmpPageSecRef{sec:option:channel} \\
\end{tabular}

%-----------------------------------------------------------------------
%
%-----------------------------------------------------------------------

\newpage

\subsection{As a Lua\LaTeX{} package}

Run |lualatex lualatex-test.tex| to show a node tree using
Lua\LaTeX{}. In Lua\LaTeX{} you can omit a call to
|\NodetreeRegisterCallback{postline}|, since |\usepackage{nodetree}|
registers the |post_linebreak_filter| by default. Use
|\NodetreeUnregisterCallback{postline}| if you don’t want to debug the
|post_linebreak_filter|.

\begin{minted}{latex}
\documentclass{article}
\usepackage{nodetree}

\begin{document}
Lorem ipsum dolor.
\end{document}
\end{minted}

\subsubsection{Available macros}

\begin{tabular}{ll}
\textbf{Macro name} &
\textbf{Reference} \\

\TmpTabularMacrosPlainTeX

\cmd{\NodetreeSet}\marg{kv-options} &
\TmpPageSecRef{sec:cmd:nodetree-set} \\
\end{tabular}

\subsubsection{Available options}

\begin{tabular}{ll}
\textbf{Option name} &
\textbf{Reference} \\

\TmpTabularOptionsBase

|channel| &
\TmpPageSecRef{sec:option:channel} \\
\end{tabular}

%-----------------------------------------------------------------------
%
%-----------------------------------------------------------------------

\newpage

\subsection{As a Lua module}

Import the Lua module of the package inside
\mintinline{latex}{\directlua{}}
with this command:
\mintinline{lua}{local nodetree = require('nodetree')}.
Then use the Lua function
\mintinline{lua}{nodetree.print(head, options)}
to debug nodes inside your Lua code.

\begin{minted}{lua}
local nodetree = require('nodetree')

local rule1 = node.new('rule')
rule1.width  = 20 * 65536
rule1.height = 10 * 65536
rule1.depth  = 10 * 65536
nodetree.print(vbox)
\end{minted}

\noindent
The function \mintinline{lua}{nodetree.print()} takes as a second
argument a Lua table to configure the output.

\begin{minted}{lua}
nodetree.print(vbox, { verbosity = 2, unit = 'cm' })
\end{minted}

\noindent
These are the default options:

\begin{minted}{lua}
options =  {
  callback = 'post_linebreak_filter',
  channel = 'term',
  color = 'colored',
  decimalplaces = 2,
  unit = 'pt',
  verbosity = 0,
  firstline = 1,
  lastline = -1,
}
\end{minted}

Options |firstline| and |lastline| only have an effect on function
\mintinline{lua}{nodetree.input(filename)}, which is used to implement
\cmd{\NodetreeEmbedInput} \TmpSecRef{sec:cmd:nodetree-embed-input}.

The following code snippet demonstrates the usage in Lua\TeX{}.
|head| is the current node.

\begin{minted}{latex}
\directlua{
  local nodetree = require('nodetree')
  local test = function (head)
    nodetree.print(head)
  end
  callback.register('post_linebreak_filter', test)
}

Lorem ipsum dolor.
\bye
\end{minted}

\noindent
This example illustrates how the function has to be applied in
Lua\LaTeX{}.

\begin{minted}{latex}
\documentclass{article}
\usepackage{nodetree}

\begin{document}

\directlua{
  local nodetree = require('nodetree')
  local test = function (head)
    nodetree.print(head)
  end
  luatexbase.add_to_callback('post_linebreak_filter', test, 'test')
}

Lorem ipsum dolor.
\end{document}
\end{minted}

%-----------------------------------------------------------------------
%
%-----------------------------------------------------------------------

\newpage
\subsection{The package \texttt{nodetree-embed}}

The single purpose of this auxiliary package is to provide a view
similar to a terminal (console) output.
This view mimics the output
of |nodetree| in a terminal.
The view can be embedded in a Lua\LaTeX{} file. You have to
compile documents using this embedded view with the option
|--shell-escape|.
The main environment of this package is |NodetreeEmbed|.
Markup
inside this environment is written into a temporary \LaTeX{} file.
This file is compiled in the background by |latexmk| and the
|nodetree| output is embedded into this view.
The following list shows each intermediate step:

\begin{enumerate}

\item |jobname.tex|

\begin{minted}{latex}
\begin{NodetreeEmbedEnv}
nodetree
\end{NodetreeEmbedEnv}
\end{minted}

\item |_nodetree-jobname/1.tex|

\begin{minted}{latex}
%!TEX program = lualatex
\documentclass{article}
\usepackage{nodetree}
\NodetreeSetOption[channel]{tex}
\NodetreeSetOption[verbosity]{0}
\NodetreeSetOption[unit]{pt}
\NodetreeSetOption[decimalplaces]{2}
\NodetreeUnregisterCallback{post_linebreak_filter}
\NodetreeRegisterCallback{post_linebreak_filter}
\begin{document}
nodetree
\end{document}
\end{minted}

\item |_nodetree-jobname/1.nttex|: This temporary Lua\LaTeX{} file is
compiled using |latexmk| and embedded in the environment |NodetreeEmbed|
(the trailing |\| character indicates line continuation).

\begin{minted}{latex}
Callback: \textcolor{NTEred}{post\_linebreak\_filter}\par
------------------------------------------\par
\mbox{├─\textcolor{NTEmagentabright}{GLUE}\hspace{0.5em}(baselineskip)\
  \textcolor{NTEyellow}{wd} 5.06\textcolor{NTEwhite}{pt}}\par
...
\end{minted}

\item Finally the result:

\begin{NodetreeEmbedEnv}
nodetree
\end{NodetreeEmbedEnv}

\end{enumerate}

\subsubsection{Available macros}

\begin{tabular}{ll}
\textbf{Macro name} &
\textbf{Reference} \\

\TmpTabularMacrosPlainTeX

\cmd{\NodetreeSet}\marg{kv-options} &
\TmpPageSecRef{sec:cmd:nodetree-set} \\

\cmd{\NodetreeEmbedCmd}\oarg{kv-options}\marg{tex-markup} &
\TmpPageSecRef{sec:cmd:nodetree-embed-cmd} \\

\cmd{\NodetreeEmbedInput}\oarg{kv-options}\marg{nttex-file} &
\TmpPageSecRef{sec:cmd:nodetree-embed-input} \\
\end{tabular}

\subsubsection{Available environment}

\begin{tabular}{ll}
\textbf{Environment name} &
\textbf{Reference} \\

|\begin{NodetreeEmbedEnv}|\oarg{kv-options} & % |\end{NodetreeEmbedEnv}|
\TmpPageSecRef{sec:env:nodetree-embed-env} \\
\end{tabular}

\subsubsection{Available options}

\begin{tabular}{ll}
\textbf{Option name} &
\textbf{Reference} \\

\TmpTabularOptionsBase

|theme| &
\TmpPageSecRef{sec:option:theme} \\

|thememode| &
\TmpPageSecRef{sec:option:thememode} \\

|font| &
\TmpPageSecRef{sec:option:font} \\

|fontsize| &
\TmpPageSecRef{sec:option:fontsize} \\

|firstline| &
\TmpPageSecRef{sec:option:firstlastline} \\

|lastline| &
\TmpPageSecRef{sec:option:firstlastline} \\
\end{tabular}

%-----------------------------------------------------------------------
% Macros
%-----------------------------------------------------------------------
\newpage
\section{Macros}

%%
% \NodetreeRegisterCallback
%%

\subsection{\TmpMacroName{NodetreeRegisterCallback}}
\label{sec:cmd:nodetree-register-callback}

\DescribeMacro{\NodetreeRegisterCallback}
\cmd{\NodetreeRegisterCallback}\marg{callbacks}: Globally register
\marg{callbacks}, which is a comma-separated list of callback aliases
\TmpSecRef{sec:option:callback}.

%%
% \NodetreeUnregisterCallback
%%

\subsection{\TmpMacroName{NodetreeUnregisterCallback}}
\label{sec:cmd:nodetree-unregister-callback}

\DescribeMacro{\NodetreeUnregisterCallback}
\cmd{\NodetreeUnregisterCallback}\marg{callbacks}: Globally unregister
\marg{callbacks}, which is a separated list of callback aliases
\TmpSecRef{sec:option:callback}.

%%
% \NodetreeSetOption
%%

\subsection{\TmpMacroName{NodetreeSetOption}}
\label{sec:cmd:nodetree-set-option}

\DescribeMacro{\NodetreeSetOption}
\cmd{\NodetreeSetOption}\oarg{option}\marg{value}: Globally set a
single \oarg{option} to \marg{value} \TmpSecRef{sec:options}.

%%
% \NodetreeResetOption
%%

\subsection{\TmpMacroName{NodetreeResetOption}}
\label{sec:cmd:nodetree-reset-option}

\DescribeMacro{\NodetreeResetOption}
\cmd{\NodetreeResetOption}\marg{option}: Globally reset a single
\marg{option} to its default value \TmpSecRef{sec:options}.

%%
% \NodetreeSet
%%

\subsection{\TmpMacroName{NodetreeSet}}
\label{sec:cmd:nodetree-set}

\DescribeMacro{\NodetreeSet}
\cmd{\NodetreeSet}\marg{kv-options}: Globally set multiple options at
once. It can only be used along with Lua\LaTeX{}. \marg{kv-options}
are key-value pairs.

\begin{code}
\NodetreeSet{color=no,callbacks={hpack,vpack},verbosity=2}
\end{code}

%%
% \NodetreeReset
%%

\subsection{\TmpMacroName{NodetreeReset}}
\label{sec:cmd:nodetree-reset}

\DescribeMacro{\NodetreeReset}
\cmd{\NodetreeReset}: Globally reset multiple options to their default
values.

%%
%
%%

\subsection{\TmpMacroName{NodetreeEmbedCmd}}
\label{sec:cmd:nodetree-embed-cmd}

\DescribeMacro{\NodetreeEmbedCmd}
\cmd{\NodetreeEmbedCmd}\oarg{kv-options}\marg{tex-markup}:

Main macro (cmd) to evaluate some \TeX{} markup and generate a
node tree from it. See environment version
\TmpSecRef{sec:cmd:nodetree-embed-cmd}. Uses |xparse|'s |+v| option to
grab the verbatim content. \marg{kv-options} are key-value pairs and
set locally only.

Only available in package |nodetree-embed|; you need option
|--shell-escape|.

%%
% \NodetreeEmbedInput
%%

\subsection{\TmpMacroName{NodetreeEmbedInput}}
\label{sec:cmd:nodetree-embed-input}

\DescribeMacro{\NodetreeEmbedInput}
\cmd{\NodetreeEmbedInput}\oarg{kv-options}\marg{nttex-file}: The path or
file name of the |*.nttex| file without the extension.
\marg{kv-options} are key-value pairs and set locally only.

Only available in package |nodetree-embed|. This command works without
option |--shell-escape|.

%-----------------------------------------------------------------------
% Environments
%-----------------------------------------------------------------------

\newpage
\section{Environments}

\subsection{\texttt{NodetreeEmbedEnv}}
\label{sec:env:nodetree-embed-env}

\DescribeEnv{NodetreeEmbedEnv}
|\begin{NodetreeEmbedEnv}|\oarg{kv-options}\\
\dots{} \textit{\TeX{} markup for evaluation} \dots\\
|\end{NodetreeEmbedEnv}|

Main environment (env) to evaluate some \TeX{} markup and generate a
node tree from it. See command version
\TmpSecRef{sec:cmd:nodetree-embed-cmd}. Uses the \cmd{\detokenize}
command to grab the verbatim content. \marg{kv-options} are key-value
pairs and set locally only.

Only available in package |nodetree-embed|; you need option
|--shell-escape|.

%-----------------------------------------------------------------------
% Options
%-----------------------------------------------------------------------
\newpage
\section{Options}
\label{sec:options}

%%
% callback
%%

\subsection{Option \texttt{callback}}
\label{sec:option:callback}

The option |callback| is the most important setting of the package. It
is possible to specify an alias to select the callback. Take a look at
the overview of callbacks (\rightarrow{} Figure~\ref{fig:callback}).
|nodetree| supports all node-related callbacks as listed in the
Lua\TeX{} reference manual.

These macros process callback options:

\begin{quote}
  \cmd{\NodetreeRegisterCallback}\marg{callbacks}\\
  \cmd{\NodetreeUnregisterCallback}\marg{callbacks}\\
  \cmd{\NodetreeSet}\marg{callback=<callbacks>}\\
  \cmd{\usepackage}\oarg{callback=<callbacks>}\marg{nodetree}
\end{quote}

The |nodetree| package can watch the node tree before and after the
functions of a callback are executed: It is possible to prepend and/or
append a colon (|:|) to indicate the desired watchpoint position,
which defaults to 'before' if no colon is used.

Use commas to specify multiple callbacks; trailing and leading
whitespace is ignored.  For example, this call

\begin{code}
\NodetreeRegisterCallback{:preline, line, :postline:}
\end{code}

\noindent
watches the node tree before the |preline| callback functions, before
the |line| callback functions, and before and after the |postline|
callback functions.  In case there are no callback functions
registered for one of the |hyphenate|, |kerning|, |ligaturing|, and
|mlist_to_hlist| callbacks, Lua\TeX{} executes some internal code
instead.  It thus makes sense to watch the node tree before and after
these (empty) callbacks even in this case.

Wrap your callback aliases in curly braces for the macro
|\NodetreeSet|. Note that no whitespace between |=| and |{| is
allowed.

\begin{code}
\NodetreeSet{callback={:preline, line, :postline:}}
\end{code}

The same applies for the macro |\usepackage|:

\begin{code}
\usepackage{callback={:preline, line, :postline:}}
\end{code}

The callbacks in Figure~\ref{fig:callback} are listed in the same
order as in the Lua\TeX{} reference manual. Note that the |ligaturing|
and |kerning| callbacks only have an effect on ligatures and kernings,
respectively, if the |luaotfload| package (which is the default for
Lua\LaTeX{}, and an optional package for Lua\TeX{}) handles the
affected font with |mode=base| (see the
\href{http://mirrors.ctan.org/macros/luatex/generic/luaotfload/luaotfload-latex.pdf}
{documentation} for more details).

%%
% Tabular callbacks
%%

\newcommand{\TmpCallbackRow}[3]{
  \TmpLuaCodeInline{#1} & \TmpLuaCodeInline{#2} & \TmpLuaCodeInline{\footnotesize#3} \\
}

\begin{figure}
\begin{tabular}{lll}
\textbf{Callback} & \textbf{Alias} & \textbf{Alias (longer)} \\
\TmpCallbackRow{contribute_filter}
{contribute}
{contributefilter}

\TmpCallbackRow{buildpage_filter}
{buildfilter} %
{buildpagefilter}

% new
\TmpCallbackRow{build_page_insert}
{buildinsert}
{buildpageinsert}

\TmpCallbackRow{pre_linebreak_filter}
{preline}
{prelinebreakfilter}

\TmpCallbackRow{linebreak_filter}
{line}
{linebreakfilter}

\TmpCallbackRow{append_to_vlist_filter}
{append}
{appendtovlistfilter}

\TmpCallbackRow{post_linebreak_filter}
{postline}
{postlinebreakfilter}

\TmpCallbackRow{hpack_filter}
{hpack}
{hpackfilter}

\TmpCallbackRow{vpack_filter}
{vpack}
{vpackfilter}

\TmpCallbackRow{hpack_quality}
{hpackq}
{hpackquality}

\TmpCallbackRow{vpack_quality}
{vpackq}
{vpackquality}

\TmpCallbackRow{process_rule}
{process}
{processrule}

\TmpCallbackRow{pre_output_filter}
{preout}
{preoutputfilter}

\TmpCallbackRow{hyphenate}
{hyph}
{}

\TmpCallbackRow{ligaturing}
{liga}
{}

\TmpCallbackRow{kerning}
{kern}
{}

\TmpCallbackRow{insert_local_par}
{insert}
{insertlocalpar}

\TmpCallbackRow{mlist_to_hlist}
{mhlist}
{mlisttohlist}
\end{tabular}

\caption{The callback aliases}
\label{fig:callback}
\end{figure}

%%
% channel
%%

\subsection{Option \texttt{channel}}
\label{sec:option:channel}

You can select the debug output channel with this option. The default
value for the option |channel| is |term|, which displays the node tree in
the current terminal. Specify |log| and the package creates a log file
named |<jobname>.ntlog|. Specify |tex| and a log file named
|<jobname>.nttex| is created. |nt...| stands for |nodetree|.
|<jobname>| is the basename of your file you want to debug. The debug
channel is only useful for the auxiliary package |nodetree-embed|. Paste
the markup in the environment |NodetreeEmbedView| and you get a
terminal-like view in your document.

%%
% verbosity
%

\subsection{Option \texttt{verbosity}}
\label{sec:option:verbosity}

Higher integer values result in a more verbose output. The default value
for this option is~|0|. At the moment verbosity levels |0| to~|3| are
implemented.

\def\TmpExampleVerbosity#1{
  \subsubsection{Example: \texttt{verbosity=#1}}
  \begin{NodetreeEmbedEnv}[verbosity=#1,callback=pre_linebreak_filter,
                           fontsize=\fontsize{5.5}{6.6}\selectfont]
  .
  \end{NodetreeEmbedEnv}
}

\TmpExampleVerbosity{0}
\TmpExampleVerbosity{1}
\TmpExampleVerbosity{2}
\TmpExampleVerbosity{3}

%%
% color
%%

\subsection{Option \texttt{color}}
\label{sec:option:color}

The default option for |color| is |colored|. Use any other string (for
example |none| or |no|) to disable the colored terminal output of the
package.

\begin{code}
\usepackage[color=no]{nodetree}
\end{code}

%%
% unit
%%

\subsection{Option \texttt{unit}}
\label{sec:option:unit}

The option |unit| sets the length unit to display all length values of
the nodes. The default option for |unit| is |pt|. See figures
\ref{fig:fixed-units} and~\ref{fig:relative-units} for possible values.

\begin{figure}
\begin{tabular}{lp{10cm}}
\textbf{Unit} &
\textbf{Description} \\

pt &
Point 1/72.27 inch. The conversion to metric units, to two decimal
places, is 1 point = 2.85 mm = 28.45 cm. \\

pc &
Pica, 12 pt \\

in &
Inch, 72.27 pt \\

bp &
Big point, 1/72 inch. This length is the definition of a point in
PostScript and many desktop publishing systems. \\

cm &
Centimeter \\

mm &
Millimeter \\

dd &
Didot point, 1.07 pt \\

cc &
Cicero, 12 dd \\

sp &
Scaled point, 1/65536 pt \\
\end{tabular}
\caption{Fixed units}
\label{fig:fixed-units}
\end{figure}

\begin{figure}
\begin{tabular}{lp{10cm}}
\textbf{Unit} &
\textbf{Description} \\

ex &
x-height of the current font \\

em &
Width of the capital letter M \\
\end{tabular}
\caption{Relative units}
\label{fig:relative-units}
\end{figure}

\def\TmpExampleUnit#1{
  \subsubsection{Example: \texttt{unit=#1}}
  \begin{NodetreeEmbedEnv}[unit=#1,callback=pre_linebreak_filter]
  Lorem.
  \end{NodetreeEmbedEnv}
}

\TmpExampleUnit{pt}
\TmpExampleUnit{sp}
\TmpExampleUnit{cm}

%%
% decimalplaces
%%

\subsection{Option \texttt{decimalplaces}}
\label{sec:option:decimalplaces}

The options |decimalplaces| sets the number of decimal places for some
node fields. If |decimalplaces| is set to |0| only integer values are shown.

\begin{code}
\NodetreeSetOption[decimalplaces]{4}
\end{code}

\def\TmpExampleDecimalplaces#1{
  \subsubsection{Example: \texttt{decimalplaces=#1}}
  \begin{NodetreeEmbedEnv}[unit=cc,decimalplaces=#1,callback=pre_linebreak_filter]
  Lorem.
  \end{NodetreeEmbedEnv}
}

\TmpExampleDecimalplaces{0}
\TmpExampleDecimalplaces{2}
\TmpExampleDecimalplaces{5}

%%
% theme and thememode
%%

\def\TmpExampleTheme#1#2{
  \subsubsection{Example: \texttt{theme=#1} \texttt{thememode=#2}}
  \begin{NodetreeEmbedEnv}[callback=pre_linebreak_filter,theme=#1,thememode=#2,fontsize=\small]
  .
  \end{NodetreeEmbedEnv}
}

\subsection{Option \texttt{theme} and \texttt{thememode}}
\label{sec:option:theme}
\label{sec:option:thememode}

% bw
\TmpExampleTheme{bwdark}{dark}
\TmpExampleTheme{bwlight}{light}

% monokaisoda
\TmpExampleTheme{monokaisoda}{dark}
\TmpExampleTheme{monokaisoda}{light}

%%
% font
%%

\subsection{Option \texttt{font}}
\label{sec:option:font}

\NodetreeSet{fontsize=\footnotesize}

\def\TmpExampleFont#1{
  \subsubsection{Example: \texttt{font=\{#1\}}}
  \begin{NodetreeEmbedEnv}[font={#1}]
  .
  \end{NodetreeEmbedEnv}
}

|nodetree-embed| passes the option |font| down to the
command |\setmonofont{}| of the |fontspec| package. The used font
should be monospaced and have some box drawing glyphs (see
table~\ref{fig:unicode}).

\TmpExampleFont{Liberation Mono}
\TmpExampleFont{Ubuntu Mono}

%%
% fontsize
%%

\subsection{Option \texttt{fontsize}}
\label{sec:option:fontsize}

\def\TmpExampleFontSize#1{
  \subsubsection{Example: \TmpMacroName{#1}}
  \begin{NodetreeEmbedEnv}[callback=pre_linebreak_filter,
                           fontsize=\csname #1\endcsname]
  .
  \end{NodetreeEmbedEnv}
}

\TmpExampleFontSize{small}
\TmpExampleFontSize{tiny}

\subsection{Options \texttt{firstline} and \texttt{lastline}}
\label{sec:option:firstlastline}

These two options are for function \cmd{\NodetreeEmbedInput} only
\TmpSecRef{sec:cmd:nodetree-embed-input}.  They specify the first
and last shown line of the read |*.nttex| file.  Values |1|, |2|,
\ldots, corresponds to the first line, second, line, etc.  Values
|-1|, |-2|, \ldots, correspond to the last line, the line before the
last line, etc.  The default values are |firstline = 1| and
|lastline = -1| to display the whole file.

%-----------------------------------------------------------------------
% Visual tree structure
%-----------------------------------------------------------------------
\newpage
\section{Visual tree structure}

%%
% Two different connections
%%

\subsection{Two different connections}

Nodes in Lua\TeX{} are connected. The |nodetree| package distinguishes
between \emph{list} and \emph{field} connections.

\begin{itemize}
 \item list: Nodes that are doubly connected by |next| and
       |previous| fields.
 \item field: Connections to nodes by other fields than |next| and
       |previous|, for example, using |head| and |pre|.
\end{itemize}

%%
% Unicode characters
%%

\subsection{Unicode characters to show the tree view}

\renewcommand{\arraystretch}{1.5}

The package |nodetree| uses the unicode box drawing symbols. Your
default terminal font should contain this characters to obtain the tree
view. Eight box drawing characters are necessary.

\begin{figure}
{
\fontspec{DejaVu Sans Mono}
\begin{tabular}{lcl}
\textbf{Code} & \textbf{Character} & \textbf{Name} \\
U+2500 & ─ & BOX DRAWINGS LIGHT HORIZONTAL \\
U+2502 & │ & BOX DRAWINGS LIGHT VERTICAL \\
U+2514 & └ & BOX DRAWINGS LIGHT UP AND RIGHT \\
U+251C & ├ & BOX DRAWINGS LIGHT VERTICAL AND RIGHT \\
U+2550 & ═ & BOX DRAWINGS DOUBLE HORIZONTAL \\
U+2551 & ║ & BOX DRAWINGS DOUBLE VERTICAL \\
U+255A & ╚ & BOX DRAWINGS DOUBLE UP AND RIGHT \\
U+2560 & ╠ & BOX DRAWINGS DOUBLE VERTICAL AND RIGHT \\
\end{tabular}
}
\caption{The Unicode box drawings glyphs}
\label{fig:unicode}
\end{figure}

\noindent
For |list| connections \emph{light} characters are shown.

{
\setmonofont{DejaVu Sans Mono}
\begin{code}
│ │
│ ├─list1
│ └─list2
└─list3
\end{code}
}

\noindent
|field| connections are visialized by \emph{Double} characters.

{
\setmonofont{DejaVu Sans Mono}
\begin{code}
║ ║
║ ╠═field1
║ ╚═field2
╚═field3
\end{code}
}

%-----------------------------------------------------------------------
% Examples
%-----------------------------------------------------------------------
\newpage
\section{Examples}

This section lists some examples of the |nodetree| output.

%%
% packagename
%%

\subsection{The node list of the package name}

\begin{NodetreeEmbedEnv}[showmarkup=true,callback=post_linebreak_filter]
nodetree
\end{NodetreeEmbedEnv}

%%
% math
%%

\subsection{The node list of a mathematical formula}

\begin{NodetreeEmbedEnv}[showmarkup=true,callback=post_linebreak_filter]
$1+2$
\end{NodetreeEmbedEnv}

%%
% ligatures
%%

\subsection{The node list of the word \emph{Office}}

The characters \emph{ffi} are deeply nested in a discretionary node.

\begin{NodetreeEmbedEnv}[showmarkup=true,decimalplaces=0]
Office
\end{NodetreeEmbedEnv}

%-----------------------------------------------------------------------
% Node types
%-----------------------------------------------------------------------

\section{Node types}

This section shows some node types in a |nodetree| view.

\newcommand{\TmpHeadingNodeTypeSub}[4]{
  \subsection{Type \texttt{#1(#2)}, subtype \texttt{#3(#4)}}
}

\newcommand{\TmpNodeTypeSub}[6]{
  \subsection{Type \texttt{#1(#2)}, subtype \texttt{#3\_#4(#5)}}
  \TmpVerbExample{#2#1#5#3#4}{#6}
}

\newcommand{\TmpHeadingNodeType}[2]{
  \subsection{Type \texttt{#1(#2)}}
}

%%
%
%%

\TmpHeadingNodeTypeSub{hlist}{0}{line}{1}

\begin{NodetreeEmbedEnv}[showmarkup=true]
Lorem
\end{NodetreeEmbedEnv}

%%
%
%%

\TmpHeadingNodeTypeSub{hlist}{0}{box}{2}

\begin{NodetreeEmbedEnv}[showmarkup=true]
L\hbox to 40pt{ore}m
\end{NodetreeEmbedEnv}

%%
%
%%

\TmpHeadingNodeTypeSub{hlist}{0}{indent}{3}

\begin{NodetreeEmbedEnv}[showmarkup=true,unit=cm]
\setlength{\parindent}{5cm}
I
\end{NodetreeEmbedEnv}

%%
%
%%

\TmpHeadingNodeType{vlist}{1}

\begin{NodetreeEmbedEnv}[showmarkup=true,decimalplaces=1]
L\vbox to 40pt{O}L
\end{NodetreeEmbedEnv}

%%
%
%%

\TmpHeadingNodeType{rule}{2}

\begin{NodetreeEmbedEnv}[showmarkup=true,unit=mm]
\rule[-2mm]{10mm}{4mm}
\end{NodetreeEmbedEnv}

%%
%
%%

\TmpHeadingNodeType{mark}{4}

\begin{NodetreeEmbedEnv}[showmarkup=true,callback=pre_output_filter]
\mark{Lorem}.
\end{NodetreeEmbedEnv}

%%
%
%%

\TmpHeadingNodeTypeSub{disc}{7}{discretionary}{0}

\begin{NodetreeEmbedEnv}[showmarkup=true]
L\discretionary{}{}{}L
\end{NodetreeEmbedEnv}

%%
%
%%

\TmpHeadingNodeTypeSub{disc}{7}{explicit}{1}

\begin{NodetreeEmbedEnv}[showmarkup=true]
L\-O\-L
\end{NodetreeEmbedEnv}

%%
%
%%

\TmpHeadingNodeTypeSub{disc}{7}{regular}{3}

\begin{NodetreeEmbedEnv}[showmarkup=true,decimalplaces=0]
Office
\end{NodetreeEmbedEnv}

%%
%
%%

\TmpNodeTypeSub{whatsit}{8}{pdf}{action}{22}{\fontsize{5.5}{6.6}\selectfont}
\TmpNodeTypeSub{whatsit}{8}{pdf}{colorstack}{28}{\footnotesize}

%%
%
%%

\TmpHeadingNodeType{dir}{10}

\begin{NodetreeEmbedEnv}[showmarkup=true]
\textdir TRT nur {\textdir TLT run \textdir TRT NUR} nur
\end{NodetreeEmbedEnv}

%%
%
%%

\TmpHeadingNodeTypeSub{glue}{12}{baselineskip}{2}

\NodetreeEmbedCmd[showmarkup=true,unit=cm]{
\baselineskip=5cm
Lorem

Lorem
}

%%
%
%%

\TmpHeadingNodeTypeSub{glue}{12}{parskip}{3}

\NodetreeEmbedCmd[showmarkup=true,callback=pre_output_filter]{
\parskip=5cm
Lorem

Lorem
}

%%
%
%%

\TmpHeadingNodeTypeSub{glue}{12}{spaceskip}{13}

\begin{NodetreeEmbedEnv}[showmarkup=true]
\spaceskip=5cm
a a
\end{NodetreeEmbedEnv}

%%
%
%%

\TmpHeadingNodeTypeSub{glue}{12}{leaders}{100}

\begin{NodetreeEmbedEnv}[showmarkup=true]
a \leavevmode\leaders\hbox{ . }\hfill\kern0pt a
\end{NodetreeEmbedEnv}

%%
%
%%

\TmpHeadingNodeTypeSub{glue}{12}{cleaders}{101}

\begin{NodetreeEmbedEnv}[showmarkup=true]
a \leavevmode\cleaders\hbox{ . }\hfill\kern0pt a
\end{NodetreeEmbedEnv}

%%
%
%%

\TmpHeadingNodeTypeSub{glue}{12}{xleaders}{102}

\begin{NodetreeEmbedEnv}[showmarkup=true]
a \leavevmode\xleaders\hbox{ . }\hfill\kern0pt a
\end{NodetreeEmbedEnv}

%%
%
%%

\TmpHeadingNodeTypeSub{glue}{12}{gleaders}{102}

\begin{NodetreeEmbedEnv}[showmarkup=true]
a \leavevmode\gleaders\hbox{ . }\hfill\kern0pt a
\end{NodetreeEmbedEnv}

%%
%
%%

\TmpHeadingNodeTypeSub{kern}{13}{userkern}{0}

\begin{NodetreeEmbedEnv}[showmarkup=true]
a\kern2pt
\end{NodetreeEmbedEnv}

%%
%
%%

\TmpHeadingNodeTypeSub{kern}{13}{fontkern}{1}

\begin{NodetreeEmbedEnv}[showmarkup=true]
Ve
\end{NodetreeEmbedEnv}

%%
%
%%

\TmpHeadingNodeTypeSub{kern}{13}{accentkern}{2}

\begin{NodetreeEmbedEnv}[showmarkup=true]
\accent96 a
\end{NodetreeEmbedEnv}

%%
%
%%

\TmpHeadingNodeTypeSub{kern}{13}{italiccorrection}{3}

\begin{NodetreeEmbedEnv}[showmarkup=true]
\textit{L}\/OL
\end{NodetreeEmbedEnv}

%%
%
%%

\TmpHeadingNodeType{penalty}{14}

\begin{NodetreeEmbedEnv}[showmarkup=true]
L \penalty 23 OL
\end{NodetreeEmbedEnv}

%%
%
%%

\TmpHeadingNodeType{glyph}{29}

\begin{NodetreeEmbedEnv}[showmarkup=true]
abc
\end{NodetreeEmbedEnv}

%%
%
%%

% It seems that 'attribute' nodes don't appear in node trees.
%
% \TmpHeadingNodeType{attribute}{38}
%
% \begin{NodetreeEmbedEnv}[showmarkup=true]
% {\attribute0=1 A}
% \end{NodetreeEmbedEnv}

%%
%
%%

\TmpHeadingNodeType{attributelist}{40}

\begin{NodetreeEmbedEnv}[showmarkup=true,callback=hpackfilter]
{\attribute0=1 A}
\end{NodetreeEmbedEnv}

\DocInput{nodetree.dtx}

\subsection{The file \texttt{nodetree.lua}}

% Compilation failure:
% lualatex: ../../../texk/web2c/luatexdir/lang/texlang.c:986: hnj_hyphenation: Assertion `(((varmem[(wordstart)].hh.u.B1) & (1 << 0)) && !((varmem[(wordstart)].hh.u.B1) & (1 << 1) ) && !((varmem[(wordstart)].hh.u.B1) & (1 << 2) ))' failed
% \inputminted{lua}{nodetree.lua}
There is a source code documentation of the file nodetree.lua compiled
with Ldoc on Github:
\url{http://josef-friedrich.github.io/nodetree/}

\end{document}
