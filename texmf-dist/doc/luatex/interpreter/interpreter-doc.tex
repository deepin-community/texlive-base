% This is the master file producing interpreter-doc.pdf. The version of the
% documentation readable in a text editor is interpreter-doc.txt (input below).
% 
% Paul Isambert - zappathustra AT free DOT fr - June 2012

\input pitex

\overfullrule=0pt

\OutputRoutine remove {headers}{shipout}

\setparameter page :
  hsize  = 25pc
  left   = 60pt
  width  = "\dimexpr 25pc + 120pt\relax"
  lines  = 35
  height = 20cm

\setparameter section :
  font = \bf
  link = true
  number = none
  numbercommand = \llap
  beforeskip = 1

\setparameter navigator :
  title  = "Interpreter documentation"
  author = "Paul Isambert"
  mode   = outlines

\setfont\mainfont:
  name     = "Chaparral Pro"
  bold     = Semibold
  big      = 18pt

\setfont\codefont:
  name        = "Lucida Console"
  slant       = 15
  bold italic = none
  size        = 8pt
  features    = "-tlig, -trep, space = mono"

\parfillskip=0pt plus 1fill
\def\describe#1#2#3{%
  \unless\ifdim\lastskip=\baselineskip
    \vskip\baselineskip
  \fi
  \needspace{2\baselineskip}%
  \noindent\color{.8 0 0}{%
    {\outline{#3}{\directlua{%
      local t = string.gsub("\luaescapestring{#1}", "[ (].*", "")
      tex.print(t)}}%
     \codefont#1}%
  \reverse\iffemptystring{#2}
    {\kern1em \hfil\penalty0\hbox{\ital{(#2)}}}}%
  \par
  \removenextindent}

\newverbatim\source{}
                   {\vskip\baselineskip
                    \hfuzz=1em
                    \codefont\parindent=0pt
                    \pdfcolorstack0 push {.8 0 0 rg}
                    \printverbatim
                    \pdfcolorstack0 pop
                    \vskip\baselineskip}

% The verbatim facilities in PiTeX aren't gated yet, so I must rely on
% this horrible hack!
\directlua{
function do_verbatim (name, exec)
	if exec then
    tex.print(pitex.verbatims[name])
  else
    for n, l in ipairs(pitex.verbatims[name]) do
      if n == \string#pitex.verbatims[name] then
        tex.print("\noexpand\\penalty\noexpand\\widowpenalty")
      end
      tex.print(pitex.verbatims[name].regime, l)
      if n == 1 and \string#pitex.verbatims[name] > 2 then
        tex.print("\noexpand\\penalty\noexpand\\widowpenalty")
      end
    end
  end
end
}

\def\arg#1{{\codefont\char"2039 #1\char"203A}}
\pdfdef\ital#1{#1}
\pdfdef\verb`#1`{#1}

% Not optimal, but hey, with all the "intepreter.core.classes" stuff...
\hyphenation{li-nes cla-sses}

\input interpreter

% Title
\vbox to 3\baselineskip{
\hbox to \hsize{\big Interpreter\hfil\normalsize Paul Isambert}
\hbox to \hsize{v.1.2, June 2012 \hfil \tcode{zappathustra AT free DOT fr}}
\vfil
}



% Bulk of the doc.
\interpretfile{doc}{interpreter-doc.txt}
\vskip0pt plus 1filll
\noindent
\bgroup\it
Typeset with Lua\TeX\ 0.71 in Chaparral Pro and Lucida Console
... nonetheless this documentation looks dull, I don't know why.
\egroup

\bye
