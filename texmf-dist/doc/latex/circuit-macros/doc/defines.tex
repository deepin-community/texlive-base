%
\newdimen{\leftwid}\setlength{\leftwid}{140pt}
\newdimen{\libwid}\setlength{\libwid}{35pt}
\newdimen\rightwid
\setlength{\rightwid}{\textwidth}
\addtolength{\rightwid}{-\leftwid}
\addtolength{\rightwid}{-\libwid}
%
\newbox\Defbox

%\newcommand{\macx}[2]{%
%  \phantomsection\pdfbookmark[subsection]{#1}{#2}#1}

% This is alchemy but good enough for now.
\newcommand{\macrodef}[5]{\paragraph*{\hfill}\noindent%
  \phantomsection\pdfbookmark[subsubsection]{#1}{#2}\hspace*{-\parindent}%
  \setbox\Defbox=\hbox{\tt #1#3}%
  \ifdim\wd\Defbox>\textwidth%
    \setbox\Defbox=\hbox{\parbox[t]{\textwidth}{\tt #1#3}}\fi%
  \nopagebreak%
  \ifdim\wd\Defbox>\leftwid%
    \box\Defbox\hfill\break\hspace*{-1ex}\hbox to \leftwid{\hfill}%
  \else\hbox to \leftwid{\box\Defbox\hfill}\fi%
  \hbox to \libwid{#4\hfill}%
  \begin{minipage}[t]{\rightwid}\raggedright#5\end{minipage}}%

\newcommand{\seesect}[1]{ (\SR{#1})}

\newcommand{\Letter}[1]{\noindent%
  \pdfbookmark[subsection]{#1}{#1}%
  \hskip-2em\hbox to 2em{\par#1\label{#1}}\relax}%

\newcommand{\LR}[1]{\hyperref[#1]{{#1}}\hskip1em}
\newcommand{\fra}[1]{\hbox{\sl frac#1}}%

%%%%%%%%%%%%%%%%%%%%%%%%%%%%%%%%%%%%%%%%%%%%%%%%%%%%%%%%%%%%%%%%

%\raggedbottom
\vspace{\abovedisplayskip}
\noindent
\LR{A} \LR{B} \LR{C} \LR{D} \LR{E} \LR{F} \LR{G} \LR{H} \LR{I} \LR{J}
\LR{K} \LR{L} \LR{M} \LR{N} \LR{O} \LR{P}
\LR{R} \LR{S} \LR{T} \LR{U} \LR{V} \LR{W} \LR{X} \LR{Y}
\\

\Letter{A}%
\macrodef{above\_}{above_}{}%
  {gen}%
  {string position above relative to current direction}%
\macrodef{abs\_}{abs_}{({\sl number})}%
  {gen}%
  {absolute value function}%
\macrodef{ACsymbol}{ACsymbol}{(at {\sl position, len, ht,} %
  [n:][A]U|D|L|R|{\sl degrees}) }%
  {cct}%
  {draw a stack of $n$ (default 1) AC symbols
   (1-cycle sine waves); If arg 4 contains A, two arcs
   are used.  The current drawing direction is default, otherwise Up, Down,
   Left, Right, or at {\sl degrees} slant; \seesect{Twoterminal:} e.g.,\\ 
   {\tt ebox; $\lbrace$ACsymbol(at last [],{,}dimen\_/8)$\rbrace$}}%
\macrodef{adc}{adc}{({\sl width, height, nIn, nN, nOut, nS})}%
  {cct}%
  {Analog-digital converter with defined width, height, and number of
   inputs {\tt In$i$}, top terminals {\tt N$i$}, ouputs {\tt Out$i$},
    and bottom terminals~{\tt S$i$}}%
\macrodef{addtaps}{addtaps}{([{\sl arrowhd}%
  | type={\sl arrowhd};name={\sl Name}],
    {\sl fraction, length, fraction, length,} $\cdots$)}%
  {cct}%
  {Add taps to the previous two-terminal element.
   {\sl arrowhd} is blank or one of {\tt . - <- -> <->}.
   Each fraction determines the position along the element body of the tap.
   A negative length draws the tap to the right of the current
   direction; positive length to the left.
   Tap names are Tap1, Tap2, $\cdots$ by default  or
   Name1, Name2, $\cdots$ if specified 
   \seesect{Composite:}}%
\macrodef{along\_}{along_}{({\sl linear object name})}%
  {gen}%
  {short for {\tt between {\sl name}.start and {\sl name}.end}}%
\macrodef{Along\_}{Along_}{({\sl LinearObj,distance,}[R])}%
  {gen}%
  {Position arg2 (default all the way) along a linear object
   from {\tt .start} to {\tt .end}
   (from {\tt .end} to {\tt .start} if arg3={\tt R}) }%
\macrodef{amp}{amp}{(\linespec, {\sl size}, {\sl attributes})}%
  {cct}%
  {amplifier\seesect{Twoterminal:}}%
\macrodef{And, Or, Not, Nand, Nor, Xor, Nxor, Buffer}%
 {And, Or, Not, Nand, Nor, Xor, Nxor, Buffer}{}%
  {log}%
  {Wrappers of {\tt AND\_gate}, $\ldots$ for use in the {\tt Autologix}
   macro}%
\macrodef{AND\_gate}{AND_gate}{({\tt {\sl n}, [N][B],
   [{\sl wid}, [{\sl ht}]], {\sl attributes}})}%
  {log}%
  {`and' gate, 2 or {\sl n\/} inputs ($0 \leq n \leq 16$) drawn in the
   current direction;
   {\tt N}: negated inputs; {\tt B}: box shape.
   Alternatively, {\tt AND\_gate({\sl chars}, [B], {\sl wid},
    {\sl ht}, {\sl attributes})}, where
   arg1 is a sequence of letters {\tt P|N} to define
   normal or negated inputs.
    \seesect{Logicgates:}}%
\macrodef{AND\_gen}{AND_gen}{($n$, {\sl chars}, [{\sl wid}, [{\sl ht}]],
  {\sl attributes})}%
  {log}%
  {general AND gate: $n$=number of inputs $(0\leq n\leq 16)$;
           {\sl chars:}
           B=base and straight sides; A=Arc;
           [N]NE,[N]SE,[N]I,[N]N,[N]S=inputs or circles;
           [N]O=output; C=center.
   Otherwise, arg1 can be a sequence of letters {\tt P|N}
   to define normal or negated inputs; arg2 is as above except that
   {\tt [N]I} is ignored. Arg 5 contains body attributes.}%
\macrodef{AND\_ht}{AND_ht}{}%
  {log}%
  {height of basic `and' and `or' gates in {\tt L\_unit}s}%
\macrodef{AND\_wd}{AND_wd}{}%
  {log}%
  {width of basic `and' and `or' gates in {\tt L\_unit}s}%
\macrodef{antenna}{antenna}{%
  (at {\sl location}, T, A|L|T|S|D|P|F, U|D|L|R|{\sl degrees})}%
  {cct}%
  {antenna, without stem for nonblank 2nd arg; arg3 is\\
    {\tt A}: aerial (default),\\
    {\tt L}: loop,\\
    {\tt T}: triangle,\\
    {\tt S}: diamond,\\
    {\tt D}: dipole,\\
    {\tt P}: phased,\\
    {\tt F}: fork;\\
    arg4 specifies Up, Down, Left, Right, or angle from horizontal (default 90) 
   \seesect{Composite:}}%
\macrodef{arca}{arca}{({\sl absolute chord linespec}, ccw|cw, {\sl radius},
   {\sl modifiers}) }%
  {gen}%
  {arc with acute angle (obtuse if radius is negative), drawn in a [ ] block}%
\macrodef{ArcAngle}{ArcAngle}{({\sl position, position, position, radius,
  modifiers, label}) }%
  {gen}%
  {Arc angle symbol drawn ccw at arg2. Arg4 is the radius from arg2;
     arg5 contains line attributes, e.g., {\tt thick linethick/2 ->};
     arg6 is an optional label at mid-arc}%
\macrodef{arcd}{arcd}{({\sl center},
  {\sl radius},{\sl start degrees},{\sl end degrees}) }%
  {gen}%
  {Arc definition (see {\tt arcr}), angles in degrees
    \seesect{Positions:}}%
\macrodef{arcdimension\_}{arcdimension_}{({\sl arcspec},{\sl offset},%
{\sl label},
    D|H|W|{\sl blank width},{\sl tic offset},{\sl arrowhead })}%
  {gen}%
  {like {\tt dimension\_}, for drawing arcs for dimensioning diagrams;
      {\sl arrowhead=}{\tt -> | <-}. Uses the first argument as the attributes
      of an invisible arc: {\tt arc invis }{\sl arg1}.  {\sl Arg2} is the
      radial displacement (possibly negative) of the dimension arrows.
      If {\sl arg3} is {\tt s\_box(\ldots)} or {\sl rs\_box(\ldots)} and
      {\sl arg4=}{\tt D|H|W} then {\sl arg4} means:
      {\tt D:} blank width is the diagonal length of {\sl arg3};
      {\tt H:} blank width is the height of {\sl arg3} + {\tt textoffset*2};
      {\tt W:} blank width is the width of {\sl arg3} + {\tt textoffset*2};
      otherwise {\sl arg4} is the absolute blank width}%
\macrodef{arcr}{arcr}{({\sl center},{\sl radius},{\sl start angle},{\sl
    end angle,modifiers,ht}) }%
  {gen}%
  {Arc definition.  If arg5 contains {\tt <-} or {\tt ->} then a midpoint
    arrowhead of height equal to arg6 is added. Arg5 can contain modifiers
    (e.g. outlined "red"), for the arc and arrowhead.  Modifiers following
    the macro affect the arc only,
    e.g., {\tt arcr(A,r,0,pi\_/2,->) dotted ->}%
   \seesect{Positions:}}%
\macrodef{arcto}{arcto}{({\sl position 1}, {\sl position 2},
    {\sl radius}, [dashed|dotted])}%
  {gen}%
  {line toward position 1 with rounded corner toward position 2}%
\macrodef{array}{array}{({\sl variable, expr1, expr2, \ldots})}%
  {dpictools}%
  {$\;\;$  Populate a singly-subscripted array:
    {\tt {\sl var}[1]={\sl expr1;} {\sl var}[2]={\sl expr2;} \ldots}.}%
\macrodef{array2}{array2}{({\sl variable, expr1, expr2, \ldots})}%
  {dpictools}%
  {$\;\;$  Populate a doubly-subscripted array:
    {\tt {\sl var}[{\sl expr1},1]={\sl expr2;}
         {\sl var}[{\sl expr1},2]={\sl expr3;} \ldots}.}%
\macrodef{arraymax}{arraymax}{({\sl data array, n, index name, value})}%
  {dpictools}%
  {$\;\;$ Find the index in {\sl array[1:n]} of the first occurrence
   of the maximum array element value.  The value is assigned if arg4 is
   nonblank; example: 
   {\tt array(x,4,9,8,6); arraymax( x,4,i )}
    assigns 2 to {\sl i,} and {\tt arraymax( x,4,i,m )}
    assigns 2 to {\sl i} and 9 to {\sl m.}}%
\macrodef{arraymin}{arraymin}{({\sl data array, n, index name, value})}%
  {dpictools}%
  {$\;\;$ Find the index in {\sl array[1:n]} of the first occurrence
   of the minimum array element value.  The value is assigned if arg4 is
   nonblank; see {\tt arraymax}.}%
\macrodef{arrester}{arrester}{(\linespec,{\sl chars},
                                   {\sl len}[:{\sl arrowhead ht}],
                                   {\sl ht}[:{\sl arrowhead wid}],
                                   {\sl attributes} )}%
  {cct}%
{Arg2 {\sl chars:}\\
    {\tt G=} spark gap (default)\\
    {\tt g=} general (dots)\\
    {\tt E=} gas discharge\\
    {\tt S=} box enclosure\\
    {\tt C=} carbon block\\
    {\tt A=} electrolytic cell\\
    {\tt H=} horn gap\\
    {\tt P=} protective gap\\
    {\tt s=} sphere gap\\
    {\tt F=} film element\\
    {\tt M=} multigap\\
    {\sl Modifiers appended to arg2:}\\
    {\tt R=} right orientation\\
    {\tt L=} left orientation\\
    {\tt D=} add terminals {\sl A, B, G} for {\tt S, E} only\\
    Arg5 applies to options C, E, S
    \seesect{Twoterminal:}}%
\macrodef{arrowline}{arrowline}{(\linespec)}%
  {cct}%
  {line (dotted, dashed permissible) with centred arrowhead
    \seesect{Twoterminal:}}%
\macrodef{assign3}{assign3}{({\sl name, name, name, arg4, arg5, arg6})}%
  {gen}%
  {Assigns \$1 = arg4 if \$1 is nonblank; similarly \$2 = arg5 and \$3 = arg6}%
\macrodef{AutoGate}{AutoGate}{\relax}%
  {log}%
  {Draw the tree for a gate as in the {\tt Autologix} macro.  No inputs
   or external connections are drawn.  The names of the internal gate
   inputs are stacked in {\tt `AutoInNames'}}%
\macrodef{Autologix}{Autologix}%
 {({\sl Boolean function sequence},%
    [N[oconnect]][L[eftinputs]][R][V][M][;offset={\sl value}])}%
  {log}%
  {Draw the Boolean expressions defined in function notation.
   The first argument is a semicolon (;)-separated sequence of
   Boolean function specifications using the functions
   {\tt And, Or, Not, Buffer, Xor, Nand, Nor, Nxor}
   with variables, e.g.,
   {\tt Autologix(And(Or(x1,\~{}x2),Or(\~{}x1,x2)));}.\\
   Each function specification is of the form\\
   {\tt {\sl function}({\sl arguments}) [@{\sl attributes}]}.
   \par
   Function outputs are aligned vertically but appending
   {\tt@}{\sl attributes} to a function can be used to place it; e.g.,
   {\tt Nand(\~{}A,B) @with .n at last [].s+(0,-2bp\_\_).}
   \par
   The function arguments are variable names or nested Boolean functions.
   Each unique variable {\sl var} causes an input point {\tt In}{\sl var}
   to be defined.  Preceding the variable by a {\tt \~{}} causes a NOT gate
   to be drawn at the input.
   The inputs are drawn in a row at the upper left by default.
   An {\tt L} in arg2 draws the inputs in a column at the left;
   {\tt R} reverses the order of the drawn inputs;
   {\tt V} scans the expression from right to left
   when listing inputs;
   {\tt M} draws the left-right mirror image of the diagram;
   and {\tt N} draws only the function tree without the input array.
   The inputs are labelled {\tt In1}, {In2}, \ldots and the function
   outputs are {\tt Out1}, {Out2}, \dots.
   Each variable {\sl var} corresponds also to one of the input array
   points with label {\tt In}{\sl var}.
   Setting {\tt offset=}{\sl value} displaces the
   drawn input list in order to disambiguate the input connections when {\tt L}
   is used.
   \par
   In the (possibly rare) case where one or more inputs of a normal function
   gate is to have a NOT-circle, an additional first argument of the function
   is inserted, of the form {\tt [{\sl charseq}]}, where {\sl charseq}
   is a string containing the characters {\tt P} for a normal input or
   {\tt N} for a negated input, the length of the string equal to the number
   of gate inputs.  Example:
   {\tt Autologix(Xor([PN],And(x,y),And(x,y)),LRV)}}%

\Letter{B}%
\macrodef{basename\_}{basename_}{({\sl string sequence, separator})}%
  {gen}%
  {Extract the rightmost name from a sequence of names separated by arg2
   (default dot ``.'')}%
\macrodef{battery}{battery}{(\linespec,{\sl n},R)}%
  {cct}%
  {n-cell battery: default 1 cell,
  R=reversed polarity\seesect{Twoterminal:}}%
\macrodef{b\_}{b_}{}%
  {gen}%
  {blue color value}%
\macrodef{b\_current}{b_current}{({\sl label}, {\sl pos}, In|Out, Start|End,
    {\sl frac})}%
  {cct}%
  {labelled branch-current arrow to {\sl frac} between branch end and body
    \seesect{Branchcurrent:}}%
\macrodef{beginshade}{beginshade}{({\sl gray value})}%
  {gen}%
  {begin gray shading, see {\tt shade}%
   e.g., {\tt beginshade(.5);} {\sl closed line specs}; {\tt endshade}}%
\macrodef{bell}{bell}{( U|D|L|R|{\sl degrees}, {\sl size})}%
  {cct}%
  {bell, {\sl In1} to {\sl In3} defined
   \seesect{Composite:}}%
\macrodef{below\_}{below_}{}%
  {gen}%
  {string position relative to current direction}%
\macrodef{Between\_}{Between_}{({\sl Pos1, Pos2,distance,}[R])}%
  {gen}%
  {Position {\sl distance} from {\sl Pos1} toward {\sl Pos2}.  If
   the fourth arg is {\tt R} then from {\sl Pos2} toward {\sl Pos1}.}%
\macrodef{binary\_}{binary_}{($n$, [$m$])}%
  {gen}%
  {binary representation of $n,$ left padded to $m$ digits if the second
   argument is nonblank}%
\macrodef{bisect}{bisect}{%
  ({\sl function name, left bound, right bound, tolerance, variable}))}%
  {dpictools}%
  {$\;\;$ Solve $\hbox{\sl function}(x) = 0$ by the method of 
   bisection.
   Like {\tt findroot} but uses recursion and is without a {\tt[]} box.
   The calculated value is assigned to the variable named in the
   last argument (\SR{Libraries:}). Example:\\
   {\tt define parabola \{ \$2 = (\$1)\char94{}2 - 1 \};\\
   bisect( parabola, 0, 2, 1e-8, x )}. }%
\macrodef{bi\_trans}{bi_trans}{(\linespec,L|R,{\sl chars},E)}%
  {cct}%
  { bipolar transistor, core left or right; chars:\\
   {\tt BU}: bulk line\\
   {\tt B}: base line and label\\
   {\tt S}: Schottky base hooks\\
   {\tt uEn|dEn}: emitters E0 to En\\
   {\tt uE|dE}: single emitter\\
   {\tt Cn|uCn|dCn}: collectors C0 to Cn; {\tt u} or {\tt d} add an arrow\\
   {\tt C}: single collector; {\tt u} or {\tt d} add an arrow\\
   {\tt G}: gate line and location\\
   {\tt H}: gate line;\\
   {\tt L}: L-gate line and location\\
   {\tt [d]D}: named parallel diode\\
   {\tt d}: dotted connection\\
   {\tt [u]T}: thyristor trigger line\\
   arg 4 = {\tt E}: envelope
    \seesect{Semiconductors:}}%
\macrodef{bi\_tr}{bi_tr}{(\linespec,L|R,P,E)}%
  {cct}%
  {left or right, N- or P-type bipolar transistor, without or with envelope 
    \seesect{Semiconductors:}}%
\macrodef{boxcoord}{boxcoord}{({\sl planar obj}, {\sl x fraction},
    {\sl y fraction})}%
  {gen}%
  {internal point in a planar object}%
\macrodef{boxdim}{boxdim}{({\sl name},h|w|d|v,{\sl default})}%
  {gen}%
  {Evaluate, e.g.\ {\sl name}{\tt \_w} if defined, else {\sl default\/}
   if given, else 0. {\tt v} gives sum of {\tt d} and {\tt h} values
     \seesect{Interaction:}}%
\macrodef{BOX\_gate}{BOX_gate}{({\sl inputs, output, swid, sht, label,
    attributes })}%
  {log}%
  {output=[{\tt P|N}], inputs=[{\tt P|N}]$\ldots$, sizes swid and sht
  in {\tt L\_unit}s (default {\tt AND\_wd} = 7)
    \seesect{Logicgates:}}%
\macrodef{bp\_\_}{bp__}{}%
  {gen}%
  {big-point-size factor, in scaled inches, ({\tt *scale/72})}%
\macrodef{bswitch}{bswitch}{(\linespec, [L|R],chars)}%
  {cct}%
  {pushbutton switch R=right orientation (default L=left);
     chars: O= normally open, C=normally closed }%
\macrodef{BUFFER\_gate}{BUFFER_gate}{(\linespec, [N|B],
  {\sl wid, ht,} [N|P]\char42, [N|P]\char42, [N|P]\char42,
    {\sl attributes})}%
  {log}%
  {basic buffer, dfault 1 input or as a 2-terminal element,\\
    arg2: {\tt N}: negated input, {\tt B}: box gate;\\
    arg 5: normal ({\tt P}) or negated {\tt N}) inputs labeled In1 
    \seesect{Logicgates:}}%
\macrodef{BUFFER\_gen}{BUFFER_gen}{({\sl chars,wd,ht},[N|P]*,[N|P]*,[N|P]*,
    {\sl attributes})}%
  {log}%
  {general buffer, {\sl chars:}\\
            {\tt T}: triangle,\\
            {\tt [N]O}: output location {\tt Out}
            ({\tt NO} draws circle {\tt N\_Out});\\
            {\tt [N]I, [N]N, [N]S, [N]NE, [N]SE} input locations;\\
            {\tt C}: centre location.\\
            Args 4-6 allow alternative
            definitions of respective {\tt In, NE,}
            and {\tt SE} argument sequences }%
\macrodef{BUF\_ht}{BUF_ht}{}%
  {log}%
  {basic buffer gate height in {\tt L\_unit}s}%
\macrodef{BUF\_wd}{BUF_wd}{}%
  {log}%
  {basic buffer gate width in {\tt L\_unit}s}%
\macrodef{buzzer}{buzzer}{( U|D|L|R|{\sl degrees, size,}[C])}%
  {cct}%
  {buzzer, {\sl In1} to {\sl In3} defined, C=curved
   \seesect{Composite:}}%

\Letter{C}%
\macrodef{cangle}{cangle}{({\sl Start, End},[d])}%
  {gen}%
  {Angle in radians of the sector at arg2 with arm ends given
   by arg1 and arg3 (degrees if arg4=d).}%
\macrodef{capacitor}{capacitor}{(\linespec,{\sl chars},R,
    {\sl height}, {\sl wid})}%
  {cct}%
  {capacitor, {\sl chars}:\\
    {\tt F} or blank: flat plate\\
    {\tt dF} flat plate with hatched fill\\
    {\tt C} curved-plate\\
    {\tt dC} curved-plate with variability arrowhead\\
    {\tt CP} constant phase element\\
    {\tt E} polarized boxed plates\\
    {\tt K} filled boxed plates\\
    {\tt M} unfilled boxes\\
    {\tt N} one rectangular plate\\
    {\tt P} alternate polarized\\
    {\tt +} adds a polarity sign\\
    {\tt +L} polarity sign to the left of drawing direction\\
    arg3:  {\tt R=}reversed polarity\\
    arg4: height (defaults {\tt F:} {\tt dimen\_}$/3$,
      {\tt C,P:} {\tt dimen\_}$/4$,
      {\tt E,K:} {\tt dimen\_}$/5$)\\
    arg5: wid (defaults {\tt F:} {\sl height}*0.3,
      {\tt C,P:} {\sl height}*0.4,
      {\tt CP:} {\sl height}*0.8,
      {\tt E,K:} {\sl height})
    \seesect{Twoterminal:}}%
\macrodef{case}{case}{({\sl i, alt1, alt2, \ldots})}%
  {dpictools}%
  {$\;\;$ Case statement for dpic; execute alternative {\sl i.}
   Example: {\tt case( 2, x=5, x=10, x=15 )} sets {\tt x} to 10.
   Note: this is a macro so {\tt \${}n} refers to the {\sl n}-th
	 argument of {\tt case}.}%
\macrodef{cbreaker}{cbreaker}{(\linespec, L|R, D|Th|TS, body name)}%
  {cct}%
  {circuit breaker to left or right, {\tt D}: with dots; {\tt Th}: thermal;
   {\tt TS}: squared thermal; default body bounding box name is
     {\sl Br}\seesect{Twoterminal:}}%
\macrodef{ccoax}{ccoax}{(at {\sl location}, M|F, {\sl diameter, attributes})}%
  {cct}%
  {coax connector, {\tt M}: male, {\tt F}: female
    \seesect{Composite:}}%
\macrodef{cct\_init}{cct_init}{}%
  {cct}%
  {initialize circuit-diagram environment (reads {\tt libcct.m4})}%
\macrodef{centerline\_}{centerline_}{({\sl linespec, thickness{\tt|}color,
   minimum long dash len, short dash len, gap len}}%
  {gen}%
  {Technical drawing centerline}%
\macrodef{c\_fet}{c_fet}{(\linespec,R,P)}%
  {cct}%
  {left or right, plain or negated pin simplified MOSFET}%
\macrodef{Cintersect}{Cintersect}{({\sl Pos1, Pos2, rad1, rad2,} [R])}%
  {gen}%
  {Upper (lower if arg5={\tt R}) intersection of circles at
    {\sl Pos1} and {\sl Pos2}, radius {\sl rad1} and {\sl rad2}}
\macrodef{clabel}{clabel}{( {\sl label}, {\sl label}, {\sl label},
    {\sl relative position}, {\sl block name})}%
  {cct}%
  {Triple label along the drawing axis of the body of an element in the
    current direction \seesect{Labels:}.  Labels are placed at the
    beginning, centre, and end of the last {\tt []} block (or a {\tt
    []} block named or enumerated in arg5). Each label is treated as
    math by default, but is copied literally if it is in double quotes
    or sprintf.  {\sl Arg4} can be {\tt above,} {\tt below,} {\tt left,}
    or {\tt right} to supplement the default relative position.}%
\macrodef{cm\_\_}{cm__}{}%
  {gen}%
  {absolute centimetres}%
\macrodef{cmyktorgb}{cmyktorgb}{({\sl c, m, y, k, r, g, b})}%
  {dpictools}%
  {$\;\;$ cmyk values in percent, i.e., 0 to 100, to rgb.}%
\macrodef{consource}{consource}{(\linespec,V|I|v|i,R,{\sl attributes})}%
  {cct}%
  {voltage or current controlled source with alternate forms;
   {\tt R}: reversed polarity\seesect{Twoterminal:}}%
\macrodef{ColoredV}{ColoredV}{%
   (box|circle|ellipse,({\sl r,g,b})|(({\sl colorseq}))[:{\sl nlines}],%
     {\sl attributes})}%
  {gen}%
  {box (default), circle, or ellipse in a {\tt []} block.
   If arg2 is blank
   then all formatting is in arg3; if parenthesized r,g,b, the object
   is shaded top to bottom white to the specified rgb color; if a
   double-parenthesized {\sl colorseq} then the {\sl colorseq}
   defines the internal
   shading top to bottom.  A {\sl colorseq} is of the form
   {\sl 0,r0,g0,b0,\\
    frac1,r1,g1,b1,\\
    frac2,r2,g2,b2,\\
   \ldots\\
    1,rn,gn,bn}\\
   with $0 < \hbox{\sl frac1} < \hbox{\sl frac2} \ldots 1$.
   The number of {\sl colorseq} lines can be specified with the colon
   (default {\tt height/(line thickness)*2)}.
   Examples: {\tt ColoredV(circle,(1,0,0));
   ColoredV(ellipse,(1,0.04,1),wid 0.75 ht 1 \textbackslash\\
    $\;$ outlined "magenta" "Goodbye");
   ColoredV(box,((0,1,1,0, 1,0,0,1)):50, \\
    $\;$ outlined "blue" rad 0.1).} }%
\macrodef{contact}{contact}{({\sl chars})}%
  {cct}%
  {single-pole contact:
   {\tt O}:  normally open\\
   {\tt C}: normally closed (default)\\
   {\tt I}: open circle contacts\\
   {\tt P}: three position\\
   {\tt R}: right orientation\\
   {\tt T}: T contacts\\
   {\tt U}: U contacts
   \seesect{Composite:}}%
\macrodef{contacts}{contacts}{({\sl count, chars})}%
  {cct}%
  {multiple ganged single-pole contacts:
   {\tt P}: three position\\
   {\tt O}: normally open\\
   {\tt C}: normally closed\\
   {\tt D}: dashed ganging line over contact armatures
   {\tt I}: open circle contacts\\
   {\tt R}: right orientation\\
   {\tt T}: T contacts\\
   {\tt U}: U contact lines parallel to drawing direction
   \seesect{Composite:}}%
\macrodef{contline}{contline}{({\sl line})}%
  {gen}%
  {evaluates to {\tt continue}
    if processor is {\bf dpic}, otherwise to first arg (default {\tt line})}%
\macrodef{copy3}{copy3}{({\sl vector1,vector2})}%
  {dpictools}%
  {$\;\;$ Copy vector1 into vector named by arg2.}%
\macrodef{copythru}{copythru}{({\sl dpic macro name,} "{\sl file name}")}%
  {dpictools}%
  {$\;\;$ Implements the gpic {\tt copy {\sl filename} thru {\sl macro-name}}
   for file data separated by commas, spaces, or tabs.}%
\macrodef{corner}{corner}{({\sl line thickness,attributes,turn radians})}%
  {gen}%
  { Mitre (default filled square) drawn at end of last line or at a
    given position.
    arg1 default: current line thickness;
    arg2: e.g. {\tt outlined} {\sl string}; if arg2 starts with
    {\tt at} {\sl position} then a manhattan (right-left-up-down) corner
    is drawn;
    arg3= radians (turn angle, +ve is ccw, default $\pi/2$).
    The corner is enclosed in braces
    in order to leave {\tt Here} unchanged unless arg2 begins with {\tt at}%
    \seesect{Corners:}}%
\macrodef{Cos}{Cos}{({\sl integer})}%
  {gen}%
  {cosine function, {\sl integer\/} degrees}%
\macrodef{cosd}{cosd}{({\sl arg})}%
  {gen}%
  {cosine of an expression in degrees}%
\macrodef{Cosine}{Cosine}{( {\sl amplitude}, {\sl freq}, {\sl time},
    {\sl phase} )}%
  {gen}%
  {function $a\times\cos(\omega t + \phi)$ }%
\macrodef{cross3}{cross3}{({\sl vec1, vec2, vec3})}%
  {dpictools}%
  {$\;\;$ The 3-vector cross product
    $\hbox{\sl vec3} = \hbox{\sl vec1} \times \hbox{\sl vec2}$.}%
\macrodef{cross3D}{cross3D}{({\sl x1,y1,z1,x2,y2,z2})}%
  {3D}%
  {cross product of two triples}%
\macrodef{cross}{cross}{(at {\sl location, size}|{\sl keys})}%
  {gen}%
  {Plots a small cross.  The possible key-value pairs are:
    {\tt size={\sl expr};},
    {\tt line={\sl attributes};} }%
\macrodef{crossover}{crossover}{(\linespec, [L|R][:{\sl line attributes}],
  {\sl Linename1, Linename2,} $\ldots$)}%
  {cct}%
  {line jumping left or right over ordered named
   lines\seesect{Semiconductors:}}%
\macrodef{crosswd\_}{crosswd_}{}%
  {gen}%
  {cross dimension}%
\macrodef{csdim\_}{csdim_}{}%
  {cct}%
  {controlled-source width}%

\Letter{D}%
\macrodef{dabove}{dabove}{(at {\sl location})}%
  {darrow}%
  {above (displaced dlinewid/2)}%
\macrodef{dac}{dac}{({\sl width,height,nIn,nN,nOut,nS})}%
  {cct}%
  {DAC with defined width, height, and number of inputs {\tt In$i$},
    top terminals {\tt N$i$}, ouputs {\tt Out$i$},
    and bottom terminals~{\tt S$i$} \seesect{Logicgates:}}%
\macrodef{Darc}{Darc}{({\sl center position},
  {\sl radius}, {\sl start radians}, {\sl end radians},
  {\sl parameters})}%
  {darrow}%
  {Wrapper for {\tt darc}.
   CCW arc in {\tt dline} style, with closed ends or (dpic only) arrowheads.
  Semicolon-separated {\sl parameters}:\\
  {\tt thick=}{\sl value};\\
  {\tt wid=}{\sl value};\\
  {\tt ends=} {\sl x}{\tt -},
  {\tt -}{\sl x}, {\sl x}{\tt -}{\sl x}, {\tt ->}, {\sl x}{\tt ->},
  {\tt <-}, {\tt <-}{\sl x}, {\tt <->},
  where {\sl x} is {\tt |} or (half-thickness line) {\tt !}.}%
\macrodef{darc}{darc}{({\sl center position},
  {\sl radius}, {\sl start radians}, {\sl end radians}, {\sl dline thickness},
  {\sl arrowhead wid}, {\sl arrowhead ht},
  {\sl end symbols}, {\sl outline attributes}, {\sl inner attributes})}%
  {darrow}%
  {See also {\tt Darc}.
   CCW arc in {\tt dline} style, with closed ends or (dpic only) arrowheads.
  Permissible {\sl end symbols}:
  {\sl x}{\tt -},
  {\tt -}{\sl x}, {\sl x}{\tt -}{\sl x}, {\tt ->}, {\sl x}{\tt ->},
  {\tt <-}, {\tt <-}{\sl x}, {\tt <->}
  where {\sl x} is {\tt |} or (half-thickness line) {\tt !}.
  An inner arc is drawn overlaying the outer arc.
  Example: {\tt darc(,{,},{,},{,},{,}outlined "red",outlined "yellow")}.}%
\macrodef{Darlington}{Darlington}{(L|R,{\sl chars})}%
  {cct}%
  {Composite Darlington pair Q1 and Q2 with internal locations E, B, C;
   Characters in {\sl arg2:}\\
   E= envelope\\
   P= P-type\\
   B1= internal base lead\\
   D= damper diode\\
   R1= Q1 bias resistor; E1= ebox\\
   R2= Q2 bias resistor; E1= ebox\\
   Z= zener bias diode 
    \seesect{Semiconductors:}}%
\macrodef{darrow}{darrow}{(\linespec,
  t, t, {\sl width}, {\sl arrowhd wd}, {\sl arrowhd ht}, {\sl parameters}, 
    {\sl color attributes})}%
  {darrow}%
  {See also {\tt Darrow}.
   Double arrow, truncated at beginning (arg2={\tt t}) or
    end (arg3={\tt t}), specified sizes,
    with arrowhead and optional closed stem. The parameters (arg7) are
    {\tt {\sl x}-} or {\tt ->} or {\tt {\sl x}->} or {\tt <-} or
    {\tt <-{\sl x}}
    or {\tt <->} where {\sl x} is {\tt |} or {\tt !}.
    The {\tt !-} or {\tt -!} parameters close
    the stem with half-thickness lines to simplify butting to other objects.
    The color attributes are, e.g.,
    {\tt outlined "{\sl color}" shaded "{\sl color}"}.
    Example:
    {\tt linethick=5; darrow(down\_\ 2,{,},0.5,0.75,0.75,|,outlined "red")}. }%
\macrodef{Darrow}{Darrow}{(\linespec, {\sl parameters})}%
  {darrow}%
  {Wrapper for {\tt darrow}.
  Semicolon-separated {\sl parameters}:\\
    {\tt S;}, {\tt E;} truncate at start or end by dline thickness/2\\
    {\tt thick=}{\sl val};   (total thicknes, ie width)\\
    {\tt wid=}{\sl val};     (arrowhead width)\\
    {\tt ht=}{\sl val};      (arrowhead height)\\
    {\tt ends=}
      {\sl x}{\tt -}{\sl x} or
      {\tt -}{\sl x} or
      {\sl x}{\tt -} where {\sl x} is {\tt !} (half-width line)
       or {\tt |} (full-width line).\\
    Examples:
    {\tt define(`dfillcolor',`1,0.85,0') linethick=5;
     rgbdraw(1,0,0,Darrow(down\_\ 2,thick=0.5; wid=0.75; ht=0.75; ends=|->))},
     which is equivalent to
    {\tt Darrow(down\_\ 2,thick=0.5; wid=0.75; ht=0.75; \
      ends=|->; outline="red")}.  }%
\macrodef{darrow\_init}{darrow_init}{}%
  {darrow}%
  {Initialize darrow drawing parameters (reads library file {\tt darrow.m4})}%
\macrodef{dashline}{dashline}{(\linespec,{\sl thickness}|{\sl color}|<->,
                                  {\sl dash len, gap len},G)}%
  {gen}%
  {dashed line with dash at end ({\tt G} ends with gap)}%
\macrodef{dbelow}{dbelow}{(at {\sl location})}%
  {darrow}%
  {below (displaced dlinewid/2)}%
\macrodef{dcosine3D}{dcosine3D}{({\sl i,x,y,z})}%
  {3D}%
  {extract i-th entry of triple x,y,z}%
\macrodef{DCsymbol}{DCsymbol}{(at {\sl position, len, ht,}%
  U|D|L|R|{\sl degrees}) }%
  {cct}%
  {A DC symbol (a dashed line below a solid line).
   The current drawing direction is default, otherwise Up, Down,
   Left, Right, or at {\sl degrees} slant; e.g., 
   {\tt source(up\_ dimen\_); $\lbrace$ DCsymbol(at last [],{},{},R) $\rbrace$}
   \seesect{Twoterminal:} }%
\macrodef{DefineCMYKColor}{DefineCMYKColor}{({\sl color-name, c, m, y, k})}%
  {dpictools}%
  {$\;\;$ Like {\tt DefineRGBColor} but takes arguments in percent, i.e.,
    the range $[0,100].$ Define dpic macro {\sl colorname}
    according to the postprocessor specified by dpic command-line option.
    The macro evaluates to a string.}%
\macrodef{DefineHSVColor}{DefineHSVColor}{({\sl color-name, h, s, v})}%
  {dpictools}%
  {$\;\;$ Like {\tt DefineRGBColor} but takes argument {\sl h} in the
    range $[0,360],$ {\sl s} in $[0,1],$ and {\sl v} in $[0,1].$
    Define dpic macro {\sl colorname}
    according to the postprocessor specified by dpic command-line option.
    The macro evaluates to a string.}%
\macrodef{DefineRGBColor}{DefineRGBColor}{({\sl color-name, r, g, b})}%
  {dpictools}%
  {$\;\;$ Arguments are in the range 0 to 1. Define dpic macro {\sl colorname}
    according to the postprocessor specified by dpic command-line option.
    The macro evaluates to a string.}%
\macrodef{definergbcolor}{definergbcolor}{({\sl color-name, r, g, b})}%
  {gen}%
  {Arguments are in the range 0 to 1. Define color name
    according to the postprocessor.  Similar to dpictools {\tt DefineRGBColor}
    but the color name is an m4 macro, not a string.}%
\macrodef{delay}{delay}{(\linespec,{\sl size},{\sl attributes})}%
  {cct}%
  {delay element\seesect{Twoterminal:}}%
\macrodef{delay\_rad\_}{delay_rad_}{}%
  {cct}%
  {delay radius}%
\macrodef{deleminit\_}{deleminit_}{}%
  {darrow}%
  {sets drawing direction for dlines}%
\macrodef{Deltasymbol}{Deltasymbol}{(at {\sl position},keys,
    U|D|L|R|{\sl degrees}) (default {\tt U} for up)}%
  {cct}%
  {Delta symbol for power-system diagrams.
   {\sl keys:} {\tt size={\sl expression;}}
    {\tt type=C|O} (default {\tt C} for closed;
      {\tt O} draws an ``open'' symbol); }%
\macrodef{Demux}{Demux}{({\sl n},{\sl label},
  {\tt [L][B|H|X][N[{\sl n}]|S[{\sl n}]][[N]OE],
   {\sl wid},{\sl ht},{\sl attributes}})}%
  {log}%
  {binary demultiplexer, $n$ inputs\\
    {\tt L} reverses input pin numbers\\
    {\tt B} displays binary pin numbers\\
    {\tt H} displays hexadecimal pin numbers\\
    {\tt X} do not print pin numbers\\
    {\tt N[{\sl n}]} puts Sel or Sel$0$ .. Sel$n$ at the top
    (i.e., to the left of the drawing direction)\\
    {\tt S[{\sl n}]} puts the Sel inputs at the bottom (default)
    {\tt OE} ({\tt N=}negated) {\tt OE} pin
    \seesect{Logicgates:}}%
\macrodef{dend}{dend}{(at {\sl location},
  {\sl line thickness}{\tt |}{\sl attributes})}%
  {darrow}%
  {Close (or start) double line (Note specifying {\tt dends=} for {\tt Dline}
   is a similar function.  Arg2 is dline thickness or atributes:\\
   {\tt thick={\sl expression};} (dline thickness in drawing units)\\
   {\tt outline=({\sl r,g,b})|"{\sl color}";}}%
\macrodef{d\_fet}{d_fet}{(\linespec,R,P,E|S)}%
  {cct}%
  {left or right, N or P depletion MOSFET, envelope or simplified
    \seesect{Semiconductors:}}%
\macrodef{dfillcolor}{dfillcolor}{}%
  {darrow}%
  {dline fill color (default white)}%
\macrodef{diff3}{diff3}{({\sl vec1, vec2, vec3})}%
  {dpictools}%
  {$\;\;$ The 3-vector subtraction
    $\hbox{\sl vec3} = \hbox{\sl vec1} - \hbox{\sl vec2}$.}%
\macrodef{dfitcurve}{dfitcurve}{({\sl Name, n, linetype, m})}%
  {dpictools}%
  {$\;\;$ Draw a spline through {\sl Name}{\tt [m],} \ldots {\sl Name}{\tt [n]}
   with attribute {\sl linetype} {\tt dotted,} for example.
   The calculated control points {\sl P[i]} satisfy approximately:
   {\sl P[0] = V[0];} {\sl P[i-1]/8 + P[i]*3/4 + P[i+1]/8 = V[i];}
   {\sl P[n] = V[n].}  See m4 macro {\tt fitcurve.}}%
\macrodef{dfitpoints}{dfitpoints}{({\sl V,n,m,P,mp})}%
  {dpictools}%
  {$\;\;$ Compute the control locations {\tt P[mP], P[mP+1]...} for
  the spline passing throught points {\tt V[m]...V[n].}  Used by
  macro {\tt dfitcurve}.}%
\macrodef{diff3D}{diff3D}{({\sl x1,y1,z1,x2,y2,z2})}%
  {3D}%
  {difference of two triples}%
\macrodef{diff\_}{diff_}{({\sl a},{\sl b})}%
  {gen}%
  {difference function}%
\macrodef{dimen\_}{dimen_}{}%
  {cct}%
  {size parameter for scaling circuit element bodies \seesect{Circuitscaling:}}%
\macrodef{dimension\_}{dimension_}{(\linespec,{\sl offset},{\sl label},
    D|H|W|{\sl blank width},{\sl tic offset},{\sl arrowhead })}%
  {gen}%
  {macro for dimensioning diagrams; {\sl arrowhead=}{\tt -> | <-}}%
\macrodef{diode}{diode}{(\linespec,
B|b|CR|D|G|L|LE[R]|P[R]|S|Sh|T|U|V|v|w|Z|chars,
[R][E])}%
  {cct}%
  {diode:
   {\tt B}: bi-directional\\
   {\tt b}: bi-directional with outlined zener crossbar\\
   {\tt CR}: current regulator\\
   {\tt D}: diac\\
   {\tt G}: Gunn\\
   {\tt L}: open form with centre line\\
   {\tt LE[R]}: LED [right]\\
   {\tt P[R]}: photodiode [right]\\
   {\tt S}: Schottky\\
   {\tt Sh}: Shockley\\
   {\tt T}: tunnel\\
   {\tt U}: limiting\\
   {\tt V}: varicap\\
   {\tt v}: varicap (curved plate)\\
   {\tt w}: varicap (reversed polarity)\\
   {\tt Z}: zener\\
   appending {\tt K} to arg 2 draws open arrowheads;
   arg 3: {\tt R}: reversed polarity, {\tt E}: enclosure \seesect{Twoterminal:}}%
%\macrodef{DIP}{DIP}{({\sl pin count, attributes})}%
%  {log}%
%  {Dual in-line package diagram. Default pin count = 8.
%   Arg2 ({\sl attributes})= semicolon-separated list of optional terms:
%   {\tt bodywid=}{\sl expr} (default 0.25$\,$in${} {}$5{\tt *L\_unit}),
%   {\tt bodylen=}{\sl expr} (default {\sl pin count} $\times$ {\sl pin pitch}),
%   {\tt pinpitch=}{\sl expr} (default 0.1),
%   {\tt pinwid=}{\sl expr} (default 0.06),
%   {\tt pinlen=}{\sl expr} (default 0.05),
%   {\tt direct=U|D|L|R} (default {\tt U} for up),
%   {\tt type=I|Q} (default {\tt I}; Q=pins of alternating length)
%   \seesect{Logicgates:}}%
\macrodef{dir\_}{dir_}{}%
  {darrow}%
  {used for temporary storage of direction by darrow macros}%
\macrodef{distance}{distance_}{({\sl Position 1}, {\sl Position2})}%
  {gen}%
  {distance between named positions}%
\macrodef{distance}{distance}{({\sl position}, {\sl position})}%
  {gen}%
  {distance between positions}%
\macrodef{dlabel}{dlabel}{({\sl long},{\sl lat},{\sl label},{\sl
     label},{\sl label},{\sl chars})}%
  {cct}%
  {general triple label; {\sl chars:}
    $X$ displacement {\tt {\sl long, lat}} with respect to the drawing
      direction is from the centre of the last
      line rather than the centre of the last {\tt [ ]};
    L,R,A,B align labels ljust, rjust, above,
      or below (absolute) respectively  \seesect{Labels:}}%
\macrodef{dleft}{dleft}{({\tt at} {\sl position}, {\sl line thickness},
      {\sl attributes})}%
  {darrow}%
  {Double line left turn 90 degrees.  Attributes can be
    {\tt outline=({\sl r, g, b})|"{\sl color}";}
    {\tt innershade=({\sl r, g, b})|"{\sl color}";}
    where rgb values in parentheses or a defined color is specified.}%
\macrodef{Dline}{Dline}{(\linespec, {\sl parameters})}%
  {darrow}%
  {Wrapper for {\tt dline}.
  The semicolon-separated {\sl parameters} are:\\
    {\tt S;}, {\tt E;} truncate at start or end by dline thickness/2;\\
    {\tt thick={\sl val};}   (total thicknes, ie width);\\
    {\tt outline={\sl color};} (e.g., {\tt "red"} or {\tt (1,0,0)}),\\
    {\tt innershade={\sl color};} (e.g., {\tt (0,1,1)} or {\tt "cyan"}),\\
    {\tt name={\sl Name};},\\
    {\tt ends=}%
      {\sl x}{\tt -}{\sl x} or
      {\tt -}{\sl x} or
      {\sl x}{\tt -} where {\sl x} is {\tt !} (half-width line)
       or {\tt |;} (full-width line).}%
\macrodef{dline}{dline}{(\linespec,t,t,{\sl width},{\sl parameters})}%
  {darrow}%
  {See also {\tt Dline}.
   Double line, truncated by half width at either end, closed
    at either or both ends.
   {\sl parameters=}
      {\sl x}{\tt -}{\sl x} or
      {\tt -}{\sl x} or
      {\sl x}{\tt -} where {\sl x} is {\tt !} (half-width line)
       or {\tt |} (full-width line).}%
\macrodef{dlinewid}{dlinewid}{}%
  {darrow}%
  {width of double lines}%
\macrodef{dljust}{dljust}{(at {\sl location})}%
  {darrow}%
  {ljust (displaced dlinewid/2)}%
\macrodef{dna\_}{dna_}{}%
  {cct}%
  {internal character sequence that specifies which subcomponents are drawn}%
\macrodef{dn\_}{dnx}{}%
  {gen}%
  {down with respect to current direction}%
%\macrodef{dnm\_}{dnm_}{}%
%  {cct}%
%  {similar to dna\_}%
\macrodef{dot3}{dot3}{({\sl vec1, vec2})}%
  {dpictools}%
  {$\;\;$ Expands to the dot (scalar) product of the two 3-vector arguments:
   $(\${}1[1]\cdot\${}2[1] + \${}1[2]\cdot\${}2[2] + \${}1[3]\cdot\${}2[3])$.}%
\macrodef{dot3D}{dot3D}{({\sl x1,y1,z1,x2,y2,z2})}%
  {3D}%
  {dot product of two triples}%
\macrodef{dot}{dot}{(at {\sl location},{\sl radius}|{\sl keys},{\sl fill})}%
  {gen}%
  {Filled circle (third arg= gray value: 0=black, 1=white). The possible
   key-value pairs are:
    {\tt rad={\sl expr};} and
    {\tt circle={\sl attributes};} }%
\macrodef{dotrad\_}{dotrad_}{}%
  {gen}%
  {dot radius}%
\macrodef{down\_}{down_}{}%
  {gen}%
  {sets current direction to down \seesect{Placing:}}%
\macrodef{dpquicksort}{dpquicksort}{({\sl array name, lo, hi, ix})}%
  {dpictools}%
  {$\;\;$ Given array {\sl a[lo:hi]} and index array {\sl ix[lo:hi] =
    lo, lo+1, lo+2,\ldots hi}, sort {\sl a[lo:hi]} and do identical exchanges
    on {\sl ix}.}%
\macrodef{dprot}{dprot}{({\sl radians, x, y})}%
  {dpictools}%
  {$\;\;$ Evaluates to a rotated pair (see m4 {\tt rot\_}).}%
\macrodef{dprtext}{dprtext}{({\sl degrees, text})}%
  {dpictools}%
  {$\;\;$ Rotated PStricks or pgf text in a {\tt []} box.}%
\macrodef{dright}{dright}{({\tt at} {\sl position}, {\sl line thickness},
      {\sl attributes})}%
  {darrow}%
  {Double line right turn 90 degrees.  Attributes can be\\
    {\tt outline=({\sl r, g, b})|"{\sl color}";}\\
    {\tt innershade=({\sl r, g, b})|"{\sl color}";}\\
    where rgb values in parentheses or a defined color is specified.}%
\macrodef{drjust}{drjust}{(at {\sl location})}%
  {darrow}%
  {rjust (displaced dlinewid/2)}%
\macrodef{dswitch}{dswitch}{(\linespec, L|R, W[ud]B {\sl chars},
     {\sl attributes})}%
  {cct}%
  {
                                 Comprehensive IEEE-IEC single-pole switch:
                                 arg2={\tt R}: orient to the right of
                                     drawing dir
                                 arg4 is a key-value sequence for the body of
                                 {\tt GC} and {\tt GX} options:
                                   {\tt GC} keys: {\tt diam, circle;}%
                                   {\tt GX} keys: {\tt lgth, wdth, box, text}.
\par
                                 Arg 3:
                                   blank means {\tt WB} by default\\
                                   {\tt B}: contact blade open\\
                                   {\tt Bc}: contact blade closed\\
                                   {\tt Bm}: mirror blade\\
                                   {\tt Bo}: contact blade more widely open\\
                                   {\tt dB}: contact blade to the right of
                                     direction\\
                                   {\tt Cb}: circuit breaker function 
                                     (IEC S00219)\\
                                   {\tt Co}: contactor function (IEC S00218)\\
                                   {\tt C}: external operating mechanism\\
                                   {\tt D}: circle at contact and hinge
                                      ({\tt dD} = hinge only,
                                       {\tt uD}: contact only)\\
                                   {\tt DI}: Disconnector, isolator
                                     (IEC S00288)\\
                                   {\tt E}: emergency button\\
                                   {\tt EL}: early close (or late open)\\
                                   {\tt LE}: late close (or early open)\\
                                   {\tt F}: fused\\
                                   {\tt GC}: disk control mechanism, attribs:
                                    {\tt diam={\sl expr};
                                         circle={\sl circle attribs};}\\
                                   {\tt GX}: box control mechanism, attribs:
                                    {\tt lgth={\sl expr}; wdth={\sl expr};
                                         box={\sl box attr};}
                                    {\tt text={\sl char};}\\
                                   {\tt H}: time delay closing\\
                                   {\tt uH}: time delay opening\\
                                   {\tt HH}: time delay opening and closing\\
                                   {\tt K}: vertical closing contact line
                                     use {\tt WdBK} for a normally-closed
                                     switch\\
                                   {\tt L}: limit\\
                                   {\tt M}: maintained (latched)\\
                                   {\tt MM}: momentary contact on make\\
                                   {\tt MR}: momentary contact on release\\
                                   {\tt MMR}: momentary contact on make and
                                     release\\
                                   {\tt O}: hand operation button\\
                                   {\tt P}: pushbutton\\
                                   {\tt Pr[T|M]}: proximity (touch-sensitive or
                                       magnetically controlled)\\
                                   {\tt R}: time-delay operating arm\\
                                   {\tt Sd}: Switch-disconnector\\
                                   {\tt Th}: thermal control linkage\\
                                   {\tt Tr}: tripping\\
                                   {\tt W}: baseline with gap\\
                                   {\tt Y}: pull switch\\
                                   {\tt Z}: turn switch
   \seesect{Twoterminal:}}%
\macrodef{dtee}{dtee}{([L|R], {\sl line thickness}, {\sl attributes})}%
  {darrow}%
  {Double arrow tee junction with tail to left,
   right, or (default) back along current direction, leaving the current
   location at the tee centre; e.g.,
   {\tt dline(right\_,{,}t); dtee(R);
    \{ darrow(down\_,t) \}; darrow(right\_,t)}.
   The attributes are
     {\tt thick={\sl expr};} (line thickness in drawing units),
     {\tt innershade=({\sl r,g,b})|"{\sl color}";}
     {\tt outline=({\sl r,g,b})|"{\sl color}";}. }%
\macrodef{dtor\_}{dtor_}{}%
  {gen}%
  {degrees to radians conversion constant}%
\macrodef{dturn}{dturn}{({\sl degrees ccw},{\sl line thickness},
  {\sl attributes})}%
  {darrow}%
  {Tturn dline arg1 degrees left (ccw).
  Attributes can be
    {\tt outline=({\sl r, g, b})|"{\sl color}";}
    {\tt innershade=({\sl r, g, b})|"{\sl color}";}
    where rgb values in parentheses or a defined color is specified.}%

\Letter{E}%

\macrodef{earphone}{earphone}{( U|D|L|R|{\sl degrees, size})}%
  {cct}%
  {earphone, {\sl In1} to {\sl In3} defined
   \seesect{Composite:}}%
\macrodef{ebox}{ebox}{(\linespec,{\sl lgth},{\sl wdth},{\sl fill value},
  {\sl box attributes})}%
  {cct}%
  { two-terminal box element with adjustable dimensions and fill
   value 0 (black) to 1 (white). {\sl lgth} (length) and {\sl wdth} (width)
   are relative
   to the direction of \linespec. Alternatively, argument 1 is the
   \linespec\ and argument 2 is a semicolon-separated sequence of key=value
   terms.  The possible keys are {\tt lgth, wdth, text, box},
   e.g., {\tt lgth=0.2; text=\char34{}XX\char34;
     box=shaded \char34{}green\char34}\seesect{Twoterminal:}}%
\macrodef{E\_\_}{E__}{}%
  {gen}%
  {the constant $e$}%
\macrodef{e\_}{e_}{}%
  {gen}%
  {.e relative to current direction}%
\macrodef{e\_fet}{e_fet}{(\linespec,R,P,E|S)}%
  {cct}%
  {left or right, N or P enhancement MOSFET, normal
   or simplified, without or with envelope
    \seesect{Semiconductors:}}%
\macrodef{elchop}{elchop}{({\sl Name1,Name2})}%
  {gen}%
  {{\tt chop} for ellipses: evaluates to {\tt chop} $r$ where $r$ is
    the distance from the centre of ellipse Name1 to the intersection of
    the ellipse with a line to location Name2;
    e.g., {\tt line from A to E elchop(E,A)}}%
\macrodef{eleminit\_}{eleminit_}{(\linespec)}%
  {cct}%
  {internal line initialization}%
\macrodef{elen\_}{elen_}{}%
  {cct}%
  {default element length}%
\macrodef{ellipsearc}{ellipsearc}{({\sl width, height, startangle, endangle,
   rotangle, {\tt cw|ccw}, line attributes})}%
  {gen}%
  {Arc of a rotated ellipse in a {\tt [ ]} block. Angles are in radians.
   Arg5 is the angle of the width axis;
   e.g., {\tt ellipsearc(2,1,0,pi\_,pi\_/4,{},dashed ->)}.  Internal locations
   are {\tt Start, End, C} (for centre). }%
\macrodef{em\_arrows}{em_arrows}{({\sl type}|{\sl keys, angle, length})}%
  {cct}%
  { Radiation arrows: {\sl type} {\tt N|I|E [D|T]}:\\
  {\tt N}: nonionizing,\\
  {\tt I}: ionizing,\\
  {\tt E}: simple;\\
  {\tt D}: dot on arrow stem;\\
  {\tt T}: anchor tail;\\
  {\sl keys:} {\tt type=}{\sl chars} as above;\\
  {\tt angle=}{\sl degrees}; (absolute direction)\\
  {\tt lgth=}{\sl expr};\\
  {\tt sep=}{\sl expr}; arrow separation
  \seesect{Twoterminal:}}%
\macrodef{endshade}{endshade}{}%
  {gen}%
  {end gray shading, see {\tt beginshade}}%
\macrodef{Equidist3}{Equidist3}{({\sl Pos1, Pos2, Pos3, Result, distance})}%
  {gen}%
  {Calculates location named {\sl Result} equidistant from the first three
   positions, i.e.\ the centre of the circle passing through the three
   positions.  If arg5 is nonblank, it is returned equated to the radius.}%
\macrodef{expe}{expe}{}%
  {gen}%
  {exponential, base $e$}%

\Letter{F}%

\macrodef{f\_box}{f_box}{({\sl boxspecs},{\sl text},{\sl expr1},$\cdots$)}%
  {gen}%
  {like {\tt s\_box} but the text is overlaid on a box of identical size.
   If there is only one argument then the default box
   is invisible and filed white
    \seesect{Interaction:}}%
\macrodef{Fector}{Fector}{({\sl x1,y1,z1,x2,y2,z2})}%
  {3D}%
  {vector projected on current view plane with top face
   of 3-dimensonal arrowhead normal to x2,y2,z2 }%
\macrodef{Fe\_fet}{Fe_fet}{(\linespec,R,{\sl chars})}%
  {cct}%
  {FET with superimposed ferroelectric symbol. Args 1 to 3 are as for
   the {\tt mosfet} macro
    \seesect{Semiconductors:}}%
\macrodef{FF\_ht}{FF_ht}{}%
  {cct}%
  {flipflop height parameter in {\tt L\_unit}s}%
\macrodef{FF\_wid}{FF_wid}{}%
  {cct}%
  {flipflop width parameter in {\tt L\_unit}s}%
\macrodef{fill\_}{fill_}{({\sl number})}%
  {gen}%
  {fill macro, 0=black, 1=white\seesect{Semiconductors:}}%
\macrodef{findroot}{findroot}{%
  ({\sl function name, left bound, right bound, tolerance, variable}))}%
  {dpictools}%
  {$\;\;$ Solve $\hbox{\sl function}(x) = 0$ by the method of 
   bisection.  The calculated value is assigned to the variable named in the
   last argument (\SR{Libraries:}). Example:
   {\tt define parabola \{ \$2 = (\$1)\char94{}2 - 1 \};
   findroot( parabola, 0, 2, 1e-8, x )}. }%
\macrodef{fitcurve}{fitcurve}{(V, n, {\sl attributes,} m (default 0))}%
  {gen}%
  {Draw a spline through positions V[m], \ldots V[n]: Works only with dpic.}%
\macrodef{FlipFlop}{FlipFlop}{(D|T|RS|JK,{\sl label},{\sl boxspec},%
 {\sl pinlength})}%
  {log}%
  {flip-flops,
  {\sl boxspec} e.g.,\ ht x wid y \seesect{Logicgates:}}%
\macrodef{FlipFlopX}{FlipFlopX}{({\sl boxspec, label, leftpins, toppins,
   rightpins, bottompins, pinlength})}%
  {log}%
  {General flipflop.
   Arg 1 modifies the box (labelled Chip) default specification.
   Each of args 3 to 6 is null or a string of {\sl pinspecs}
   separated by semicolons ({\tt;}).  A {\sl Pinspec} is either empty
   or of the form
   {\tt[}{\sl pinopts}{\tt]:[}{\sl label}{\tt[:}{\sl Picname}{\tt]]}.
   The first colon draws the pin.
   Pins are placed top to bottom or left to right along the box edges with
   null {\sl pinspecs} counted for placement. Pins are named by side and number
   by default; eg {\tt W1, W2, ..., N1, N2, ..., E1, ..., S1, ...} ; however,
   if {\tt:}{\sl Picname} is present in a {\sl pinspec} then {\sl Picname}
   replaces the default name.
   A {\sl pinspec} label is text placed at the pin base. Semicolons are
   not allowed in labels; use, e.g., {\tt \char92{}char59\char123\char125}
   instead.
   To put a bar over a label, use {\tt lg\_bartxt(}{\sl label}{\tt)}.
   The {\sl pinopts} are {\tt[N|L|M][E]};
   {\tt N}: pin with not circle;
   {\tt L}: active low out;
   {\tt M}: active low in;
   {\tt E}: edge trigger \seesect{Logicgates:}.\\
   Optional arg 7 is the length of pins}%
\macrodef{foreach\_}{foreach_}{(`{\sl variable}',{\sl actions},%
  {\sl value1, value2, $\ldots$})}%
  {gen}%
  {Clone of Loopover\_ by a different name:
   Repeat {\sl actions} with {\sl variable} set successively to
   {\sl value1, value2, $\ldots$}, setting macro {\tt m4Lx} to 1, 2,
   $\ldots$, terminating if {\sl variable} is nul}%
\macrodef{for\_}{for_}{({\sl start},{\sl end},{\sl increment},`{\sl actions}')}%
  {gen}%
  {integer for loop with index variable {\tt m4x} \seesect{Looping:}}%
\macrodef{FTcap}{FTcap}{({\sl chars})}%
  {cct}%
  {Feed-through capacitor; example of a composite element derived from
   a two-terminal element.  Defined points:
   {\sl .Start, .End, .C, .T1, .T2, T}\\
   Arg 1: {\tt A}:  type A (default),
   {\tt B}: type B,
   {\tt C}: type C 
   \seesect{Composite:}}%
\macrodef{fuse}{fuse}{({\sl linespec, type, wid, ht, attributes})}%
  {cct}%
  {fuse symbol, type$=$
  {\tt  A|B|C|D|S|HB|HC|SB} or {\tt dA=D}\seesect{Twoterminal:}}%

\Letter{G}%

\macrodef{gap}{gap}{(\linespec,{\sl fill},A)}%
  {cct}%
  {gap with (filled) dots, A=chopped arrow between dots\seesect{Twoterminal:}}%
\macrodef{gen\_init}{gen_init}{}%
  {gen}%
  {initialize environment for general diagrams
    (customizable, reads {\tt libgen.m4})}%
\macrodef{g\_fet}{g_fet}{(\linespec,R,P,{\sl shade spec})}%
  {cct}%
  {left or right, N or P graphene FET, without or with shading
    \seesect{Semiconductors:}}%
\macrodef{g\_}{g_}{}%
  {gen}%
  {green color value}%
\macrodef{G\_hht}{G_hht}{}%
  {log}%
  {gate half-height in {\tt L\_unit}s}%
\macrodef{geiger}{geiger}{(\linespec, r, {\sl diameter}, R,
  {\sl body attributes, body name})}%
  {cct}%
  {Wrapper that calls {\tt source} with identical arguments except 
   arg2, which is blank or {\tt r} for right orientation.} 
%\macrodef{glabel\_}{glabel_}{}%
%  {cct}%
%  {internal general labeller}%
%\macrodef{gpar\_}{gpar_}{({\sl element},{\sl element},{\sl separation})}%
%  {cct}%
%  {two same-direction elements in parallel}%
\macrodef{gpolyline\_}{gpolyline_}{({\sl fraction}, {\sl location}, ...)}%
  {gen}%
  {internal to {\tt gshade}}%
\macrodef{graystring}{graystring}{({\sl gray value})}%
  {gen}%
  {evaluates to a string compatible with the postprocessor in use
   to go with {\tt colored}, {\tt shaded}, or {\tt outlined} attributes.
   (PSTricks, metapost, pgf-tikz, pdf, postscript, svg).
   The argument is a fraction in the range $[0,1]$; see {\tt rgbstring}}%
\macrodef{grid\_}{grid_}{({\sl x},{\sl y})}%
  {log}%
  {absolute grid location}%
\macrodef{ground}{ground}{(at {\sl location}, T|{\sl stem length},
   N|F|S|L|P[A]|E, U|D|L|R|{\sl degrees})}%
  {cct}%
  { ground, without stem for 2nd arg = T;\\
    {\tt N}: normal,\\
    {\tt F}: frame,\\
    {\tt S}: signal,\\
    {\tt L}: low-noise,\\
    {\tt P}: protective,\\
    {\tt PA}: protective alternate,\\
    {\tt E}: European; up, down, left, right, or angle
    from horizontal (default -90) \\
   \seesect{Composite:}}%
\macrodef{gshade}{gshade}{({\sl gray value},A,B,...,Z,A,B)}%
  {gen}%
  {(Note last two arguments).  Shade a polygon with named
    vertices, attempting to avoid sharp corners}%
\macrodef{gyrator}{gyrator}{({\sl box specs,space ratio,pin lgth,}[N][V])}%
  {cct}%
  {Gyrator two-port wrapper for {\tt nport}, {\tt N} omits pin dots; {\tt V}
   gives a vertical orientation
   \seesect{Composite:}}%

\Letter{H}%

\macrodef{hatchbox}{hatchbox}{({\sl boxspec,hashsep,hatchspec,angle})
  {\rm or} hatchbox({\sl keys})}%
  {gen}%
% {Alternate invocation: {\tt hatchbox({\sl keys})}.
  {Manhattan box hatched at a specified separation
    and angle (default 45 degrees).
   Defined keys are:\\
    {\tt wid={\sl expr};}\\
    {\tt ht={\sl expr};}\\
    {\tt box={\sl attributes};} (e.g. {\tt dashed outline "{\sl color}"})\\
    {\tt fillcolor="{\sl color}"|({\sl r,g,b});}\\
    {\tt hatchsep={\sl expr};}\\
    {\tt hatchspec={\sl attributes};}\\
    {\tt angle={\sl expr};}
 e.g.,
   {\tt hatchbox(outlined "blue",{},dashed outlined "green" thick 0.4)};\\
   also
   {\tt DefineRGBColor(mycolor,1,0.2,0.5);\\
        hatchbox(box=dashed outlined mycolor)}}%
\macrodef{Header}{Header}{(1|2,{\sl rows,wid,ht,box attributes})}%
  {log}%
  {Header block with 1 or 2 columns and square Pin 1:
   arg1 = number of columns;
   arg2 = pins per column;
   arg3,4 = custom wid, ht;
   arg5 = e.g., {\tt fill\_(0.9)}%
    \seesect{Composite:}}%
\macrodef{HeaderPin}{HeaderPin}{({\sl location, type, Picname},%
n|e|s|w,{\sl length})}%
  {log}%
  {General pin for {\tt Header} macro; arg 4 specifies pin direction
   with respect to the current drawing direction)}%
\macrodef{heatere}{heatere}{({\sl linespec, keys,} [R][T])}%
  {cct}%
  {Heater element with curved sides\seesect{Twoterminal:}.
    {\tt R} means right orientation;
    {\tt T} truncates leads to the width of the body.
    The {\sl keys} for the body are\\
    {\tt lgth={\sl expr};}
    {\tt wdth={\sl expr};} (default {\tt lgth*2/5});
    {\tt cycles={\sl expr};}%
    {\tt line={\sl attributes};} (e.g., {\tt dotted, dashed, outlined})}%
\macrodef{heater}{heater}{({\sl linespec, ndivisions|keys, wid, ht,
    boxspec}|[E[R][T]])}%
  {cct}%
  {Heater element\seesect{Twoterminal:}. If arg 5 contains {\tt E,}
    draws an {\tt heatere({\sl linespec, keys,} [R][T]),}
    otherwise a
    {\tt heatert({\sl linespec, nparts, wid, ht, boxspec})}}%
\macrodef{heatert}{heatert}{({\sl linespec, nparts|keys, wid, ht, boxspec})}%
  {cct}%
  {Two-terminal rectangular heater element\seesect{Twoterminal:}.
    The {\sl keys} for the body are
    {\tt parts={\sl expr};}
    {\tt lgth={\sl expr};}
    {\tt wdth={\sl expr};} (default {\tt lgth*2/5});
    {\tt box={\sl body attributes};}
      (e.g., {\tt dotted, dashed, outlined, shaded}).
    Args 3--5 are unused if any key is given}%
\macrodef{hexadecimal\_}{hexadecimal_}{($n$, [$m$])}%
  {gen}%
  {hexadecimal representation of $n,$ left padded to $m$ digits if the second
   argument is nonblank}%
\macrodef{hex\_digit}{hex_digit}{($n$)}%
  {gen}%
  {hexadecimal digit for $0 \leq n < 16$}%
\macrodef{H\_ht}{H_ht}{}%
  {log}%
  {hysteresis symbol dimension in {\tt L\_unit}s}%
\macrodef{histbins}{histbins}{({\sl data-array name, n, min, max, nbins,
  bin array name})}%
  {dpictools}%
  {$\;\;$ Generate the distribution of {\sl n} values in {\sl data-array}.
    If given, arg3 and arg4 specify maximum and minimum data values,
    otherwise they are calculated. Bins have index 0 to arg5-1.}%
\macrodef{hlth}{hlth}{}%
  {gen}%
  {current line half thickness in drawing units}%
\macrodef{hoprad\_}{hoprad_}{}%
  {cct}%
  {hop radius in crossover macro}%
\macrodef{hsvtorgb}{hsvtorgb}{({\sl h, s, v, r, g, b})}%
  {dpictools}%
  {$\;\;$ hsv color triple to rgb; {\sl h} has range 0 to 360.}%
\macrodef{ht\_}{ht_}{}%
  {gen}%
  {height relative to current direction}%

\Letter{I}%

\macrodef{ifdpic}{ifdpic}{({\sl if true},{\sl if false})}%
  {gen}%
  {test if dpic has been specified as pic processor}%
\macrodef{ifgpic}{ifgpic}{({\sl if true},{\sl if false})}%
  {gen}%
  {test if gpic has been specified as pic processor}%
\macrodef{ifinstr}{ifinstr}{({\sl string},{\sl string},{\sl if true},{\sl
    if false})}%
  {gen}%
  {test if the second argument is a substring of the first; also
  {\tt ifinstr({\sl string},{\sl string},{\sl if true},{\sl
  string},{\sl string},{\sl if true}, $\ldots$ {\sl if false})} }%
\macrodef{ifmfpic}{ifmfpic}{({\sl if true},{\sl if false})}%
  {gen}%
  {test if mfpic has been specified as pic post-processor}%
\macrodef{ifmpost}{ifmpost}{({\sl if true},{\sl if false})}%
  {gen}%
  {test if MetaPost has been specified as pic post-processor}%
\macrodef{ifpgf}{ifpgf}{({\sl if true},{\sl if false})}%
  {gen}%
  {test if \TPGF~has been specified as pic post-processor}%
\macrodef{ifpostscript}{ifpostscript}{({\sl if true},{\sl if false})}%
  {gen}%
  {test if Postscript ({\tt dpic -r}) has been specified as pic output format}%
\macrodef{ifpsfrag}{ifpsfrag}{({\sl if true},{\sl if false})}%
  {gen}%
  {Test if either {\tt psfrag} or {\tt psfrag\_} has been defined. For
   postscript with psfrag strings, one or the other should be defined
   prior to or at the beginning of the diagram}%
\macrodef{ifpstricks}{ifpstricks}{({\sl if true},{\sl if false})}%
  {gen}%
  {test if \PSTricks~has been specified as post-processor}%
\macrodef{ifroff}{ifroff}{({\sl if true},{\sl if false})}%
  {gen}%
  {test if {\bf troff} or {\bf groff} has been specified as post-processor}%
\macrodef{ifxfig}{ifxfig}{({\sl if true},{\sl if false})}%
  {gen}%
  {test if Fig 3.2 ({\tt dpic -x}) has been specified as pic output format}%
\macrodef{igbt}{igbt}{(\linespec,L|R,[L][[d]D])}%
  {cct}%
  {left or right IGBT, L=alternate gate type, D=parallel diode,
   dD=dotted connections }%
\macrodef{inductor}{inductor}{(\linespec, W|L, {\sl cycles}, M[n]|P[n]|K[n],
   {\sl loop wid})}%
  {cct}%
  {inductor, arg2: (default narrow), {\tt W}: wide, {\tt L}: looped;\\
   arg3: number of arcs or cycles (default 4);\\
   arg4:
   {\tt M}: magnetic core,
   {\tt P}: powder (dashed) core,
   {\tt K}: long-dashed core,
     n={\sl integer} (default 2) number of core lines named
     {\sl M4Core1, M4Core2,} $\ldots$;\\
   arg5: loop width (default {\tt L, W}: {\tt dimen\_}/5;
     other: {\tt dimen\_}/8)
   \seesect{Twoterminal:}}%
\macrodef{in\_\_}{in__}{}%
  {gen}%
  {absolute inches}%
\macrodef{inner\_prod}{inner_prod}{({\sl linear obj},{\sl linear obj})}%
  {gen}%
  {inner product of (x,y) dimensions of two linear objects}%
\macrodef{integrator}{integrator}{(\linespec,{\sl size})}%
  {cct}%
  {integrating amplifier\seesect{Twoterminal:}}%
\macrodef{intersect\_}{intersect_}{({\sl line1}.start,{\sl line1}.end,
{\sl line2}.start,{\sl line2}.end)}%
  {gen}%
  {intersection of two lines}%
\macrodef{Intersect\_}{Intersect_}{({\sl Name1},{\sl Name2})}%
  {gen}%
  {intersection of two named lines}%
\macrodef{Int\_}{Int_}{}%
  {gen}%
  {corrected (old) gpic $int()$ function}%
\macrodef{IOdefs}{IOdefs}{(\linespec,{\sl label},[P|N]*,L|R) }%
  {log}%
  {Define locations {\sl label}{\tt 1}, $\ldots$ {\sl label}{\tt n}
    along the line; {\tt P}: label only;
    {\tt N}: with {\tt NOT\_circle};
    {\tt R}: circle to right of current direction }%

\Letter{J}%

\macrodef{jack}{jack}{(U|D|L|R|{\sl degrees},{\sl chars})}%
  {cct}%
  {arg1: drawing direction; string arg2: {\tt R}: right orientation,
   one or more {\tt L[M][B]} for L and auxiliary contacts with make or break
   points; {\tt S[M][B]} for S and auxiliary contacts
   \seesect{Composite:}}%
\macrodef{j\_fet}{j_fet}{(\linespec,L|R,P,E)}%
  {cct}%
  {left or right, N or P JFET, without or with envelope
    \seesect{Semiconductors:}}%
\macrodef{jumper}{jumper}{({\sl linespec, chars}|{\sl keys})}%
  {cct}%
  { Two-terminal solder jumper with named body parts.
    The {\sl chars} character sequence specifies the jumper components,
    and normally begins with {\tt C} and ends with {\tt D.} The character
    {\tt E} is an empty (blank) gap, {\tt J} is a filled gap, {\tt B}
    is a box component.  The components are named {\sl T1, T2, \ldots}
    Examples: {\tt CED} is a simple open jumper (the default); {\tt CJD}
    closed; {\tt CEBED} three-contact open; {\tt CJBED} three-contact
    open and closed.
    The {\sl keys} are: {\tt type=}{\sl chars} as previously;
                        {\tt body=}{\sl attributes} (e.g. {\tt fill\_(0.5)});
                        {\tt wdth=}{\sl expr};
                        {\tt name=}{\sl chars} (the body name)%
\seesect{Twoterminal:}}%

\Letter{K}%

\macrodef{KelvinR}{KelvinR}{({\sl cycles},[R],{\sl cycle wid})}%
  {cct}%
  {IEEE resistor in a {\tt [ ]} block with Kelvin taps {\sl T1} and {\sl T2}
   \seesect{Composite:}}%

\Letter{L}%

\macrodef{lamp}{lamp}{(\linespec, [R][T])}%
  {cct}%
  {Two-terminal incandescent lamp. {\tt T} truncates leads to the body width.
   \seesect{Twoterminal:}}%
\macrodef{langle}{langle}{({\sl Start, End})}%
  {gen}%
  {Angle in radians from horizontal of the line from {\sl Start}
  to {\sl End}.}%
\macrodef{larrow}{larrow}{({\sl label},{\tt ->|<-},{\sl dist})}%
  {cct}%
  {arrow {\sl dist} to left of last-drawn 2-terminal element
   \seesect{Branchcurrent:}}%
\macrodef{lbox}{lbox}{({\sl wid}, {\sl ht}, {\sl attributes})}%
  {gen}%
  {box oriented in current direction, arg 3= e.g.\ {\tt dashed shaded "red"}}%
\macrodef{LCintersect}{LCintersect}{({\sl line name, Centre, rad,} [R],
  [{\sl Line start, End}])}%
  {gen}%
  { First (second if arg4 is R) intersection of a line with a circle.
    Solves $|V.{\tt start} + tV| = {\sl radius}$ for $t$ where $V$ is the line.
    If arg1 is blank then the line start and end are given in arg5 and arg6.}%
\macrodef{LCtangent}{LCtangent}{({\sl Pos1, Centre, rad,} [R])}%
  {gen}%
  { Left (right if arg4=R) tangent point of line
     from Pos1 to circle at Centre with radius arg3}%
\macrodef{left\_}{left_}{}%
  {gen}%
  {left with respect to current direction \seesect{Placing:}}%
\macrodef{LEintersect}{LEintersect}{({\sl line name, Centre, ellipse wid,
  ellipse ht}, [R], [{\sl Line start, End}])}%
  {gen}%
  { First (second if arg5 is R) intersection of a line with an ellipse.
    If arg1 is blank then the line start and end are given in arg6 and arg7.}%
\macrodef{length3}{length3}{({\sl vector})}%
  {dpictools}%
  {$\;\;$ Euclidean length of 3-vector argument.}%
\macrodef{length3D}{length3D}{(x,y,z)}%
  {3D}%
  {Euclidean length of triple x,y,z}%
\macrodef{LEtangent}{LEtangent}{({\sl Pos1, Centre, ellips wid, ellipse ht,}
 [R])}%
  {gen}%
  { Left (right if arg5=R) tangent point of line
     from Pos1 to ellipse at Centre with given width and height}%
\macrodef{lg\_bartxt}{lg_bartxt}{}%
  {log}%
  {draws an overline over logic-pin text (except for xfig)}%
\macrodef{lg\_pin}{lg_pin}{({\sl location, label, Picname},
  n|e|s|w[L|M|I|O][N][E], {\sl pinno, optlen})}%
  {log}%
  {comprehensive logic pin;\\
   {\sl label}: text (indicating logical pin function, usually),\\
   {\sl Picname}: pic label for referring to the pin (line),\\
   {\tt n|e|s|w}: orientation (north, south, east, west),\\
   {\tt L}: active low out,\\
   {\tt M}: active low in,\\
   {\tt I}: inward arrow,\\
   {\tt O}: outward arrow,\\
   {\tt N}: negated,\\
   {\tt E}: edge trigger}%
\macrodef{lg\_pintxt}{lg_pintxt}{}%
  {log}%
  {reduced-size text for logic pins}%
\macrodef{lg\_plen}{lg_plen}{}%
  {log}%
  {logic pin length in in {\tt L\_unit}s}%
\macrodef{LH\_symbol}{LH_symbol}{([U|D|L|R|{\sl degrees}][I])}%
  {log}%
  {logic-gate hysteresis symbol; {\tt I:} inverted}%
\macrodef{lin\_ang}{lin_ang}{({\sl line-reference}[,d])}%
  {gen}%
  {the angle of a line or move from {\tt .start} to {\tt .end} of a linear
   object (in degrees if arg2={\tt d})}%
\macrodef{linethick\_}{linethick_}{({\sl number})}%
  {gen}%
  {set line thickness in points}%
\macrodef{lin\_leng}{lin_leng}{({\sl line-reference})}%
  {gen}%
  {length of a line, equivalent to {\sl line-reference}{\tt .len}
   with dpic}%
\macrodef{ljust\_}{ljust_}{}%
  {gen}%
  {ljust with respect to current direction}%
\macrodef{llabel}{llabel}{( {\sl label}, {\sl label}, {\sl label},
    {\sl relative position}, {\sl block name})}%
  {cct}%
  {Triple label on the left of the body of an element with respect to the
    current direction \seesect{Labels:}.  Labels are placed at the
    beginning, centre, and end of the last {\tt []} block (or a {\tt []} block
    named or enumerated in arg5). Each label is treated as math by
    default, but is copied literally if it is in double quotes or defined
    by sprintf.  {\sl Arg4} can be {\tt above,} {\tt below,} {\tt left,}
    or {\tt right} to supplement the default relative position.}%
\macrodef{loc\_}{loc_}{({\sl x}, {\sl y})}%
  {gen}%
  {location adjusted for current direction}%
\macrodef{log10E\_}{log10E_}{}%
  {gen}%
  {constant $\log_{10}(e)$}%
\macrodef{loge}{loge}{}%
  {gen}%
  {logarithm, base $e$}%
\macrodef{log\_init}{log_init}{}%
  {log}%
  {initialize environment for logic diagrams
    (customizable, reads {\tt liblog.m4})}%
\macrodef{loop}{loop}{({\sl initial assignments, test, loop end, statements})}%
  {dpictools}%
  {$\;\;$ C-like loop.  Commas in arg3 and arg4 must
    be in quotes or parentheses. Example:\\
    {\tt loop(i=1, i<=3, i+=1, print i)} prints 1, 2, 3.}%
\macrodef{Loopover\_}{Loopover_}{(`{\sl variable}',{\sl actions},{\sl
  value1, value2, $\ldots$})}%
  {gen}%
  {Repeat {\sl actions} with {\sl variable} set successively to
   {\sl value1, value2, $\ldots$}, setting macro {\tt m4Lx} to 1, 2,
   $\ldots$, terminating if {\sl variable} is nul}%
\macrodef{lpop}{lpop}{({\sl xcoord}, {\sl ycoord}, {\sl radius},
{\sl fill},
  {\sl zero ht})} {gen}%
  {for lollipop graphs: filled circle with stem to
    {\tt ({\sl xcoord,zeroht})}}%
\macrodef{lp\_xy}{lp_xy}{}%
  {log}%
  {coordinates used by {\tt lg\_pin}}%
\macrodef{lswitch}{lswitch}{( \linespec, L|R, {\sl chars} )}%
  {cct}%
  {knife switch R=right orientation (default L=left);
    {\sl chars}: [O{\tt|}C][D][K][A] O=opening arrow; C=closing arrow;
    D=dots; K=closed switch; A=blade arrowhead \seesect{Twoterminal:}}%
\macrodef{lthick}{lthick}{}%
  {gen}%
  {current line thickness in drawing units}%
\macrodef{lt\_}{lt_}{}%
  {gen}%
  {left with respect to current direction}%
\macrodef{LT\_symbol}{LT_symbol}{(U|D|L|R|{\sl degrees})}%
  {log}%
  {logic-gate triangle symbol}%
\macrodef{L\_unit}{L_unit}{}%
  {log}%
  {logic-element grid size}%

\Letter{M}%

\macrodef{m4\_arrow}{m4_arrow}{(\linespec,{\sl ht},{\sl wid})}%
  {gen}%
  {arrow with adjustable head, filled when possible}%
\macrodef{m4dupstr}{m4dupstr}{({\sl string},{\sl n},`{\sl name}')}%
  {gen}%
  {Defines {\sl name} as {\sl n} concatenated copies of {\sl
  string}.}%
\macrodef{m4lstring}{m4lstring}{({\sl arg1},{\sl arg2})}%
  {gen}%
  {expand {\sl arg1} if it begins
    with {\tt sprintf} or {\tt "}, otherwise {\sl arg2}}%
\macrodef{m4xpand}{m4xpand}{({\sl arg})}%
  {gen}%
  {Evaluate the argument as a macro}%
\macrodef{m4xtract}{m4xtract}{(`{\sl string1}',{\sl string2})}%
  {gen}%
  {delete {\sl string2} from {\sl string1}, return 1 if present}%
\macrodef{manhattan}{manhattan}{}%
  {gen}%
  {sets direction cosines for left, right, up, down}%
\macrodef{Magn}{Magn}{({\sl length, height,} U|D|L|R|{\sl degrees})}%
  {cct}%
  {magnetic action symbol.}%
\macrodef{Max}{Max}{({\sl arg, arg, $\ldots$})}%
  {gen}%
  {Max of an arbitrary number of inputs}%
\macrodef{memristor}{memristor}{({\sl linespec, wid, ht, attributes})}%
  {cct}%
  {memristor element\seesect{Twoterminal:}}%
\macrodef{microphone}{microphone}{( A|U|D|L|R|{\sl degrees, size, attributes})}%
  {cct}%
  {microphone; if arg1 = A: upright mic, otherwise arg1 sets direction
   of standard microphone with {\sl In1} to {\sl In3} defined
   \seesect{Composite:}}%
\macrodef{Min}{Min}{({\sl arg, arg, $\ldots$})}%
  {gen}%
  {Min of an arbitrary number of inputs}%
\macrodef{Mitre\_}{Mitre_}{%
 ({\sl Line1,Line2,length,line attributes})}%
  {gen}%
  {e.g., {\tt Mitre\_(L,M)} draws angle at intersection of lines
   L and M with legs of length arg3 (default {\tt linethick bp\_\_/2});
   sets {\tt Here} to intersection
    \seesect{Corners:}}%
\macrodef{mitre\_}{mitre_}{%
 ({\sl Position1,Position2,Position3,length,line attributes})}%
  {gen}%
  {e.g., {\tt mitre\_(A,B,C)} draws angle ABC with legs
   of length arg4 (default {\tt linethick bp\_\_/2}); sets {\tt Here}
   to Position2
    \seesect{Corners:}}%
\macrodef{mm\_\_}{mm__}{}%
  {gen}%
  {absolute millimetres}%
\macrodef{mosfet}{mosfet}{(\linespec,L|R,{\sl chars},E)}%
  {cct}%
  {MOSFET left or right, included components defined by characters,
  envelope.
   arg 3 chars:\\
   {\tt [u][d]B:} center bulk connection pin\\
   {\tt D:} D pin and lead\\
   {\tt E:} dashed substrate\\
   {\tt F:} solid-line substrate\\
   {\tt [u][d]G:} G pin to substrate at source\\
   {\tt [u][d]H:} G pin to substrate at center\\
   {\tt L:} G pin to channel (obsolete)\\
   {\tt [u][d]M:} G pin to channel, u: at drain end, d: at source end\\
   {\tt [u][d]M{\sl n}:} multiple gates G0 to G{\sl n}\\
   {\tt Py:} parallel diode\\
   {\tt Pz:} parallel zener diode\\
   {\tt Q:} connect B pin to S pin\\
   {\tt R:} thick channel\\
   {\tt [u][d]S:} S pin and lead u: arrow up, d: arrow down\\
   {\tt [d]T:} G pin to center of channel d: not circle\\
   {\tt X:} XMOSFET terminal\\
   {\tt Z:} simplified complementary MOS
   \seesect{Semiconductors:}}%
\macrodef{Mux\_ht}{Mux_ht}{}%
  {cct}%
  {Mux height parameter in {\tt L\_unit}s}%
\macrodef{Mux}{Mux}{({\sl n},{\sl label},
   {\tt [L][B|H|X][N[{\sl n}]|S[{\sl n}]][[N]OE], {\sl wid}, {\sl ht},
     {\sl attributes}})}%
  {log}%
  {binary multiplexer, $n$ inputs,
    {\tt L} reverses input pin numbers, {\tt B} display binary pin
    numbers, {\tt H} display hexadecimal pin numbers, {\tt X} do not
    print pin numbers, {\tt N[{\sl n}]} puts Sel or Sel$0$ .. Sel$n$
    at the top (i.e., to the left of the drawing direction), {\tt
    S[{\sl n}]} puts the Sel inputs at the bottom (default) {\tt OE}
    ({\tt N}: negated) {\tt OE} pin \seesect{Logicgates:}}%
\macrodef{Mux\_wid}{Mux_wid}{}%
  {cct}%
  {Mux width parameter in {\tt L\_unit}s}%
\macrodef{Mx\_pins}{Mx_pins}{}%
  {log}%
  {max number of gate inputs without wings}%

\Letter{N}%

\macrodef{NAND\_gate}{NAND_gate}{({\tt {\sl n}, [N][B],
   [{\sl wid}, [{\sl ht}]], {\sl attributes}})}%
  {log}%
  {`nand' gate, 2 or {\sl n\/} inputs ($0 \leq n \leq 16$);
   {\tt N}: negated inputs; {\tt B}: box shape.
   Alternatively, {\tt NAND\_gate({\sl chars}, [B], {\sl wid},
    {\sl ht}, {\sl attributes})}, where
   arg1 is a sequence of letters {\tt P|N} to define
   normal or negated inputs.
    \seesect{Logicgates:}}%
\macrodef{N\_diam}{N_diam}{}%
  {log}%
  {diameter of `not' circles in {\tt L\_unit}s}%
\macrodef{NeedDpicTools}{NeedDpicTools}{}%
  {gen}%
  {executes {\tt copy "HOMELIB\_/dpictools.pic"} if the file has
  not been read}%
\macrodef{neg\_}{neg_}{}%
  {gen}%
  {unary negation}%
\macrodef{ne\_}{ne_}{}%
  {gen}%
  {.ne with respect to current direction}%
\macrodef{n\_}{n_}{}%
  {gen}%
  {.n with respect to current direction}%
\macrodef{norator}{norator}{(\linespec,{\sl width},{\sl ht},{\sl attributes})}%
  {cct}%
  { norator two-terminal element \seesect{Twoterminal:}}%
\macrodef{NOR\_gate}{NOR_gate}{({\sl n},N)}%
  {log}%
  {`nor' gate, 2 or {\sl n\/} inputs; {\tt N}: negated input.
   Otherwise, arg1 can be a sequence of letters {\tt P|N} to define
   normal or negated inputs.
    \seesect{Logicgates:}}%
\macrodef{NOT\_circle}{NOT_circle}{}%
  {log}%
  {`not' circle}%
\macrodef{NOT\_gate}{NOT_gate}{(\linespec,[B][N|n],{\sl wid},{\sl height},
  {\sl attributes})}%
  {log}%
  {`not' gate.
   When {\sl linespec} is blank then the element is composite and In1,
   Out, C, NE, and SE are defined; otherwise the element is drawn as a
   two-terminal element. arg2: {\tt B}: box gate, {\tt N}: not circle at
   input and output, {\tt n}: not circle at input only
    \seesect{Logicgates:}}%
\macrodef{NOT\_rad}{NOT_rad}{}%
  {log}%
  {`not' radius in absolute units}%
\macrodef{NPDT}{NPDT}{({\sl npoles,}[R])}%
  {cct}%
  {Double-throw switch; {\sl npoles:} number of poles;
   {\tt R}: right orientation with respect to drawing direction
   \seesect{Composite:}}%
\macrodef{nport}{nport}{({\sl box spec{\tt ;}other commands,
nw,nn,ne,ns,space ratio,pin lgth,style, other commands})}%
  {cct}%
  { Default is a standard-box twoport.  Args 2 to 5 are
    the number of ports to be drawn on w, n, e, s sides.  The port pins
    are named by side, number, and by a or b pin, e.g., W1a, W1b, W2a,
    $\ldots$ Arg 6 specifies the ratio of port width to interport space
    (default 2), and arg 7 is the pin length.  Set arg 8 to N to omit
    the dots on the port pins. Arguments 1 and 9 allow customizations
   \seesect{Composite:}}%
\macrodef{N\_rad}{N_rad}{}%
  {log}%
  {radius of `not' circles in {\tt L\_unit}s}%
\macrodef{nterm}{nterm}{({\sl box spec{\tt ;}other commands,
nw,nn,ne,ns,pin lgth,style, other commands})}%
  {cct}%
  {n-terminal box macro (default three pins).
   Args 2 to 5 are the number of pins to be drawn on W, N, E, S
   sides.  The pins are named by side and number, e.g. W1, W2, N1,
   $\ldots$ Arg 6 is the pin length.  Set arg 7 to N to omit the
   dots on the pins. Arguments 1 and 8 allow customizations, e.g.
   {\tt nterm(,{,},{,},{,}N,"\$a\$" at Box.w ljust,"\$b\$" at Box.e rjust,
      "\$c\$" at Box.s above)} }%
\macrodef{nullator}{nullator}{(\linespec,{\sl width},{\sl ht},{\sl attributes})}%
  {cct}%
  { nullator two-terminal element \seesect{Twoterminal:}}%
\macrodef{nw\_}{nw_}{}%
  {gen}%
  {.nw with respect to current direction}%
\macrodef{NXOR\_gate}{NXOR_gate}{({\sl n},N)}%
  {log}%
  {`nxor' gate, 2 or {\sl n\/} inputs; {\tt N}: negated input.
   Otherwise, arg1 can be a sequence of letters {\tt P|N} to define
   normal or negated inputs.
    \seesect{Logicgates:}}%

\Letter{O}%

\macrodef{opamp}{opamp}{\tt (\linespec,{\sl label, label, size|keys, chars,
  other commands})}%
  {cct}%
  {operational amplifier with $-,$ $+$ or other internal labels and
    specified size, drawn in a {\tt [ ]} block.
    {\sl chars:} {\tt P} add power connections, {\tt
    R} swap {\sl In1, In2 labels,} {\tt T} truncated point.  The internally
    defined positions are {\sl W, N, E, S, Out, NE, SE, In, In2}, and
    the (obsolete) positions {\sl E1 = NE, E2 = SE}. Instead of a size
    value, arg4 can be a key-value sequence.  The keys are:
    {\tt lgth={\sl expr};}, 
    {\tt wdth={\sl expr};}, 
    {\tt body={\sl attributes};}, e.g., {\tt body=shaded "{\sl color}"}. 
   \seesect{Composite:}}%
\macrodef{open\_arrow}{open_arrow}{(\linespec,{\sl ht},{\sl wid})}%
  {gen}%
  {arrow with adjustable open head}%
\macrodef{OR\_gate}{OR_gate}{({\sl n},[N][B], {\sl wid,} {\sl ht,}%
   {\sl attributes})}%
  {log}%
  {Or gate, $n$ inputs ($0 \leq n \leq 16$); arg2: {\tt N}: negated inputs;
   {\tt B}: box gate.
   Otherwise, arg1 can be a sequence of letters {\tt P|N} to define
   normal or negated inputs.
    \seesect{Logicgates:}}%
\macrodef{OR\_gen}{OR_gen}{($n$,{\sl chars},[{\sl wid},[{\sl ht}]],
   {\sl attributes})}%
  {log}%
  {General OR gate: $n$=number of inputs $(0\leq n\leq 16)$;
    {\sl chars:}%
    {\tt B}: base and straight sides;\\ 
    {\tt A}: arcs;\\
    {\tt [N]NE,[N]SE,[N]I,[N]N,[N]S}: inputs or circles;\\ 
    {\tt [N]P}: XOR arc;\\
    {\tt [N]O}: output; C=center.\\
   Otherwise, arg1 can be a sequence of letters {\tt P|N} to define
   normal or negated inputs.  If arg5 contains {\tt shaded rgbstring(...)}
   the arguments of {\tt rgbstring} may not contain parentheses.}%
\macrodef{OR\_rad}{OR_rad}{}%
  {log}%
  {radius of OR input face in {\tt L\_unit}s}%

\Letter{P}%

\macrodef{parallel\_}{parallel_}{(\char96{\sl elementspec}\char39,%
 \char96{\sl elementspec}\char39 $\ldots$)}%
  {cct}%
  { Parallel combination of two-terminal elements in a {\tt []} block.
    Each argument is a {\em quoted} elementspec of the form\\
    {\tt[Sep={\sl val};][{\sl Label}:]{\sl element};[{\sl attributes}]} where
    an {\sl attribute} is of the form\\
    {\tt[llabel($\ldots$);]|[rlabel($\ldots$);]|[b\_current($\ldots$);]}.\\
    An argument may also be {\tt series\_($\ldots$)} or {\tt
    parallel\_($\ldots$)} {\em without} attributes or quotes.  Sep={\sl
    val}; in the first branch sets the default separation of all branches
    to {\sl val}; in a later element Sep={\sl val}; applies only to
    that branch.  An element may have normal arguments but should not
    change the drawing direction.
   \seesect{Seriesandparallel:}}%
\macrodef{pconnex}{pconnex}{(R|L|U|D|{\sl degrees},{\sl chars, attributes})}%
  {cct}%
  {power connectors, arg 1: drawing direction; {\sl chars:}\\
   {\tt R} (right orientation)\\
   {\tt M|F} (male, female)\\
   {\tt A[B]|AC} (115V 3-prong, B: default box, C: circle)\\
   {\tt P} (PC connector)\\
   {\tt D} (2-pin connector)\\
   {\tt G|GC} (GB 3-pin)\\
   {\tt J} (110V 2-pin)
    \seesect{Composite:}}%
\macrodef{pc\_\_}{pc__}{}%
  {gen}%
  {absolute points}%
\macrodef{perpto}{perpto}{({\sl Pos1, Line, Point})}%
  {gen}%
  {{\sl Point} is the label for the point on {\sl Line} of the
  perpendicular
   from {\sl Point} to {\sl Line}.}%
\macrodef{PerpTo}{PerpTo}{({\sl Pos1, Pos2, Pos3})}%
  {gen}%
  {The point between Pos2 and Pos3 of intersection of the perpendicular
   to Pos1, i.e., the perpendicular projection of Pos1 onto the line
   from Pos2 to Pos3.}%
\macrodef{pi\_}{pi_}{}%
  {gen}%
  {$\pi$}%
\macrodef{plug}{plug}{\smash{(U|D|L|R|{\sl degrees},[2|3][R])}}%
  {cct}%
  {Phone plug; arg1: drawing direction; arg2: {\tt R} right orientation,
   {\tt 2|3} number of conductors\seesect{Composite:}}%
\macrodef{pmod}{pmod}{({\sl integer}, {\sl integer})}%
  {gen}%
  {+ve $\hbox{mod}(M,N)$ e.g., $\hbox{\tt pmod}(-3,5)=2$}%
\macrodef{point\_}{point_}{({\sl angle})}%
  {gen}%
  {(radians) set direction cosines}%
\macrodef{Point\_}{Point_}{({\sl integer})}%
  {gen}%
  {sets direction cosines in degrees \seesect{Placing:}}%
\macrodef{polar\_}{polar_}{({\sl x},{\sl y})}%
  {gen}%
  {rectangular-to polar conversion}%
\macrodef{polygon}{polygon}{({\sl n},{\sl keys})}%
  {gen}%
  {Regular polygon in a {\tt []} block. The keys are
    {\tt line={\sl line attributes};} (e.g., {\tt dashed shaded "blue"}),
    {\tt rot={\sl degrees};} (angle of first internal vertex {\tt V[0]}),
    {\tt side|rad={\sl expression};} size by side length or by radius.
    {\tt radv={\sl expression};} radius of rounded vertices. If this is
    nonzero then any fill has to be by {\tt rgbfill(r,g,b,polygon(...)).}
   The internal defined points are the centre {\tt C} and vertices
   {\tt V[0]} $\ldots$ {\tt V[{\sl n}]}. }%
\macrodef{posarray}{posarray}{({\sl Name, Position1, Position2, \ldots})}%
  {dpictools}%
  {$\;\;$  Populate a singly-subscripted array of positions:
    {\tt {\sl Name}[1]:{\sl Position1;}
         {\sl Name}[2]={\sl Position2;} \ldots}.}%
\macrodef{posarray2}{posarray2}{({\sl Name, expr, Position1, Position2,
  \ldots})}%
  {dpictools}%
  {$\;\;$  Populate a doubly-subscripted array of positions:
    {\tt {\sl Name}[{\sl expr},1]={\sl Position1;}
         {\sl Name}[{\sl expr},2]={\sl Position2;} \ldots}.}%

\macrodef{potentiometer}{potentiometer}{(\linespec,%
  {\sl cycles},{\sl fractional pos},{\sl length},$\cdots$)} {cct}%
  {resistor with taps T1, T2, $\ldots$
    with specified fractional positions and lengths (possibly neg)
   \seesect{Composite:}}%
\macrodef{print3D}{print3D}{(x,y,z)}%
  {3D} {write out triple for debugging}%
\macrodef{prod\_}{prod_}{({\sl a},{\sl b})}%
  {gen}%
  {binary multiplication}%
\macrodef{project}{project}{({\sl x},{\sl y},{\sl z})}%
  {3D}%
  {3D to 2D projection onto the plane perpendicular to the view
  vector {\tt View3D} with angles defined by
  {\tt setview({\sl azimuth, elevation, rotation})}.}%
\macrodef{Proxim}{Proxim}{({\sl size,} U|D|L|R|{\sl degrees, attributes})}%
  {cct}%
  {proximity detector with fillable body.}%
\macrodef{proximity}{proximity}{(\sl linespec)}%
  {cct}%
  {proximity detector (= {\tt consource(,P)})}%
\macrodef{psset\_}{psset_}{({\sl PSTricks settings})}%
  {gen}%
  {set PSTricks parameters}%
\macrodef{PtoL}{PtoL}{({\sl position}, U|D|L|R|{\sl degrees}, {\sl length}) }%
  {gen}%
  {Evaluates to {\tt from {\sl position} to
   {\sl position} + Rect\_({\sl length, angle}) }
   from the polar-coordinate data in the arguments }%
\macrodef{pt\_\_}{pt__}{}%
  {gen}%
  {\TeX\ point-size factor, in scaled inches, ({\tt *scale/72.27})}%
\macrodef{ptrans}{ptrans}{(\linespec, [R|L])}%
  {cct}%
  {pass transistor; {\tt L=} left orientation
    \seesect{Semiconductors:}}%
\macrodef{pushkey\_}{pushkey_}{({\sl string, key, default value,}[N])}%
  {gen}%
  {Key-value definition.
  If {\sl string} contains the substring
  {\sl key}{\tt =}{\sl expr} then macro {\tt m4{\sl key}}
  is defined using {\tt pushdef()} to
  expand to {\tt ({\sl expr})}, or to {\tt ({\sl default value})} if the
  substring is missing.  Arg 1 can contain several such substrings separated
  by semicolons.
  If arg4 is nonblank, the parentheses are omitted.
   \seesect{Macroarguments:}}%
\macrodef{pushkeys\_}{pushkeys_}{({\sl string, key sequence})}%
  {gen}%
  {Multiple key-value definitions. Arg2 is a semicolon-separated
   sequence of terms of the form {\tt {\sl key}:{\sl default-value}[:N]}
   which must contain no semicolons and the default values contain no colons.
   A key may not be the tail of another key.
   Macro {\tt pushkey\_} is applied to each of the terms in order.
   Quote arg2 for robustness and, if an argument depends on a previous
   argument, add quotes to delay expansion; for example
   {\tt pushkeys\_(\char96\$1\char39,\char96{}hght:0.5;
     wdth:m4\char96\char39hght/2\char39)}.
   \seesect{Macroarguments:}}%
\macrodef{pvcell}{pvcell}{({\sl linespec, width, height, attributes})}%
  {cct}%
  {PV cell}%
\macrodef{px\_\_}{px__}{}%
  {gen}%
  {absolute SVG screen pixels}%

\Letter{R}%

\macrodef{randn}{randn}{({\sl array name, $n,$ mean, stddev})}%
  {dpictools}%
  {$\;\;$ Assign $n$ Gaussian random numbers in array
   $\hbox{\sl name}[1], \hbox{\sl name}[2], \ldots \hbox{\sl name}[n]$
   with given mean and standard deviation.}%
\macrodef{rarrow}{rarrow}{({\sl label,{\tt ->|<-},{\sl dist}})}%
  {cct}%
  {arrow {\sl dist} to right of last-drawn 2-terminal element
   \seesect{Branchcurrent:}}%
\macrodef{Rect\_}{Rect_}{({\sl radius},{\sl angle})}%
  {gen}%
  {(deg) polar-to-rectangular conversion}%
\macrodef{rect\_}{rect_}{({\sl radius},{\sl angle})}%
  {gen}%
  {(radians) polar-rectangular conversion}%
\macrodef{reed}{reed}{({\sl linespec, width, height, box attribues},
[R][C])}%
  {cct}%
  {Enclosed reed two-terminal contact;
   {\tt R}: right orientation; {\tt C}: closed contact; e.g., {\tt
   reed(,{,}dimen\_/5,shaded "lightgreen"} \seesect{Composite:}}%
\macrodef{relaycoil}{relaycoil}{({\sl chars, wid, ht,} R|L|U|D|{\sl
degrees, attributes})}%
  {cct}%
  {chars:
    {\tt X}: or default: external lines from A2 and B2;\\
    {\tt AX}: external lines at positions A1,A3;\\
    {\tt BX}: external lines at positions B1,B3;\\
    {\tt NX}: no lines at positions A1,A2,A3,B1,B2,B3;\\
    {\tt SO}: slow operating;\\
    {\tt SOR}: slow operating and release;\\
    {\tt SR}: slow release;\\
    {\tt HS}: high speed;\\
    {\tt S}:  diagonal slash;\\
    {\tt NAC}: unaffected by AC current;\\
    {\tt AC}: AC current;\\
    {\tt ML}: mechanically latched;\\
    {\tt PO}: polarized;\\
    {\tt RM}: remanent;\\
    {\tt RH}: remanent;\\
    {\tt TH}: thermal;\\
    {\tt EL}: electronic
   \seesect{Composite:}}%
\macrodef{relay}{relay}{({\sl number of poles, chars, attributes})}%
  {cct}%
  {relay: n poles (default 1),\\
    {\sl chars:} {\tt O}: normally open,\\
    {\tt C}: normally closed,\\
    {\tt P}: three position, default double throw,\\
    {\tt L}: drawn left (default),\\
    {\tt R}: drawn right,\\
    {\tt Th}: thermal.\seesect{Composite:}}%
\macrodef{resetdir\_}{resetdir_)}{}%
  {gen}%
  {resets direction set by {\tt setdir\_}}%
\macrodef{resetrgb}{resetrgb}{}%
  {gen}%
  {cancel {\tt r\_, g\_, b\_} color definitions}%
\macrodef{resistor}{resistor}{(\linespec, n, {\sl chars}, {\sl cycle wid})}%
  {cct}%
  {resistor, n cycles (default 3), {\sl chars:}\\
    {\tt AC}: general complex element,\\
    {\tt E}: {\tt ebox},\\
    {\tt ES}: {\tt ebox} with slash,\\
    {\tt F}: FDNR (frequency-dependent negative resistor),\\
    {\tt Q}: offset,\\
    {\tt H}: squared,\\
    {\tt N}: IEEE,\\
    {\tt B}: not burnable,\\
    {\tt V}: varistor variant,\\
    {\tt R}: right-oriented;\\
    Arg4: {\sl cycle width} (default {\tt dimen\_}$/6$)
   \seesect{Twoterminal:}}%
\macrodef{resized}{resized}{({\sl factor},`{\sl macro name}',args)}%
  {cct}%
  {scale the element body size by {\sl factor}}%
\macrodef{restorem4dir}{restorem4dir}{([`{\sl stack name}'])}%
  {gen}%
  {Restore m4 direction parameters from the named stack;
    default {\tt `savm4dir\_'}}%
\macrodef{reversed}{reversed}{(`{\sl macro name}',args)}%
  {cct}%
  {reverse polarity of 2-terminal element}%
\macrodef{rgbdraw}{rgbdraw}{({\sl color triple}, {\sl drawing commands})}%
  {gen}%
  {color drawing for PSTricks, pgf, MetaPost, SVG postprocessors;
   (color entries are 0 to 1),
   see {\tt setrgb} \seesect{Semiconductors:}.  Exceptionally, the color
   of SVG arrows other than the default black has to be defined using the
   {\tt outlined }{\sl string} and {\tt shaded }{\sl string} constructs.}%
\macrodef{rgbfill}{rgbfill}{({\sl color triple}, {\sl closed path})}%
  {gen}%
  {fill with arbitrary color (color entries are 0 to 1); see {\tt
   setrgb}\seesect{Semiconductors:}}%
\macrodef{rgbstring}{rgbstring}{({\sl color triple or color name})}%
  {gen}%
  {evaluates to a string compatible with the postprocessor in use
   to go with {\tt colored}, {\tt shaded}, or {\tt outlined} attributes.
   (PSTricks, metapost, pgf-tikz, pdf, postscript, svg).  The arguments
   are fractions in the range $[0,1]$; For example, {\tt box outlined
   rgbstring(0.1,0.2,0.7) shaded rgbstring(0.75,0.5,0.25)}.  For those
   postprocessors that allow it, there can be one argument which is the
   name of a defined color}%
\macrodef{rgbtocmyk}{rgbtocmyk}{({\sl r, g, b, c, m, y, k})}%
  {dpictools}%
  {$\;\;$ rgb to cmyk values in the range 0 to 100.}%
\macrodef{rgbtohsv}{rgbtohsv}{({\sl r, g, b, h, s, v})}%
  {dpictools}%
  {$\;\;$ rgb color triple to hsv with {\sl h} range 0 to 360.}%
\macrodef{RightAngle}{RightAngle}{({\sl Pos1, Pos2, Pos3, line len,
  attributes})}%
  {gen}%
  {Draw a right-angle symbol at {\sl Pos2}, of size given by arg4.
   Arg5 = line attributes, e.g., {\tt outlined "gray"} or e.g. to add a dot,
    {\tt ;dot(at last line.c)}}%
\macrodef{right\_}{right_}{}%
  {gen}%
  {set current direction right \seesect{Placing:}}%
\macrodef{rjust\_}{rjust_}{}%
  {gen}%
  {right justify with respect to current direction}%
\macrodef{rlabel}{rlabel}{( {\sl label}, {\sl label}, {\sl label},
    {\sl relative position}, {\sl block name})}%
  {cct}%
  {Triple label on the right of the body of an element with respect to the
    current direction \seesect{Labels:}.  Labels are placed at the
    beginning, centre, and end of the last {\tt []} block (or a {\tt []} block
    named or enumerated in arg5). Each label is treated as math by
    default, but is copied literally if it is in double quotes or defined
    by sprintf.  {\sl Arg4} can be {\tt above,} {\tt below,} {\tt left,}
    or {\tt right} to supplement the default relative position.}%
\macrodef{rot3Dx}{rot3Dx}{({\sl radians,x,y,z})}%
  {3D} {rotates x,y,z about x axis}%
\macrodef{rot3Dy}{rot3Dy}{({\sl radians,x,y,z})}%
  {3D} {rotates x,y,z about y axis}%
\macrodef{rot3Dz}{rot3Dz}{({\sl radians,x,y,z})}%
  {3D} {rotates x,y,z about z axis}%
\macrodef{rotbox}{rotbox}{({\sl wid,ht,attributes},[r|t={\sl val}])}%
  {gen}%
  {box oriented in current direction in {\tt [ ]} block;
   {\sl attributes:} e.g. {\tt dotted shaded "green".}  Defined internal
   locations: N, E, S, W (and NE, SE, NW, SW if arg4 is blank).  If arg4
   is {\tt r=}{\sl val} then corners have radius {\sl val}.  If arg4 is
   {\tt t=}{\sl val} then a spline with tension {\sl val} is used to draw
   a ``superellipse,'' and the bounding box is then only approximate. }%
\macrodef{rotellipse}{rotellipse}{({\sl wid,ht,attributes})}%
  {gen}%
  {ellipse oriented in current direction in {\tt [ ]} block;
   e.g. {\tt Point\_(45); rotellipse(,{},dotted fill\_(0.9)).} Defined
   internal locations: N, S, E, W.}%
\macrodef{Rot\_}{Rot_}{({\sl position, degrees})}%
  {gen}%
  {rotate position by degrees}%
\macrodef{rot\_}{rot_}{({\sl x, y, angle})}%
  {gen}%
  {rotate {\sl x,y} by theta radians}%
\macrodef{round}{round}{(at {\sl location,line thickness,attributes})}%
  {gen}%
  {filled circle for rounded corners; attributes={\tt colored "gray"}
   for example; leaves {\tt Here} unchanged if arg1 is blank
    \seesect{Corners:}}%
\macrodef{rpoint\_}{rpoint_}{(\linespec)}%
  {gen}%
  {set direction cosines}%
\macrodef{rpos\_}{rpos_}{({\sl position})}%
  {gen}%
  {Here + {\sl position}}%
\macrodef{r\_}{r_}{}%
  {gen}%
  {red color value}%
\macrodef{rrot\_}{rrot_}{({\sl x, y, angle})}%
  {gen}%
  {\tt Here + vrot\_({\sl x, y, cos(angle), sin(angle))}}%
\macrodef{rs\_box}{rs_box}{([angle={\sl degrees};]
     {\sl text},{\sl expr1},$\cdots$)}%
  {gen}%
  {like {\tt s\_box} but the text is rotated by {\tt text\_ang}%
   (default 90) degrees, unless
   the first argument begins with {\tt angle={\sl decimal number};},
   in which case the number defines the rotation angle.
   Two or more args are passed to {\tt sprintf()}.
   If the first argument begins with {\tt angle={\sl expr};} then
   the specified angle is used.
   The examples {\tt define(`text\_ang',45); rs\_box(Hello World)} and
   {\tt rs\_box(angle=45; Hello World)} are equivalent
   \seesect{Interaction:}, \seesect{Pstricks:}}%
\macrodef{rsvec\_}{rsvec_}{({\sl position})}%
  {gen}%
  {Here + {\sl position}}%
\macrodef{r\_text}{r_text}{({\sl degrees},{\sl text},at {\sl position})}%
  {gen}%
  {Rotate text by arg1 degrees (provides a single command for
   PSTricks, PGF, or SVG only) placed at position in arg3.
   The first argument is a decimal constant (not an expression) and
   the text is a simple string without quotes.
   \seesect{Interaction:}, \seesect{Pstricks:}}%
\macrodef{rtod\_\_}{rtod__}{}%
  {gen}%
  {constant, degrees/radian}%
\macrodef{rtod\_}{rtod_}{}%
  {gen}%
  {constant, degrees/radian}%
\macrodef{rt\_}{rt_}{}%
  {gen}%
  {right with respect to current direction}%
\macrodef{rvec\_}{rvec_}{({\sl x},{\sl y})}%
  {gen}%
  {location relative to current direction}%

\Letter{S}%

\macrodef{sarrow}{sarrow}{(\linespec,{\sl keys})}%
  {gen}%
  {Single-segment, single-headed special arrows with {\sl keys}:\\
       {\tt type=}{\tt O[pen]} (default)
          | {\tt D[iamond]} | {\tt C[rowfoot]} | {\tt P[lain]}
          | {\tt R[ight]} | {\tt L[eft]} {\tt ;}\\
       {\tt wdth=}{\sl expression}{\tt ;} (default {\tt arrowwid})\\
       {\tt lgth=}{\sl expression}{\tt ;} (default {\tt arrowht})\\
       {\tt shaft=}{\sl shaft attributes} (e.g., {\tt dashed}){\tt ;}\\
       {\tt head=}{\sl head attributes} (e.g., {\tt shaded}){\tt ;}\\
       {\tt name=}{\sl Name}{\tt ;} (default {\sl Sarrow\_})
  \seesect{Macroarguments:}}%
\macrodef{savem4dir}{savem4dir}{([`{\sl stack name}'])}%
  {gen}%
  {Stack m4 direction parameters in the named stack
    (default {\tt `savm4dir\_'})}%
\macrodef{s\_box}{s_box}{({\sl text},{\sl expr1},$\cdots$)}%
  {gen}%
  {generate dimensioned text string using {\tt\char92{}boxdims} from
    {\tt boxdims.sty}. Two or more args are passed to {\tt sprintf()}
   (default 90) degrees \seesect{Interaction:}}%

\macrodef{sbs}{sbs}{({\sl linespec, chars, label})}%
  {cct}%
  {Wrapper to place an SBS thyristor as a two-terminal element with
   {\tt [ ]} block label given by the third argument
    \seesect{Semiconductors:}}%
\macrodef{sc\_draw}{sc_draw}{({\sl dna string, chars, iftrue, iffalse})}%
  {cct}%
  {test if chars are in string, deleting chars from string}%
\macrodef{scr}{scr}{({\sl linespec, chars, label})}%
  {cct}%
  {Wrapper to place an SCR thyristor as a two-terminal element with
   {\tt [ ]} block label given by the third argument
    \seesect{Semiconductors:}}%
\macrodef{scs}{scs}{({\sl linespec, chars, label})}%
  {cct}%
  {Wrapper to place an SCS thyristor as a two-terminal element with
   {\tt [ ]} block label given by the third argument
    \seesect{Semiconductors:}}%
\macrodef{s\_dp}{s_dp}{({\sl name},{\sl default})}%
  {gen}%
  {depth of the most recent (or named) {\tt s\_box}
    \seesect{Interaction:}}%
\macrodef{series\_}{series_}{({\sl elementspec}, {\sl elementspec},
$\ldots$)}%
  {cct}%
  { Series combination in a {\tt []} block of elements
    with shortened default length.
    Each argument is an elementspec of the form\\
    {\tt[Sep={\sl val};][{\sl Label}:]{\sl element};[{\sl attributes}]} where
    an {\sl attribute} is of the form\\
    {\tt[llabel($\ldots$);]|[rlabel($\ldots$);]|[b\_current($\ldots$);]}.\\
    An argument may also be {\tt series\_($\ldots$)} or {\tt
    parallel\_($\ldots$)} {\em without} attributes or quotes.
    An element may have normal arguments but should not
    change the drawing direction.
    Internal points {\tt Start}, {\tt End}, and {\tt C} are defined
   \seesect{Seriesandparallel:} }%
\macrodef{se\_}{se_}{}%
  {gen}%
  {.se with respect to current direction}%
\macrodef{setdir\_}{setdir_}{(R|L|U|D|{\sl degrees}, {\sl default}%
    U|D|R|L|{\sl degrees})}%
  {gen}%
  {store drawing direction and set it to
    up, down, left, right, or angle in degrees (reset by {\tt
    resetdir\_}).  The directions may be spelled out, i.e., Right,
    Left, $\ldots$
   \seesect{Seriesandparallel:}}%
\macrodef{setkey\_}{setkey_}{({\sl string, key, default,}[N])}%
  {gen}%
  {Key-value definition, like {\tt pushkey\_()} but the resulting
  macro is defined using {\tt define()} rather than {\tt pushdef().}
   \seesect{Macroarguments:}}%
\macrodef{setkeys\_}{setkeys_}{({\sl string, key sequence})}%
  {gen}%
  {Multiple key-value definition using {\tt define()} rather than
   {\tt pushdef().} See macro {\tt pushkeys\_}.
   \seesect{Macroarguments:}}%
\macrodef{setrgb}{setrgb}{({\sl red value, green value, blue value},[{\sl
   name}])}%
  {gen}%
  {define colour for lines and text, optionally named (default
   {\tt lcspec}); \seesect{Semiconductors:}}%
\macrodef{setview}{setview}{({\sl azimuth degrees},{\sl elevation
   degrees}, {\sl rotation degrees})}%
  {3D} {Set projection viewpoint for the {\tt project} macro.
   The view vector is obtained by looking in along the $x$ axis,
   then rotating about $-x,$ $-y,$ and $z$ in that order.  The components
   {\tt view3D1, view3D2,} and {\tt view3D3} are defined, as well as
   positions {\tt UPx\_, UPy\_,} and {\tt UPz\_} which are the projections
   of unit vectors {\tt (1,0,0), (0,1,0),} and {\tt (0,0,1)} respectively
   onto the plane.}%
\macrodef{sfgabove}{sfgabove}{}%
  {cct}%
  {like above but with extra space}%
\macrodef{sfgarc}{sfgarc}{(\linespec,{\sl text},{\sl text
justification},cw|ccw,
    {\sl height scale factor},{\sl arc attributes})}%
  {cct}%
  {Directed arc drawn between nodes, with text label
    and a height-adjustment parameter. Example:
    {\tt sfgarc(from B to A,-B/M,below,{,}1.1,outlined "red")} }%
\macrodef{sfgbelow}{sfgbelow}{}%
  {cct}%
  {like below but with extra space}%
\macrodef{sfg\_init}{sfg_init}{({\sl default line len, node rad, arrowhd len,
  arrowhd wid}), (reads {\tt libcct.m4})}
  {cct}%
  {initialization of signal flow graph macros}%
\macrodef{sfgline}{sfgline}{(\linespec,{\sl text},{\tt
  sfgabove|sfgbelow|ljust|rjust},{\sl line attributes})}%
  {cct}%
  {Directed straight line chopped by node radius, with text label,
   e.g., {\tt sfgline(,K/M,{,}dashed colored "orange")} }%
\macrodef{sfgnode}{sfgnode}{(at {\sl location},{\sl text},above|below,{\sl
   circle attributes})}%
  {cct}%
  {small circle default white interior, with text label. The default
  label position is inside if the diameter is bigger than {\tt textht}
  and {\tt textwid}; otherwise it is {\tt sfgabove.} Options such as
  color, fill, or line thickness can be given, e.g.,
  {\tt thick 0.8 outlined "red" shaded "orange"}. }%
\macrodef{sfgself}{sfgself}{(at {\sl location}, U|D|L|R|{\sl degrees},
   {\sl text label}, {\sl text justification}, cw|ccw, {\sl scale factor},
    [-> | <- | <->], {\sl attributes})}%
  {cct}%
  {Self-loop drawn at an angle from a node,
     with text label, specified arrowheads, and a size-adjustment parameter.
     The attributes can set thickness and color, for example. }%
\macrodef{shade}{shade}{({\sl gray value},{\sl closed line specs})}%
  {gen}%
  {Fill arbitrary closed curve. Note: when producing pdf via pdflatex, line
   thickness changes within this macro must be made via the {\tt linethick}
   environment variable rather than by the {\tt thickness} line attribute}%
\macrodef{shadebox}{shadebox}{(box {\sl attributes, shade width})}%
  {gen}%
  {Box with edge shading.  Arg2 is in points. See also {\tt shaded} }%
\macrodef{shadedball}{shadedball}{({\sl radius, highlight radius,
  highlight degrees, initial gray, final gray} | ({\sl rf,gf,bf}))}%
  {3D}%
  { Shaded ball in {\tt [ ]} box.
    The highlight is by default at {\sl radius}*3/5 and angle 110 deg
    (or arg2 deg); if setlight has been invoked then its azimuth and
    elevation arguments determine highlight position.  Arg5 can be a
    parenthesized rgb color.}
\macrodef{ShadedPolygon}{ShadedPolygon}{({\sl vertexseq, line attributes,
  degrees, colorseq})} {gen}%
  {Draws the polygon specified in arg1 and
  shades the interior according
   to arg4 by drawing lines perpendicular to the angle in arg3.  The {\sl
   vertexseq} is a colon ({\tt:}) separated sequence of vertex positions
   (or names) of the polygon in cw or ccw order. A {\sl colorseq} is of
   the form 0, r0,g0,b0, {\sl frac1},r1,g1,b1, {\sl frac2},r2,g2,b2,
   \ldots 1,rn,gn,bn with $0 < \hbox{\sl frac1} < \hbox{\sl frac2} \ldots 1$ }%
\macrodef{ShadeObject}{ShadeObject}{({\sl drawroutine, n, colorseq})}%
  {dpictools}%
  {$\;\;$
  Fill an area in a {\tt []} block with graded color
  defined by {\sl colorseq,} an indexed sequence of rgb colors:
    \fra{0}{\sl,r0,g0,b0,}
    \fra{1}{\sl,r1,g1,b1,}
    \ldots
    \fra{n}{\sl,rn,gn,bn} with\\
  $0 \leq \fra{0} < \fra{1}$
   $< \fra{2} <$
   $\ldots$ 
   $\fra{n} \leq 1.$\\  
  (Often $\fra{0} = 0$ and $\fra{n} = 1.$)
  The dpic macro\\
  {\sl drawroutine{\tt(}frac, r, g, b{\tt)}}\\
  typically draws a colored line
  and must be defined according to the area to be filled.
  This routine is called $n{+}1$ times for
  $\fra{}= \fra{0},\;$
   $\fra{0} + 1/n\times(\fra{n}{-}\fra{0}),$
   $\fra{0} + 2/n\times(\fra{n}{-}\fra{0}),$
   \ldots
   \fra{n}\\
  (i.e., often
  $\fra{} = 0, 1/n, 2/n, \ldots 1$)\\
  with rgb
  arguments interpolated in hsv space between {\sl colorseq} points
  (which are specified in rgb-space).
  Example (shade a box with 101 graded-color lines):\\ 
  {\tt B: box\\
   define HorizShade \lbr\ line right B.wid \textbackslash\\
   $\;$ from (0,-(\char36{}1)*B.ht) \textbackslash\\
   $\;$ outlined rgbstring(\char36{}2,\char36{}3,\char36{}4) \rbr;}\\
  {\tt ShadeObject(HorizShade, B.ht/lthick, 0,1,0,0, 1,0,0,1) at B}.
  }%
\macrodef{shadowed}{shadowed}{(box|circle|ellipse|line,
  {\sl position spec, keys})}%
  {gen}%
  { Object with specified shadow.  {\sl possspec} is e.g.,
    {\tt with .w at ...} or {\tt at} {\sl position}.
    The {\sl keys} are
       {\tt attrib=}{\sl object attributes}{\tt ;}
       {\tt shadowthick=}{\sl expr}{\tt ;} (default {\tt linethick*)}5/4),
       {\tt shadowcolor=}{\sl string}{\tt ;} (default {\tt "gray"}),
       {\tt shadowangle=}{\sl expr}{\tt ;} (default $-45$)
       for box only: {\tt rad=}{\sl expr}{\tt ;}%
 }%
\macrodef{shielded}{shielded}{(`{\sl two-terminal element}',
  L|U, {\sl line attributes})} {cct}%
  {shielding in a {\tt [ ]} box for
  two-terminal element. Arg2= blank
    (default) to enclose the element body; L for the left side with
    respect to drawing direction, R for right. Internal points {\tt
    .Start, .End,} and {\tt .C} are defined}%
\macrodef{s\_ht}{s_ht}{({\sl name},{\sl default})}%
  {gen}%
  {height of the most recent (or named) {\tt s\_box}%
    \seesect{Interaction:}}%
\macrodef{SIdefaults}{SIdefaults}{}%
  {gen}%
  {Sets {\tt scale = 25.4} for drawing units in mm, and sets
   pic parameters {\tt lineht = 12, linewid = 12, moveht = 12,
    movewid = 12,
    arcrad = 6, circlerad = 6, boxht = 12, boxwid = 18, ellipseht = 12,
    ellipsewid = 18, dashwid = 2, arrowht = 3, arrowwid = arrowht/2,}}%
\macrodef{sign\_}{sign_}{({\sl number})}%
  {gen}%
  {sign function}%
\macrodef{sinc}{sinc}{({\sl number})}%
  {gen}%
  {the $\hbox{sinc}(x)$ function}%
\macrodef{sind}{sind}{({\sl arg})}%
  {gen}%
  {sine of an expression in degrees}%
\macrodef{s\_init}{s_init}{({\sl name})}%
  {gen}%
  {initialize {\tt s\_box} string label to {\sl name} which should
  be unique
    \seesect{Interaction:}}%
\macrodef{Sin}{Sin}{({\sl integer})}%
  {gen}%
  {sine function, {\sl integer\/} degrees}%
\macrodef{sinusoid}{sinusoid}{({\sl amplitude, frequency, phase, tmin,
  tmax, linetype})}%
  {gen}%
  {draws a sinusoid over the interval $(t_{\hbox{\scriptsize min}},
  t_{\hbox{\scriptsize max}})$;
   e.g., to draw a dashed sine curve, amplitude {\sl a},
   of {\sl n} cycles of length {\sl x} from {\sl A}, {\tt
   sinusoid(a,twopi\_*n/x,-pi\_/2,0,x,dashed) with .Start at A}}%

\macrodef{sl\_box}{sl_box}{({\sl stem linespec, keys, stem object})}%
  {SLD}%
  {One-terminal SLD element: argument 1 is a \linespec\ to define the stem
   or, in the case of a zero-length stem, one of {\tt U, D, L, R,} or an
   angle in degrees, optionally followed by {\tt at {\sl position}}.
   The position is {\sl Here} by default.

   Argument 2 contains semicolon (;)-separated key-value attributes
   of the head:
   {\tt name={\sl{}Name}} (default {\sl Head});
   {\tt lgth={\sl{}expr}};
   {\tt wdth={\sl{}expr}};
   {\tt text="{\sl{}text}"},
   {\tt box={\sl{}box pic attributes}}.

   If argument 3 is null then a plain stem is drawn; if it is of the
   form {\tt S:}{\sl keys} or {\tt S$n$:}{\sl keys} an $n$-line slash
   symbol is overlaid on the stem; otherwise the keys are for an overlaid
   breaker, so that a {\tt C} specifies a default closed breaker, {\tt
   O} an open breaker, {\tt X,} {\tt /,} or \bsl\ for these marks, or
   \MR{sl_ttbox}{\tt sl\_ttbox} key-value pairs defining box attributes
   for the breaker (default name {\sl Br})

   \seesect{SingleLine:}}%
\macrodef{sl\_breaker}{sl_breaker}{({\sl linespec,} {\tt type=[A|C][D];}%
  {\sl ttbox args})}%
  {SLD}%
  {Two-terminal SLD element:
   type {\tt A} (the default) is for a box breaker; type 
   {\tt C} for a curved breaker; adding a {\tt D} puts drawout elements
   in the input and output leads.
   Otherwise, the arguments are as for 
   \MR{sl_ttbox}{\tt sl\_ttbox}
   \seesect{SingleLine:}}%
\macrodef{sl\_busbar}{sl_busbar}{({\sl linespec, np, keys})}%
  {SLD}%
  {Composite SLD element drawn in a {\tt [ ]} block.  A busbar is
   essentially a thick straight line 
   drawn along the {\sl linespec} with positions evenly distributed
   along it.  For example,
     {\tt line right\_; sl\_busbar(, up\_ 4.5, 5) with .P3 at Here}.

   Argument 1 is a \linespec\ to define the direction and length of the
     busbar (but not its position, since it is drawn in a {\tt [ ]} block).

   Argument 2 is the number $np$ of evenly spaced positions
     $P1, P2, \ldots Pnp$ along the line with $P1$ and $Pnp$ indented
     from the ends of the line.

   Argument 3 contains semicolon (;)-separated key-value attributes
     of the line:
   {\tt port=D} (for a dot at each port position);
   {\tt line=}{\sl pic line attributes}.
   {\tt indent=}{\sl indent distance}.
   \seesect{SingleLine:}}%
\macrodef{sl\_ct}{sl_ct}{%
    ({\tt at}{\sl position},{\sl keys},{\tt R|L|U|D|}{\sl degrees})}%
  {SLD}%
  {Composite SLD element drawn in a {\tt [ ]} block:

   The keys are as follows:
   {\tt type=L|N|S[n]} (default {\tt L;} {\tt S$n$} draws an $n$-line slash
     symbol, default 2); {\tt N} means no stem);
   {\tt scale={\sl expr} (default 1.0)};
   {\tt grnd={\sl expr} attached ground at given angle
     (type {\tt S} or {\tt N}))};
   {\tt sep={\sl{}expr}};
   {\tt stemlgth={\sl{}expr}};
   {\tt wdth={\sl{}expr}};
   {\tt direct=U|D|L|R|{\sl degrees}} (drawing direction).

   Key {\tt stemlgth} is the length of the leads at the start, centre, and end,
     with labeled ends {\sl Tstart, Tc,} and {\sl Tend}.
     The {\tt L} (default) variant also defines internal labels
     Internal labels {\sl L} and {\sl C} are included.

   Key {\tt sep} is the type-{\tt S} separation from the head to the centre
     of the slash symbol.

   Key {\tt scale} allows scaling (default scale 1.0) but, with \dpic,
     the {\tt scaled} directive can also be used.
   \seesect{SingleLine:}}%
\macrodef{sl\_disk}{sl_disk}{({\sl stem linespec, keys, breaker})}%
  {SLD}%
  {One-terminal SLD element: argument 1 is a \linespec\ to define the stem
   or, in the case of a zero-length stem, one of {\tt U, D, L, R,} or an
   angle in degrees, optionally followed by {\tt at {\sl position}}.
   The position is {\sl Here} by default.

   Argument 2 contains semicolon (;)-separated key-value attributes
   of the head:
   {\tt name={\sl{}Name}} (default {\sl Head});
   {\tt text="{\sl{}text}"};
   {\tt diam={\sl{}expr}};
   {\tt circle={\sl{}circle pic attributes}}.

   Argument 3 is null for no breaker in the stem, {\tt C} for a default
   closed breaker, {\tt O} for an open breaker, {\tt X,} {\tt /,} or \bsl\ for
   these marks, or
   \MR{sl_ttbox}{\tt sl\_ttbox}
   key-value pairs defining box attributes for the breaker
   (default name {\sl Br})
   \label{sl_disk}
   \seesect{SingleLine:}}%
\macrodef{sl\_drawout}{sl_drawout}{({\sl linespec, keys,} R)}%
  {SLD}%
  {Two-terminal SLD element: argument 1 is a \linespec\ as for ordinary
   two-terminal elements.

   Argument 2 contains semicolon (;)-separated key-value body attributes:\\
   {\tt type=T} (for truncated leads);
   {\tt lgth={\sl{}expr},}
   {\tt wdth={\sl{}expr}} (body size);
   {\tt name={\sl{}Name}} (default {\sl Body});
   {\tt line={\sl{}pic line attributes}}; (e.g., {\tt thick 2})

   Argument 3 is {\tt R} to reverse the direction of the drawn chevrons.
   \seesect{SingleLine:}}%
\macrodef{sl\_generator}{sl_generator}{({\sl stem linespec, keys, breaker})}%
  {SLD}%
  {One-terminal SLD element: argument 2 is
   {\tt type=AC|WT|BS|StatG|PV|Y|Delta} and,
   if {\tt type=PV,} the {\tt SL\_box} keys;
   otherwise, the {\tt sl\_disk} body keys.

   Argument 3 is null for no breaker in the stem, {\tt C} for a default
   closed breaker, {\tt O} for an open breaker, {\tt X,} {\tt /,} or \bsl\ for
   these marks, or
   \MR{sl_ttbox}{\tt sl\_ttbox}
   key-value pairs defining box attributes for the breaker
   (default name {\sl Br})
   \seesect{SingleLine:}}%
\macrodef{sl\_grid}{sl_grid}{({\sl stem linespec, keys, breaker})}%
  {SLD}%
  {One-terminal SLD element: argument 1 is a \linespec\ to define the stem
   or, in the case of a zero-length stem, one of {\tt U, D, L, R,} or an
   angle in degrees, optionally followed by {\tt at {\sl position}}.
   The position is {\sl Here} by default.

   Argument 2 contains semicolon (;)-separated key-value attributes
   of the head:
   {\tt name={\sl{}Name}} (default {\sl Head});
   {\tt lgth={\sl{}expr}};
   {\tt wdth={\sl{}expr}}.

   Argument 3 is null for no breaker in the stem, {\tt C} for a default
   closed breaker, {\tt O} for an open breaker, {\tt X,} {\tt /,} or \bsl\ for
   these marks, or
   \MR{sl_ttbox}{\tt sl\_ttbox}
   key-value pairs defining box attributes for the breaker
   (default name {\sl Br})
   \seesect{SingleLine:}}%
\macrodef{sl\_inverter}{sl_inverter}{({\sl ttbox args})}%
  {SLD}%
  {Two-terminal SLD element: the arguments are as for 
   \MR{sl_ttbox}{\tt sl\_ttbox}
   \seesect{SingleLine:}}%

\macrodef{sl\_lamp}{sl_lamp}{({\sl stem linespec, keys, breaker})}%
  {SLD}%
  {One-terminal SLD element: the arguments are as for 
   \MR{sl_disk}{\tt sl\_disk}
   \seesect{SingleLine:}}%
\macrodef{sl\_load}{sl_load}{({\sl stem linespec, keys, breaker})}%
  {SLD}%
  {One-terminal SLD element: argument 1 is a \linespec\ to define the stem
   or, in the case of a zero-length stem, one of {\tt U, D, L, R,} or an
   angle in degrees, optionally followed by {\tt at {\sl position}}.
   The position is {\sl Here} by default.

   Argument 2 contains semicolon (;)-separated key-value attributes
   of the head:
   {\tt name={\sl{}Name}} (default {\sl Head});
   {\tt lgth={\sl{}expr}};
   {\tt wdth={\sl{}expr}};
   {\tt head={\sl{}arrowhead pic attributes}}.

   Argument 3 is null for no breaker in the stem, {\tt C} for a default
   closed breaker, {\tt O} for an open breaker, {\tt X,} {\tt /,} or \bsl\ for
   these marks, or
   \MR{sl_ttbox}{\tt sl\_ttbox}
   key-value pairs defining box attributes for the breaker
   (default name {\sl Br})
   \seesect{SingleLine:}}%
\macrodef{sl\_meterbox}{sl_meterbox}{({\sl stem linespec, keys, breaker})}%
  {SLD}%
  {One-terminal SLD element: argument 1 is a \linespec\ to define the stem
   or, in the case of a zero-length stem, one of {\tt U, D, L, R,} or an
   angle in degrees, optionally followed by {\tt at {\sl position}}.
   The position is {\sl Here} by default.

   Argument 2 contains semicolon (;)-separated key-value attributes
   of the head:
   {\tt name={\sl{}Name}} (default {\sl Head});
   {\tt lgth={\sl{}expr}};
   {\tt wdth={\sl{}expr}};
   {\tt text="{\sl{}text}"},
   {\tt box={\sl{}box pic attributes}}.

   Argument 3 is null for no breaker in the stem, {\tt C} for a default
   closed breaker, {\tt O} for an open breaker, {\tt X,} {\tt /,} or \bsl\ for
   these marks, or
   \MR{sl_ttbox}{\tt sl\_ttbox}
   key-value pairs defining box attributes for the breaker
   (default name {\sl Br})
   \seesect{SingleLine:}}%
\macrodef{sl\_reactor}{sl_reactor}{({\sl stem linespec, keys, breaker keys,
    breaker keys})}%
  {SLD}%
  {Two-terminal SLD element: argument 1 is a \linespec\ as for ordinary
   two-terminal elements.

   Argument 2 contains semicolon (;)-separated key-value body attributes:
   {\tt name={\sl{}Name}} (default {\sl Body});
   {\tt diam={\sl{}expr}}.

   Argument 3 is null for no breaker in the input lead, {\tt C} for a default
   closed breaker, {\tt O} for an open breaker, {\tt X,} {\tt /,} or \bsl\ for
   these marks, or key-value pairs as above defining breaker attributes
   except that the default breaker name is {\sl BrI}.

   Argument 4 defines the breaker in the output lead as for argument 3
   except that the default breaker name is {\sl BrO}.
   \seesect{SingleLine:}}%
\macrodef{sl\_rectifier}{sl_rectifier}{({\sl ttbox args})}%
  {SLD}%
  {Two-terminal SLD element: the arguments are as for 
   \MR{sl_ttbox}{\tt sl\_ttbox}
   \seesect{SingleLine:}}%
\macrodef{sl\_slash}{sl_slash}{(at {\sl position, keys,}%
   [$n$:]R|L|U|D|{\sl degrees})}%
  {SLD}%
  {Slash symbol for SLD elements: draws $n$ slashes in a {\tt [] } block.
   The keys are
   {\tt lines={\sl line attributes,} e.g., dotted thick {\sl expr}};
   {\tt size=}{\sl expr} (default {\tt ht dimen\_/3}).
   \seesect{SingleLine:}}%
%
\macrodef{sl\_transformer3}{sl_transformer3}{({\sl linespec, keys,
    breaker keys, symbol keys})}%
  {SLD}%
  {Composite (block) SLD element: argument 1 is a \linespec\ that can be used
   to set the direction and distance between primary terminals but not
   position.

   Argument 2 contains semicolon (;)-separated key-value body attributes:
   {\tt name={\sl{}Name}} (default {\sl Body});
   {\tt type=S|C} (default {\tt S});
   {\tt scale={\sl expr}} (body size factor, default 1.0);
   {\tt direct=L|R} (default {\tt L}) direction of the tertiary
     circle and terminal relative to the drawing direction;
   {\tt body={\sl circle attributes}}.

   Argument 3 is colon (:)-separated sequence of up to three breaker
   attribute specifications for the input, output, and teriary breaker
   in order.  A null or blank means no breaker, {\tt tt\_breaker}
   specifications otherwise. Default breaker names are {\sl BrI}
   and {\sl BrO} as for
   {\tt sl\_transformer,} and {\sl Br} for the third breaker. 

   Argument 4 is colon (:)-separated sequence of up to three symbol
   specifications for the input, output, and teriary circle
   in order.  A null or blank means no symbol;
   {\tt Y} for a Y-symbol;
   {\tt Delta} for a $\Delta$ symbol;
   otherwise, other customization commands expanded in a {\tt \lbr\rbr} pair.
   \seesect{SingleLine:}}%
%
\macrodef{sl\_transformer}{sl_transformer}{({\sl linespec, keys,
    input breaker keys, output breaker keys,
    input circle inner object, output circle inner object})}%
  {SLD}%
  {Two-terminal SLD element: argument 1 is a \linespec\ as for ordinary
   two-terminal elements.

   Argument 2 contains semicolon (;)-separated key-value body attributes:
   {\tt name={\sl{}Name}} (default {\sl Body});
   {\tt scale={\sl expr}} (body size factor, default 1.0);
   {\tt type=I|S|A[R]} (default {\tt I}).

   Additional type {\tt I} keys are
   {\tt cycles={\sl{}integer}} (default 4);
   {\tt core=A|M[$n$]|P[$n$]|K[$n$]}, $n$={\sl integer} (default 2 lines).

   Additional type {\tt S} keys are
   {\tt body={\sl circle pic attributes}} e.g., {\tt shaded "{\sl color}"}.

   Type {\tt A} keys are
   {\tt body={\sl circle pic attributes}}.  Type {\tt AR} means right
   orientation.

   Argument 3 is null for no breaker in the input lead, {\tt C} for a default
   closed breaker, {\tt O} for an open breaker, {\tt X,} {\tt /,} or \bsl\ for
   these marks, or
   key-value pairs as above defining breaker attributes
   except that the default breaker name is {\sl BrI}.

   Argument 4 defines the breaker in the output lead as for argument 3
   except that the default breaker name is {\sl BrO}.

   Argumentss 5 and 6 for the input and output circles respectively are:
   {\tt Y} for a Y-symbol;
   {\tt YN} for a Y-symbol with ground;
   {\tt Delta} for a $\Delta$ symbol;
   otherwise, other customization commands expanded in a {\tt \lbr\rbr} pair.
   \seesect{SingleLine:}}%
%
\macrodef{sl\_ttbox}{sl_ttbox}{({\sl linespec, keys, input breaker keys,
    output breaker keys})}%
  {SLD}%
  {Two-terminal SLD element: argument 1 is a \linespec\ as for ordinary
   two-terminal elements.

   Argument 2 contains semicolon (;)-separated key-value body attributes:
   {\tt name={\sl{}Name}} (default {\sl Body});
   {\tt lgth={\sl{}expr}};
   {\tt wdth={\sl{}expr}};
   {\tt text="{\sl{}text}"};
   {\tt box={\sl{}box pic attributes}};
   {\tt supp={\sl{}additional {\tt rotbox} commands}}.

   Argument 3 is null for no breaker in the input lead, {\tt C} for a default
   closed breaker, {\tt O} for an open breaker, {\tt X,} {\tt /,} or \bsl\ for
   these marks, or
   key-value pairs as above defining breaker attributes
   except that the default breaker name is {\sl BrI}.

   Argument 4 defines the breaker in the output lead as for argument 3
   except that the default breaker name is {\sl BrO}.
   \label{sl_ttbox}
   \seesect{SingleLine:}}%
\macrodef{s\_name}{s_name}{}%
  {gen}%
  {the value of the last {\tt s\_init} argument
    \seesect{Interaction:}}%
\macrodef{sourcerad\_}{sourcerad_}{}%
  {cct}%
  {default source radius}%
\macrodef{slantbox}{slantbox}{%
  ({\sl wid, height, x offset, y offset,  attributes})}%
  {dpictools}%
  {$\;\;$ Trapezoid formed from a box with top corners displaced right by
    $x$ offset and right corners displaced up by $y$ offset.}%
\macrodef{source}{source}{(\linespec, {\sl char or chars}, {\sl diameter},R,
  {\sl body attributes, body name})}%
  {cct}%
  {Source; arg2 blank or:\\
               {\tt AC}: AC source;\\
               {\tt B}: bulb;\\
               {\tt F}: fluorescent;\\
               {\tt G}: generator;\\
               {\tt H}: step function;\\
               {\tt I}: current source;\\
               {\tt i}: alternate current source;\\
               {\tt ii}: double arrow current source;\\
               {\tt ti}: truncated-bar alternate current source;\\
               {\tt L}: lamp;\\
               {\tt N}: neon;\\
               {\tt NA}: neon 2;\\
               {\tt NB}: neon 3;\\
               {\tt P}: pulse;\\
               {\tt Q}: charge;\\
               {\tt R}: ramp;
               {\tt S}: sinusoid;\\
               {\tt SC}: quarter arc, {\tt SCr} right orientation;\\
               {\tt SE}: arc, {\tt SEr} right orientation;\\
               {\tt T}: triangle;\\
               {\tt U}: square-wave;\\
               {\tt V}: voltage source;\\
               {\tt X}: interior X;\\
               {\tt v}: alternate voltage source;\\
               {\tt tv}: truncated-bar alternate voltage source;\\
               other: custom interior label or waveform;\\
        arg 4: R: reversed polarity;\\
        arg 5 modifies the circle (body) with e.g., color or fill;\\
        arg 6 names the body {\tt [ ]} block
   \seesect{Twoterminal:}}%
\macrodef{speaker}{speaker}{( U|D|L|R|{\sl degrees},{\sl size},H,attributes)}%
  {cct}%
  {speaker, {\sl In1} to {\sl In7} defined; {\tt H}: horn
   \seesect{Composite:}}%
\macrodef{sprod3}{sprod3}{({\sl scalar, vec1, vec2})}%
  {dpictools}%
  {$\;\;$ Multiplied vector by scalar arg1: {\sl vec2 = vec1 * arg1}.}%
\macrodef{sprod3D}{sprod3D}{(a,x,y,z)}%
  {3D} {scalar product of triple x,y,z by arg1}%
\macrodef{sp\_}{sp_}{}%
  {gen}%
  {evaluates to medium space for gpic strings}%
\macrodef{sqrta}{sqrta}{({\sl arg})}%
  {gen}%
  {square root of the absolute value of {\sl arg}; i.e.,
   {\tt sqrt(abs({\sl arg}))}}%
\macrodef{SQUID}{SQUID}{({\sl n, diameter, initial angle}, {\tt ccw|cw})}%
  {cct}%
  {Superconducting quantum interface device
   with {\sl }n junctions labeled {\tt J1, ... J}{\sl n} placed around
   a circle with initial angle -90 deg (by default) with respect to the
   current drawing direction. The default diameter is {\tt dimen\_} }%
\macrodef{s\_}{s_}{}%
  {gen}%
  {.s with respect to current direction}%
\macrodef{stackargs\_}{stackargs}{(`{\sl stackname}',{\sl args})}%
  {gen}%
  {Stack arg 2, arg 3, ... onto the named stack up to a blank arg}%
\macrodef{stackcopy\_}{stackcopy_}{(`{\sl name 1}',`{\sl name 2}')}%
  {gen}%
  {Copy stack 1 into stack 2, preserving the order of pushed elements}%
\macrodef{stackdo\_}{stackdo}{(`{\sl stackname}',{\sl commands})}%
  {gen}%
  {Empty the stack to the first blank entry, performing arg 2}%
\macrodef{stackexec\_}{stackexec_}{(`{\sl name 1}',`{\sl name 2}',%
 {\sl commands})}%
  {gen}%
  {Copy stack 1 into stack 2, performing arg3 for each nonblank
  entry}%
\macrodef{stackprint\_}{stackprint_}{(`{\sl stack name}')}%
  {gen}%
  {Print the contents of the stack to the terminal}%
%\macrodef{stackpromote\_}{stackpromote_}{({\sl prefix},% 
% `{\sl stack name}',{\sl In name})}%
%  {gen}%
%  {Define locations {\tt In1} or {\sl In name }{\tt 1}, $\ldots$ corresponding %   to the locations in stack {\sl stack name}, as created by the
%   {\tt AutoGate} and {\tt Autologic} macros.  Each location is prefixed
%   by argument 1 ``.''}%
\macrodef{stackreverse\_}{stackreverse_}{(`{\sl stack name}')}%
  {gen}%
  {Reverse the order of elements in a stack, preserving the name}%
\macrodef{stacksplit\_}{stacksplit_}{(`{\sl stack name}',{\sl string},{\sl
   separator})}%
  {gen}%
  {Stack the fields of {\sl string} left to right separated
  by nonblank
    {\sl separator} (default .).  White space preceding the fields
    is ignored.}%
\macrodef{sum3}{sum3}{({\sl vec1, vec2, vec3})}%
  {dpictools}%
  {$\;\;$ The 3-vector sum
    $\hbox{\sl vec3} = \hbox{\sl vec1} + \hbox{\sl vec2}$.}%
\macrodef{sum3D}{sum3D}{({\sl x1,y1,z1,x2,y2,z2})}%
  {3D} {sum of two triples}%
\macrodef{sum\_}{sum_}{({\sl a},{\sl b})}%
  {gen}%
  {binary sum}%
\macrodef{sus}{sus}{({\sl linespec, chars, label})}%
  {cct}%
  {Wrapper to place an SUS thyristor as a two-terminal element with
   {\tt [ ]} block label given by the third argument
    \seesect{Semiconductors:}}%
\macrodef{svec\_}{svec_}{({\sl x},{\sl y})}%
  {log}%
  {scaled and rotated grid coordinate vector}%
\macrodef{s\_wd}{s_wd}{({\sl name},{\sl default})}%
  {gen}%
  {width of the most recent (or named) {\tt s\_box}%
    \seesect{Interaction:}}%
\macrodef{switch}{switch}{(\linespec,L|R,[C|O][D],[B|D])}%
  {cct}%
  {SPST switch (wrapper for bswitch, lswitch, and dswitch),
    arg2: {\tt R}: right orientation (default {\tt L} for left);\\
     if arg4=blank (knife switch): arg3 = {\tt [O|C][D][A]},
       {\tt O}: opening, {\tt C}: closing, {\tt D}:dots,
       {\tt A}: blade arrowhead;\\
     if arg4=B (button switch): arg3 = 
       {\tt O|C}: {\tt O}: normally open, {\tt C}: normally closed;\\
     if arg4={\tt D}: arg3 = same as for dswitch \seesect{Twoterminal:}}%
\macrodef{sw\_}{sw_}{}%
  {gen}%
  {.sw with respect to current direction}%

\Letter{T}%

\macrodef{tapped}{tapped}{(`{\sl two-terminal element}',
  [{\sl arrowhd} | type={\sl arrowhd};name={\sl Name}],
    {\sl fraction, length, fraction, length,} $\cdots$)}%
  {cct}%
  {Draw the two-terminal element with taps in a [ ] block (see
  {\tt addtaps}).
   {\sl arrowhd} = blank or one of {\tt . - <- -> <->}.  Each fraction
   determines the position along the element body of the tap.  A negative
   length draws the tap to the right of the current direction; positive
   length to the left.  Tap names are Tap1, Tap2, $\cdots$ by default
   or Name1, Name2, $\cdots$ if specified.  Internal block names are
   {\tt .Start, .End,} and {.C} corresponding to the drawn element,
   and the tap names \seesect{Composite:} }%
\macrodef{ta\_xy}{ta_xy}{({\sl x, y})}%
  {cct}%
  {macro-internal coordinates adjusted for {\tt L|R}}%
\macrodef{tbox}{tbox}{({\sl text,wid,ht},<|>|<>,{\sl attributes})}%
  {cct}%
  {Pointed terminal box. The {\sl text} is placed at the rectangular
  center
   in math mode unless the text begins with {\tt "} or {\tt sprintf} in
   which case the arument is used literally.  Arg 4 determines whether
   the point is forward, backward, or both with respect to the current
   drawing direction.
    \seesect{Composite:}}%
\macrodef{tconn}{tconn}{({\sl linespec, chars}|{\sl keys}, {\sl wid})}%
  {cct}%
  {Terminal connector drawn on a linespec, with head enclosed in a {\tt []}%
   block.  The permissible {\sl chars} are:
   {\tt > | >{}> | < | <{}< | A | AA | M | O | OF}.
   Type {\tt O} draws a node (circle); {\tt OF} a filled circle.
   Type {\tt M} is a black bar; {\tt A} is an open arc end; type {\tt AA}
   a double open arc.  Type {\tt >} (the default) is an arrow-like output
   connector; {\tt <} and {\tt <<} input connectors.  Arg 3 is arrowhead
   width or circle diameter when key-value pairs are not used.
   If keys are specified, they are {\tt type=}{\sl chars} as previously;
   {\tt wdth=}{\sl expr}; {\tt lgth=}{\sl expr}; {\tt sep=}{\sl expr};
   {\tt head=}{\sl attributes except} {\tt lgth, wdth.}
   The key {\tt sep=} is the double-head separation
    \seesect{Composite:}}%
\macrodef{testexpr}{testexpr}{({\sl variable, expr1, expr2, \ldots})}%
  {dpictools}%
  {$\;\;$ Set the variable given by arg1 to the index of the first true
  alternative in a sequence of logical expressions, e.g.,
  {\tt testexpr(i, 1>2, 1<2 )} sets {\sl i} to 2. The variable is set to 0
   if no test is true.}%
\macrodef{tgate}{tgate}{({\sl linespec,} [B][R|L])}%
  {cct}%
  {transmission gate, {\tt B=} ebox type; {\tt L=} oriented left
    \seesect{Semiconductors:}}%
\macrodef{thermocouple}{thermocouple}{({\sl linespec, wid, ht,} L|R [T])}%
  {cct}%
  { Thermocouple drawn to the left (by default) of the
    {\sl linespec} line.  A {\tt T} argument truncates the leads so
    only the two branches appear.  {\tt R=} right orientation.
    \seesect{Twoterminal:}}%
\macrodef{thicklines\_}{thicklines_}{({\sl number})}%
  {gen}%
  {set line thickness in points}%
\macrodef{thinlines\_}{thinlines_}{({\sl number})}%
  {gen}%
  {set line thickness in points}%
\macrodef{threeD\_init}{threeD_init}{}%
  {3D} {initialize 3D transformations (reads {\tt lib3D.m4})}%
\macrodef{thyristor}{thyristor}{(\linespec,%
 {\tt [SCR|SCS|SUS|SBS|IEC][{\sl chars}]})}%
  {cct}%
  {Composite thyristor element in {\tt []}block, types:\\
                     {\tt SCR}: silicon controlled rectifier (default),\\
                     {\tt SCS}: silicon controlled switch,\\
                     {\tt SUS}: silicon unilateral switch,\\
                     {\tt SBS}: silicon bilateral switch,\\
                     {\tt IEC}: type IEC.\\
   {\sl Chars} to modify or define the element:\\
                     {\tt K}: open arrowheads,\\
                     {\tt A}: arrowhead,\\
                     {\tt F}: half arrowhead,\\
                     {\tt B}: bidirectional diode,\\
                     {\tt E}: adds envelope,\\
                     {\tt H}: perpendicular gate (endpoint {\sl G}),\\
                     {\tt N}: anode gate (endpoint {\sl Ga}),\\
                     {\tt U}: centre line in diodes,\\
                     {\tt V}: perpendicular gate across arrowhead centre,\\
                     {\tt R}: right orientation,\\
                     {\tt E}: envelope
    \seesect{Semiconductors:}}%
\macrodef{thyristor\_t}{thyristor_t}{({\sl linespec, chars, label})}%
  {cct}%
  {Wrapper to place a thyristor as a two-terminal element with
   {\tt [ ]} block label given by the third argument
    \seesect{Semiconductors:}}%
\macrodef{tikznode}{tikznode}{({\sl \Tikz node name, position}) }%
  {pgf}%
  {insert \Tikz code to define a zero-size \Tikz node at {\sl location}%
    (default {\tt Here}) to assist with inclusion of \pic code output
    in \Tikz diagrams.  This macro must be invoked in the outermost
    \pic scope.  \seesect{Tikzwithpic:}}%
\macrodef{tline}{tline}{(\linespec,{\sl wid},{\sl ht}) }%
  {cct}%
  {transmission line, manhattan direction\seesect{Twoterminal:}}%
\macrodef{ToPos}{ToPos}{({\sl position}, U|D|L|R|{\sl degrees}, {\sl length}) }%
  {gen}%
  {Evaluates to {\tt from {\sl position} - Rect\_({\sl length, angle}) to
   {\sl position}} from the polar-coordinate data in the arguments }%
\macrodef{transformer}{transformer}{(\linespec,L|R,{\sl np},%
[A|P][W|L][D1|D2|D12|D21],{\sl ns})}%
  {cct}%
  {2-winding transformer or choke with terminals {\sl P1, P2, TP, S1, S2, TS}:\\
    arg2: {\tt L}: left, {\tt R}: right,\\
    arg3: np primary arcs,\\
    arg5: ns secondary arcs,\\
    arg4:
    {\tt A}: air core,\\
    {\tt P}: powder (dashed) core,\\
    {\tt W}: wide windings,\\
    {\tt L}: looped windings,\\
    {\tt D1}: phase dots at {\sl P1} and {\sl S1} end;\\
    {\tt D2}: at {\sl P2} and {\sl S2} end;\\
    {\tt D12}: at {\sl P1} and {\sl S2} end;\\
    {\tt D21} at {\sl P2} and {\sl S1} end
   \seesect{Composite:}}%
\macrodef{tr\_xy\_init}{tr_xy_init}{({\sl origin, unit size, sign })}%
  {cct}%
  {initialize {\tt tr\_xy}}%
\macrodef{tr\_xy}{tr_xy}{({\sl x, y})}%
  {cct}%
  {relative macro internal coordinates adjusted for {\tt L|R}}%
\macrodef{tstrip}{tstrip}{(R|L|U|D|{\sl degrees, nterms, chars})}%
  {cct}%
  {terminal strip, chars:
   I (invisible terminals), C (default circle terminals), D (dot terminals),
   O (omitted separator lines), {\tt wid=}value{\tt ;} total strip width,
   {\tt ht=}value{\tt ;} strip height, {\tt box={\sl shaded etc.};}
    \seesect{Composite:}}%
\macrodef{ttmotor}{ttmotor}{({\sl linespec, string, diameter, brushwid,
brushht})}%
  {cct}%
  {motor with label\seesect{Twoterminal:}}%
\macrodef{twopi\_}{twopi_}{}%
  {gen}%
  {$2\pi$}%

\Letter{U}%

\macrodef{ujt}{ujt}{(\linespec,R,P,E)}%
  {cct}%
  {unijunction transistor, right, P-channel, envelope
    \seesect{Semiconductors:}}%
\macrodef{unit3D}{unit3D}{(x,y,z)}%
  {3D} {unit triple in the direction of triple x,y,z}%
\macrodef{up\_\_}{up__}{}%
  {gen}%
  {up with respect to current direction}%
\macrodef{up\_}{up_}{}%
  {gen}%
  {set current direction up \seesect{Placing:}}%

\Letter{V}%

\macrodef{variable}{variable}{(`{\sl element}', {\sl chars}, [+|-]{\sl angle},
   {\sl length}, at position)}%
% {\tt [A|P|L|[u]N|[u]NN][C|S]},
  {cct}%
  {Overlaid arrow or line to indicate variable 2-terminal element:
    The {\sl chars} are\\
    {\tt A}: arrow,\\
    {\tt P}: preset,\\
    {\tt L}: linear,\\
    {\tt N}:  symmetric nonlinear,\\
    {\tt C}: continuous,\\
    {\tt S}: setpwise;\\
    {\tt u} changes the nonlinearity direction.  The angle is absolute
    but preceding it with a sign makes the angle (often -30 or -45)
    relative to the element drawing direction.\\
    If arg5 is blank the symbol is placed over the last {\tt []} block
   \seesect{Twoterminal:}}%
\macrodef{Vcoords\_}{Vcoords_}{({\sl position})}%
  {gen}%
  {The $x, y$ coordinate pair of the position}%
\macrodef{Vdiff\_}{Vdiff_}{({\sl position},{\sl position})}%
  {gen}%
  {{\tt Vdiff\_(A,B)} evaluates to {\tt A-(B)} with dpic, {\tt A-(B.x,B.y)}
   with gpic}%
\macrodef{vec\_}{vec_}{({\sl x},{\sl y})}%
  {gen}%
  {position rotated with respect to current direction}%
\macrodef{vec3}{vec3}{({\sl vector})}%
  {dpictools}%
  {$\;\;$ Expands to the threee components of the vector argument separated
    by commas.}%
\macrodef{View3D}{View3D}{}%
  {3D} {The view vector (triple) defined by {\tt setview({\sl azimuth,
   elevation, rotation})}. The {\tt project} macro projects onto the plane
   through {\tt (0,0)} and orthogonal to this vector.}%
\macrodef{vlength}{vlength}{({\sl x},{\sl y})}%
  {gen}%
  {vector length $\sqrt{x^2+y^2}$}%
\macrodef{vperp}{vperp}{({\sl linear object})}%
  {gen}%
  {unit-vector pair CCW-perpendicular to linear object}%
\macrodef{Vperp}{Vperp}{({\sl position name}, {\sl position name})}%
  {gen}%
  {unit-vector pair CCW-perpendicular to line joining two named positions}%
\macrodef{vrot\_}{vrot_}{({\sl x},{\sl y},{\sl xcosine},{\sl ycosine})}%
  {gen}%
  {rotation operator}%
\macrodef{vscal\_}{vscal_}{({\sl number},{\sl x},{\sl y})}%
  {gen}%
  {vector scale operator}%
\macrodef{Vsprod\_}{Vsprod_}{({\sl position}, {\sl expression})}%
  {gen}%
  {The vector in arg 1 multiplied by the scalar in arg 2}%
\macrodef{Vsum\_}{Vsum_}{({\sl position},{\sl position})}%
  {gen}%
  {{\tt Vsum\_(A,B)} evaluates to {\tt A+B} with dpic, {\tt A+(B.x,B.y)}
   with gpic}%

\Letter{W}%

\macrodef{while\_}{while_}{(`{\sl test}',`{\sl actions}')}%
  {gen}%
  {Integer m4 while loop}%
\macrodef{wid\_}{wid_}{}%
  {gen}%
  {width with respect to current direction}%
\macrodef{winding}{winding}{(L|R, {\sl diam, pitch, turns, core wid,
   core color})}%
  {cct}%
  {core winding drawn in the current direction; {\tt R}: right-handed
   \seesect{Composite:}}%
\macrodef{w\_}{w_}{}%
  {gen}%
  {.w with respect to current direction}%
\macrodef{XOR\_gate}{XOR_gate}{({\sl n},N)}%
  {log}%
  {`xor' gate, 2 or {\sl n\/} inputs; {\tt N}: negated input.
   Otherwise, arg1 can be a sequence of letters {\tt P|N} to define
   normal or negated inputs.
    \seesect{Logicgates:}}%
\macrodef{XOR\_off}{XOR_off}{}%
  {log}%
  {XOR and NXOR offset of input face}%

\Letter{X}%

\macrodef{xtal}{xtal}{(\linespec,{\sl keys})}%
  {cct}%
  {Quartz crystal. The {\sl keys} are
     {\tt type=N} (default) or {\tt type=R} (round);\\
            type {\tt N} keys:\\
              {\tt lgth=}{\sl expr} (body length);\\
              {\tt wdth=}{\sl expr} (body width);\\
              {\tt bxwd=}{\sl expr} (body inner box width);\\
              {\tt box=} box attributes ({\tt shaded} $\ldots$);\\
            type {\tt R} keys:\\
              {\tt outerdiam=}{\sl expr};\\
              {\tt innerdiam=}{\sl expr};\\
              {\tt outer=} outer circle attributes ({\tt dotted} $\ldots$);\\
              {\tt inner=} inner circle attributes ({\tt shaded} $\ldots$)%
\seesect{Twoterminal:}}%
\macrodef{xtract}{xtract}{({\sl string, substr1, substr2, $\ldots$})}%
  {gen}%
  {returns substrings if present}%

\Letter{Y}%

\macrodef{Ysymbol}{Ysymbol}{(at {\sl position},keys,
    U|D|L|R|{\sl degrees}) (default {\tt U} for up)}%
  {cct}%
  {Y symbol for power-system diagrams
   {\sl keys:} {\tt size={\sl expression}; type=G}}%
%  \end{tabbing}%
