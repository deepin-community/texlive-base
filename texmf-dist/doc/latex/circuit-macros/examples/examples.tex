% PSTricks examples.tex

\documentclass[11pt]{article}
\usepackage[dvipsnames]{xcolor}
\usepackage{times}
\input mode
\usepackage{rotating}
\usepackage{graphicx}
\usepackage{boxdims}
%\usepackage{upgreek}
\usepackage{siunitx}
\usepackage{amssymb}

\input header

\global\psttrue

\begin{document}
  \hfill
  {\large\bf Examples:
    Version 10.6
}
  \hfill\break

  This is a collection of diagrams the author has had occasion to produce
  using m4 circuit macros and others, and dpic or gpic.  In some cases
  there are other or better m4 or pic constructs for producing the same
  drawings, but names of the actual source-files are shown for reference.
  Some of the later examples test the boundaries of what can be done
  when employing a ``little language'' like pic.  Most of the examples
  can be processed using either dpic~-p, dpic~-g, or, with exceptions,
  gpic~-t, but the possibility of other postprocessing has meant that
  sometimes the source is slightly more complicated than it would be if
  only one workflow had been assumed.  The most simplicity and elegance
  is achieved by sticking to one pic interpreter and one postprocessor.

  Color and other embellishments are not included in the standards
  documents for circuit elements but examples of their use to call
  attention to particular elements are included.

  This document duplicates a few diagrams from the manual
  Circuit\_macros.pdf.  There are also a few files in the
  examples directory that are not included in this document. To process
  {\sl file}.m4, for example, type "make {\sl file}.pdf".

\input files
\endinput
