\newpage
\section{Working with lua} \label{calc_with_lua}

\subsubsection{Option \code{lua}} % (fold)
\label{ssub:option_code_lua}

% subsubsection option_code_lua (end)
You can now use the \ItkzPopt{tkz-euclide}{lua} option with \tkzname{\tkznameofpack} version 5.
You just have to write in your preamble

 |usepackage[lua]{tkz-euclide}|. 
 Obviously, you'll need to compile with LuaLaTeX. Nothing changes for the syntax.

Without the option you can use \tkzname{\tkznameofpack} with the proposed code of version 4.25.

This version is not yet finalized although the documentation you are currently reading has been compiled with this option.

Some information about the method used and the results obtained. Concerning the method, I considered two possibilities. The first one was simply to replace everywhere I could the calculations made by \code{xfp} or sometimes by \code{lua}. This is how I went from \code{fp} to \code{xfp} and now to \code{lua}. The second and more ambitious possibility would have been to associate to each point a complex number and to make the calculations on the complexes with \code{lua}. Unfortunately for that I have to use libraries for which I don't know the license. 

Otherwise the results are good. This documentation with \code{LualaTeX} and \code{xfp} compiles in 47s while with \code{lua} it takes only 30s for 236 pages.

Another document of 61 pages is compiled 16s with \code{pdflaTeX} and \code{xfp} and 13s with  \code{LualaTeX} and \code{xfp}.

This documentation compiles with |\usepackage{tkz-base}|  and |\usepackage[lua]{tkz-euclide}| but I didn't test all the interactions thoroughly. 

\subsubsection{Option \code{mini}} % (fold)
\label{ssub:option_code_mini}

When you use \tkzNamePack{tkz-elements} solely to determine the points in your figures, it is not necessary to load all the modules of \tkzname{\tkznameofpack}. In this case, by using the \ItkzPopt{tkz-euclide}{mini} option |\usepackage[mini]{tkz-euclide}| , you will only load the modules necessary for the drawings.

% subsubsection option_code_mini (end)
\endinput