\section{Different authors}

\subsection{Code from Andrew Swan}

\begin{tkzexample}[latex=7cm]
\begin{tikzpicture}[scale=1.25]
\def\radius{4}
\def\angle{40}
\pgfmathsetmacro{\htan}{tan(\angle)}
\tkzDefPoint(0,0){A} \tkzDefPoint(0,\radius){F}
\tkzDefPoint(\radius,0){B}
\tkzDefPointBy[rotation= center A angle \angle](B)
\tkzGetPoint{C}
\tkzDefLine[perpendicular=through B,K=1](A,B)
\tkzGetPoint{b}
\tkzInterLL(A,C)(B,b) \tkzGetPoint{D}
\tkzDefLine[perpendicular=through C,K=-1](A,B)
\tkzGetPoint{c}
\tkzInterLL(C,c)(A,B) \tkzGetPoint{E}
\tkzDrawSector[fill=blue,opacity=0.1](A,B)(C)
\tkzDrawArc[thin](A,B)(F)
\tkzMarkAngle(B,A,C)
\tkzLabelAngle[pos=0.8](B,A,C){$x$}
\tkzDrawPolygon(A,B,D)
\tkzDrawSegments(C,B)
\tkzDrawSegments[dashed,thin](C,E)
\tkzLabelPoints[below left](A)
\tkzLabelPoints[below right](B)
\tkzLabelPoints[above](C)
\tkzLabelPoints[above right](D)
\begin{scope}[/pgf/decoration/raise=5pt]
\draw [decorate,decoration={brace,mirror,
   amplitude=10pt},xshift=0pt,yshift=-4pt]
(A) -- (B) node [black,midway,yshift=-20pt]
{\footnotesize $1$};
\draw [decorate,decoration={brace,amplitude=10pt},
       xshift=4pt,yshift=0pt]
(D) -- (B) node [black,midway,xshift=27pt]
{\footnotesize $\tan x$};
\draw [decorate,decoration={brace,amplitude=10pt},
       xshift=4pt,yshift=0pt]
(E) -- (C) node [black,midway,xshift=-27pt]
{\footnotesize $\sin x$};
\end{scope}
\end{tikzpicture}
\end{tkzexample}


\subsection{Example: Dimitris Kapeta}

You need in this example to use \tkzname{mkpos=.2} with \tkzcname{tkzMarkAngle} because the measure of $ \widehat{CAM}$ is too small.
Another possiblity is to use \tkzcname{tkzFillAngle}.


\begin{tkzexample}[latex=7cm,small]
\begin{tikzpicture}[scale=1]
  \tkzDefPoint(0,0){O}
  \tkzDefPoint(2.5,0){N}
  \tkzDefPoint(-4.2,0.5){M}
  \tkzDefPointBy[rotation=center O angle 30](N)
  \tkzGetPoint{B}
  \tkzDefPointBy[rotation=center O angle -50](N)
  \tkzGetPoint{A}
  \tkzInterLC[common=B](M,B)(O,B) \tkzGetFirstPoint{C}
  \tkzInterLC[common=A](M,A)(O,A) \tkzGetFirstPoint{A'}
  \tkzMarkAngle[mkpos=.2, size=0.5](A,C,B)
  \tkzMarkAngle[mkpos=.2, size=0.5](A,M,C)
  \tkzDrawSegments(A,C M,A M,B A,B)
  \tkzDrawCircle(O,N)
  \tkzLabelCircle[above left](O,N)(120){%
                 $\mathcal{C}$}
  \begin{scope}[veclen] 
  \tkzMarkAngle[mkpos=.2, size=1.2](C,A,M)
  \end{scope}
  \tkzDrawPoints(O, A, B, M, B, C)
  \tkzLabelPoints[right](O,A,B)
  \tkzLabelPoints[above left](M,C)
  \tkzLabelPoint[below left](A'){$A'$}
\end{tikzpicture}
\end{tkzexample}


\subsection{Example :  John Kitzmiller }
Prove that $\dfrac{AC}{CE}=\dfrac{BD}{DF}$.

Another interesting example from John, you can see how to use some extra options like\\\ \tkzname{decoration} and \tkzname{postaction}  from \TIKZ\ with \tkzname{tkz-euclide}.

\begin{tkzexample}[vbox,small]
\begin{tikzpicture}[scale=1.5,decoration={markings,
  mark=at position 3cm with {\arrow[scale=2]{>}}}]
  \tkzDefPoints{0/0/E, 6/0/F, 0/1.8/P, 6/1.8/Q, 0/3/R, 6/3/S}
  \tkzDrawLines[postaction={decorate}](E,F P,Q R,S)
  \tkzDefPoints{3.5/3/A, 5/3/B}
  \tkzDrawSegments(E,A F,B)
  \tkzInterLL(E,A)(P,Q) \tkzGetPoint{C}
  \tkzInterLL(B,F)(P,Q) \tkzGetPoint{D}
  \tkzLabelPoints[above right](A,B)
  \tkzLabelPoints[below](E,F)
  \tkzLabelPoints[above left](C)
  \tkzDrawSegments[style=dashed](A,F)
  \tkzInterLL(A,F)(P,Q) \tkzGetPoint{G}
  \tkzLabelPoints[above right](D,G)
  \tkzDrawSegments[color=teal, line width=3pt, opacity=0.4](A,C A,G)
  \tkzDrawSegments[color=magenta, line width=3pt, opacity=0.4](C,E G,F)
  \tkzDrawSegments[color=teal, line width=3pt, opacity=0.4](B,D)
  \tkzDrawSegments[color=magenta, line width=3pt, opacity=0.4](D,F)
\end{tikzpicture}
\end{tkzexample}


\subsection{Example 1: from Indonesia}

\begin{tkzexample}[vbox,small]
\begin{tikzpicture}[scale=3]
   \tkzDefPoints{0/0/A,2/0/B}
   \tkzDefSquare(A,B) \tkzGetPoints{C}{D}
   \tkzDefPointBy[rotation=center D angle 45](C)\tkzGetPoint{G}
   \tkzDefSquare(G,D)\tkzGetPoints{E}{F}
   \tkzInterLL(B,C)(E,F)\tkzGetPoint{H}
   \tkzFillPolygon[gray!10](D,E,H,C,D)
   \tkzDrawPolygon(A,...,D)\tkzDrawPolygon(D,...,G)
   \tkzDrawSegment(B,E)
   \tkzMarkSegments[mark=|,size=3pt,color=gray](A,B B,C C,D D,A E,F F,G G,D D,E)
   \tkzMarkSegments[mark=||,size=3pt,color=gray](B,E E,H)
   \tkzLabelPoints[left](A,D)
   \tkzLabelPoints[right](B,C,F,H)
   \tkzLabelPoints[above](G)\tkzLabelPoints[below](E)
   \tkzMarkRightAngles(D,A,B D,G,F)
\end{tikzpicture}
\end{tkzexample}

\subsection{Example 2: from Indonesia}
\begin{tkzexample}[vbox,overhang,small]
  \begin{tikzpicture}[pol/.style={fill=brown!40,opacity=.2},
      seg/.style={tkzdotted,color=gray}, hidden pt/.style={fill=gray!40},
       mra/.style={color=gray!70,tkzdotted,/tkzrightangle/size=.2},scale=2]
  \tkzDefPoints{0/0/A,2.5/0/B,1.33/0.75/D,0/2.5/E,2.5/2.5/F}
  \tkzDefLine[parallel=through D](A,B)  \tkzGetPoint{I1}
  \tkzDefLine[parallel=through B](A,D)  \tkzGetPoint{I2}
  \tkzInterLL(D,I1)(B,I2)               \tkzGetPoint{C}
  \tkzDefLine[parallel=through E](A,D)  \tkzGetPoint{I3}
  \tkzDefLine[parallel=through D](A,E)  \tkzGetPoint{I4}
  \tkzInterLL(E,I3)(D,I4)               \tkzGetPoint{H}
  \tkzDefLine[parallel=through F](E,H)  \tkzGetPoint{I5}
  \tkzDefLine[parallel=through H](E,F)  \tkzGetPoint{I6}
  \tkzInterLL(F,I5)(H,I6)               \tkzGetPoint{G}
  \tkzDefMidPoint(G,H) \tkzGetPoint{P}  \tkzDefMidPoint(G,C) \tkzGetPoint{Q}
  \tkzDefMidPoint(B,C) \tkzGetPoint{R}  \tkzDefMidPoint(A,B) \tkzGetPoint{S}
  \tkzDefMidPoint(A,E) \tkzGetPoint{T}  \tkzDefMidPoint(E,H) \tkzGetPoint{U}
  \tkzDefMidPoint(A,D) \tkzGetPoint{M}  \tkzDefMidPoint(D,C) \tkzGetPoint{N}
  \tkzInterLL(B,D)(S,R)\tkzGetPoint{L} \tkzInterLL(H,F)(U,P) \tkzGetPoint{K}
  \tkzDefLine[parallel=through K](D,H)  \tkzGetPoint{I7}
  \tkzInterLL(K,I7)(B,D)                \tkzGetPoint{O}
  \tkzFillPolygon[pol](P,Q,R,S,T,U)
  \tkzDrawSegments[seg](K,O K,L P,Q R,S T,U C,D H,D A,D M,N B,D)
  \tkzDrawSegments(E,H B,C G,F G,H G,C Q,R S,T U,P H,F)
  \tkzDrawPolygon(A,B,F,E)
  \tkzDrawPoints(A,B,C,E,F,G,H,P,Q,R,S,T,U,K) \tkzDrawPoints[hidden pt](M,N,O,D)
  \tkzMarkRightAngle[mra](L,O,K)
  \tkzMarkSegments[mark=|,size=1pt,thick,color=gray](A,S B,S B,R C,R
                    Q,C Q,G G,P H,P E,U H,U E,T A,T)
  \tkzLabelAngle[pos=.3](K,L,O){$\alpha$}
  \tkzLabelPoints[below](O,A,S,B)    \tkzLabelPoints[above](H,P,G)
  \tkzLabelPoints[left](T,E)         \tkzLabelPoints[right](C,Q)
  \tkzLabelPoints[above left](U,D,M) \tkzLabelPoints[above right](L,N)
  \tkzLabelPoints[below right](F,R)  \tkzLabelPoints[below left](K)
\end{tikzpicture}
\end{tkzexample}
\newpage

\subsection{Illustration of  the Morley theorem by Nicolas François}
\begin{tkzexample}[vbox,small]
  \begin{tikzpicture}
    \tkzInit[ymin=-3,ymax=5,xmin=-5,xmax=7]
    \tkzClip
    \tkzDefPoints{-2.5/-2/A,2/4/B,5/-1/C}
    \tkzFindAngle(C,A,B) \tkzGetAngle{anglea}
    \tkzDefPointBy[rotation=center A angle 1*\anglea/3](C) \tkzGetPoint{TA1}
    \tkzDefPointBy[rotation=center A angle 2*\anglea/3](C) \tkzGetPoint{TA2}
    \tkzFindAngle(A,B,C) \tkzGetAngle{angleb}
    \tkzDefPointBy[rotation=center B angle 1*\angleb/3](A) \tkzGetPoint{TB1}
    \tkzDefPointBy[rotation=center B angle 2*\angleb/3](A) \tkzGetPoint{TB2}
    \tkzFindAngle(B,C,A) \tkzGetAngle{anglec}
    \tkzDefPointBy[rotation=center C angle 1*\anglec/3](B) \tkzGetPoint{TC1}
    \tkzDefPointBy[rotation=center C angle 2*\anglec/3](B) \tkzGetPoint{TC2}
    \tkzInterLL(A,TA1)(B,TB2) \tkzGetPoint{U1}
    \tkzInterLL(A,TA2)(B,TB1) \tkzGetPoint{V1}
    \tkzInterLL(B,TB1)(C,TC2) \tkzGetPoint{U2}
    \tkzInterLL(B,TB2)(C,TC1) \tkzGetPoint{V2}
    \tkzInterLL(C,TC1)(A,TA2) \tkzGetPoint{U3}
    \tkzInterLL(C,TC2)(A,TA1) \tkzGetPoint{V3}
    \tkzDrawPolygons(A,B,C U1,U2,U3 V1,V2,V3)
    \tkzDrawLines[add=2 and 2,very thin,dashed](A,TA1 B,TB1 C,TC1 A,TA2 B,TB2 C,TC2)
    \tkzDrawPoints(U1,U2,U3,V1,V2,V3)
    \tkzLabelPoint[left](V1){$s_a$} \tkzLabelPoint[right](V2){$s_b$}
    \tkzLabelPoint[below](V3){$s_c$} \tkzLabelPoint[above left](A){$A$}
    \tkzLabelPoints[above right](B,C) \tkzLabelPoint(U1){$t_a$}
    \tkzLabelPoint[below left](U2){$t_b$} \tkzLabelPoint[above](U3){$t_c$}
  \end{tikzpicture}
  \end{tkzexample}

\newpage
\subsection{Gou gu theorem / Pythagorean Theorem by  Zhao Shuang}
\begin{tikzpicture}
\node [mybox,title={Gou gu theorem / Pythagorean Theorem by  Zhao Shuang}] (box){%
\begin{minipage}{0.90\textwidth}
  {\emph{Pythagoras was not the first person who discovered this theorem around the world. Ancient China discovered this theorem much earlier than him. So there is another name for the Pythagorean theorem in China, the Gou-Gu theorem.
Zhao Shuang was an ancient Chinese mathematician. He rediscovered the “Gou gu therorem”, which is actually the Chinese version of the “Pythagorean theorem”. Zhao Shuang used a method called the “cutting and compensation principle”, he  created a picture of “Pythagorean Round Square”
Below the figure used to illustrate the proof of the “Gou gu theorem.”  (code from Nan Geng)
}} 
\end{minipage}
};
\end{tikzpicture}
  
\begin{tkzexample}[latex=7cm,small]
\begin{tikzpicture}[scale=.8]
  \tkzDefPoint(0,0){A} \tkzDefPoint(4,0){A'}
  \tkzInterCC[R](A, 5)(A', 3)
  \tkzGetSecondPoint{B}
  \tkzDefSquare(A,B)   \tkzGetPoints{C}{D}
  \tkzCalcLength(A,A') \tkzGetLength{lA}
  \tkzCalcLength(A',B) \tkzGetLength{lB}
  \pgfmathparse{\lA-\lB}
  \tkzInterLC[R](A,A')(A',\pgfmathresult)
  \tkzGetFirstPoint{D'}
  \tkzDefSquare(D',A')\tkzGetPoints{B'}{C'}
  \tkzDefLine[orthogonal=through D](D,D') 
   \tkzGetPoint{d}
  \tkzDefLine[orthogonal=through A](A,A')
   \tkzGetPoint{a}
  \tkzDefLine[orthogonal=through C](C,C') 
   \tkzGetPoint{c}  
  \tkzInterLL(D,d)(C,c) \tkzGetPoint{E}
  \tkzInterLL(D,d)(A,a) \tkzGetPoint{F}
  \tkzDefSquare(E,F)\tkzGetPoints{G}{H}
  \tkzDrawPolygons[fill=teal!10](A,B,A' B,C,B'
     C,D,C' A,D',D)  
  \tkzDrawPolygons(A,B,C,D E,F,G,H)
  \tkzDrawPolygon[fill=green!10](A',B',C',D')
  \tkzDrawSegment[dim={$a$,-10pt,}](D,C')
  \tkzDrawSegment[dim={$b$,-10pt,}](C,C')
  \tkzDrawSegment[dim={$c$,-10pt,}](C,D)
  \tkzDrawPoints[size=2](A,B,C,D,A',B',C',D')
  \tkzLabelPoints[left](A)
  \tkzLabelPoints[below](B)
  \tkzLabelPoints[right](C)
  \tkzLabelPoints[above](D)
  \tkzLabelPoints[right](A')
  \tkzLabelPoints[below right](B')
  \tkzLabelPoints[below left](C') 
  \tkzLabelPoints[below](D')
 \end{tikzpicture}
\end{tkzexample}

\newpage
\subsection{Reuleaux-Triangle}
\begin{tikzpicture}
\node [mybox,title={Reuleaux-triangle by  Stefan Kottwitz}] (box){%
\begin{minipage}{0.90\textwidth}
  {\emph{A well-known classic field of mathematics is geometry.
You may know Euclidean geometry from school, with constructions
by compass and ruler. Math teachers may be very interested in
drawing geometry constructions and explanations. Underlying
constructions can help us with general drawings where we would
need intersections and tangents of lines and circles, even if
it does not look like geometry.
So, here, we will remember school geometry drawings.
We will use the tkz-euclide package, which works on top of TikZ.
We will construct an equilateral triangle.
Then we extend it to get a Reuleaux triangle, and add annotations.
The code is fully explained in the LaTeX Cookbook, Chapter 10,
Advanced Mathematics, Drawing geometry pictures.
 Stefan Kottwitz
}} 
\end{minipage}
};
\end{tikzpicture}

\begin{tkzexample}[vbox,small]
  \begin{tikzpicture}
    \tkzDefPoint(0,0){A} \tkzDefPoint(4,1){B}
    \tkzInterCC(A,B)(B,A) \tkzGetPoints{C}{D}
    \tkzInterLC(A,B)(B,A) \tkzGetPoints{F}{E}
    \tkzDrawCircles[dashed](A,B B,A)
    \tkzDrawPolygons(A,B,C A,E,D)
    \tkzCompasss[color=red, very thick](A,C B,C A,D B,D)
    \begin{scope}
      \tkzSetUpArc[thick,delta=0]
      \tkzDrawArc[fill=blue!10](A,B)(C)
      \tkzDrawArc[fill=blue!10](B,C)(A)
      \tkzDrawArc[fill=blue!10](C,A)(B)
    \end{scope}
    \tkzMarkAngles(D,A,E A,E,D)
    \tkzFillAngles[fill=yellow,opacity=0.5](D,A,E A,E,D) 
    \tkzMarkRightAngle[size=0.65,fill=red!20,opacity=0.2](A,D,E) 
    \tkzLabelAngle[pos=0.7](D,A,E){$\alpha$}
    \tkzLabelAngle[pos=0.8](A,E,D){$\beta$}
    \tkzLabelAngle[pos=0.5,xshift=-1.4mm](A,D,D){$90^\circ$}
    \begin{scope}[font=\small]
      \tkzLabelSegment[below=0.6cm,align=center](A,B){Reuleaux\\triangle}
      \tkzLabelSegment[above right,sloped](A,E){hypotenuse}
      \tkzLabelSegment[below,sloped](D,E){opposite}
      \tkzLabelSegment[below,sloped](A,D){adjacent}
      \tkzLabelSegment[below right=4cm](A,E){Thales circle}
    \end{scope}
    \tkzLabelPoints[below left](A)
    \tkzLabelPoints(B,D)
    \tkzLabelPoint[above](C){$C$}
    \tkzLabelPoints(E)
    \tkzDrawPoints(A,...,E)

  \end{tikzpicture}
\end{tkzexample}




\endinput