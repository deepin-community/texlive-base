% !TEX TS-program = lualatex
% encoding : utf8 
% Documentation of tkz-euclide v4
% Copyright 2022  Alain Matthes
% This work may be distributed and/or modified under the
% conditions of the LaTeX Project Public License, either version 1.3
% of this license or (at your option) any later version.
% The latest version of this license is in
% http://www.latex-project.org/lppl.txt
% and version 1.3 or later is part of all distributions of LaTeX
% version 2005/12/01 or later.
% This work has the LPPL maintenance status “maintained”.
% The Current Maintainer of this work is Alain Matthes.
\PassOptionsToPackage{unicode}{hyperref}

\documentclass[DIV         = 14,
               fontsize    = 10,
               index       = totoc,
               twoside,
               cadre,
               headings    = small
               ]{tkz-doc}
%\usepackage{etoc}
\gdef\tkznameofpack{tkz-euclide}
\gdef\tkzversionofpack{4.25c}
\gdef\tkzdateofpack{\today}
\gdef\tkznameofdoc{doc-tkz-euclide}
\gdef\tkzversionofdoc{4.25c} 
\gdef\tkzdateofdoc{\today}
\gdef\tkzauthorofpack{Alain Matthes}
\gdef\tkzadressofauthor{}
\gdef\tkznamecollection{AlterMundus}
\gdef\tkzurlauthor{http://altermundus.fr}
\gdef\tkzengine{lualatex}
\gdef\tkzurlauthorcom{http://altermundus.fr}
\nameoffile{\tkznameofpack}
% -- Packages ---------------------------------------------------          
\usepackage[dvipsnames,svgnames]{xcolor}
\usepackage{calc}
\usepackage{tkz-base,tkz-euclide,pgfornament} 
\usetikzlibrary{backgrounds}
\usepackage[colorlinks,pdfencoding=auto, psdextra]{hyperref}
\hypersetup{
      linkcolor=Gray,
      citecolor=Green,
      filecolor=Mulberry,
      urlcolor=NavyBlue,
      menucolor=Gray,
      runcolor=Mulberry,
      linkbordercolor=Gray,
      citebordercolor=Green,
      filebordercolor=Mulberry,
      urlbordercolor=NavyBlue,
      menubordercolor=Gray,
      runbordercolor=Mulberry,
      pdfsubject={Euclidean Geometry},
      pdfauthor={\tkzauthorofpack},
      pdftitle={\tkznameofpack},
      pdfcreator={\tkzengine}
}
\usepackage{tkzexample}
\usepackage{fontspec}
\setmainfont{texgyrepagella}[
  Extension = .otf,
  UprightFont = *-regular ,
  ItalicFont  = *-italic  ,
  BoldFont    = *-bold    ,
  BoldItalicFont = *-bolditalic
]
\setsansfont{texgyreheros}[
  Extension = .otf,
  UprightFont = *-regular ,
  ItalicFont  = *-italic  ,
  BoldFont    = *-bold    ,
  BoldItalicFont = *-bolditalic ,
]

\setmonofont{lmmono10-regular.otf}[
  Numbers={Lining,SlashedZero},
  ItalicFont=lmmonoslant10-regular.otf,
  BoldFont=lmmonolt10-bold.otf,
  BoldItalicFont=lmmonolt10-boldoblique.otf,
]
\newfontfamily\ttcondensed{lmmonoltcond10-regular.otf}
%% (La)TeX font-related declarations:
\linespread{1.05}      % Pagella needs more space between lines
%\usepackage{unicode-math}
\usepackage[math-style=literal,bold-style=literal]{unicode-math}
\usepackage{fourier-otf}
\let\rmfamily\ttfamily
\usepackage{multicol,lscape}
\usepackage[english]{babel}
\usepackage[normalem]{ulem}
\usepackage{multirow,multido,booktabs,cellspace}
\usepackage{shortvrb,fancyvrb,bookmark} 
\usepackage{makeidx}
\makeindex 

%<---------------------------------------------------------------------------> 
% settings styles
\tkzSetUpColors[background=white,text=black]  
\tkzSetUpCompass[color=orange, line width=.2pt,delta=10]
\tkzSetUpArc[color=gray,line width=.2pt]
\tkzSetUpPoint[size=2,color=teal]
\tkzSetUpLine[line width=.2pt,color=teal]
\tkzSetUpStyle[color=orange,line width=.2pt]{new}
\tikzset{every picture/.style={line width=.2pt}}
\tikzset{label angle style/.append style={color=teal,font=\footnotesize}} 
\tikzset{label style/.append style={below,color=teal,font=\scriptsize}}
\tikzset{new/.style={color=orange,line width=.2pt}} 

\AtBeginDocument{\MakeShortVerb{\|}} % link to shortvrb
\begin{document} 
  
\parindent=0pt
\tkzTitleFrame{tkz-euclide\\Euclidean Geometry}
\clearpage

\defoffile{\lefthand\
From version 4.00, \tkzname{\tkznameofpack} became independent from  \tkzname{tkz-base} . This has implied some changes : the next major step will be the version 5 which will see the introduction of Lua. To prepare for this change, I removed the last macros that allowed to plot and define at the same time. Indeed Lua will be there to make all the calculations and define all the necessary nodes. As for \TIKZ\ , it will remain to carry out the tracings, the markings and the labels.\\
\tkzname{\tkznameofpack} is a set of convenient macros for drawing in a plane (fundamental two-dimensional object) with a Cartesian coordinate system. It  handles the most classic situations in Euclidean Geometry. \tkzname{\tkznameofpack} is built on top of PGF and its associated front-end \TIKZ\ and is a (La)TeX-friendly drawing package. The aim is to provide a high-level user interface  to build graphics  relatively simply.  The idea is to allow you to follow step by step a construction that would be done by hand as naturally as possible.\\
English is  not my native language so there  might be some errors.
}

\presentation

\vspace*{1cm}
\lefthand\ Firstly, I would like to thank \textbf{Till Tantau} for the  beautiful \LaTeX{}  package, namely  \href{http://sourceforge.net/projects/pgf/}{\TIKZ}.

\vspace*{12pt}
\lefthand\ Acknowledgements : I received much valuable advice, remarks, corrections and examples from \tkzimp{Jean-Côme Charpentier}, \tkzimp{Josselin Noirel}, \tkzimp{Manuel Pégourié-Gonnard}, \tkzimp{Franck Pastor}, \tkzimp{David Arnold}, \tkzimp{Ulrike Fischer}, \tkzimp{Stefan Kottwitz}, \tkzimp{Christian Tellechea}, \tkzimp{Nicolas Kisselhoff}, \tkzimp{David Arnold}, \tkzimp{Wolfgang Büchel}, \tkzimp{John Kitzmiller}, \tkzimp{Dimitri Kapetas}, \tkzimp{Gaétan Marris}, \tkzimp{Mark Wibrow}, \tkzimp{Yves Combe} for his work on a protractor, \tkzimp{Paul Gaborit}, \tkzimp{Laurent Van Deik} for all his corrections, remarks and questions and \tkzimp{Muzimuzhi Z} for the code about the option "dim".

\vspace*{12pt}
\lefthand\ I would also like to thank Eric Weisstein, creator of MathWorld:
\href{http://mathworld.wolfram.com/about/author.html}{MathWorld}.

\vspace*{12pt}
\lefthand\ You can find some examples on my site:
\href{http://altermundus.fr}{altermundus.fr}. \hspace{2cm} under construction!

\vfill
Please report typos or any other comments to this documentation to: \href{mailto:al.ma@mac.com}{\textcolor{blue}{Alain Matthes}}.

This file can be redistributed and/or modified under the terms of the \LaTeX{} 
Project Public License Distributed from \href{http://www.ctan.org/}{CTAN}\  archives.

\clearpage
\tableofcontents

\clearpage
\newpage

\part{General survey : a brief but comprehensive review}
\input{TKZdoc-euclide-news.tex}
\input{TKZdoc-euclide-installation.tex}
\input{TKZdoc-euclide-presentation.tex}
\input{TKZdoc-euclide-elements.tex}
\input{TKZdoc-euclide-documentation.tex}
\part{Setting}
\input{TKZdoc-euclide-points.tex}

\part{Calculating}
\input{TKZdoc-euclide-pointsSpc.tex}
\input{TKZdoc-euclide-pointby.tex}
\input{TKZdoc-euclide-pointwith.tex}
\input{TKZdoc-euclide-lines.tex}
\input{TKZdoc-euclide-triangles.tex}
\input{TKZdoc-euclide-polygons.tex}
\input{TKZdoc-euclide-circles.tex}
\input{TKZdoc-euclide-circleby.tex}
\input{TKZdoc-euclide-intersection.tex}
\input{TKZdoc-euclide-angles.tex}
\input{TKZdoc-euclide-rnd.tex}

\part{Drawing and Filling}
\input{TKZdoc-euclide-drawing.tex}
\input{TKZdoc-euclide-filling.tex}
\input{TKZdoc-euclide-clipping.tex}

\part{Marking}
\input{TKZdoc-euclide-marking.tex}

\part{Labelling}
\input{TKZdoc-euclide-labelling.tex}

\part{Complements}
\input{TKZdoc-euclide-compass.tex}
\input{TKZdoc-euclide-show.tex}
\input{TKZdoc-euclide-rapporteur.tex}
\input{TKZdoc-euclide-tools.tex}

\part{Working with style}
\input{TKZdoc-euclide-styles.tex}

\part{Examples}
\input{TKZdoc-euclide-others.tex}
\input{TKZdoc-euclide-examples.tex}

\part{FAQ}
\input{TKZdoc-euclide-FAQ.tex}

\clearpage\newpage
\small\printindex
\end{document}