\section{Definition of points by transformation}
These transformations are:

\begin{itemize}
   \item translation;
   \item homothety;
   \item orthogonal reflection or symmetry;
   \item central symmetry;
   \item orthogonal projection;
   \item rotation (degrees or radians);
   \item inversion with respect to a circle.
\end{itemize}

\subsection{\tkzcname{tkzDefPointBy}}
The choice of transformations is made through the options. There are two macros, one for the transformation of a single point \tkzcname{tkzDefPointBy} and the other for the transformation of a list of points \tkzcname{tkzDefPointsBy}. By default the image of $A$ is $A'$. For example, we'll write:
\begin{tkzltxexample}[]
\tkzDefPointBy[translation= from A to A'](B) 
\end{tkzltxexample}
The result is in \tkzname{tkzPointResult}

\medskip
\begin{NewMacroBox}{tkzDefPointBy}{\oarg{local options}\parg{pt}}%
The argument is a simple existing point and its image is stored in \tkzname{tkzPointResult}. If you want to keep this point then the macro \tkzcname{tkzGetPoint\{M\}} allows you to assign the name \tkzname{M} to the point.

\begin{tabular}{lll}%
\toprule
arguments &  definition & examples               \\ 
\midrule
\TAline{pt}   {existing point name}   {$(A)$}
\bottomrule
\end{tabular}

\begin{tabular}{lll}%
options     &     & examples                         \\ 
\midrule
\TOline{translation}{= from \#1 to \#2}{[translation=from A to B](E)}
\TOline{homothety}  {= center \#1 ratio \#2}{[homothety=center A ratio .5](E)}
\TOline{reflection} {= over \#1--\#2}{[reflection=over A--B](E)}
\TOline{symmetry }  {= center \#1}{[symmetry=center A](E)}
\TOline{projection }{= onto \#1--\#2}{[projection=onto A--B](E)}
\TOline{rotation }  {= center \#1 angle \#2}{[rotation=center O angle 30](E)}
\TOline{rotation in rad}{= center \#1 angle \#2}{[rotation in rad=center O angle pi/3](E)} 
\TOline{rotation with nodes}{= center \#1 from \#2 to \#3}{[center O from A to B](E)} 
\TOline{inversion}{= center \#1 through \#2}{[inversion =center O through A](E)} 
\TOline{inversion negative}{= center \#1 through \#2}{...} 
\bottomrule
\end{tabular}

\medskip
\emph{The image is only defined and not drawn.}
\end{NewMacroBox} 

\subsubsection{\tkzname{translation}} 

\begin{tkzexample}[latex=7cm,small]
\begin{tikzpicture}[>=latex] 
 \tkzDefPoints{0/0/A,3/1/B,3/0/C}
 \tkzDefPointBy[translation= from B to A](C) 
 \tkzGetPoint{D} 
 \tkzDrawPoints[teal](A,B,C,D)
 \tkzLabelPoints[color=teal](A,B,C,D) 
 \tkzDrawSegments[orange,->](A,B D,C) 
\end{tikzpicture} 
\end{tkzexample}

\subsubsection{\tkzname{reflection} (orthogonal symmetry)} 

\begin{tkzexample}[latex=7cm,small] 
\begin{tikzpicture}[scale=.75]
 \tkzDefPoints{-2/-2/A,-1/-1/C,-4/2/D,-4/0/O}    
 \tkzDrawCircle(O,A)
 \tkzDefPointBy[reflection = over C--D](A)
 \tkzGetPoint{A'}
 \tkzDefPointBy[reflection = over C--D](O)
 \tkzGetPoint{O'}
 \tkzDrawCircle(O',A')
 \tkzDrawLine[add= .5 and .5](C,D)
 \tkzDrawPoints(C,D,O,O')
\end{tikzpicture}
\end{tkzexample}


\subsubsection{\tkzname{homothety} and \tkzname{projection}}

\begin{tkzexample}[latex=7cm,small] 
\begin{tikzpicture}
  \tkzDefPoints{0/1/A,5/3/B,3/4/C}
  \tkzDefLine[bisector](B,A,C) \tkzGetPoint{a} 
  \tkzDrawLine[add=0 and 0,color=magenta!50 ](A,a) 
  \tkzDefPointBy[homothety=center A ratio .5](a)   
  \tkzGetPoint{a'} 
  \tkzDefPointBy[projection = onto A--B](a') 
  \tkzGetPoint{k'}  
  \tkzDefPointBy[projection = onto A--B](a) 
  \tkzGetPoint{k} 
  \tkzDrawLines[add= 0 and .3](A,k A,C)   
  \tkzDrawSegments[blue](a',k' a,k) 
  \tkzDrawPoints(a,a',k,k',A)
  \tkzDrawCircles(a',k' a,k)   
  \tkzLabelPoints(a,a',k,A)
\end{tikzpicture}
\end{tkzexample}  


\subsubsection{\tkzname{projection}}
\begin{tkzexample}[latex=7cm,small] 
\begin{tikzpicture}[scale=1.25]  
 \tkzDefPoints{0/0/A,0/4/B}
 \tkzDefTriangle[pythagore](B,A) \tkzGetPoint{C}
 \tkzDefLine[bisector](B,C,A) \tkzGetPoint{c}
 \tkzInterLL(C,c)(A,B)  \tkzGetPoint{D}
 \tkzDefPointBy[projection=onto B--C](D) 
 \tkzGetPoint{G}
 \tkzInterLC(C,D)(D,A) \tkzGetPoints{E}{F}
 \tkzDrawPolygon(A,B,C)
 \tkzDrawSegment(C,D)
 \tkzDrawCircle(D,A)
 \tkzDrawSegment[new](D,G)
 \tkzMarkRightAngle[fill=orange!10](D,G,B)
 \tkzDrawPoints(A,C,F) \tkzLabelPoints(A,C,F)
 \tkzDrawPoints(B,D,E,G)   
 \tkzLabelPoints[above right](B,D,E)
  \tkzLabelPoints[above](G)
 \end{tikzpicture}
 \end{tkzexample} 

\subsubsection{\tkzname{symmetry} }
\begin{tkzexample}[latex=6cm,small] 
\begin{tikzpicture}[scale=1]
  \tkzDefPoints{2/-1/A,2/2/B,0/0/O}
  \tkzDefPointsBy[symmetry=center O](B,A){}
  \tkzDrawLine(A,A')
  \tkzDrawLine(B,B')
  \tkzMarkAngle[mark=s,arc=lll,
      size=1.5,mkcolor=red](A,O,B) 
  \tkzLabelAngle[pos=2,circle,draw,
    fill=blue!10,font=\scriptsize](A,O,B){$60^{\circ}$}
  \tkzDrawPoints(A,B,O,A',B') 
  \tkzLabelPoints(B,B')
  \tkzLabelPoints[below](A,O,A')  
\end{tikzpicture}   
\end{tkzexample}

\subsubsection{\tkzname{rotation} } 
\begin{tkzexample}[latex=7cm,small] 
\begin{tikzpicture}[scale=0.75] 
 \tkzDefPoints{0/0/A,5/0/B}
 \tkzDrawSegment(A,B)
 \tkzDefPointBy[rotation=center A angle 60](B) 
 \tkzGetPoint{C} 
 \tkzDefPointBy[symmetry=center C](A) 
 \tkzGetPoint{D} 
 \tkzDrawSegment(A,tkzPointResult) 
 \tkzDrawLine(B,D)
 \tkzDrawArc(A,B)(C) \tkzDrawArc(B,C)(A)
 \tkzDrawArc(C,D)(D)  
 \tkzMarkRightAngle(D,B,A) 
 \tkzDrawPoints(A,B) 
 \tkzLabelPoints(A,B)
 \tkzLabelPoints[above](C)
 \tkzLabelPoints[right](D)
\end{tikzpicture}  
\end{tkzexample}  

\subsubsection{\tkzname{rotation in radian}} 
\begin{tkzexample}[latex=6cm,small]
\begin{tikzpicture}
  \tkzDefPoint["$A$" left](1,5){A}
  \tkzDefPoint["$B$" right](4,3){B}
  \tkzDefPointBy[rotation in rad= center A angle pi/3](B)
  \tkzGetPoint{C}  
  \tkzDrawSegment(A,B)
  \tkzDrawPoints(A,B,C) 
  \tkzCompass(A,C)
  \tkzCompass(B,C) 
  \tkzLabelPoints(C)
\end{tikzpicture}
\end{tkzexample} 

\subsubsection{\tkzname{rotation with nodes}} 
\begin{tkzexample}[latex=6cm,small]
\begin{tikzpicture}
 \tkzDefPoint(0,0){O}    
 \tkzDefPoint(0:2){A} 
 \tkzDefPoint(40:2){B}  
 \tkzDefPoint(20:4){C}
 \tkzDrawLine(O,A)
 \tkzDefPointBy[rotation with nodes%
             =center O from A to B](C)  
 \tkzGetPoint{D}
\tkzDrawPoints(A,B,C,D)
\tkzDrawCircle(O,A)
\tkzLabelPoints(A,C,D)
\tkzLabelPoints[above](B)
\end{tikzpicture}
\end{tkzexample} 

\subsubsection{\tkzname{inversion }}

Inversion is the process of transforming points to a corresponding set of points known as their inverse points. Two points $P$ and $P'$ are said to be inverses with respect to an inversion circle having inversion center $O$ and inversion radius $k$ if $P'$ is the perpendicular foot of the altitude of $OQP$, where  $Q$ is a point on the circle such that $OQ$ is perpendicular to $PQ$.\\
 The quantity $k^2$ is known as the circle power (Coxeter 1969, p. 81).
 
(\url{https://mathworld.wolfram.com/Inversion.html})

Some propositions :
\begin{itemize}
\item The inverse of a circle (not through the center of inversion) is a circle.
\item The inverse of a circle through the center of inversion is a line.
\item The inverse of a line (not through the center of inversion) is a circle through the center of inversion.
\item A circle orthogonal to the circle of inversion is its own inverse.
\item A line through the center of inversion is its own inverse.
\item Angles are preserved in inversion.
\end{itemize}

Explanation:

Directly 
(Center O power=$k^2={OA}^2=OP \times OP'$)

\begin{tkzexample}[latex=6cm,small]
\begin{tikzpicture}[scale=.5]
  \tkzDefPoints{4/0/A,6/0/P,0/0/O}
  \tkzDefPointBy[inversion = center O through A](P)
  \tkzGetPoint{P'}
  \tkzDrawSegments(O,P)
  \tkzDrawCircle(O,A)
  \tkzLabelPoints[above right,font=\scriptsize](O,A,P,P')
  \tkzDrawPoints(O,A,P,P')
\end{tikzpicture}
\end{tkzexample} 

\begin{tkzexample}[latex=6cm,small]
\begin{tikzpicture}[scale=.5]
  \tkzDefPoints{4/0/A,6/0/P,0/0/O}
  \tkzDefLine[orthogonal=through P](O,P)
  \tkzGetPoint{L}
  \tkzDefLine[tangent from = P](O,A) \tkzGetPoints{R}{Q}
  \tkzDefPointBy[projection=onto O--A](Q) \tkzGetPoint{P'}
  \tkzDrawSegments(O,P O,A)
  \tkzDrawSegments[new](O,P O,Q P,Q Q,P')
  \tkzDrawCircle(O,A)
  \tkzDrawLines[add=1 and 0](P,L)
  \tkzLabelPoints[below,font=\scriptsize](O,P')
  \tkzLabelPoints[above right,font=\scriptsize](P,Q)
  \tkzDrawPoints(O,P) \tkzDrawPoints[new](Q,P')
  \tkzLabelSegment[above](O,Q){$k$}
  \tkzMarkRightAngles(A,P',Q P,Q,O)
  \tkzLabelCircle[above=.5cm,
      font=\scriptsize](O,A)(100){inversion circle}
  \tkzLabelPoint[left,font=\scriptsize](O){inversion center}
  \tkzLabelPoint[left,font=\scriptsize](L){polar}
\end{tikzpicture}
\end{tkzexample} 


\subsubsection{\tkzname{Inversion of lines} ex 1}
\begin{tkzexample}[latex=6cm,small]  
\begin{tikzpicture}[scale=.5]
\tkzDefPoints{0/0/O,3/0/I,4/3/P,6/-3/Q}
\tkzDrawCircle(O,I)
\tkzDefPointBy[projection= onto P--Q](O) \tkzGetPoint{A}
\tkzDefPointBy[inversion = center O through I](A)
\tkzGetPoint{A'}
\tkzDefPointBy[inversion = center O through I](P)
\tkzGetPoint{P'}
\tkzDefCircle[diameter](O,A')\tkzGetPoint{o}
\tkzDrawCircle[new](o,A')
\tkzDrawLines[add=.25 and .25,red](P,Q)
\tkzDrawLines[add=.25 and .25](O,A)
\tkzDrawSegments(O,P)
\tkzDrawPoints(A,P,O) \tkzDrawPoints[new](A',P')
\end{tikzpicture}
\end{tkzexample} 

\subsubsection{\tkzname{inversion of lines} ex 2}
\begin{tkzexample}[latex=6cm,small]  
\begin{tikzpicture}[scale=.8]
\tkzDefPoints{0/0/O,3/0/I,3/2/P,3/-2/Q}
\tkzDrawCircle(O,I)
\tkzDefPointBy[projection= onto P--Q](O) \tkzGetPoint{A}
\tkzDefPointBy[inversion = center O through I](A)
\tkzGetPoint{A'}
\tkzDefPointBy[inversion = center O through I](P)
\tkzGetPoint{P'}
\tkzDefCircle[diameter](O,A')\tkzGetPoint{o}
\tkzDrawCircle[new](o,A')
\tkzDrawLines[add=.25 and .25,red](P,Q)
\tkzDrawLines[add=.25 and .25](O,A)
\tkzDrawSegments(O,P)
\tkzDrawPoints(A,P,O) \tkzDrawPoints[new](A',P')
\end{tikzpicture}
\end{tkzexample}

\subsubsection{\tkzname{inversion of lines} ex 3}
\begin{tkzexample}[latex=6cm,small]  
\begin{tikzpicture}[scale=.8]
\tkzDefPoints{0/0/O,3/0/I,2/1/P,2/-2/Q}
\tkzDrawCircle(O,I)
\tkzDefPointBy[projection= onto P--Q](O) \tkzGetPoint{A}
\tkzDefPointBy[inversion = center O through I](A)
\tkzGetPoint{A'}
\tkzDefPointBy[inversion = center O through I](P)
\tkzGetPoint{P'}
\tkzDefCircle[diameter](O,A')
\tkzDrawCircle[new](I,A')
\tkzDrawLines[add=.25 and .75,red](P,Q)
\tkzDrawLines[add=.25 and .25](O,A')
\tkzDrawSegments(O,P')
\tkzDrawPoints(A,P,O) \tkzDrawPoints[new](A',P')
\end{tikzpicture}
\end{tkzexample}

\subsubsection{\tkzname{inversion} of circle and \tkzname{homothety} }
\begin{tkzexample}[latex=7cm,small]
\begin{tikzpicture}[scale=.7]
\tkzDefPoints{0/0/O,3/2/A,2/1/P}
\tkzDefLine[tangent from = O](A,P) \tkzGetPoints{T}{X}
\tkzDefPointsBy[homothety = center O%
                ratio 1.25](A,P,T){}
\tkzInterCC(A,P)(A',P') \tkzGetPoints{C}{D}
\tkzCalcLength(A,P)
\tkzGetLength{rAP}
\tkzDefPointOnCircle[R=center A angle 190 radius \rAP]
\tkzGetPoint{M}
\tkzDefPointBy[inversion = center O through C](M)
\tkzGetPoint{M'}
\tkzDrawCircles[new](A,P A',P')
\tkzDrawCircle(O,C)
\tkzDrawLines[add=0 and .5](O,T' O,A' O,M' O,P')
\tkzDrawPoints(A,A',P,P',O,T,T',M,M')
\tkzLabelPoints(O,T,T',M,M')
\tkzLabelPoints[below](P,P')
\end{tikzpicture}
\end{tkzexample}


\subsubsection{\tkzname{inversion} of Triangle  with respect to the Incircle}
\begin{tkzexample}[latex=6cm,small] 
\begin{tikzpicture}[scale=1]
\tkzDefPoints{0/0/A,5/1/B,3/6/C}
\tkzDefTriangleCenter[in](A,B,C) \tkzGetPoint{O} 
\tkzDefPointBy[projection= onto A--C](O) \tkzGetPoint{b}
\tkzDefPointBy[projection= onto A--C](O) \tkzGetPoint{b}
\tkzDefPointBy[projection= onto B--C](O) \tkzGetPoint{a}
\tkzDefPointBy[projection= onto A--B](O) \tkzGetPoint{c}
\tkzDefPointsBy[inversion = center O through b](a,b,c)%
                                             {Ia,Ib,Ic}
\tkzDefMidPoint(O,Ia) \tkzGetPoint{Ja}
\tkzDefMidPoint(O,Ib) \tkzGetPoint{Jb}
\tkzDefMidPoint(O,Ic) \tkzGetPoint{Jc}
\tkzInterCC(Ja,O)(Jb,O) \tkzGetPoints{O}{x}
\tkzInterCC(Ja,O)(Jc,O) \tkzGetPoints{y}{O}
\tkzInterCC(Jb,O)(Jc,O) \tkzGetPoints{O}{z}
\tkzDrawPolygon(A,B,C)
\tkzDrawCircle(O,b)\tkzDrawPoints(A,B,C,O)
\tkzDrawCircles[dashed,gray](Ja,y Jb,x Jc,z)
\tkzDrawArc[line width=1pt,orange,delta=0](Jb,x)(z)
\tkzDrawArc[line width=1pt,orange,delta=0](Jc,z)(y)
\tkzDrawArc[line width=1pt,orange,delta=0](Ja,y)(x)
\tkzLabelPoint[below](A){$A$}\tkzLabelPoint[above](C){$C$}
\tkzLabelPoint[right](B){$B$}\tkzLabelPoint[below](O){$O$}
\end{tikzpicture}
\end{tkzexample}

\subsubsection{\tkzname{inversion}: orthogonal circle with inversion circle} 
The inversion circle itself, circles orthogonal to it, and lines through the inversion center are invariant under inversion. If the circle meets the reference circle, these invariant points of intersection are also on the inverse circle. See I and J in the next figure.

\begin{tkzexample}[latex=5cm,small]
\begin{tikzpicture}[scale=1]
\tkzDefPoint(0,0){O}\tkzDefPoint(1,0){A}
\tkzDefPoint(-1.5,-1.5){z1} 
\tkzDefPoint(1.5,-1.25){z2} 
\tkzDefCircle[orthogonal through=z1 and z2](O,A)
\tkzGetPoint{c} 
\tkzDrawCircle[new](c,z1) 
\tkzDefPointBy[inversion =  center O through A](z1)
\tkzGetPoint{Z1} 
\tkzInterCC(O,A)(c,z1) \tkzGetPoints{I}{J}
\tkzDefPointBy[inversion =  center O through A](I)
\tkzGetPoint{I'}
\tkzDrawCircle(O,A)
\tkzDrawPoints(O,A,z1,z2) 
\tkzDrawPoints[new](c,Z1,I,J) 
\tkzLabelPoints(O,A,z1,z2,c,Z1,I,J)
\end{tikzpicture}
\end{tkzexample}



For a more complex example see \tkzname{Pappus} \ref{pappus}

\subsubsection{\tkzname{inversion negative}}
It's an inversion followed by a symmetry of center $O$
\begin{tkzexample}[latex=5cm,small]  
\begin{tikzpicture}[scale=1.5]
  \tkzDefPoints{1/0/A,0/0/O}
  \tkzDefPoint(-1.5,-1.5){z1}
  \tkzDefPoint(0.35,-2){z2} 
  \tkzDefPointBy[inversion negative = center O through A](z1)
  \tkzGetPoint{Z1} 
  \tkzDefPointBy[inversion negative = center O through A](z2)
  \tkzGetPoint{Z2}
  \tkzDrawCircle(O,A)  
  \tkzDrawPoints[color=black, fill=red,size=4](Z1,Z2)    
  \tkzDrawSegments(z1,Z1 z2,Z2)
  \tkzDrawPoints[color=black, fill=red,size=4](O,z1,z2)
  \tkzLabelPoints[font=\scriptsize](O,A,z1,z2,Z1,Z2)  
\end{tikzpicture}
\end{tkzexample} 


\newpage
\subsection{Transformation of multiple points; \tkzcname{tkzDefPointsBy} }
Variant of the previous macro for defining multiple images.
You must give the names of the images as arguments, or indicate that the names of the images are formed from the names of the antecedents, leaving the argument empty. 

\begin{tkzltxexample}[]
\tkzDefPointsBy[translation= from A to A'](B,C){}
\end{tkzltxexample}
The images are $B'$ and $C'$.

\begin{tkzltxexample}[]
\tkzDefPointsBy[translation= from A to A'](B,C){D,E} 
\end{tkzltxexample}
The images are $D$ and $E$.

\begin{tkzltxexample}[]
\tkzDefPointsBy[translation= from A to A'](B)
\end{tkzltxexample}
The image is $B'$.
\begin{NewMacroBox}{tkzDefPointsBy}{\oarg{local options}\parg{list of points}\marg{list of points}}%
\begin{tabular}{lll}%
arguments &  examples  &                  \\ 
\midrule
\TAline{\parg{list of points}\marg{list of pts}}{(A,B)\{E,F\}}{$E$,$F$ images of $A$, $B$}   \\
\bottomrule
\end{tabular}

\medskip
If the list of images is empty then the name of the image is the name of the antecedent to which " ' " is added.

\medskip
\begin{tabular}{lll}%
\toprule
options     &     & examples                         \\ 
\midrule
\TOline{translation = from \#1 to \#2}{}{[translation=from A to B](E)\{\}}
\TOline{homothety = center \#1 ratio \#2}{}{[homothety=center A ratio .5](E)\{F\}}
\TOline{reflection = over \#1--\#2}{}{[reflection=over A--B](E)\{F\}}
\TOline{symmetry = center \#1}{}{[symmetry=center A](E)\{F\}}
\TOline{projection = onto \#1--\#2}{}{[projection=onto A--B](E)\{F\}}
\TOline{rotation = center \#1 angle \#2}{}{[rotation=center  angle 30](E)\{F\}}
\TOline{rotation in rad = center \#1 angle \#2}{}{for instance angle pi/3}
\TOline{rotation with nodes = center \#1 from \#2 to \#3}{}{[center O from A to B](E)\{F\}} 
\TOline{inversion = center \#1 through \#2}{}{[inversion = center O through A](E)\{F\}} 
\TOline{inversion negative = center \#1 through \#2}{}{...} 
\bottomrule
\end{tabular}

\medskip
The points are only defined and not drawn.
\end{NewMacroBox}

\subsubsection{\tkzname{translation} of multiple points}
\begin{tkzexample}[latex=7cm,small]
\begin{tikzpicture}[>=latex] 
 \tkzDefPoints{0/0/A,3/0/B,3/1/A',1/2/C}
 \tkzDefPointsBy[translation= from A to A'](B,C){} 
 \tkzDrawPolygon(A,B,C)
 \tkzDrawPolygon[new](A',B',C')
 \tkzDrawPoints(A,B,C)
 \tkzDrawPoints[new](A',B',C') 
 \tkzLabelPoints(A,B,A',B')  
 \tkzLabelPoints[above](C,C')
 \tkzDrawSegments[color = gray,->,
              style=dashed](A,A' B,B' C,C') 
\end{tikzpicture}
\end{tkzexample}

\subsubsection{\tkzname{symmetry} of multiple points: an oval}

\begin{tkzexample}[latex=7cm,small]
\begin{tikzpicture}[scale=0.4]
  \tkzDefPoint(-4,0){I}
  \tkzDefPoint(4,0){J}
  \tkzDefPoint(0,0){O} 
  \tkzInterCC(J,O)(O,J) \tkzGetPoints{L}{H}
  \tkzInterCC(I,O)(O,I) \tkzGetPoints{K}{G} 
  \tkzInterLL(I,K)(J,H) \tkzGetPoint{M}
  \tkzInterLL(I,G)(J,L) \tkzGetPoint{N}
  \tkzDefPointsBy[symmetry=center J](L,H){D,E} 
  \tkzDefPointsBy[symmetry=center I](G,K){C,F}
  \begin{scope}[line style/.style = {very thin,teal}]
    \tkzDrawLines[add=1.5 and 1.5](I,K I,G J,H J,L) 
    \tkzDrawLines[add=.5 and .5](I,J) 
    \tkzDrawCircles(O,I I,O J,O) 
    \tkzDrawArc[delta=0,orange](N,D)(C) 
    \tkzDrawArc[delta=0,orange](M,F)(E) 
    \tkzDrawArc[delta=0,orange](J,E)(D) 
    \tkzDrawArc[delta=0,orange](I,C)(F) 
  \end{scope}   
\end{tikzpicture} 
\end{tkzexample}

\endinput