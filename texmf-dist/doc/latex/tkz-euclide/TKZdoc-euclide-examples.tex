\section{Some interesting examples}

\subsection{Square root of the integers}
\begin{tikzpicture}
\node [mybox,title={Square root of the integers}] (box){%
\begin{minipage}{0.90\textwidth}
  {\emph{How to get $1$, $\sqrt{2}$, $\sqrt{3}$ with a rule and a compass.
}} 
\end{minipage}
};
\end{tikzpicture}

\begin{tkzexample}[latex=7cm,small]
\begin{tikzpicture}
  \tkzDefPoint(0,0){O}
  \tkzDefPoint(1,0){a0}
   \tkzDrawSegment(O,a0)
  \foreach \i [count=\j] in {0,...,16}{%
    \tkzDefPointWith[orthogonal normed](a\i,O)
    \tkzGetPoint{a\j}
       \pgfmathsetmacro{\c}{5*\i} 
    \tkzDrawPolySeg[fill=teal!\c](a\i,a\j,O)}
 \end{tikzpicture}
\end{tkzexample}

\subsection{About right triangle}
\begin{tikzpicture}
\node [mybox,title={About right triangle}] (box){%
\begin{minipage}{0.90\textwidth}
  {\emph{We have a segment $[AB]$ and we want to determine a point $C$ such that $AC=8$~cm    and $ABC$ is a right triangle in $B$.
}} 
\end{minipage}
};
\end{tikzpicture}

\begin{tkzexample}[latex=7cm,small]
\begin{tikzpicture}[scale=.5]
  \tkzDefPoint["$A$" left](2,1){A}
  \tkzDefPoint["$B$" right](6,4){B}
  \tkzDefPointWith[orthogonal,K=-1](B,A)
  \tkzDrawLine[add = .5 and .5](B,tkzPointResult)
  \tkzInterLC[R](B,tkzPointResult)(A,8)
  \tkzGetPoints{J}{C}
  \tkzDrawSegment(A,B)
  \tkzDrawPoints(A,B,C)
  \tkzCompass(A,C)
  \tkzMarkRightAngle(A,B,C)
  \tkzDrawLine[color=gray,style=dashed](A,C)
  \tkzLabelPoint[above](C){$C$}
\end{tikzpicture}
\end{tkzexample}

\subsection{Archimedes}
\begin{tikzpicture}
\node [mybox,title={Archimedes}] (box){%
\begin{minipage}{0.90\textwidth}
  {\emph{This is an ancient problem   proved by the great Greek mathematician Archimedes .
The figure below shows a semicircle, with diameter $AB$. A tangent line is drawn and  touches the semicircle at $B$.   An other tangent line at a point, $C$, on the semicircle is drawn. We project the point $C$ on the line segment $[AB]$  on a point $D$. The two tangent lines intersect at the point $T$. Prove that the line $(AT)$ bisects $(CD)$
}} 
\end{minipage}
};
\end{tikzpicture}

\begin{tkzexample}[]
\begin{tikzpicture}[scale=1]
  \tkzDefPoint(0,0){A}\tkzDefPoint(6,0){D}
  \tkzDefPoint(8,0){B}\tkzDefPoint(4,0){I}
  \tkzDefLine[orthogonal=through D](A,D)
  \tkzInterLC[R](D,tkzPointResult)(I,4) \tkzGetSecondPoint{C}
  \tkzDefLine[orthogonal=through C](I,C)    \tkzGetPoint{c}
  \tkzDefLine[orthogonal=through B](A,B)    \tkzGetPoint{b}
  \tkzInterLL(C,c)(B,b) \tkzGetPoint{T}
  \tkzInterLL(A,T)(C,D) \tkzGetPoint{P}
  \tkzDrawArc(I,B)(A)
  \tkzDrawSegments(A,B A,T C,D I,C) \tkzDrawSegment[new](I,C)
  \tkzDrawLine[add = 1 and 0](C,T)   \tkzDrawLine[add = 0 and 1](B,T)
  \tkzMarkRightAngle(I,C,T)
  \tkzDrawPoints(A,B,I,D,C,T)
  \tkzLabelPoints(A,B,I,D)  \tkzLabelPoints[above right](C,T)
  \tkzMarkSegment[pos=.25,mark=s|](C,D) \tkzMarkSegment[pos=.75,mark=s|](C,D)
\end{tikzpicture}
\end{tkzexample}

\newpage
\subsubsection{Square and rectangle of same area; Golden section}

\begin{tikzpicture}
\node [mybox,title={Book II, proposition XI  \_Euclid's Elements\_}] (box){%
\begin{minipage}{0.90\textwidth}
{\emph{To construct Square and rectangle of same area.}
} 
\end{minipage}
};
\end{tikzpicture}% 

\begin{tkzexample}[vbox,small]
 \begin{tikzpicture}[scale=.75]
  \tkzDefPoint(0,0){D} \tkzDefPoint(8,0){A}
  \tkzDefSquare(D,A) \tkzGetPoints{B}{C}
  \tkzDefMidPoint(D,A) \tkzGetPoint{E}
  \tkzInterLC(D,A)(E,B)\tkzGetSecondPoint{F}
  \tkzInterLC[near](B,A)(A,F)\tkzGetFirstPoint{G}
  \tkzDefSquare(A,F)\tkzGetFirstPoint{H}
  \tkzInterLL(C,D)(H,G)\tkzGetPoint{I}
  \tkzFillPolygon[teal!10](I,G,B,C)
  \tkzFillPolygon[teal!10](A,F,H,G)
  \tkzDrawArc[angles](E,B)(0,120)
  \tkzDrawSemiCircle(A,F)
  \tkzDrawSegments(A,F E,B H,I F,H)
  \tkzDrawPolygons(A,B,C,D)
  \tkzDrawPoints(A,...,I)
  \tkzLabelPoints[below right](A,E,D,F,I)
  \tkzLabelPoints[above right](C,B,G,H)
 \end{tikzpicture}
\end{tkzexample}

\newpage

\subsubsection{Steiner Line and Simson Line}

\begin{tikzpicture}
\node [mybox,title={Steiner Line and Simson Line}] (box){%
\begin{minipage}{0.90\textwidth}
{\emph{Consider the triangle ABC and a point M on its circumcircle. The projections  of M on the sides of the triangle are on a line (Steiner Line),  The three closest points to M on lines AB, AC, and BC are collinear. It's the Simson Line.
}} 
\end{minipage}
};
\end{tikzpicture}%

\begin{tkzexample}[latex=7cm,small]
\begin{tikzpicture}[scale=.75,rotate=-20]
  \tkzDefPoint(0,0){B} 
  \tkzDefPoint(2,4){A} \tkzDefPoint(7,0){C}
  \tkzDefCircle[circum](A,B,C)  
  \tkzGetPoint{O}
  \tkzDrawCircle(O,A) 
  \tkzCalcLength(O,A)  
  \tkzGetLength{rOA} 
  \tkzDefShiftPoint[O](40:\rOA){M}
  \tkzDefShiftPoint[O](60:\rOA){N}  
  \tkzDefTriangleCenter[orthic](A,B,C)
  \tkzGetPoint{H}
  \tkzDefSpcTriangle[orthic,name=H](A,B,C){a,b,c}
  \tkzDefPointsBy[reflection=over A--B](M,N){P,P'}
  \tkzDefPointsBy[reflection=over A--C](M,N){Q,Q'}
  \tkzDefPointsBy[reflection=over C--B](M,N){R,R'}
  \tkzDefMidPoint(M,P)\tkzGetPoint{I}
  \tkzDefMidPoint(M,Q)\tkzGetPoint{J}
  \tkzDefMidPoint(M,R)\tkzGetPoint{K} 
  \tkzDrawSegments[new](P,R M,P M,Q M,R N,P'%
   N,Q' N,R' P',R' I,K)
  \tkzDrawPolygons(A,B,C)
  \tkzDrawPoints(A,B,C,H,M,N,P,Q,R,P',Q',R',I,J,K)
  \tkzLabelPoints(A,B,C,H,M,N,P,Q,R,P',Q',R',I,J,K)
\end{tikzpicture}
\end{tkzexample}

\newpage
\subsection{Lune of Hippocrates}

\begin{tikzpicture}
\node [mybox,title={Lune of Hippocrates}] (box){%
\begin{minipage}{0.90\textwidth}
  { \emph{From wikipedia : In geometry, the lune of Hippocrates, named after Hippocrates of Chios, is a lune bounded by arcs of two circles, the smaller of which has as its diameter a chord spanning a right angle on the larger circle.In the first figure, the area of the lune is equal to the area of the triangle ABC. Hippocrates of Chios (ancient Greek mathematician,)
}} 
\end{minipage}
};
\end{tikzpicture}% 

\begin{tkzexample}[latex=7cm,small]
\begin{tikzpicture}
 \tkzInit[xmin=-2,xmax=5,ymin=-1,ymax=6]
 \tkzClip % allows you to define a bounding box 
   % large enough
  \tkzDefPoint(0,0){A}\tkzDefPoint(4,0){B}
  \tkzDefSquare(A,B) 
  \tkzGetFirstPoint{C} 
  \tkzDrawPolygon[fill=green!5](A,B,C)
   \begin{scope}
     \tkzClipCircle[out](B,A)
     \tkzDefMidPoint(C,A) \tkzGetPoint{M}
     \tkzDrawSemiCircle[fill=teal!5](M,C)
   \end{scope}
   \tkzDrawArc[delta=0](B,C)(A)
\end{tikzpicture}
\end{tkzexample}

\subsection{Lunes of Hasan Ibn al-Haytham}

\begin{tikzpicture}
\node [mybox,title={Lune of Hippocrates}] (box){%
\begin{minipage}{0.90\textwidth}
  { \emph{From wikipedia : the Arab mathematician Hasan Ibn al-Haytham (Latinized name Alhazen) showed that two lunes, formed on the two sides of a right triangle, whose outer boundaries are semicircles and whose inner boundaries are formed by the circumcircle of the triangle, then the areas of these two lunes added together are equal to the area of the triangle. The lunes formed in this way from a right triangle are known as the lunes of Alhazen.
}} 
\end{minipage}};
\end{tikzpicture}% 

\begin{tkzexample}[latex=7cm,small]
\begin{tikzpicture}[scale=.5,rotate=180]
  \tkzInit[xmin=-1,xmax=11,ymin=-4,ymax=7]
  \tkzClip
  \tkzDefPoints{0/0/A,8/0/B}
  \tkzDefTriangle[pythagore,swap](A,B) 
  \tkzGetPoint{C}
  \tkzDrawPolygon[fill=green!5](A,B,C)
  \tkzDefMidPoint(C,A) \tkzGetPoint{I}
  \begin{scope}
    \tkzClipCircle[out](I,A)
    \tkzDefMidPoint(B,A) \tkzGetPoint{x}
    \tkzDrawSemiCircle[fill=teal!5](x,A)
    \tkzDefMidPoint(B,C) \tkzGetPoint{y}
    \tkzDrawSemiCircle[fill=teal!5](y,B)
  \end{scope}
  \tkzSetUpCompass[/tkzcompass/delta=0]
      \tkzDefMidPoint(C,A) \tkzGetPoint{z}
  \tkzDrawSemiCircle(z,A)
\end{tikzpicture}
\end{tkzexample}

\newpage
\subsection{About clipping circles}\label{About clipping circles}
\begin{tikzpicture}
\node [mybox,title={About clipping circles}] (box){%
\begin{minipage}{0.90\textwidth}
  { \emph{The problem is the management of the bounding box. First you have to define a rectangle in which the figure will be inserted. This is done with the first two lines.
}} 
\end{minipage}
};
\end{tikzpicture}% 

\begin{tkzexample}[latex=7cm,small]
\begin{tikzpicture}
  \tkzInit[xmin=0,xmax=6,ymin=0,ymax=6]
  \tkzClip
  \tkzDefPoints{0/0/A, 6/0/B}
  \tkzDefSquare(A,B)      \tkzGetPoints{C}{D}
  \tkzDefMidPoint(A,B)        \tkzGetPoint{M}
  \tkzDefMidPoint(A,D)        \tkzGetPoint{N}
  \tkzDefMidPoint(B,C)        \tkzGetPoint{O}
  \tkzDefMidPoint(C,D)        \tkzGetPoint{P}
 \begin{scope}
  \tkzClipCircle[out](M,B) \tkzClipCircle[out](P,D)
  \tkzFillPolygon[teal!20](M,N,P,O)
 \end{scope}
 \begin{scope}
   \tkzClipCircle[out](N,A) \tkzClipCircle[out](O,C)
   \tkzFillPolygon[teal!20](M,N,P,O)
 \end{scope}
 \begin{scope}
   \tkzClipCircle(P,C) \tkzClipCircle(N,A)      
   \tkzFillPolygon[teal!20](N,P,D)
 \end{scope}
 \begin{scope}
     \tkzClipCircle(O,C) \tkzClipCircle(P,C) 
     \tkzFillPolygon[teal!20](P,C,O)
 \end{scope}
 \begin{scope}
     \tkzClipCircle(M,B)  \tkzClipCircle(O,B)
     \tkzFillPolygon[teal!20](O,B,M)
 \end{scope}
 \begin{scope}
     \tkzClipCircle(N,A) \tkzClipCircle(M,A)  
     \tkzFillPolygon[teal!20](A,M,N)
 \end{scope}
 \tkzDrawSemiCircles(M,B N,A O,C P,D)
 \tkzDrawPolygons(A,B,C,D M,N,P,O)
 \end{tikzpicture}
 \end{tkzexample}

\newpage
\subsection{Similar isosceles triangles}

\begin{tikzpicture}
\node [mybox,title={Similar isosceles triangles}] (box){%
\begin{minipage}{0.90\textwidth}
  { \emph{The following is from the excellent site \textbf{Descartes et les Mathématiques}. I did not modify the text and I am only the author of the programming of the figures.
\url{http://debart.pagesperso-orange.fr/seconde/triangle.html}
}} 
\end{minipage}
};
\end{tikzpicture}% 

The following is from the excellent site \textbf{Descartes et les Mathématiques}. I did not modify the text and I am only the author of the programming of the figures.

\url{http://debart.pagesperso-orange.fr/seconde/triangle.html}

Bibliography:

\begin{itemize}
\item   Géométrie au Bac - Tangente, special issue no. 8 - Exercise 11, page 11

\item   Elisabeth Busser and Gilles Cohen: 200 nouveaux problèmes du "Monde" - POLE 2007 (200 new problems of "Le Monde")

\item   Affaire de logique n° 364 - Le Monde February 17, 2004
\end{itemize}


Two statements were proposed, one by the magazine \textit{Tangente} and the other by \textit{Le Monde}.

\vspace*{2cm}
\emph{Editor of the magazine "Tangente"}: \textcolor{orange}{Two similar isosceles triangles $AXB$ and $BYC$ are constructed with main vertices $X$ and $Y$, such that $A$, $B$ and $C$ are aligned and that these triangles are "indirect". Let $\alpha$ be the angle at vertex $\widehat{AXB}$ = $\widehat{BYC}$. We then construct a third isosceles triangle $XZY$ similar to the first two, with main vertex $Z$ and "indirect".
We ask to demonstrate that point $Z$ belongs to the straight line $(AC)$.}

\vspace*{2cm}
\emph{Editor of  "Le Monde"}: \textcolor{orange}{We construct two similar isosceles triangles $AXB$ and $BYC$ with principal vertices $X$ and $Y$, such that $A$, $B$ and $C$ are aligned and that these triangles are "indirect". Let $\alpha$ be the angle at vertex $\widehat{AXB}$ = $\widehat{BYC}$. The point Z of the line segment $[AC]$ is equidistant from the two vertices $X$ and $Y$.\\
At what angle does he see these two vertices?}

\vspace*{2cm} The constructions and their associated codes are on the next two pages, but you can search before looking. The programming respects (it seems to me ...) my reasoning in both cases.

\subsection{Revised version of "Tangente"}
\begin{tkzexample}[]
\begin{tikzpicture}[scale=.8,rotate=60]
  \tkzDefPoint(6,0){X}   \tkzDefPoint(3,3){Y}
  \tkzDefShiftPoint[X](-110:6){A}    \tkzDefShiftPoint[X](-70:6){B}
  \tkzDefShiftPoint[Y](-110:4.2){A'} \tkzDefShiftPoint[Y](-70:4.2){B'}
  \tkzDefPointBy[translation= from A' to B ](Y) \tkzGetPoint{Y}
  \tkzDefPointBy[translation= from A' to B ](B') \tkzGetPoint{C}
  \tkzInterLL(A,B)(X,Y) \tkzGetPoint{O}
  \tkzDefMidPoint(X,Y) \tkzGetPoint{I}
  \tkzDefPointWith[orthogonal](I,Y)
  \tkzInterLL(I,tkzPointResult)(A,B) \tkzGetPoint{Z}
  \tkzDefCircle[circum](X,Y,B) \tkzGetPoint{O}
  \tkzDrawCircle(O,X)
  \tkzDrawLines[add = 0 and 1.5](A,C) \tkzDrawLines[add = 0 and 3](X,Y)
  \tkzDrawSegments(A,X B,X B,Y C,Y)   \tkzDrawSegments[color=red](X,Z Y,Z)
  \tkzDrawPoints(A,B,C,X,Y,O,Z)
  \tkzLabelPoints(A,B,C,Z)   \tkzLabelPoints[above right](X,Y,O)
\end{tikzpicture}
\end{tkzexample}

\subsection{"Le Monde" version}

\begin{tkzexample}[]
\begin{tikzpicture}[scale=1.25]
  \tkzDefPoint(0,0){A}
  \tkzDefPoint(3,0){B}
  \tkzDefPoint(9,0){C}
  \tkzDefPoint(1.5,2){X}
  \tkzDefPoint(6,4){Y}
  \tkzDefCircle[circum](X,Y,B) \tkzGetPoint{O}
  \tkzDefMidPoint(X,Y)               \tkzGetPoint{I}
  \tkzDefPointWith[orthogonal](I,Y)  \tkzGetPoint{i}
  \tkzDrawLines[add = 2 and 1,color=orange](I,i)
  \tkzInterLL(I,i)(A,B)              \tkzGetPoint{Z}
  \tkzInterLC(I,i)(O,B)              \tkzGetFirstPoint{M}
  \tkzDefPointWith[orthogonal](B,Z)  \tkzGetPoint{b}
  \tkzDrawCircle(O,B)
  \tkzDrawLines[add = 0 and 2,color=orange](B,b)
  \tkzDrawSegments(A,X B,X B,Y C,Y A,C X,Y)
  \tkzDrawSegments[color=red](X,Z Y,Z)
  \tkzDrawPoints(A,B,C,X,Y,Z,M,I)
  \tkzLabelPoints(A,B,C,Z)
  \tkzLabelPoints[above right](X,Y,M,I)
\end{tikzpicture}
\end{tkzexample}

\subsection{Triangle altitudes}

\begin{tikzpicture}
\node [mybox,title={Triangle altitudes}] (box){%
\begin{minipage}{0.90\textwidth}
  { \emph{From Wikipedia : The following is again from the excellent site \textbf{Descartes et les Mathématiques} (Descartes and the Mathematics).
\url{http://debart.pagesperso-orange.fr/geoplan/geometrie_triangle.html}.
The three altitudes of a triangle intersect at the same H-point. 
}} 
\end{minipage}
};
\end{tikzpicture}% 

\begin{tkzexample}[vbox,small]
\begin{tikzpicture}
   \tkzDefPoint(0,0){C} \tkzDefPoint(7,0){B}
   \tkzDefPoint(5,6){A}
   \tkzDefMidPoint(C,B) \tkzGetPoint{I}
   \tkzInterLC(A,C)(I,B)
   \tkzGetFirstPoint{B'}
   \tkzInterLC(A,B)(I,B)
   \tkzGetSecondPoint{C'}
   \tkzInterLL(B,B')(C,C') \tkzGetPoint{H}
   \tkzInterLL(A,H)(C,B) \tkzGetPoint{A'}
   \tkzDefCircle[circum](A,B',C') \tkzGetPoint{O}
   \tkzDrawArc(I,B)(C)
   \tkzDrawPolygon(A,B,C)
   \tkzDrawCircle[color=red](O,A)
   \tkzDrawSegments[color=orange](B,B' C,C' A,A')
   \tkzMarkRightAngles(C,B',B B,C',C C,A',A)
   \tkzDrawPoints(A,B,C,A',B',C',H)
   \tkzLabelPoints[above right](A,B',C',H)
   \tkzLabelPoints[below right](B,C,A')
\end{tikzpicture}
\end{tkzexample}

\subsection{Altitudes - other construction}

\begin{tkzexample}[vbox,small]
\begin{tikzpicture}
\tkzDefPoint(0,0){A} \tkzDefPoint(8,0){B} 
\tkzDefPoint(5,6){C} 
\tkzDefMidPoint(A,B)\tkzGetPoint{O} 
\tkzDefPointBy[projection=onto A--B](C) \tkzGetPoint{P}
\tkzInterLC[common=A](C,A)(O,A)
\tkzGetFirstPoint{M}
\tkzInterLC(C,B)(O,A)
\tkzGetSecondPoint{N}
\tkzInterLL(B,M)(A,N)\tkzGetPoint{I}
\tkzDefCircle[diameter](A,B)\tkzGetPoint{x}
\tkzDefCircle[diameter](I,C)\tkzGetPoint{y}
\tkzDrawCircles(x,A y,C)
\tkzDrawSegments(C,A C,B A,B B,M A,N)
\tkzMarkRightAngles[fill=brown!20](A,M,B A,N,B A,P,C)
\tkzDrawSegment[style=dashed,color=orange](C,P)
\tkzLabelPoints(O,A,B,P)
\tkzLabelPoint[left](M){$M$} 
\tkzLabelPoint[right](N){$N$} 
\tkzLabelPoint[above](C){$C$} 
\tkzLabelPoint[above right](I){$I$} 
\tkzDrawPoints[color=red](M,N,P,I) 
\tkzDrawPoints[color=brown](O,A,B,C)
\end{tikzpicture}
\end{tkzexample}

\newpage
\subsection{Three circles  in an Equilateral Triangle }
\begin{tikzpicture}
\node [mybox,title={Three circles  in an Equilateral Triangle}] (box){%
\begin{minipage}{0.90\textwidth}
  { \emph{From Wikipedia : In geometry, the Malfatti circles are three circles inside a given triangle such that each circle is tangent to the other two and to two sides of the triangle. They are named after Gian Francesco Malfatti, who made early studies of the problem of constructing these circles in the mistaken belief that they would have the largest possible total area of any three disjoint circles within the triangle. Below is a study of a particular case with an equilateral triangle and three identical circles. 
}} 
\end{minipage}
};
\end{tikzpicture}% 
\begin{tkzexample}[latex=7cm,small]
\begin{tikzpicture}[scale=.8]
  \tkzDefPoints{0/0/A,8/0/B,0/4/a,8/4/b,8/8/c}
  \tkzDefTriangle[equilateral](A,B) \tkzGetPoint{C}
  \tkzDefMidPoint(A,B) \tkzGetPoint{M}
  \tkzDefMidPoint(B,C) \tkzGetPoint{N}
  \tkzDefMidPoint(A,C) \tkzGetPoint{P}
  \tkzInterLL(A,N)(M,a) \tkzGetPoint{Ia}
  \tkzDefPointBy[projection = onto A--B](Ia)
  \tkzGetPoint{ha}
  \tkzInterLL(B,P)(M,b) \tkzGetPoint{Ib}
  \tkzDefPointBy[projection = onto A--B](Ib)
  \tkzGetPoint{hb}
  \tkzInterLL(A,c)(M,C) \tkzGetPoint{Ic}
  \tkzDefPointBy[projection = onto A--C](Ic)
  \tkzGetPoint{hc}
  \tkzInterLL(A,Ia)(B,Ib) \tkzGetPoint{G}
  \tkzDefSquare(A,B) \tkzGetPoints{D}{E}
  \tkzDrawPolygon(A,B,C)
  \tkzClipBB
  \tkzDrawSemiCircles[gray,dashed](M,B A,M 
  A,B B,A G,Ia)
  \tkzDrawCircles[gray](Ia,ha Ib,hb Ic,hc)
  \tkzDrawPolySeg(A,E,D,B)
  \tkzDrawPoints(A,B,C,G,Ia,Ib,Ic)
  \tkzDrawSegments[gray,dashed](C,M A,N B,P
   M,a M,b A,a a,b b,B A,D Ia,ha)
\end{tikzpicture}
\end{tkzexample}

\newpage
\subsection{Law of sines}
\begin{tikzpicture}
\node [mybox,title={Law of sines}] (box){%
\begin{minipage}{0.90\textwidth}
  {From wikipedia : \emph{In trigonometry, the law of sines, sine law, sine formula, or sine rule is an equation relating the lengths of the sides of a triangle (any shape) to the sines of its angles.
}} 
\end{minipage}
};
\end{tikzpicture}% 

\begin{tkzexample}[latex=7cm,small]
  \begin{tikzpicture}
  \tkzDefPoints{0/0/A,5/1/B,2/6/C}
  \tkzDefTriangleCenter[circum](A,B,C)
   \tkzGetPoint{O} 
  \tkzDefPointBy[symmetry= center O](B) 
   \tkzGetPoint{D} 
  \tkzDrawPolygon[color=brown](A,B,C)
  \tkzDrawCircle(O,A)
  \tkzDrawPoints(A,B,C,D,O)
  \tkzDrawSegments[dashed](B,D A,D)
  \tkzLabelPoint[left](D){$D$}
  \tkzLabelPoint[below](A){$A$}
  \tkzLabelPoint[above](C){$C$}
  \tkzLabelPoint[right](B){$B$}
  \tkzLabelPoint[below](O){$O$}
  \tkzLabelSegment(B,C){$a$}
  \tkzLabelSegment[left](A,C){$b$}
  \tkzLabelSegment(A,B){$c$}
  \end{tikzpicture}
\end{tkzexample}

In the triangle $ABC$ 

\begin{equation}
\frac{a}{\sin A} = \frac{b}{\sin B} =\frac{c}{\sin C}
\end{equation}

\[\widehat{C} = \widehat{D}\] 
\begin{equation}
\frac{c}{2R} = \sin D = \sin C 
\end{equation}

Then \[ \frac{c}{\sin C} = 2R\]

\newpage
\subsection{Flower of Life}
\begin{tikzpicture}
\node [mybox,title={Book IV, proposition XI  \_Euclid's Elements\_}] (box){%
\begin{minipage}{0.90\textwidth}
  {\emph{Sacred geometry can be described as a belief system attributing a religious or cultural value to many of the fundamental forms of space and time. According to this belief system, the basic patterns of existence are perceived as sacred because in contemplating them one is contemplating the origin of all things. By studying the nature of these forms and their relationship to each other, one may seek to gain insight into the scientific, philosophical, psychological, aesthetic and mystical laws of the universe.
The Flower of Life is considered to be a symbol of sacred geometry, said to contain ancient, religious value depicting the fundamental forms of space and time. In this sense, it is a visual expression of the connections life weaves through all mankind, believed by some to contain a type of Akashic Record of basic information of all living things.
}} 
\end{minipage}
};
\end{tikzpicture}% 

One of the beautiful arrangements of circles found at the Temple of Osiris at Abydos, Egypt (Rawles 1997). \\
Weisstein, Eric W. "Flower of Life." From MathWorld--A Wolfram Web Resource.\\ \url{http://mathworld.wolfram.com/FlowerofLife.html}
 
\begin{tkzexample}[vbox,small]
\begin{tikzpicture}[scale=.75]
  \tkzSetUpLine[line width=2pt,color=teal!80!black]
  \tkzSetUpCompass[line width=2pt,color=teal!80!black]
   \tkzDefPoint(0,0){O}  \tkzDefPoint(2.25,0){A}
   \tkzDrawCircle(O,A)
\foreach \i in {0,...,5}{
   \tkzDefPointBy[rotation= center O angle 30+60*\i](A)\tkzGetPoint{a\i}
   \tkzDefPointBy[rotation= center {a\i} angle  120](O)\tkzGetPoint{b\i}
   \tkzDefPointBy[rotation= center {a\i} angle  180](O)\tkzGetPoint{c\i}
   \tkzDefPointBy[rotation= center {c\i} angle  120](a\i)\tkzGetPoint{d\i}
   \tkzDefPointBy[rotation= center {c\i} angle   60](d\i)\tkzGetPoint{f\i}
   \tkzDefPointBy[rotation= center {d\i} angle   60](b\i)\tkzGetPoint{e\i} 
   \tkzDefPointBy[rotation= center {f\i} angle   60](d\i)\tkzGetPoint{g\i} 
   \tkzDefPointBy[rotation= center {d\i} angle   60](e\i)\tkzGetPoint{h\i}
   \tkzDefPointBy[rotation= center {e\i} angle  180](b\i)\tkzGetPoint{k\i}   
   \tkzDrawCircle(a\i,O)
   \tkzDrawCircle(b\i,a\i)
   \tkzDrawCircle(c\i,a\i)
   \tkzDrawArc[rotate](f\i,d\i)(-120)
   \tkzDrawArc[rotate](e\i,d\i)(180)
   \tkzDrawArc[rotate](d\i,f\i)(180)
   \tkzDrawArc[rotate](g\i,f\i)(60)
   \tkzDrawArc[rotate](h\i,d\i)(60)
   \tkzDrawArc[rotate](k\i,e\i)(60) 
}
   \tkzClipCircle(O,f0)
\end{tikzpicture}
\end{tkzexample}


\newpage
\subsection{Pentagon in a circle}
\begin{tikzpicture}
\node [mybox,title={Book IV, proposition XI  \_Euclid's Elements\_}] (box){%
\begin{minipage}{0.90\textwidth}
  {\emph{To inscribe an equilateral and equiangular pentagon in a given circle.
}} 
\end{minipage}
};
\end{tikzpicture}% 

\begin{tkzexample}[vbox,small]
\begin{tikzpicture}[scale=.75]
   \tkzDefPoint(0,0){O} 
   \tkzDefPoint(5,0){A}
   \tkzDefPoint(0,5){B}
   \tkzDefPoint(-5,0){C} 
   \tkzDefPoint(0,-5){D}
   \tkzDefMidPoint(A,O)             \tkzGetPoint{I}
   \tkzInterLC(I,B)(I,A)            \tkzGetPoints{F}{E}
   \tkzInterCC(O,C)(B,E)            \tkzGetPoints{D3}{D2}
   \tkzInterCC(O,C)(B,F)            \tkzGetPoints{D4}{D1}
   \tkzDrawArc[angles](B,E)(180,360)
   \tkzDrawArc[angles](B,F)(220,340)
   \tkzDrawLine[add=.5 and .5](B,I)
   \tkzDrawCircle(O,A)
   \tkzDefCircle[diameter](O,A)     \tkzGetPoint{x}
   \tkzDrawCircle(x,A)
   \tkzDrawSegments(B,D C,A) 
   \tkzDrawPolygon[new](D,D1,D2,D3,D4)
   \tkzDrawPoints(A,...,D,O)
   \tkzDrawPoints[new](E,F,I,D1,D2,D4,D3)
   \tkzLabelPoints[below left](A,...,D,O)
   \tkzLabelPoints[new,below right](I,E,F,D1,D2,D4,D3)  
\end{tikzpicture}
\end{tkzexample}

 \newpage
 \subsection{Pentagon in a square}
 \begin{tikzpicture}
 \node [mybox,title={Pentagon in a square}] (box){%
 \begin{minipage}{0.90\textwidth}
   {: \emph{To inscribe an equilateral and equiangular pentagon in a given square.
 }} 
 \end{minipage}
 };
 \end{tikzpicture}%
    
\begin{tkzexample}[vbox,small]
\begin{tikzpicture}[scale=.75]
  \tkzDefPoints{0/0/O,-5/-5/A,5/-5/B}
  \tkzDefSquare(A,B)   \tkzGetPoints{C}{D}
  \tkzDefMidPoint(A,B) \tkzGetPoint{F}
  \tkzDefMidPoint(C,D) \tkzGetPoint{E}
  \tkzDefMidPoint(B,C) \tkzGetPoint{G}
  \tkzDefMidPoint(A,D) \tkzGetPoint{K}
  \tkzInterLC(D,C)(E,B)                    \tkzGetSecondPoint{T}
  \tkzDefMidPoint(D,T)                     \tkzGetPoint{I}
  \tkzInterCC[with nodes](O,D,I)(E,D,I)    \tkzGetSecondPoint{H}
  \tkzInterLC(O,H)(O,E)                    \tkzGetSecondPoint{M}
  \tkzInterCC(O,E)(E,M)                    \tkzGetFirstPoint{Q}
  \tkzInterCC[with nodes](O,O,E)(Q,E,M)    \tkzGetFirstPoint{P}
  \tkzInterCC[with nodes](O,O,E)(P,E,M)    \tkzGetFirstPoint{N}
  \tkzCompasss(O,H E,H)
  \tkzDrawArc(E,B)(T)
  \tkzDrawPolygons[purple](A,B,C,D M,E,Q,P,N) 
  \tkzDrawCircle(O,E)
  \tkzDrawSegments(T,I O,H E,H E,F G,K)
  \tkzDrawPoints(T,M,Q,P,N,I)
  \tkzLabelPoints(A,B,O,N,P,Q,M,H)
  \tkzLabelPoints[above right](C,D,E,I,T)
\end{tikzpicture} 
\end{tkzexample}

\newpage
 \subsection{Hexagon Inscribed}
 \begin{tikzpicture}
 \node [mybox,title={Hexagon Inscribed}] (box){%
 \begin{minipage}{0.90\textwidth}
   {\emph{To inscribe a regular hexagon in a given equilateral triangle  perfectly inside it (no boarders).
 }} 
 \end{minipage}
 };
 \end{tikzpicture}%
 
\subsubsection{Hexagon Inscribed version 1} % (fold)
\label{ssub:hexagon_inscribed_version_1}
\begin{tkzexample}[latex=7cm,small]
  \begin{tikzpicture}[scale=.5]
   \pgfmathsetmacro{\c}{6} 
   \tkzDefPoints{0/0/A,\c/0/B}
   \tkzDefTriangle[equilateral](A,B)\tkzGetPoint{C}
   \tkzDefTriangleCenter[centroid](A,B,C) 
   \tkzGetPoint{I}
   \tkzDefPointBy[homothety=center A ratio 1./3](B) 
   \tkzGetPoint{c1}
   \tkzInterLC(B,C)(I,c1) \tkzGetPoints{a1}{a2}
   \tkzInterLC(A,C)(I,c1) \tkzGetPoints{b1}{b2}
   \tkzInterLC(A,B)(I,c1) \tkzGetPoints{c1}{c2}
   \tkzDrawPolygon(A,B,C)
   \tkzDrawCircle[thin,orange](I,c1)
   \tkzDrawPolygon[red,thick](a2,a1,b2,b1,c2,c1)
 \end{tikzpicture} 
\end{tkzexample}
% subsubsection hexagon_inscribed_version_1 (end)

\subsubsection{Hexagon Inscribed version 2} % (fold)
\label{ssub:hexagon_inscribed_version_2}
\begin{tkzexample}[latex=7cm,small]
\begin{tikzpicture}[scale=.5]
 \pgfmathsetmacro{\c}{6} 
 \tkzDefPoints{0/0/A,\c/0/B}
 \tkzDefTriangle[equilateral](A,B)\tkzGetPoint{C}
 \tkzDefTriangleCenter[centroid](A,B,C) 
 \tkzGetPoint{I}
 \tkzDefPointsBy[rotation= center I%
                 angle 60](A,B,C){a,b,c}
 \tkzDrawPolygon[fill=teal!20,opacity=.5](A,B,C)
 \tkzDrawPolygon[fill=purple!20,opacity=.5](a,b,c)
\end{tikzpicture} 
\end{tkzexample}
% subsubsection hexagon_inscribed_version_2 (end)

\newpage
\subsection{Power of a point with respect to a circle}

\begin{tikzpicture}
\node [mybox,title={Power of a point with respect to a circle}] (box){%
\begin{minipage}{0.90\textwidth}
  {\emph{$\overline{MA} \times \overline{MB}={MT}^2={MO}^2-{OT}^2$} } 
\end{minipage}
};
\end{tikzpicture}% 

\begin{tkzexample}[vbox,small]
\begin{tikzpicture}
 \pgfmathsetmacro{\r}{2}%
 \pgfmathsetmacro{\xO}{6}%
 \pgfmathsetmacro{\xE}{\xO-\r}%
 \tkzDefPoints{0/0/M,\xO/0/O,\xE/0/E}
 \tkzDefCircle[diameter](M,O)
 \tkzGetPoint{I}
 \tkzInterCC(I,O)(O,E) \tkzGetPoints{T}{T'}
 \tkzDefShiftPoint[O](45:2){B}
 \tkzInterLC(M,B)(O,E) \tkzGetPoints{A}{B}
 \tkzDrawCircle(O,E)
 \tkzDrawSemiCircle[dashed](I,O)
 \tkzDrawLine(M,O)
 \tkzDrawLines(M,T O,T M,B)
 \tkzDrawPoints(A,B,T)
 \tkzLabelPoints[above](A,B,O,M,T)
\end{tikzpicture}
\end{tkzexample}

\newpage
\subsection{Radical axis of two non-concentric circles}
\begin{tikzpicture}
\node [mybox,title={Radical axis of two non-concentric circles}] (box){%
\begin{minipage}{0.90\textwidth}
  {From Wikipedia : \emph{In geometry, the radical axis of two non-concentric circles is the set of points whose power with respect to the circles are equal. For this reason the radical axis is also called the power line or power bisector of the two circles.  The notation radical axis was used by the French mathematician M. Chasles as axe radical.
}} 
\end{minipage}
};
\end{tikzpicture}% 

\begin{tkzexample}[vbox,small]
\begin{tikzpicture}
\tkzDefPoints{0/0/A,4/2/B,2/3/K}
\tkzDefCircle[R](A,1)\tkzGetPoint{a}
\tkzDefCircle[R](B,2)\tkzGetPoint{b}
\tkzDefCircle[R](K,3)\tkzGetPoint{k}
\tkzDrawCircles(A,a B,b)
\tkzDrawCircle[dashed,new](K,k)
\tkzInterCC(A,a)(K,k) \tkzGetPoints{a}{a'}
\tkzInterCC(B,b)(K,k) \tkzGetPoints{b}{b'}
\tkzDrawLines[new,add=2 and 2](a,a')
\tkzDrawLines[new,add=1 and 1](b,b')
\tkzInterLL(a,a')(b,b') \tkzGetPoint{X}
\tkzDefPointBy[projection= onto A--B](X) \tkzGetPoint{H}
\tkzDrawPoints(A,B,H,X,a,b,a',b')
\tkzDrawLine(A,B)
\tkzDrawLine[add= 1 and 2,new](X,H)
\tkzLabelPoints(A,B,H,X,a,b,a',b')
\end{tikzpicture}
\end{tkzexample}

\newpage
\subsection{External homothetic center}
\begin{tikzpicture}
\node [mybox,title={External homothetic center}] (box){%
\begin{minipage}{0.90\textwidth}
  {From Wikipedia : \emph{ Given two nonconcentric circles, draw radii parallel and in the same direction. Then the line joining the extremities of the radii passes through a fixed point on the line of centers which divides that line externally in the ratio of radii. This point is called the external homothetic center, or external center of similitude (Johnson 1929, pp. 19-20 and 41).
}} 
\end{minipage}
};
\end{tikzpicture}% 

\begin{tkzexample}[vbox,small]
\begin{tikzpicture}
\tkzDefPoints{0/0/A,4/2/B,2/3/K}
\tkzDefCircle[R](A,1)\tkzGetPoint{a}
\tkzDefCircle[R](B,2)\tkzGetPoint{b}
\tkzDrawCircles(A,a B,b)
\tkzDrawLine(A,B)
\tkzDefShiftPoint[A](60:1){M}
\tkzDefShiftPoint[B](60:2){M'}
\tkzInterLL(A,B)(M,M') \tkzGetPoint{O}
\tkzDefLine[tangent from = O](B,M') \tkzGetPoints{X}{T'}
\tkzDefLine[tangent from = O](A,M) \tkzGetPoints{X}{T}
\tkzDrawPoints(A,B,O,T,T',M,M')
\tkzDrawLines[new](O,B O,T' O,M')
\tkzDrawSegments[new](A,M B,M')
\tkzLabelPoints(A,B,O,T,T',M,M')
\end{tikzpicture}
\end{tkzexample}

\newpage
\subsection{Tangent lines to two circles}

\begin{tikzpicture}
\node [mybox,title={Tangent lines to two circles}] (box){%
\begin{minipage}{0.90\textwidth}
  {\emph{For two circles, there are generally four distinct lines that are tangent to both  if the two circles are outside each other.  For two of these, the external tangent lines, the circles fall on the same side of the line; the external tangent lines intersect in the external homothetic center}}
\end{minipage}
};
\end{tikzpicture}%

\begin{tkzexample}[vbox,small]
\begin{tikzpicture}
 \pgfmathsetmacro{\r}{1}%
 \pgfmathsetmacro{\R}{2}%
 \pgfmathsetmacro{\rt}{\R-\r}%
 \tkzDefPoints{0/0/A,4/2/B,2/3/K}
 \tkzDefMidPoint(A,B) \tkzGetPoint{I}
 \tkzInterLC[R](A,B)(B,\rt) \tkzGetPoints{E}{F}
 \tkzInterCC(I,B)(B,F) \tkzGetPoints{a}{a'}
 \tkzInterLC[R](B,a)(B,\R) \tkzGetPoints{X'}{T'}
 \tkzDefLine[tangent at=T'](B) \tkzGetPoint{h}
 \tkzInterLL(T',h)(A,B) \tkzGetPoint{O}
 \tkzInterLC[R](O,T')(A,\r) \tkzGetPoints{T}{T}
 \tkzDefCircle[R](A,\r)  \tkzGetPoint{a}         
 \tkzDefCircle[R](B,\R)  \tkzGetPoint{b}
 \tkzDefCircle[R](B,\rt)  \tkzGetPoint{c}
 \tkzDrawCircles(A,a)  
 \tkzDrawCircles[orange](B,b B,c)           
 \tkzDrawCircle[orange,dashed](I,B)
 \tkzDrawPoints(O,A,B,a,a',E,F,T',T)
 \tkzDrawLines(O,B A,a B,T' A,T)
 \tkzDrawLines[add= 1 and 8](T',h)
 \tkzLabelPoints(O,A,B,a,a',E,F,T,T')
\end{tikzpicture}
\end{tkzexample}

\newpage
\subsection{Tangent lines to two circles with radical axis}

\begin{tikzpicture}
\node [mybox,title={Tangent lines to two circles with radical axis}] (box){%
\begin{minipage}{0.90\textwidth}
  {\emph{As soon as two circles are not concentric, we can construct their radical axis, the set of points of equal power with respect to the two circles. We know that the radical axis is a line orthogonal to the line of the centers. Note that if we specify $P$ and $Q$ as the points of contact of one of the common exterior tangents with the two circles and $D$ and $E$ as the points of the circles outside [AB], then (DP) and (EQ) intersect on the radical axis of the two circles. We will show that this property is always true and that it allows us to construct common tangents, even when the circles have the same radius. }}
\end{minipage}
};
\end{tikzpicture}%


\begin{tkzexample}[vbox,small]
\begin{tikzpicture}
\tkzDefPoints{0/0/A,4/2/B,2/3/K}
\tkzDefCircle[R](A,1) \tkzGetPoint{a}
\tkzDefCircle[R](B,3) \tkzGetPoint{b}
\tkzInterCC[R](A,1)(K,3) \tkzGetPoints{a}{a'}
\tkzInterCC[R](B,3)(K,3) \tkzGetPoints{b}{b'}
\tkzInterLL(a,a')(b,b')  \tkzGetPoint{X}
\tkzDefPointBy[projection= onto A--B](X) \tkzGetPoint{H}
\tkzGetPoint{C}
\tkzInterLC[R](A,B)(B,3) \tkzGetPoints{b1}{E}
\tkzInterLC[R](A,B)(A,1) \tkzGetPoints{D}{a2}
\tkzDefMidPoint(D,E) \tkzGetPoint{I}
\tkzDrawCircle[orange](I,D)
\tkzInterLC(X,H)(I,D) \tkzGetPoints{M}{M'}
\tkzInterLC(M,D)(A,D) \tkzGetPoints{P}{P'}
\tkzInterLC(M,E)(B,E) \tkzGetPoints{Q'}{Q}
\tkzInterLL(P,Q)(A,B) \tkzGetPoint{O}
\tkzDrawCircles(A,a B,b)
\tkzDrawSegments[orange](A,P I,M B,Q)
\tkzDrawPoints(A,B,D,E,M,I,O,P,Q,X,H)
\tkzDrawLines(O,E M,D M,E O,Q)
\tkzDrawLine[add= 3 and 4,orange](X,H)
\tkzLabelPoints(A,B,D,E,M,I,O,P,Q,X,H)
\end{tikzpicture}
\end{tkzexample}

\newpage
\subsection{Middle of a  segment with a compass}

\begin{tikzpicture}
\node [mybox,title={Tangent lines to two circles with radical axis}] (box){%
\begin{minipage}{0.90\textwidth}
  {\emph{This example involves determining the middle of a segment, using only a compass.}}
\end{minipage}
};
\end{tikzpicture}%

\begin{tkzexample}[vbox,small]
\begin{tikzpicture}
\tkzDefPoint(0,0){A}
\tkzDefRandPointOn[circle= center A radius 4]    \tkzGetPoint{B}
\tkzDefPointBy[rotation= center A angle 180](B)  \tkzGetPoint{C}
\tkzInterCC(A,B)(B,A)                            \tkzGetPoints{I}{I'}
\tkzInterCC(A,I)(I,A)                            \tkzGetPoints{J}{B}
\tkzInterCC(B,A)(C,B)                            \tkzGetPoints{D}{E}
\tkzInterCC(D,B)(E,B)                            \tkzGetPoints{M}{M'}
\tkzSetUpArc[color=orange,style=solid,delta=10]
\tkzDrawArc(C,D)(E)
\tkzDrawArc(B,E)(D)
\tkzDrawCircle[color=teal,line width=.2pt](A,B)
\tkzDrawArc(D,B)(M) 
\tkzDrawArc(E,M)(B)
\tkzCompasss[color=orange,style=solid](B,I I,J J,C)
\tkzDrawPoints(A,B,C,D,E,M)
\tkzLabelPoints(A,B,M)
\end{tikzpicture}
\end{tkzexample}
 
\newpage

\subsection{Definition of a circle  \_Apollonius\_}

\begin{tikzpicture}
\node [mybox,title={Definition of a circle  \_Apollonius\_}] (box){%
\begin{minipage}{0.90\textwidth}
  {From Wikipedia : \emph{Apollonius showed that a circle can be defined as the set of points in a plane that have a specified ratio of distances to two fixed points, known as foci. This Apollonian circle is the basis of the Apollonius pursuit problem. ... The solutions to this problem are sometimes called the circles of Apollonius.}} 
\end{minipage}
};
\end{tikzpicture}% 

Explanation

A circle is the set of points in a plane that are equidistant from a given point O. The distance r from the center is called the radius, and the point O is called the center. It is the simplest definition but it is not the only one. Apollonius of Perga gives another definition :
The set of all points whose distances from two fixed points are in a constant ratio is a circle.

With \pkg{tkz-euclide} is easy to show you the last definition

\begin{tkzexample}[vbox, small]
\begin{tikzpicture}[scale=1.5]
    % Firstly we defined two fixed point. 
    % The figure depends of these points and the ratio K
\tkzDefPoint(0,0){A}
\tkzDefPoint(4,0){B}
    % tkz-euclide.sty knows about the apollonius's circle
    % with K=2 we search some points like  I such as IA=2 x IB
\tkzDefCircle[apollonius,K=2](A,B) \tkzGetPoints{K1}{k}
\tkzDefPointOnCircle[through=  center K1 angle 30 point k]
\tkzGetPoint{I}
\tkzDefPointOnCircle[through= center K1 angle 280  point k]
\tkzGetPoint{J}
\tkzDrawSegments[new](A,I I,B A,J J,B)  
\tkzDrawCircle[color = teal,fill=teal!20,opacity=.4](K1,k)
\tkzDrawPoints(A,B,K1,I,J)
\tkzDrawSegment(A,B)
\tkzLabelPoints[below,font=\scriptsize](A,B,K1,I,J)
\end{tikzpicture}
\end{tkzexample}

\subsection{Application of Inversion : \tkzname{Pappus chain} }\label{pappus}
\begin{tikzpicture}
\node [mybox,title={Pappus chain}] (box){%
\begin{minipage}{0.90\textwidth}
From Wikipedia  {\emph{In geometry, the Pappus chain is a ring of circles between two tangent circles investigated by Pappus of Alexandria in the 3rd century AD.}}
\end{minipage}
};
\end{tikzpicture}%

\begin{tkzexample}[vbox,small]
\begin{tikzpicture}[ultra thin]
  \pgfmathsetmacro{\xB}{6}%
  \pgfmathsetmacro{\xC}{9}%
  \pgfmathsetmacro{\xD}{(\xC*\xC)/\xB}%
  \pgfmathsetmacro{\xJ}{(\xC+\xD)/2}%
  \pgfmathsetmacro{\r}{\xD-\xJ}%
  \pgfmathsetmacro{\nc}{16}%
  \tkzDefPoints{0/0/A,\xB/0/B,\xC/0/C,\xD/0/D}
  \tkzDefCircle[diameter](A,C) \tkzGetPoint{x}
  \tkzDrawCircle[fill=teal!30](x,C)
  \tkzDefCircle[diameter](A,B) \tkzGetPoint{y}
  \tkzDrawCircle[fill=teal!30](y,B)
  \foreach \i in {-\nc,...,0,...,\nc}
  {\tkzDefPoint(\xJ,2*\r*\i){J}
   \tkzDefPoint(\xJ,2*\r*\i-\r){H}
   \tkzDefCircleBy[inversion = center A through C](J,H)
   \tkzDrawCircle[fill=teal](tkzFirstPointResult,tkzSecondPointResult)}
\end{tikzpicture}
\end{tkzexample}

\subsection{Book of lemmas proposition 1 Archimedes}
\begin{tikzpicture}
\node [mybox,title={Book of lemmas proposition 1 Archimedes}] (box){%
\begin{minipage}{0.90\textwidth}
  {\emph{If two circles touch at $A$, and if $[CD]$, $[EF]$ be parallel diameters in them, $A$, $C$ and $E$ are aligned.}}
\end{minipage}
};
\end{tikzpicture}%

\begin{tkzexample}[latex=7cm,small]
  \begin{tikzpicture}[scale=.75]
    \tkzDefPoints{0/0/O_1,0/1/O_2,0/3/A}
    \tkzDefPoint(15:3){F}
     \tkzDefPointBy[symmetry=center O_1](F) \tkzGetPoint{E}
     \tkzDefLine[parallel=through O_2](E,F) \tkzGetPoint{x}
     \tkzInterLC(x,O_2)(O_2,A) \tkzGetPoints{D}{C}
     \tkzDrawCircles(O_1,A O_2,A)
     \tkzDrawSegments[orange](O_1,A E,F C,D)
     \tkzDrawSegments[purple](A,E A,F)
     \tkzDrawPoints(A,O_1,O_2,E,F,x,C,D)
     \tkzLabelPoints(A,O_1,O_2,E,F,x,C,D)
  \end{tikzpicture}
\end{tkzexample}

$(CD) \parallel (EF)$ $(AO_1)$ is secant to these two lines so
$\widehat{A0_2C} = \widehat{A0_1E}$.

Since the triangles $AO_2C$ and $AO_1E$ are isosceles the angles at the base are equal $widehat{AC0_2} = \widehat{AE0_1} = \widehat{CA0_2} = \widehat{EA0_1}$. Thus $A$,$C$ and $E$ are aligned

\subsection{Book of lemmas proposition 6 Archimedes}
\begin{tikzpicture}
\node [mybox,title={Book of lemmas proposition 6 Archimedes}] (box){%
\begin{minipage}{0.90\textwidth}
  {\emph{Let $AC$, the diameter of a semicircle, be divided at $B$ so that $AC/AB =\phi$ or in any ratio. Describe semicircles within the first semicircle and on $AB$, $BC$ as diameters, and suppose a circle drawn touching the all three semicircles. If $GH$ be the diameter of this circle, to find relation between $GH$ and $AC$.}}
\end{minipage}
};
\end{tikzpicture}%


\begin{tkzexample}[vbox,overhang,small]
\begin{tikzpicture}
\tkzDefPoints{0/0/A,12/0/C}
\tkzDefGoldenRatio(A,C)                  \tkzGetPoint{B}
\tkzDefMidPoint(A,C)                     \tkzGetPoint{O}
\tkzDefMidPoint(A,B)                     \tkzGetPoint{O_1}
\tkzDefMidPoint(B,C)                     \tkzGetPoint{O_2}
\tkzDefExtSimilitudeCenter(O_1,A)(O_2,B) \tkzGetPoint{M_0}
\tkzDefIntSimilitudeCenter(O,A)(O_1,A)   \tkzGetPoint{M_1}
\tkzDefIntSimilitudeCenter(O,C)(O_2,C)   \tkzGetPoint{M_2}
\tkzInterCC(O_1,A)(M_2,C)                \tkzGetFirstPoint{E}
\tkzInterCC(O_2,C)(M_1,A)                \tkzGetSecondPoint{F}
\tkzInterCC(O,A)(M_0,B)                  \tkzGetFirstPoint{D}
\tkzInterLL(O_1,E)(O_2,F)                \tkzGetPoint{O_3}
\tkzDefCircle[circum](E,F,B)             \tkzGetPoint{0_4}
\tkzInterLC(A,D)(O_1,A)                  \tkzGetFirstPoint{I}
\tkzInterLC(C,D)(O_2,B)                  \tkzGetSecondPoint{K}
\tkzInterLC[common=D](A,D)(O_3,D)        \tkzGetFirstPoint{G}
\tkzInterLC[common=D](C,D)(O_3,D)        \tkzGetFirstPoint{H}
\tkzInterLL(C,G)(B,K)                    \tkzGetPoint{M}
\tkzInterLL(A,H)(B,I)                    \tkzGetPoint{L}
\tkzInterLL(L,G)(A,C)                    \tkzGetPoint{N}
\tkzInterLL(M,H)(A,C)                    \tkzGetPoint{P}  
\tkzDrawCircles[red,thin](O_3,F)
\tkzDrawCircles[new,thin](0_4,B)
\tkzDrawSemiCircles[teal](O,C O_1,B O_2,C)
\tkzDrawSemiCircles[green](M_2,C)
\tkzDrawSemiCircles[green,swap](M_1,A)
\tkzDrawSegment(A,C)
\tkzDrawSegments[new](O_1,O_3 O_2,O_3)
\tkzDrawSegments[new,very thin](B,H C,G A,H G,N H,P)
\tkzDrawSegments[new,very thin](B,D A,D C,D G,H I,B K,B B,G)
\tkzDrawPoints(A,B,C,M_1,M_2,E,O_3,F,D,0_4,O_1,O_2,I,K,G,H,L,P,N,M)  
\tkzLabelPoints[font=\scriptsize](A,B,C,M_1,M_2,F,O_1,O_2,I,K,G,H,L,M,N)
\tkzLabelPoints[font=\scriptsize,right](E,O_3,D,0_4,P)
\end{tikzpicture}
\end{tkzexample}

Let $GH$ be the diameter of the circle which is parallel to $AC$, and let the circle touch the semicircles on $AC$, $AB$, $BC$ in $D$, $E$, $F$ respectively.

Then, by Prop. 1 $A$,$G$ and $D$ are aligned, ainsi que $D$, $H$ and $C$.\\
 For a like reason $A$ $E$ and $H$ are aligned, $C$ $F$ and $G$are aligned, as also are $B$ $E$ and $G$, $B$ $F$ and $H$.
 
Let $(AD)$ meet the semicircle on $[AC]$ at $I$, and let $(BD)$ meet the semicircle on $[BC]$ in $K$. Join CI, CK meeting AE, BF in L, M, and let GL, HM produced meet AB in N, P respectively.

Now, in the triangle $AGB$, the perpendiculars from $A$, $C$ on the opposite sides meet in $L$. Therefore by the properties of triangles, $(GN)$ is perpendicular to $(AC)$.
Similarly $(HP)$ is perpendicular to $(BC)$.\\
Again, since the angles at $I$, $K$, $D$ are right, $(CK)$ is parallel to $(AD)$, and $(CI)$ to $(BD)$.

 Therefore\\
\[\frac{AB}{BC} = \frac{AL}{LH}    =  \frac{AN}{NP}  \quad\text{and} \quad \frac{BC}{AB} = \frac{CM}{MG}    =  \frac{PC}{NP} \]

hence

\[ \frac{AN}{NP}    =  \frac{NP}{PC} \quad\text{so} \quad {NP}^2 = AN \times PC  \]

Now suppose that $B$ divides $[AC]$ according to the divine proportion that is :
\[\phi = \frac{AB}{BC} =  \frac{AC}{AB} \quad\text{then}  \quad AN = \phi NP \text{and}\quad  NP = \phi PC \]

We have 
\[ AC = AN + NP + PC\quad \text{either} \quad AB + BC = = AN + NP + PC \quad \text{or} \quad (\phi + 1) BC = AN + NP + PC \]

we get 

\[ (\phi + 1) BC = \phi NP + NP + PC =(\phi + 1)NP + PC = \phi(\phi + 1)PC + PC = {\phi}^2 + \phi + 1)PC \]

as 
\[ {\phi}^2 = \phi + 1 \quad \text{then} \quad (\phi + 1) BC = 2(\phi + 1) PC \quad\text{i.e.}\quad BC = 2 PC \]

That is,
$p$ is the middle of the segment $BC$.

Part of the proof from \url{https://www.cut-the-knot.org}


\subsection{ "The" Circle of APOLLONIUS}

\begin{tikzpicture}
\node [mybox,title={The Apollonius circle of a triangle  \_Apollonius\_}] (box){%
\begin{minipage}{0.90\textwidth}
  {\emph{The circle which touches all three excircles of a triangle and encompasses them is often known as "the" Apollonius circle (Kimberling 1998, p. 102)}}
\end{minipage}
};
\end{tikzpicture}%

Explanation

The purpose of the first  examples was to show the simplicity with which we could recreate these propositions. With TikZ you need to do calculations and use trigonometry while with \pkg{tkz-euclide} you only need to build simple objects

But don't forget that behind or far above \pkg{tkz-euclide} there is TikZ. I'm only creating an interface between TikZ and the user of my package.

The last example is very complex and it is to show you all that we can do with \pkg{tkz-euclide}.


\begin{tkzexample}[vbox,small]
\begin{tikzpicture}[scale=.6]
\tkzDefPoints{0/0/A,6/0/B,0.8/4/C}
\tkzDefTriangleCenter[euler](A,B,C)        \tkzGetPoint{N} 
\tkzDefTriangleCenter[circum](A,B,C)       \tkzGetPoint{O} 
\tkzDefTriangleCenter[lemoine](A,B,C)      \tkzGetPoint{K}
\tkzDefTriangleCenter[ortho](A,B,C)        \tkzGetPoint{H}
\tkzDefSpcTriangle[excentral,name=J](A,B,C){a,b,c}
\tkzDefSpcTriangle[centroid,name=M](A,B,C){a,b,c}
\tkzDefCircle[in](Ma,Mb,Mc)                \tkzGetPoint{Sp}  % Sp Spieker center
\tkzDefProjExcenter[name=J](A,B,C)(a,b,c){Y,Z,X}
\tkzDefLine[parallel=through Za](A,B)      \tkzGetPoint{Xc}
\tkzInterLL(Za,Xc)(C,B)                    \tkzGetPoint{C'}
\tkzDefLine[parallel=through Zc](B,C)      \tkzGetPoint{Ya}
\tkzInterLL(Zc,Ya)(A,B)                    \tkzGetPoint{A'}
\tkzDefPointBy[reflection= over Ja--Jc](C')\tkzGetPoint{Ab}
\tkzDefPointBy[reflection= over Ja--Jc](A')\tkzGetPoint{Cb}
\tkzInterLL(K,O)(N,Sp)                     \tkzGetPoint{Q}
\tkzInterLC(A,B)(Q,Cb)                     \tkzGetFirstPoint{Ba}
\tkzInterLC(A,C)(Q,Cb)                     \tkzGetPoints{Ac}{Ca}
\tkzInterLC(B,C')(Q,Cb)                    \tkzGetFirstPoint{Bc}
\tkzInterLC[next to=Ja](Ja,Q)(Q,Cb)        \tkzGetFirstPoint{F'a}
\tkzInterLC[next to=Jc](Jc,Q)(Q,Cb)        \tkzGetFirstPoint{F'c}
\tkzInterLC[next to=Jb](Jb,Q)(Q,Cb)        \tkzGetFirstPoint{F'b}
\tkzInterLC[common=F'a](Sp,F'a)(Ja,F'a)    \tkzGetFirstPoint{Fa}
\tkzInterLC[common=F'b](Sp,F'b)(Jb,F'b)    \tkzGetFirstPoint{Fb}
\tkzInterLC[common=F'c](Sp,F'c)(Jc,F'c)    \tkzGetFirstPoint{Fc}
\tkzInterLC(Mc,Sp)(Q,Cb)                   \tkzGetFirstPoint{A''}
\tkzDefCircle[euler](A,B,C)                \tkzGetPoints{E}{e}
\tkzDefCircle[ex](C,A,B)                   \tkzGetPoints{Ea}{a}
\tkzDefCircle[ex](A,B,C)                   \tkzGetPoints{Eb}{b}
\tkzDefCircle[ex](B,C,A)                   \tkzGetPoints{Ec}{c}
% Calculations are done, now you can draw, mark and label
\tkzDrawCircles(Q,Cb E,e)%
\tkzDrawCircles(Eb,b Ea,a Ec,c)
\tkzDrawPolygon(A,B,C)
\tkzDrawSegments[dashed](A,A' C,C' A',Zc Za,C' B,Cb B,Ab A,Ca)
\tkzDrawSegments[dashed](C,Ac Ja,Xa Jb,Yb Jc,Zc)
\begin{scope}
   \tkzClipCircle(Q,Cb) % We limit the drawing of the lines
   \tkzDrawLine[add=5 and 12,orange](K,O)
   \tkzDrawLine[add=12 and 28,red!50!black](N,Sp)
\end{scope}
\tkzDrawPoints(A,B,C,K,Ja,Jb,Jc,Q,N,O,Sp,Mc,Xa,Xb,Yb,Yc,Za,Zc)
\tkzDrawPoints(A',C',A'',Ab,Cb,Bc,Ca,Ac,Ba,Fa,Fb,Fc,F'a,F'b,F'c)
\tkzLabelPoints(Ja,Jb,Jc,Q,Xa,Xb,Za,Zc,Ab,Cb,Bc,Ca,Ac,Ba,F'b)
\tkzLabelPoints[above](O,K,F'a,Fa,A'')
\tkzLabelPoints[below](B,F'c,Yc,N,Sp,Fc,Mc)
\tkzLabelPoints[left](A',C',Fb)
\tkzLabelPoints[right](C)
\tkzLabelPoints[below right](A)
\tkzLabelPoints[above right](Yb)
\tkzDrawSegments(Fc,F'c Fb,F'b Fa,F'a)
\tkzDrawSegments[color=green!50!black](Mc,N Mc,A'' A'',Q)
\tkzDrawSegments[color=red,dashed](Ac,Ab Ca,Cb Ba,Bc Ja,Jc A',Cb C',Ab)
\tkzDrawSegments[color=red](Cb,Ab Bc,Ac Ba,Ca A',C')
\tkzMarkSegments[color=red,mark=|](Cb,Ab Bc,Ac Ba,Ca)
\tkzMarkRightAngles(Jc,Zc,A Ja,Xa,B Jb,Yb,C)
\tkzDrawSegments[green,dashed](A,F'a B,F'b C,F'c)
\end{tikzpicture}
\end{tkzexample}

\endinput