\section{FAQ} 

\subsection{Most common errors}
 For the moment, I'm basing myself on my own, because having changed syntax several times, I've made a number of mistakes. This section is going to be expanded. With version 4.05 new problems may appear.
 
\begin{itemize}\setlength{\itemsep}{10pt}
  \item The mistake I make most often is to forget to put an "s" in the macro used to draw more than one object: like \tkzcname{tkzDrawSegment(s)} or \tkzcname{tkzDrawCircle(s)} ok like in this example \tkzcname{tkzDrawPoint(A,B)} when you need  \tkzcname{tkzDrawPoints(A,B)};
  
  \item Don't forget that since version 4 the unit is obligatorily the "cm" it is thus necessary to withdraw the unit like here \tkzcname{tkzDrawCircle[R](O,3cm)} which becomes \tkzcname{tkzDrawCircle[R](O,3)}. The traditional options of \tkzname{TikZ} keep their units example\tkzname{ below right = 12pt} on the other hand one will write \tkzname{size=1.2} to position an arc in \tkzcname{tkzMarkAngle};
  
  \item The following error still happens to me from time to time. A point that is created has its name in brackets while a point that is used either as an option or as a parameter has its name in braces. Example \tkzcname{tkzGetPoint(A)} When defining an object, use braces and not brackets, so write: \tkzcname{tkzGetPoint\{A\}};
  
  \item The changes in obtaining the points of intersection between lines and circles sometimes exchange the solutions, this leads either to a bad figure or to an error.
  
  \item \tkzcname{tkzGetPoint\{A\}} in place of \tkzcname{tkzGetFirstPoint\{A\}}. When a macro gives two points as results, either we retrieve these points using \tkzcname{tkzGetPoints\{A\}\{B\}}, or we retrieve only one of the two points, using \tkzcname{tkzGetFirstPoint\{A\}} or 
  \tkzcname{tkzGetSecondPoint\{A\}}. These two points can be used with the reference \tkzname{tkzFirstPointResult} or 
  \tkzname{tkzSecondPointResult}. It is possible that a third point is given as\\ \tkzname{tkzPointResult};

\item Mixing options and arguments; all macros that use a circle need to know the radius of the circle. If the radius is given by a measure then the option includes a \tkzname{R}.  


\item The angles are given in degrees, more rarely in radians.  

\item If an error occurs in a calculation when passing parameters, then it is better to make these calculations before calling the macro.
 
\item Do not mix the syntax of \tkzNamePack{pgfmath} and \tkzNamePack{xfp}. I've often chosen \tkzNamePack{xfp} but if you prefer pgfmath then do your calculations before passing parameters.
  
 \item  Error "dimension too large"  : In some cases, this error occurs. One way to avoid it is to use the "\tkzname{veclen}" option. When this option is used in an scope, the "veclen" function is replaced by a function dependent on "xfp".  Do not use intersection macros in this scope. For example, an error occurs if you use the macro \tkzcname{tkzDrawArc}
 with too small an angle. The error is produced by the \NameLib{decoration} library when you want to place a mark on an arc. Even if the mark is absent, the error is still present.

\end{itemize}    
\endinput