% !TeX root = mercatormap.tex
% !TeX encoding=UTF-8
% !TeX spellcheck=en_US
% include file of mercatormap.tex (manual of the LaTeX package mercatormap)
\clearpage
\section{Map Definition and Map Coordinates}\label{sec:map_definition}%

\tikzsetnextfilename{def_mrcdefinemap}%
\begin{dispExample}
\begin{tikzpicture}
\mrcdefinemap{west=11,east=13,north=49,south=48}
\mrcdrawmap[draw=path]
\mrcdrawnetwork
\draw       (mrc cs:latitude=48.53475,longitude=12.15087)   circle (2mm);
\draw[red]  (mrc cs:lat=48.53475,lon=12.15087)              circle (3mm);
\draw[blue] (mrcq cs:48.53475:12.15087)                     circle (4mm);
\ifmrcinmap{48.53475}{12.15087}{     \draw[yellow] (mrcpos) circle (5mm);}{}
\ifmrcinvicinity{48.53475}{12.15087}{\draw[cyan]   (mrcpos) circle (6mm);}{}
\end{tikzpicture}
\end{dispExample}

%-------------------------------------------------------------------------------
\subsection{Option Setting}
\begin{docCommand}{mermapset}{\marg{options}}
  Sets \meta{options} for all following maps inside the current \TeX\ group.
  All options share the common prefix |mermap/|, e.g. for setting
  \refKey{mermap/vicinity} use
  \begin{dispListing}
    \mermapset{vicinity=3cm}
  \end{dispListing}
  Also see  \refCom{mrcdefinemap}, \refCom{mermapsetsupply},
  and \refCom{mermapsetmarker}.\par
  Note that the options by \refCom{mermapset} are |expl3| \cite{l3kernel:interfaces}
  keys while \tikzname\ \cite{package:tikz} uses its own key management.
\end{docCommand}


\clearpage
%-------------------------------------------------------------------------------
\subsection{Manual Map Definition}
The following map definition is only relevant, if no script setup is used
and maps are generated completely manually.
See \Fullref{sec:automated_map} for script aided map definitions.

\begin{docCommand}{mrcdefinemap}{\marg{options}}
  Establishes a map inside a |tikzpicture| environment following
  and applying the given \meta{options}.
  All options share the common prefix |mermap/mapdef/|.
  After \refCom{mrcdefinemap} is applied, map drawing and map coordinates
  can be used.
  \begin{itemize}
  \item\refCom{mrcdefinemap} can be used directly, if no tile download
    and no script setup is intended.
  \item\refCom{mrcdefinemap} is implicitly used with
    \refCom{mrcapplymap} and \refCom{mrcmap}. In this case, all options are
    also set implicitly.
  \end{itemize}
\end{docCommand}


\begin{docMrcKey}[mapdef]{north}{=\meta{map north latitude}}{no default, initially 50}
  Northern latitude degree of the visible map, possibly negative for the southern hemisphere,
  lower than $90$ but always larger than \refKey{mermap/mapdef/south}.
  It is accessible as \docAuxCommand{mrcmapnorth} (use read-only).
\end{docMrcKey}

\begin{docMrcKey}[mapdef]{south}{=\meta{map south latitude}}{no default, initially 48}
  Southern latitude degree of the visible map, possibly negative for the southern hemisphere,
  larger than $-90$ but always lower than \refKey{mermap/mapdef/north}.
  It is accessible as \docAuxCommand{mrcmapsouth} (use read-only).
\end{docMrcKey}

\begin{docMrcKey}[mapdef]{west}{=\meta{map west longitude}}{no default, initially 11}
  Western longitude degree of the visible map, possibly negative for the western hemisphere,
  possibly shifted periodically, but always lower than \refKey{mermap/mapdef/east}.
  It is accessible as \docAuxCommand{mrcmapwest} (use read-only).
\end{docMrcKey}

\begin{docMrcKey}[mapdef]{east}{=\meta{map east longitude}}{no default, initially 13}
  Eastern longitude degree of the visible map, possibly negative for the western hemisphere,
  possibly shifted periodically, but always larger than \refKey{mermap/mapdef/west}.
  It is accessible as \docAuxCommand{mrcmapeast} (use read-only).
\end{docMrcKey}


%-------------------------------------------------------------------------------
\subsection{Further Map Definition Options}


The following options are typically implicitly set by \refCom{mrcapplymap}
and not manually by \refCom{mrcdefinemap}. However, some values are
computationally used in all cases. They can be ignored as pure technical
information.

\begin{docMrcKey}[mapdef]{xmin}{=\meta{map tile x minimum}}{no default, initially 271}
  Minimal $x$ coordinate of the map tiles.
\end{docMrcKey}

\begin{docMrcKey}[mapdef]{xmax}{=\meta{map tile x maximum}}{no default, initially 275}
  Maximal $x$ coordinate of the map tiles.
\end{docMrcKey}

\begin{docMrcKey}[mapdef]{ymin}{=\meta{map tile y minimum}}{no default, initially 173}
  Minimal $y$ coordinate of the map tiles.
\end{docMrcKey}

\begin{docMrcKey}[mapdef]{ymax}{=\meta{map tile y maximum}}{no default, initially 177}
  Maximal $y$ coordinate of the map tiles.
\end{docMrcKey}

\begin{docMrcKey}[mapdef]{zoom}{=\meta{map zoom}}{no default, initially 9}
  Map tile zoom factor alias $z$ coordinate of the map tiles.
\end{docMrcKey}

\begin{docMrcKey}[mapdef]{pixelwidth}{=\meta{map width in pixels}}{no default, initially 100}
  Width of the visible map expressed in pixels of the source file(s).
  It is accessible as \docAuxCommand{mrcpixelwidth} (use read-only).
\end{docMrcKey}

\begin{docMrcKey}[mapdef]{pixelheight}{=\meta{map height in tiles}}{no default, initially 100}
  Height of the visible map expressed in pixels of the source file(s).
  It is accessible as \docAuxCommand{mrcpixelheight} (use read-only).
\end{docMrcKey}

\begin{docMrcKey}[mapdef]{westoffset}{=\meta{map tile offset (west)}}{no default, initially 0}
  Distance of the visible map from the western edge of the most western tile
  expressed in tiles (range from 0 to 1).
\end{docMrcKey}

\begin{docMrcKey}[mapdef]{northoffset}{=\meta{map tile offset (north)}}{no default, initially 0}
  Distance of the visible map from the northern edge of the most northern tile
  expressed in tiles (range from 0 to 1).
\end{docMrcKey}

\begin{docMrcKey}[mapdef]{southoffset}{=\meta{map tile offset (south)}}{no default, initially 0}
  Distance of the visible map from the southern edge of the most southern tile
  expressed in tiles (range from 0 to 1).
\end{docMrcKey}

\begin{docMrcKey}[mapdef]{basename}{=\meta{map tile base name}}{no default, initially \texttt{tiles/tile}}
  File base name for the tiles.
\end{docMrcKey}

\begin{docMrcKey}[mapdef]{attribution}{=\meta{attribution text}}{no default, initially empty}
  Attribution text for the map source. Typically, it acknowledges the copyright
  of the map data provider. It may contain hyperlinks.
  It is accessible as \docAuxCommand{mrcmapattribution} (use read-only).
\end{docMrcKey}

\begin{docMrcKey}[mapdef]{attribution print}{=\meta{attribution text}}{no default, initially empty}
  Attribution text for the map source.
  In contrast to \refKey{mermap/mapdef/attribution} it is intended for media
  that does not support hyperlinks like printed posters, books, etc.
  It is accessible as \docAuxCommand{mrcmapattributionprint} (use read-only).
\end{docMrcKey}


\begin{docMrcKey}[mapdef]{resource}{=\meta{map resource}}{no default, initially |none|}
  Available map resource with following feasible values:
  \begin{itemize}
  \item\docValue{none}: No tiles and no merged map.
  \item\docValue{tiles}: Map tiles locally available.
  \item\docValue{mergedmap}: Single map picture file merged from tiles locally available.
  \item\docValue{wmsmap}: Single map picture file locally available.
  \end{itemize}
\end{docMrcKey}

\begin{docMrcKey}[mapdef]{tile size}{=\meta{length}}{no default, initially |32.512mm|}
  Typically set computationally. It is identical to \refKey{mermap/tile size}
  which is the recommended user option for manual setup.
\end{docMrcKey}



\clearpage
%-------------------------------------------------------------------------------
\subsection{Map Coordinate System}
After a map is defined inside a |tikzpicture| environment
by \refCom{mrcdefinemap}, \refCom{mrcapplymap}, or \refCom{mrcmap},
a Mercator map coordinate system can be used.
The border of the visible map is denoted by a \tikzname\ node \docNode{mrcmap}.

\tikzsetnextfilename{def_coordinate_system}%
\begin{dispExample}
\begin{tikzpicture}
  \mrcNPdef{nuremberg}{49.45522}{11.07631}
  \mermapset{tile size=2cm}
  \mrcdefinemap{west=5,east=15,south=47,north=55,zoom=7}
  \path[draw,fill=green!10] (mrcmap.south west) rectangle (mrcmap.north east);
  \mrcdrawnetwork
  \fill (mrc cs:latitude=48.137222,longitude=11.575556) circle (2pt)
    node[below] {M\"unchen};
  \fill (mrc cs:lat=53.550556,lon=9.993333) circle (2pt)
    node[above] {Hamburg};
  \fill (mrcq cs:52.518611:13.408333) circle (2pt)
    node[left] {Berlin};
  \fill (\mrcNPcs{nuremberg}) circle (2pt) node[above] {N\"urnberg};
  \ifmrcinmap{50.938056}{6.956944}{
    \fill (mrcpos) circle (2pt) node[right] {K\"oln};}{}
  \ifmrcNPinmap{nuremberg}{\draw[red] (mrcpos) circle (1cm);}{}
\end{tikzpicture}
\end{dispExample}


\medskip
The |mrc cs| coordinate system defines a map point by
\refKey{mermap/cs/latitude} and \refKey{mermap/cs/longitude}

\begin{docMrcKey}[cs]{latitude}{=\meta{latitude}}{no default}
  Sets the \meta{latitude} of a map point.
\end{docMrcKey}

\begin{docMrcKey}[cs]{longitude}{=\meta{longitude}}{no default}
  Sets the \meta{longitude} of a map point.
\end{docMrcKey}

\begin{dispListing}
  \fill (mrc cs:latitude=48.137222,longitude=11.575556) circle (2pt);
\end{dispListing}

\medskip
A map point can also be defined by shorter variants
\refKey{mermap/cs/lat} and \refKey{mermap/cs/lon}

\begin{docMrcKey}[cs]{lat}{=\meta{latitude}}{no default}
  Sets the \meta{latitude} of a map point.
\end{docMrcKey}

\begin{docMrcKey}[cs]{lon}{=\meta{longitude}}{no default}
  Sets the \meta{longitude} of a map point.
\end{docMrcKey}

\begin{dispListing}
  \fill (mrc cs:lat=48.137222,lon=11.575556) circle (2pt);
\end{dispListing}

\medskip
A map point can be defined even quicker by
|(mrcq cs:|\meta{latitude}|:|\meta{longitude}|)|.

\begin{dispListing}
  \fill (mrcq cs:48.137222:11.575556) circle (2pt);
\end{dispListing}

\medskip

\begin{docCommand}{mrcpgfpoint}{\marg{latitude}\marg{longitude}}
  Yields a low level |pgf| point location given by
  \meta{latitude} and \meta{longitude}.
  This can be used like |\pgfpoint|.
  \begin{dispListing}
    \pgfpathcircle{\mrcpgfpoint{49.45522}{11.07631}}{2pt}
    \pgfusepath{fill}
  \end{dispListing}
\end{docCommand}




\clearpage
%-------------------------------------------------------------------------------
\subsection{Named Positions}\label{sec:names_positions}


\begin{docCommand}{mrcNPdef}{\marg{name}\marg{latitude}\marg{longitude}}
  A coordinate pair of \meta{latitude} and \meta{longitude}
  can be saved as \emph{named position} (NP) to a \meta{name} for later use.
  The \emph{named position} just stores the given values as evaluated
  floating points but without coordinate system processing.
  Therefore, a named position can be used outside a map definition
  or |tikzpicture| environment, even as a preset for the whole document.
  Note that this saving is not global but only effective inside the
  current \TeX\ group.
  \begin{dispListing}
    \mrcNPdef{nuremberg}{49.45522}{11.07631}
  \end{dispListing}
\end{docCommand}


\begin{docCommand}{mrcNPfrompoint}{\marg{name}\marg{\tikzname\ point}}
  \emph{Latitude} and \emph{longitude} of a given \meta{\tikzname\ point} are
  calculated and saved as \emph{named position} (NP) with given \meta{name}.
  \refCom{mrcNPfrompoint} can only be used after a valid a map definition
  inside a |tikzpicture| environment.
  \begin{dispListing}
    \mrcNPfrompoint{mapcenter}{mrcmap.center}
    \mrcNPfrompoint{mytest}{[xshift=1cm,yshift=1cm]mrcmap.south west}
  \end{dispListing}
\end{docCommand}


\begin{docCommand}{mrcNPcs}{\marg{name}}
  A map point definition from the \meta{name} of a previously saved
  \emph{named position} (NP).
  \begin{dispListing}
    \fill (\mrcNPcs{nuremberg}) circle (2pt) node[above] {N\"urnberg};
  \end{dispListing}
\end{docCommand}


\begin{docCommand}[doc updated=2024-08-02]{mrcNPlat}{\marg{name}}
  Inserts the \emph{latitude} of a \emph{named position} with given \meta{name}.
  \refCom{mrcNPlat} is expandable and may be used in floating point expressions.
  If the \meta{name} does not exist, the macro expands to |0|.
  \begin{dispExample}
    \mrcNPdef{nuremberg}{49.45522}{11.07631}
    Latitude: \mrcNPlat{nuremberg}\\
    Longitude: \mrcNPlon{nuremberg}
  \end{dispExample}
\end{docCommand}


\begin{docCommand}[doc updated=2024-08-02]{mrcNPlon}{\marg{name}}
  Inserts the \emph{longitude} of a \emph{named position} with given \meta{name}.
  \refCom{mrcNPlon} is expandable and may be used in floating point expressions.
  If the \meta{name} does not exist, the macro expands to |0|.
\end{docCommand}


\begin{docCommand}[doc new=2024-08-02]{ifmrcNPexists}{\marg{name}\marg{true}\marg{false}}
  If the given \emph{named position} with given \meta{name} exists, the \meta{true} code is executed, otherwise
  the \meta{false} code.
\end{docCommand}


\clearpage
%-------------------------------------------------------------------------------
\subsection{Tests for Points to be inside or outside a Map}

When a map is drawn, \refCom{mrcclipmap} can be used to set up a
\tikzname\ clip environment which automatically removes all content which
is not inside the defined map. However, the \tikzname\ position of a geographic
point has to be computed first to decide, if this point is to be drawn.
Since \TeX\ length registers do not allow large dimensions, compiler errors
are possible to happen.

The following tests check given geographic coordinates before they are
transformed to \TeX\ dimensions and avoid such compiler errors.

\begin{docCommand}{ifmrcinmap}{\marg{latitude}\marg{longitude}\marg{true}\marg{false}}
  If the given \meta{latitude} and \meta{longitude} describes a point
  inside the visible map, the \meta{true} code is executed, otherwise
  the \meta{false} code.\par
  Inside the \meta{true} code a \tikzname\ coordinate \docNode{mrcpos}
  describes the given point. Also, \docNode{mrclastpos} denotes the
  \emph{last} position before.
  \begin{dispListing}
    \ifmrcinmap{48.137222}{11.575556}{\fill (mrcpos) circle (2pt);}{}
  \end{dispListing}
\end{docCommand}

\begin{docCommand}{ifmrcNPinmap}{\marg{name}\marg{true}\marg{false}}
  If the given \emph{named position} (NP) \meta{name} describes a point
  inside the visible map, the \meta{true} code is executed, otherwise
  the \meta{false} code.
  \begin{dispListing}
    \mrcNPdef{munich}{48.137222}{11.575556}
    \ifmrcNPinmap{munich}{\fill (mrcpos) circle (2pt);}{}
  \end{dispListing}
\end{docCommand}


Very similar to \refCom{ifmrcinmap} is \refCom{ifmrcinvicinity}.

\begin{docCommand}{ifmrcinvicinity}{\marg{latitude}\marg{longitude}\marg{true}\marg{false}}
  If the given \meta{latitude} and \meta{longitude} describes a point
  inside a vicinity of the visible map, i.e. the map \emph{plus} a margin of \refKey{mermap/vicinity},
  the \meta{true} code is executed, otherwise
  the \meta{false} code.\par
  Inside the \meta{true} code a \tikzname\ coordinate \docNode{mrcpos}
  describes the given point. Also, \docNode{mrclastpos} denotes the
  \emph{last} position before.\par
  \refCom{ifmrcinvicinity} may be used for objects of a certain size like
  markers which could be partly visible even when their reference point
  is outside the visible map (but nearby).
\begin{dispListing}
  \ifmrcinvicinity{48.137222}{11.575556}{\fill (mrcpos) circle (2pt);}{}
\end{dispListing}
\end{docCommand}


\begin{docCommand}{ifmrcNPinvicinity}{\marg{name}\marg{true}\marg{false}}
  If the given \emph{named position} (NP) \meta{name} describes a point
  inside a vicinity of the visible map, the \meta{true} code is executed, otherwise
  the \meta{false} code, see \refCom{ifmrcinvicinity}.
\begin{dispListing}
  \mrcNPdef{munich}{48.137222}{11.575556}
  \ifmrcNPinvicinity{munich}{\fill (mrcpos) circle (2pt);}{}
\end{dispListing}
\end{docCommand}


\begin{docMrcKey}{vicinity}{=\meta{width}}{no default, initially |2cm|}
  The vicinity of the map is the given map plus a border in all directions
  with the given \meta{width}.
\end{docMrcKey}


\clearpage
%-------------------------------------------------------------------------------
\subsection{Formatted Coordinate Output}

\begin{docCommand}{mrcformlat}{\oarg{options}\marg{latitude}}
  Formatted output for a given \meta{latitude} following given \meta{options}.
  Formatting \meta{options} are described in the following.
  \begin{dispExample}
    Latitude from \mrcformlat{-24.29} to \mrcformlat{12.3456789}.
  \end{dispExample}
\end{docCommand}

\begin{docCommand}{mrcformlon}{\oarg{options}\marg{longitude}}
  Formatted output for a given \meta{longitude} following given \meta{options}.
  Formatting \meta{options} are described in the following.
  \begin{dispExample}
    Longitude from \mrcformlon{-24.29} to \mrcformlon{12.3456789}.
  \end{dispExample}
\end{docCommand}


\begin{docMrcKey}{format angle}{=\meta{type}}{no default, initially |decimal-4|}
  The \meta{type} defines some formatting settings for
  \refCom{mrcformlat} and \refCom{mrcformlon}. Internally, the |\ang| macro
  from package |siunitx| \cite{package:siunitx} is used which can be controlled by further settings
  of |siunitx| like digit grouping or changing the decimal marker.\par
  Feasible values for \meta{type} are
  \begin{itemize}
  \item\docValue{decimal}: decimal output without rounding.
    \begin{dispExample}
      \mermapset{format angle=decimal}
      Longitude from \mrcformlon{-24.29} to \mrcformlon{12.3456789}.
    \end{dispExample}
  \item\docValue{decimal-0}: decimal output with rounding to full degrees.
    \begin{dispExample}
      \mermapset{format angle=decimal-0}
      Longitude from \mrcformlon{-24.29} to \mrcformlon{12.3456789}.
    \end{dispExample}
  \item\docValue{decimal-1}: decimal output with rounding to one place.
    \begin{dispExample}
      \mermapset{format angle=decimal-1}
      Longitude from \mrcformlon{-24.29} to \mrcformlon{12.3456789}.
    \end{dispExample}
  \item\docValue{decimal-2}: decimal output with rounding to two places.
    \begin{dispExample}
      \mermapset{format angle=decimal-2}
      Longitude from \mrcformlon{-24.29} to \mrcformlon{12.3456789}.
    \end{dispExample}
  \pagebreak
  \item\docValue{decimal-3}: decimal output with rounding to three places.
    \begin{dispExample}
      \mermapset{format angle=decimal-3}
      Longitude from \mrcformlon{-24.29} to \mrcformlon{12.3456789}.
    \end{dispExample}
  \item\docValue{decimal-4}: decimal output with rounding to four places.
    \begin{dispExample}
      \mermapset{format angle=decimal-4}
      Longitude from \mrcformlon{-24.29} to \mrcformlon{12.3456789}.
    \end{dispExample}
  \item\docValue{degree}: output with rounding to full degrees.
      This is an alias for \docValue{decimal-0}.
    \begin{dispExample}
      \mermapset{format angle=degree}
      Longitude from \mrcformlon{-24.29} to \mrcformlon{12.3456789}.
    \end{dispExample}
  \item\docValue{minute}: output with rounding to degrees and full minutes.
    \begin{dispExample}
      \mermapset{format angle=minute}
      Longitude from \mrcformlon{-24.29} to \mrcformlon{12.3456789}.
    \end{dispExample}
  \item\docValue{second}: output with rounding to degrees, minutes, and full seconds.
    \begin{dispExample}
      \mermapset{format angle=second}
      Longitude from \mrcformlon{-24.29} to \mrcformlon{12.3456789}.
    \end{dispExample}
  \end{itemize}
\end{docMrcKey}


\begin{docMrcKey}{format south}{=\meta{code}}{no default, initially \texttt{\#1\textbackslash,S}}
  Defines the format \meta{code} for a negative latitude.
  Use \texttt{\#1} to place the number (without sign).
  \begin{dispExample}
    \mermapset{format south={$-#1$}}
    Latitude \mrcformlat{-24.29}.
  \end{dispExample}
\end{docMrcKey}

\begin{docMrcKey}{format north}{=\meta{code}}{no default, initially \texttt{\#1\textbackslash,N}}
  Defines the format \meta{code} for a non-negative latitude.
  Use \texttt{\#1} to place the number.
  \begin{dispExample}
    \mermapset{format north={#1 North}}
    Latitude \mrcformlat{12.3456789}.
  \end{dispExample}
\end{docMrcKey}

\pagebreak
\begin{docMrcKey}{format east}{=\meta{code}}{no default, initially \texttt{\#1\textbackslash,E}}
  Defines the format \meta{code} for a positive longitude.
  Use \texttt{\#1} to place the number (without sign).
  \begin{dispExample}
    \mermapset{format east={#1\,O}}
    \sisetup{output-decimal-marker={,}}
    L\"angengrad \mrcformlon{12.3456789}.
  \end{dispExample}
\end{docMrcKey}

\begin{docMrcKey}{format west}{=\meta{code}}{no default, initially \texttt{\#1\textbackslash,W}}
  Defines the format \meta{code} for a negative longitude.
  Use \texttt{\#1} to place the number.
  \begin{dispExample}
    \mermapset{format west={West: #1}}
    Longitude \mrcformlon{-24.29}.
  \end{dispExample}
\end{docMrcKey}


\begin{docMrcKey}{format NEWS numeric}{}{no value}
  Defines the format for north, east, west, and south as numeric value
  without N, E, W, S.
  \begin{dispExample}
    \mermapset{format NEWS numeric}
    Longitude \mrcformlon{-24.29} and \mrcformlon{12.3456789}.\\
    Latitude  \mrcformlat{-24.29} and \mrcformlat{12.3456789}.
  \end{dispExample}
\end{docMrcKey}


\begin{docMrcKey}{format NEWS absolute}{}{no value}
  Defines the format for north, east, west, and south as absolute value
  without N, E, W, S and without algebraic sign.
  \begin{dispExample}
    \mermapset{format NEWS absolute}
    Longitude \mrcformlon{-24.29} and \mrcformlon{12.3456789}.\\
    Latitude  \mrcformlat{-24.29} and \mrcformlat{12.3456789}.
  \end{dispExample}
\end{docMrcKey}

