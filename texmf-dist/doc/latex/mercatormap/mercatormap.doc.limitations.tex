% !TeX root = mercatormap.tex
% !TeX encoding=UTF-8
% !TeX spellcheck=en_US
% include file of mercatormap.tex (manual of the LaTeX package mercatormap)
\clearpage
\section{Limitations and Caveats}%

%-------------------------------------------------------------------------------
\subsection{No Polar Regions}
  The standard Mercator projection is not suited for north or south polar
  regions.
  The Web Mercator projections only covers
  positions between \mrcformlat{-85.0511} and \mrcformlat{85.0511}
  with map tiles. Equally, the |mercatormap| package does only provide support for
  this area.

  If your map overlaps to south of \mrcformlat{-85.0511} or to north of  \mrcformlat{85.0511},
  compiler errors are possible because of internal computation limitations.
  In any case, there are no map tiles.

  \tikzsetnextfilename{limit_polar}%
  \begin{dispExample}
    \begin{tikzpicture}
      \mrcmap[type=reference,latitude=-85.0511,longitude=0,zoom=3,
        source=opentopomap,
        tex width=\linewidth, tex height=8cm]{}
      \mrcdrawmap
      \node[above=4mm,font=\fontsize{7pt}{7pt}\sffamily] at (mrcmap.south)
          {\mrcmapattribution};
      \mrcclipmap
      \mrcdrawnetwork[network pieces=10,network distance=10mm]
      \path[draw] (mrcmap.south west) rectangle (mrcmap.north east);
      \mrcmarker{type=pin,position=-85.0511:0,
        contents={End of the mapped world}}
    \end{tikzpicture}
  \end{dispExample}

%-------------------------------------------------------------------------------
\clearpage
\subsection{International Dateline (180th Meridian)}\label{sec:dateline}
  If your map includes the 180th meridian (more or less the international dateline),
  remember that the eastern hemisphere lies \emph{west} and
  the western hemisphere lies \emph{east} for your map.
  Consider the following example displaying Vladivostok and Anchorage.
  Here, \refKey{mermap/supply/west} is set to \ang{130}\,E (|130|)
  and \refKey{mermap/supply/east} is set to \ang{149}\,W (|211| instead of |-149|).

  There is some automatic correction for positions to fit inside the
  defined map (see Anchorage in the example below), but
  \emph{west} should be lower than \emph{east}.

  \tikzsetnextfilename{limit_dateline}%
  \begin{dispExample}
    \begin{tikzpicture}
      \mrcmap[type=areafit,tex width=\linewidth,tex height=8cm,
        source=topplusopen web,
        south=42,north=62,west=130,east=-149+360]{}
        \mrcdrawmap
        \node[below,font=\fontsize{7pt}{7pt}\sffamily] at (mrcmap.south)
          {\mrcmapattribution};
        \mrcclipmap
      \draw (mrcmap.south west) rectangle (mrcmap.north east);
      \draw (mrc cs:lat=\mrcmapsouth,lon=180)
        -- node[sloped,above] {180th meridian}
           node[sloped,below,red] {add 360 to longitude for eastern border}
      (mrc cs:lat=\mrcmapnorth,lon=180);
      \mrcmarker{type=pin,position=61.22:-149.88,contents={Anchorage}}
      \mrcmarker{type=pin,position=43.12:131.9,  contents={Vladivostok}}
    \end{tikzpicture}
  \end{dispExample}

%-------------------------------------------------------------------------------
\clearpage
\subsection{Very small Scale Maps}
  If a map has a very small scale, a coordinate point may appear more than once
  on the map. But, the map coordinates of this package will only
  pilot to a single point:

  \tikzsetnextfilename{limit_small_scale_1}%
  \begin{dispExample}
    \begin{tikzpicture}
      \mrcmap[type=reference,latitude=48.14,longitude=11.57,
        source=opentopomap,
        tex width=\linewidth,tex height=4cm,zoom=1]{}
      \mrcdrawmap
      \node[below,font=\fontsize{7pt}{7pt}\sffamily] at (mrcmap.south)
        {\mrcmapattribution};   \mrcclipmap
      \path[draw] (mrcmap.south west) rectangle (mrcmap.north east);
      \mrcmarker{type=pin,position=61.22:-149.88,contents={Anchorage}}
      \mrcmarker{type=pin,position=43.12:131.9,contents={Vladivostok}}
    \end{tikzpicture}
  \end{dispExample}

  You cannot trust too much in figures from
  \refCom{mrcprettymapscale},
  \refCom{mrcprettymapwidth},
  \refCom{mrcprettymapheight}, etc
  for large and medium scale maps.
  For small scale maps, these figures are worse and even misleading.
  Better do not use them:

  \tikzsetnextfilename{limit_small_scale_2}%
  \begin{dispExample}
    \mermapset{supply/source=opentopomap}
    \begin{tikzpicture}
      \mrcmap[type=reference,latitude=48.14,longitude=11.57,
        tex width=\linewidth,tex height=6cm,zoom=2]{}
      \mrcdrawmap
      \node[below,font=\fontsize{7pt}{7pt}\sffamily] at (mrcmap.south)
        {\mrcmapattribution};   \mrcclipmap  \mrcdrawinfo
    \end{tikzpicture}
  \end{dispExample}


%-------------------------------------------------------------------------------
\clearpage
\subsection{Zoom is not Scale}

  Using the same zoom value for different latitudes can give completely
  different scales. The following example for Oslo and Rio de Janeiro
  at zoom 12 scales to 1:151000 and otherwise 1:277000.
  \tikzexternaldisable

  \tikzsetnextfilename{limit_zoom_scale_1}
  \begin{dispExample}
    \mermapsetsupply{source=topplusopen web, type=reference, zoom=12,
      tex width=7cm, tex height=5cm}
    \begin{tikzpicture}
      \mrcmap[latitude=59.91,longitude=10.75]{oslo1}   \mrcdrawmap
    \end{tikzpicture}\hfill\begin{tikzpicture}
      \mrcmap[latitude=-22.91,longitude=-43.20]{rio1}  \mrcdrawmap
    \end{tikzpicture}
    \begin{center}\fontsize{7pt}{7pt}\sffamily\mrcmapattribution\end{center}
  \end{dispExample}

  Same example again, but with \refKey{mermap/flex scale} which gives
  1:150000 for both cities and comparable maps.

  \tikzsetnextfilename{limit_zoom_scale_2}
  \begin{dispExample}
    \mermapsetsupply{source=topplusopen web, type=reference}
    \begin{tikzpicture}
      \mrcmap[latitude=59.91,longitude=10.75, flex reference scale=150000,
        tex width=7cm, tex height=5cm]{oslo2}    \mrcdrawmap
    \end{tikzpicture}\hfill\begin{tikzpicture}
      \mrcmap[latitude=-22.91,longitude=-43.20, flex reference scale=150000,
        tex width=7cm, tex height=5cm]{rio2}     \mrcdrawmap
    \end{tikzpicture}
    \begin{center}\fontsize{7pt}{7pt}\sffamily\mrcmapattribution\end{center}
  \end{dispExample}


%-------------------------------------------------------------------------------
\clearpage
\subsection{Distances}

There are at least three kinds of distance measures between two points
on the map:
\begin{itemize}
\item Measure the distance with a ruler on the printed (or displayed) map
  and multiply with the \refCom{mrcmapscaledenominator}.
  Note that the map scale denominator is only (approximately) correct
  for the map center.
\item Compute the loxodromic distance, e.g. with \refCom{mrcstoreloxodistance}
  which gives the (approximated) distance following a loxodrome.
  On our Mercator map, this would be equal to the ruler method from above,
  if the map scale would be constant.
\item Compute the orthodromic distance, e.g. with \refCom{mrcstoreorthodistance}
  which gives the (approximated) shortest distance following an orthodrome.
\end{itemize}
Apart from the approximate nature of the implementation, these distances
are expected to be quite identical for large scale maps, but not for
(very) small scale maps.

\clearpage
\tikzsetnextfilename{limit_distance1}%
\begin{dispExample*}{}
\begin{tikzpicture}
  \mrcNPdef{munich}{48.137222}{11.575556}
  \mrcNPdef{fraunberg}{48.3685075}{11.9894209}
  \mrcmap[type=reference, named position=munich,
    flex reference scale=1000000,
    source=topplusopen web,
    tex width=\linewidth, tex height=10cm]{}
  \mrcdrawmap
  \node[below left,font=\fontsize{7pt}{7pt}\sffamily] at (mrcmap.south east)
        {\mrcmapattribution};
  \node[below right,font=\fontsize{7pt}{7pt}\sffamily] at (mrcmap.south west)
        {\mrcprettymapscale};
  \mrcclipmap
  \path[draw] (\mrcNPcs{munich}) circle (\mrckmtotex{40});
  \node[above,fill=white,opacity=0.85,text opacity=1]
    at ([yshift=\mrckmtotex{40}]\mrcNPcs{munich}) {\SI{40}{km}};
  \draw[blue,very thick] (\mrcNPcs{munich}) -- (\mrcNPcs{fraunberg});
  \mrcNPdraworthodrome[red,very thick] {munich}{fraunberg}
  \path (\mrcNPcs{munich}) --
    node[sloped,above=3mm,red,fill=white,opacity=0.85,text opacity=1]
         { \mrcNPprettyorthodistance{munich}{fraunberg} }
    node[sloped,below=3mm,blue,fill=white,opacity=0.85,text opacity=1]
         { \mrcNPprettyloxodistance{munich}{fraunberg} }
    (\mrcNPcs{fraunberg});
  \mrcmarker{type=pin,named position=munich,contents={M\"unchen}}
  \mrcmarker{type=pin,named position=fraunberg,contents={Fraunberg}}
\end{tikzpicture}
\end{dispExample*}

\clearpage
The next example gives notable different distances. The aberration is
dependent from the center angle.

\tikzsetnextfilename{limit_distance2}%
\begin{dispExample*}{breakable}
\begin{tikzpicture}
  \mrcNPdef{munich}{48.137222}{11.575556}
  \mrcNPdef{barcelona}{41.3947688}{2.0787285}
  \mrcmap[type=reference, named position=munich,
    flex reference scale=25000000,
    source=topplusopen web,
    tex width=\linewidth, tex height=10cm]{}
  \mrcdrawmap
  \node[below left,font=\fontsize{7pt}{7pt}\sffamily] at (mrcmap.south east)
        {\mrcmapattribution};
  \node[below right,font=\fontsize{7pt}{7pt}\sffamily] at (mrcmap.south west)
        {\mrcprettymapscale};
  \mrcclipmap
  \path[draw] (\mrcNPcs{munich}) circle (\mrckmtotex{1000});
  \node[above,fill=white,opacity=0.85,text opacity=1]
    at ([yshift=\mrckmtotex{1000}]\mrcNPcs{munich}) {\SI{1000}{km}};
  \draw[blue,very thick] (\mrcNPcs{munich}) -- (\mrcNPcs{barcelona});
  \mrcNPdraworthodrome[red,very thick] {munich}{barcelona}
  \path (\mrcNPcs{munich}) --
    node[sloped,above=3mm,red,fill=white,opacity=0.85,text opacity=1]
         { \mrcNPprettyorthodistance{munich}{barcelona} }
    node[sloped,below=3mm,blue,fill=white,opacity=0.85,text opacity=1]
         { \mrcNPprettyloxodistance{munich}{barcelona} }
    (\mrcNPcs{barcelona});
  \mrcmarker{type=pin,named position=munich,contents={M\"unchen}}
  \mrcmarker{type=pinflip,named position=barcelona,contents={Barcelona}}
\end{tikzpicture}
\end{dispExample*}


\clearpage
The following small scale example contains unacceptable deviations from
the displayed circle radius of \SI{3000}{\kilo\meter}. The aberration is
highly dependent from the center angle. For such small scale maps,
displaying a distance circle should be avoided.


\tikzsetnextfilename{limit_distance3}%
\begin{dispExample*}{breakable}
\begin{tikzpicture}
  \mrcNPdef{munich}{48.137222}{11.575556}
  \mrcNPdef{neskaup}{65.1446431}{-13.7420082}
  \mrcNPdef{atyrau}{47.0971204}{51.866263}
  \mrcNPdef{dakhla}{24.2026691}{-15.4883971}
  \mrcmap[type=reference, named position=munich,
    flex reference scale=75000000,
    source=topplusopen web,
    tex width=\linewidth, tex height=10cm]{}
  \mrcdrawmap
  \node[below left,font=\fontsize{7pt}{7pt}\sffamily] at (mrcmap.south east)
        {\mrcmapattribution};
  \node[below right,font=\fontsize{7pt}{7pt}\sffamily] at (mrcmap.south west)
        {\mrcprettymapscale};
  \mrcclipmap
  \path[draw] (\mrcNPcs{munich}) circle (\mrckmtotex{3000});
  \node[above,fill=white,opacity=0.85,text opacity=1]
    at ([yshift=\mrckmtotex{3000}]\mrcNPcs{munich}) {\SI{3000}{km}};
  \draw[blue,very thick] (\mrcNPcs{munich}) -- (\mrcNPcs{neskaup});
  \mrcNPdraworthodrome[red,very thick] {munich}{neskaup}
  \path (\mrcNPcs{munich}) --
    node[sloped,above=3mm,red,fill=white,opacity=0.85,text opacity=1]
         { \mrcNPprettyorthodistance{munich}{neskaup} }
    node[sloped,below=3mm,blue,fill=white,opacity=0.85,text opacity=1]
         { \mrcNPprettyloxodistance{munich}{neskaup} }
    (\mrcNPcs{neskaup});
  \draw[blue,very thick] (\mrcNPcs{munich}) -- (\mrcNPcs{atyrau});
  \mrcNPdraworthodrome[red,very thick] {munich}{atyrau}
  \path (\mrcNPcs{munich}) --
    node[sloped,above=3mm,red,fill=white,opacity=0.85,text opacity=1]
         { \mrcNPprettyorthodistance{munich}{atyrau} }
    node[sloped,below=3mm,blue,fill=white,opacity=0.85,text opacity=1]
         { \mrcNPprettyloxodistance{munich}{atyrau} }
    (\mrcNPcs{atyrau});
  \draw[blue,very thick] (\mrcNPcs{munich}) -- (\mrcNPcs{dakhla});
  \mrcNPdraworthodrome[red,very thick] {munich}{dakhla}
  \path (\mrcNPcs{munich}) --
    node[sloped,above=3mm,red,fill=white,opacity=0.85,text opacity=1]
         { \mrcNPprettyorthodistance{munich}{dakhla} }
    node[sloped,below=3mm,blue,fill=white,opacity=0.85,text opacity=1]
         { \mrcNPprettyloxodistance{munich}{dakhla} }
    (\mrcNPcs{dakhla});
  \mrcmarker{type=pinflip,named position=munich,contents={M\"unchen}}
  \mrcmarker{type=pin,named position=neskaup,contents={Neskaupsta\dh ur}}
  \mrcmarker{type=pin,named position=atyrau,contents={Atyrau}}
  \mrcmarker{type=pinflip,named position=dakhla,contents={Ad-Dakhla}}
\end{tikzpicture}
\end{dispExample*}
