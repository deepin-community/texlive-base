%% tikzviolinplots.tex
%% Copyright 2023 Pedro Callil-Soares
%
% This work may be distributed and/or modified under the
% conditions of the LaTeX Project Public License, either version 1.3
% of this license or (at your option) any later version.
% The latest version of this license is in
%   http://www.latex-project.org/lppl.txt
% and version 1.3 or later is part of all distributions of LaTeX
% version 2005/12/01 or later.
%
% This work has the LPPL maintenance status `maintained'.
%
% The Current Maintainer of this work is Pedro Callil-Soares.
%
% This work consists of the files tikzviolinplots.sty and
% tikzviolinplots.tex.

\documentclass{article}

\usepackage{amsmath}
\usepackage{pgfplots}
\pgfplotsset{width=\textwidth,compat=1.18}
\usepgfplotslibrary{statistics}
\usepackage{tikzviolinplots}
\usepgfplotslibrary{external}
\tikzexternalize
\usetikzlibrary{arrows.meta}
\usetikzlibrary{decorations.text}
\usetikzlibrary{decorations.pathmorphing}
\usepackage{minted}
\usemintedstyle{gruvbox-light}
\usepackage{scontents}
\usepackage{wasysym}
\usepackage{microtype}
\usepackage{subcaption}
\usepackage{hyperref}

\begin{scontents}[write-out=violinandboxplotexample.dat]
A	B	C	D
0.876	0.574	2.175	1.684
1.015	0.822	2.121	2.290
0.995	1.266	1.916	2.184
1.101	0.740	2.044	2.065
1.222	0.636	2.004	2.248
0.712	1.753	2.038	2.200
1.063	0.873	1.974	1.865
0.973	1.020	2.145	1.895
1.174	0.763	2.052	2.240
1.057	1.293	2.079	2.360
1.004	0.853	2.064	1.695
0.909	0.694	2.136	1.837
1.108	0.849	2.000	1.744
1.210	0.468	2.010	1.820
0.958	0.767	1.824	2.158
1.119	0.940	2.034	1.846
1.000	0.962	1.967	2.545
0.954	1.551	1.833	1.816
0.918	0.755	1.914	2.369
1.093	0.973	1.997	1.543
\end{scontents}

\begin{scontents}[write-out=example.dat]
A	B	C	D	E
0.3	-2.1	3.50	2.89	1.00
0.41	-1.9	3.55	2.88	1.06
0.45	-1.5	3.55	3.13	1.00
0.46	-1.3	3.60	2.69	1.20
0.46	-1.3	3.60	2.78	1.00
0.46	-1.27	3.60	2.83	1.35
0.47	-1.26	3.65	3.08	1.00
0.47	-1.26	3.65	3.08	1.53
0.48	-1.24	3.65	2.73	1.00
0.51	-1.2	3.65	3.08	1.73
0.57	-1.13	3.65	3.24	1.00
2.3	-1.02	3.70	3.10	1.95
2.41	-0.9	3.70	2.98	1.00
2.46	-0.2	3.70	2.98	2.21
2.47	0.0	3.75	3.04	1.00
2.48	0.1	3.80	3.24	2.49
2.51	0.3	3.85	3.16	1.00
2.57	0.5	3.85	3.30	3.04
\end{scontents}

\title{The \texttt{tikzviolinplots} package}
\author{Pedro Callil-Soares}
\date{\today}

\begin{document}

\maketitle

\tableofcontents

\begin{abstract}
	The package provides commands for violin plot creation and
	the kernel density estimations required.
\end{abstract}

\section{Introduction}

This package, through the use of the package \texttt{pgfplots}, allows the
creation of violin plots in \LaTeX. Violin plots are similar to boxplots,
but instead of a box signalling the average and quartiles, a kernel density
estimator is plotted, as in equation \ref{eq:kde}, in which the function $k$
(the kernel) is a probability distribution, the positive number $h$ (the
bandwidth) is a smoothing factor and $n$ is the sample size.

\begin{equation}
	\label{eq:kde}
	\textnormal{KDE}(x) = %
	\cfrac{1}{nh}\sum_{i=1}^nk\left(\cfrac{x-x_i}{h}\right)
\end{equation}

A comparison between the two plots, showcasing its similarities, is
shown in figures \ref{fig:example}. The violin plot in figure
\ref{graph:violin_example} assumes normal data, and the bandwidth
(smoothing factor $h$ in equation \ref{eq:kde}) is defined accordingly.

\pgfplotsset{height=1.6\linewidth}

\begin{figure}[h!]
	\centering
	\begin{subfigure}{0.5\textwidth}
		\centering
		\begin{tikzpicture}
			\begin{axis}
				[
					boxplot/draw direction=y,
					ymax=3, ymin=0,
					xmin=0, xmax=5,
					ymajorgrids=true,
					xtick={1,2,3,4},
					xticklabels={$\alpha$,$\beta$,$\gamma$,$\delta$},
					ylabel={Some property},
				]
				\addplot+[boxplot,
					blue!100!red,
					fill=blue!100!red,
					fill opacity=0.50,
					no marks]
					table [y=A] {violinandboxplotexample.dat};
				\addplot+[boxplot,
					blue!66!red,
					fill=blue!66!red,
					fill opacity=0.50,
					no marks]
					table [y=B] {violinandboxplotexample.dat};
				\addplot+[boxplot,
					blue!33!red,
					fill=blue!33!red,
					fill opacity=0.50,
					no marks]
					table [y=C] {violinandboxplotexample.dat};
				\addplot+[boxplot,
					blue!0!red,
					fill=blue!0!red,
					fill opacity=0.50,
					no marks]
					table [y=D] {violinandboxplotexample.dat};
			\end{axis}
		\end{tikzpicture}
		\caption{Box plot}
		\label{graph:box_example}
	\end{subfigure}%
	\hfill%
	\begin{subfigure}{0.5\textwidth}
		\centering
		\begin{tikzpicture}
			\violinsetoptions[
				averages,
				data points,
				scaled,
			]{
				xmin=0,xmax=5,
				ymin=0,ymax=3,
				xlabel style={
					yshift = {-2*height("a")}
				},
				ymajorgrids=true,
				ylabel={Same property},
			}
			\violinplotwholefile[%
				primary color=red,
				secondary color=blue,
				indexes={A,B,C,D},
				spacing=1.0,
				labels={%
					$\alpha$,
					$\beta$,
					$\gamma$,
					$\delta$,
				},
				col sep=tab,
				dataset size=1pt,
				dataset mark=*,
				dataset fill=black!50!white,
				dataset fill opacity=1.0,
				average mark=x,
				average size=5pt,
			]{violinandboxplotexample.dat}
		\end{tikzpicture}
		\caption{Violin plot}
		\label{graph:violin_example}
	\end{subfigure}
	\caption{Box and violin plot examples}
	\label{fig:example}
\end{figure}

\pgfplotsset{height=0.9\linewidth}

\section{Usage}

To plot a violin plot with the commands provided, one must, inside
a \texttt{tikzpicture} environment, set the general options to all
plots and insert each individual dataset.

To set the general options, the command \texttt{{\textbackslash}violinsetoptions}
is provided, and must be invoked before plotting the datasets. This should
be done with the commands \texttt{{\textbackslash}violinplot} or
\texttt{{\textbackslash}violinplotwholefile}.

\subsection{General options: \texttt{{\textbackslash}violinsetoptions}}

The command \texttt{{\textbackslash}violinsetoptions} takes two arguments,
an optional argument with package-specific options and a mandatory
argument with options to be passed to \texttt{pgfplots}.

\begin{minted}[escapeinside=||]{latex}
	\violinsetoptions[|\textit{<package-specific options>}|]%
		{|\textit{<pgfplots general options>}|}
\end{minted}

\subsubsection{Package-specific options}

There are five options specific to the package: \texttt{scaled},
\texttt{data points}, \texttt{averages}, \texttt{no mirror} and
\texttt{reverse axis}, controlling how and which information
in the datasets should be presented.

The option \texttt{scaled} controls if all plots in the graph
have the same area or same width. If passed, the kernel
distribution estimations will be scaled to the same width, as
shown in figure \ref{graph:violin_verti}; otherwise, the plots
will present the same area, as in figure \ref{graph:violin_horiz}.

The option \texttt{data points}, if passed, will show, along with
the violin plots, the distribution of points in the data set, as shown in
figure \ref{graph:violin_verti}.

If the option \texttt{averages} is passed, the average of the data set
elements is shown, as in figure \ref{graph:violin_horiz}.

The plots are mirrored by default; however, passing the option \texttt{no mirror}
will show only half the plot, as shown in figure \ref{graph:violin_horiz}.

Finally, to ``transpose'' the plots (\textit{i.e.} show the distributions
as functions of the abcissa, as in figure \ref{graph:violin_horiz},
and not as functions of ordinate, as in figure \ref{graph:violin_verti}),
one might use the option \texttt{reverse axis}.

\subsubsection{Plot limits and other \texttt{pgfplots} options}

The minima and maxima of the plot axes must be set in the second (and first
mandatory) argument to the command, and should follow \texttt{pgfplots}
syntax. For instance, to set the minimum and maximum of the $x$-axis
to -3 and 6, and of the $y$-axis to 2.5 and 7, one might use:

\begin{minted}[escapeinside=||]{latex}
	\violinsetoptions[|\textit{<package-specific options>}|]%
		{xmin=-3, xmax=6, ymin=2.5, xmin=7,%
		|\textit{<pgfplots general options>}|}
\end{minted}

Other \texttt{pgfplots} expressions such as title or axes labels may be set
in the same way in this argument.

\subsection{Options for each data set: \texttt{{\textbackslash}violinplot}}

If the data sets are not very similar and/or advanced customizations
are desired, \texttt{{\textbackslash}violinplot} should be used to
plot each data set individually.
This command takes one mandatory argument, and a list of options:

\begin{minted}[escapeinside=||]{latex}
	\violinplot[%
		|\textit{<option>}=\textit{<value>}|
	]{|\textit{filename}|}
\end{minted}

The filename (mandatory argument) must be a path to a file storing the data as
space-separated columns. The optional argument is a list of options, including:

\begin{itemize}
	\item \texttt{col sep}: Column separation character in filename.
		Defaults to \texttt{comma}, and can accept options
		\texttt{space}, \texttt{tab}, \texttt{comma},
		\texttt{colon}, \texttt{semicolon}, \texttt{braces},
		\texttt{\&} and \texttt{ampersand}.
	\item \texttt{index}: Necessary option, is the name of the column with
		the data to be plotted.
	\item \texttt{kernel}: The function to be used for the kernel density
		estimation; available values are \texttt{gaussian} (default),
		\texttt{logistic}, \texttt{parabolic}, \texttt{uniform}
		and \texttt{triangular}.
	\item \texttt{bandwidth}: Smoothing parameter for the kernel density
		estimation; defaults to $h$ in equation \ref{eq:band},
		which assumes gaussian distribution.
		\begin{equation}
			\label{eq:band}
			h=\sqrt[5]{\frac{4\times\textnormal{stddev}%
			\left(x_1,x_2,\ldots,x_n\right)^5}{3n}}
		\end{equation}
	\item \texttt{samples}: Number of points in which the kernel density
		estimation will be calculated and plotted. Defaults
		to 50.
	\item \texttt{relative position}: position of the plot in the axis
		with the labels for each dataset. Defaults to 0.
	\item \texttt{color}: Color of the plot, border and interior.
		Defaults to black. The interior is colored with 0.5
		opacity.
	\item \texttt{label}: Label of the plot, positioned at the point
		set by \texttt{relative position}.
	\item \texttt{invert}: If \texttt{true}, plot will be drawn in opposite
		side. Useful for comparing two parts of a data set, by plotting each
		group side by side.
	\item Options for average and data points:
	\begin{itemize}
		\item \texttt{average mark}, \texttt{dataset mark}: Mark used
			for averages and points of the data set (defaults
			to ``\texttt{x}'' and ``\texttt{*}'', respectively).
		\item \texttt{average size}, \texttt{dataset size}: Size
			of each mark (defaults to 3pt for averages and 1pt
			for other points).
		\item \texttt{average color}, \texttt{dataset color}: Color
			of each mark's border. Defaults to black in both
			cases.
		\item \texttt{average opacity}, \texttt{dataset opacity}:
			Border opacity. Defaults to 1.0 in both cases.
		\item \texttt{average fill}, \texttt{dataset fill}: Color
			of each mark's interior. Defaults to black for the
			points of the data set and white for the averages.
		\item \texttt{average fill opacity}, \texttt{dataset fill opacity}:
			Opacity of the marks. Defaults to 0.5 for averages and
			0.2 for the other points.
	\end{itemize}
\end{itemize}

\subsection{Simplified interface: \texttt{{\textbackslash}violinplotwholefile}}

If the data sets are similar and no customizations are required, one might
use the command \texttt{{\textbackslash}violinplotwholefile}.

\begin{minted}[escapeinside=||]{latex}
	\violinplotwholefile[%
		|\textit{<option>}=\textit{<value>}|
	]{|\textit{filename}|}
\end{minted}

This command will calculate and plot all columns named. The available options
are:

\begin{itemize}
	\item \texttt{primary color}: Primary color utilised. A gradient is built from
		the secondary color to it. Defaults to black.
	\item \texttt{secondary color}: Secondary color utilized. A
		gradient is built from it to the primary color. Defaults
		to white.
	\item \texttt{indexes}: List of columns to be plotted.
	\item \texttt{spacing}: Distance between plots. Defaults to 1.0.
	\item \texttt{labels}: Labels of the data to be plotted.
\end{itemize}

Besides, the options available to \texttt{{\textbackslash}violinplots} are also
available for this command; for instance, one might set the kernel to
be utilized as the \texttt{uniform} kernel through the option
\texttt{kernel=uniform}. Of course, \texttt{relative position} is not an option
here.

\section{Examples}

To show the usage of the package, we plotted several data sets two times, in
figures \ref{graph:violin_verti} and \ref{graph:violin_horiz}, using either the
simplified or the complete interface.

The data are in the file \texttt{example.dat}, shown below:

\colorlet{mintedbgcsv}{black!10!white}

\inputminted[%
	frame=single,
	rulecolor=mintedbgcsv,
	framesep=10pt,
	bgcolor=mintedbgcsv,
	obeytabs,
	tabsize=12,
]{text}{example.dat}

\subsection{Simplified Interface}

First we will plot this data using the simplified interface, resulting in
figure \ref{graph:violin_verti}.

\begin{figure}[h]
	\centering
	\begin{tikzpicture}
		\violinsetoptions[
			data points,
			scaled,
		]{
			xmin=0,xmax=6,
			ymin=-4,ymax=6,
			title={Plotting Some Data},
			xlabel={Groups},
			ylabel={Some property},
			xlabel style={
				yshift = {-2*height("a")}
			},
			ymajorgrids=true,
		}
		\violinplotwholefile[%
			primary color=orange,
			secondary color=black,
			indexes={C,B,A,D,E},
			spacing=1.0,
			labels={%
				$\alpha$,
				$\beta$,
				$\gamma$,
				$\delta$,
				$\epsilon$
			},
			col sep=tab,
			dataset size=2pt,
			dataset mark=diamond*,
			dataset fill=blue!50!white,
			dataset fill opacity=1.0,
		]{example.dat}
	\end{tikzpicture}
	\caption{Violin plot --- vertical example}
	\label{graph:violin_verti}
\end{figure}

Code for figure \ref{graph:violin_verti}, using
\texttt{{\textbackslash}violinplotwholefile}:

\colorlet{mintedbgverti}{orange!10!white}

\begin{minted}[%
	escapeinside=||,
	frame=single,
	rulecolor=mintedbgverti,
	framesep=10pt,
	tabsize=4,
	bgcolor=mintedbgverti]{latex}
\begin{tikzpicture}
	\violinsetoptions[
		data points,
		scaled,
	]{
		xmin=0,xmax=6,
		ymin=-4,ymax=6,
		title={Plotting Some Data},
		xlabel={Groups},
		ylabel={Some property},
		xlabel style={
			yshift = {-2*height("a")}
		},
		ymajorgrids=true,
	}
	\violinplotwholefile[%
		primary color=orange,
		secondary color=black,
		batch indexes={C,B,A,D,E},
		batch spacing=1.0,
		batch labels={%
			$\alpha$,
			$\beta$,
			$\gamma$,
			$\delta$,
			$\epsilon$
		},
		col sep=tab,
		dataset size=2pt,
		dataset mark=diamond*,
		dataset fill=blue!50!white,
		dataset fill opacity=1.0,
	]{example.dat}
\end{tikzpicture}

\end{minted}

\subsection{Complete Interface}

Then the data will be plotted using the complete interface, resulting
in figure \ref{graph:violin_horiz}.

\begin{figure}[h]
	\centering
	\begin{tikzpicture}
		\violinsetoptions[
			no mirror,
			averages,
			reverse axis,
		]{%
			xmin=-4,xmax=6,
			ymin=0.7,ymax=7,
			title={Plotting Some Data},
			ylabel={Groups},
			xlabel={Some property},
			ylabel style={
				yshift = {2*width("a")}
			},
			xmajorgrids=true,
		}
		\violinplot[%
			index=C,
			relative position=1,
			color=blue!0!black,
			label={$\alpha$},
			col sep=tab,
		]{example.dat}
		\violinplot[%
			index=B,
			relative position=2,
			color=blue!25!black,
			label={$\beta$},
			col sep=tab,
		]{example.dat}
		\violinplot[%
			index=A,
			relative position=3,
			color=blue!50!black,
			label={$\gamma$},
			col sep=tab,
			kernel=parabolic,
			samples=100,
			bandwidth=0.4,
			average mark=otimes*,
			average size=2pt,
			average fill=blue!70!black,
			average fill opacity=0.7,
		]{example.dat}
		\violinplot[%
			index=D,
			relative position=4,
			color=blue!75!black,
			label={$\delta$},
			col sep=tab,
		]{example.dat}
		\violinplot[%
			index=E,
			relative position=5,
			color=blue!100!black,
			label={$\epsilon$},
			col sep=tab,
		]{example.dat}
	\end{tikzpicture}
	\caption{Violin plot --- horizontal example}
	\label{graph:violin_horiz}
\end{figure}

Code for figure \ref{graph:violin_horiz}, using
\texttt{{\textbackslash}violinplot}:

\colorlet{mintedbghoriz}{blue!10!white}

\begin{minted}[%
	escapeinside=||,
	frame=single,
	rulecolor=mintedbghoriz,
	framesep=10pt,
	tabsize=4,
	bgcolor=mintedbghoriz]{latex}
\begin{tikzpicture}
	\violinsetoptions[
		no mirror,
		averages,
		reverse axis,
	]{%
		xmin=-4,xmax=6,
		ymin=0.7,ymax=7,
		title={Plotting Some Data},
		ylabel={Groups},
		xlabel={Some property},
		ylabel style={
			yshift = {2*width("a")}
		},
		xmajorgrids=true,
	}
	\violinplot[%
		index=C,
		relative position=1,
		color=blue!0!black,
		label={$\alpha$},
		col sep=tab,
	]{example.dat}
	\violinplot[%
		index=B,
		relative position=2,
		color=blue!25!black,
		label={$\beta$},
		col sep=tab,
	]{example.dat}
	\violinplot[%
		index=A,
		relative position=3,
		color=blue!50!black,
		label={$\gamma$},
		col sep=tab,
		kernel=parabolic,
		samples=100,
		bandwidth=0.4,
		average mark=otimes*,
		average size=2pt,
		average fill=blue!70!black,
		average fill opacity=0.7,
	]{example.dat};
	\violinplot[%
		index=D,
		relative position=4,
		color=blue!75!black,
		label={$\delta$},
		col sep=tab,
	]{example.dat}
	\violinplot[%
		index=E,
		relative position=5,
		color=blue!100!black,
		label={$\epsilon$},
		col sep=tab,
	]{example.dat}
\end{tikzpicture}
\end{minted}

\subsection{Drawings and Annotations}

As everything happens inside a \texttt{tikzpicture} environment, drawings
and annotations are straightforward, as seen in figure \ref{graph:violin_annotations}.

\begin{figure}[h]
	\centering
	\begin{tikzpicture}
		\violinsetoptions[
			averages,
			data points,
		]{
			xmin=0,xmax=6,
			ymin=-2,ymax=5,
			xlabel = {Bandwidth choice},
			ylabel = {Some property},
			title={Problems with Different Distributions},
			xlabel style={
				yshift = {-3*height("a")}
			},
			ymajorgrids=true,
		}
		\violinplot[%
			index=A,
			relative position=1,
			color=green!33!black,
			label={Estimated bandwidth},
			col sep=tab,
		]{example.dat}
		\violinplot[%
			index=A,
			bandwidth=0.1,
			relative position=4,
			color=green!12!black,
			label={Smaller bandwidth},
			col sep=tab,
		]{example.dat}
		\begin{axis}[
			xmin=0,xmax=6,
			ymin=-2,ymax=5,
		]
			\draw[-{Stealth}] (axis cs: 1.75,0.45) %
				.. controls (axis cs:1.75,0.85) and %
				(axis cs:3.9,0.5) .. (axis cs: 3.9,1.05);
			\draw[decoration={raise=2pt,text along path, text=%
				{|\tiny| Better bandwidth}, text align={center}}, %
				decorate] %
				(axis cs: 1.75,0.45) %
				.. controls (axis cs:1.75,0.85) and %
				(axis cs:3.9,0.5) .. (axis cs: 3.9,1.05);
			\draw[-{Stealth}] %
				(axis cs: 1.5,0.45) .. controls (axis cs:1.5,0.85) %
				and (axis cs:1.25,1.1) .. (axis cs: 1.25,1.5) .. %
				controls (axis cs:1.25,1.9) and (axis cs:1.35,1.9) %
				.. (axis cs:1.35,2.6);
			\draw[decoration={raise=2pt,text along path, text=%
				{|\tiny|Very oversmoothed}, text align={center}}, %
				decorate] %
				(axis cs: 1.35,2.6) .. controls (axis cs:1.35,1.9) %
				and (axis cs:1.25,1.9) .. (axis cs: 1.25,1.5) .. %
				controls (axis cs:1.25,1.1) and (axis cs:1.5,0.85) %
				.. (axis cs:1.5,0.45);
		\end{axis}
	\end{tikzpicture}
	\caption{Annotations in a plot}
	\label{graph:violin_annotations}
\end{figure}

Figure \ref{graph:violin_annotations} is compiled from the code excerpt below.

\colorlet{mintedbgannot}{green!10!white}

\begin{minted}[%
	escapeinside=||,
	frame=single,
	rulecolor=mintedbgannot,
	framesep=10pt,
	tabsize=4,
	bgcolor=mintedbgannot]{latex}
\begin{tikzpicture}
	\violinsetoptions[
		averages,
		data points,
	]{
		xmin=0,xmax=6,
		ymin=-2,ymax=5,
		xlabel = {Bandwidth choice},
		ylabel = {Some property},
		title={Problems with Different %
			Distributions},
		xlabel style={
			yshift = {-3*height("a")}
		},
		ymajorgrids=true,
	}
	\violinplot[%
		index=A,
		relative position=1,
		color=green!33!black,
		label={Estimated bandwidth},
		col sep=tab,
	]{example.dat}
	\violinplot[%
		index=A,
		bandwidth=0.1,
		relative position=4,
		color=green!12!black,
		label={Smaller bandwidth},
		col sep=tab,
	]{example.dat}
	\begin{axis}[
		xmin=0,xmax=6,
		ymin=-2,ymax=5,
	]
	\draw[-{Stealth}] (axis cs: 1.75,0.45) %
	.. controls (axis cs:1.75,0.85) and %
	(axis cs:3.9,0.5) .. (axis cs: 3.9,1.05);
	\draw[decoration={raise=2pt,text along path,%
	text={|\tiny| Better bandwidth}, %
	text align={center}}, decorate] %
	(axis cs: 1.75,0.45) %
	.. controls (axis cs:1.75,0.85) and %
	(axis cs:3.9,0.5) .. (axis cs: 3.9,1.05);
	\draw[-{Stealth}] %
	(axis cs: 1.5,0.45) .. controls %
	(axis cs:1.5,0.85) and (axis cs:1.25,1.1) %
	.. (axis cs: 1.25,1.5) .. controls %
	(axis cs:1.25,1.9) and (axis cs:1.35,1.9) %
	.. (axis cs:1.35,2.6);
	\draw[decoration={raise=2pt,text along path, %
	text={|\tiny|Very oversmoothed}, %
	text align={center}}, decorate] %
	(axis cs: 1.35,2.6) .. controls %
	(axis cs:1.35,1.9) and (axis cs:1.25,1.9) %
	.. (axis cs: 1.25,1.5) .. %
	controls (axis cs:1.25,1.1) and %
	(axis cs:1.5,0.85) .. (axis cs:1.5,0.45);
	\end{axis}
\end{tikzpicture}
\end{minted}

\subsection{Standalone Kernel Density Estimation}

Plotting only one dataset and choosing adequate plot limits, one
may obtain a simple representation of a kernel density estimation.
This can be seen in figure \ref{graph:kde_standalone}, obtained from the
code excerpt below.

\begin{figure}[h]
	\centering
	\begin{tikzpicture}
		\violinsetoptions[
			no mirror,
			reverse axis,
			data points,
		]{%
			xmin=3.3,xmax=4,
			ymin=-1,ymax=12,
			title={Plotting Some Data},
			ylabel={$p(x)$},
			xlabel={$x$},
			xmajorgrids=true,
			ymajorticks=true,
			ytick={0.0,2.5,5,7.5,10},
			yticklabels={0.0,2.5,5,7.5,10},
		}
		\violinplot[%
			index=C,
			relative position=0,
			color=red!30!black,
			label={},
			col sep=tab,
			dataset size=5pt,
			dataset color=red!10!black,
			dataset mark=*,
			dataset fill=red!90!black,
			dataset fill opacity=0.2,
		]{example.dat}
	\end{tikzpicture}
	\caption{Standalone kernel density estimation example}
	\label{graph:kde_standalone}
\end{figure}

\colorlet{mintedbgstand}{red!10!white}

\begin{minted}[%
	escapeinside=||,
	frame=single,
	rulecolor=mintedbgstand,
	framesep=10pt,
	tabsize=4,
	bgcolor=mintedbgstand]{latex}
\begin{tikzpicture}
	\violinsetoptions[
		no mirror,
		reverse axis,
		data points,
	]{%
		xmin=3.3,xmax=4,
		ymin=-1,ymax=12,
		title={Plotting Some Data},
		ylabel={$p(x)$},
		xlabel={$x$},
		xmajorgrids=true,
		ymajorticks=true,
		ytick={0.0,2.5,5,7.5,10},
		yticklabels={0.0,2.5,5,7.5,10},
	}
	\violinplot[%
		index=C,
		relative position=0,
		color=red!30!black,
		label={},
		col sep=tab,
		dataset size=5pt,
		dataset color=red!10!black,
		dataset mark=*,
		dataset fill=red!90!black,
		dataset fill opacity=0.2,
	]{example.dat}
\end{tikzpicture}
\end{minted}

\subsection{Asymmetrical Violin Plots --- Real World Data}

Using the key \texttt{invert=true} in the options of
\texttt{\textbackslash violinplot} one can plot two different sets of data side
by side, in an asymmetrical violin plot. This can be seen in figure
\ref{graph:asymmetrical}, that exhibits the male and female life expectancy at
birth, in 2019, for several countries, segregated in the WHO regions.\footnote{
	\url{https://www.who.int/data/gho/publications/world-health-statistics}.
}.

\colorlet{malecolor}{cyan!60!white}
\colorlet{femalecolor}{cyan!60!black}

\begin{figure}[h]
	\centering
	\begin{tikzpicture}
		\violinsetoptions[
			no mirror,
			scaled,
		]{%
			xmin=0.8,xmax=13.2,
			ymin=30,ymax=100,
			title={Life Expectancy Per WHO Region},
			xlabel={WHO Region},
			ylabel={Life Expectancy in Years},
			xlabel style={
				yshift = {-3*width("a")}
			},
			ymajorgrids=true,
		}
		\violinplot[%
			index=LifeExpectancyAtBirthMale,
			relative position=2,
			color=malecolor,
			label={Africa},
			col sep=comma,
			invert=true
		]{AFR.csv}
		\violinplot[%
			index=LifeExpectancyAtBirthFemale,
			relative position=2,
			color=femalecolor,
			label={},
			col sep=comma,
		]{AFR.csv}
		\violinplot[%
			index=LifeExpectancyAtBirthMale,
			relative position=4,
			color=malecolor,
			label={Middle-East},
			col sep=comma,
			invert=true
		]{EMR.csv}
		\violinplot[%
			index=LifeExpectancyAtBirthFemale,
			relative position=4,
			color=femalecolor,
			label={},
			col sep=comma,
		]{EMR.csv}
		\violinplot[%
			index=LifeExpectancyAtBirthMale,
			relative position=6,
			color=malecolor,
			label={Pacific},
			col sep=comma,
			invert=true
		]{WPR.csv}
		\violinplot[%
			index=LifeExpectancyAtBirthFemale,
			relative position=6,
			color=femalecolor,
			label={},
			col sep=comma,
		]{WPR.csv};
		\violinplot[%
			index=LifeExpectancyAtBirthMale,
			relative position=8,
			color=malecolor,
			label={SE. Asia},
			col sep=comma,
			invert=true
		]{SEAR.csv}
		\violinplot[%
			index=LifeExpectancyAtBirthFemale,
			relative position=8,
			color=femalecolor,
			label={},
			col sep=comma,
		]{SEAR.csv}
		\violinplot[%
			index=LifeExpectancyAtBirthMale,
			relative position=10,
			color=malecolor,
			label={Americas},
			col sep=comma,
			invert=true
		]{AMR.csv}
		\violinplot[%
			index=LifeExpectancyAtBirthFemale,
			relative position=10,
			color=femalecolor,
			label={},
			col sep=comma,
		]{AMR.csv}
		\violinplot[%
			index=LifeExpectancyAtBirthMale,
			relative position=12,
			color=malecolor,
			label={Europe},
			col sep=comma,
			invert=true
		]{EUR.csv}
		\violinplot[%
			index=LifeExpectancyAtBirthFemale,
			relative position=12,
			color=femalecolor,
			label={},
			col sep=comma,
		]{EUR.csv}
		\begin{axis}[
			xmin=0.8,xmax=13.2,
			ymin=30,ymax=100,
		]
			\draw(axis cs:2,35) node[anchor=east] {\male};
			\draw(axis cs:2,35) node[anchor=west] {\female};
			\draw(axis cs:4,35) node[anchor=east] {\male};
			\draw(axis cs:4,35) node[anchor=west] {\female};
			\draw(axis cs:6,35) node[anchor=east] {\male};
			\draw(axis cs:6,35) node[anchor=west] {\female};
			\draw(axis cs:8,35) node[anchor=east] {\male};
			\draw(axis cs:8,35) node[anchor=west] {\female};
			\draw(axis cs:10,35) node[anchor=east] {\male};
			\draw(axis cs:10,35) node[anchor=west] {\female};
			\draw(axis cs:12,35) node[anchor=east] {\male};
			\draw(axis cs:12,35) node[anchor=west] {\female};
		\end{axis}
	\end{tikzpicture}
	\caption{Example of asymmetrical plot}
	\label{graph:asymmetrical}
\end{figure}

The code for figure \ref{graph:asymmetrical} is avaliable below:

\begin{minted}[%
	escapeinside=||,
	frame=single,
	rulecolor=cyan!10!white,
	framesep=10pt,
	tabsize=4,
	bgcolor=cyan!10!white]{latex}
\begin{tikzpicture}
	\violinsetoptions[
		no mirror,
		scaled,
	]{%
		xmin=0,xmax=14,
		ymin=30,ymax=100,
		title={Life Expectancy Per WHO Region},
		xlabel={WHO Region},
		ylabel={Life Expectancy in Years},
		xlabel style={
			yshift = {-3*width("a")}
		},
		ymajorgrids=true,
	}
	\violinplot[%
		index=LifeExpectancyAtBirthMale,
		relative position=2,
		color=malecolor,
		label={AFR},
		col sep=comma,
		invert=true
	]{AFR.csv}
	\violinplot[%
		index=LifeExpectancyAtBirthFemale,
		relative position=2,
		color=femalecolor,
		label={},
		col sep=comma,
	]{AFR.csv}
	\violinplot[%
		index=LifeExpectancyAtBirthMale,
		relative position=4,
		color=malecolor,
		label={EMR},
		col sep=comma,
		invert=true
	]{EMR.csv}
	\violinplot[%
		index=LifeExpectancyAtBirthFemale,
		relative position=4,
		color=femalecolor,
		label={},
		col sep=comma,
	]{EMR.csv}
	\violinplot[%
		index=LifeExpectancyAtBirthMale,
		relative position=6,
		color=malecolor,
		label={WPR},
		col sep=comma,
		invert=true
	]{WPR.csv}
	\violinplot[%
		index=LifeExpectancyAtBirthFemale,
		relative position=6,
		color=femalecolor,
		label={},
		col sep=comma,
	]{WPR.csv};
	\violinplot[%
		index=LifeExpectancyAtBirthMale,
		relative position=8,
		color=malecolor,
		label={SEAR},
		col sep=comma,
		invert=true
	]{SEAR.csv}
	\violinplot[%
		index=LifeExpectancyAtBirthFemale,
		relative position=8,
		color=femalecolor,
		label={},
		col sep=comma,
	]{SEAR.csv}
	\violinplot[%
		index=LifeExpectancyAtBirthMale,
		relative position=10,
		color=malecolor,
		label={AMR},
		col sep=comma,
		invert=true
	]{AMR.csv}
	\violinplot[%
		index=LifeExpectancyAtBirthFemale,
		relative position=10,
		color=femalecolor,
		label={},
		col sep=comma,
	]{AMR.csv}
	\violinplot[%
		index=LifeExpectancyAtBirthMale,
		relative position=12,
		color=malecolor,
		label={EUR},
		col sep=comma,
		invert=true
	]{EUR.csv}
	\violinplot[%
		index=LifeExpectancyAtBirthFemale,
		relative position=12,
		color=femalecolor,
		label={},
		col sep=comma,
	]{EUR.csv}
	\begin{axis}[
		xmin=0,xmax=14,
		ymin=30,ymax=100,
	]
		\draw(axis cs:2,35) node[anchor=east] {\male};
		\draw(axis cs:2,35) node[anchor=west] {\female};
		\draw(axis cs:4,35) node[anchor=east] {\male};
		\draw(axis cs:4,35) node[anchor=west] {\female};
		\draw(axis cs:6,35) node[anchor=east] {\male};
		\draw(axis cs:6,35) node[anchor=west] {\female};
		\draw(axis cs:8,35) node[anchor=east] {\male};
		\draw(axis cs:8,35) node[anchor=west] {\female};
		\draw(axis cs:10,35) node[anchor=east] {\male};
		\draw(axis cs:10,35) node[anchor=west] {\female};
		\draw(axis cs:12,35) node[anchor=east] {\male};
		\draw(axis cs:12,35) node[anchor=west] {\female};
	\end{axis}
\end{tikzpicture}
\end{minted}

\section{Limitations}

As the math is handled through \TeX, generating the kernel distribution
estimations is a slow process; therefore, if possible, the number of samples
should not be very large. Also, each value should not, itself, be too large,
to avoid ``dimension too large'' errors. In this case, the data should be scaled
and the tick labels manually corrected.

As each violin plot is rendered in a different \texttt{axis} environment,
the positions of the axes' labels are not, usually, correct. One should
set it manually. In the examples, this was accomplished setting the
vertical/horizontal shift to twice the height of a letter, in the direction
opposite to the appropriate axis. This approach also leads to another problem:
using the keys ``height'' or ``width'' won't rescale the plot; one might use,
before and after the code for the figure, \texttt{\textbackslash pgfplotsset},
as shown below:

\colorlet{mintedbgtempplotsset}{red!50!blue}
\colorlet{mintedbgplotsset}{mintedbgtempplotsset!10!white}

\begin{minted}[%
	escapeinside=||,
	frame=single,
	rulecolor=mintedbgplotsset,
	framesep=10pt,
	tabsize=4,
	bgcolor=mintedbgplotsset]{latex}
\pgfplotsset{height=1.75\linewidth}
\begin{tikzpicture}
	\violinsetoptions[
		|\textit{options for \texttt{tikzviolinplots}}|
	]{
		|\textit{\texttt{pgfplots} options}|
	}
	\violinplot[%
		|\textit{options for data points}|
	]{|\textit{filename}|}
	|...|
\end{tikzpicture}
\pgfplotsset{height=0.9\linewidth}
\end{minted}

For the same reason, the options \texttt{xmin}, \texttt{ymin}, \texttt{xmax}
and \texttt{ymax} are required; no automatic placement is performed.

\end{document}
