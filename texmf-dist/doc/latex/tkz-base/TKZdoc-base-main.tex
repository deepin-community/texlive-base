%!TEX TS-program = lualatex
%  encoding: utf8 
%  documentation of tkz-base.sty  
% Copyright 2024  Alain Matthes
% This work may be distributed and/or modified under the
% conditions of the LaTeX Project Public License, either version 1.3
% of this license or (at your option) any later version.
% The latest version of this license is in
%   http://www.latex-project.org/lppl.txt
% and version 1.3 or later is part of all distributions of LaTeX
% version 2005/12/01 or later.
% This work has the LPPL maintenance status “maintained”. 
% The Current Maintainer of this work is Alain Matthes.
\PassOptionsToPackage{unicode}{hyperref}
\documentclass[DIV         = 14,
               fontsize    = 10,
               index       = totoc,
               twoside,
               cadre,
               headings    = small
               ]{tkz-doc}
%\usepackage{etoc}
\gdef\tkznameofpack{tkz-base}
\gdef\tkzversionofpack{4.21c}
\gdef\tkzdateofpack{\today}
\gdef\tkznameofdoc{doc-tkz-base}
\gdef\tkzversionofdoc{4.21c} 
\gdef\tkzdateofdoc{\today}
\gdef\tkzauthorofpack{Alain Matthes}
\gdef\tkzadressofauthor{}
\gdef\tkznamecollection{AlterMundus}
\gdef\tkzurlauthor{http://altermundus.fr}
\gdef\tkzengine{lualatex}
\gdef\tkzurlauthorcom{http://altermundus.fr}
\nameoffile{\tkznameofpack}
% -- Packages ---------------------------------------------------          
\usepackage{calc}
\usepackage{tkz-base,tkz-euclide,pgfornament}
\usepackage[colorlinks,pdfencoding=auto, psdextra]{hyperref}
\hypersetup{
      linkcolor=Gray,
      citecolor=Green,
      filecolor=Mulberry,
      urlcolor=NavyBlue,
      menucolor=Gray,
      runcolor=Mulberry,
      linkbordercolor=Gray,
      citebordercolor=Green,
      filebordercolor=Mulberry,
      urlbordercolor=NavyBlue,
      menubordercolor=Gray,
      runbordercolor=Mulberry,
      pdfsubject={Cartesian System},
      pdfauthor={\tkzauthorofpack},
      pdftitle={\tkznameofpack},
      pdfkeywords={tikz, pgf, pdf, pdflatex, graphic, euclide,lualatex,
      geometry, points, maths, line, circle, angle ,polygon},
      pdfcreator={\tkzengine}
}
\usepackage{tkzexample}
%\usepackage{mathtools}
\usepackage{fontspec}
\setmainfont{texgyrepagella}[
  Extension = .otf,
  UprightFont = *-regular ,
  ItalicFont  = *-italic  ,
  BoldFont    = *-bold    ,
  BoldItalicFont = *-bolditalic
]
\setsansfont{texgyreheros}[
  Extension = .otf,
  UprightFont = *-regular ,
  ItalicFont  = *-italic  ,
  BoldFont    = *-bold    ,
  BoldItalicFont = *-bolditalic ,
]

\setmonofont{lmmono10-regular.otf}[
  Numbers={Lining,SlashedZero},
  ItalicFont=lmmonoslant10-regular.otf,
  BoldFont=lmmonolt10-bold.otf,
  BoldItalicFont=lmmonolt10-boldoblique.otf,
]
\newfontfamily\ttcondensed{lmmonoltcond10-regular.otf}
\linespread{1.05}   
\usepackage[math-style=literal,bold-style=literal]{unicode-math}
\usepackage{fourier-otf}
\usepackage{datetime,multicol,lscape}
\usepackage[english]{babel}
\usepackage[autolanguage]{numprint}
\usepackage[normalem]{ulem}
%\usepackage{microtype}
\usepackage{array,multirow,multido,booktabs}
\usepackage{shortvrb,fancyvrb,bookmark} 
\usepackage{makeidx}

\AtBeginDocument{\MakeShortVerb{\|}} % link to shortvrb
%\@twocolumnfalse
\makeindex 
%<--------------------------------------------------------------------------->% settings styles
\tkzSetUpColors[background=white,text=black]  
\tkzSetUpCompass[color=orange, line width=.2pt,delta=10]
\tkzSetUpArc[color=gray,line width=.2pt]
\tkzSetUpPoint[size=2,color=teal]
\tkzSetUpLine[line width=.2pt,color=teal]
\tkzSetUpStyle[color=orange,line width=.2pt]{new}
\tikzset{every picture/.style={line width=.2pt}}
\tikzset{label angle style/.append style={color=teal,font=\footnotesize}}
\tikzset{new/.style={color=orange,line width=.2pt}}  
\tikzset{label style/.append style={below,color=teal,font=\scriptsize}} 

% \def\tkzref{\arabic{section}-\arabic{subsection}-\arabic{subsubsection}}
% \renewenvironment{tkzexample}[1][]{%
%  \tkz@killienc \VerbatimOut{tkzbase-\tkzref.tex}%
%   }{%
% \endVerbatimOut
% }


\begin{document}

\parindent=0pt
\tkzTitleFrame{tkz-base\\Cartesian System}
\clearpage


\defoffile{\tkzname{\tkznameofpack} is a package based on \TIKZ\ to make graphics as simple as possible. It is the basis on which a series of packages will be built, having as a common point, the creation of drawings useful in the teaching of mathematics. The main function of \tkzname{\tkznameofpack} is to provide an orthogonal coordinate system, and to let the user choose the graphical units.  This package requires version 3 or higher of \TIKZ.{\color{red} You must load \tkzimp{tkz-base} before \tkzimp{tkz-euclide} or \tkzimp{tkz-fct}.} }

\presentation

\vspace*{1cm} 
\noindent\space I'd like to thank \textbf{Till~Tantau} for creating the wonderful tool \href{http://sourceforge.net/projects/pgf/}{\TIKZ}.

\vspace*{12pt} 
\noindent\space I thank \textbf{Yves~Combe} for sharing his work on the protractor and the compass constructions. I would also like to thank, \tkzimp{David~Arnold} who corrected a lot of errors and tested many examples, \tkzimp{Wolfgang~Büchel} who also corrected errors and built great scripts to get the example files,  \tkzimp{John~Kitzmiller} and \tkzimp{Dimitri~Kapetas} for their examples, \tkzimp{Gaétan~Marris} for his remarks and corrections, and finally \tkzimp{Laurent Van Deik} for all his corrections, remarks and questions.

\vspace*{12pt}
\noindent\space You will find many examples on my site: 
\href{http://altermundus.fr}{altermundus.fr}.

\vfill
You can send your remarks, and reports on errors you find, to the following address: \href{mailto:al.ma@mac.com}{\textcolor{pdfurlcolor}{\tkzauthorofpack}}.
 
This file can be redistributed and/or modified under the terms of the \LATEX\ 
Project Public License Distributed from \href{http://www.ctan.org/}{CTAN}\  archives.


\clearpage
\tableofcontents

\clearpage 
\newpage

\setlength{\parskip}{1ex plus 0.5ex minus 0.2ex} 
%<------------- includes   -----------------------------------------------
\section{News and presentation}

This package was the foundation of the \tkzNamePack{tkz-euclide} and \tkzNamePack{tkz-fct} in particular. Now \tkzimp{tkz-euclide} is independent of \tkzname{\tkznameofpack}.  \tkzimp{tkz-euclide} should be used only for Euclidean geometry.  The package has been modified and object transfers between 
\tkzimp{tkz-base} and \tkzimp{tkz-euclide} have been performed. 

\tkzimp{tkz-base} provides a Cartesian system that will be defined by the macro \tkzcname{tkzInit}. The big difference now between \tkzname{\tkznameofpack} and \tkzimp{tkz-euclide} is the role of the units. The unit in \tkzimp{tkz-euclide} is the cm and is fixed. This is not the case with \tkzimp{tkz-base}.

The main novelty is the recent replacement of the \tkzNamePack{fp} package by \tkzNamePack{xfp}. The appearance of this one is a step towards version 3 of \LATEX.
 The next step will be the creation of a new package.

Here are some of the changes. The  \tkzimp{tkz-euclide} package brings more new features.  \tkzimp{tkz-euclide} is used for some examples in this documentation.

\vspace{2cm}
 \begin{itemize}\setlength{\itemsep}{10pt} 
\item  Code Improvement;
\item  Bug correction;
\item  The bounding box is now controlled in each macro (hopefully) to avoid the use of \tkzcname{tkzInit} followed by \tkzcname{tkzClip};
\item  Logically most macros accept \TIKZ\ options. So I removed the "duplicate" options;
\item  Removing the option "label options";
\item  Random points are now in \tkzimp{tkz-euclide} and the macro \tkzcname{tkzGetRandPointOn} is replaced by \tkzcname{tkzDefRandPointOn}. For homogeneity reasons, the points must be retrieved with \tkzcname{tkzGetPoint};
\item The options \tkzimp{end} and \tkzimp{start} which allowed to give a label to a line are removed. You must now use the macro \tkzcname{tkzLabelLine};

\item Introduction of the libraries \NameLib{quotes} and \NameLib{angles} they allows to give a label to a point.even if I am not in favour of this practice;

\item Appearance of the macro \tkzcname{usetkztool}, which allows to load new "tools".
\end{itemize}

\endinput
\section{Installation}

\tkzname{\tkznameofpack} is now on the server of the \tkzname{CTAN}\footnote{\tkzname{\tkznameofpack} is part of \NameDist{TeXLive} and \tkzname{tlmgr} allows you to install them. This package is also part of \NameDist{MiKTeX} under \NameSys{Windows}.}. If you want to test a beta version, just put the following files in a texmf folder that your system can find.
You will have to check several points:

\begin{itemize}\setlength{\itemsep}{5pt}
\item  The \tkzname{\tkznameofpack} folder must be located on a path recognized by \tkzname{latex}.
\item  The  \tkzname{\tkznameofpack} uses \tkzNamePack{xfp}.
\item This documentation and all examples were obtained with \tkzname{lualatex} but \tkzname{pdflatex} or \tkzname{xelatex} should be suitable.
\end{itemize}

\endinput

\include{TKZdoc-base-compilation}
\include{TKZdoc-base-initialisation}
\include{TKZdoc-base-axes}
\include{TKZdoc-base-grid}
\include{TKZdoc-base-point}
\include{TKZdoc-base-style}
\section{Controlling Bounding Box}
From the \tkzimp{PgfManual} :"When you add the clip option, the current path is used for clipping subsequent drawings. Clipping never enlarges the clipping area. Thus, when you clip against a certain path and then clip again against another path, you clip against the intersection of both.
The only way to enlarge the clipping path is to end the {pgfscope} in which the clipping was done. At the end of a {pgfscope} the clipping path that was in force at the beginning of the scope is reinstalled."


First of all, you don't have to deal with \TIKZ\ the size of the bounding box. Early versions of \tkzNamePack{tkz-euclide} did not control the size of the bounding box, now with \tkzNamePack{\tkznameofpack} 4 the size of the bounding box is limited.

The initial bounding box after using the macro \tkzcname{tkzInit} is defined by the rectangle based on the points $(0,0)$ and $(10,10)$. The \tkzcname{tkzInit} macro allows this initial bounding box to be modified using the arguments (\tkzname{xmin}, \tkzname{xmax}, \tkzname{ymin}, and \tkzname{ymax}). Of course any external trace modifies the bounding box. \TIKZ\ maintains that bounding box. It is possible to influence this behavior either directly with commands or options in \TIKZ\ such as a command like \tkzcname{useasboundingbox} or the option \tkzname{use as bounding box}. A possible consequence is to reserve a box for a figure but the figure may overflow the box and spread over the main text.
The following command \tkzcname{pgfresetboundingbox} clears a bounding box and establishes a new one.

\subsection{Utility of \tkzcname{tkzInit}} 
 However, it is sometimes necessary to control the size of what will be displayed.
 To do this, you need to have prepared the bounding box you are going to work in, this is the role of the   macro \tkzNameMacro{tkzInit}.  For some drawings, it is interesting to fix the extreme values (xmin,xmax,ymin and ymax) and to "clip" the definition rectangle in order to control the size of the figure as well as possible.

The two macros that are useful for controlling the bounding box:
\begin{itemize}
   \item \tkzcname{tkzInit}
   \item \tkzcname{tkzClip}
\end{itemize}
\vspace{20pt}

To this, I added macros directly linked to the bounding box. You can now view it, backup it, restore it (see the  section Bounding Box).

\subsection{\tkzcname{tkzInit}}

\begin{NewMacroBox}{tkzInit}{\oarg{local options}}\hypertarget{init}{}%
\begin{tabular}{lll}%    
options  & default & definition             \\
\midrule    
\TOline{xmin} {0} {minimum value of the abscissae in cm}
\TOline{xmax} {10} {maximum value of the abscissae in cm}
\TOline{xstep}{1} {difference between two graduations in $x$}
\TOline{ymin} {0} {minimum y-axis value in cm }
\TOline{ymax} {10} {maximum y-axis value in cm}
\TOline{ystep}{1} {difference between two graduations in $y$}  
\bottomrule    
\end{tabular}

\medskip 

The role of \tkzcname{tkzInit} is to define a \textcolor{red}{orthogonal} coordinates system and a rectangular part of the plane in which you will place your drawings using Cartesian coordinates. 
This macro allows you to define your working environment as with a calculator. With \tkzname{\tkznameofpack} 4 \tkzcname{xstep}  and \tkzcname{ystep} are always 1. Logically it is no longer useful to use \tkzcname{tkzInit}, except for an action like "Clipping Out".
\end{NewMacroBox}


\subsection{\tkzcname{tkzClip}}

\subsection{tkzClip}
\begin{NewMacroBox}{tkzClip}{\oarg{local options}}
The role of this macro is to make invisible what is outside the rectangle defined by (xmin~;~ymin) and (xmax~;~ymax).

\medskip
\begin{tabular}{lll}
\hline
options  & default & definition             \\
\midrule
\TOline{space} {1} {added value on the right, left, bottom and top of the background}
\bottomrule
\end{tabular}

\medskip

The role of the \tkzname{space} option is to enlarge the visible part of the drawing. This part becomes the rectangle defined by (xmin-space~;~ymin-space) and (xmax+space~;~ymax+space).  \tkzname{space} can be negative!  The unit is cm and should not be specified.
\end{NewMacroBox}

The role of this macro is to "clip" the initial rectangle so that only the paths contained in this rectangle are drawn.

\begin{tkzexample}[latex=8cm,small]
\begin{tikzpicture}
 \tkzInit[xmax=4, ymax=3]
 \tkzDefPoints{-1/-1/A,5/2/B}
 \tkzDrawX \tkzDrawY 
 \tkzGrid
 \tkzClip
 \tkzDrawSegment(A,B)
\end{tikzpicture}
\end{tkzexample} 

It is possible to add a bit of space
\begin{tkzltxexample}[]
  \tkzClip[space=1]
\end{tkzltxexample} 

\subsection{\tkzcname{tkzClip} and the option \tkzname{space}} 
This option allows you to add some space around the "clipped" rectangle.
\begin{tkzexample}[latex=8cm,small]
\begin{tikzpicture}
 \tkzInit[xmax=4, ymax=3]
 \tkzDefPoints{-1/-1/A,5/2/B}
 \tkzDrawX \tkzDrawY 
 \tkzGrid
 \tkzClip[space=1]
 \tkzDrawSegment(A,B)
\end{tikzpicture}
\end{tkzexample}   
The dimensions of the "clipped" rectangle are \tkzname{xmin-1}, \tkzname{ymin-1}, \tkzname{xmax+1} and \tkzname{ymax+1}. 

%<--------------------------------------------------------------------------->
%              tkzShowBB
%<--------------------------------------------------------------------------->
\subsection{tkzShowBB}
The simplest macro. 
\begin{NewMacroBox}{tkzShowBB}{\oarg{local options}}% 
This macro displays the bounding box. A rectangular frame surrounds the bounding box. This macro accepts \TIKZ\ options.
\end{NewMacroBox} 

\subsubsection{Example with \tkzcname{tkzShowBB}}
\begin{tkzexample}[latex=8cm,small]
\begin{tikzpicture}[scale=.5]
  \tkzInit[ymax=5,xmax=8]
  \tkzGrid  
  \tkzDefPoint(3,0){A}
   \begin{scope}
    \tkzClipBB
    \tkzDefCircle[R](A,5)
    \tkzDrawCircle(A,tkzPointResult)
     \tkzShowBB[line width = 4pt,fill=teal!10,opacity=.4]
   \end{scope}
\tkzDefCircle[R](A,4)
\tkzDrawCircle(A,tkzPointResult)
\end{tikzpicture}
\end{tkzexample}
%<--------------------------------------------------------------------------->
%         tkzClipBB
%<--------------------------------------------------------------------------->
\subsection{tkzClipBB}
\begin{NewMacroBox}{tkzClipBB}{}%
The idea is to limit future constructions to the current bounding box.
\end{NewMacroBox}

\subsubsection{Example with \tkzcname{tkzClipBB} and the bisectors}

\begin{tkzexample}[latex=6cm,small]
  \begin{tikzpicture}
  \tkzInit[xmin=-3,xmax=6, ymin=-1,ymax=6]
  \tkzDefPoint(0,0){O}\tkzDefPoint(3,1){I}
  \tkzDefPoint(1,4){J}
  \tkzDefLine[bisector](I,O,J) \tkzGetPoint{i}
  \tkzDefLine[bisector out](I,O,J) \tkzGetPoint{j}
  \tkzDrawPoints(O,I,J,i,j)
  \tkzClipBB
  \tkzDrawLines[add = 1 and 2,color=red](O,I O,J)
  \tkzDrawLines[add = 1 and 2,color=blue](O,i O,j)
  \tkzShowBB
  \end{tikzpicture}
\end{tkzexample}

\endinput
\include{TKZdoc-base-obj}
\include{TKZdoc-base-rep}
\include{TKZdoc-base-divers}
\include{TKZdoc-base-marks}
\include{TKZdoc-base-texte}
\include{TKZdoc-base-faq}
%<------------------------------------------------------------------------
\clearpage\newpage 
\makeatletter

\begin{multicols}{2}
\small\printindex
\end{multicols}
\end{document}

