%!TEX TS-program = lualatex
%  encoding: utf8 
%  documentation of tkz-base.sty  
% Copyright 2024  Alain Matthes
% This work may be distributed and/or modified under the
% conditions of the LaTeX Project Public License, either version 1.3
% of this license or (at your option) any later version.
% The latest version of this license is in
%   http://www.latex-project.org/lppl.txt
% and version 1.3 or later is part of all distributions of LaTeX
% version 2005/12/01 or later.
% This work has the LPPL maintenance status “maintained”. 
% The Current Maintainer of this work is Alain Matthes.
\PassOptionsToPackage{unicode}{hyperref}
\documentclass[DIV         = 14,
               fontsize    = 10,
               index       = totoc,
               twoside,
               cadre,
               headings    = small
               ]{tkz-doc}
%\usepackage{etoc}
\gdef\tkznameofpack{tkz-base}
\gdef\tkzversionofpack{4.21c}
\gdef\tkzdateofpack{\today}
\gdef\tkznameofdoc{doc-tkz-base}
\gdef\tkzversionofdoc{4.21c} 
\gdef\tkzdateofdoc{\today}
\gdef\tkzauthorofpack{Alain Matthes}
\gdef\tkzadressofauthor{}
\gdef\tkznamecollection{AlterMundus}
\gdef\tkzurlauthor{http://altermundus.fr}
\gdef\tkzengine{lualatex}
\gdef\tkzurlauthorcom{http://altermundus.fr}
\nameoffile{\tkznameofpack}
% -- Packages ---------------------------------------------------          
\usepackage{calc}
\usepackage{tkz-base,tkz-euclide,pgfornament}
\usepackage[colorlinks,pdfencoding=auto, psdextra]{hyperref}
\hypersetup{
      linkcolor=Gray,
      citecolor=Green,
      filecolor=Mulberry,
      urlcolor=NavyBlue,
      menucolor=Gray,
      runcolor=Mulberry,
      linkbordercolor=Gray,
      citebordercolor=Green,
      filebordercolor=Mulberry,
      urlbordercolor=NavyBlue,
      menubordercolor=Gray,
      runbordercolor=Mulberry,
      pdfsubject={Cartesian System},
      pdfauthor={\tkzauthorofpack},
      pdftitle={\tkznameofpack},
      pdfkeywords={tikz, pgf, pdf, pdflatex, graphic, euclide,lualatex,
      geometry, points, maths, line, circle, angle ,polygon},
      pdfcreator={\tkzengine}
}
\usepackage{tkzexample}
%\usepackage{mathtools}
\usepackage{fontspec}
\setmainfont{texgyrepagella}[
  Extension = .otf,
  UprightFont = *-regular ,
  ItalicFont  = *-italic  ,
  BoldFont    = *-bold    ,
  BoldItalicFont = *-bolditalic
]
\setsansfont{texgyreheros}[
  Extension = .otf,
  UprightFont = *-regular ,
  ItalicFont  = *-italic  ,
  BoldFont    = *-bold    ,
  BoldItalicFont = *-bolditalic ,
]

\setmonofont{lmmono10-regular.otf}[
  Numbers={Lining,SlashedZero},
  ItalicFont=lmmonoslant10-regular.otf,
  BoldFont=lmmonolt10-bold.otf,
  BoldItalicFont=lmmonolt10-boldoblique.otf,
]
\newfontfamily\ttcondensed{lmmonoltcond10-regular.otf}
\linespread{1.05}   
\usepackage[math-style=literal,bold-style=literal]{unicode-math}
\usepackage{fourier-otf}
\usepackage{datetime,multicol,lscape}
\usepackage[english]{babel}
\usepackage[autolanguage]{numprint}
\usepackage[normalem]{ulem}
%\usepackage{microtype}
\usepackage{array,multirow,multido,booktabs}
\usepackage{shortvrb,fancyvrb,bookmark} 
\usepackage{makeidx}

\AtBeginDocument{\MakeShortVerb{\|}} % link to shortvrb
%\@twocolumnfalse
\makeindex 
%<--------------------------------------------------------------------------->% settings styles
\tkzSetUpColors[background=white,text=black]  
\tkzSetUpCompass[color=orange, line width=.2pt,delta=10]
\tkzSetUpArc[color=gray,line width=.2pt]
\tkzSetUpPoint[size=2,color=teal]
\tkzSetUpLine[line width=.2pt,color=teal]
\tkzSetUpStyle[color=orange,line width=.2pt]{new}
\tikzset{every picture/.style={line width=.2pt}}
\tikzset{label angle style/.append style={color=teal,font=\footnotesize}}
\tikzset{new/.style={color=orange,line width=.2pt}}  
\tikzset{label style/.append style={below,color=teal,font=\scriptsize}} 

% \def\tkzref{\arabic{section}-\arabic{subsection}-\arabic{subsubsection}}
% \renewenvironment{tkzexample}[1][]{%
%  \tkz@killienc \VerbatimOut{tkzbase-\tkzref.tex}%
%   }{%
% \endVerbatimOut
% }


\begin{document}

\parindent=0pt
\tkzTitleFrame{tkz-base\\Cartesian System}
\clearpage


\defoffile{\tkzname{\tkznameofpack} is a package based on \TIKZ\ to make graphics as simple as possible. It is the basis on which a series of packages will be built, having as a common point, the creation of drawings useful in the teaching of mathematics. The main function of \tkzname{\tkznameofpack} is to provide an orthogonal coordinate system, and to let the user choose the graphical units.  This package requires version 3 or higher of \TIKZ.{\color{red} You must load \tkzimp{tkz-base} before \tkzimp{tkz-euclide} or \tkzimp{tkz-fct}.} }

\presentation

\vspace*{1cm} 
\noindent\space I'd like to thank \textbf{Till~Tantau} for creating the wonderful tool \href{http://sourceforge.net/projects/pgf/}{\TIKZ}.

\vspace*{12pt} 
\noindent\space I thank \textbf{Yves~Combe} for sharing his work on the protractor and the compass constructions. I would also like to thank, \tkzimp{David~Arnold} who corrected a lot of errors and tested many examples, \tkzimp{Wolfgang~Büchel} who also corrected errors and built great scripts to get the example files,  \tkzimp{John~Kitzmiller} and \tkzimp{Dimitri~Kapetas} for their examples, \tkzimp{Gaétan~Marris} for his remarks and corrections, and finally \tkzimp{Laurent Van Deik} for all his corrections, remarks and questions.

\vspace*{12pt}
\noindent\space You will find many examples on my site: 
\href{http://altermundus.fr}{altermundus.fr}.

\vfill
You can send your remarks, and reports on errors you find, to the following address: \href{mailto:al.ma@mac.com}{\textcolor{pdfurlcolor}{\tkzauthorofpack}}.
 
This file can be redistributed and/or modified under the terms of the \LATEX\ 
Project Public License Distributed from \href{http://www.ctan.org/}{CTAN}\  archives.


\clearpage
\tableofcontents

\clearpage 
\newpage

\setlength{\parskip}{1ex plus 0.5ex minus 0.2ex} 
%<------------- includes   -----------------------------------------------
\include{TKZdoc-base-news}
\include{TKZdoc-base-installation}
\include{TKZdoc-base-compilation}
\include{TKZdoc-base-initialisation}
\include{TKZdoc-base-axes}
\include{TKZdoc-base-grid}
\section{The points}

I made a distinction between the point used in Euclidean geometry and the point used to represent an element of a statistical cloud. In the first case, I use as object a \tkzname{node}, which means that the representation of the point cannot be modified by a \tkzname{scale}; in the second case, I use as object a \tkzname{plot mark}. The latter can be scaled and have more varied forms than the node.

The new macro is \tkzNameMacro{tkzDefPoint}, it allows to use \TIKZ-specific options as a shift and the values are processed with tkz-base. Moreover, if calculations are needed then the \tkzNamePack{xfp} package takes care of them. You can use Cartesian or polar coordinates.

\subsection{Defining a point in Cartesian coordinates: \tkzcname{tkzDefPoint}} \hypertarget{tdp}{}

\begin{NewMacroBox}{tkzDefPoint}{\oarg{local options}\parg{x,y}\var{name} or \parg{a:r}\var{name}}%
\begin{tabular}{lll}%
arguments &  default & definition  \\ 
\midrule
\TAline{x,y}{no default}{$x$ and $y$ are two dimensions, by default in cm.}
\TAline{a:r}{no default}{$a$ is an angle in degrees, $r$ is a dimension}
\bottomrule
\end{tabular}

\medskip
\noindent{The mandatory arguments of this macro are two dimensions expressed with decimals, in the first case they are two measures of length, in the second case they are a measure of length and the measure of an angle in degrees}.

\medskip
\begin{tabular}{lll}%
\toprule
options             & default & definition   \\ 
\midrule
\TOline{shift} {(0,0)} {value spacing}
 \bottomrule
\end{tabular}

\medskip
\noindent{All the options of \TIKZ\ that we can apply to \tkzname{coordinate}, are applicable (well I hope!) as for example the option \tkzname{label} defined with the library \tkzname{quotes}.}
\end{NewMacroBox}

\subsubsection{Use of \tkzname{shift}}
\tkzname{shift} allows the points to be placed in relation to each other. 

\begin{tkzexample}[latex=7cm,small]
\begin{tikzpicture}[trim left=-1cm]
 \tkzDefPoint(2,3){A}
 \tkzDefPoint[shift={(2,3)}](31:3){B}  
 \tkzDefPoint[shift={(2,3)}](158:3){C}
 \tkzDrawSegments[color=red,line width=1pt](A,B A,C) 
 \tkzDrawPoints[color=red](A,B,C)
\end{tikzpicture}
\end{tkzexample}

\subsection{Placing a label with the library \tkzname{quotes}}
I prefer not to mix operations and use \tkzcname{tkzLabelPoint} to place labels. See the section "The  Quotes Syntax" in the \TIKZ\ manual.

\begin{tkzexample}[latex=7cm,small]
\begin{tikzpicture}[trim left=-1cm]
 \tkzDefPoint["-60:$A_n$" ](2,3){A}
 \tkzDefPoint[shift={(2,3)},%
    "$B_n$" above left](31:3){B}  
 \tkzDefPoint[shift={(2,3)},%
     "$C_n$" above right](158:3){C}
 \tkzDrawSegments[color=red,%
          line width=1pt](A,B A,C) 
 \tkzDrawPoints[color=red](A,B,C)
\end{tikzpicture}
\end{tkzexample}


\subsubsection{Rotation with \tkzname{shift} and \tkzname{scope}}  
Preferable to rotate is to use a \tkzNameEnv{scope} environment. 
                    
\begin{tkzexample}[latex=7cm,small] 
\begin{tikzpicture}[scale=.75,rotate=90] 
 \tkzDefPoint[label=right:$A_n$](2,3){A} 
 \begin{scope}[shift={(A)}]
   \tkzDefPoint[label= right:$B_n$](31:3){B} 
   \tkzDefPoint[label= right:$C_n$](158:3){C} 
 \end{scope}
  \tkzDrawSegments[color=red,%
           line width=1pt](A,B A,C) 
  \tkzDrawPoints[color=red](A,B,C)
 \end{tikzpicture}
\end{tkzexample}

\subsubsection{Forms and coordinates}
Here we must follow the syntax of \tkzNamePack{xfp}. It is always possible to go through \tkzNamePack{pgfmath} but in this case, the coordinates must be calculated before using the macro \tkzcname{tkzDefPoint}.

\begin{tkzexample}[latex=6cm,small]
\begin{tikzpicture}[scale=.75]
  \tkzInit[xmax=6,ymax=6]
  \tkzGrid
  \tkzSetUpPoint[shape = circle,color = red,%
                 size = 4,fill = red!30]
  \tkzDefPoint(-1+1,-1+4){O}
  \tkzDefPoint({3*ln(exp(1))},{exp(1)}){A}
  \tkzDefPoint({4*sin(pi/6)},{4*cos(pi/6)}){B}
  \tkzDefPoint({4*sin(pi/3)},{4*cos(pi/3)}){B'}
  \tkzDefPoint[shift={(1,3)}](30:3){A'} 
  \tkzDrawPoints(O,A,B) 
  \tkzDrawPoints[color=red,shape=cross out](B',A') 
  \tkzLabelPoints(A,O,B,B',A') 
\end{tikzpicture}
\end{tkzexample}

\subsubsection{Scope and \tkzcname{tkzDefPoint}}
First, we can use the \tkzNameEnv{scope} of \TIKZ. 
In the following example, we have a way to define an isosceles triangle.

\begin{tkzexample}[latex=7cm,small]
\begin{tikzpicture}[scale=1]
 \begin{scope}[rotate=30]
  \tkzDefPoint(2,3){A}
  \begin{scope}[shift=(A)]
     \tkzDefPoint(90:5){B}
     \tkzDefPoint(30:5){C}
  \end{scope}
 \end{scope}
\tkzDrawSegments[color=blue](A,B B,C C,A) 
\tkzDrawPoints(A,B,C)
\tkzLabelPoints[above](B,C)
\tkzLabelPoints[below](A) 
\end{tikzpicture}
\end{tkzexample}
%<--------------------------------------------------------------------------->
\subsection{Definition of points in Cartesian coordinates: \tkzcname{tkzDefPoints}} \hypertarget{tdps}{} 
 
\begin{NewMacroBox}{tkzDefPoints}{\oarg{local options}\var{$x_1/y_1/n_1,x_2/y_2/n_2$, ...}}%
$x_1$ and $y_1$ are the coordinates of a referenced point $n_1$ 

\begin{tabular}{lll}%
\toprule
arguments &  example  &   \\ 
\midrule
\TAline{$x_i/y_i/n_i$}{\tkzcname{tkzDefPoints\{0/0/O,2/2/A\}}}{}
\end{tabular}
\end{NewMacroBox}

\subsubsection{Definition of points}
\begin{tkzexample}[latex=6cm,small]
\begin{tikzpicture}[scale=1]
 \tkzDefPoints{0/0/A,2/0/B,2/2/C,0/2/D}
 \tkzDrawSegments(D,A A,B B,C C,D)
 %with tkz-euclide \tkzDrawPolygon(A,...,D)
 \tkzDrawPoints(A,B,C,D) 
\end{tikzpicture}
\end{tkzexample}   

%<--------------------------------------------------------------------------->
\subsection{Point relative to another: \tkzcname{tkzDefShiftPoint}} 
\hypertarget{tdsp}{} 
\begin{NewMacroBox}{tkzDefShiftPoint}{\oarg{Point}\parg{x,y}\var{name} ou \parg{a:r}\var{name}}%
\begin{tabular}{lll}%
arguments &  default & definition \\ 
\midrule
\TAline{(x,y)}{no default}{$x$ and $y$ are two dimensions, by default in cm.}
\TAline{(a:r)}{no default}{$a$ is an angle in degrees, $r$ is a dimension}
\TAline{point} {no default} {\tkzcname{tkzDefShiftPoint}[A](0:4)\{B\}} 
\bottomrule
\end{tabular}

No options. The name of the point is mandatory.
\end{NewMacroBox}

\subsubsection{Example with  \tkzcname{tkzDefShiftPoint}}
This macro allows you to place one point relative to another. This is equivalent to a translation. Here is how to construct an isosceles triangle with main vertex $A$ and angle at vertex of $30^\circ$. 

\begin{tkzexample}[latex=7cm,small]
\begin{tikzpicture}[rotate=-30]
 \tkzDefPoint(2,3){A}
 \tkzDefShiftPoint[A](0:4){B}
 \tkzDefShiftPoint[A](30:4){C} 
 \tkzDrawSegments(A,B B,C C,A)
 \tkzMarkSegments[mark=|,color=red](A,B A,C)
 \tkzDrawPoints(A,B,C) 
 \tkzLabelPoints[above](A,C)   
 \tkzLabelPoints(B) 
\end{tikzpicture}
\end{tkzexample}

\subsection{Point relative to another: \tkzcname{tkzDefShiftPointCoord}}
\begin{NewMacroBox}{tkzDefShiftPointCoord}{\oarg{a,b}\parg{x,y}\var{name} or \parg{a:r}\var{name}}%
{This involves performing a $(a,b)$ vector translation at the defined point relative to the origin.}

\medskip
\begin{tabular}{lll}%
\toprule
arguments &  default & definition \\ 
\midrule
\TAline{(x,y)}{no default}{$x$ and $y$ are two dimensions, by default in cm.}
\TAline{(a:r)}{no default}{$a$ is an angle in degrees, $r$ is a dimension}
\bottomrule
\end{tabular}

\medskip
\begin{tabular}{lll}%
options             & default & example   \\ 
\midrule
\TOline{a,b} {no default} {\tkzcname{tkzDefShiftPointCoord}[2,3](0:4)\{B\}}
 \bottomrule
\end{tabular}

The option is mandatory
\end{NewMacroBox}

  
\subsubsection{Equilateral triangle with \tkzcname{tkzDefShiftPointCoord}}
Let's see how to get an equilateral triangle (there is much simpler)

\begin{tkzexample}[latex=7cm,small]
\begin{tikzpicture}[scale=1]
 \tkzDefPoint(2,3){A}
 \tkzDefShiftPointCoord[2,3](30:4){B}
 \tkzDefShiftPointCoord[2,3](-30:4){C} 
 \tkzDrawSegments(A,B B,C C,A) 
 % or \tkzDrawPolygon
  \tkzDrawPoints(A,B,C)
  \tkzLabelPoints(B,C)  
  \tkzLabelPoint[left](A){$A$} 
\end{tikzpicture}
\end{tkzexample} 

\subsubsection{Isosceles triangle with \tkzcname{tkzDefShiftPointCoord}}
Let's see how to obtain an isosceles triangle with a principal angle of 30 degrees. Rotation is possible. $AB=AC=5$ and $\widehat{BAC}$

\begin{tkzexample}[latex=7cm,small]
\begin{tikzpicture}[rotate=15]
 \tkzDefPoint(2,3){A}
 \tkzDefShiftPointCoord[2,3](15:5){B}
 \tkzDefShiftPointCoord[2,3](-15:5){C} 
 \tkzDrawSegments(A,B B,C C,A) 
 \tkzDrawPoints(A,B,C)
 \tkzLabelPoints(B,C)
 \tkzLabelPoint[left](A){$A$}
\end{tikzpicture}
\end{tkzexample}

%<--------------------------------------------------------------------------->
\subsection{Drawing a point \tkzcname{tkzDrawPoint}}
\begin{NewMacroBox}{tkzDrawPoint}{\oarg{local options}\parg{point}}%
\begin{tabular}{lll}%
arguments &  default & definition                 \\ 
\midrule
\TAline{point} {no default}  {a name or reference is requested}
\bottomrule
\end{tabular}

\medskip
\noindent{The argument is mandatory, but it is not necessary (although recommended) to use a reference; a pair of coordinates placed between braces is accepted. The disk takes the color of the circle, but 50\% lighter. It is possible to modify everything. The point is a node and is therefore invariant if the drawing is modified by scaling..}

\medskip
\begin{tabular}{lll}%
\toprule
options             & default & definition  \\ 
\midrule
\TOline{shape}  {circle}{Possible \tkzname{cross} or \tkzname{cross out}} 
\TOline{size}   {2 pt} {disk size}
\TOline{color}  {black}{the default color can be changed}
\bottomrule
\end{tabular}

\medskip
\noindent{We can create other forms such as \tkzname{cross}}
\end{NewMacroBox}

\subsubsection{Default stitch style} 
\begin{tkzexample}[latex=5cm,small]
  \begin{tikzpicture}
   \tkzDefPoint(1,3){A}
   \tkzDrawPoint(A)
  \end{tikzpicture}
\end{tkzexample}  

\subsubsection{Changing the style} 
The default definition is in the file \tkzname{tkz-base.cfg}

\begin{tkzltxexample}[small]
\tikzset{point style/.style={draw         = \tkz@euc@pointcolor,
                             inner sep    = 0pt,
                             shape        = \tkz@euc@pointshape,
                             minimum size = \tkz@euc@pointsize,
                             fill         = \tkz@euc@pointcolor!50}}
\end{tkzltxexample}

\begin{tkzexample}[latex=5cm,small]
  \begin{tikzpicture}
   \tikzset{point style/.style={%
     draw         = blue,
     inner sep    = 0pt,
     shape        = circle,
     minimum size = 6pt,
     fill         = red!20}}
   \tkzDefPoint(1,3){A}
   \tkzDefPoint(4,1){B}
   \tkzDefPoint(0,0){O}
   \tkzDrawPoint(A)
   \tkzDrawPoint(B)
   \tkzDrawPoint(O)
  \end{tikzpicture} 
\end{tkzexample} 

\subsubsection{Example of point plots}
Note that \tkzname{scale} does not affect the shape of the dots. Which is normal.  Most of the time, we are satisfied with a single point shape that we can define from the beginning, either with a macro or by modifying a configuration file. 

\begin{tkzexample}[latex=5cm,small]
  \begin{tikzpicture}[scale=.5]
   \tkzDefPoint(1,3){A}
   \tkzDefPoint(4,1){B}
   \tkzDefPoint(0,0){O}
   \tkzDrawPoint[shape=cross out,size=12,color=red](A)
   \tkzDrawPoint[shape=cross,size=12,color=blue](B)
   \tkzDrawPoint[size=12,color=green](O)
   \tkzDrawPoint[size=12,color=blue,fill=yellow]({2,2})
  \end{tikzpicture}
\end{tkzexample}

It is possible to draw several points at once, but this macro is a little slower than the previous one. Moreover, we have to make do with the same options for all the points.                               

\subsection{Drawing points \tkzcname{tkzDrawPoints}}
\hypertarget{tdrps}{}
\begin{NewMacroBox}{tkzDrawPoints}{\oarg{local options}\parg{liste}}%
\begin{tabular}{lll}%
arguments &  default & definition \\ 
\midrule
\TAline{points list}{no default}{example \tkzcname{tkzDrawPoints(A,B,C)}}
\bottomrule
\end{tabular}

\medskip
Warning at the final "s", an oversight leads to cascading errors if you attempt to plot multiple points. The options are the same as for the previous macro. 
\end{NewMacroBox}

\subsubsection{Example with \tkzcname{tkzDefPoint} and \tkzcname{tkzDrawPoints} } 
\begin{tkzexample}[latex=7cm,small]
  \begin{tikzpicture}[scale=.5]
   \tkzDefPoint(1,3){A}
   \tkzDefPoint(4,1){B}
   \tkzDefPoint(0,0){O}
   \tkzDrawPoints[size=8,color=red](A,B,O)
  \end{tikzpicture}
\end{tkzexample} 
  
\subsubsection{More complex example } 
\begin{tkzexample}[latex=7cm]
\begin{tikzpicture}[scale=.5]
 \tkzDefPoint(2,3){A}  \tkzDefPoint(5,-1){B}  
 \tkzDefPoint[label=below:$\mathcal{C}$,
               shift={(2,3)}](-30:5.5){E}
 \begin{scope}[shift=(A)]
    \tkzDefPoint(30:5){C}
 \end{scope}   
 \tkzDrawCircle(A,B)
 \tkzDrawSegment(A,B)
 \tkzDrawPoints(A,B,C) 
 \tkzLabelPoints(B,C)
 \tkzLabelPoints[above](A)
\end{tikzpicture}
\end{tkzexample}  

%<--------------------------------------------------------------------------->
\subsection{Add a label to a point \tkzcname{tkzLabelPoint}} 
\hypertarget{tlp}{}
It is possible to add several labels at the same point by using this macro several times.  

\begin{NewMacroBox}{tkzLabelPoint}{\oarg{local options}\parg{point}\var{label}}%
\begin{tabular}{lll}%
arguments &  example  &                  \\ 
\midrule
\TAline{point}{\tkzcname{tkzLabelPoint(A)\{\$A\_1\$\}}}{}
options  & default & definition\\
\midrule
\TOline{TikZ options}{}{colour, position etc.}
\bottomrule
\end{tabular}

\medskip
Optionally, we can use any style of \TIKZ, especially placement with above, right, dots...
\end{NewMacroBox}

\subsubsection{Example with \tkzcname{tkzLabelPoint}} 
\begin{tkzexample}[latex=7cm,small]  
\begin{tikzpicture}
  \tkzDefPoint(0,0){A}
  \tkzDefPoint(4,0){B}
  \tkzDefPoint(0,3){C}
  \tkzDrawSegments(A,B B,C C,A)
  \tkzDrawPoints(A,B,C)
  \tkzLabelPoint[left,red](A){$A$}
  \tkzLabelPoint[right,blue](B){$B$}
  \tkzLabelPoint[above,purple](C){$C$}  
\end{tikzpicture} 
\end{tkzexample} 

\subsubsection{Label and reference}
 The reference of a point is the object that allows to use the point, the label is the name of the point that will be displayed.
 
\begin{tkzexample}[latex=8cm,small]
 \begin{tikzpicture}
    \tkzInit[xmax=1,xstep=0.15,ymax=.5]
    \tkzAxeX \tkzDrawY[noticks]
    \tkzDefPoint(0.22,0.25){A} 
    \tkzDrawPoint(A)
    \tkzLabelPoint[above](A){$A_1$}  
  \end{tikzpicture}
 \end{tkzexample}
%<--------------------------------------------------------------------------->
\subsection{Add labels to points \tkzcname{tkzLabelPoints}}
It is possible to place several labels quickly when the point references are identical to the labels and when the labels are placed in the same way in relation to the points. By default, \tkzname{below right} is chosen.
\hypertarget{tlps}{}  

\begin{NewMacroBox}{tkzLabelPoints}{\oarg{local options}\parg{$A_1,A_2,...$}}%
\begin{tabular}{lll}
arguments &  example & result                 \\ 
\midrule
\TAline{list of points}{\tkzcname{tkzLabelPoints(A,B,C)}}{Display of $A$, $B$ and $C$}
\bottomrule
\end{tabular}

\medskip
This macro reduces the number of lines of code, but it is not obvious that all points need the same label positioning.
\end{NewMacroBox}

\subsubsection{Example with \tkzcname{tkzLabelPoints}}   
\begin{tkzexample}[latex = 7cm,small]  
\begin{tikzpicture}
  \tkzDefPoint(2,3){A}
  \tkzDefShiftPoint[A](30:2){B}
  \tkzDefShiftPoint[A](30:5){C}
  \tkzDrawPoints(A,B,C)
  \tkzLabelPoints(A,B,C) 
\end{tikzpicture} 
\end{tkzexample}
%<--------------------------------------------------------------------------->
%                       tkzAutoLabelPoints
%<--------------------------------------------------------------------------->
\subsection{Automatic position of labels \tkzcname{tkzAutoLabelPoints}}
The label of a point is placed in a direction defined by a \tkzname{center} and a point . The distance to the point is determined by a percentage of the distance between the center and the point. This percentage is given by \tkzname{dist}.
\begin{NewMacroBox}{tkzLabelPoints}{\oarg{local options}\parg{$A_1,A_2,...$}}%
\begin{tabular}{lll}
arguments &  example & result                 \\ 
\midrule
\TAline{list of points}{\tkzcname{tkzLabelPoint(A,B,C)}}{Display of $A$, $B$ and $C$}
\end{tabular}

\medskip
\begin{tabular}{lll}
options &  default & definition                 \\ 
\midrule
\TOline{center}{no default}{you need to deisgn a center}
\TOline{dist}{0.15}{percentage change in the distance between the center and the points} 
\end{tabular}
\end{NewMacroBox}

\subsubsection{Example 1 with \tkzcname{tkzAutoLabelPoints}} 
Here the points are positioned relative to the center of gravity of $A,B,C \ \text{et}\ O$.
\begin{tkzexample}[latex=5cm,small]
\begin{tikzpicture}[scale=1.25]
  \tkzDefPoint(2,1){O}
  \tkzDefRandPointOn[circle=center O radius 1.5]
  \tkzGetPoint{A}
  \tkzDrawCircle(O,A) 
  \tkzDefPointBy[rotation=center O angle 100](A)
  \tkzGetPoint{C}
  \tkzDefPointBy[rotation=center O angle 78](A)
  \tkzGetPoint{B}
  \tkzDrawPoints(O,A,B,C) 
  \tkzDrawSegments(C,B B,A A,O O,C)
  \tkzDefBarycentricPoint(A=1,B=1,C=1,O=1)
  \tkzDrawPoint(tkzPointResult)
  \tkzAutoLabelPoints[center=tkzPointResult,
                     dist=.3,red](O,A,B,C)
\end{tikzpicture}
\end{tkzexample}

\subsubsection{Example 2 with \tkzcname{tkzAutoLabelPoints}} 
This time the reference is $O$ and the distance is by default $0.15$.
\begin{tkzexample}[latex=5cm,small]
\begin{tikzpicture}[scale=1.25]
 \tkzDefPoint(2,1){O}
 \tkzDefRandPointOn[circle=center O radius 1.5]
 \tkzGetPoint{A}
 \tkzDrawCircle(O,A) 
 \tkzDefPointBy[rotation=center O angle 100](A)
 \tkzGetPoint{C}
 \tkzDefPointBy[rotation=center O angle 78](A)
 \tkzGetPoint{B}
 \tkzDrawPoints(O,A,B,C) 
 \tkzDrawSegments(C,B B,A A,O O,C)
 \tkzAutoLabelPoints[center=O,red](A,B,C)
\end{tikzpicture}
\end{tkzexample}
%<--------------------------------------------------------------------------->

\subsection{Point style with \tkzcname{tkzSetUpPoint}}
 It is important to understand that the size of a dot depends on the size of a line.
\begin{NewMacroBox}{tkzSetUpPoint}{\oarg{local options}}%
\begin{tabular}{lll}
options &  default & definition                 \\ 
\midrule
\TOline{shape}{circle}{possible: circle, cross, cross out}
\TOline{size}{current }{the size of the point is size * line width   } 
\TOline{color}{current}{} 
\TOline{fill}{current!50}{} 
\end{tabular}
\end{NewMacroBox}

This is a macro for choosing a \hypertarget{setupoint}{style} for points.
\subsubsection{Simple example with \tkzcname{tkzSetUpPoint}} 
\begin{tkzexample}[latex=6cm,small]
\begin{tikzpicture} 
 \tkzSetUpPoint[shape = cross out,
                   color=blue] 
 \tkzInit[xmax=100,xstep=20,ymax=.5] 
 \tkzDefPoint(20,1){A} 
 \tkzDefPoint(80,0){B} 
 \tkzDrawLine(A,B)
 \tkzDrawPoints(A,B)
\end{tikzpicture}
\end{tkzexample}

\subsubsection{Second example with \tkzcname{tkzSetUpPoint}} 
\begin{tkzexample}[latex=7cm,small]
\begin{tikzpicture}
  \tkzInit[ymin=-0.5,ymax=3,xmin=-0.5,xmax=7]
  \tkzDefPoint(0,0){A}
  \tkzDefPoint(02.25,04.25){B}
  \tkzDefPoint(4,0){C}
  \tkzDefPoint(3,2){D}
  \tkzDrawSegments(A,B A,C A,D)  
  \tkzSetUpPoint[shape=cross out,size=4,]
  \tkzDrawPoints(A,B,C,D)
  \tkzLabelPoints(A,B,C,D) 
\end{tikzpicture}
\end{tkzexample}

\subsubsection{Using \tkzcname{tkzSetUpPoint} in a group}
Only the points in the group are affected by the changes.
 
\begin{tkzexample}[latex=7cm,small]   
\begin{tikzpicture}
  \tkzInit[ymin=-0.5,ymax=3,xmin=-0.5,xmax=7]
  \tkzDefPoint(0,0){A}
  \tkzDefPoint(02.25,04.25){B}
  \tkzDefPoint(4,0){C}
  \tkzDefPoint(3,2){D}
  \tkzDrawSegments(A,B A,C A,D)  
{\tkzSetUpPoint[shape=cross out,
            fill= blue!70!black!!50,
            size=4,color=blue!70!black!30]
  \tkzDrawPoints(A,B)}
  \tkzSetUpPoint[fill= blue!70!black!!50,size=4,
               color=blue!70!black!30]
   \tkzDrawPoints(C,D) 
  \tkzLabelPoints(A,B,C,D) 
\end{tikzpicture} 
\end{tkzexample} 
%<--------------------------------------------------------------------------->
\subsection{Show point coordinates} 
This macro allows you to display the coordinates of a point and to draw arrows to specify the abscissa and ordinate. The point is given by its reference (its name). It is possible to give a couple of coordinates.
 \hypertarget{tpsc}{} 
\begin{NewMacroBox}{tkzPointShowCoord}{\oarg{local options}\parg{point}}%
\begin{tabular}{lll}%
argument     & example & explanation                         \\ 
\midrule
\TAline{\parg{ref}}{\tkzcname{tkzPointShowCoord}(A)}{shows the coordinates of point $A$}
\bottomrule
\end{tabular}
 
\medskip 
\begin{tabular}{lll}%
option             & default    & explication                         \\ 
\midrule
\TOline{xlabel}{empty}{label abscissa}
\TOline{xstyle}{empty}{style for the abscissa label node example |text=red|}
\TOline{noxdraw}{false}{boolean for not draw an arrow to the X-axis $(x'x)$}
\TOline{ylabel}{empty}{idem}
\TOline{ystyle}{empty}{idem}
\TOline{noydraw}{false}{idem}
\end{tabular} 
\end{NewMacroBox}   

\subsubsection{Default styles}
Here are some of the main styles:
\begin{tkzltxexample}[small]
\tikzset{arrow coord style/.style={dashed,
                             \tkz@euc@linecolor,
                             >=latex',
                             ->}}
\tikzset{xcoord style/.style={\tkz@euc@labelcolor,
                           font=\normalsize,text height=1ex,
                           inner sep = 0pt,
                           outer sep = 0pt,
                           fill=\tkz@fillcolor,
                           below=3pt}} 
\tikzset{ycoord style/.style={\tkz@euc@labelcolor,
                           font=\normalsize,text height=1ex, 
                           inner sep = 0pt,
                           outer sep = 0pt, 
                           fill=\tkz@fillcolor,
                           left=3pt}}
\end{tkzltxexample}

\subsubsection{Example with \tkzcname{tkzPointShowCoord}} 

\begin{tkzexample}[latex=7cm,small]
  \begin{tikzpicture}[scale=1.5]
   \tkzInit[xmax=3,ymax=2]
   \tkzAxeXY
   \tkzDefPoint(2,1){a}
   \tkzPointShowCoord(a) 
   \tkzDrawPoint(a)
   \tkzLabelPoint(a){$A_1$}
   \tkzPointShowCoord({1,2}) 
   \tkzDrawPoint({1,2})
   \tkzLabelPoint({1,2}){$A_2$}
  \end{tikzpicture}  
\end{tkzexample}

\subsubsection{Example with \tkzcname{tkzPointShowCoord} and \tkzname{xstep}} 

\begin{tkzexample}[latex=7cm,small]
  \begin{tikzpicture}[xscale=3,yscale=2]
   \tkzInit[xmax=15,ymax=15,
           xstep=10,ystep=10]
   \tkzAxeXY
   \tkzDefPoint(10,10){a} \tkzDrawPoint(a)
   \tkzPointShowCoord(a)
   \tkzLabelPoint(a){$A_1$}
  \end{tikzpicture}  
\end{tkzexample}  


\subsection{\tkzcname{tkzDefSetOfPoints}} % (fold)
It was already possible to create a scatter plot with the macro \tkzcname{tkzDefPoints}, but this requires making a reference (a name) to each point, which is sometimes tedious. The macro \tkzcname{tkzSetOfPoints} allows to define points \tkzname{tkzPt1}, \tkzname{tkzPt2}, etc.

This is frequently referred to as \hypertarget{label_tkzDefSetOfPoints}{ "scatter plot"}. The difference from the macro \tkzcname{tkzDefPoints} is that the reference to the points is given by a prefix (default \tkzname{tkzPt}) and the point number. 
The points are not drawn. 

\begin{NewMacroBox}{tkzDefSetOfPoints}{\oarg{local options}\var{$x_1/y_1,x_2/y_2,\ldots,x_n/y_n$}}%
\begin{tabular}{lll}%
arguments &  default & definition  \\ 
\midrule
\TAline{$x_n/y_n$}{no default}{List of couples $x_n/y_n$ separated by commas}
\bottomrule
\end{tabular}

\medskip
\begin{tabular}{lll}%
options             & default & definition   \\ 
\midrule
\TOline{prefix} {tkzPt} {prefix for point names}
\end{tabular}
\end{NewMacroBox} 
 
\subsubsection{Creating a scatter plot with \tkzcname{tkzDefSetOfPoints}} 
\begin{tkzexample}[latex=7cm,small]
\begin{tikzpicture}
  \tkzInit[ymax=4,xmax=5]
  \tkzAxeXY
  \tkzDefSetOfPoints[prefix=P]%
           {1/2,4/3,2/2.5}
  \tkzDrawPoints(P1,P2,P3) 
  \tkzLabelPoints(P1,P2,P3)
\end{tikzpicture}
\end{tkzexample}  
   
\endinput

 
\include{TKZdoc-base-style}
\include{TKZdoc-base-BB}
\include{TKZdoc-base-obj}
\include{TKZdoc-base-rep}
\include{TKZdoc-base-divers}
\include{TKZdoc-base-marks}
\include{TKZdoc-base-texte}
\include{TKZdoc-base-faq}
%<------------------------------------------------------------------------
\clearpage\newpage 
\makeatletter

\begin{multicols}{2}
\small\printindex
\end{multicols}
\end{document}

