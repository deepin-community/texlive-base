% !TeX TXS-program:compile = txs:///arara
% arara: pdflatex: {shell: no, synctex: no, interaction: batchmode}
% arara: pdflatex: {shell: no, synctex: no, interaction: batchmode} if found('log', '(undefined references|Please rerun|Rerun to get)')

\documentclass[english,11pt,a4paper]{article}
\usepackage[utf8]{inputenc}
\usepackage[T1]{fontenc}
%\usepackage{DejaVuSerif}
%\usepackage[scale=1.125]{inconsolata}
\usepackage{pictochrono}
\usepackage{soul}
\usepackage{tabularray}
\usepackage{lipsum}
\usepackage{codehigh}
\usepackage{fontawesome5}
\usepackage{fancyvrb}
\usepackage{fancyhdr}
\fancyhf{}
\renewcommand{\headrulewidth}{0pt}
%\rhead{\sffamily\small\affloetalab[Legende]}
\lfoot{\sffamily\small [pictochrono]}
\cfoot{\sffamily\small - \thepage{} -}
\rfoot{\hyperlink{matoc}{\small\faArrowAltCircleUp[regular]}}
\usepackage{hologo}
\usepackage{xspace}
\providecommand\tikzlogo{Ti\textit{k}Z}
\providecommand\TeXLive{\TeX{}Live\xspace}
\providecommand\PSTricks{\textsf{PSTricks}\xspace}
\let\pstricks\PSTricks
\let\TikZ\tikzlogo

\usepackage{hyperref}
\urlstyle{same}
\hypersetup{pdfborder=0 0 0}
\usepackage[margin=2cm]{geometry}
\setlength{\parindent}{0pt}

\def\TPversion{0.1.0}
\def\TPdate{13/02/2024}
\usepackage{tcolorbox}

\sethlcolor{lightgray!25}
\NewDocumentCommand\MontreCode{ m }{%
	\hl{\vphantom{\texttt{pf}}\texttt{#1}}%
}

\usepackage{babel}

\begin{document}

\pagestyle{fancy}

\thispagestyle{empty}

\begin{center}
	\begin{minipage}{0.88\linewidth}
	\begin{tcolorbox}[colframe=yellow,colback=yellow!15]
		\begin{center}
			\begin{tabular}{c}
				{\Huge \texttt{pictochrono}}\\
				\\
				{\LARGE A chronometer, made with Ti\textit{k}Z,} \\
				\\
				{\LARGE for inline insertion.} \\
				\\
				{\small \texttt{Version \TPversion{} -- \TPdate}}
		\end{tabular}
		\end{center}
	\end{tcolorbox}
\end{minipage}
\end{center}

\begin{center}
	\begin{tabular}{c}
	\texttt{Cédric Pierquet}\\
	{\ttfamily c pierquet -- at -- outlook . fr}\\
	\texttt{\url{https://github.com/cpierquet/pictochrono}}
\end{tabular}
\end{center}

\hrule

\phantomsection

\hypertarget{matoc}{}

\tableofcontents

\vspace*{5mm}

\hrule

\vspace*{5mm}

\vfill

\begin{tcolorbox}[colframe=lightgray,colback=lightgray!10]
\hfill
{\pictochrono[Height=4cm]{20}}
\hfill~
\end{tcolorbox}

\medskip

\begin{tcolorbox}[colframe=lightgray,colback=lightgray!10]
{\LARGE\bfseries\sffamily Exercise n°1 (\pictochrono[ColTime=red]{10})\dotfill(5 points)}\\
{\LARGE\bfseries\sffamily Exercise n°2 (\pictochrono[ColTime=blue]{25})\dotfill(9 points)}\\
{\LARGE\bfseries\sffamily Exercise n°3 (\pictochrono[Delta=15,ColTime=orange]{35})\dotfill(11 points)}
\end{tcolorbox}

\vfill~

\pagebreak

\section{The package pictochrono}

\subsection{Ideas}

The idea is to display a pictogram like "chronometer" to present times for exercises, for example.

The pictogram can be inserted inline with automatic height and vertical offset, or with a manual height and offset.

\subsection{Loading}

The package loads within the preamble, with \MontreCode{\textbackslash usepackage\{pictochrono\}}.

The only loaded packages are \MontreCode{simplekv}, \MontreCode{calc}, \MontreCode{simplekv}, \MontreCode{xstring}, \MontreCode{etoolbox} and \MontreCode{tikz} (with \MontreCode{calc} library).

\begin{codehigh}[language=latex/latex2,style/main=cyan!10,style/code=cyan!10]
\usepackage{pictochrono}
\end{codehigh}

\section{The command}

\subsection{Usage}

The command is \MontreCode{\textbackslash pictochrono}.

\begin{codehigh}[language=latex/latex2,style/main=cyan!10,style/code=cyan!10]
\pictochrono[keys]{duration}<v.align>
\end{codehigh}

\medskip

The optional keys are :

\begin{itemize}
	\item the key \MontreCode{Height} :
	\begin{itemize}
		\item can use an automatic height, with \MontreCode{auto} (by default, global height of current fonte + small vertical offset) ;
		\item can specify a global height ;
	\end{itemize}
	\item the key \MontreCode{Color} for the main color (\MontreCode{black} by default) ;
	\item the key \MontreCode{ColTime} for the color of duration, if specified (\MontreCode{lightgray} by default) ;
	\item the key \MontreCode{Delta} for the graduations of minutes (within \texttt{0/5/10/15/20/30}) (\MontreCode{5} by default) ;
	\item the key \MontreCode{Offset} for manual vertical offset (\MontreCode{auto} by default) ;
\end{itemize}

\medskip

The mandatory argument, between \MontreCode{\{...\}}, is the duration to be displayed.

\medskip

The optional argument, between \MontreCode{<...>}, can specify a vertical alignment of the pictogram (within \texttt{vcenter/vbottom/vtop}, or empty for default positioning).

\subsection{Examples}

\begin{demohigh}[language=latex/latex2,style/main=cyan!10,style/code=cyan!10]
%automatic inline insertion
This is a test, with a \pictochrono{25} minutes chronometer.
\end{demohigh}

\begin{demohigh}[language=latex/latex2,style/main=cyan!10,style/code=cyan!10]
%automatic inline insertion, with user colors.
{\Huge This is an other test, with a 
\pictochrono[Delta=15,Color=blue,ColTime=red]{20} minutes chronometer.}
\end{demohigh}

\begin{demohigh}[language=latex/latex2,style/main=cyan!10,style/code=cyan!10]
%manual inline insertion, with manual offset.
This a \pictochrono[Height=3em,Offset=-0.75em]{45} minutes chronometer.
\end{demohigh}

\begin{demohigh}[language=latex/latex2,style/main=cyan!10,style/code=cyan!10]
%manual insertion
\pictochrono[Height=5cm,ColTime=teal]{33}
\end{demohigh}

\begin{demohigh}[language=latex/latex2,style/main=cyan!10,style/code=cyan!10]
%wihtin a tblr
\begin{tblr}{hlines,vlines,colspec={Q[m,c]Q[8cm,m,j]}}
  \pictochrono[Height=4cm,ColTime=purple]{20}
  &
  \lipsum[1][1-9] \\
\end{tblr}
\end{demohigh}

\begin{demohigh}[language=latex/latex2,style/main=cyan!10,style/code=cyan!10]
	%wihtin a tblr
\begin{tblr}{hlines,vlines,colspec={Q[m,c]Q[8cm,m,j]}}
  \pictochrono[Height=4cm,ColTime=purple]{20}<vcenter>
  &
  \lipsum[1][1-9] \\
\end{tblr}
\end{demohigh}

\vfill

\section{History}

\verb|v0.1.0|~:~~~~Initial version

\vspace*{15mm}

\end{document}