%  encoding : utf8 
%  tkz-doc-graph
%  Created by Alain Matthes  on 2021/01/20.
%  Copyright (C) 2021 Alain Matthes  
%
% This file may be distributed and/or modified
%
% 1. under the LaTeX Project Public License , either version 1.3
% of this license or (at your option) any later version and/or
% 2. under the GNU Public License.
%
% See the file doc/generic/pgf/licenses/LICENSE for more details.%
% See http://www.latex-project.org/lppl.txt for details.
%
%
% ``tkzdoc-graph-fr'' is the french doc of tkz-graph
%
%
%%%%%%%%%%%%%%%%%%%%%%%%%%%%%%%%%%%%%%%%%%%%%%%%%%%%%%%%%%%%%%%%%
%                                                               %
%        tkz-doc-graph    encodage : utf8                       %
%                                                               %
%%%%%%%%%%%%%%%%%%%%%%%%%%%%%%%%%%%%%%%%%%%%%%%%%%%%%%%%%%%%%%%%%
%                                                               %
%           Created by Alain Matthes     2007/09/02             %
%  Copyright (c) 2021 __Altermundus__ All rights reserved.      %
%        version : 2.0                                          %
%%%%%%%%%%%%%%%%%%%%%%%%%%%%%%%%%%%%%%%%%%%%%%%%%%%%%%%%%%%%%%%%%


\documentclass[DIV         = 14,
               fontsize    = 10,
               headinclude = false,
               footinclude = false,
               index       = totoc,
               twoside,
               headings    = small]{tkz-doc}   
\usepackage{etoc}
\gdef\tkznameofpack{tkz-graph}
\gdef\tkzversionofpack{2.0c}
\gdef\tkzdateofpack{2021/01/20}
\gdef\tkznameofdoc{doc-tkz-graph}
\gdef\tkzversionofdoc{2.0c} 
\gdef\tkzdateofdoc{2021/01/20}
\gdef\tkzauthorofpack{Alain Matthes}
\gdef\tkzadressofauthor{}
\gdef\tkznamecollection{AlterMundus}
\gdef\tkzurlauthor{}
\gdef\tkzengine{lualatex}
\gdef\tkzurlauthorcom{http://altermundus.fr}

% -- Packages ---------------------------------------------------          
\usepackage[dvipsnames,svgnames]{xcolor}
\usepackage{calc}
\usepackage{tkz-berge} 
\usetikzlibrary{calc,positioning,shapes}
\usepackage[colorlinks]{hyperref}
\hypersetup{
      linkcolor=Gray,
      citecolor=Green,
      filecolor=Mulberry,
      urlcolor=NavyBlue,
      menucolor=Gray,
      runcolor=Mulberry,
      linkbordercolor=Gray,
      citebordercolor=Green,
      filebordercolor=Mulberry,
      urlbordercolor=NavyBlue,
      menubordercolor=Gray,
      runbordercolor=Mulberry,
      pdfsubject={Euclidean Geometry},
      pdfauthor={\tkzauthorofpack},
      pdftitle={\tkznameofpack},
      pdfcreator={\tkzengine}
}
 \usepackage{bookmark}
\usepackage{tkzexample}
\usepackage{fontspec}
\setmainfont{texgyrepagella}%
 [Extension = .otf ,
  UprightFont = *-regular,
  ItalicFont = *-italic,
  BoldFont = *-bold,
  BoldItalicFont = *-bolditalic]
\setsansfont{texgyreheros}[
  Extension = .otf,
  UprightFont = *-regular ,
  ItalicFont  = *-italic  ,
  BoldFont    = *-bold    ,
  BoldItalicFont = *-bolditalic ,
]

\setmonofont{lmmono10-regular.otf}[
  Numbers={Lining,SlashedZero},
  ItalicFont=lmmonoslant10-regular.otf,
  BoldFont=lmmonolt10-bold.otf,
  BoldItalicFont=lmmonolt10-boldoblique.otf,
]
\newfontfamily\ttcondensed{lmmonoltcond10-regular.otf}
%% (La)TeX font-related declarations:
\linespread{1.05}      % Pagella needs more space between lines

\usepackage{unicode-math}
\usepackage{fourier-otf,zorna}
\usepackage{datetime,multicol,lscape}
\usepackage[french]{babel}
\usepackage[autolanguage]{numprint}
\usepackage{array,multirow,multido,booktabs}
\usepackage{shortvrb,fancyvrb} 
\usepackage{fancybox}
\usepackage{stmaryrd}
\usepackage{xkeyval,array} 
\usepackage[weather]{ifsym}
\RequirePackage{makeidx} 
\makeindex 

\title{The package : tkz-graph.sty}
\author{Alain Matthes}

\AtBeginDocument{\MakeShortVerb{\|}}

\begin{document} 
  
\parindent=0pt
\author{\tkzauthorofpack}  
\title{\tkznameofpack}
\date{\today}
\clearpage
\thispagestyle{empty}
\maketitle
\definecolor{iceberg}{rgb}{0.44, 0.65, 0.82}

\AddToShipoutPicture*{%
\setlength\unitlength{1mm}
\put(70,120){%
\begin{tikzpicture}[scale=4]
   \SetVertexNoLabel
   \tikzstyle{VertexStyle}=[draw,
                            shape        = circle,
                            shading      = ball,
                            ball color   = blue!50,
                            inner sep    = 10pt,
                            outer sep    = 0pt]
    \tikzstyle{EdgeStyle}= [thick,
                            double = blue,%
                            double distance = 1pt] 
    \draw (0,0)  node[circle,draw,shade,
                      ball color    = iceberg,
                      minimum size = 2cm] (am){\textbf{tkz-graph}};
     \grIcosahedral[RA=1.4,RB=0.8]
\end{tikzpicture}
}    
} 


\clearpage
\tkzSetUpColors[background=white,text=darkgray]
\let\rmfamily\ttfamily

\nameoffile{\tkznameofpack} 
\defoffile{Le package \tkzname{tkz-graph.sty} est un package pour créer à l'aide de \TIKZ\ des graphes le plus simplement possible. Il fera partie d'une série de modules ayant comme point commun, la création de dessins utiles dans l'enseignement  des mathématiques. La lecture de cette documentation va , je l'espère, vous permettre d'apprécier la simplicité d'utilisation de \TIKZ\ et vous permettre de commencer à le pratiquer. Il est accompagné du package \tkzname{tkz-berge.sty} qui permet de tracer des graphes particuliers de la théorie des graphes.}

\presentation

\vspace*{1cm}  
\lefthand\ Je souhaite remercier \textbf{Till Tantau} pour avoir créé le merveilleux outil \href{http://sourceforge.net/projects/pgf/}{Ti\emph{k}Z}. 


\vspace*{12pt}
\lefthand\ Vous trouverez de nombreux exemples sur mon site~: 
\href{http://altermundus.fr/pages/download.html}{altermundus.fr}    

\vfill   
Vous pouvez envoyer vos remarques, et les rapports sur des erreurs que vous aurez constatées à l'adresse suivante~: \href{mailto:al.ma@mac.com}{\textcolor{blue}{Alain Matthes}}.
 
This file can be redistributed and/or modified under the terms of the LATEX 
Project Public License Distributed from CTAN archives in directory \url{CTAN:// 
macros/latex/base/lppl.txt}.    



 \clearpage
 \tableofcontents
 \clearpage


Liste des macros dans l'ordre d'apparition :

\medskip
\begin{itemize}
\item \tkzcname{SetVertexLabelOut}
\item \tkzcname{SetVertexLabelIn}
\item \tkzcname{SetVertexMath}
\item \tkzcname{SetVertexNoMath}
\item \tkzcname{SetUpVertex}
\item \tkzcname{Vertex}
\item \tkzcname{EA}
\item \tkzcname{WE}
\item \tkzcname{NO}
\item \tkzcname{SO}
\item \tkzcname{NOEA}
\item \tkzcname{NOWE}
\item \tkzcname{SOEA}
\item \tkzcname{SOWE}
\item \tkzcname{Vertices}
\item \tkzcname{SetUpEdge}
\item \tkzcname{Edge}
\item \tkzcname{Edges}
\item \tkzcname{Loop}
\item \tkzcname{grProb}
\item \tkzcname{SetGraphShadeColor}
\item \tkzcname{SetGraphArtColor}
\item \tkzcname{SetGraphColor}
\item \tkzcname{AddVertexColor}
\end{itemize}

\vfill
%<-------------------------------------------------------------------------->
\renewcommand*{\VertexLightFillColor}{fondpaille} 
%\include{TKZdoc-gr-installation}
\include{TKZdoc-gr-presentation}
\include{TKZdoc-gr-vertex}
\include{TKZdoc-gr-vertices}
\include{TKZdoc-gr-label}
\include{TKZdoc-gr-edge}
\include{TKZdoc-gr-style}
\include{TKZdoc-gr-prob}
\include{TKZdoc-gr-Welsh}
\include{TKZdoc-gr-annales}
\include{TKZdoc-gr-Dijkstra}
%<-------------------------------------------------------------------------->

\clearpage\newpage
\small\printindex

\end{document}

