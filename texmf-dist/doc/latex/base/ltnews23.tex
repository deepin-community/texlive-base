% \iffalse meta-comment
%
% Copyright (C) 2015-2024
% The LaTeX Project and any individual authors listed elsewhere
% in this file.
%
% This file is part of the LaTeX base system.
% -------------------------------------------
%
% It may be distributed and/or modified under the
% conditions of the LaTeX Project Public License, either version 1.3c
% of this license or (at your option) any later version.
% The latest version of this license is in
%    http://www.latex-project.org/lppl.txt
% and version 1.3c or later is part of all distributions of LaTeX
% version 2008 or later.
%
% This file has the LPPL maintenance status "maintained".
%
% The list of all files belonging to the LaTeX base distribution is
% given in the file `manifest.txt'. See also `legal.txt' for additional
% information.
%
% The list of derived (unpacked) files belonging to the distribution
% and covered by LPPL is defined by the unpacking scripts (with
% extension .ins) which are part of the distribution.
%
% \fi
% Filename: ltnews23.tex
%
% This is issue 23 of LaTeX News.

\documentclass{ltnews}
\usepackage[T1]{fontenc}

\usepackage{lmodern,url,hologo}

\makeatletter % -- provide command introduced in new release
              %    so this typesets with an old format

% Check we are not in the preamble of a composite document
\def\@tempa{\@latex@error{Can be used only in preamble}\@eha}
\ifx\DeclareTextCommandDefault\@tempa
\else
  \DeclareTextCommandDefault\textcommabelow[1]
    {\hmode@bgroup\ooalign{\null#1\crcr\hidewidth\raise-.31ex
     \hbox{\check@mathfonts\fontsize\ssf@size\z@
     \math@fontsfalse\selectfont,}\hidewidth}\egroup}
\fi
\makeatother

\publicationmonth{October}
\publicationyear{2015}

\publicationissue{23}

\providecommand\pkg[1]{\texttt{#1}}
\providecommand\cls[1]{\texttt{#1}}
\providecommand\option[1]{\texttt{#1}}
\providecommand\env[1]{\texttt{#1}}
\providecommand\file[1]{\texttt{#1}}

\begin{document}

\maketitle

\tableofcontents

\section{Enhanced support for \hologo{LuaTeX}}

As noted in \LaTeX\ News 22, the 2015/01/01 release of \LaTeX{}
introduced built-in support for extended \TeX\ systems.

The range of allocated register numbers (for example, for count
registers) is now set according to the underlying engine capabilities
to 256, 32768 or 65536. Additional allocators were also added for the
facilities added by \hologo{eTeX} (\verb|\newmark|) and \hologo{XeTeX}
(\verb|\newXeTeXintercharclass|). At that time, however, the work to
incorporate additional allocators for \hologo{LuaTeX} was not ready for
distribution.

The main feature of this release is  that by default it includes
allocators for \hologo{LuaTeX}-provided features, such as Lua
functions, bytecode registers, catcode tables and Lua callbacks.
Previously these features have been provided by the contributed
\pkg{luatex} (Heiko Oberdiek) and \pkg{luatexbase}
(\'{E}lie Roux,
  Manuel P\'{e}gouri\'{e}-Gonnard and Philipp Gesang)
packages. However, just as
noted with the \pkg{etex} package in the previous release, it is
better if allocation is handled by the format to avoid problems with
conflicts between different allocation schemes, or definitions made
before a package-defined allocation scheme is enabled.

The facilities incorporated into the format with this release, and
described below, are closely modelled on the \pkg{luatexbase}
package and we thank the authors, and especially \'{E}lie Roux, for
help in arranging this transition.

The implementation of these \hologo{LuaTeX} features has been
redesigned to match the allocation system introduced in the 2015/01/01
\LaTeX\ release, and there are some other differences from the previous
\pkg{luatexbase} package. However, as noted below,
\pkg{luatexbase} is being updated in line with this \LaTeX\ release
to provide the previous interface as a wrapper around the new
implementation, so we expect the majority of documents using
\pkg{luatexbase} to work without change.

\subsection{Names of \hologo{LuaTeX} primitive commands}

The 2015/01/01 \LaTeX\ release for the first time initialised
\hologo{LuaTeX} in \file{latex.ltx} if \hologo{LuaTeX} is being
used. Following the convention used in the contributed
\file{lualatex.ini} file used to set up the format for earlier
releases, most \hologo{LuaTeX}-specific primitives were defined with
names prefixed by \texttt{luatex}. This was designed to minimize name
clashes but had the disadvantage that names did not match the
\hologo{LuaTeX} manual, or the names used in other formats, and
produced some awkward command names such as \verb|\luatexluafunction|.
From this release the names are enabled without the \texttt{luatex}
prefix.

In practice this change should not affect many documents; relatively
few packages access the primitive commands, and many of those are
already set up to work with prefixed or unprefixed names, so that they
work with multiple formats.

For package writers, if you want to ensure that your code works with
this and earlier releases, use unprefixed names in the package and
ensure that they are defined by using code such as:
\begin{verbatim}
\directlua{tex.enableprimitives("",
             tex.extraprimitives(
               "omega", "aleph", "luatex"))}
\end{verbatim}
Conversely if your document  uses a package relying on prefixed names
then you can add:
\begin{verbatim}
\directlua{tex.enableprimitives("luatex",
             tex.extraprimitives(
               "omega", "aleph", "luatex"))}
\end{verbatim}
to your document.

Note the compatibility layer offered by the \pkg{luatexbase} package
described below makes several commands available under both names.

As always, this change can be reverted using:\\
\verb|\RequirePackage[2015/01/01]{latexrelease}|\\
at the start of the document.



\subsection{\TeX\ commands for allocation in \hologo{LuaTeX}}
For detailed descriptions of the new allocation commands see the
documented sources in \file{ltluatex.dtx} or chapter N of
\file{source2e}; however, the following new allocation commands are
defined by default in \hologo{LuaTeX}:
\verb|\newattribute|,
\verb|\newcatcodetable|,
\verb|\newluafunction| and
\verb|\newwhatsit|.
In addition, the commands \verb|\setattribute| and
\verb|\unsetattribute| are defined to set and unset Lua attributes
(integer values similar to counters, but attached to nodes). Finally
several catcode tables are predefined:
\verb|\catcodetable@initex|,
\verb|\catcodetable@string|,
\verb|\catcodetable@latex|,
\verb|\catcodetable@atletter|.

\subsection{Predefined Lua functions}
If used with \hologo{LuaTeX}, \LaTeX\ will initialise a Lua table,
\texttt{luatexbase}, with functions supporting allocation and also
the registering of Lua callback functions.

\subsection{Support for older releases and plain \TeX}
The \hologo{LuaTeX} allocation functionality made available in this
release is also available in plain \TeX\ and older \LaTeX\ releases
in the files \file{ltluatex.tex} and \file{ltluatex.lua} which may be
used simply by including the \TeX\ file: \verb|%%
%% This is file `ltluatex.tex',
%% generated with the docstrip utility.
%%
%% The original source files were:
%%
%% ltluatex.dtx  (with options: `tex,plain')
%% 
%% This is a generated file.
%% 
%% The source is maintained by the LaTeX Project team and bug
%% reports for it can be opened at https://latex-project.org/bugs.html
%% (but please observe conditions on bug reports sent to that address!)
%% 
%% 
%% Copyright (C) 1993-2024
%% The LaTeX Project and any individual authors listed elsewhere
%% in this file.
%% 
%% This file was generated from file(s) of the LaTeX base system.
%% --------------------------------------------------------------
%% 
%% It may be distributed and/or modified under the
%% conditions of the LaTeX Project Public License, either version 1.3c
%% of this license or (at your option) any later version.
%% The latest version of this license is in
%%    https://www.latex-project.org/lppl.txt
%% and version 1.3c or later is part of all distributions of LaTeX
%% version 2008 or later.
%% 
%% This file has the LPPL maintenance status "maintained".
%% 
%% This file may only be distributed together with a copy of the LaTeX
%% base system. You may however distribute the LaTeX base system without
%% such generated files.
%% 
%% The list of all files belonging to the LaTeX base distribution is
%% given in the file `manifest.txt'. See also `legal.txt' for additional
%% information.
%% 
%% The list of derived (unpacked) files belonging to the distribution
%% and covered by LPPL is defined by the unpacking scripts (with
%% extension .ins) which are part of the distribution.
\ifx\newluafunction\undefined\else\expandafter\endinput\fi
\ifx
  \ProvidesFile\undefined\begingroup\def\ProvidesFile
  #1#2[#3]{\endgroup\immediate\write-1{File: #1 #3}}
\fi
\ProvidesFile{ltluatex.tex}%
[2023/08/03 v1.2c
  LuaTeX support for plain TeX (core)
]
\edef\etatcatcode{\the\catcode`\@}
\catcode`\@=11
\ifnum\luatexversion<60 %
  \wlog{***************************************************}
  \wlog{* LuaTeX version too old for ltluatex support *}
  \wlog{***************************************************}
  \expandafter\endinput
\fi
\long\def\@gobble#1{}
\long\def\@firstofone#1{#1}
\directlua{tex.enableprimitives("",tex.extraprimitives("luatex"))}
\ifx\e@alloc\@undefined
  \ifx\documentclass\@undefined
    \ifx\loccount\@undefined
      \input{etex.src}%
    \fi
    \catcode`\@=11 %
    \outer\expandafter\def\csname newfam\endcsname
                          {\alloc@8\fam\chardef\et@xmaxfam}
  \else
    \RequirePackage{etex}
    \expandafter\def\csname newfam\endcsname
                    {\alloc@8\fam\chardef\et@xmaxfam}
    \expandafter\let\expandafter\new@mathgroup\csname newfam\endcsname
  \fi
\edef \et@xmaxregs {\ifx\directlua\@undefined 32768\else 65536\fi}
\edef \et@xmaxfam {\ifx\Umathcode\@undefined\sixt@@n\else\@cclvi\fi}
\count 270=\et@xmaxregs % locally allocates \count registers
\count 271=\et@xmaxregs % ditto for \dimen registers
\count 272=\et@xmaxregs % ditto for \skip registers
\count 273=\et@xmaxregs % ditto for \muskip registers
\count 274=\et@xmaxregs % ditto for \box registers
\count 275=\et@xmaxregs % ditto for \toks registers
\count 276=\et@xmaxregs % ditto for \marks classes
\expandafter\let\csname newcount\expandafter\expandafter\endcsname
                \csname globcount\endcsname
\expandafter\let\csname newdimen\expandafter\expandafter\endcsname
                \csname globdimen\endcsname
\expandafter\let\csname newskip\expandafter\expandafter\endcsname
                \csname globskip\endcsname
\expandafter\let\csname newbox\expandafter\expandafter\endcsname
                \csname globbox\endcsname
\chardef\e@alloc@top=65535
\let\e@alloc@chardef\chardef
\def\e@alloc#1#2#3#4#5#6{%
  \global\advance#3\@ne
  \e@ch@ck{#3}{#4}{#5}#1%
  \allocationnumber#3\relax
  \global#2#6\allocationnumber
  \wlog{\string#6=\string#1\the\allocationnumber}}%
\gdef\e@ch@ck#1#2#3#4{%
  \ifnum#1<#2\else
    \ifnum#1=#2\relax
      #1\@cclvi
      \ifx\count#4\advance#1 10 \fi
    \fi
    \ifnum#1<#3\relax
    \else
      \errmessage{No room for a new \string#4}%
    \fi
  \fi}%
\expandafter\csname newcount\endcsname\e@alloc@attribute@count
\expandafter\csname newcount\endcsname\e@alloc@ccodetable@count
\expandafter\csname newcount\endcsname\e@alloc@luafunction@count
\expandafter\csname newcount\endcsname\e@alloc@whatsit@count
\expandafter\csname newcount\endcsname\e@alloc@bytecode@count
\expandafter\csname newcount\endcsname\e@alloc@luachunk@count
\fi
\ifx\e@alloc@attribute@count\@undefined
  \countdef\e@alloc@attribute@count=258
  \e@alloc@attribute@count=\z@
\fi
\def\newattribute#1{%
  \e@alloc\attribute\attributedef
    \e@alloc@attribute@count\m@ne\e@alloc@top#1%
}
\def\setattribute#1#2{#1=\numexpr#2\relax}
\def\unsetattribute#1{#1=-"7FFFFFFF\relax}
\ifx\e@alloc@ccodetable@count\@undefined
  \countdef\e@alloc@ccodetable@count=259
  \e@alloc@ccodetable@count=\z@
\fi
\def\newcatcodetable#1{%
  \e@alloc\catcodetable\chardef
    \e@alloc@ccodetable@count\m@ne{"8000}#1%
  \initcatcodetable\allocationnumber
}
\newcatcodetable\catcodetable@initex
\newcatcodetable\catcodetable@string
\begingroup
  \def\setrangecatcode#1#2#3{%
    \ifnum#1>#2 %
      \expandafter\@gobble
    \else
      \expandafter\@firstofone
    \fi
      {%
        \catcode#1=#3 %
        \expandafter\setrangecatcode\expandafter
          {\number\numexpr#1 + 1\relax}{#2}{#3}
      }%
  }
  \@firstofone{%
    \catcodetable\catcodetable@initex
      \catcode0=12 %
      \catcode13=12 %
      \catcode37=12 %
      \setrangecatcode{65}{90}{12}%
      \setrangecatcode{97}{122}{12}%
      \catcode92=12 %
      \catcode127=12 %
      \savecatcodetable\catcodetable@string
    \endgroup
  }%
\newcatcodetable\catcodetable@latex
\newcatcodetable\catcodetable@atletter
\begingroup
  \def\parseunicodedataI#1;#2;#3;#4\relax{%
    \parseunicodedataII#1;#3;#2 First>\relax
  }%
  \def\parseunicodedataII#1;#2;#3 First>#4\relax{%
    \ifx\relax#4\relax
      \expandafter\parseunicodedataIII
    \else
      \expandafter\parseunicodedataIV
    \fi
      {#1}#2\relax%
  }%
  \def\parseunicodedataIII#1#2#3\relax{%
    \ifnum 0%
      \if L#21\fi
      \if M#21\fi
      >0 %
      \catcode"#1=11 %
    \fi
  }%
  \def\parseunicodedataIV#1#2#3\relax{%
    \read\unicoderead to \unicodedataline
    \if L#2%
      \count0="#1 %
      \expandafter\parseunicodedataV\unicodedataline\relax
    \fi
  }%
  \def\parseunicodedataV#1;#2\relax{%
    \loop
      \unless\ifnum\count0>"#1 %
        \catcode\count0=11 %
        \advance\count0 by 1 %
    \repeat
  }%
  \def\storedpar{\par}%
  \chardef\unicoderead=\numexpr\count16 + 1\relax
  \openin\unicoderead=UnicodeData.txt %
  \loop\unless\ifeof\unicoderead %
    \read\unicoderead to \unicodedataline
    \unless\ifx\unicodedataline\storedpar
      \expandafter\parseunicodedataI\unicodedataline\relax
    \fi
  \repeat
  \closein\unicoderead
  \@firstofone{%
    \catcode64=12 %
    \savecatcodetable\catcodetable@latex
    \catcode64=11 %
    \savecatcodetable\catcodetable@atletter
   }
\endgroup
\ifx\e@alloc@luafunction@count\@undefined
  \countdef\e@alloc@luafunction@count=260
  \e@alloc@luafunction@count=\z@
\fi
\def\newluafunction{%
  \e@alloc\luafunction\e@alloc@chardef
    \e@alloc@luafunction@count\m@ne\e@alloc@top
}
\def\newluacmd{%
  \e@alloc\luafunction\luadef
    \e@alloc@luafunction@count\m@ne\e@alloc@top
}
\def\newprotectedluacmd{%
  \e@alloc\luafunction{\protected\luadef}
    \e@alloc@luafunction@count\m@ne\e@alloc@top
}
\ifx\e@alloc@whatsit@count\@undefined
  \countdef\e@alloc@whatsit@count=261
  \e@alloc@whatsit@count=\z@
\fi
\def\newwhatsit#1{%
  \e@alloc\whatsit\e@alloc@chardef
    \e@alloc@whatsit@count\m@ne\e@alloc@top#1%
}
\ifx\e@alloc@bytecode@count\@undefined
  \countdef\e@alloc@bytecode@count=262
  \e@alloc@bytecode@count=\z@
\fi
\def\newluabytecode#1{%
  \e@alloc\luabytecode\e@alloc@chardef
    \e@alloc@bytecode@count\m@ne\e@alloc@top#1%
}

\ifx\e@alloc@luachunk@count\@undefined
  \countdef\e@alloc@luachunk@count=263
  \e@alloc@luachunk@count=\z@
\fi
\def\newluachunkname#1{%
  \e@alloc\luachunk\e@alloc@chardef
    \e@alloc@luachunk@count\m@ne\e@alloc@top#1%
    {\escapechar\m@ne
    \directlua{lua.name[\the\allocationnumber]="\string#1"}}%
}
\def\now@and@everyjob#1{%
  \everyjob\expandafter{\the\everyjob
    #1%
  }%
  #1%
}
  \begingroup
    \attributedef\attributezero=0 %
    \chardef     \charzero     =0 %
    \countdef    \CountZero    =0 %
    \dimendef    \dimenzero    =0 %
    \mathchardef \mathcharzero =0 %
    \muskipdef   \muskipzero   =0 %
    \skipdef     \skipzero     =0 %
    \toksdef     \tokszero     =0 %
    \directlua{require("ltluatex")}
  \endgroup
\catcode`\@=\etatcatcode\relax
\endinput
%%
%% End of file `ltluatex.tex'.
|.
An alternative for old \LaTeX\ releases is to use:\\
\verb|\RequirePackage[2015/10/01]{latexrelease}|\\
which will update the kernel to the current release, including
\hologo{LuaTeX} support.

\subsection{Additional \hologo{LuaTeX} support packages}
In addition to the base \LaTeX\ release two packages have been
contributed to the \textsf{contrib} area on CTAN. The
\pkg{ctablestack} package offers some commands to help package
writers control the \hologo{LuaTeX} \texttt{catcodetable}
functionality, and the \pkg{luatexbase} package replaces the
previously available package of the same name, providing a compatible
interface but implemented over the \pkg{ltluatex} code.

\section{More Floats and Inserts}
If \hologo{eTeX} is available, the number of registers allocated in
the format to hold floats such as figures is increased from 18 to 52.

The extended allocation system introduced in 2015/01/01 means that in
most cases it is no longer necessary to load the \pkg{etex}
package. Many classes and packages that previously loaded this package
no longer do so. Unfortunately in some circumstances where a package
or class previously used the \pkg{etex} \verb|\reserveinserts|
command, it is possible for a document that previously worked to
generate an error ``no room for a new insert''. In practice this error
can always be avoided by declaring inserts earlier, before the
registers below 256 are all allocated. However, it is better not
to require packages to be re-ordered and in some cases the re-ordering
is complicated due to delayed allocations in \verb|\AtBeginDocument|.

In this release, a new implementation of
\verb|\newinsert| is used which allocates inserts from the previously
allocated float lists once the classical register allocation has run
out. This allows an extra 52 (or in \hologo{LuaTeX}, 64~thousand)
insert allocations which is more than enough for practical documents
(by default, \LaTeX\ only uses two insert allocations).

\section{Updated Unicode data}


The file \file{unicode-letters.def} recording catcodes, upper and
lower case mappings and other properties for Unicode characters has
been regenerated using the data files from Unicode~8.0.0.

\section{Support for Comma Accent}
The command \verb|\textcommabelow| has been added to the format.
This is mainly used for the Romanian letters
\textcommabelow{S}\textcommabelow{s}\textcommabelow{T}\textcommabelow{t}.
This was requested in latex/4414 in the \LaTeX\ bug tracker.

\section{Extended \pkg{inputenc}}
The \option{utf8} option for \pkg{inputenc} has been extended to support
the letters s and t with comma accent,
U+0218\,--\,U+021b. Similarly circumflex w and y U+0174\,--\,U+0177 are defined.
Also U+00a0 and U+00ad are declared by default, and defined to be
\verb|\nobreakspace| and \verb|\-| respectively.

The error message given on undefined UTF-8 input characters
now displays the Unicode number
in U+\textit{hex} format in addition
to showing the character.

\section{Pre-release Releases}
The patch level mechanism has been used previously to identify \LaTeX\
releases that have small patches applied to the main release, without
changing the main format date.

The mechanism has now been extended to allow identification of
pre-release versions of the software (which may or may not be released
via CTAN) but can be identified with a banner such as\\
{\catcode`\<=13 \def<{\string<} \catcode`\>=13 \def>{\string>}%
\verb|LaTeX2e <2015/10/01> pre-release-1|}\\
Internally this is identified as a patch release with a negative patch
level.

\section{Updates in tools}

The \pkg{multicol} package has been updated to fix the interaction
with ``here'' floats that land on the same page as the start or end of
a \env{multicols} environment.

\end{document}
