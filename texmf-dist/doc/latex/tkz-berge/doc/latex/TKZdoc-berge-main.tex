%  encoding : utf8 
%  tkz-berge.tex
%  Created by Alain Matthes  on 2008-01-19.
%  Copyright (C) 2009 Alain Matthes  
%
% This file may be distributed and/or modified
%
% 1. under the LaTeX Project Public License , either version 1.3
% of this license or (at your option) any later version and/or
% 2. under the GNU Public License.
%
% See the file doc/generic/pgf/licenses/LICENSE for more details.%
% See http://www.latex-project.org/lppl.txt for details.
%
%
% ``tkzdoc-berge-us'' is the english doc of tkz-berge
%
%
%%%%%%%%%%%%%%%%%%%%%%%%%%%%%%%%%%%%%%%%%%%%%%%%%%%%%%%%%%%%%%%%%
%                                                               %
%        tkz-berge.sty    encodage : utf8                       %
%                                                               %
%%%%%%%%%%%%%%%%%%%%%%%%%%%%%%%%%%%%%%%%%%%%%%%%%%%%%%%%%%%%%%%%%
%                                                               %
%           Créé par Alain Matthes le 19/02/2007                %
%  Copyright (c) 2006 __Collège Sévigné__ All rights reserved.  %
%        version : 2.7 c                                        %
%%%%%%%%%%%%%%%%%%%%%%%%%%%%%%%%%%%%%%%%%%%%%%%%%%%%%%%%%%%%%%%%%
% Fichier  .tex de présentation du package tkz-graph.sty
% d'après le code de DTK.

\documentclass[DIV         = 14,
               fontsize    = 10,
               headinclude = false,
               footinclude = false,
               index       = totoc,
               twoside,
               headings    = small]{tkz-doc}   
\usepackage{etoc}
\gdef\tkznameofpack{tkz-berge}
\gdef\tkzversionofpack{v 1.00 c} 
\gdef\tkzdateofpack{2011/05/25}
\gdef\tkznameofdoc{doc-tkz-berge}
\gdef\tkzversionofdoc{v 1.00 c}
\gdef\tkzdateofdoc{2011/05/25}
\gdef\tkzauthorofpack{Alain Matthes}
\gdef\tkzadressofauthor{}
\gdef\tkznamecollection{AlterMundus}
\gdef\tkzurlauthor{http://altermundus.fr}
\gdef\tkzengine{lualatex}
\gdef\tkzurlauthorcom{http://altermundus.fr}   

% -- Packages ---------------------------------------------------          
\usepackage[dvipsnames,svgnames]{xcolor}
\usepackage{calc}
\usepackage{tkz-berge} 
\usetikzlibrary{calc,positioning,shapes}
\usepackage[colorlinks,pdfencoding=auto]{hyperref}
\hypersetup{
      linkcolor=Gray,
      citecolor=Green,
      filecolor=Mulberry,
      urlcolor=NavyBlue,
      menucolor=Gray,
      runcolor=Mulberry,
      linkbordercolor=Gray,
      citebordercolor=Green,
      filebordercolor=Mulberry,
      urlbordercolor=NavyBlue,
      menubordercolor=Gray,
      runbordercolor=Mulberry,
      pdfsubject={Euclidean Geometry},
      pdfauthor={\tkzauthorofpack},
      pdftitle={\tkznameofpack},
      pdfcreator={\tkzengine}
}
\usepackage{tkzexample}
\usepackage{fontspec}
\setmainfont{texgyrepagella}%
 [Extension = .otf ,
  UprightFont = *-regular,
  ItalicFont = *-italic,
  BoldFont = *-bold,
  BoldItalicFont = *-bolditalic]
\setsansfont{texgyreheros}[
  Extension = .otf,
  UprightFont = *-regular ,
  ItalicFont  = *-italic  ,
  BoldFont    = *-bold    ,
  BoldItalicFont = *-bolditalic ,
]

\setmonofont{lmmono10-regular.otf}[
  Numbers={Lining,SlashedZero},
  ItalicFont=lmmonoslant10-regular.otf,
  BoldFont=lmmonolt10-bold.otf,
  BoldItalicFont=lmmonolt10-boldoblique.otf,
]
\newfontfamily\ttcondensed{lmmonoltcond10-regular.otf}
%% (La)TeX font-related declarations:
\linespread{1.05}      % Pagella needs more space between lines
                
\usepackage{unicode-math}
\usepackage{fourier-otf,zorna}
\usepackage{datetime,multicol,lscape}
\usepackage[french]{babel}
\usepackage[autolanguage]{numprint}
\usepackage{microtype}
\usepackage{array,multirow,multido,booktabs}
\usepackage{shortvrb,fancyvrb} 
\usepackage{fancybox}
\usepackage{stmaryrd}
\usepackage{xkeyval,array} 
\usepackage[weather]{ifsym}
\usepackage[format=hang,margin=10pt]{caption}
\usepackage{multicol}
\usepackage{makeidx} 
\makeindex 

\title{The package : tkz-berge.sty}
\author{Alain Matthes}

\AtBeginDocument{\MakeShortVerb{\|}}



\begin{document}
  
\parindent=0pt
\author{\tkzauthorofpack}  
\title{\tkznameofpack}
\date{\today}
\clearpage
\thispagestyle{empty}
\maketitle

\clearpage
 
\tkzSetUpColors[background=white,text=darkgray]

\let\rmfamily\ttfamily

\nameoffile{\tkznameofpack} 
\defoffile{The package \tkzname{\tkznameofpack} is a collection of some useful macros if you want to draw some classic graphs of the graph theory or to make others graphs. The kind of graphs that I will present, are sometimes called combinatorial graphs to distinguish them from the graphs of functions. Often, the word graph is short for graph of a function. A combinatorial graph is a very simple structure, a bunch of dots, some of which are connected by lines. Some of  graphs  have names, sometimes inspired by the graph's topology, and sometimes after their discoverer.\hfil\break
Why tkz-berge.sty ?\hfil\break
Claude Berge (1926 – 2002) was a French mathematician, recognized as one of the modern founders of combinatorics and graph theory. He played a major role in the renaissance of combinatorics and he is  remembered for his famous conjecture on perfect graphs, solved some months after his death.}

\presentation

\vfill   
\lefthand\  Firstly, I would like to thank \textbf{Till Tantau} for the  beautiful LATEX package, namely TikZ.

\lefthand I am grateful to  \textbf{Michel Bovani} for providing the \tkzname{fourier} font.

\lefthand\ I received much valuable advice and guidance on Graph Theory from \textbf{Rafael Villarroel}\\ \url{http://graphtheoryinlatex.blogspot.com/}.

\lefthand\ The names of  graphs can be found here  \href{http://mathworld.wolfram.com/topics/SimpleGraphs.html}%
           {\textcolor{blue}{MathWorld - SimpleGraphs}} by \href{http://en.wikipedia.org/wiki/Eric_W._Weisstein}%
           {\textcolor{blue}{E.Weisstein}}


\vspace{1cm}
Please report typos or any other comments to this documentation to \href{mailto:al.ma@mac.com}{\textcolor{blue}{Alain Matthes}}
This file can be redistributed and/or modified under the terms of the LATEX 
Project Public License Distributed from CTAN archives in directory \url{CTAN:// 
macros/latex/base/lppl.txt}. 

\clearpage
\tableofcontents
\clearpage

\newpage 
List of the main macros :

\medskip

\begin{multicols}{2}
	\begin{itemize}
	\item \tkzcname{grEmptyCycle}
	\item \tkzcname{grEmptyPath}
	\item \tkzcname{grEmptyStar}
	\item \tkzcname{grEmptyGrid}
	\item \tkzcname{grEmptyLadder}
	\item \tkzcname{EdgeInGraphFromOneToComp}
	\item \tkzcname{EdgeInGraphLoop}
	\item \tkzcname{EdgeInGraphSeq}
	\item \tkzcname{EdgeInGraphMod}
	\item \tkzcname{EdgeInGraphMod*}
	\item \tkzcname{grCompleteBipartite}
	\item \tkzcname{EdgeInGraphModLoop}
	\item \tkzcname{EdgeIdentity}
	\item \tkzcname{EdgeIdentity*}
	\item \tkzcname{EdgeFromOneToAll}
	\item \tkzcname{EdgeFromOneToSeq}
	\item \tkzcname{EdgeFromOneToSel}
	\item \tkzcname{EdgeFromOneToComp}
	\item \tkzcname{EdgeMod}
	\item \tkzcname{EdgeMod*}
	\item \tkzcname{EdgeDoubleMod}
	\item \tkzcname{grPath}
	\item \tkzcname{grCycle}
	\item \tkzcname{grComplete}
	\item \tkzcname{grCirculant}
	\item \tkzcname{grStar}
	\item \tkzcname{grSQCycle}
	\item \tkzcname{grWheel}
	\item \tkzcname{grLadder}
	\item \tkzcname{grPrism}
	\item \tkzcname{grCompleteBipartite}
	\item \tkzcname{grTriangularGrid}
	\item \tkzcname{grLCF}
	\item \tkzcname{grWriteExplicitLabels}
	\item \tkzcname{grWriteExplicitLabel}
	\item \tkzcname{AssignVertexLabel}
	\end{itemize} 
\end{multicols}

Classic graphs :

\medskip

\begin{multicols}{2}
	\begin{itemize}
\item \tkzcname{grAndrasfai}
\item \tkzcname{grBalaban}
\item \tkzcname{grChvatal}
\item \tkzcname{grCocktailParty}
\item \tkzcname{grCrown}
\item \tkzcname{grCubicalGraph}
\item \tkzcname{grDesargues}
\item \tkzcname{grDodecahedral}
\item \tkzcname{grDoyle}
\item \tkzcname{grFoster}
\item \tkzcname{grFolkman}
\item \tkzcname{grFranklin}
\item \tkzcname{grGeneralizedPetersen}
\item \tkzcname{grGrotzsch}
\item \tkzcname{grHeawood}
\item \tkzcname{grIcosahedral}
\item \tkzcname{grKonisberg}
\item \tkzcname{grLevi}
\item \tkzcname{grMcGee}
\item \tkzcname{grMobiusKantor}
\item \tkzcname{grMobiusLadder}
\item \tkzcname{grOctahedral}
\item \tkzcname{grPappus}
\item \tkzcname{grPetersen}
\item \tkzcname{grRobertson}
\item \tkzcname{grRobertsonWegner}
\item \tkzcname{grTetrahedral}
\item \tkzcname{grTutteCoxeter}
\item \tkzcname{grWong}
\end{itemize} 
\end{multicols}


See the document "NamedGraph" for all the classic named graphs that you can draw with the package \tkzname{tkz-berge.sty}.
\include{TKZdoc-berge-macros}
\include{TKZdoc-berge-macros-e}
\include{TKZdoc-berge-classic}
\include{TKZdoc-berge-style}

\clearpage\newpage
\small\printindex

\end{document}



 