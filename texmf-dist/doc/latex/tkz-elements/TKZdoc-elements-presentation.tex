\newpage
\section{Presentation}

\subsection{With Lua} % (fold)
\label{sub:with_lua}

The primary function of tkz-elements is to calculate dimensions and define points, which is achieved using Lua. You can view  tkz-elements as a kernel that is utilized either by tkz-euclide or by TikZ,
Definitions and calculations take place within the  environment \tkzNameEnv{tkzelements}, which is based on \tkzNameEnv{luacode}.

\begin{minipage}[t]{.52\textwidth}\vspace{0pt}%
   The key points are: 
   \begin{itemize}
      \item The source file must be \tkzEHand\ {\color{red}\uline{ \color{black}UTF8}}  encoded.
      \item Compilation is done with  \tkzEHand\ {\color{red}\uline{ \color{black}Lua\LATEX{}}}.
      \item You need to load \tkzimp{\TIKZ}{} or \tkzimp{tkz-euclide} and \tkzimp{tkz-elements}.
      \item Definitions and calculations are performed in an (orthonormal) Cartesian coordinate system, using Luawithin the  \tkzimp{tkzelements} environment.
   \end{itemize}
   
 On the right, you can see the minimum template.
  
The code is divided into two parts, represented by  two environments \tkzNameEnv{tkzelements} and \tkzNameEnv{tikzpicture}. In the first environment, you place your Lua code, while in the second, you use tkz-euclide commands.

\vspace*{4.1 cm}%
\end{minipage}\hspace*{\fill}
\begin{minipage}[t]{.45\textwidth}\vspace{0pt}%
\begin{mybox}
\begin{Verbatim}
% !TEX TS-program = lualatex
% Created by Alain Matthes
\documentclass{standalone} 
\usepackage{tkz-euclide}
% or simply TikZ
\usepackage{tkz-elements}
begin{document} 
    
\begin{tkzelements}
   scale = 1
% definition of some points
z.A = point : new (   ,   )
z.B = point : new (   ,   )

 ...code...
\end{tkzelements}

\begin{tikzpicture}
% point transfer to Nodes
\tkzGetNodes

\end{tikzpicture}
\end{document}
\end{Verbatim}
\end{mybox}
\end{minipage}
% subsection with_lua (end)

\subsection{The main process} % (fold)
\label{sub:the_main_process}

\tikzset{concept/.append style={fill={none}}}
\tikzset{root concept/.style=   {minimum size=3cm,text width=2.8cm}}%
\tikzset{level 1 concept/.append style={minimum size=4cm, font=\large, text width=3cm}}%
\begin{tikzpicture}
  \path[mindmap,concept color=orange,text=black]
    node[concept] {Transfers\\\textcolor{orange}{ \textbackslash{tkzGetNodes}}}
    child[concept color=red,grow=right] {
      node[concept] {Drawings\\\textcolor{red}{tkz-euclide}\\\textcolor{red}{\TIKZ}} }
    child[concept color=purple,grow=left] { 
    node[concept] {Definitions\\Calculations\\\textcolor{purple}{tkz-elements}} };
\end{tikzpicture}

After obtaining all the necessary points for the drawing, they must be transformed into \tkzname{nodes} so that \pkg{TikZ} or \pkg{tkz-euclide} can render the figure. This is accomplished using the macro \tkzcname{tkzGetNodes}. This macro iterates through  all the elements of the table |z| using the key (which is essentially the name of the point) and retrieves the associated values, namely the coordinates of the point (node).
% subsection the_main_process (end)

\newpage
\subsection{Complete example: Pappus circle} % (fold)
\label{sub:the_figure_pappus_circle}

\subsubsection{The figure} 

\begin{tkzelements}
 scale = 1.2
  z.A  = point: new (0 , 0)
  z.B  = point: new (10 , 0)
  L.AB = line:  new ( z.A, z.B)
  z.C  = L.AB:  gold_ratio () 
  L.AC = line:  new ( z.A, z.C)
  L.CB = line:  new ( z.C, z.B)
  L.AB = line:  new ( z.A, z.B)
  z.O_0    = L.AB.mid
  z.O_1    = L.AC.mid
  z.O_2    = L.CB.mid
  C.AB = circle: new ( z.O_0, z.B) 
  C.AC = circle: new ( z.O_1, z.C) 
  C.CB = circle: new ( z.O_2, z.B)
  z.P  = C.CB.north  
  z.Q  = C.AC.north
  z.O  = C.AB.south
  z.c  = z.C : north (2)
  C.PC = circle: new ( z.P, z.C) 
  C.QA = circle: new ( z.Q, z.A)  
  z.P_0    = intersection (C.PC,C.AB)
  z.P_1    = intersection (C.PC,C.AC)
  _,z.P_2  = intersection (C.QA,C.CB)
  z.O_3 = triangle: new ( z.P_0, z.P_1, z.P_2).circumcenter
\end{tkzelements}
\hspace*{\fill}
   \begin{tikzpicture}
  \tkzGetNodes
  \tkzDrawCircle[black,fill=yellow!20,opacity=.4](O_0,B)
  \tkzDrawCircles[teal,fill=teal!40,opacity=.6](O_1,C O_2,B)
  \tkzDrawCircle[purple,fill=purple!20,opacity=.4](O_3,P_0)
  \tkzDrawArc[cyan,delta=10](Q,A)(P_0)
  \tkzDrawArc[cyan,delta=10](P,P_0)(B)
  \tkzDrawArc[cyan,delta=10](O,B)(A)
  \tkzDrawPoints(A,B,C,O_0,O_1,O_2,P,Q,P_0,P_0,P_1,P_2,O)
  \tkzLabelPoints(A,B,C,O_0,O_1,O_2,P,Q,P_0,P_0,P_1,P_2,O)
   \end{tikzpicture}
\hspace*{\fill}
% subsection the_figure_pappus_circle (end)

\subsubsection{The code} % (fold)
\label{ssub:the_code}

\begin{Verbatim}
% !TEX TS-program = lualatex
\documentclass{article}
\usepackage{tkz-euclide}
\usepackage{tkz-elements}
\begin{document}

\begin{tkzelements}
z.A     = point: new (0 , 0)
z.B     = point: new (10 , 0)        --  creation of two fixed points $A$ and $B$
L.AB    = line:  new ( z.A, z.B)
z.C     = L.AB:  gold_ratio ()       --  use of a method linked to “line”
z.O_0   = line:  new ( z.A, z.B).mid  -- midpoint of segment with an attribute of “line”
z.O_1   = line:  new ( z.A, z.C).mid --  objects are not stored and cannot be reused.
z.O_2   = line:  new ( z.C, z.B).mid   
C.AB    = circle: new ( z.O_0, z.B)  --  new object “circle” stored and reused
C.AC    = circle: new ( z.O_1, z.C) 
C.CB    = circle: new ( z.O_2, z.B)
z.P     = C.CB.north                 --  “north” atrributes of a circle
z.Q     = C.AC.north
z.O     = C.AB.south
z.c     = z.C : north (2)           --   “north” method of a point (needs a parameter)
C.PC    = circle: new ( z.P, z.C) 
C.QA    = circle: new ( z.Q, z.A)  
z.P_0   = intersection (C.PC,C.AB)   --  search for intersections of two circles.
z.P_1   = intersection (C.PC,C.AC)  --   idem
_,z.P_2 = intersection (C.QA,C.CB)   --  idem
z.O_3   = triangle: new ( z.P_0, z.P_1, z.P_2).circumcenter 
                      -- circumcenter attribute of “triangle”
\end{tkzelements}
\end{Verbatim}

\begin{Verbatim}
\begin{tikzpicture}
  \tkzGetNodes
  \tkzDrawCircle[black,fill=yellow!20,opacity=.4](O_0,B)
  \tkzDrawCircles[teal,fill=teal!40,opacity=.6](O_1,C O_2,B)
  \tkzDrawCircle[purple,fill=purple!20,opacity=.4](O_3,P_0)
  \tkzDrawArc[cyan,delta=10](Q,A)(P_0)
  \tkzDrawArc[cyan,delta=10](P,P_0)(B)
  \tkzDrawArc[cyan,delta=10](O,B)(A)
  \tkzDrawPoints(A,B,C,O_0,O_1,O_2,P,Q,P_0,P_0,P_1,P_2,O)
  \tkzLabelPoints(A,B,C,O_0,O_1,O_2,P,Q,P_0,P_0,P_1,P_2,O)
\end{tikzpicture}
\end{document}
\end{Verbatim}
% subsubsection the_code (end)

\subsection{Another example with comments: South Pole} % (fold)
\label{sub:south_pole}

Here's another example with comments

\begin{Verbatim}
% !TEX TS-program = lualatex
\documentclass{standalone}
\usepackage{tkz-euclide,tkz-elements}
\begin{document}
\begin{tkzelements}                       
   z.A      = point: new (2 , 4)          -- we create environment tkzelements
   z.B      = point: new (0 , 0)          -- three fixed points are used
   z.C      = point: new (8 , 0)
   T.ABC    = triangle: new (z.A,z.B,z.C) -- we create a new triangle object
   C.ins    = T.ABC: in_circle ()         -- we get the incircle of this triangle
   z.I      = C.ins.center                -- center is an attribute of the circle
   z.T      = C.ins.through               -- through is also an attribute
   -- z.I,z.T  = get_points (C.ins)       -- get_points is a shortcut
   C.cir    = T.ABC : circum_circle ()    -- we get the  circumscribed circle
   z.W      = C.cir.center                -- we get the center of this circle   
   z.O      = C.cir.south                 -- now we get the south pole of this circle
   L.AO     = line: new (z.A,z.O)         -- we create an object "line"
   L.BC     = T.ABC.bc                    -- we get the line (BC)
   z.I_A    = intersection (L.AO,L.BC)    --  we search the intersection of the last lines
\end{tkzelements}
\end{Verbatim}
\begin{tkzelements}
   scale    = 1.2
   z.A      = point: new (2 , 4)
   z.B      = point: new (0 , 0)
   z.C      = point: new (8 , 0)
   T.ABC    = triangle: new (z.A,z.B,z.C)
   C.ins    = T.ABC: in_circle ()
   z.I      = C.ins.center
   z.T      = C.ins.through
-- z.I,z.T  = get_points (C.ins)
   C.cir    = T.ABC : circum_circle ()
   z.W      = C.cir.center
   z.O      = C.cir.south
   L.AO     = line: new (z.A,z.O)
   L.BC     = T.ABC.bc
   z.I_A    = intersection (L.AO,L.BC)
\end{tkzelements}

\hspace*{\fill}
\begin{tikzpicture}
\tkzGetNodes
\tkzDrawCircles(W,A I,T)
\tkzDrawArc(O,C)(B)
\tkzDrawPolygon(A,B,C)
\tkzDrawSegments[new](A,O B,O C,O)
\tkzDrawLine(B,I)
\tkzDrawPoints(A,B,C,I,I_A,W,O)
\tkzFillAngles[green!20,opacity=.3](A,O,B A,C,B)
\tkzFillAngles[teal!20,opacity=.3](O,B,C B,C,O B,A,O O,A,C)
\tkzLabelPoints(I,I_A,W,B,C,O)
\tkzLabelPoints[above](A)
\end{tikzpicture}
\hspace*{\fill}

Here's the tikzpicture environment to obtain the drawing:
\begin{Verbatim}
\begin{tikzpicture}
\tkzGetNodes
\tkzDrawCircles(W,A I,T)
\tkzDrawArc(O,C)(B)
\tkzDrawPolygon(A,B,C)
\tkzDrawSegments[new](A,O B,O C,O)
\tkzDrawLine(B,I)
\tkzDrawPoints(A,B,C,I,I_A,W,O)
\tkzFillAngles[green!20,opacity=.3](A,O,B A,C,B)
\tkzFillAngles[teal!20,opacity=.3](O,B,C B,C,O B,A,O O,A,C)
\tkzLabelPoints(I,I_A,W,B,C,O)
\tkzLabelPoints[above](A)
\end{tikzpicture}
\end{Verbatim}
% subsection south_pole (end)
\endinput