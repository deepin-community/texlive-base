% !TeX TXS-program:compile = txs:///arara
% arara: pdflatex: {shell: yes, synctex: no, interaction: batchmode}
% arara: pdflatex: {shell: yes, synctex: no, interaction: batchmode} if found('log', '(undefined references|Please rerun|Rerun to get)')

\documentclass[english,a4paper,11pt]{article}
\usepackage[margin=2cm,includefoot]{geometry}
\def\TPversion{0.1.3}
\def\TPdate{12/06/2023}
\usepackage[utf8]{inputenc}
\usepackage[T1]{fontenc}
\usepackage{amsmath,amssymb}
\usepackage{postit}
\usepackage{awesomebox}
\usepackage{fontawesome5}
\usepackage{footnote}
\makesavenoteenv{tabular}
\usepackage{enumitem}
\usepackage{tabularray}
\usepackage{wrapstuff}
\usepackage{lipsum}
\usepackage{fancyvrb}
\usepackage{fancyhdr}
\fancyhf{}
\renewcommand{\headrulewidth}{0pt}
\lfoot{\sffamily\small [postit]}
\cfoot{\sffamily\small - \thepage{} -}
\rfoot{\hyperlink{matoc}{\small\faArrowAltCircleUp[regular]}}

%\usepackage{hvlogos}
\usepackage{hologo}
\usepackage{xspace}
\providecommand\tikzlogo{Ti\textit{k}Z}
\providecommand\TeXLive{\TeX{}Live\xspace}
\providecommand\PSTricks{\textsf{PSTricks}\xspace}
\let\pstricks\PSTricks
\let\TikZ\tikzlogo
\newcommand\TableauDocumentation{%
	\begin{tblr}{width=\linewidth,colspec={X[c]X[c]X[c]X[c]X[c]X[c]},cells={font=\sffamily}}
		{\LARGE \LaTeX} & & & & &\\
		& {\LARGE \hologo{pdfLaTeX}} & & & & \\
		& & {\LARGE \hologo{LuaLaTeX}} & & & \\
		& & & {\LARGE \TikZ} & & \\
		& & & & {\LARGE \TeXLive} & \\
		& & & & & {\LARGE \hologo{MiKTeX}} \\
	\end{tblr}
}

\usepackage{hyperref}
\urlstyle{same}
\hypersetup{pdfborder=0 0 0}
\setlength{\parindent}{0pt}
\definecolor{LightGray}{gray}{0.9}

\usepackage{babel}
%\AddThinSpaceBeforeFootnotes
%\FrenchFootnotes

\usepackage{listings}

\usepackage{newverbs}
\newverbcommand{\motcletex}{\color{cyan!75!black}}{}
\newverbcommand{\packagetex}{\color{violet!75!black}}{}

\tcbuselibrary{listingsutf8}
\newtcblisting{DemoCode}[1][]{%
	enhanced,width=0.95\linewidth,center,%
	bicolor,size=title,%
	colback=cyan!2!white,%
	colbacklower=cyan!1!white,%
	colframe=cyan!75!black,%
	listing options={%
		breaklines=true,%
		breakatwhitespace=true,%
		style=tcblatex,basicstyle=\small\ttfamily,%
		tabsize=4,%
		commentstyle={\itshape\color{gray}},
		keywordstyle={\color{blue}},%
		classoffset=0,%
		keywords={},%
		alsoletter={-},%
		keywordstyle={\color{blue}},%
		classoffset=1,%
		alsoletter={-},%
		morekeywords={center,right,justify,left,\lipsum},%
		keywordstyle={\color{violet}},%
		classoffset=2,%
		alsoletter={-},%
		morekeywords={PostItNote,\MiniPostIt},%
		keywordstyle={\color{green!50!black}},%
		classoffset=3,%
		morekeywords={Color,PinColor,Pin,Width,Hieght,Rotate,Shadow,Corner,PinShift,AlignH,AlignV,AlignPostIt,Border,ExtraRightMargin,Render,Title,FontTitle,StorePostIt},%
		keywordstyle={\color{orange}}
	},%
	#1
}

\tcbset{vignettes/.style={%
	nobeforeafter,box align=base,boxsep=0pt,enhanced,sharp corners=all,rounded corners=southeast,%
	boxrule=0.75pt,left=7pt,right=1pt,top=0pt,bottom=0.25pt,%
	}
}

\tcbset{vignetteMaJ/.style={%
	fontupper={\vphantom{pf}\footnotesize\ttfamily},
	vignettes,colframe=purple!50!black,coltitle=white,colback=purple!10,%
	overlay={\begin{tcbclipinterior}%
			\fill[fill=purple!75]($(interior.south west)$) rectangle node[rotate=90]{\tiny \sffamily{\textcolor{black}{\scalebox{0.66}[0.66]{\textbf{MàJ}}}}} ($(interior.north west)+(5pt,0pt)$);%
	\end{tcbclipinterior}}
	}
}

\newcommand\Cle[1]{{\small\sffamily\textlangle \textcolor{orange}{#1}\textrangle}}
\newcommand\cmaj[1]{\tcbox[vignetteMaJ]{#1}\xspace}

\begin{document}

\setlength{\aweboxleftmargin}{0.07\linewidth}
\setlength{\aweboxcontentwidth}{0.93\linewidth}
\setlength{\aweboxvskip}{8pt}

\pagestyle{fancy}

\thispagestyle{empty}

\vspace{2cm}

\begin{center}
	\begin{minipage}{0.75\linewidth}
	\begin{tcolorbox}[colframe=yellow,colback=yellow!15]
		\begin{center}
			\begin{tabular}{c}
				{\Huge \texttt{postit} [en]}\\
				\\
				{\LARGE Small Post-It notes,} \\
				\\
				{\LARGE with \textsf{tcolorbox} or \textsf{Ti\textit{k}Z}.} \\
			\end{tabular}
			
			\bigskip
			
			{\small \texttt{Version \TPversion{} -- \TPdate}}
		\end{center}
	\end{tcolorbox}
\end{minipage}
\end{center}

\begin{center}
	\begin{tabular}{c}
	\texttt{Cédric Pierquet}\\
	{\ttfamily c pierquet -- at -- outlook . fr}\\
	\texttt{\url{https://github.com/cpierquet/postit}}
\end{tabular}
\end{center}

\vspace{0.25cm}

{$\blacktriangleright$~~Display and customize Post-It or \textit{mini-}Post-It.}

\vspace{0.25cm}

{$\blacktriangleright$~~Custom width, height, rotation, decoration\ldots}

\vspace{1cm}

\begin{PostItNote}[StorePostIt=PI1]<center>
	This is a small Post-It ! For example \[(a+b)^2=a^2+2ab+b^2.\]
\end{PostItNote}

\begin{PostItNote}[Render=tikz,Width=8cm,Color=orange,Pin=Paperclip,PinColor=blue,Rotate=-5,AlignPostIt=center,Title={- With a title -},FontTitle={\color{blue!50!black}\bfseries\small\sffamily},StorePostIt=PI2]
\lipsum[1][1-4]
\end{PostItNote}
\hfill
\begin{PostItNote}[Height=6cm,AlignV=center,Color=pink,Pin=Scotch,Rotate=15,Corner,AlignPostIt=center,StorePostIt=PI3]
\lipsum[1][1-4]
\end{PostItNote}

\begin{tikzpicture}[remember picture,overlay]
	\draw[very thick,->,>=latex] (PI1-S)to[out=-90,in=90](PI2-N) ;
	\draw[very thick,lime,densely dashed,->,>=latex] (PI2-E)to[out=0,in=180](PI3-S-W) ;
\end{tikzpicture}

\vspace{0.5cm}

%\hfill{}\textit{Merci à Denis Bitouzé et à Gilles Le Bourhis pour leurs retours et idées !}

\smallskip

\vfill

\hrule

\medskip

\TableauDocumentation

\medskip

\hrule

\medskip

\newpage

\phantomsection
\hypertarget{matoc}{}

\tableofcontents

\vfill

\section{History}

\verb|v0.1.3|~:~~~~Nodes for anchor points.

\verb|v0.1.2|~:~~~~English version.

\verb|v0.1.1|~:~~~~\motcletex!\vphantom! for \textit{mini-}Post-It + Bugfixes + \TikZ{} rendering + optional title .

\verb|v0.1.0|~:~~~~Initial version.

\newpage

\section{The package postit}

\subsection{Introduction}

\begin{noteblock}
The package proposes small Post-It notes, in a \textsf{tex} doc, created with \packagetex!tcolorbox! or \packagetex!tikz!, with option(s) in order to :

\begin{itemize}
	\item change dimensions orcolor ;
	\item use pin deocration like Paperclip, Pushpin or Scotch ;
	\item customize border and/or corner ;
	\item use anchor points for each Post-It.
\end{itemize}

The package propose a command to display \textit{mini-}Post-It (created with \motcletex!tcbox!), with color and shadow customization.
\end{noteblock}

\subsection{Loading of the package, and option}

\begin{importantblock}
The package Scrabble loads within the preamble.

There's no option, and \packagetex!xcolor! isn't loaded.
\end{importantblock}

\begin{DemoCode}[listing only]
\documentclass{article}
\usepackage{postit}

\end{DemoCode}

\begin{noteblock}
\packagetex!postit! loads the following packages and libraries :

\begin{itemize}
	\item \packagetex!tcolorbox! with library \packagetex!tcbox.skins! ;
	\item \packagetex!tikz! libraries :
	\begin{itemize}
		\item \packagetex!tikz.calc! ;
		\item \packagetex!tikz.babel! ;
		\item \packagetex!tikz.decorations! ;
		\item \packagetex!tikz.decorations.pathmorphing! ;
	\end{itemize}
	\item \packagetex!settobox!, \packagetex!xstring!, \packagetex!varwidth! and \packagetex!simplekv!.
\end{itemize}

It’s mostly compatible with \textsf{latex}, \textsf{pdflatex}, \textsf{lualatex} or \textsf{xelatex} compilation !
\end{noteblock}

\subsection{Compatibility}

\begin{cautionblock}
If an other package loads \packagetex!tcolorbox!, with \Cle{[most]} option, it's better to load \packagetex!postit! after, to avoid \motcletex!option clash error...!.
\end{cautionblock}

\begin{DemoCode}[listing only]
\documentclass{article}
\usepackage[<librairies>]{tcolorbox}
\usepackage{postit}
...

\end{DemoCode}

\vfill~

\pagebreak

\section{Post-It Environment}

\subsection{Environment}

\begin{cautionblock}
The environment to display a Post-It note is \packagetex!PostItNote!.

It works with keys, between \texttt{[...]} and, with \texttt{<...>}, it's possible to parse options to  the \motcletex!tcbox! (not necessary with \motcletex!tikz!) !
\end{cautionblock}

\begin{DemoCode}[listing only]
\begin{PostIt}[keys]<options tcbox>
...
...
\end{PostIt}
\end{DemoCode}

\begin{noteblock}
As mentionned in the introduction, the Post-It note is create with a \motcletex!tcbox! or a  \motcletex!tikzpicture!.

Most of the \motcletex!tcbox!/\motcletex!tikzpicture! parameters are fixed by the code, but some of them are configurable !
\end{noteblock}

\begin{DemoCode}[]
%default rendering (tcbox), with lipsum paragraph
\begin{PostItNote}
\lipsum[1][1-2]
\end{PostItNote}
\end{DemoCode}

\begin{DemoCode}[]
%tikz rendering, with lipsum paragraph
\begin{PostItNote}[Render=tikz]
\lipsum[1][1-2]
\end{PostItNote}
%tikzv2 rendering, with lipsum paragraph
\begin{PostItNote}[Render=tikzv2]
\lipsum[1][1-2]
\end{PostItNote}
\end{DemoCode}

\begin{tipblock}
The colors must be used as \textit{single}, without \textit{mixes} (with \motcletex|CouleurA!...!CouleurB|) proposed by \packagetex!xcolor!. 


However, every predefined color can be used within the Post-It.
\end{tipblock}

\begin{tipblock}
The Post-It can be used with a \motcletex!minipage! or a \motcletex!wrapstuff! if needed.

For horizontal alignement, \motcletex!\hfill! or \motcletex!flush...! can be used.
\end{tipblock}

\begin{warningblock}
With a overlapping pin and the \textsf{tcbox} rendering, a vertical spacing before can be necessary, like \motcletex!\vspace! or \motcletex!\bigskip!\ldots
\end{warningblock}

\subsection{Keys and options}

\begin{tipblock}
The first argument, mandatory and between \texttt{[...]}, proposes the following \Cle{keys} :

\begin{itemize}
	\item \cmaj{0.1.3} \Cle{StorePostIt} : name (for futher \TikZ{} code) of the Post-It ; \hfill{}default : \Cle{PostIt}
	\item \Cle{Width} : width (in cm) of the Post-It  ; \hfill{}default : \Cle{6cm}
	\item \Cle{Color} : color of the Post-It (border is a bit darker) ; \hfill{}default : \Cle{yellow}
	\item \Cle{Height} : hieght (in cm, if needed) of the Post-It (\textit{automatic} by default) ;
	
	\hfill{}default : \Cle{auto}
	\item \cmaj{0.1.1} \Cle{Render} : engine, within \Cle{tcbox / tikz / tikv2} ; \hfill{}default : \Cle{tcbox}
	\item \Cle{Rotation} : rotation of the Post-It ; \hfill{}default : \Cle{0}
	\item \Cle{Shadow} : boolean for shadow ; \hfill{}default : \Cle{true}
	\item \Cle{Border} : boolean for a thin border ; \hfill{}default : \Cle{true}
	\item \Cle{Corner} : boolean to the corner decoration (\motcletex!tcbox!) ; \hfill{}default : \Cle{false}
	\item \Cle{Pin} : decoration, within \Cle{Paperclip / Pushpin / None / Scotch} ;
	
	\hfill{}default : \Cle{Pushpin}
	\item \Cle{PinColor} : color of the pin ; \hfill{}default : \Cle{red}
	\item \Cle{PinsShift} : horizontal shift (without unity, but in cm) of the orginal position of the pin \hfill{}default : \Cle{0}
	\item \cmaj{0.1.1} \Cle{Title} : add a title (1st line and/or under the pin) ; \hfill{}default : \Cle{empty}
	\item \cmaj{0.1.1} \Cle{Fonttitle} : font of the titel ; \hfill{}default : \Cle{\textbackslash normalfont\textbackslash normalfont}
	\item \cmaj{0.1.1} \Cle{ExtraRightMargin} : add (with \packagetex!tikz! rendering, and in cm) à right margin ;
	
	\hfill{}default : \Cle{0cm}
	\item \Cle{AlignV} : vertical alignement in the Post-It (within \Cle{top/center/bottom}) ;
	
	\hfill{}default : \Cle{top}
	\item \Cle{AlignH} : horizontal alignment in the Post-It (within \Cle{left/center/right/justify}) ;
	
	\hfill{}default : \Cle{justify}
	\item \Cle{AlignPostIt} : vertical alignemnt of the Post-It (within \Cle{top/center/bottom}).
	
	\hfill{}default : \Cle{bottom}
\end{itemize}
\vspace*{-\baselineskip}\leavevmode
\end{tipblock}

\begin{tipblock}
The second argument, optional and between \texttt{<...>} is used to parse options to the \motcletex!tcolorbox!.

They can be used to modify locally options not present in the keys.
\end{tipblock}

\subsection{Anchor points}

\begin{tipblock}
Some anchor points are created with the code :

\begin{itemize}
	\item \motcletex!(<name>-N)!, \motcletex!(<name>-E)!, \motcletex!(<name>-S)! et \motcletex!(<name>-W)! for North/East/South/West ;
	\item \motcletex!(<name>-N-W)!, \motcletex!(<name>-N-E)!, \motcletex!(<name>-S-E)! and \motcletex!(<name>-S-W)! for North East/North West/\ldots.
\end{itemize}
\end{tipblock}

\begin{DemoCode}[]
\begin{center}
\begin{PostItNote}[Rotate=10,Pin=None,Render=tikz,StorePostIt=MySmallNote1]
	\lipsum[1][1-2]
\end{PostItNote}
\end{center}
\end{DemoCode}

\begin{tikzpicture}[remember picture,overlay]
	\foreach \dir/\pos in {N-W/above left,N/above,N-E/above right,E/right, S-E/below right,S/below,S-W/below left,W/left} 
	{%
		\draw[draw=blue,fill=red] (MySmallNote1-\dir) circle[radius=2pt] node[text=gray,\pos,font=\scriptsize\ttfamily] {MySmallNote1-\dir};%
	}
\end{tikzpicture}

\begin{DemoCode}[]
\begin{PostItNote}[StorePostIt=NoteY]<center>
	This is a small Post-It ! For example \[(a+b)^2=a^2+2ab+b^2.\]
\end{PostItNote}\\
\begin{PostItNote}[Render=tikz,Width=8cm,Color=blue,Rotate=-5,StorePostIt=NoteZ]
	\lipsum[1][1-2]
\end{PostItNote}

\begin{tikzpicture}[remember picture,overlay]
	\draw[very thick,->,>=latex] (NoteY-S)to[out=-90,in=90](NoteZ-N) ;
\end{tikzpicture}
\end{DemoCode}

\subsection{Examples}

\begin{DemoCode}[]
\begin{PostItNote}%tcbox rendering
	[Color=cyan,Pin=Paperclip,Width=10cm,Rotate=10]<center,right=1.5cm>
\lipsum[1][1-3]
\end{PostItNote}
\end{DemoCode}

\begin{DemoCode}[]
\hfill\begin{PostItNote}%tikz rendering
	[Render=tikz,Color=violet,Width=9cm,Rotate=-10,Pin=Paperclip,
	PinColor=black,ExtraRightMargin=1cm,Title={Small title},
	FontTitle={\color{white}\bfseries\small\sffamily}]
\lipsum[1][1-3]
\end{PostItNote}\hfill~
\end{DemoCode}

\begin{DemoCode}[]
\hfill\begin{PostItNote}%tikzv2 rendering
	[Render=tikzv2,Color=orange,Width=9cm,rotate=-10,Pin=Scotch, Title={Try},
	FontTitle={\color{blue!50!black}\bfseries\itshape\small\ttfamily}]
\lipsum[1][1-3]
\end{PostItNote}\hfill~
\end{DemoCode}

\begin{DemoCode}[]
%usepackage{wrapstuff}
\begin{wrapstuff}[r,top=1]
\begin{PostItNote}[Rotate=5,Corner,Color=pink,PinkColor=blue,Border=false]
\lipsum[1][1-2]
\end{PostItNote}
\end{wrapstuff}

\lipsum[1]
\end{DemoCode}

\begin{DemoCode}[]
%usepackage{wrapstuff}
\begin{wrapstuff}[r,top=1]
\begin{PostItNote}[Rotate=5,Render=tikz,Color=pink, PinkColor=blue,Border=false]
\lipsum[1][1-2]
\end{PostItNote}
\end{wrapstuff}

\lipsum[1]
\end{DemoCode}

\begin{DemoCode}[]
%usepackage{wrapstuff}
\begin{wrapstuff}[r,top=1]
\begin{PostItNote}[Rotate=5,Render=tikzv2,Pin=Scotch,Color=pink]
\lipsum[1][1-2]
\end{PostItNote}
\end{wrapstuff}

\lipsum[1]
\end{DemoCode}

\begin{DemoCode}[]
A small Post-It, and vertically aligned :
%
\hfill\begin{PostItNote}[Rotate=-10,Color=orange,Width=5cm,Height=5cm, AlignV=center,Corner,PinColor=yellow, PinShift=-1,AlignPostIt=center]

\textsf{\small\lipsum[1][1-2]}
\[\mathsf{\displaystyle\sum_{k=1}^{n} k = \dfrac{n(n+1)}{2}}\]
\end{PostItNote}
\end{DemoCode}

\pagebreak

\section{Simple inline Post-It Note}

\subsection{Command}

\begin{cautionblock}
The inline \textit{mini-}Post-It note is \motcletex!MiniPostIt!.

The only optional paramater for the \motcletex!tcbox! Post-It is the color

\smallskip

Dimensions can't be changed, a \motcletex!\vphantom! is insered at beginning to prevent different heights.
\end{cautionblock}

\begin{DemoCode}[listing only]
\MiniPostIt(*)[color]{text}
\end{DemoCode}

\subsection{Arguments}

\begin{noteblock}
The starred version show the shadow og the \textit{mini-}Post-It.

The color (\Cle{yellow}), is the only optional argument, between \texttt{[...]}.
\end{noteblock}

\subsection{Examples}

\begin{DemoCode}[]
To solve Diophantine equations, we can use \MiniPostIt*[orange]{Bezout's thorem}, and \MiniPostIt{Gauss' theorem}, with the \MiniPostIt*[cyan]{reciprocal}.

It's classic and good to know !
\end{DemoCode}

\pagebreak

\section{Gallery of styles}

\subsection{Render by tcbox}

\begin{DemoCode}[text only]
\hfill\begin{PostItNote}
\texttt{Shadow/PushPin/Border}
\end{PostItNote}
\begin{PostIt}[Shadow=false]
\texttt{Pin/Border}
\end{PostIt}\hfill~

\medskip

\hfill\begin{PostItNote}[Border=false]
\texttt{Shadow/Pushpin}
\end{PostItNote}
\begin{PostItNote}[Border=false,Shadow=false]
\texttt{Pushpin}
\end{PostItNote}\hfill~

\medskip

\hfill\begin{PostItNote}[Pin=Paperclip]
\texttt{Shadow/Paperclip/Border}\\
~
\end{PostItNote}
\begin{PostItNote}[Pin=Scotch]
\texttt{Shadow/Scotch/Border}\\
~
\end{PostItNote}\hfill~

\medskip

\hfill\begin{PostItNote}[Pin=None]
\texttt{Shadow/Border}
\end{PostItNote}
\begin{PostItNote}[Corner,Pin=None]
\texttt{Shadow/Border/Corner}
\end{PostItNote}\hfill~

\vspace{1cm}

\hfill\begin{PostItNote}[Title={Lipsum[1][1-4]},FontTitle={\large\sffamily},Rotate=5,Color=pink,Height=6cm,Pin=Scotch,AlignV=center,Corner]
\lipsum[1][1-4]
\end{PostItNote}\hfill~
\end{DemoCode}

\pagebreak

\subsection{Render by tikz}

\begin{DemoCode}[text only]
\hfill\begin{PostItNote}[Render=tikz]
\texttt{Shadow/Pushpin/Border}
\end{PostItNote}
\begin{PostItNote}[Shadow=false,Render=tikz]
\texttt{Pushpin/Border}
\end{PostItNote}\hfill~

\medskip

\hfill\begin{PostItNote}[Border=false,Render=tikz]
\texttt{Shadow/Pushpin}
\end{PostItNote}
\begin{PostItNote}[Border=false,Shadow=false,Render=tikz]
\texttt{Pushpin}
\end{PostItNote}\hfill~

\medskip

\hfill\begin{PostItNote}[Pin=Paperclip,Render=tikz]
\texttt{Shadow/Paperclip/Border}\\
~
\end{PostItNote}
\begin{PostItNote}[Pin=Scotch,Render=tikz]
\texttt{Shadow/Scotch/Border}\\
~
\end{PostItNote}\hfill~

\medskip

\hfill\begin{PostItNote}[Pin=None,Render=tikz]
\texttt{Shadow/Border}
\end{PostItNote}\hfill~

\vspace{1cm}

\hfill\begin{PostItNote}[Render=tikz,Title={Lipsum[1][1-4]},FontTitle={\large\sffamily},Rotate=5,Color=pink,Height=6cm,Pin=Scotch,AlignV=center,Corner]
\lipsum[1][1-4]
\end{PostItNote}\hfill~
\end{DemoCode}

\subsection{Render by tikzv2}

\begin{DemoCode}[text only]
\hfill\begin{PostItNote}[Render=tikzv2]
\texttt{Shadow/Pushpin/Border}
\end{PostItNote}
\begin{PostItNote}[Shadow=false,Render=tikzv2]
\texttt{Pushpin/Border}
\end{PostItNote}\hfill~

\medskip

\hfill\begin{PostItNote}[Border=false,Render=tikzv2]
\texttt{Shadow/Pushpin}
\end{PostItNote}
\begin{PostItNote}[Border=false,Shadow=false,Render=tikzv2]
\texttt{Pushpin}
\end{PostItNote}\hfill~

\medskip

\hfill\begin{PostItNote}[Pin=Paperclip,Render=tikzv2]
\texttt{Shadow/Paperclip/Border}\\
~
\end{PostItNote}
\begin{PostItNote}[Pin=Scotch,Render=tikzv2]
\texttt{Shadow/Scotch/Border}\\
~
\end{PostItNote}\hfill~

\medskip

\hfill\begin{PostItNote}[Pin=None,Render=tikzv2]
\texttt{Shadow/Border}
\end{PostItNote}\hfill~

\vspace{1cm}

\hfill\begin{PostItNote}[Render=tikzv2,Title={Lipsum[1][1-4]},FontTitle={\large\sffamily},Rotate=5,Color=pink,Height=6cm,Pin=Scotch,AlignV=center,Corner]
\lipsum[1][1-4]
\end{PostItNote}\hfill~
\end{DemoCode}

\end{document}