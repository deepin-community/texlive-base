%!TEX root = tzplot-doc.tex
%\begin{document}

%%%%%===== PART: NOT YET!!! ==============
%%%%%%%%==================================
%%%%%\part{Miscellaneous Macros and Examples}
%%%%%%%%==================================
%%%%%\label{p:miscandexamples}
%%%%%
%%==================================

%%==================================
\chapter{Miscellany}
\label{c:misc}

%%------------------------------------------------------------
\section{Middle arrows}
\label{s:midarrows}

Four styles for middle arrow tips are predefined: |-->--|, |--o--|, |--x--| and |--/--|.

\subsection{Controllable middle arrow tips: \texttt{-->--} and \protect\cmd{\settzmidarrow}}

The middle arrow tip style \ixxw{-->--} prints a middle arrow tip of |stealth| by default.
It accepts one argument that changes the position (|0.5| by default) of a \iisw{middle arrow tip}, like |-->--=.75|.


\begin{tzcode}{.3}
\begin{tikzpicture}
% middle arrow tip: -->--
\tzhelplines(4,3)
\tzline[-->--](0,3)(4,3)
\tzline[-->--=.75,blue,thick](0,2.5)(4,2.5)
\tzto[-->--,->,out=45](0,0)(4,1)
\end{tikzpicture}
\end{tzcode}


\paragraph{How to control middle arrow tips}
The default options for middle arrow tips are |-|, |thin|, |solid|, and |shorten >=0|, and |bend right=0|. The defaults can be changed by |\settzmidarrow|.

The macro \icmd{\settzmidarrow} controls the position, style, and other options of middle arrow tips. The effect is valid until the end of the |tikzpicture| environment, unless changed again.

\begin{tzdef}
% syntax:
\settzmidarrow<position>{<arrow tip style>}[<opt>]
% all arguments are optional
% defaults:
  <0.5>{stealth}[-,thin,solid,shorten <=0,shorten >=0,bend right=0]
\end{tzdef}


\begin{tzcode}{.3}
% \settzmidarrow
\begin{tikzpicture}
\tzhelplines(4,3)
\settzmidarrow<.25>{Stealth[reversed]}[red,scale=2] %
\tzline[-->--](0,3)(4,3)
\tzline[-->--=.75,blue,thick](0,2.5)(4,2.5)
\tzto[-->--,->,out=45](0,0)(4,1)
\end{tikzpicture}
\end{tzcode}


\begin{tzcode}{.3}
% various arrow tip styles: arrows.meta
\begin{tikzpicture}
\tzhelplines(4,3)
\settzmidarrow{Circle[open]}[blue,scale=2]
\tzline[-->--](0,3)(4,3)
\tikzset{>=to}                             %%
\settzmidarrow{>>>}[blue,scale=2]
\tzline[-->--](0,2.5)(4,2.5)
\settzmidarrow<.25>{Rays}[blue,scale=2]
\tzline[-->--=.75,blue,thick](0,2)(4,2)
\tzto[-->--,->,red,out=45](0,0)(4,1)
\end{tikzpicture}
\end{tzcode}

You can use various styles of arrow tips. (See \Tikz\ manual on |arrows.meta| library, for more details.)



\subsection{Fixed middle arrow tip styles: \texttt{--o--}, \texttt{--x--}, \texttt{--/--}}

Three styles for middle arrow tips are predefined: \xem{circle} |--o--| , \xem{cross} |--x--| , and \xem{diagonal} |--/--| middle arrow tips.
These middle arrow tip styles are fixed and cannot be changed.
|\settzmidarrow| can only be used with |[<opt>]| to control these middle arrow tips.
The other options |<position>| and |{<arrow tip style>}| options are ignored for these styles.

\paragraph{Circle middle arrow tips:}

The circle middle arrow tip style \ixxw{--o--} takes one argument to change the position (|0.5| by default).

The circle is drawn like this:

\begin{verbatim}
    \draw [<opt>] (0,0) circle (1.2pt) ;
\end{verbatim}

|\settzmidarrow| controls the circle arrow tip with |<opt>|.

\begin{tzcode}{.3}
\begin{tikzpicture}
% circle middle arrow tip: --o--
\tzhelplines(4,3)
\tzline[--o--](0,3)(4,3)
\tzline[--o--=.75,blue,thick](0,2.5)(4,2.5)
\tzto[--o--,->,out=45](0,0)(4,1)
\end{tikzpicture}
\end{tzcode}

\begin{tzcode}{.3}
\begin{tikzpicture}
% \settzmidarrow: difference
\tzhelplines(4,3)
\settzmidarrow<.25>{Circle[open,fill=green]}[scale=2] %
\tzline[--o--,thick](0,3)(4,3)
\tzline[-->--=.5,blue,thick](0,2.5)(4,2.5)
\tzline[-->--,blue,thick](0,2)(4,2)
\settzmidarrow[scale=2,thick,fill=green]
\tzto[--o--=.25,->,out=45](0,0)(4,1)
\end{tikzpicture}
\end{tzcode}


\paragraph{Cross middle arrow tips:}

The cross middle arrow tip style \ixxw{--x--} takes one argument to change the position (|0.5| by default).

The cross mark is drawn like this:

\begin{verbatim}
    \draw [<opt>] (2pt,2pt)  to (-2pt,-2pt) ;
    \draw [<opt>] (-2pt,2pt) to (2pt,-2pt) ;
\end{verbatim}

|\settzmidarrow| controls the cross arrow tip with |<opt>|.

\begin{tzcode}{.3}
\begin{tikzpicture}
% cross middle arrow tip: --x--
\tzhelplines(4,3)
\tzline[--x--](0,3)(4,3)
\tzline[--x--=.75,blue,thick](0,2.5)(4,2.5)
\tzto[--x--,->,out=45](0,0)(4,1)
\end{tikzpicture}
\end{tzcode}

\begin{tzcode}{.3}
\begin{tikzpicture}
% \settzmidarrow: scale, tzextend
\tzhelplines(4,3)
\settzmidarrow<.25>{Rays[red]}[scale=2] %
\tzline[--x--,thick](0,3)(4,3)
\tzline[-->--=.5,blue,thick](0,2.5)(4,2.5)
\tzline[-->--,blue,thick](0,2)(4,2)
\settzmidarrow[thick,blue,tzextend={8pt}{8pt}]
\tzto[--x--=.25,->,out=45](0,0)(4,1)
\end{tikzpicture}
\end{tzcode}


\paragraph{Diagonal middle arrow tips}

The diagonal middle arrow tip style \ixxw{--/--} takes one argument to change the position (|0.5| by default).

The diagonal line is drawn like this:

\begin{verbatim}
    \draw [<opt>] (2pt,2pt)  to (-2pt,-2pt) ;
\end{verbatim}

|\settzmidarrow| controls the diagonal arrow tip line with |[<opt>]|.

\begin{tzcode}{.3}
\begin{tikzpicture}
% diagonal middle arrow tip: --/--
\tzhelplines(4,3)
\tzline[--/--](0,3)(4,3)
\tzline[--/--=.75,blue,thick](0,2.5)(4,2.5)
\tzto[--/--,->,out=45](0,0)(4,1)
\end{tikzpicture}
\end{tzcode}

\begin{tzcode}{.3}
\begin{tikzpicture}
% \settzmidarrow: rotate, tzextend
\tzhelplines(4,3)
\tzline[--/--,thick](0,3)(4,3)
\tzline[--/--=.25,blue,thick](0,2.5)(4,2.5)
\settzmidarrow[rotate=90,red,tzextend={3pt}{3pt}]
\tzline[--/--,blue,thick](0,2)(4,2)
\tzto[--/--=.75,->,out=45](0,0)(4,1)
\end{tikzpicture}
\end{tzcode}

\begin{tzcode}{.3}
\begin{tikzpicture}
% \settzmidarrow: bend left
\tzhelplines(4,3)
\settzmidarrow[scale=2,rotate=45,bend left=60]
\tzline[--/--,thick](0,3)(4,3)
\tzline[--/--=.25,blue,thick](0,2.5)(4,2.5)
\settzmidarrow[rotate=90,red,tzextend={3pt}{3pt}]
\tzline[--/--,blue,thick](0,2)(4,2)
\tzto[--/--=.75,->,out=45](0,0)(4,1)
\end{tikzpicture}
\end{tzcode}



%%------------------------------------------------------------
\section{\protect\cmd{\tzbrace(')}}
\label{s:tzbrace}

\icmd{\tzbrace} takes two coordinates as mandatory arguments to draw a \xem{calligraphic brace} connecting them.

\begin{tzdef}
% syntax: minimum
\tzbrace(<coor>)(<coor>)
% syntax: medium
\tzbrace(<coor>)(<coor>){<text>}[<node opt>]
% syntax: full
\tzbrace[<draw opt>]{<raise>}[<decoration opt>]<shift coor>
        (<coor>)(<coor>){<text>}[<node opt>]
% defaults
  []{5pt}[amplitude=5pt]<>(<m>)(<m>){}[]
\end{tzdef}

The |raise| value of a brace is |5pt| by default and the value can be changed by the first curly brace optional argument |{<raise>}|. 

The |amplitude| of a brace is |5pt| by default. You can control the amplitude by writing the option |amplitude=<dim>| in the second bracket option |[<decoration opt>]|.

\begin{tztikz}
\tzbrace[thick](0,0)(3,1) % works like:
  \draw [thick,decorate,decoration={calligraphic brace, amplitude=5pt, raise=5pt}]
        (0,0) to (3,1);
\end{tztikz}

The \iisw{swap version} \icmd{\tzbrace'} swaps the coordinates.
So it prints a mirror image of |\tzbrace|.
For example, |\tzbrace'(0,0)(3,1)| is equivalent to |\tzbrace(3,1)(0,0)|.

\begin{tzcode}{.3}
% \tzbrace(')
\begin{tikzpicture}[sloped]
\tzhelplines(4,2)
\tzline(0,0)(3,1)
\tzbrace(0,0)(3,1){AAA}[above=10pt,blue]
\tzbrace'[red](0,0)(3,1){BBB}[below=10pt]
\end{tikzpicture}
\end{tzcode}

You can change the style of the decorating brace by the second bracket optional argument |[<decoration opt>]|.

The color of the calligraphic brace can be changed by the option |pen colour| in the list of |[<draw option>]|.

\begin{tzcode}{.3}
% \tzbrace('): decoration options
\begin{tikzpicture}[sloped]
\tzhelplines(4,3)
\tzline(0,0)(3,1)
\tzbrace [very thick,pen colour=blue]
         [amplitude=10pt]
         (0,0)(3,1){AAA}[a=15pt]
\tzbrace'[red,very thick]{10pt}
         [brace,amplitude=10pt]
         (0,0)(3,1){BBB}[b=20pt]
\end{tikzpicture}
\end{tzcode}

You can also move a brace by specifying the option |<shift coor>| immediately before the the first mandatory coordinate.
The empty shift option |<>| is not allowed.

\begin{tzcode}{.3}
% \tzbrace('): shift
\begin{tikzpicture}[sloped]
\tzhelplines(4,3)
\tzline(0,0)(3,1)
\tzbrace [very thick,pen colour=blue][amplitude=10pt]
         <.5,.5>(0,0)(3,1){AAA}[a=15pt]
\tzbrace'[red,very thick]{10pt}[brace,amplitude=10pt]
         (0,0)(3,1){BBB}[b=20pt]
\end{tikzpicture}
\end{tzcode}




%%------------------------------------------------------------
\section{\protect\cmd{\tzsnake(+)}: Snake lines (Experimental)}
\label{s:tzsnake}

\icmd{\tzsnake} connects two points with a snaked line, with many default values, using \Tikz's |to| operation.

\begin{tztikz}
\tzsnake(0,1)(3,2) % works like
  \draw [ decorate , decoration={ %% many defaults
                                  snake,
                                  segment length=5pt, % controlled by \tzsnake
                                  amplitude=2.5pt,
                                  pre length=5pt,
                                  post length=5pt
                                }
        ]  (0,1) to (3,2) ;
\end{tztikz}

\begin{tzcode}{.3}
\begin{tikzpicture}
\tzhelplines(4,3)
\tzsnake(0,1)(3,2)
\tzsnake[->,blue,bend right](1,0)(4,3)
\end{tikzpicture}
\end{tzcode}

\begin{tzdef}
% syntax: minimum
\tzsnake(<coor>)(<coor>)
% syntax: full
\tzsnake[<opt>]{<segment length>}[<decoration opt>]<shift coor>
        (<coor>){<text>}[<node opt>]
        (<coor>){<text>}[<node opt>]<code.append>
% defaults
  []{5pt}[many defaults above]<>(<m>){}[](<m>){}[]<>
\end{tzdef}

The key |segment length| is controlled by the first curly brace option and you should write down all the other keys in the second bracket option |[<decoration opt>]| to change the values. Not specifying the option |{<segment length>}|, when the first bracket option is empty, you need empty brackets |[]|, like |\tzsnake[][<decoration opt>]...|.

\begin{tzcode}{.3}
% \tzsnake: {segment length} option
\begin{tikzpicture}
\tzhelplines(4,3)
\tzsnake[blue]{3pt}[coil](0,1){spring}[a,sloped](3,2)
\tzsnake[->,bend right]{3pt}
        [coil,amplitude=5pt,post length=20pt]
        (1,0){coil line}[b,sloped](4,3){Up!}[draw,r]
\end{tikzpicture}
\end{tzcode}

The plus version \icmd{\tzsnake+} uses the second mandatory coordinate as a relative coordinate to the first. Everything else is the same as in |\tzsnake|.


\begin{tzcode}{.3}
% \tzsnake(+): shift
\begin{tikzpicture}
\tzhelplines(4,3)
\tzsnake[->][post length=20pt](0,0)(3,0)
\tzsnake[->,red][post length=20pt]<1,1>(0,0)(3,0) %%
\tzsnake+[->,bend left,blue]{10pt}[post length=10pt]
  (0,1){snake line}[a,sloped](4,2){what?}[r]
\end{tikzpicture}
\end{tzcode}


%%%%%%%% (under construction)
%%%%%
%%%%%
%%%%%%%------------------------------------------------------------
%%%%%\section{\protect\cmd{\tzspy} (Experimental)}
%%%%%\label{s:tzspy}
%%%%%
%%%%%
%%%%%\begin{tzdef}
%%%%%% syntax: minimum
%%%%%\tzspy(<coor>)(<coor>)
%%%%%% syntax: full
%%%%%\tzbrace[<opt>]{<size>}[<maginificaltion>]
%%%%%        (<on-coor>)(<in-coor>)[<node opt>]<code.append>
%%%%%% defaults
%%%%%  [circle]{2cm}{5}(<m>)(<m>)[]<>
%%%%%\end{tzdef}
%%%%%
%%%%%To use |\tzspy| you need to tell \Tikz\ that a picture will be magnified, by specifying some special keys in the |tikzpicture| environment option, such as |spy using outlines|,  |spy using overlays|, etc., as follows:
%%%%%\begin{verbatim}
%%%%%    \begin{tikzpicture}[spy using outlines]
%%%%%      <tzplot code>
%%%%%      <tikz code>
%%%%%    \tzspy...
%%%%%    \end{tikzpicture}
%%%%%\end{verbatim}
%%%%%
%%%%%\begin{tzcode}[text above listing]
%%%%%\begin{tikzpicture}[scale=.1,font=\scriptsize,spy using outlines]
%%%%%\settzdotsize{1.2pt}
%%%%%\tzaxes(-15,0)(60,60){$Q$}{$P$}
%%%%%
%%%%%\def\Dem{50-\x}
%%%%%\def\Sup{.5*\x+5}
%%%%%\def\Supa{.5*\x+6.5}
%%%%%
%%%%%\tzfn{\Dem}[0:50]{$D$}
%%%%%\tzfn{\Sup}[-10:50]{$S$}[r]
%%%%%\tzfn{\Supa}[-10:50]{$S'$}
%%%%%
%%%%%\tzXpoint*{Dem}{Sup}(eqm)
%%%%%\tzXpoint*{Dem}{Supa}(eqma)
%%%%%\tzvXpointat*{Sup}{29}(eqmb)
%%%%%
%%%%%\tzproj[densely dotted](eqm){\rlap{300}}{200}
%%%%%\tzproj[densely dotted](eqma){\llap{290}}{\raisebox{1em}{210}}
%%%%%\tzprojy[densely dotted](eqmb){\raisebox{-2em}{195}}
%%%%%
%%%%%\tzticks{-10/-100,50/500}{5/50,50/500}
%%%%%
%%%%%\tzpath*[blue](eqm)(eqma)(eqmb);
%%%%%
%%%%%\tzspy($(eqma)!.5!(eqmb)$)(100,30)
%%%%%
%%%%%%\spy [blue,draw,circle,size=5cm, magnification = 5,connect spies]
%%%%%%on ($(eqma)!.5!(eqmb)$) in node at (100,30);
%%%%%\end{tikzpicture}
%%%%%\end{tzcode}
%%%%%
%%%%%
%%%%%\begin{tzcode}[text above listing]
%%%%%\begin{tikzpicture}[scale=.1,font=\scriptsize,spy using overlays]
%%%%%\settzdotsize{1.2pt}
%%%%%\tzaxes(-15,0)(60,60){$Q$}{$P$}
%%%%%
%%%%%\def\Dem{50-\x}
%%%%%\def\Sup{.5*\x+5}
%%%%%\def\Supa{.5*\x+6.5}
%%%%%
%%%%%\tzfn{\Dem}[0:50]{$D$}
%%%%%\tzfn{\Sup}[-10:50]{$S$}[r]
%%%%%\tzfn{\Supa}[-10:50]{$S'$}
%%%%%
%%%%%\tzXpoint*{Dem}{Sup}(eqm)
%%%%%\tzXpoint*{Dem}{Supa}(eqma)
%%%%%\tzvXpointat*{Sup}{29}(eqmb)
%%%%%
%%%%%\tzproj[densely dotted](eqm){\rlap{300}}{200}
%%%%%\tzproj[densely dotted](eqma){\llap{290}}{\raisebox{1em}{210}}
%%%%%\tzprojy[densely dotted](eqmb){\raisebox{-2em}{195}}
%%%%%
%%%%%\tzticks{-10/-100,50/500}{5/50,50/500}
%%%%%
%%%%%\tzpath*[blue](eqm)(eqma)(eqmb);
%%%%%
%%%%%\tzspy[connect spies]{5cm}[6]($(eqma)!.5!(eqmb)$)(100,30)[draw,fill=red,thick]
%%%%%\end{tikzpicture}
%%%%%\end{tzcode}
%%%%%



%%%%%%%% (under construction)
%%%%%\section{\protect\cmd{\tzplotfile}: plotting data}
%%%%%\label{s:tzplotfile}
%%%%%
%%%%%\icmd{\tzplotfile} plots data.
%%%%%
%%%%%\begin{tcblisting}{listing only}
%%%%%% syntax
%%%%%\tzplotfile[<opt>]{<file name>}[<plot opt>]
%%%%%\end{tcblisting}
%%%%%
%%%%%
%%%%%\begin{tcblisting}{listing above text}
%%%%%\begin{tikzpicture}
%%%%%\tzplotfile{sine.data}
%%%%%\end{tikzpicture}
%%%%%\end{tcblisting}
%%%%%
%%%%%\begin{tcblisting}{listing above text}
%%%%%\begin{tikzpicture}
%%%%%\tzplotfile[blue]{sine.data}[mark=x]
%%%%%\end{tikzpicture}
%%%%%\end{tcblisting}





